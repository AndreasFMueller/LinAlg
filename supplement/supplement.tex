%
% supplement.tex -- Supplement zur Vorlesung Lineare Algebra
%                   gehalten an der Hochschule Rapperswil im Wintersemester 17
%
% (c) 2017 Prof. Dr. Andreas Mueller, HSR
%
\documentclass{book}
\usepackage{german}
\usepackage{times}
\usepackage{amsmath}
\usepackage{amssymb}
\usepackage{amsfonts}
\usepackage{amsthm}
\usepackage{graphicx}
\usepackage{fancyhdr}
\usepackage{textcomp}
\usepackage[all]{xy}
\usepackage{txfonts}
\usepackage{array}
\usepackage{makeidx}
\usepackage{verbatim}
\usepackage{pdflscape}
\usepackage{paralist}
\usepackage{epic}
\usepackage[colorlinks=true]{hyperref}
\usepackage{geometry}
\geometry{papersize={170mm,240mm},total={140mm,200mm},top=21mm,bindingoffset=10mm}
\setlength{\unitlength}{1pt}
\usepackage{color}
\usepackage{enumitem}
\input ../skript/linsys.tex
\makeindex
\setlength{\headheight}{15pt}
\begin{document}
\pagestyle{fancy}
\lhead{Tabea's Supplement}
\frontmatter
\newcommand\HRule{\noindent\rule{\linewidth}{1.5pt}}
\begin{titlepage}
\vspace*{\stretch{1}}
\HRule
\vspace*{10pt}
\begin{flushright}
{\Huge
Lineare Algebra}

\vspace*{10pt}
{\LARGE
Tabea's Supplement}
\end{flushright}
\HRule
\begin{flushright}
\vspace{30pt}
\LARGE
Andreas M"uller
\end{flushright}
\vspace*{\stretch{2}}
\begin{center}
Hochschule f"ur Technik, Rapperswil, 2017
\end{center}
\end{titlepage}
\hypersetup{
    linktoc=all,
    linkcolor=blue
}
\rhead{Inhaltsverzeichnis}
\tableofcontents
\newtheorem{satz}{Satz}[chapter]
\newtheorem{hilfssatz}[satz]{Hilfssatz}
\newtheorem{definition}[satz]{Definition}
\newtheorem{annahme}[satz]{Annahme}
\newtheorem{aufgabe}[satz]{Aufgabe}
% Beispiel
\newenvironment{beispiel}[1][Beispiel]{%
\begin{proof}[\bf #1]%
\renewcommand{\qedsymbol}{$\bigcirc$}%
}{\end{proof}}
\mainmatter
\allowdisplaybreaks
\input einleitung.tex
\input endlichekoerper.tex
\input homologie.tex
\input lie.tex
\input nichtstandardanalysis.tex
\vfill
\pagebreak
\ifodd\value{page}\else\null\clearpage\fi
\input supplement.ind
\appendix
\end{document}
