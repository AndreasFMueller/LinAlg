%
% hilbertraum.tex -- Theorie der Hilbertr"aume
%
% (c) 2017 Prof Dr Andreas Müller, Hochschule Rapperswil
%
\chapter{Hilbertr"aume
\label{chapter:hilbertspaces}}
Die Theorie der Vektorr"aume mit einem Skalarprodukt wird im Unterricht
"ublicherweise nur am Beispiel des Vektorraums $\mathbb R^n$ dargestellt.
Dabei werden die Studierenden kurze Zeit sp"ater mit der Theorie
der Fourier-Reihen konfrontiert, die in nat"urlicher Weise in einem
unendlichdimensionalen Vektorraum mit einem Skalarprodukt situiert ist.

\section{Skalarprodukt und Norm}
Da die Theorie eines Vektorraumes mit Skalarprodukt formuliert werden
soll unabh"angig von der konkreten Darstellung der Vektoren als 
Spaltenvektoren, muss das Skalaprodukt ebenfalls unabh"angig von einer
solchen Darstellung definiert werden.
Die axiomatische Methode bietet sich hierf"ur an.

\begin{definition}
\label{hilbert:hermitescheform}
Sei $V$ ein komplexer Vektorraum.
Eine Funktion $\langle\;,\;\rangle\colon V\times V\to\mathbb C$ heisst
{\em hermitesch}, wenn sie folgende Eigenschaften hat
\begin{enumerate}[label={\bf HF.\arabic*},itemsep=0mm]
\item $\langle\;,\;\rangle$ ist linear im ersten und konjugiert-linear
im zweiten Argument:
\begin{align*}
\langle x+x',y\rangle &=\langle x,y\rangle + \langle x',y\rangle
&
\langle x,y+y'\rangle &=\langle x,y\rangle + \langle x,y'\rangle
\\
\langle\lambda x,y\rangle&=\lambda\langle x,y\rangle
&
\langle x,\lambda y\rangle&=\overline\lambda\langle x,y\rangle
\end{align*}
f"ur alle $x,x',y,y'\in V$.
\item F"ur alle $x,y\in V$ gilt
$\langle x,y\rangle=\overline{\langle y,x\rangle}$.
\end{enumerate}
\end{definition}
Eine hermitsche Form ist noch nicht geeignet, die Rolle eines
Skalarproduktes zu "ubernehmen, denn wir m"ochten ja $\langle x,x\rangle$
als Quadrat der L"ange eines Vektors $x$ interpretieren.
Dazu muss zun"achst mal sichergestellt sein, dass $\langle x,x\rangle$
eine reelle Zahl ist.
Doch dies folgt aus Axiom (HF.2) einer hermiteschen Form.
Vertauscht man die beiden Argument in $\langle x,x\rangle$, "andert
sich nat"urlich nichts, und daher folgt
\[
\langle x,x\rangle = \overline{\langle x,x\rangle}
\qquad\Rightarrow\qquad
\langle x,x\rangle\in\mathbb R.
\]

Damit $\langle x,x\rangle$ das Quadrat der L"ange sein kann, darf
$\langle x,x\rangle$ nicht negativ wird, was
durch die Axiome nicht sichergestellt wird.
Ist n"amlich $\langle\;,\;\rangle$ eine hermitesche Form, dann ist
auch $-\langle\;,\;\rangle$ eine hermitesche Form.
Die Definition muss daher verfeinert werden:

\begin{definition}
\label{hilbert:postiveform}
Eine hermitesche Form heisst {\em positiv}, wenn gilt
$\langle x,x\rangle \ge 0$ f"ur alle $x\in V$.
\end{definition}

Positive hermitesche Formen haben bereits eine wichtige Eigenschaft,
die f"ur die Definition der L"ange erforderlich.

\begin{satz} Ist $\langle\;,\;\rangle$ eine positive hermitesche
Form, dann gilt die Cauchy-Schwarz-Ungleichung
\[
|\langle x,y\rangle|^2 \le \langle x,x\rangle\langle y,y\rangle
\]
f"ur beliebige Vektoren $x,y\in V$.
\end{satz}

\begin{proof}[Beweis]
Wir betrachten den Vektor $x+\lambda y$.
Da $\langle\;,\;\rangle$ positiv ist, gilt 
$\langle x+\lambda y,x+\lambda y\rangle \ge 0$.
Mit der Linearit"at von $\langle\;,\;\rangle$ k"onnen wir dies
aber auch ausrechnen:
\begin{align*}
0&\le
\langle x+\lambda y,x+\lambda y\rangle
\\
&=
\langle x,x\rangle
+
\langle \lambda y,x\rangle
+
\langle x,\lambda y\rangle
+
\langle \lambda y,\lambda y\rangle
\\
&=
\langle x,x\rangle
+
\lambda\langle y,x\rangle
+
\overline{\lambda}\langle x,y\rangle
+
\lambda\overline{\lambda}\langle y,y\rangle
\\
&=
\langle x,x\rangle
+
\lambda\langle y,x\rangle
+
\overline{\lambda}\langle y,x\rangle
+
|\lambda|^2\langle y,y\rangle
\end{align*}
Falls $\langle y,y\rangle >0$
setzt man $\lambda = - \langle x,y\rangle/\langle y,y\rangle$.
Dann folgt
\begin{align*}
0
&\le
\langle x,x\rangle
-
2\frac{\langle x,y\rangle}{\langle y,y\rangle}\langle y,x\rangle
+
\frac{|\langle y,x\rangle|^2}{\langle y,y\rangle^2}\langle y,y\rangle
\\
&=
\langle x,x\rangle
-2
\frac{|\langle x,y\rangle|^2}{\langle y,y\rangle}
+
\frac{|\langle y,x\rangle|^2}{\langle y,y\rangle^2}\langle y,y\rangle
\\
&=
\langle x,x\rangle
-
\frac{|\langle x,y\rangle|^2}{\langle y,y\rangle}.
\end{align*}
Durch Multiplizieren mit $\langle y,y\rangle$ erh"alt man daraus
\begin{align*}
0&\le
\langle x,x\rangle\langle y,y\rangle
-
|\langle x,y\rangle|^2
\\
\Rightarrow\qquad
|\langle x,y\rangle|^2
&\le
\langle x,x\rangle\langle y,y\rangle.
\end{align*}

"Ahnlich kann vorgegangen werden, wenn $\langle x,x\rangle > 0$ ist und
$\langle y,y\rangle =0$.

Im Fall $\langle x,x\rangle=\langle y,y\rangle=0$ verwendet man 
$\lambda = -\langle x,y\rangle$ und erh"alt
\begin{align*}
0
&\le
\langle x,x\rangle
+
\lambda\langle y,x\rangle
+
\overline{\lambda}\langle y,x\rangle
+
|\lambda|^2\langle y,y\rangle
\\
&=
-\langle x,y\rangle\langle y,x\rangle
-\langle y,x\rangle\langle x,y\rangle
=
-2|\langle x,y\rangle|^2
\\
\Rightarrow\qquad
2|\langle x,y\rangle|^2
&\le 0.
\end{align*}
Daraus folgt $\langle x,y\rangle = 0$, ein Spezialfall der
Cauchy-Schwarz-Ungleichung.
\end{proof}

Auch mit dieser Definition ist es immer noch m"oglich, dass Vektoren,
die von $0$ verschieden sind, Skalarprodukt $0$ haben.
Dies verhindert die folgend Erweiterung der Definition

\begin{definition}
\label{hilbert:postivdefiniteform}
Eine hermitesche Form $\langle\;,\;\rangle$ heisst {\em positiv definit},
wenn $\langle x,x\rangle > 0$ f"ur alle Vektoren $x\ne 0$ in $V$.
\end{definition}

Eine positiv definite hermitesche Form erlaubt nun die Norm
eines Vektors zu definieren

\begin{definition}
Ist $\langle \;,\;\rangle$ eine positiv definite hermitesche Form
auf $V$, dann heisst $\|x\| = \sqrt{\langle x,x\rangle}$ die {\em Norm}
des Vektors $x\in V$.
\end{definition}

Diese Norm erf"ullt die Dreiecksungleichung, wie man mit folgender
Rechnung einsehen kann:
\begin{align*}
\|x+y\|^2
&=
\langle x+y,x+y\rangle
\\
&=
\langle x,x\rangle
+
\langle x,y\rangle
+
\langle y,x\rangle
+
\langle y,y\rangle
\\
&=
\|x\|^2 + \|y\|^2
+ 2\operatorname{Re}\langle x,y\rangle
\\
&\le
\|x\|^2 + 2\|x\|\cdot \|y\| + \|y\|^2 = (\|x\| + \|y\|)^2
\\
\Rightarrow\qquad
\|x+y\|
&\le
\|x\| + \|y\|
\end{align*}
Damit ist die Dreiecksungleichung bewiesen.

\begin{definition}
Ein (komplexer) Pr"ahilbertraum ist ein komplexer Vektorraum mit
einer positiv definiten hermiteschen Form $\langle\;,\;\rangle$
auch genannt das {\em Skalarprodukt} des Pr"ahlibertraumes.
\end{definition}

\section{Hilbertraum}
Die rationalen komplexen Zahlen $\mathbb Q(i)=\{a+bi\,|\,a,b\in\mathbb Q\}$
k"onnen als komplexer Hilbertraum mit dem Skalarprodukt
$\langle x,y\rangle = \overline{x}y$ betrachtet werden.
Die Norm von $x=a+bi$ ist $\|x\|^2=a^2+b^2$.
Auf den rationalen Zahlen ist die Norm genau der Betrag, Cauchy-Folgen
in $\mathbb Q$ sind also automatisch auch Cauchy-Folgen im
Pr"ahilbertraum $\mathbb Q(i)$.
Es ist aber auch wohlbekannt dass, eine Cauchy-Folge in $\mathbb Q$
nicht konvergent sein muss.

Wir m"ochten den Begriff des Hilbertraums als die nat"urlich B"uhne zum
Beispiel f"ur die Fourier-Theorie etablieren, in der das Summieren von
Reihen wie einer Fourier-Reihe
\[
f(x) = \frac{a_0}2+\sum_{k=1}^\infty (a_k\cos kx+b_k\sin kx)
\]
wohldefiniert sein soll.
Wir m"ussen daher fordern, dass Cauchy-Folgen einen Grenzwert haben,
und definieren daher eine Hilbertraum wie folgt.

\begin{definition}
Ein Pr"ahilbertraum $H$ heisst ein {\em Hilbertraum}, wenn jede er 
vollst"andig ist, d.~h.~wenn jede Cauchy-Folge einen Grenzwert in $H$ hat.
\end{definition}

\begin{beispiel}[Endlichdimensionale komplexe Vektorr"aume]
Der Vektorraum $\mathbb C^n$ ist in nat"urlicher Weise ein Pr"ahilbertraum
mit dem Skalarprodukt $\langle u,v\rangle = \overline{u}^t v$.
Da $\mathbb C$ vollst"andig ist, ist $\mathbb C^n$ aber auch ein
Hilbertraum.
\end{beispiel}

\begin{beispiel}[Periodische Funktionen]
Sei $C(S^1, \mathbb C)$ die Menge der stetigen, komplexwertigen Funktionen 
von $S^1 = \mathbb R/2\pi\mathbb Z$.
Das Skalarprodukt
\[
\langle f,g\rangle = \frac1{\sqrt{2\pi}}\int_0^{2\pi} \overline{f(x)} g(x)\,dx
\]
macht $C(S^1,\mathbb C)$ zu einem Pr"ahlibertraum.
Er kann aber kein Hilbertraum sein, denn die Funktionenfolge
\[
f_n(x)=\root{2n+1}\of{\sin x},\qquad n\in\mathbb N
\]
konvergiert gegen die Rechteckfunktion
\[
f(x)=\begin{cases}
1&\qquad 0<x<\pi\\
-1&\qquad \pi<x<2\pi\\
0&\qquad\text{sonst}
\end{cases}
\]
Diese Funktion ist nicht stetig, also nicht in $C(S^1,\mathbb C)$, es existiert
also zwar ein Grenzwert, aber er liegt nicht im Raum $H$ drin.
\end{beispiel}

\section{Orthonormalbasis}

\section{Operatoren auf einem Hilbert-Raum}

\section{Unit"are Operatoren}

\section{Fourier-Transformation}

\section{Spektral-Theorie}




