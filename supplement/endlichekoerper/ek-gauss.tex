%
% ek-gauss.tex -- Anwendungen des Gauss-Algorithmus in endlichen Körpern
%
% (c) 2017 Prof Dr Andreas Mueller, Hochschule Rapperswil
%
\section{Gauss-Algorithmus in $\mathbb F_p$}
Der Gauss-Algorithmus ist die Basis sehr vieler Untersuchungen in der
linearen Algebra.
Man kann damit Gleichungssysteme lösen, Determinanten bestimmen oder
Matrizen in einfachere Faktoren zerlegen.
Er verwendet nur Körper-Operationen, und ist daher unmittelbar
auf endliche Körper übertragbar.
Wir illustrieren dies an Hand einiger Beispiele.

\subsection{Ein Gleichungssystem über $\mathbb F_2$}
Besonders einfach ist die Arithmetik im Körper $\mathbb F_2$,
da es nur ein einziges Element gibt, welches von $0$ verschieden ist,
nämlich $1$.
Ausserdem ist die Addition nichts anderes als die XOR-Verknüpfung.

Man kann also das folgende Gleichungssystem über $\mathbb F_2$
\begin{equation}
\begin{linsys}{3}
x_1& &   &+&x_3&=&1\\
x_1&+&x_2& &   &=&1\\
   & &x_2& &   &=&1\\
\end{linsys}
\label{ffield:gleichung}
\end{equation}
mit dem Gauss-Algorithmus wie folgt lösen:
\begin{align*}
\begin{tabular}{|>{$}c<{$}>{$}c<{$}>{$}c<{$}|>{$}c<{$}|}
\hline
1&0&1&1\\
1&1&0&1\\
0&1&0&1\\
\hline
\end{tabular}
&
\rightarrow
\begin{tabular}{|>{$}c<{$}>{$}c<{$}>{$}c<{$}|>{$}c<{$}|}
\hline
1&0&1&1\\
0&1&1&0\\
0&1&0&1\\
\hline
\end{tabular}
\rightarrow
\begin{tabular}{|>{$}c<{$}>{$}c<{$}>{$}c<{$}|>{$}c<{$}|}
\hline
1&0&1&1\\
0&1&1&0\\
0&0&1&1\\
\hline
\end{tabular}
\rightarrow
\begin{tabular}{|>{$}c<{$}>{$}c<{$}>{$}c<{$}|>{$}c<{$}|}
\hline
1&0&0&0\\
0&1&0&1\\
0&0&1&1\\
\hline
\end{tabular}
\end{align*}
Daraus kann man die Lösung
\[
\begin{pmatrix}x_1\\x_2\\x_3\end{pmatrix}
=
\begin{pmatrix}0\\1\\1\end{pmatrix}
\]
ablesen.

Natürlich kann man auch die inverse Matrix bestimmen:
\begin{align*}
\begin{tabular}{|>{$}c<{$}>{$}c<{$}>{$}c<{$}|>{$}c<{$}>{$}c<{$}>{$}c<{$}|}
\hline
1&0&1&1&0&0\\
1&1&0&0&1&0\\
0&1&0&0&0&1\\
\hline
\end{tabular}
&
\rightarrow
\begin{tabular}{|>{$}c<{$}>{$}c<{$}>{$}c<{$}|>{$}c<{$}>{$}c<{$}>{$}c<{$}|}
\hline
1&0&1&1&0&0\\
0&1&1&1&1&0\\
0&1&0&0&0&1\\
\hline
\end{tabular}
\rightarrow
\begin{tabular}{|>{$}c<{$}>{$}c<{$}>{$}c<{$}|>{$}c<{$}>{$}c<{$}>{$}c<{$}|}
\hline
1&0&1&1&0&0\\
0&1&1&1&1&0\\
0&0&1&1&1&1\\
\hline
\end{tabular}
\\
&
\rightarrow
\begin{tabular}{|>{$}c<{$}>{$}c<{$}>{$}c<{$}|>{$}c<{$}>{$}c<{$}>{$}c<{$}|}
\hline
1&0&0&0&1&1\\
0&1&0&0&0&1\\
0&0&1&1&1&1\\
\hline
\end{tabular}
\end{align*}
Daraus liest man ab
\[
\begin{pmatrix}
1&0&1\\
1&1&0\\
0&1&0
\end{pmatrix}^{-1}
=
\begin{pmatrix}
0&1&1\\
0&0&1\\
1&1&1\\
\end{pmatrix}.
\]
Auch die eben gefundene Lösung des Gleichungssystems~\eqref{ffield:gleichung}
kann jetzt mit der inversen Matrix bestimmt werden:
\[
\begin{pmatrix}x_1\\x_2\\x_3\end{pmatrix}
=
\begin{pmatrix}
0&1&1\\
0&0&1\\
1&1&1\\
\end{pmatrix}
\begin{pmatrix}1\\1\\1\end{pmatrix}
=
\begin{pmatrix}0\\1\\1\end{pmatrix}.
\]

\subsection{LU- und LR-Zerlegung in $\mathbb F_5$}
Die Operationen in $\mathbb F_p$ werden durch die folgenden Additions-
bzw.~Multiplikationstabellen beschrieben.
\begin{center}
\begin{tabular}{|>{$}c<{$}|>{$}c<{$}>{$}c<{$}>{$}c<{$}>{$}c<{$}>{$}c<{$}|}
\hline
+&0&1&2&3&4\\
\hline
0&0&1&2&3&4\\
1&1&2&3&4&0\\
2&2&3&4&0&1\\
3&3&4&0&1&2\\
4&4&0&1&2&3\\
\hline
\end{tabular}
\qquad
\begin{tabular}{|>{$}c<{$}|>{$}c<{$}>{$}c<{$}>{$}c<{$}>{$}c<{$}>{$}c<{$}|}
\hline
\cdot&0&1&2&3&4\\
\hline
   0 &0&0&0&0&0\\
   1 &0&1&2&3&4\\
   2 &0&2&4&1&3\\
   3 &0&3&1&4&2\\
   4 &0&4&3&2&1\\
\hline
\end{tabular}
\end{center}
Damit ist es jetzt einfach, den Algorithmus zur Bestimmung der LU- und
LR-Zerlegung durchzuführen.
Wir suchen die LU- und die LR-Zerlegung der Matrix
\[
A=\begin{pmatrix}
2&2&4\\
2&3&2\\
0&3&3
\end{pmatrix}.
\]
Der Gauss-Algorithmus liefert
\begin{align*}
\begin{tabular}{|>{$}c<{$}>{$}c<{$}>{$}c<{$}|}
\hline
2&2&4\\
2&3&2\\
0&3&3\\
\hline
\end{tabular}
&
\rightarrow
\begin{tabular}{|>{$}c<{$}>{$}c<{$}>{$}c<{$}|}
\hline
1&1&2\\
0&1&3\\
0&3&3\\
\hline
\end{tabular}
\rightarrow
\begin{tabular}{|>{$}c<{$}>{$}c<{$}>{$}c<{$}|}
\hline
1&1&2\\
0&1&3\\
0&0&4\\
\hline
\end{tabular}
\end{align*}
Daraus liest man die LU-Zerlegung ab:
\[
L
=
\begin{pmatrix}
2&0&0\\
2&1&0\\
0&3&4
\end{pmatrix},
\qquad
U
=
\begin{pmatrix}
1&1&2\\
0&1&3\\
0&0&1
\end{pmatrix}
\qquad
\Rightarrow
\qquad
LU=
\begin{pmatrix}
2&0&0\\
2&1&0\\
0&3&4
\end{pmatrix}
\begin{pmatrix}
1&1&2\\
0&1&3\\
0&0&1
\end{pmatrix}
=
\begin{pmatrix}
2&2&4\\
2&3&2\\
0&3&3
\end{pmatrix}
\]
Für die LR-Zerlegung muss $U$ mit $\operatorname{diag}(2,1,4)$
multipliziert werden und $L$ mit der Inversen:
\begin{align*}
L'
&=
L\operatorname{diag}(2,1,4)^{-1}
=
\begin{pmatrix}
2&0&0\\
2&1&0\\
0&3&4
\end{pmatrix}
\begin{pmatrix}
3&0&0\\
0&1&0\\
0&0&4
\end{pmatrix}
=
\begin{pmatrix}
1&0&0\\
1&1&0\\
0&3&1
\end{pmatrix}
\\
R'
&=
\operatorname{diag}(2,1,4)
\begin{pmatrix}
1&1&2\\
0&1&3\\
0&0&1
\end{pmatrix}
=
\begin{pmatrix}
2&0&0\\
0&1&0\\
0&0&4
\end{pmatrix}
\begin{pmatrix}
1&1&2\\
0&1&3\\
0&0&1
\end{pmatrix}
=
\begin{pmatrix}
2&2&4\\
0&1&3\\
0&0&4
\end{pmatrix}
\end{align*}
Kontrolle:
\[
L'R'
=
\begin{pmatrix}
1&0&0\\
1&1&0\\
0&3&1
\end{pmatrix}
\begin{pmatrix}
2&2&4\\
0&1&3\\
0&0&4
\end{pmatrix}
=
\begin{pmatrix}
2&2&4\\
2&3&2\\
0&3&3
\end{pmatrix}.
\]

\subsection{Inverse Matrix in $\mathbb F_7$}
Die Additions- und Multiplikationstabellen für $\mathbb F_7$ sind
\begin{center}
\begin{tabular}{|>{$}c<{$}|>{$}c<{$}>{$}c<{$}>{$}c<{$}>{$}c<{$}>{$}c<{$}>{$}c<{$}>{$}c<{$}|}
\hline
+&0&1&2&3&4&5&6\\
\hline
0&0&1&2&3&4&5&6\\
1&1&2&3&4&5&6&0\\
2&2&3&4&5&6&0&1\\
3&3&4&5&6&0&1&2\\
4&4&5&6&0&1&2&3\\
5&5&6&0&1&2&3&4\\
6&6&0&1&2&3&4&5\\
\hline
\end{tabular}
\qquad
\begin{tabular}{|>{$}c<{$}|>{$}c<{$}>{$}c<{$}>{$}c<{$}>{$}c<{$}>{$}c<{$}>{$}c<{$}>{$}c<{$}|}
\hline
\cdot&0&1&2&3&4&5&6\\
\hline
  0  &0&0&0&0&0&0&0\\
  1  &0&1&2&3&4&5&6\\
  2  &0&2&4&6&1&3&5\\
  3  &0&3&6&2&5&1&4\\
  4  &0&4&1&5&2&6&3\\
  5  &0&5&3&1&6&4&2\\
  6  &0&6&5&4&3&2&1\\
\hline
\end{tabular}
\end{center}
Damit können wir die inverse Matrix von
\[
A
=
\begin{pmatrix}
3&6&5\\
3&1&0\\
0&6&1
\end{pmatrix}
\]
mit dem Gauss-Algorithmus bestimmen:
\begin{align*}
\begin{tabular}{|ccc|ccc|}
\hline
3&6&5&1&0&0\\
3&1&0&0&1&0\\
0&6&1&0&0&1\\
\hline
\end{tabular}
&
\rightarrow
\begin{tabular}{|ccc|ccc|}
\hline
1&2&4&5&0&0\\
0&2&2&6&1&0\\
0&6&1&0&0&1\\
\hline
\end{tabular}
\rightarrow
\begin{tabular}{|ccc|ccc|}
\hline
1&2&4&5&0&0\\
0&1&1&3&4&0\\
0&0&2&3&4&1\\
\hline
\end{tabular}
\\
&
\rightarrow
\begin{tabular}{|ccc|ccc|}
\hline
1&2&0&6&6&5\\
0&1&0&5&2&3\\
0&0&1&5&2&4\\
\hline
\end{tabular}
\rightarrow
\begin{tabular}{|ccc|ccc|}
\hline
1&0&0&3&2&6\\
0&1&0&5&2&3\\
0&0&1&5&2&4\\
\hline
\end{tabular}
\end{align*}
Daraus liest man ab:
\[
A^{-1}
=
\begin{pmatrix}
3&2&6\\
5&2&3\\
5&2&4
\end{pmatrix}
\qquad
\Rightarrow
\qquad
AA^{-1}
=
\begin{pmatrix}
3&6&5\\
3&1&0\\
0&6&1
\end{pmatrix}
\begin{pmatrix}
3&2&6\\
5&2&3\\
5&2&4
\end{pmatrix}
=
\begin{pmatrix}
64&28&56\\
14& 8&21\\
35&14&22
\end{pmatrix}
=
\begin{pmatrix}
1&0&0\\
0&1&0\\
0&0&1
\end{pmatrix}.
\]
Alternativ können wir dafür auch Minoren verwenden.
Dazu brauchen wir zunächst die Determinante, die wir mit der Sarrus-Formel
berechnen können:
\begin{align*}
\det(A)
&
=
\left|
\begin{matrix}
3&6&5\\
3&1&0\\
0&6&1
\end{matrix}
\right|
=
3+5\cdot3\cdot6-1\cdot3\cdot 6
=
3+6-4=5.
\end{align*}
Damit wird die inverse Matrix
\begin{align*}
A^{-1}
&=
\frac1{\det(A)}
{
\def\arraystretch{2.2}
\begin{pmatrix}
\def\arraystretch{1}
\phantom{-}
\left|\begin{matrix} 1&0\\6&1 \end{matrix}\right|&
\def\arraystretch{1}
-
\left|\begin{matrix} 6&5\\6&1 \end{matrix}\right|&
\def\arraystretch{1}
\phantom{-}
\left|\begin{matrix} 6&5\\1&0 \end{matrix}\right|
\\
\def\arraystretch{1}
-
\left|\begin{matrix} 3&0\\0&1 \end{matrix}\right|&
\def\arraystretch{1}
\phantom{-}
\left|\begin{matrix} 3&5\\0&1 \end{matrix}\right|&
\def\arraystretch{1}
-
\left|\begin{matrix} 3&5\\3&0 \end{matrix}\right|
\\
\def\arraystretch{1}
\phantom{-}
\left|\begin{matrix} 3&1\\0&6 \end{matrix}\right|&
\def\arraystretch{1}
-
\left|\begin{matrix} 3&6\\0&6 \end{matrix}\right|&
\def\arraystretch{1}
\phantom{-}
\left|\begin{matrix} 3&6\\3&1 \end{matrix}\right|
\end{pmatrix}
}
=
3\cdot
\begin{pmatrix}
 1\cdot 1-0\cdot 6&-6\cdot 1+5\cdot 6& 6\cdot 0-5\cdot 1\\
-3\cdot 1+0\cdot 0& 3\cdot 1-5\cdot 0&-3\cdot 0+5\cdot 3\\
 3\cdot 6-1\cdot 0&-3\cdot 6+6\cdot 0& 3\cdot 1-6\cdot 3
\end{pmatrix}
\\
&=
3\cdot
\begin{pmatrix}
1&3&2\\
4&3&1\\
4&3&6
\end{pmatrix}
=
\begin{pmatrix}
3&2&6\\
5&2&3\\
5&2&4
\end{pmatrix},
\end{align*}
in "Ubereinstimmung mit der Rechnung mit dem Gauss-Algorithmus.

