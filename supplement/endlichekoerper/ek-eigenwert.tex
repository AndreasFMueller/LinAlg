%
% ek-eigenwert.tex -- Eigenwertproblem in endlichen Körpern
%
% (c) 2017 Prof Dr Andreas Mueller, Hochschule Rapperswil
%
\section{Eigenwerte und Eigenvektoren}
\rhead{Eigenwerte und Eigenvektoren}
Auch die Theorie der Eigenwerte und Eigenvektoren kann auf endliche Körper
ausgedehnt werden.
Der erste Schritt bei der Bestimmung der Eigenvektoren verlangt, dass
die Nullstellen des charakteristischen Polynoms bestimmt werden.
Wie schon bei Polynomen über $\mathbb Q$ ist die Bestimmung der Nullstellen
von Polynomen über $\mathbb F_p$ alles andere als einfach.
Sie wird aber noch kompliziert dadurch, dass über $\mathbb F_p$
zusätzlich der Frobenius-Automorphismus bei $p$-ten Potenzen zu
neuen, unintuitiven Effekten führt.

\subsection{Eigenvektoren über $\mathbb F_7$}
Wir suchen Eigenwerte und Eigenvektoren der Matrix
\[
A
=
\begin{pmatrix}
0&1&4\\
3&5&6\\
3&4&4
\end{pmatrix}
\]
über $\mathbb F_7$.

Wir berechnen zuerst das charakteristische Polynom:
\begin{align*}
\det(A-\lambda E)
&
=
\left|\begin{matrix}
-\lambda&1        &4\\
3       &5-\lambda&6\\
3       &4        &4-\lambda
\end{matrix}\right|
\\
&=
-\lambda(5-\lambda)(4-\lambda)
+1\cdot6\cdot3
+4\cdot3\cdot 4
-3\cdot(5-\lambda)\cdot 4
+4\cdot6\cdot\lambda
-(4-\lambda)\cdot3\cdot 1
\\
&=
-\lambda(6-2\lambda+\lambda^2)
\\
&=
-6\lambda+2\lambda^2-\lambda^3
+4
+6
-4+5\lambda
+3\lambda
-5+3\lambda
\\
&=
-\lambda^3+2\lambda^2+5\lambda+1.
\end{align*}
Um die Eigenwerte zu finden, müssen wir also die Lösungen der
Gleichung
\[
-\chi_A(\lambda)
=
\lambda^3
-2\lambda^2
-5\lambda
-1
=
\lambda^3
+5\lambda^2
+2\lambda
+6
=
0
\]
finden.
Da $\mathbb F_7$ nur $7$ Elemente hat, kann man alle Werte durchprobieren,
und findet $1$, $3$ und $5$ als Nullstellen.
Zur Kontrolle berechnen wir das Produkt
\begin{align*}
(\lambda -1)(\lambda-3)(\lambda-5)
&
=
(\lambda^2-4\lambda+3)(\lambda-5)
\\
&=
\lambda^3-4\lambda^2+3\lambda
-5\lambda^2+6\lambda-1
\\
&=
\lambda^3+5\lambda^2+2\lambda+6
\\
&=
-\chi_A(\lambda),
\end{align*}
wir haben also alle Eigenwerte gefunden.

Zu jedem Eigenwert müssen jetzt noch ein Eigenvektor gefunden werde.
Für $\lambda=1$ verwenden wir 
\begin{align*}
\begin{tabular}{|>{$}c<{$}>{$}c<{$}>{$}c<{$}|}
\hline
0-1&1&4\\
3&5-1&6\\
3&4&4-1\\
\hline
\end{tabular}
&=
\begin{tabular}{|>{$}c<{$}>{$}c<{$}>{$}c<{$}|}
\hline
6&1&4\\
3&4&6\\
3&4&3\\
\hline
\end{tabular}
\rightarrow
\begin{tabular}{|>{$}c<{$}>{$}c<{$}>{$}c<{$}|}
\hline
1&6&3\\
0&0&4\\
0&0&1\\
\hline
\end{tabular}
\rightarrow
\begin{tabular}{|>{$}c<{$}>{$}c<{$}>{$}c<{$}|}
\hline
1&6&3\\
0&0&1\\
0&0&0\\
\hline
\end{tabular}
\rightarrow
\begin{tabular}{|>{$}c<{$}>{$}c<{$}>{$}c<{$}|}
\hline
1&6&0\\
0&0&1\\
0&0&0\\
\hline
\end{tabular}
\end{align*}
Daraus liest man ab, dass die dritte Komponente verschwinden muss, und dass
die zweite Variable frei wählbar ist.
Wählen wir sie als $1$, dann ist der Eigenvektor zum Eigenwert $\lambda=1$
\[
v_1
=
\begin{pmatrix}1\\1\\0\end{pmatrix}
\qquad\text{Kontrolle:}\qquad
Av_1
=
\begin{pmatrix}
0&1&4\\
3&5&6\\
3&4&4
\end{pmatrix}
\begin{pmatrix}1\\1\\0\end{pmatrix}
=
\begin{pmatrix}1\\1\\0\end{pmatrix}
\]

Für $\lambda = 3$ bestimmen wir einen Eigenvektor mit dem folgenden
Gauss-Tableau:
\begin{align*}
\begin{tabular}{|>{$}c<{$}>{$}c<{$}>{$}c<{$}|}
\hline
0-3&1&4\\
3&5-3&6\\
3&4&4-3\\
\hline
\end{tabular}
&=
\begin{tabular}{|>{$}c<{$}>{$}c<{$}>{$}c<{$}|}
\hline
4&1&4\\
3&2&6\\
3&4&1\\
\hline
\end{tabular}
\rightarrow
\begin{tabular}{|>{$}c<{$}>{$}c<{$}>{$}c<{$}|}
\hline
1&2&1\\
0&3&3\\
0&5&5\\
\hline
\end{tabular}
\rightarrow
\begin{tabular}{|>{$}c<{$}>{$}c<{$}>{$}c<{$}|}
\hline
1&2&1\\
0&1&1\\
0&0&0\\
\hline
\end{tabular}
\rightarrow
\begin{tabular}{|>{$}c<{$}>{$}c<{$}>{$}c<{$}|}
\hline
1&0&6\\
0&1&1\\
0&0&0\\
\hline
\end{tabular}
\end{align*}
Diesmal ist die dritte Komponente frei wählbar, wir wählen sie wieder als
$1$ und erhalten den Eigenvektor zum Eigenwert $\lambda=3$
\[
v_3=\begin{pmatrix}1\\6\\1\end{pmatrix}
\qquad\text{Kontrolle:}\qquad
Av_3
=
\begin{pmatrix}
0&1&4\\
3&5&6\\
3&4&4
\end{pmatrix}
\begin{pmatrix}1\\6\\1\end{pmatrix}
=
\begin{pmatrix}3\\4\\3\end{pmatrix}
=
3\begin{pmatrix}1\\6\\1\end{pmatrix}
=
3v_3.
\]

Schliesslich untersuchen wir den Eigenwert $\lambda=5$.
\begin{align*}
\begin{tabular}{|>{$}c<{$}>{$}c<{$}>{$}c<{$}|}
\hline
0-3&1&4\\
3&5-3&6\\
3&4&4-3\\
\hline
\end{tabular}
&=
\begin{tabular}{|>{$}c<{$}>{$}c<{$}>{$}c<{$}|}
\hline
2&1&4\\
3&0&6\\
3&4&6\\
\hline
\end{tabular}
\rightarrow
\begin{tabular}{|>{$}c<{$}>{$}c<{$}>{$}c<{$}|}
\hline
1&4&2\\
0&2&0\\
0&6&0\\
\hline
\end{tabular}
\rightarrow
\begin{tabular}{|>{$}c<{$}>{$}c<{$}>{$}c<{$}|}
\hline
1&0&2\\
0&1&0\\
0&0&0\\
\hline
\end{tabular}
\end{align*}
Die zweite Komponente ist $0$, die dritte frei wählber.
\[
v_5
=
\begin{pmatrix}5\\0\\1\end{pmatrix}
\qquad\text{Kontrolle:}\qquad
Av_5
=
\begin{pmatrix}
0&1&4\\
3&5&6\\
3&4&4
\end{pmatrix}
\begin{pmatrix}5\\0\\1\end{pmatrix}
=
\begin{pmatrix}4\\0\\5\end{pmatrix}
=
5\begin{pmatrix}5\\0\\1\end{pmatrix}
=
5v_5.
\]








