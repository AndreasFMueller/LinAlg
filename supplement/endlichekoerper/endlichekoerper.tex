%
% endlichekoerper.tex
%
% (c) 2017 Prof Dr Andreas Mueller, Hochschule Rapperswil
%
\chapter{Lineare Algebra in endlichen K"orpern
\label{chapter:endlichekoerper}}
\rhead{Endliche K"orper}

Die Konstruktionen in Kapitel~1 und 2 des Skripts verlangen nicht mehr
als die Grundoperationen.
Die vorgestellten Algorithmen sollten daher in jeder Zahlenmenge
durchf"uhrbar sein, sobald die arithmetischen Operationen zur
Verf"ugung stehen.
Zum Beispiel wird die Theorie zwar fast ausschliesslich f"ur reelle
Zahlen entwickelt, doch funktioniert sie genau gleich f"ur rationale
oder komplexe Zahlen.
Alle drei Zahlenmengen sind sogenannte K"orper, sie enthalten
unendlich viele Elemente.
Es zeigt sich, dass es auch K"orper gibt, die nur endliche viele
Elemente enthalten.
Die K"orper werden jeweils durch eine Primzahl $p$ charakterisiert.
In der Codierungstheorie und der Kryptographie haben vor allem die
K"orper zur Primzahl $p=2$ eine besondere Bedeutung.
In diesem Kapitel wird gezeigt, wie man in diesen K"orpern rechnet
und es wird an Beispielen illustriert, wie die wohlbekannten Algorithmen
der linearen Algebra sich auf diese K"orper "ubertragen.

%
% ek-grundlagen.tex -- Grundlagen zur Theorie der endlichen Körper
%
% (c) 2017 Prof Dr Andreas Mueller, Hochschule Rapperswil
%
\section{Endliche Körper}
Im Unterricht in linearer Algebra werden die Eigenschaften der reellen
Zahlen als Grundlage ohne weitere Diskussion akzeptiert.
Im Analysis-Unterricht wird etwas sorgfältiger analyisiert, was denn
genau für Eigenschaften notwendig sind.
Dabei werden erst die rationalen Zahlen $\mathbb Q$ als die Menge der
Brüche der ganzen Zahlen konstruiert.
In einem zweiten Schritt werden diese unter Verwendung der Ordnungsrelation
(Dedekindsche Schnitte)
oder des Abstandsbegriffs (Topologie, metrischer Raum) zu den reellen
Zahlen $\mathbb R$ vervollständigt.
Der zweite Schritt ist aus der Sicht der linearen Algebra nicht nötig.
Die lineare Algebra könnte auch gänzlich über den rationalen Zahlen
entwickelt werden.
Erst bei der Cholesky-Zerlegung oder beim Eigenwertproblem wird es
nötig, die rationalen Zahlen so zu erweitern, dass Quadratwurzeln
(Cholesky-Zerlegung) oder Nullstellen von Polynomen höheren Grades
gefunden werden können.

Die Analysis nutzt aus, dass die rationalen Zahlen nicht nur eine algebraische
Struktur haben, sondern auch eine Ordnungsstruktur.
Insbesondere ist es für zwei beliebige positive rationale Zahlen
$a$ und $b$ möglich, eine Zahl $N$ anzugeben, so dass $Na>b$.
Man sagt, die rationalen Zahlen bilden einen archimedischen Körper.
Die Ordnungsrelation ist also verträglich mit den Rechenoperationen.
In der Analysis äussert sich das dadurch, dass die Multiplikation mit
einer Zahl eine stetige Abbildung ist.
Für die lineare Algebra ist diese Eigenschaft jedoch bedeutungslos:
in den beiden ersten Kapiteln des Skriptes wird die Ordnungsrelation
kein einziges Mal gebraucht!

In diesem Kapitel soll gezeigt werde, wie man ausgehend von den
ganzen Zahlen $\mathbb Z$ auch andere Körper konstruieren kann,
für die die archimedische Eigenschaft nicht erfüllt ist.
Wir verwenden hier einen axiomatischen Zugang um ganz sicher zu sein,
dass wir die konstruierten Zahlkörper nicht versehentlich mit weiteren
Eigenschaften ausstatten, die nicht benötigt werden.
Erst im letzten Abschnitt über das Eigenwert-Problem werden wir
feststellen, dass hierfür die einfachen Körper erweitert werden
müssen.
Dies geschieht durch hinzufügen von geeigneten neuen Zahlen, den
Quadratwurzeln oder anderen Nullstellen.
Im Falle von $\mathbb R$ waren in $\mathbb R$ bereits alle denkbaren
Quadratwurzeln von positiven Zahlen vorhanden, und sie waren beliebig
nahe an rationalen Zahlen.
Da die endlichen Körper keine Ordnungsrelation haben, und damit auch
keinen (offensichtlichen) Abstandsbegriff, finden wir bereits die
notwendigen Quadratwurzeln ``weit ausserhalb'' des Ausgangskörpers.

%
% Definition eines Körpers und Beispiele
%
\subsection{Körper}
Der Begriff des Körpers fasst die Eigenschaften zusammen, die für
die Skalare in der linearen Algebra benötigt werden.

Eine {\em Gruppe} ist eine Menge $G$ mit einer Verknüpfung, die zwei Elementen
$a,b\in G$ das Element $ab\in G$ zuordnet.
Ausserdem müssen folgen Axiome erfüllt sein:
\begin{enumerate}[label={\bf G.\arabic*},itemsep=0mm]
\item
Die Verknüpfung ist assoziativ, d.~h.~$(ab)c=a(bc)$ für alle $a,b,c\in G$.
\item
Es gibt ein Element $e\in G$ mit der Eigenschaft $eg=ge=g$ für alle $g\in G$,
genant das Neutralelement.
\item
Für jedes Element $g\in G$ gibt es ein Element $g^{-1}\in G$ welches
$gg^{-1}=g^{-1}g=e$.
\end{enumerate}
Eine Gruppe heisst {\em abelsch} wenn $ab=ba$ für alle $a,b\in G$.

Ein {\em Ring} ist eine Menge $R$ mit zwei Verknüpfungen, der Addition
und der Multiplikation, mit folgenden Eigenschaften:
\begin{enumerate}[label={\bf R.\arabic*},itemsep=0mm]
\item $R$ ist bezüglich der Addition eine abelsche Gruppe.
\item Die Multiplikation in $R$ ist assoziativ und hat ein Einselement.
\item Für drei Element $x,y,z\in R$ gilt $(x+y)z=xz+yz$ und
$z(x+y)=zx+zy$.
\end{enumerate}

Die Menge $\mathbb Z$ der ganzen Zahlen trägt die Struktur eines Ringes.
Die Menge der $n\times n$-Matrizen mit Einträgen in $\mathbb Z$ ist ebenfalls
ein Ring.
Ist $R$ ein Ring, dann ist die Menge 
\[
R[X]=\{ a_0+a_1X +a_2X^2+\dots +a_nX^n\,|\,a_i\in R\}
\]
der Polynome in der Variablen $X$ ein Ring.

Ein {\em Körper} ist ein Ring, so dass die Menge der von $0$ verschiedenen
Elemente eine abelsche Gruppe bezüglich der Multiplikation bilden.

Die Menge $\mathbb Q$ der rationalen Zahlen trägt die Struktur eines
Körpers.

%
% Körper gebildet mit Restklassen
%
\subsection{Reste}
Die ganzen Zahlen $\mathbb Z$ bilden einen Ring.
Dies bleibt auch wahr für die Menge der Reste $\mathbb Z/n\mathbb Z$.
Die Restklasse $\llbracket r\rrbracket$ von $r$ ist die Menge
\[
\llbracket
r
\rrbracket
=
\{ z\in\mathbb Z\,|\, z\equiv r\mod n\}
\]
der ganzen Zahlen, die den gleichen Rest bei Teilung durch $n$ haben
wie $r$.
Mit Resten kann man wie gewohnt rechnen:
\begin{align*}
\llbracket a \rrbracket
\pm
\llbracket b \rrbracket
&=
\llbracket a\pm b \rrbracket
\\
\llbracket a \rrbracket
\llbracket b \rrbracket
&=
\llbracket ab \rrbracket
\end{align*}
Im allgemeinen ist die Menge der Reste kein Ring.
Ist nämlich $n=pq$ ein Produkt von Zahlen, dann ist das Produkt der
Restklassen
$\llbracket p\rrbracket$
und
$\llbracket q\rrbracket$
\[
\llbracket p\rrbracket
\llbracket q\rrbracket
=
\llbracket pq\rrbracket
=
\llbracket n\rrbracket
=
\llbracket 0\rrbracket.
\]
Insbesondere ist das Produkt der Restklassen von $p$ und $q$ die
Restklasse von $0$, es ist daher nicht möglich, ein multiplikativ
inverses Element für $\llbracket p \rrbracket$ zu finden.
Ursache für dieses pathologische Verhalten ist natürlich, dass $n=pq$
faktorisierbar ist.
Diese Möglichkeit entfällt, wenn $n$ eine Primzahl ist.
Tatsächlich gilt

\begin{satz}
Wenn $p$ eine Primzahl ist, dann ist der Restklassenring
$\mathbb Z/p\mathbb Z=\mathbb F_p$ ein Körper.
\end{satz}

%
% Charakteristik eines Körpers
%
\subsection{Charakteristik}

%
% Der Frobenius-Automorphismus ist eine zusätzliche Struktur, die nur die
% Körper mit Charaketeristik != 0 haben
%
\subsection{Frobenius-Automorphismus}
Für Körper mit Charakteristik $0$ ist das Potenzieren zwar bezüglich der
Multiplikation ein Homomorphismus:
$
(ab)^k = a^kb^k
$
aber nicht bezüglich der Addition, wo die binomische Formel
\begin{equation}
(a+b)^k
= 
a^k + \binom{k}{1} a^{k-1}b + \binom{k}{2} a^{k-2}b^2
+ \binom{k}{3}a^{k-3}b^3 + \dots + \binom{k}{k-1}ab^{k-1} + b^k
\label{ff:binom}
\end{equation}
gilt.
Man kann zeigen, dass die Binomial-Koeffizienten auf Zeile $k$ durch 
$k$ teilbar sind.
Reduzieren wir die binomische Formel \eqref{ff:binom} mit $k=p$ Modulo $p$,
dann folgt
\begin{align*}
(a+b)^p
&= 
a^p + \binom{p}{1} a^{p-1}b + \binom{p}{2} a^{p-2}b^2
+ \binom{p}{3}a^{p-3}b^3 + \dots + \binom{p}{p-1}ab^{p-1} + b^p
\\
&\equiv
a^p + 0\cdot a^{p-1}b + 0\cdot a^{p-1}b^2
+ 0\cdot a^{p-3}b^3 + \dots + 0\cdot ab^{p-1} + b^p
\quad\mod p
\\
&=
a^p + b^p.
\end{align*}
Im Körper $\mathbb F_p$ ist daher die Abbildung $a\mapsto a^p$
ein Homomorphismus von Körpern.
Er heisst der {\em Frobenius-Automorphismus}.

Speziell gilt auch, dass mehrfache Anwendung des Frobenius-Automorphismus
ebenfalls ein Automorphismus ist.
Die $k$-fache Anwendung des Frobenius-Automorphismus ist nichts anderes
als das erheben in die $p^k$-te Potenz.
Dies bedeutet, dass auch die die meisten Binomial-Koeffizienten zu $p^k$ 
durch $p$ teilbar sind:
\[
\binom{p^k}{l}\equiv 0\quad\mod p
\qquad
0<l<p^k.
\]

\begin{beispiel}[Beispiel: Frobenius-Automorphismus in $\mathbb F_2$]
In $\mathbb F_2$ besagt der Frobenius-Automorphismus, dass das Quadrieren
ein Automorphismus ist.
Da $0^2=0$ und $1^2=1$ ist, ist der Frobenius-Automorphismus in $\mathbb F_2$
die identische Abbildung.
\end{beispiel}

Der kleine Satz von Euler besagt, dass für eine Primzahl $p$ und jede
beliebige ganze Zahl $a$ gilt
\[
a^{p-1}\cong 1\mod p
\qquad\Rightarrow\qquad
a^p\cong a\mod p.
\]
Dies bedeutet, dass der Frobenius-Automorphismus auf dem Körper $\mathbb F_p$
immer wie die Identität wirkt.
Seine Wirkung wird also erst sichtbar, wenn man zu Körpererweiterungen
von $\mathbb F_p$ übergeht.
Im Lichte der Galois-Theorie ist $\mathbb F_p$ ein Fix-Körper
unter der Wirkung des Frobenius-Automorphismus.

\begin{beispiel}[Beispiel: Frobenius-Automorphismus in $\mathbb F_2(\alpha)$]
Wir konstruieren eine Körper-Erweiterung vom Grad zwei über dem Körper
$\mathbb F_2$.
Dazu verwenden betrachten wir das Polynom $m(x)=x^2 + x + 1$.
Wir suchen nach Nullstellen des Polynoms.
Durch Einsetzen von $1$ und $0$ in $m(x)$ kann man erkennen, dass
$m(x)$ keine Nullstellen $x\in\mathbb F_2$ hat.

Man kann dies alternativ auch einsehen, indem man versucht, das Polynom
$m(x)$ in Faktoren zu zerlegen.
Da es nur zwei Polynome ersten Grades gibt, kann man durch durchprobieren
aller Polynome ersten Grades herausfinden, ob $m(x)$ irreduzibel ist.
$m(x)$ ist offensichtlich nicht durch $x$ teilbar.
Andererseits ist $m(x)$ auch nicht durch $x+1$ teilbar, denn da $x$ kein
Faktor sein kann, müsste dann $m(x)=(x+1)^2=x^2 +  2x + 1=x^2+x$ sein,
was offensichtlich nicht zutrifft.
Somit ist $m(x)$ irreduzibel über $\mathbb F_2$.

Wir postulieren jetzt zwei neue ``Zahlen'' $\alpha$ und $\beta$, die
zu $\mathbb F_2$ hinzugefügt werden sollen, und die Nullstellen von 
$m(x)$ sein sollen.
Sie müssen daher die Gleichungen
\[
\left.
\begin{aligned}
\alpha^2 + \alpha +1&=0
\\
\beta^2 + \beta +1&=0
\end{aligned}
\right\}
\qquad\Rightarrow\qquad
\left\{
\begin{aligned}
\alpha^2&=\alpha+1\\
\beta^2&=\beta+1
\end{aligned}
\right.
\]
Die Zahl $\alpha+1$ ist ebenfalls eine Nullstelle von $m(x)$:
\[
m(\alpha+1)
=
(\alpha+1)^2+(\alpha+1) + 1
=
\underbrace{\alpha^2 + 1}_{\text{Frobenius}}\mathstrut + \alpha + 1 + 1
=
\underbrace{\alpha+1}_{\alpha^2} + 1+ (\alpha + 1) + 1=0.
\]
Da $\alpha+1\ne \alpha$ ist, aber auch eine Nullstelle, gibt es nur
noch die eine Möglichkeit, dass $\alpha+1=\beta$ ist.
Dann ist aber auch
\[
\beta^2
=
(\alpha + 1)^2
= 
\alpha^2 + 1
=
\alpha + 1 + 1
=
\alpha.
\]
Der Frobenius-Automorphismus vertauscht die beiden Nullstellen.
In diesem Sinne ist der Frobenius-Automorphismus analog zum
von $i\mapsto -i$ indizierten Automorphismus von
$\mathbb C = \mathbb R(i)$.

Man beachte, dass das hier betrachtet Polynom $m(x)=x^2+x+1$ auch
als Polynom über $\mathbb Q$ irreduzibel ist.
Seine Nullstellen in $\mathbb R$ sind nämlich
\[
x_{\pm}=-\frac12 \pm\frac{\sqrt{3}}{2}i\not\in\mathbb Q.
\]
Man kann aber auch hier nachrechnen, dass 
\begin{align*}
x_+^2
&=
\biggl(-\frac12+\frac{\sqrt{3}}{2}i\biggr)^2
=
\frac14-2\frac{\sqrt{3}}{4}i-\frac34
=
-\frac12-\frac{\sqrt{3}}{2}i
=
-x_+-1,
\\
x_-^2
&=
\biggl(-\frac12-\frac{\sqrt{3}}{2}i\biggr)^2
=
\frac14+2\frac{\sqrt{3}}{4}i-\frac34
=
-\frac12+\frac{\sqrt{3}}{2}i
=
-x_--1
\end{align*}
gilt.

Wir hätten $m(x)=x^2+x+1$ als Polynom über $\mathbb F_2$ aber auch
als die Reduktion $\mod 2$ des Polynoms $q(x)=x^2-x-1$ über $\mathbb Z$
betrachten können, Vorzeichenwechsel sind in $\mathbb F_2$ ja nicht
möglich.
Die Nullstellen von $q(x)$ sind aber
\[
x_{\pm}
=
\frac12\pm\frac{\sqrt{5}}2.
\]
Hier haben wir also ein Polynom, welches über $\mathbb R$ reduzibel ist.
Dies illustriert, dass verschiedene Polynome mit ganzlich unterschiedlicher
Nullstellen das gleiche algebraische Verhalten als Polynome über $\mathbb F_2$
zeigen.
Die Zahlen
\[
-\frac12\pm\frac{\sqrt{3}}{2}i
\qquad\text{und}\qquad
\frac12\pm\frac{\sqrt{5}}{2}
\]
haben also die gleichen algebraischen Eigenschaften wie die beiden Zahlen
$\alpha$ und $\beta$ als Nullstellen von $m(x)$ über $\mathbb F_2$.
\end{beispiel}



%
% ek-gauss.tex -- Anwendungen des Gauss-Algorithmus in endlichen Körpern
%
% (c) 2017 Prof Dr Andreas Mueller, Hochschule Rapperswil
%
\section{Gauss-Algorithmus in $\mathbb F_p$}
Der Gauss-Algorithmus ist die Basis sehr vieler Untersuchungen in der
linearen Algebra.
Man kann damit Gleichungssysteme l"osen, Determinanten bestimmen oder
Matrizen in einfachere Faktoren zerlegen.
Er verwendet nur K"orper-Operationen, und ist daher unmittelbar
auf endliche K"orper "ubertragbar.
Wir illustrieren dies an Hand einiger Beispiele.

\subsection{Ein Gleichungssystem "uber $\mathbb F_2$}
Besonders einfach ist die Arithmetik im K"orper $\mathbb F_2$,
da es nur ein einziges Element gibt, welches von $0$ verschieden ist,
n"amlich $1$.
Ausserdem ist die Addition nichts anderes als die XOR-Verkn"upfung.

Man kann also das folgende Gleichungssystem "uber $\mathbb F_2$
\begin{equation}
\begin{linsys}{3}
x_1& &   &+&x_3&=&1\\
x_1&+&x_2& &   &=&1\\
   & &x_2& &   &=&1\\
\end{linsys}
\label{ffield:gleichung}
\end{equation}
mit dem Gauss-Algorithmus wie folgt l"osen:
\begin{align*}
\begin{tabular}{|>{$}c<{$}>{$}c<{$}>{$}c<{$}|>{$}c<{$}|}
\hline
1&0&1&1\\
1&1&0&1\\
0&1&0&1\\
\hline
\end{tabular}
&
\rightarrow
\begin{tabular}{|>{$}c<{$}>{$}c<{$}>{$}c<{$}|>{$}c<{$}|}
\hline
1&0&1&1\\
0&1&1&0\\
0&1&0&1\\
\hline
\end{tabular}
\rightarrow
\begin{tabular}{|>{$}c<{$}>{$}c<{$}>{$}c<{$}|>{$}c<{$}|}
\hline
1&0&1&1\\
0&1&1&0\\
0&0&1&1\\
\hline
\end{tabular}
\rightarrow
\begin{tabular}{|>{$}c<{$}>{$}c<{$}>{$}c<{$}|>{$}c<{$}|}
\hline
1&0&0&0\\
0&1&0&1\\
0&0&1&1\\
\hline
\end{tabular}
\end{align*}
Daraus kann man die L"osung
\[
\begin{pmatrix}x_1\\x_2\\x_3\end{pmatrix}
=
\begin{pmatrix}0\\1\\1\end{pmatrix}
\]
ablesen.

Nat"urlich kann man auch die inverse Matrix bestimmen:
\begin{align*}
\begin{tabular}{|>{$}c<{$}>{$}c<{$}>{$}c<{$}|>{$}c<{$}>{$}c<{$}>{$}c<{$}|}
\hline
1&0&1&1&0&0\\
1&1&0&0&1&0\\
0&1&0&0&0&1\\
\hline
\end{tabular}
&
\rightarrow
\begin{tabular}{|>{$}c<{$}>{$}c<{$}>{$}c<{$}|>{$}c<{$}>{$}c<{$}>{$}c<{$}|}
\hline
1&0&1&1&0&0\\
0&1&1&1&1&0\\
0&1&0&0&0&1\\
\hline
\end{tabular}
\rightarrow
\begin{tabular}{|>{$}c<{$}>{$}c<{$}>{$}c<{$}|>{$}c<{$}>{$}c<{$}>{$}c<{$}|}
\hline
1&0&1&1&0&0\\
0&1&1&1&1&0\\
0&0&1&1&1&1\\
\hline
\end{tabular}
\\
&
\rightarrow
\begin{tabular}{|>{$}c<{$}>{$}c<{$}>{$}c<{$}|>{$}c<{$}>{$}c<{$}>{$}c<{$}|}
\hline
1&0&0&0&1&1\\
0&1&0&0&0&1\\
0&0&1&1&1&1\\
\hline
\end{tabular}
\end{align*}
Daraus liest man ab
\[
\begin{pmatrix}
1&0&1\\
1&1&0\\
0&1&0
\end{pmatrix}^{-1}
=
\begin{pmatrix}
0&1&1\\
0&0&1\\
1&1&1\\
\end{pmatrix}.
\]
Auch die eben gefundene L"osung des Gleichungssystems~\eqref{ffield:gleichung}
kann jetzt mit der inversen Matrix bestimmt werden:
\[
\begin{pmatrix}x_1\\x_2\\x_3\end{pmatrix}
=
\begin{pmatrix}
0&1&1\\
0&0&1\\
1&1&1\\
\end{pmatrix}
\begin{pmatrix}1\\1\\1\end{pmatrix}
=
\begin{pmatrix}0\\1\\1\end{pmatrix}.
\]

\subsection{LU- und LR-Zerlegung in $\mathbb F_5$}
Die Operationen in $\mathbb F_p$ werden durch die folgenden Additions-
bzw.~Multiplikationstabellen beschrieben.
\begin{center}
\begin{tabular}{|>{$}c<{$}|>{$}c<{$}>{$}c<{$}>{$}c<{$}>{$}c<{$}>{$}c<{$}|}
\hline
+&0&1&2&3&4\\
\hline
0&0&1&2&3&4\\
1&1&2&3&4&0\\
2&2&3&4&0&1\\
3&3&4&0&1&2\\
4&4&0&1&2&3\\
\hline
\end{tabular}
\qquad
\begin{tabular}{|>{$}c<{$}|>{$}c<{$}>{$}c<{$}>{$}c<{$}>{$}c<{$}>{$}c<{$}|}
\hline
\cdot&0&1&2&3&4\\
\hline
   0 &0&0&0&0&0\\
   1 &0&1&2&3&4\\
   2 &0&2&4&1&3\\
   3 &0&3&1&4&2\\
   4 &0&4&3&2&1\\
\hline
\end{tabular}
\end{center}
Damit ist es jetzt einfach, den Algorithmus zur Bestimmung der LU- und
LR-Zerlegung durchzuf"uhren.
Wir suchen die LU- und die LR-Zerlegung der Matrix
\[
A=\begin{pmatrix}
2&2&4\\
2&3&2\\
0&3&3
\end{pmatrix}.
\]
Der Gauss-Algorithmus liefert
\begin{align*}
\begin{tabular}{|>{$}c<{$}>{$}c<{$}>{$}c<{$}|}
\hline
2&2&4\\
2&3&2\\
0&3&3\\
\hline
\end{tabular}
&
\rightarrow
\begin{tabular}{|>{$}c<{$}>{$}c<{$}>{$}c<{$}|}
\hline
1&1&2\\
0&1&3\\
0&3&3\\
\hline
\end{tabular}
\rightarrow
\begin{tabular}{|>{$}c<{$}>{$}c<{$}>{$}c<{$}|}
\hline
1&1&2\\
0&1&3\\
0&0&4\\
\hline
\end{tabular}
\end{align*}
Daraus liest man die LU-Zerlegung ab:
\[
L
=
\begin{pmatrix}
2&0&0\\
2&1&0\\
0&3&4
\end{pmatrix},
\qquad
U
=
\begin{pmatrix}
1&1&2\\
0&1&3\\
0&0&1
\end{pmatrix}
\qquad
\Rightarrow
\qquad
LU=
\begin{pmatrix}
2&0&0\\
2&1&0\\
0&3&4
\end{pmatrix}
\begin{pmatrix}
1&1&2\\
0&1&3\\
0&0&1
\end{pmatrix}
=
\begin{pmatrix}
2&2&4\\
2&3&2\\
0&3&3
\end{pmatrix}
\]
F"ur die LR-Zerlegung muss $U$ mit $\operatorname{diag}(2,1,4)$
multipliziert werden und $L$ mit der Inversen:
\begin{align*}
L'
&=
L\operatorname{diag}(2,1,4)^{-1}
=
\begin{pmatrix}
2&0&0\\
2&1&0\\
0&3&4
\end{pmatrix}
\begin{pmatrix}
3&0&0\\
0&1&0\\
0&0&4
\end{pmatrix}
=
\begin{pmatrix}
1&0&0\\
1&1&0\\
0&3&1
\end{pmatrix}
\\
R'
&=
\operatorname{diag}(2,1,4)
\begin{pmatrix}
1&1&2\\
0&1&3\\
0&0&1
\end{pmatrix}
=
\begin{pmatrix}
2&0&0\\
0&1&0\\
0&0&4
\end{pmatrix}
\begin{pmatrix}
1&1&2\\
0&1&3\\
0&0&1
\end{pmatrix}
=
\begin{pmatrix}
2&2&4\\
0&1&3\\
0&0&4
\end{pmatrix}
\end{align*}
Kontrolle:
\[
L'R'
=
\begin{pmatrix}
1&0&0\\
1&1&0\\
0&3&1
\end{pmatrix}
\begin{pmatrix}
2&2&4\\
0&1&3\\
0&0&4
\end{pmatrix}
=
\begin{pmatrix}
2&2&4\\
2&3&2\\
0&3&3
\end{pmatrix}
\]

\subsection{Inverse Matrix in $\mathbb F_7$}
Die Additions- und Multiplikationstabellen f"ur $\mathbb F_7$ sind
\begin{center}
\begin{tabular}{|>{$}c<{$}|>{$}c<{$}>{$}c<{$}>{$}c<{$}>{$}c<{$}>{$}c<{$}>{$}c<{$}>{$}c<{$}|}
\hline
+&0&1&2&3&4&5&6\\
\hline
0&0&1&2&3&4&5&6\\
1&1&2&3&4&5&6&0\\
2&2&3&4&5&6&0&1\\
3&3&4&5&6&0&1&2\\
4&4&5&6&0&1&2&3\\
5&5&6&0&1&2&3&4\\
6&6&0&1&2&3&4&5\\
\hline
\end{tabular}
\qquad
\begin{tabular}{|>{$}c<{$}|>{$}c<{$}>{$}c<{$}>{$}c<{$}>{$}c<{$}>{$}c<{$}>{$}c<{$}>{$}c<{$}|}
\hline
\cdot&0&1&2&3&4&5&6\\
\hline
  0  &0&0&0&0&0&0&0\\
  1  &0&1&2&3&4&5&6\\
  2  &0&2&4&6&1&3&5\\
  3  &0&3&6&2&5&1&4\\
  4  &0&4&1&5&2&6&3\\
  5  &0&5&3&1&6&4&2\\
  6  &0&6&5&4&3&2&1\\
\hline
\end{tabular}
\end{center}
Damit k"onnen wir die inverse Matrix von
\[
A
=
\begin{pmatrix}
3&6&5\\
3&1&0\\
0&6&1
\end{pmatrix}
\]
mit dem Gauss-Algorithmus bestimmen:
\begin{align*}
\begin{tabular}{|ccc|ccc|}
\hline
3&6&5&1&0&0\\
3&1&0&0&1&0\\
0&6&1&0&0&1\\
\hline
\end{tabular}
&
\rightarrow
\begin{tabular}{|ccc|ccc|}
\hline
1&2&4&5&0&0\\
0&2&2&6&1&0\\
0&6&1&0&0&1\\
\hline
\end{tabular}
\rightarrow
\begin{tabular}{|ccc|ccc|}
\hline
1&2&4&5&0&0\\
0&1&1&3&4&0\\
0&0&2&3&4&1\\
\hline
\end{tabular}
\\
&
\rightarrow
\begin{tabular}{|ccc|ccc|}
\hline
1&2&0&6&6&5\\
0&1&0&5&2&3\\
0&0&1&5&2&4\\
\hline
\end{tabular}
\rightarrow
\begin{tabular}{|ccc|ccc|}
\hline
1&0&0&3&2&6\\
0&1&0&5&2&3\\
0&0&1&5&2&4\\
\hline
\end{tabular}
\end{align*}
Daraus liest man ab:
\[
A^{-1}
=
\begin{pmatrix}
3&2&6\\
5&2&3\\
5&2&4
\end{pmatrix}
\qquad
\Rightarrow
\qquad
AA^{-1}
=
\begin{pmatrix}
3&6&5\\
3&1&0\\
0&6&1
\end{pmatrix}
\begin{pmatrix}
3&2&6\\
5&2&3\\
5&2&4
\end{pmatrix}
=
\begin{pmatrix}
64&28&56\\
14& 8&21\\
35&14&22
\end{pmatrix}
=
\begin{pmatrix}
1&0&0\\
0&1&0\\
0&0&1
\end{pmatrix}.
\]
Alternativ k"onnen wir daf"ur auch Minoren verwenden.
Dazu brauchen wir zun"achst die Determinante, die wir mit der Sarrus-Formel
berechnen k"onnen:
\begin{align*}
\det(A)
&
=
\left|
\begin{matrix}
3&6&5\\
3&1&0\\
0&6&1
\end{matrix}
\right|
=
3+5\cdot3\cdot6-1\cdot3\cdot 6
=
3+6-4=5.
\end{align*}
Damit wird die inverse Matrix
\begin{align*}
A^{-1}
&=
\frac1{\det(A)}
{
\def\arraystretch{2.2}
\begin{pmatrix}
\def\arraystretch{1}
\phantom{-}
\left|\begin{matrix} 1&0\\6&1 \end{matrix}\right|&
\def\arraystretch{1}
-
\left|\begin{matrix} 6&5\\6&1 \end{matrix}\right|&
\def\arraystretch{1}
\phantom{-}
\left|\begin{matrix} 6&5\\1&0 \end{matrix}\right|
\\
\def\arraystretch{1}
-
\left|\begin{matrix} 3&0\\0&1 \end{matrix}\right|&
\def\arraystretch{1}
\phantom{-}
\left|\begin{matrix} 3&5\\0&1 \end{matrix}\right|&
\def\arraystretch{1}
-
\left|\begin{matrix} 3&5\\3&0 \end{matrix}\right|
\\
\def\arraystretch{1}
\phantom{-}
\left|\begin{matrix} 3&1\\0&6 \end{matrix}\right|&
\def\arraystretch{1}
-
\left|\begin{matrix} 3&6\\0&6 \end{matrix}\right|&
\def\arraystretch{1}
\phantom{-}
\left|\begin{matrix} 3&6\\3&1 \end{matrix}\right|
\end{pmatrix}
}
=
3\cdot
\begin{pmatrix}
 1\cdot 1-0\cdot 6&-6\cdot 1+5\cdot 6& 6\cdot 0-5\cdot 1\\
-3\cdot 1+0\cdot 0& 3\cdot 1-5\cdot 0&-3\cdot 0+5\cdot 3\\
 3\cdot 6-1\cdot 0&-3\cdot 6+6\cdot 0& 3\cdot 1-6\cdot 3
\end{pmatrix}
\\
&=
3\cdot
\begin{pmatrix}
1&3&2\\
4&3&1\\
4&3&6
\end{pmatrix}
=
\begin{pmatrix}
3&2&6\\
5&2&3\\
5&2&4
\end{pmatrix},
\end{align*}
in "Ubereinstimmung mit der Rechnung mit dem Gauss-Algorithmus.


%
% ek-gruppen.tex -- algebraische Gruppen über endlichen Körpern
%
% (c) 2017 Prof Dr Andreas Mueller, Hochschule Rapperswil
%
\section{Matrizengruppen in $\mathbb F_p$}
\rhead{Matrizengruppen in $\mathbb F_p$}
Die Gruppen $\textrm{GL}_n(\mathbb R)$ und $\textrm{SL}_n(\mathbb R)$
sind Beispiele algebraischer Gruppen über den reellen Zahlen.
In diesem Abschnitt sollen diese Gruppen auf den Fall endlicher Körper
verallgemeinert werden.

\subsection{$\textrm{GL}_n(\mathbb F_p)$ und $\textrm{SL}_n(\mathbb F_p)$}
Die Gruppe $\textrm{GL}_n(\mathbb F_p)$ besteht aus den
invertierbaren $n\times n$-Matrizen.
Da Invertierbarkeit wie im reellen Fall mit Hilfe der Determinante
festgestellt werden kann, die Gruppe $\textrm{GL}_n(\mathbb F_p)$ ist 
daher
\[
\textrm{GL}_n(\mathbb F_p)
=
\{ A\,|\,\text{$A$ ist eine $n\times n$-Matrix und $\det(A)\ne 0$}\}.
\]

\begin{beispiel}[Die Gruppe $\textrm{GL}_2(\mathbb F_2)$]
Da $\mathbb F_2$ nur zwei Element hat, können wir alle $2\times 2$-Matrizen
auflisten und die Determinanten bestimmen:
\begin{equation}
\begin{aligned}
A_0   &=\begin{pmatrix}0&0\\0&0\end{pmatrix} &&\Rightarrow&\det(A_0   )&=0&&&
A_1   &=\begin{pmatrix}1&0\\0&0\end{pmatrix} &&\Rightarrow&\det(A_1   )&=0\\
A_2   &=\begin{pmatrix}0&1\\0&0\end{pmatrix} &&\Rightarrow&\det(A_2   )&=0&&&
A_3   &=\begin{pmatrix}1&1\\0&0\end{pmatrix} &&\Rightarrow&\det(A_3   )&=0\\
A_4   &=\begin{pmatrix}0&0\\1&0\end{pmatrix} &&\Rightarrow&\det(A_4   )&=0&&&
A_5   &=\begin{pmatrix}1&0\\1&0\end{pmatrix} &&\Rightarrow&\det(A_5   )&=0\\
A_6   &=\begin{pmatrix}0&1\\1&0\end{pmatrix} &&\Rightarrow&\det(A_6   )&=1&&&
A_7   &=\begin{pmatrix}1&1\\1&0\end{pmatrix} &&\Rightarrow&\det(A_7   )&=1\\
A_8   &=\begin{pmatrix}0&0\\0&1\end{pmatrix} &&\Rightarrow&\det(A_8   )&=0&&&
A_9   &=\begin{pmatrix}1&0\\0&1\end{pmatrix} &&\Rightarrow&\det(A_9   )&=1\\
A_{10}&=\begin{pmatrix}0&1\\0&1\end{pmatrix} &&\Rightarrow&\det(A_{10})&=0&&&
A_{11}&=\begin{pmatrix}1&1\\0&1\end{pmatrix} &&\Rightarrow&\det(A_{11})&=1\\
A_{12}&=\begin{pmatrix}0&0\\1&1\end{pmatrix} &&\Rightarrow&\det(A_{12})&=0&&&
A_{13}&=\begin{pmatrix}1&0\\1&1\end{pmatrix} &&\Rightarrow&\det(A_{13})&=1\\
A_{14}&=\begin{pmatrix}0&1\\1&1\end{pmatrix} &&\Rightarrow&\det(A_{14})&=1&&&
A_{15}&=\begin{pmatrix}1&1\\1&1\end{pmatrix} &&\Rightarrow&\det(A_{15})&=0
\end{aligned}
\label{ff-f2matrizen}
\end{equation}
In der Gruppe $\textrm{GL}_2(\mathbb F_2)$ sind also nur die Matrizen
$A_6$, $A_7$, $A_9$, $A_{11}$, $A_{13}$ und $A_{14}$.
$A_9$ ist die Einheitsmatrix.
Die Matrizen haben die Multiplikationstabelle
\begin{center}
\begin{tabular}{|>{$}c<{$}|>{$}c<{$}>{$}c<{$}>{$}c<{$}>{$}c<{$}>{$}c<{$}>{$}c<{$}|}
\hline
\cdot &   E    &  A_6   &   A_7  & A_{11} & A_{13} & A_{14} \\
\hline
  E   &   E    &  A_6   &   A_7  & A_{11} & A_{13} & A_{14} \\
 A_6  &  A_6   &  E     & A_{13} & A_{14} &  A_7   & A_{11} \\
 A_7  &  A_7   & A_{11} & A_{14} & A_{13} &  A_6   &   E    \\
A_{11}& A_{11} &  A_7   &   A_6  &   E    & A_{14} & A_{13} \\
A_{13}& A_{13} & A_{14} & A_{11} &  A_7   &   E    &  A_6   \\
A_{14}& A_{14} & A_{13} &   E    &  A_6   & A_{11} &  A_7   \\
\hline
\end{tabular}
\end{center}
Es gibt nur zwei Gruppen mit sechs Elementen, nämlich die zyklische
Gruppe $\mathbb Z/6\mathbb Z$, die auch abelsch ist, und die
Permutationsgruppe $S_3$ auf $3$ Objekten, die nicht abelsch ist.
Wegen $A_6A_7=A_{13}\ne A_{11}=A_7A_6$ ist $\textrm{GL}_2(\mathbb F_2)$
nicht abelsch, also muss $\textrm{GL}_2(\mathbb F_2)\cong S_3$ sein.

In der Tat gibt es in $\textrm{GL}_2(\mathbb F_2)$ drei Matrizen,
$A_6$, $A_{11}$ und $A_{13}$,
deren Quadrat die Einheitsmatrix ist, genauso wie es drei solche
Permutationen gibt, nämlich jene, die genau zwei Elemente vertauschen,
die Transpositionen.
Indem wir diesen Matrizen Transpositionen zuordnen, können wir einen
Isomorphismus $\textrm{GL}_2(\mathbb F_2)\cong S_3$ konstruieren:
\[
\left.
\begin{aligned}
A_6   &\mapsto\begin{pmatrix}1&2&3\\2&1&3\end{pmatrix}
\\
A_{11}&\mapsto\begin{pmatrix}1&2&3\\3&2&1\end{pmatrix}
\\
A_{13}&\mapsto\begin{pmatrix}1&2&3\\1&3&2\end{pmatrix}
\end{aligned}
\right\}
\quad\Rightarrow\quad
\left\{
\begin{aligned}
A_7   &=A_6A_{13}\mapsto
\begin{pmatrix}1&2&3\\2&1&3\end{pmatrix}
\begin{pmatrix}1&2&3\\1&3&2\end{pmatrix}
=
\begin{pmatrix}1&2&3\\3&1&2\end{pmatrix}
\\
A_{14}&=A_6A_{11}\mapsto
\begin{pmatrix}1&2&3\\2&1&3\end{pmatrix}
\begin{pmatrix}1&2&3\\3&2&1\end{pmatrix}
=
\begin{pmatrix}1&2&3\\2&3&1\end{pmatrix}
\end{aligned}
\right.
\]
Wir kontrollieren diese Zuordnung, indem wir $A_7A_{14}$ ausrechnen,
wir müssten die identische Permutation erhalten:
\[
\begin{pmatrix}1&2&3\\3&1&2\end{pmatrix}
\begin{pmatrix}1&2&3\\2&3&1\end{pmatrix}
=
\begin{pmatrix}1&2&3\\1&2&3\end{pmatrix}
\]
in "Ubereinstimmung mit der Multiplikationstabelle von
$\textrm{GL}_2(\mathbb F_2)$.
\end{beispiel}



\subsection{Orthogonale Gruppen}
Auch die Bedingung der Orthogonalität lässt sich direkt auf lineare
Gruppen über endlichen Körpern ausdehnen.
Die Gruppe
\[
\textrm{O}(n,\mathbb F_p)
=
\{
A\,|\,
\text{$A$ ist eine $n\times n$-Matrix über $\mathbb F_p$ und $A^tA=E$}
\}
\]
besteht aus Matrizen $A$, die die quadratische Form
$
\langle u,v\rangle
=
u^tv
$
erhält, denn es gilt
\[
\langle Au,Av\rangle
=
(Au)^t Av
=
u^tA^tAv
=
u^tv
=
\langle u,v\rangle.
\]

Nicht übertragbar sind dagegen die Ideen, die aus der Interpretation
von $\langle u,u\rangle=u^tu$ als Länge eines Vektors hervorgehen.
Für diese Ideen ist wesentlich, dass $u^tu=0$ gleichbedeutend ist
mit $u=0$, eine Eigenschaft, die in endlichen Körpern nicht mehr gilt.
"Uber dem Körper $\mathbb F_p$ betrachten wir den $p$-dimensionalen
Vektor $u\in\mathbb F_p^p$ bestehend aus lauter Einsen, dann gilt
\[
u^tu
=
\begin{pmatrix}1&\dots&1\end{pmatrix}
\begin{pmatrix}1\\\vdots\\1\end{pmatrix}
=
\underbrace{1+\dots+1}_{\text{$p$ Summanden}}=0\quad\text{in $\mathbb F_p$}
\]
Damit sind Orthonormalisierungsalgorithmen wie der Gram-Schmidt-Algorithmus
nicht auf den Fall endlicher Körper $\mathbb F_p$.

\begin{beispiel}[Die Gruppe $\textrm{O}(2,\mathbb F_2)$]
Orthogonale Matrizen sind solche, die $A^tA=E$ erfüllen.
Da jede orthogonale Matrix auch invertierbar ist, können
wir die Liste der invertierbaren Matrizen von
Seite~\eqref{ff-f2matrizen} verwenden, um diejenigen Matrizen zu finden,
die auch orthogonal sind.
\[
\begin{aligned}
%A_0^t   &=A_0   &&\Rightarrow&A_0^t   A_0   &=A_0   A_0   =A_0   \\
%A_1^t   &=A_1   &&\Rightarrow&A_1^t   A_1   &=A_1   A_1   =A_0   \\
%A_2^t   &=A_4   &&\Rightarrow&A_2^t   A_2   &=A_4   A_2   =A_0   \\
%A_3^t   &=A_5   &&\Rightarrow&A_3^t   A_3   &=A_5   A_3   =A_0   \\
%A_4^t   &=A_2   &&\Rightarrow&A_4^t   A_4   &=A_2   A_4   =A_0   \\
%A_5^t   &=A_3   &&\Rightarrow&A_5^t   A_5   &=A_3   A_5   =A_0   \\
 A_6^t   &=A_6   &&\Rightarrow&A_6^t   A_6   &=A_6   A_6   =E     \\
 A_7^t   &=A_7   &&\Rightarrow&A_7^t   A_7   &=A_7   A_7   =A_{14}\\
%A_8^t   &=A_8   &&\Rightarrow&A_8^t   A_8   &=A_8   A_8   =A_0   \\
   E^t   &=E     &&\Rightarrow&  E^t   E     &=E     E     =E     \\
%A_{10}^t&=A_{12}&&\Rightarrow&A_{10}^tA_{10}&=A_{12}A_{10}=A_{10}\\
 A_{11}^t&=A_{13}&&\Rightarrow&A_{11}^tA_{11}&=A_{13}A_{11}=A_7   \\
%A_{12}^t&=A_{10}&&\Rightarrow&A_{12}^tA_{12}&=A_{10}A_{12}=A_{12}\\
 A_{13}^t&=A_{11}&&\Rightarrow&A_{13}^tA_{13}&=A_{11}A_{13}=A_{14}\\
 A_{14}^t&=A_{14}&&\Rightarrow&A_{14}^tA_{14}&=A_{14}A_{14}=A_7   \\
%A_{15}^t&=A_{15}&&\Rightarrow&A_{15}^tA_{15}&=A_{15}A_{15}=A_{15}\\
\end{aligned}
\]
Die einzigen Matrizen in $\textrm{GL}_2(\mathbb F_2)$, die auch orthogonal
sind, bilden die Gruppe
\[
\textrm{O}(2,\mathbb F_2)
=
\left\{
\begin{pmatrix}1&0\\0&1\end{pmatrix},
\begin{pmatrix}0&1\\1&0\end{pmatrix}
\right\}
\qedhere
\]
\end{beispiel}





%
% ek-eigenwert.tex -- Eigenwertproblem in endlichen Körpern
%
% (c) 2017 Prof Dr Andreas Mueller, Hochschule Rapperswil
%
\section{Eigenwerte und Eigenvektoren}
Auch die Theorie der Eigenwerte und Eigenvektoren kann auf endliche Körper
ausgedehnt werden.
Der erste Schritt bei der Bestimmung der Eigenvektoren verlangt, dass
die Nullstellen des charakteristischen Polynoms bestimmt werden.
Wie schon bei Polynomen über $\mathbb Q$ ist die Bestimmung der Nullstellen
von Polynomen über $\mathbb F_p$ alles andere als einfach.
Sie wird aber noch kompliziert dadurch, dass über $\mathbb F_p$
zusätzlich der Frobenius-Automorphismus bei $p$-ten Potenzen zu
neuen, unintuitiven Effekten führt.

\subsection{Eigenvektoren über $\mathbb F_7$}
Wir suchen Eigenwerte und Eigenvektoren der Matrix
\[
A
=
\begin{pmatrix}
0&1&4\\
3&5&6\\
3&4&4
\end{pmatrix}
\]
über $\mathbb F_7$.

Wir berechnen zuerst das charakteristische Polynom:
\begin{align*}
\det(A-\lambda E)
&
=
\left|\begin{matrix}
-\lambda&1        &4\\
3       &5-\lambda&6\\
3       &4        &4-\lambda
\end{matrix}\right|
\\
&=
-\lambda(5-\lambda)(4-\lambda)
+1\cdot6\cdot3
+4\cdot3\cdot 4
-3\cdot(5-\lambda)\cdot 4
+4\cdot6\cdot\lambda
-(4-\lambda)\cdot3\cdot 1
\\
&=
-\lambda(6-2\lambda+\lambda^2)
\\
&=
-6\lambda+2\lambda^2-\lambda^3
+4
+6
-4+5\lambda
+3\lambda
-5+3\lambda
\\
&=
-\lambda^3+2\lambda^2+5\lambda+1.
\end{align*}
Um die Eigenwerte zu finden, müssen wir also die Lösungen der
Gleichung
\[
-\chi_A(\lambda)
=
\lambda^3
-2\lambda^2
-5\lambda
-1
=
\lambda^3
+5\lambda^2
+2\lambda
+6
=
0
\]
finden.
Da $\mathbb F_7$ nur $7$ Elemente hat, kann man alle Werte durchprobieren,
und findet $1$, $3$ und $5$ als Nullstellen.
Zur Kontrolle berechnen wir das Produkt
\begin{align*}
(\lambda -1)(\lambda-3)(\lambda-5)
&
=
(\lambda^2-4\lambda+3)(\lambda-5)
\\
&=
\lambda^3-4\lambda^2+3\lambda
-5\lambda^2+6\lambda-1
\\
&=
\lambda^3+5\lambda^2+2\lambda+6
\\
&=
-\chi_A(\lambda),
\end{align*}
wir haben also alle Eigenwerte gefunden.

Zu jedem Eigenwert müssen jetzt noch ein Eigenvektor gefunden werde.
Für $\lambda=1$ verwenden wir 
\begin{align*}
\begin{tabular}{|>{$}c<{$}>{$}c<{$}>{$}c<{$}|}
\hline
0-1&1&4\\
3&5-1&6\\
3&4&4-1\\
\hline
\end{tabular}
&=
\begin{tabular}{|>{$}c<{$}>{$}c<{$}>{$}c<{$}|}
\hline
6&1&4\\
3&4&6\\
3&4&3\\
\hline
\end{tabular}
\rightarrow
\begin{tabular}{|>{$}c<{$}>{$}c<{$}>{$}c<{$}|}
\hline
1&6&3\\
0&0&4\\
0&0&1\\
\hline
\end{tabular}
\rightarrow
\begin{tabular}{|>{$}c<{$}>{$}c<{$}>{$}c<{$}|}
\hline
1&6&3\\
0&0&1\\
0&0&0\\
\hline
\end{tabular}
\rightarrow
\begin{tabular}{|>{$}c<{$}>{$}c<{$}>{$}c<{$}|}
\hline
1&6&0\\
0&0&1\\
0&0&0\\
\hline
\end{tabular}
\end{align*}
Daraus liest man ab, dass die dritte Komponente verschwinden muss, und dass
die zweite Variable frei wählbar ist.
Wählen wir sie als $1$, dann ist der Eigenvektor zum Eigenwert $\lambda=1$
\[
v_1
=
\begin{pmatrix}1\\1\\0\end{pmatrix}
\qquad\text{Kontrolle:}\qquad
Av_1
=
\begin{pmatrix}
0&1&4\\
3&5&6\\
3&4&4
\end{pmatrix}
\begin{pmatrix}1\\1\\0\end{pmatrix}
=
\begin{pmatrix}1\\1\\0\end{pmatrix}
\]

Für $\lambda = 3$ bestimmen wir einen Eigenvektor mit dem folgenden
Gauss-Tableau:
\begin{align*}
\begin{tabular}{|>{$}c<{$}>{$}c<{$}>{$}c<{$}|}
\hline
0-3&1&4\\
3&5-3&6\\
3&4&4-3\\
\hline
\end{tabular}
&=
\begin{tabular}{|>{$}c<{$}>{$}c<{$}>{$}c<{$}|}
\hline
4&1&4\\
3&2&6\\
3&4&1\\
\hline
\end{tabular}
\rightarrow
\begin{tabular}{|>{$}c<{$}>{$}c<{$}>{$}c<{$}|}
\hline
1&2&1\\
0&3&3\\
0&5&5\\
\hline
\end{tabular}
\rightarrow
\begin{tabular}{|>{$}c<{$}>{$}c<{$}>{$}c<{$}|}
\hline
1&2&1\\
0&1&1\\
0&0&0\\
\hline
\end{tabular}
\rightarrow
\begin{tabular}{|>{$}c<{$}>{$}c<{$}>{$}c<{$}|}
\hline
1&0&6\\
0&1&1\\
0&0&0\\
\hline
\end{tabular}
\end{align*}
Diesmal ist die dritte Komponente frei wählbar, wir wählen sie wieder als
$1$ und erhalten den Eigenvektor zum Eigenwert $\lambda=3$
\[
v_3=\begin{pmatrix}1\\6\\1\end{pmatrix}
\qquad\text{Kontrolle:}\qquad
Av_3
=
\begin{pmatrix}
0&1&4\\
3&5&6\\
3&4&4
\end{pmatrix}
\begin{pmatrix}1\\6\\1\end{pmatrix}
=
\begin{pmatrix}3\\4\\3\end{pmatrix}
=
3\begin{pmatrix}1\\6\\1\end{pmatrix}
=
3v_3.
\]

Schliesslich untersuchen wir den Eigenwert $\lambda=5$.
\begin{align*}
\begin{tabular}{|>{$}c<{$}>{$}c<{$}>{$}c<{$}|}
\hline
0-3&1&4\\
3&5-3&6\\
3&4&4-3\\
\hline
\end{tabular}
&=
\begin{tabular}{|>{$}c<{$}>{$}c<{$}>{$}c<{$}|}
\hline
2&1&4\\
3&0&6\\
3&4&6\\
\hline
\end{tabular}
\rightarrow
\begin{tabular}{|>{$}c<{$}>{$}c<{$}>{$}c<{$}|}
\hline
1&4&2\\
0&2&0\\
0&6&0\\
\hline
\end{tabular}
\rightarrow
\begin{tabular}{|>{$}c<{$}>{$}c<{$}>{$}c<{$}|}
\hline
1&0&2\\
0&1&0\\
0&0&0\\
\hline
\end{tabular}
\end{align*}
Die zweite Komponente ist $0$, die dritte frei wählber.
\[
v_5
=
\begin{pmatrix}5\\0\\1\end{pmatrix}
\qquad\text{Kontrolle:}\qquad
Av_5
=
\begin{pmatrix}
0&1&4\\
3&5&6\\
3&4&4
\end{pmatrix}
\begin{pmatrix}5\\0\\1\end{pmatrix}
=
\begin{pmatrix}4\\0\\5\end{pmatrix}
=
5\begin{pmatrix}5\\0\\1\end{pmatrix}
=
5v_5.
\]










