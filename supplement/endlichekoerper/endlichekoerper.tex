%
% endlichekoerper.tex
%
% (c) 2017 Prof Dr Andreas Mueller, Hochschule Rapperswil
%
\chapter{Lineare Algebra in endlichen K"orpern
\label{chapter:endlichekoerper}}
\rhead{Endliche K"orper}

Die Konstruktionen in Kapitel~1 und 2 des Skripts verlangen nicht mehr
als die Grundoperationen.
Die vorgestellten Algorithmen sollten daher in jeder Zahlenmenge
durchf"uhrbar sein, sobald die arithmetischen Operationen zur
Verf"ugung stehen.
Zum Beispiel wird die Theorie zwar fast ausschliesslich f"ur reelle
Zahlen entwickelt, doch funktioniert sie genau gleich f"ur rationale
oder komplexe Zahlen.
Alle drei Zahlenmengen sind sogenannte K"orper, sie enthalten
unendlich viele Elemente.
Es zeigt sich, dass es auch K"orper gibt, die nur endliche viele
Elemente enthalten.
Die K"orper werden jeweils durch eine Primzahl $p$ charakterisiert.
In der Codierungstheorie und der Kryptographie haben vor allem die
K"orper zur Primzahl $p=2$ eine besondere Bedeutung.
In diesem Kapitel wird gezeigt, wie man in diesen K"orpern rechnet
und es wird an Beispielen illustriert, wie die wohlbekannten Algorithmen
der linearen Algebra sich auf diese K"orper "ubertragen.

%
% ek-grundlagen.tex -- Grundlagen zur Theorie der endlichen Körper
%
% (c) 2017 Prof Dr Andreas Mueller, Hochschule Rapperswil
%
\section{Endliche K"orper}
Im Unterricht in linearer Algebra werden die Eigenschaften der reellen
Zahlen als Grundlage ohne weitere Diskussion akzeptiert.
Im Analysis-Unterricht wird etwas sorgf"altiger analyisiert, was denn
genau f"ur Eigenschaften notwendig sind.
Dabei werden erst die rationalen Zahlen $\mathbb Q$ als die Menge der
Br"uche der ganzen Zahlen konstruiert.
In einem zweiten Schritt werden diese unter Verwendung der Ordnungsrelation
(Dedekindsche Schnitte)
oder des Abstandsbegriffs (Topologie, metrischer Raum) zu den reellen
Zahlen $\mathbb R$ vervollst"andigt.
Der zweite Schritt ist aus der Sicht der linearen Algebra nicht n"otig.
Die lineare Algebra k"onnte auch g"anzlich "uber den rationalen Zahlen
entwickelt werden.
Erst bei der Cholesky-Zerlegung oder beim Eigenwertproblem wird es
n"otig, die rationalen Zahlen so zu erweitern, dass Quadratwurzeln
(Cholesky-Zerlegung) oder Nullstellen von Polynomen h"oheren Grades
gefunden werden k"onnen.

Die Analysis nutzt aus, dass die rationalen Zahlen nicht nur eine algebraische
Struktur haben, sondern auch eine Ordnungsstruktur.
Insbesondere ist es f"ur zwei beliebige positive rationale Zahlen
$a$ und $b$ m"oglich, eine Zahl $N$ anzugeben, so dass $Na>b$.
Man sagt, die rationalen Zahlen bilden einen archimedischen K"orper.
Die Ordnungsrelation ist also vertr"aglich mit den Rechenoperationen.
In der Analysis "aussert sich das dadurch, dass die Multiplikation mit
einer Zahl eine stetige Abbildung ist.
F"ur die lineare Algebra ist diese Eigenschaft jedoch bedeutungslos:
in den beiden ersten Kapiteln des Skriptes wird die Ordnungsrelation
kein einziges Mal gebraucht!

In diesem Kapitel soll gezeigt werde, wie man ausgehend von den
ganzen Zahlen $\mathbb Z$ auch andere K"orper konstruieren kann,
f"ur die die archimedische Eigenschaft nicht erf"ullt ist.
Wir verwenden hier einen axiomatischen Zugang um ganz sicher zu sein,
dass wir die konstruierten Zahlk"orper nicht versehentlich mit weiteren
Eigenschaften ausstatten, die nicht ben"otigt werden.
Erst im letzten Abschnitt "uber das Eigenwert-Problem werden wir
feststellen, dass hierf"ur die einfachen K"orper erweitert werden
m"ussen.
Dies geschieht durch hinzuf"ugen von geeigneten neuen Zahlen, den
Quadratwurzeln oder anderen Nullstellen.
Im Falle von $\mathbb R$ waren in $\mathbb R$ bereits alle denkbaren
Quadratwurzeln von positiven Zahlen vorhanden, und sie waren beliebig
nahe an rationalen Zahlen.
Da die endlichen K"orper keine Ordnungsrelation haben, und damit auch
keinen (offensichtlichen) Abstandsbegriff, finden wir bereits die
notwendigen Quadratwurzeln ``weit ausserhalb'' des Ausgangsk"orpers.

\subsection{K"orper}
Der Begriff des K"orpers fasst die Eigenschaften zusammen, die f"ur
die Skalare in der linearen Algebra ben"otigt werden.

Eine {\em Gruppe} ist eine Menge $G$ mit einer Verkn"upfung, die zwei Elementen
$a,b\in G$ das Element $ab\in G$ zuordnet.
Ausserdem m"ussen folgen Axiome erf"ullt sein:
\begin{enumerate}[label={\bf G.\arabic*},itemsep=0mm]
\item
Die Verkn"upfung ist assoziativ, d.~h.~$(ab)c=a(bc)$ f"ur alle $a,b,c\in G$.
\item
Es gibt ein Element $e\in G$ mit der Eigenschaft $eg=ge=g$ f"ur alle $g\in G$,
genant das Neutralelement.
\item
F"ur jedes Element $g\in G$ gibt es ein Element $g^{-1}\in G$ welches
$gg^{-1}=g^{-1}g=e$.
\end{enumerate}
Eine Gruppe heisst {\em abelsch} wenn $ab=ba$ f"ur alle $a,b\in G$.

Ein {\em Ring} ist eine Menge $R$ mit zwei Verkn"upfungen, der Addition
und der Multiplikation, mit folgenden Eigenschaften:
\begin{enumerate}[label={\bf R.\arabic*},itemsep=0mm]
\item $R$ ist bez"uglich der Addition eine abelsche Gruppe.
\item Die Multiplikation in $R$ ist assoziativ und hat ein Einselement.
\item F"ur drei Element $x,y,z\in R$ gilt $(x+y)z=xz+yz$ und
$z(x+y)=zx+zy$.
\end{enumerate}

Die Menge $\mathbb Z$ der ganzen Zahlen tr"agt die Struktur eines Ringes.
Die Menge der $n\times n$-Matrizen mit Eintr"agen in $\mathbb Z$ ist ebenfalls
ein Ring.
Ist $R$ ein Ring, dann ist die Menge 
\[
R[X]=\{ a_0+a_1X +a_2X^2+\dots +a_nX^n\,|\,a_i\in R\}
\]
der Polynome in der Variablen $X$ ein Ring.

Ein {\em Körper} ist ein Ring, so dass die Menge der von $0$ verschiedenen
Elemente eine abelsche Gruppe bez"uglich der Multiplikation bilden.

Die Menge $\mathbb Q$ der rationalen Zahlen tr"agt die Struktur eines
K"orpers.

\subsection{Reste}

\subsection{Charakteristik}

\subsection{Frobenius-Automorphismus}
F"ur K"orper mit Charakteristik $0$ ist das Potenzieren zwar bez"uglich der
Multiplikation ein Homomorphismus:
$
(ab)^k = a^kb^k
$
aber nicht bez"uglich der Addition, wo die binomische Formel
\begin{equation}
(a+b)^k
= 
a^k + \binom{k}{1} a^{k-1}b + \binom{k}{2} a^{k-2}b^2
+ \binom{k}{3}a^{k-3}b^3 + \dots + \binom{k}{k-1}ab^{k-1} + b^k
\label{ff:binom}
\end{equation}
gilt.
Man kann zeigen, dass die Binomial-Koeffizienten auf Zeile $k$ durch 
$k$ teilbar sind.
Reduzieren wir die binomische Formel \eqref{ff:binom} mit $k=p$ Modulo $p$,
dann folgt
\begin{align*}
(a+b)^p
&= 
a^p + \binom{p}{1} a^{p-1}b + \binom{p}{2} a^{p-2}b^2
+ \binom{p}{3}a^{p-3}b^3 + \dots + \binom{p}{p-1}ab^{p-1} + b^p
\\
&\equiv
a^p + 0\cdot a^{p-1}b + 0\cdot a^{p-1}b^2
+ 0\cdot a^{p-3}b^3 + \dots + 0\cdot ab^{p-1} + b^p
\quad\mod p
\\
&=
a^p + b^p.
\end{align*}
Im K"orper $\mathbb F_p$ ist daher die Abbildung $a\mapsto a^p$
ein Homomorphismus von K"orpern.
Er heisst der {\em Frobenius-Automorphismus}.

Speziell gilt auch, dass mehrfache Anwendung des Frobenius-Automorphismus
ebenfalls ein Automorphismus ist.
Die $k$-fache Anwendung des Frobenius-Automorphismus ist nichts anderes
als das erheben in die $p^k$-te Potenz.
Dies bedeutet, dass auch die die meisten Binomial-Koeffizienten zu $p^k$ 
durch $p$ teilbar sind:
\[
\binom{p^k}{l}\equiv 0\quad\mod p
\qquad
0<l<p^k.
\]

\begin{beispiel}[Beispiel: Frobenius-Automorphismus in $\mathbb F_2$]
In $\mathbb F_2$ besagt der Frobenius-Automorphismus, dass das Quadrieren
ein Automorphismus ist.
Da $0^2=0$ und $1^2=1$ ist, ist der Frobenius-Automorphismus in $\mathbb F_2$
die identische Abbildung.
\end{beispiel}

Der kleine Satz von Euler besagt, dass f"ur eine Primzahl $p$ und jede
beliebige ganze Zahl $a$ gilt
\[
a^{p-1}\cong 1\mod p
\qquad\Rightarrow\qquad
a^p\cong a\mod p.
\]
Dies bedeutet, dass der Frobenius-Automorphismus auf dem K"orper $\mathbb F_p$
immer wie die Identit"at wirkt.
Seine Wirkung wird also erst sichtbar, wenn man zu K"orpererweiterungen
von $\mathbb F_p$ "ubergeht.
Im Lichte der Galois-Theorie ist $\mathbb F_p$ ein Fix-K"orper
unter der Wirkung des Frobenius-Automorphismus.

\begin{beispiel}[Beispiel: Frobenius-Automorphismus in $\mathbb F_2(\alpha)$]
Wir konstruieren eine K"orper-Erweiterung vom Grad zwei "uber dem K"orper
$\mathbb F_2$.
Dazu verwenden betrachten wir das Polynom $m(x)=x^2 + x + 1$.
Wir suchen nach Nullstellen des Polynoms.
Durch Einsetzen von $1$ und $0$ in $m(x)$ kann man erkennen, dass
$m(x)$ keine Nullstellen $x\in\mathbb F_2$ hat.

Man kann dies alternativ auch einsehen, indem man versucht, das Polynom
$m(x)$ in Faktoren zu zerlegen.
Da es nur zwei Polynome ersten Grades gibt, kann man durch durchprobieren
aller Polynome ersten Grades herausfinden, ob $m(x)$ irreduzibel ist.
$m(x)$ ist offensichtlich nicht durch $x$ teilbar.
Andererseits ist $m(x)$ auch nicht durch $x+1$ teilbar, denn da $x$ kein
Faktor sein kann, m"usste dann $m(x)=(x+1)^2=x^2 +  2x + 1=x^2+x$ sein,
was offensichtlich nicht zutrifft.
Somit ist $m(x)$ irreduzibel "uber $\mathbb F_2$.

Wir postulieren jetzt zwei neue ``Zahlen'' $\alpha$ und $\beta$, die
zu $\mathbb F_2$ hinzugef"ugt werden sollen, und die Nullstellen von 
$m(x)$ sein sollen.
Sie m"ussen daher die Gleichungen
\[
\left.
\begin{aligned}
\alpha^2 + \alpha +1&=0
\\
\beta^2 + \beta +1&=0
\end{aligned}
\right\}
\qquad\Rightarrow\qquad
\left\{
\begin{aligned}
\alpha^2&=\alpha+1\\
\beta^2&=\beta+1
\end{aligned}
\right.
\]
Die Zahl $\alpha+1$ ist ebenfalls eine Nullstelle von $m(x)$:
\[
m(\alpha+1)
=
(\alpha+1)^2+(\alpha+1) + 1
=
\underbrace{\alpha^2 + 1}_{\text{Frobenius}}\mathstrut + \alpha + 1 + 1
=
\underbrace{\alpha+1}_{\alpha^2} + 1+ (\alpha + 1) + 1=0.
\]
Da $\alpha+1\ne \alpha$ ist, aber auch eine Nullstelle, gibt es nur
noch die eine M"oglichkeit, dass $\alpha+1=\beta$ ist.
Dann ist aber auch
\[
\beta^2
=
(\alpha + 1)^2
= 
\alpha^2 + 1
=
\alpha + 1 + 1
=
\alpha.
\]
Der Frobenius-Automorphismus vertauscht die beiden Nullstellen.
In diesem Sinne ist der Frobenius-Automorphismus analog zum
von $i\mapsto -i$ indizierten Automorphismus von
$\mathbb C = \mathbb R(i)$.

Man beachte, dass das hier betrachtet Polynom $m(x)=x^2+x+1$ auch
als Polynom "uber $\mathbb Q$ irreduzibel ist.
Seine Nullstellen in $\mathbb R$ sind n"amlich
\[
x_{\pm}=-\frac12 \pm\frac{\sqrt{3}}{2}i\not\in\mathbb Q.
\]
Man kann aber auch hier nachrechnen, dass 
\begin{align*}
x_+^2
&=
\biggl(-\frac12+\frac{\sqrt{3}}{2}i\biggr)^2
=
\frac14-2\frac{\sqrt{3}}{4}i-\frac34
=
-\frac12-\frac{\sqrt{3}}{2}i
=
-x_+-1,
\\
x_-^2
&=
\biggl(-\frac12-\frac{\sqrt{3}}{2}i\biggr)^2
=
\frac14+2\frac{\sqrt{3}}{4}i-\frac34
=
-\frac12+\frac{\sqrt{3}}{2}i
=
-x_--1
\end{align*}
gilt.

Wir h"atten $m(x)=x^2+x+1$ als Polynom "uber $\mathbb F_2$ aber auch
als die Reduktion $\mod 2$ des Polynoms $q(x)=x^2-x-1$ "uber $\mathbb Z$
betrachten k"onnen, Vorzeichenwechsel sind in $\mathbb F_2$ ja nicht
m"oglich.
Die Nullstellen von $q(x)$ sind aber
\[
x_{\pm}
=
\frac12\pm\frac{\sqrt{5}}2.
\]
Hier haben wir also ein Polynom, welches "uber $\mathbb R$ reduzibel ist.
Dies illustriert, dass verschiedene Polynome mit ganzlich unterschiedlicher
Nullstellen das gleiche algebraische Verhalten als Polynome "uber $\mathbb F_2$
zeigen.
Die Zahlen
\[
-\frac12\pm\frac{\sqrt{3}}{2}i
\qquad\text{und}\qquad
\frac12\pm\frac{\sqrt{5}}{2}
\]
haben also die gleichen algebraischen Eigenschaften wie die beiden Zahlen
$\alpha$ und $\beta$ als Nullstellen von $m(x)$ "uber $\mathbb F_2$.
\end{beispiel}



%
% ek-gauss.tex -- Anwendungen des Gauss-Algorithmus in endlichen Körpern
%
% (c) 2017 Prof Dr Andreas Mueller, Hochschule Rapperswil
%
\section{Gauss-Algorithmus in $\mathbb F_p$}
\rhead{Gauss-Algorithmus in $\mathbb F_p$}
Der Gauss-Algorithmus ist die Basis sehr vieler Untersuchungen in der
linearen Algebra.
Man kann damit Gleichungssysteme lösen, Determinanten bestimmen oder
Matrizen in einfachere Faktoren zerlegen.
Er verwendet nur Körper-Operationen, und ist daher unmittelbar
auf endliche Körper übertragbar.
Wir illustrieren dies an Hand einiger Beispiele.

\subsection{Ein Gleichungssystem über $\mathbb F_2$}
Besonders einfach ist die Arithmetik im Körper $\mathbb F_2$,
da es nur ein einziges Element gibt, welches von $0$ verschieden ist,
nämlich $1$.
Ausserdem ist die Addition nichts anderes als die XOR-Verknüpfung.

Man kann also das folgende Gleichungssystem über $\mathbb F_2$
\begin{equation}
\begin{linsys}{3}
x_1& &   &+&x_3&=&1\\
x_1&+&x_2& &   &=&1\\
   & &x_2& &   &=&1\\
\end{linsys}
\label{ffield:gleichung}
\end{equation}
mit dem Gauss-Algorithmus wie folgt lösen:
\begin{align*}
\begin{tabular}{|>{$}c<{$}>{$}c<{$}>{$}c<{$}|>{$}c<{$}|}
\hline
1&0&1&1\\
1&1&0&1\\
0&1&0&1\\
\hline
\end{tabular}
&
\rightarrow
\begin{tabular}{|>{$}c<{$}>{$}c<{$}>{$}c<{$}|>{$}c<{$}|}
\hline
1&0&1&1\\
0&1&1&0\\
0&1&0&1\\
\hline
\end{tabular}
\rightarrow
\begin{tabular}{|>{$}c<{$}>{$}c<{$}>{$}c<{$}|>{$}c<{$}|}
\hline
1&0&1&1\\
0&1&1&0\\
0&0&1&1\\
\hline
\end{tabular}
\rightarrow
\begin{tabular}{|>{$}c<{$}>{$}c<{$}>{$}c<{$}|>{$}c<{$}|}
\hline
1&0&0&0\\
0&1&0&1\\
0&0&1&1\\
\hline
\end{tabular}
\end{align*}
Daraus kann man die Lösung
\[
\begin{pmatrix}x_1\\x_2\\x_3\end{pmatrix}
=
\begin{pmatrix}0\\1\\1\end{pmatrix}
\]
ablesen.

Natürlich kann man auch die inverse Matrix bestimmen:
\begin{align*}
\begin{tabular}{|>{$}c<{$}>{$}c<{$}>{$}c<{$}|>{$}c<{$}>{$}c<{$}>{$}c<{$}|}
\hline
1&0&1&1&0&0\\
1&1&0&0&1&0\\
0&1&0&0&0&1\\
\hline
\end{tabular}
&
\rightarrow
\begin{tabular}{|>{$}c<{$}>{$}c<{$}>{$}c<{$}|>{$}c<{$}>{$}c<{$}>{$}c<{$}|}
\hline
1&0&1&1&0&0\\
0&1&1&1&1&0\\
0&1&0&0&0&1\\
\hline
\end{tabular}
\rightarrow
\begin{tabular}{|>{$}c<{$}>{$}c<{$}>{$}c<{$}|>{$}c<{$}>{$}c<{$}>{$}c<{$}|}
\hline
1&0&1&1&0&0\\
0&1&1&1&1&0\\
0&0&1&1&1&1\\
\hline
\end{tabular}
\\
&
\rightarrow
\begin{tabular}{|>{$}c<{$}>{$}c<{$}>{$}c<{$}|>{$}c<{$}>{$}c<{$}>{$}c<{$}|}
\hline
1&0&0&0&1&1\\
0&1&0&0&0&1\\
0&0&1&1&1&1\\
\hline
\end{tabular}
\end{align*}
Daraus liest man ab
\[
\begin{pmatrix}
1&0&1\\
1&1&0\\
0&1&0
\end{pmatrix}^{-1}
=
\begin{pmatrix}
0&1&1\\
0&0&1\\
1&1&1\\
\end{pmatrix}.
\]
Auch die eben gefundene Lösung des Gleichungssystems~\eqref{ffield:gleichung}
kann jetzt mit der inversen Matrix bestimmt werden:
\[
\begin{pmatrix}x_1\\x_2\\x_3\end{pmatrix}
=
\begin{pmatrix}
0&1&1\\
0&0&1\\
1&1&1\\
\end{pmatrix}
\begin{pmatrix}1\\1\\1\end{pmatrix}
=
\begin{pmatrix}0\\1\\1\end{pmatrix}.
\]

\subsection{LU- und LR-Zerlegung in $\mathbb F_5$}
Die Operationen in $\mathbb F_p$ werden durch die folgenden Additions-
bzw.~Multiplikationstabellen beschrieben.
\begin{center}
\begin{tabular}{|>{$}c<{$}|>{$}c<{$}>{$}c<{$}>{$}c<{$}>{$}c<{$}>{$}c<{$}|}
\hline
+&0&1&2&3&4\\
\hline
0&0&1&2&3&4\\
1&1&2&3&4&0\\
2&2&3&4&0&1\\
3&3&4&0&1&2\\
4&4&0&1&2&3\\
\hline
\end{tabular}
\qquad
\begin{tabular}{|>{$}c<{$}|>{$}c<{$}>{$}c<{$}>{$}c<{$}>{$}c<{$}>{$}c<{$}|}
\hline
\cdot&0&1&2&3&4\\
\hline
   0 &0&0&0&0&0\\
   1 &0&1&2&3&4\\
   2 &0&2&4&1&3\\
   3 &0&3&1&4&2\\
   4 &0&4&3&2&1\\
\hline
\end{tabular}
\end{center}
Damit ist es jetzt einfach, den Algorithmus zur Bestimmung der LU- und
LR-Zerlegung durchzuführen.
Wir suchen die LU- und die LR-Zerlegung der Matrix
\[
A=\begin{pmatrix}
2&2&4\\
2&3&2\\
0&3&3
\end{pmatrix}.
\]
Der Gauss-Algorithmus liefert
\begin{align*}
\begin{tabular}{|>{$}c<{$}>{$}c<{$}>{$}c<{$}|}
\hline
2&2&4\\
2&3&2\\
0&3&3\\
\hline
\end{tabular}
&
\rightarrow
\begin{tabular}{|>{$}c<{$}>{$}c<{$}>{$}c<{$}|}
\hline
1&1&2\\
0&1&3\\
0&3&3\\
\hline
\end{tabular}
\rightarrow
\begin{tabular}{|>{$}c<{$}>{$}c<{$}>{$}c<{$}|}
\hline
1&1&2\\
0&1&3\\
0&0&4\\
\hline
\end{tabular}
\end{align*}
Daraus liest man die LU-Zerlegung ab:
\[
L
=
\begin{pmatrix}
2&0&0\\
2&1&0\\
0&3&4
\end{pmatrix},
\qquad
U
=
\begin{pmatrix}
1&1&2\\
0&1&3\\
0&0&1
\end{pmatrix}
\qquad
\Rightarrow
\qquad
LU=
\begin{pmatrix}
2&0&0\\
2&1&0\\
0&3&4
\end{pmatrix}
\begin{pmatrix}
1&1&2\\
0&1&3\\
0&0&1
\end{pmatrix}
=
\begin{pmatrix}
2&2&4\\
2&3&2\\
0&3&3
\end{pmatrix}
\]
Für die LR-Zerlegung muss $U$ mit $\operatorname{diag}(2,1,4)$
multipliziert werden und $L$ mit der Inversen:
\begin{align*}
L'
&=
L\operatorname{diag}(2,1,4)^{-1}
=
\begin{pmatrix}
2&0&0\\
2&1&0\\
0&3&4
\end{pmatrix}
\begin{pmatrix}
3&0&0\\
0&1&0\\
0&0&4
\end{pmatrix}
=
\begin{pmatrix}
1&0&0\\
1&1&0\\
0&3&1
\end{pmatrix}
\\
R'
&=
\operatorname{diag}(2,1,4)
\begin{pmatrix}
1&1&2\\
0&1&3\\
0&0&1
\end{pmatrix}
=
\begin{pmatrix}
2&0&0\\
0&1&0\\
0&0&4
\end{pmatrix}
\begin{pmatrix}
1&1&2\\
0&1&3\\
0&0&1
\end{pmatrix}
=
\begin{pmatrix}
2&2&4\\
0&1&3\\
0&0&4
\end{pmatrix}
\end{align*}
Kontrolle:
\[
L'R'
=
\begin{pmatrix}
1&0&0\\
1&1&0\\
0&3&1
\end{pmatrix}
\begin{pmatrix}
2&2&4\\
0&1&3\\
0&0&4
\end{pmatrix}
=
\begin{pmatrix}
2&2&4\\
2&3&2\\
0&3&3
\end{pmatrix}.
\]

\subsection{Inverse Matrix in $\mathbb F_7$}
Die Additions- und Multiplikationstabellen für $\mathbb F_7$ sind
\begin{center}
\begin{tabular}{|>{$}c<{$}|>{$}c<{$}>{$}c<{$}>{$}c<{$}>{$}c<{$}>{$}c<{$}>{$}c<{$}>{$}c<{$}|}
\hline
+&0&1&2&3&4&5&6\\
\hline
0&0&1&2&3&4&5&6\\
1&1&2&3&4&5&6&0\\
2&2&3&4&5&6&0&1\\
3&3&4&5&6&0&1&2\\
4&4&5&6&0&1&2&3\\
5&5&6&0&1&2&3&4\\
6&6&0&1&2&3&4&5\\
\hline
\end{tabular}
\qquad
\begin{tabular}{|>{$}c<{$}|>{$}c<{$}>{$}c<{$}>{$}c<{$}>{$}c<{$}>{$}c<{$}>{$}c<{$}>{$}c<{$}|}
\hline
\cdot&0&1&2&3&4&5&6\\
\hline
  0  &0&0&0&0&0&0&0\\
  1  &0&1&2&3&4&5&6\\
  2  &0&2&4&6&1&3&5\\
  3  &0&3&6&2&5&1&4\\
  4  &0&4&1&5&2&6&3\\
  5  &0&5&3&1&6&4&2\\
  6  &0&6&5&4&3&2&1\\
\hline
\end{tabular}
\end{center}
Damit können wir die inverse Matrix von
\[
A
=
\begin{pmatrix}
3&6&5\\
3&1&0\\
0&6&1
\end{pmatrix}
\]
mit dem Gauss-Algorithmus bestimmen:
\begin{align*}
\begin{tabular}{|ccc|ccc|}
\hline
3&6&5&1&0&0\\
3&1&0&0&1&0\\
0&6&1&0&0&1\\
\hline
\end{tabular}
&
\rightarrow
\begin{tabular}{|ccc|ccc|}
\hline
1&2&4&5&0&0\\
0&2&2&6&1&0\\
0&6&1&0&0&1\\
\hline
\end{tabular}
\rightarrow
\begin{tabular}{|ccc|ccc|}
\hline
1&2&4&5&0&0\\
0&1&1&3&4&0\\
0&0&2&3&4&1\\
\hline
\end{tabular}
\\
&
\rightarrow
\begin{tabular}{|ccc|ccc|}
\hline
1&2&0&6&6&5\\
0&1&0&5&2&3\\
0&0&1&5&2&4\\
\hline
\end{tabular}
\rightarrow
\begin{tabular}{|ccc|ccc|}
\hline
1&0&0&3&2&6\\
0&1&0&5&2&3\\
0&0&1&5&2&4\\
\hline
\end{tabular}
\end{align*}
Daraus liest man ab:
\[
A^{-1}
=
\begin{pmatrix}
3&2&6\\
5&2&3\\
5&2&4
\end{pmatrix}
\qquad
\Rightarrow
\qquad
AA^{-1}
=
\begin{pmatrix}
3&6&5\\
3&1&0\\
0&6&1
\end{pmatrix}
\begin{pmatrix}
3&2&6\\
5&2&3\\
5&2&4
\end{pmatrix}
=
\begin{pmatrix}
64&28&56\\
14& 8&21\\
35&14&22
\end{pmatrix}
=
\begin{pmatrix}
1&0&0\\
0&1&0\\
0&0&1
\end{pmatrix}.
\]
Alternativ können wir dafür auch Minoren verwenden.
Dazu brauchen wir zunächst die Determinante, die wir mit der Sarrus-Formel
berechnen können:
\begin{align*}
\det(A)
&
=
\left|
\begin{matrix}
3&6&5\\
3&1&0\\
0&6&1
\end{matrix}
\right|
=
3+5\cdot3\cdot6-1\cdot3\cdot 6
=
3+6-4=5.
\end{align*}
Damit wird die inverse Matrix
\begin{align*}
A^{-1}
&=
\frac1{\det(A)}
{
\def\arraystretch{2.2}
\begin{pmatrix}
\def\arraystretch{1}
\phantom{-}
\left|\begin{matrix} 1&0\\6&1 \end{matrix}\right|&
\def\arraystretch{1}
-
\left|\begin{matrix} 6&5\\6&1 \end{matrix}\right|&
\def\arraystretch{1}
\phantom{-}
\left|\begin{matrix} 6&5\\1&0 \end{matrix}\right|
\\
\def\arraystretch{1}
-
\left|\begin{matrix} 3&0\\0&1 \end{matrix}\right|&
\def\arraystretch{1}
\phantom{-}
\left|\begin{matrix} 3&5\\0&1 \end{matrix}\right|&
\def\arraystretch{1}
-
\left|\begin{matrix} 3&5\\3&0 \end{matrix}\right|
\\
\def\arraystretch{1}
\phantom{-}
\left|\begin{matrix} 3&1\\0&6 \end{matrix}\right|&
\def\arraystretch{1}
-
\left|\begin{matrix} 3&6\\0&6 \end{matrix}\right|&
\def\arraystretch{1}
\phantom{-}
\left|\begin{matrix} 3&6\\3&1 \end{matrix}\right|
\end{pmatrix}
}
=
3\cdot
\begin{pmatrix}
 1\cdot 1-0\cdot 6&-6\cdot 1+5\cdot 6& 6\cdot 0-5\cdot 1\\
-3\cdot 1+0\cdot 0& 3\cdot 1-5\cdot 0&-3\cdot 0+5\cdot 3\\
 3\cdot 6-1\cdot 0&-3\cdot 6+6\cdot 0& 3\cdot 1-6\cdot 3
\end{pmatrix}
\\
&=
3\cdot
\begin{pmatrix}
1&3&2\\
4&3&1\\
4&3&6
\end{pmatrix}
=
\begin{pmatrix}
3&2&6\\
5&2&3\\
5&2&4
\end{pmatrix},
\end{align*}
in "Ubereinstimmung mit der Rechnung mit dem Gauss-Algorithmus.


%
% ek-gruppen.tex -- algebraische Gruppen über endlichen Körpern
%
% (c) 2017 Prof Dr Andreas Mueller, Hochschule Rapperswil
%
\section{Matrizengruppen in $\mathbb F_p$}

%
% ek-eigenwert.tex -- Eigenwertproblem in endlichen Körpern
%
% (c) 2017 Prof Dr Andreas Mueller, Hochschule Rapperswil
%
\section{Eigenwerte und Eigenvektoren}
\rhead{Eigenwerte und Eigenvektoren}
Auch die Theorie der Eigenwerte und Eigenvektoren kann auf endliche Körper
ausgedehnt werden.
Der erste Schritt bei der Bestimmung der Eigenvektoren verlangt, dass
die Nullstellen des charakteristischen Polynoms bestimmt werden.
Wie schon bei Polynomen über $\mathbb Q$ ist die Bestimmung der Nullstellen
von Polynomen über $\mathbb F_p$ alles andere als einfach.
Sie wird aber noch kompliziert dadurch, dass über $\mathbb F_p$
zusätzlich der Frobenius-Automorphismus bei $p$-ten Potenzen zu
neuen, unintuitiven Effekten führt.

\subsection{Eigenvektoren über $\mathbb F_7$}
Wir suchen Eigenwerte und Eigenvektoren der Matrix
\[
A
=
\begin{pmatrix}
0&1&4\\
3&5&6\\
3&4&4
\end{pmatrix}
\]
über $\mathbb F_7$.

Wir berechnen zuerst das charakteristische Polynom:
\begin{align*}
\det(A-\lambda E)
&
=
\left|\begin{matrix}
-\lambda&1        &4\\
3       &5-\lambda&6\\
3       &4        &4-\lambda
\end{matrix}\right|
\\
&=
-\lambda(5-\lambda)(4-\lambda)
+1\cdot6\cdot3
+4\cdot3\cdot 4
-3\cdot(5-\lambda)\cdot 4
+4\cdot6\cdot\lambda
-(4-\lambda)\cdot3\cdot 1
\\
&=
-\lambda(6-2\lambda+\lambda^2)
\\
&=
-6\lambda+2\lambda^2-\lambda^3
+4
+6
-4+5\lambda
+3\lambda
-5+3\lambda
\\
&=
-\lambda^3+2\lambda^2+5\lambda+1.
\end{align*}
Um die Eigenwerte zu finden, müssen wir also die Lösungen der
Gleichung
\[
-\chi_A(\lambda)
=
\lambda^3
-2\lambda^2
-5\lambda
-1
=
\lambda^3
+5\lambda^2
+2\lambda
+6
=
0
\]
finden.
Da $\mathbb F_7$ nur $7$ Elemente hat, kann man alle Werte durchprobieren,
und findet $1$, $3$ und $5$ als Nullstellen.
Zur Kontrolle berechnen wir das Produkt
\begin{align*}
(\lambda -1)(\lambda-3)(\lambda-5)
&
=
(\lambda^2-4\lambda+3)(\lambda-5)
\\
&=
\lambda^3-4\lambda^2+3\lambda
-5\lambda^2+6\lambda-1
\\
&=
\lambda^3+5\lambda^2+2\lambda+6
\\
&=
-\chi_A(\lambda),
\end{align*}
wir haben also alle Eigenwerte gefunden.

Zu jedem Eigenwert müssen jetzt noch ein Eigenvektor gefunden werde.
Für $\lambda=1$ verwenden wir 
\begin{align*}
\begin{tabular}{|>{$}c<{$}>{$}c<{$}>{$}c<{$}|}
\hline
0-1&1&4\\
3&5-1&6\\
3&4&4-1\\
\hline
\end{tabular}
&=
\begin{tabular}{|>{$}c<{$}>{$}c<{$}>{$}c<{$}|}
\hline
6&1&4\\
3&4&6\\
3&4&3\\
\hline
\end{tabular}
\rightarrow
\begin{tabular}{|>{$}c<{$}>{$}c<{$}>{$}c<{$}|}
\hline
1&6&3\\
0&0&4\\
0&0&1\\
\hline
\end{tabular}
\rightarrow
\begin{tabular}{|>{$}c<{$}>{$}c<{$}>{$}c<{$}|}
\hline
1&6&3\\
0&0&1\\
0&0&0\\
\hline
\end{tabular}
\rightarrow
\begin{tabular}{|>{$}c<{$}>{$}c<{$}>{$}c<{$}|}
\hline
1&6&0\\
0&0&1\\
0&0&0\\
\hline
\end{tabular}
\end{align*}
Daraus liest man ab, dass die dritte Komponente verschwinden muss, und dass
die zweite Variable frei wählbar ist.
Wählen wir sie als $1$, dann ist der Eigenvektor zum Eigenwert $\lambda=1$
\[
v_1
=
\begin{pmatrix}1\\1\\0\end{pmatrix}
\qquad\text{Kontrolle:}\qquad
Av_1
=
\begin{pmatrix}
0&1&4\\
3&5&6\\
3&4&4
\end{pmatrix}
\begin{pmatrix}1\\1\\0\end{pmatrix}
=
\begin{pmatrix}1\\1\\0\end{pmatrix}
\]

Für $\lambda = 3$ bestimmen wir einen Eigenvektor mit dem folgenden
Gauss-Tableau:
\begin{align*}
\begin{tabular}{|>{$}c<{$}>{$}c<{$}>{$}c<{$}|}
\hline
0-3&1&4\\
3&5-3&6\\
3&4&4-3\\
\hline
\end{tabular}
&=
\begin{tabular}{|>{$}c<{$}>{$}c<{$}>{$}c<{$}|}
\hline
4&1&4\\
3&2&6\\
3&4&1\\
\hline
\end{tabular}
\rightarrow
\begin{tabular}{|>{$}c<{$}>{$}c<{$}>{$}c<{$}|}
\hline
1&2&1\\
0&3&3\\
0&5&5\\
\hline
\end{tabular}
\rightarrow
\begin{tabular}{|>{$}c<{$}>{$}c<{$}>{$}c<{$}|}
\hline
1&2&1\\
0&1&1\\
0&0&0\\
\hline
\end{tabular}
\rightarrow
\begin{tabular}{|>{$}c<{$}>{$}c<{$}>{$}c<{$}|}
\hline
1&0&6\\
0&1&1\\
0&0&0\\
\hline
\end{tabular}
\end{align*}
Diesmal ist die dritte Komponente frei wählbar, wir wählen sie wieder als
$1$ und erhalten den Eigenvektor zum Eigenwert $\lambda=3$
\[
v_3=\begin{pmatrix}1\\6\\1\end{pmatrix}
\qquad\text{Kontrolle:}\qquad
Av_3
=
\begin{pmatrix}
0&1&4\\
3&5&6\\
3&4&4
\end{pmatrix}
\begin{pmatrix}1\\6\\1\end{pmatrix}
=
\begin{pmatrix}3\\4\\3\end{pmatrix}
=
3\begin{pmatrix}1\\6\\1\end{pmatrix}
=
3v_3.
\]

Schliesslich untersuchen wir den Eigenwert $\lambda=5$.
\begin{align*}
\begin{tabular}{|>{$}c<{$}>{$}c<{$}>{$}c<{$}|}
\hline
0-3&1&4\\
3&5-3&6\\
3&4&4-3\\
\hline
\end{tabular}
&=
\begin{tabular}{|>{$}c<{$}>{$}c<{$}>{$}c<{$}|}
\hline
2&1&4\\
3&0&6\\
3&4&6\\
\hline
\end{tabular}
\rightarrow
\begin{tabular}{|>{$}c<{$}>{$}c<{$}>{$}c<{$}|}
\hline
1&4&2\\
0&2&0\\
0&6&0\\
\hline
\end{tabular}
\rightarrow
\begin{tabular}{|>{$}c<{$}>{$}c<{$}>{$}c<{$}|}
\hline
1&0&2\\
0&1&0\\
0&0&0\\
\hline
\end{tabular}
\end{align*}
Die zweite Komponente ist $0$, die dritte frei wählber.
\[
v_5
=
\begin{pmatrix}5\\0\\1\end{pmatrix}
\qquad\text{Kontrolle:}\qquad
Av_5
=
\begin{pmatrix}
0&1&4\\
3&5&6\\
3&4&4
\end{pmatrix}
\begin{pmatrix}5\\0\\1\end{pmatrix}
=
\begin{pmatrix}4\\0\\5\end{pmatrix}
=
5\begin{pmatrix}5\\0\\1\end{pmatrix}
=
5v_5.
\]










