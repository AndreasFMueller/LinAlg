%
% ek-gruppen.tex -- algebraische Gruppen über endlichen Körpern
%
% (c) 2017 Prof Dr Andreas Mueller, Hochschule Rapperswil
%
\section{Matrizengruppen in $\mathbb F_p$}
\rhead{Matrizengruppen in $\mathbb F_p$}
Die Gruppen $\textrm{GL}_n(\mathbb R)$ und $\textrm{SL}_n(\mathbb R)$
sind Beispiele algebraischer Gruppen über den reellen Zahlen.
In diesem Abschnitt sollen diese Gruppen auf den Fall endlicher Körper
verallgemeinert werden.

\subsection{$\textrm{GL}_n(\mathbb F_p)$ und $\textrm{SL}_n(\mathbb F_p)$}
Die Gruppe $\textrm{GL}_n(\mathbb F_p)$ besteht aus den
invertierbaren $n\times n$-Matrizen.
Da Invertierbarkeit wie im reellen Fall mit Hilfe der Determinante
festgestellt werden kann, die Gruppe $\textrm{GL}_n(\mathbb F_p)$ ist 
daher
\[
\textrm{GL}_n(\mathbb F_p)
=
\{ A\,|\,\text{$A$ ist eine $n\times n$-Matrix und $\det(A)\ne 0$}\}.
\]

\begin{beispiel}[Die Gruppe $\textrm{GL}_2(\mathbb F_2)$]
Da $\mathbb F_2$ nur zwei Element hat, können wir alle $2\times 2$-Matrizen
auflisten und die Determinanten bestimmen:
\begin{equation}
\begin{aligned}
A_0   &=\begin{pmatrix}0&0\\0&0\end{pmatrix} &&\Rightarrow&\det(A_0   )&=0&&&
A_1   &=\begin{pmatrix}1&0\\0&0\end{pmatrix} &&\Rightarrow&\det(A_1   )&=0\\
A_2   &=\begin{pmatrix}0&1\\0&0\end{pmatrix} &&\Rightarrow&\det(A_2   )&=0&&&
A_3   &=\begin{pmatrix}1&1\\0&0\end{pmatrix} &&\Rightarrow&\det(A_3   )&=0\\
A_4   &=\begin{pmatrix}0&0\\1&0\end{pmatrix} &&\Rightarrow&\det(A_4   )&=0&&&
A_5   &=\begin{pmatrix}1&0\\1&0\end{pmatrix} &&\Rightarrow&\det(A_5   )&=0\\
A_6   &=\begin{pmatrix}0&1\\1&0\end{pmatrix} &&\Rightarrow&\det(A_6   )&=1&&&
A_7   &=\begin{pmatrix}1&1\\1&0\end{pmatrix} &&\Rightarrow&\det(A_7   )&=1\\
A_8   &=\begin{pmatrix}0&0\\0&1\end{pmatrix} &&\Rightarrow&\det(A_8   )&=0&&&
A_9   &=\begin{pmatrix}1&0\\0&1\end{pmatrix} &&\Rightarrow&\det(A_9   )&=1\\
A_{10}&=\begin{pmatrix}0&1\\0&1\end{pmatrix} &&\Rightarrow&\det(A_{10})&=0&&&
A_{11}&=\begin{pmatrix}1&1\\0&1\end{pmatrix} &&\Rightarrow&\det(A_{11})&=1\\
A_{12}&=\begin{pmatrix}0&0\\1&1\end{pmatrix} &&\Rightarrow&\det(A_{12})&=0&&&
A_{13}&=\begin{pmatrix}1&0\\1&1\end{pmatrix} &&\Rightarrow&\det(A_{13})&=1\\
A_{14}&=\begin{pmatrix}0&1\\1&1\end{pmatrix} &&\Rightarrow&\det(A_{14})&=1&&&
A_{15}&=\begin{pmatrix}1&1\\1&1\end{pmatrix} &&\Rightarrow&\det(A_{15})&=0
\end{aligned}
\label{ff-f2matrizen}
\end{equation}
In der Gruppe $\textrm{GL}_2(\mathbb F_2)$ sind also nur die Matrizen
$A_6$, $A_7$, $A_9$, $A_{11}$, $A_{13}$ und $A_{14}$.
$A_9$ ist die Einheitsmatrix.
Die Matrizen haben die Multiplikationstabelle
\begin{center}
\begin{tabular}{|>{$}c<{$}|>{$}c<{$}>{$}c<{$}>{$}c<{$}>{$}c<{$}>{$}c<{$}>{$}c<{$}|}
\hline
\cdot &   E    &  A_6   &   A_7  & A_{11} & A_{13} & A_{14} \\
\hline
  E   &   E    &  A_6   &   A_7  & A_{11} & A_{13} & A_{14} \\
 A_6  &  A_6   &  E     & A_{13} & A_{14} &  A_7   & A_{11} \\
 A_7  &  A_7   & A_{11} & A_{14} & A_{13} &  A_6   &   E    \\
A_{11}& A_{11} &  A_7   &   A_6  &   E    & A_{14} & A_{13} \\
A_{13}& A_{13} & A_{14} & A_{11} &  A_7   &   E    &  A_6   \\
A_{14}& A_{14} & A_{13} &   E    &  A_6   & A_{11} &  A_7   \\
\hline
\end{tabular}
\end{center}
Es gibt nur zwei Gruppen mit sechs Elementen, nämlich die zyklische
Gruppe $\mathbb Z/6\mathbb Z$, die auch abelsch ist, und die
Permutationsgruppe $S_3$ auf $3$ Objekten, die nicht abelsch ist.
Wegen $A_6A_7=A_{13}\ne A_{11}=A_7A_6$ ist $\textrm{GL}_2(\mathbb F_2)$
nicht abelsch, also muss $\textrm{GL}_2(\mathbb F_2)\cong S_3$ sein.

In der Tat gibt es in $\textrm{GL}_2(\mathbb F_2)$ drei Matrizen,
$A_6$, $A_{11}$ und $A_{13}$,
deren Quadrat die Einheitsmatrix ist, genauso wie es drei solche
Permutationen gibt, nämlich jene, die genau zwei Elemente vertauschen,
die Transpositionen.
Indem wir diesen Matrizen Transpositionen zuordnen, können wir einen
Isomorphismus $\textrm{GL}_2(\mathbb F_2)\cong S_3$ konstruieren:
\[
\left.
\begin{aligned}
A_6   &\mapsto\begin{pmatrix}1&2&3\\2&1&3\end{pmatrix}
\\
A_{11}&\mapsto\begin{pmatrix}1&2&3\\3&2&1\end{pmatrix}
\\
A_{13}&\mapsto\begin{pmatrix}1&2&3\\1&3&2\end{pmatrix}
\end{aligned}
\right\}
\quad\Rightarrow\quad
\left\{
\begin{aligned}
A_7   &=A_6A_{13}\mapsto
\begin{pmatrix}1&2&3\\2&1&3\end{pmatrix}
\begin{pmatrix}1&2&3\\1&3&2\end{pmatrix}
=
\begin{pmatrix}1&2&3\\3&1&2\end{pmatrix}
\\
A_{14}&=A_6A_{11}\mapsto
\begin{pmatrix}1&2&3\\2&1&3\end{pmatrix}
\begin{pmatrix}1&2&3\\3&2&1\end{pmatrix}
=
\begin{pmatrix}1&2&3\\2&3&1\end{pmatrix}
\end{aligned}
\right.
\]
Wir kontrollieren diese Zuordnung, indem wir $A_7A_{14}$ ausrechnen,
wir müssten die identische Permutation erhalten:
\[
\begin{pmatrix}1&2&3\\3&1&2\end{pmatrix}
\begin{pmatrix}1&2&3\\2&3&1\end{pmatrix}
=
\begin{pmatrix}1&2&3\\1&2&3\end{pmatrix}
\]
in "Ubereinstimmung mit der Multiplikationstabelle von
$\textrm{GL}_2(\mathbb F_2)$.
\end{beispiel}



\subsection{Orthogonale Gruppen}
Auch die Bedingung der Orthogonalität lässt sich direkt auf lineare
Gruppen über endlichen Körpern ausdehnen.
Die Gruppe
\[
\textrm{O}(n,\mathbb F_p)
=
\{
A\,|\,
\text{$A$ ist eine $n\times n$-Matrix über $\mathbb F_p$ und $A^tA=E$}
\}
\]
besteht aus Matrizen $A$, die die quadratische Form
$
\langle u,v\rangle
=
u^tv
$
erhält, denn es gilt
\[
\langle Au,Av\rangle
=
(Au)^t Av
=
u^tA^tAv
=
u^tv
=
\langle u,v\rangle.
\]

Nicht übertragbar sind dagegen die Ideen, die aus der Interpretation
von $\langle u,u\rangle=u^tu$ als Länge eines Vektors hervorgehen.
Für diese Ideen ist wesentlich, dass $u^tu=0$ gleichbedeutend ist
mit $u=0$, eine Eigenschaft, die in endlichen Körpern nicht mehr gilt.
"Uber dem Körper $\mathbb F_p$ betrachten wir den $p$-dimensionalen
Vektor $u\in\mathbb F_p^p$ bestehend aus lauter Einsen, dann gilt
\[
u^tu
=
\begin{pmatrix}1&\dots&1\end{pmatrix}
\begin{pmatrix}1\\\vdots\\1\end{pmatrix}
=
\underbrace{1+\dots+1}_{\text{$p$ Summanden}}=0\quad\text{in $\mathbb F_p$}
\]
Damit sind Orthonormalisierungsalgorithmen wie der Gram-Schmidt-Algorithmus
nicht auf den Fall endlicher Körper $\mathbb F_p$.

\begin{beispiel}[Die Gruppe $\textrm{O}(2,\mathbb F_2)$]
Orthogonale Matrizen sind solche, die $A^tA=E$ erfüllen.
Da jede orthogonale Matrix auch invertierbar ist, können
wir die Liste der invertierbaren Matrizen von
Seite~\eqref{ff-f2matrizen} verwenden, um diejenigen Matrizen zu finden,
die auch orthogonal sind.
\[
\begin{aligned}
%A_0^t   &=A_0   &&\Rightarrow&A_0^t   A_0   &=A_0   A_0   =A_0   \\
%A_1^t   &=A_1   &&\Rightarrow&A_1^t   A_1   &=A_1   A_1   =A_0   \\
%A_2^t   &=A_4   &&\Rightarrow&A_2^t   A_2   &=A_4   A_2   =A_0   \\
%A_3^t   &=A_5   &&\Rightarrow&A_3^t   A_3   &=A_5   A_3   =A_0   \\
%A_4^t   &=A_2   &&\Rightarrow&A_4^t   A_4   &=A_2   A_4   =A_0   \\
%A_5^t   &=A_3   &&\Rightarrow&A_5^t   A_5   &=A_3   A_5   =A_0   \\
 A_6^t   &=A_6   &&\Rightarrow&A_6^t   A_6   &=A_6   A_6   =E     \\
 A_7^t   &=A_7   &&\Rightarrow&A_7^t   A_7   &=A_7   A_7   =A_{14}\\
%A_8^t   &=A_8   &&\Rightarrow&A_8^t   A_8   &=A_8   A_8   =A_0   \\
   E^t   &=E     &&\Rightarrow&  E^t   E     &=E     E     =E     \\
%A_{10}^t&=A_{12}&&\Rightarrow&A_{10}^tA_{10}&=A_{12}A_{10}=A_{10}\\
 A_{11}^t&=A_{13}&&\Rightarrow&A_{11}^tA_{11}&=A_{13}A_{11}=A_7   \\
%A_{12}^t&=A_{10}&&\Rightarrow&A_{12}^tA_{12}&=A_{10}A_{12}=A_{12}\\
 A_{13}^t&=A_{11}&&\Rightarrow&A_{13}^tA_{13}&=A_{11}A_{13}=A_{14}\\
 A_{14}^t&=A_{14}&&\Rightarrow&A_{14}^tA_{14}&=A_{14}A_{14}=A_7   \\
%A_{15}^t&=A_{15}&&\Rightarrow&A_{15}^tA_{15}&=A_{15}A_{15}=A_{15}\\
\end{aligned}
\]
Die einzigen Matrizen in $\textrm{GL}_2(\mathbb F_2)$, die auch orthogonal
sind, bilden die Gruppe
\[
\textrm{O}(2,\mathbb F_2)
=
\left\{
\begin{pmatrix}1&0\\0&1\end{pmatrix},
\begin{pmatrix}0&1\\1&0\end{pmatrix}
\right\}
\qedhere
\]
\end{beispiel}




