%
% homologie.tex
%
% (c) 2017 Prof Dr Andreas Müller, Hochschule Rapperswil
%
\chapter{Homologie}
\rhead{Homologie}
Im Skript wird gezeigt, wie man zu jedem Graphen die Menge der Zyklen
finden kann.
In den Anwendungen wird dies daf"ur gebraucht, eine linear unabh"angige
Menge von Maschen f"ur die Gleichungen von Kirchhoff zu bestimmen.
Es zeigt sich jedoch auch, dass der Begriff des Zyklus sehr viel
allgemeiner formuliert werden kann, und dass sich damit ein algebraisches
Objekt konstruieren l"asst, mit welchem sich topologische Eigenschaften
eines K"orpers beschreiben lassen.
Der Begriff der Homologie-Gruppen verallgemeinert eine grundlegende
Beobachtung, die schon Euler "uber Polyeder gemacht hat.

\section{Kern und Bild}
Seien $U$ und $V$ Vektorr"aume "uber $\mathbb R$ und sei $\varphi\colon U\to V$
eine lineare Abbildung.
Dann sind die Mengen
\begin{align*}
\operatorname{ker}\varphi&=\{u\in U\,|\, \varphi(u) = 0\}
\\
\operatorname{im}\varphi&=\{\varphi(u)\,|\, u\in U\}
\end{align*}
Vektorr"aume.
In der Tat folgt aus $u_1,u_2\in\operatorname{ker}\varphi$
und $\lambda\in\mathbb R$
\begin{align*}
\varphi(u_1+u_2)&=\varphi(u_1)+\varphi(u_2) = 0
&
&\Rightarrow&
u_1+u_2&\in\operatorname{ker}\varphi
\\
\varphi(\lambda u_1)&=\lambda\varphi(u_1)
&
&\Rightarrow&
\lambda u_1&\in\operatorname{ker}\varphi
\end{align*}
was beweist das $\operatorname{ker}\varphi$ ein Vektorraum ist.
F"ur $\operatorname{im}\varphi$ nehmen wir zun"achst
$v_1,v_2\in\operatorname{im}\varphi$.
Dann gibt es Vektoren $u_1,u_2\in U$ derart, dass
$\varphi(u_1)=v_1$ und $\varphi(u_2)=v_2$.
Dann folgt
\begin{align*}
v_1+v_2&=\varphi(u_1)+\varphi(u_2)=\varphi(u_1+u_2)
&
&\Rightarrow&
v_1+v_w&\in\operatorname{im}\varphi
\\
\lambda v_1&=\lambda\varphi(u_1)=\varphi(\lambda u_1)
&
&\Rightarrow&
\lambda v_1&\in\operatorname{im}\varphi,
\end{align*}
was wieder beweist, dass $\operatorname{im}\varphi$ ein Vektor ist.
Es ist klar dass $\operatorname{ker}\varphi\subset U$ und
$\operatorname{im}\varphi\subset V$.

\section{Quotientenraum}
Sei jetzt $V$ ein reeller Vekor und $U\subset V$ ein Unterraum.
Dann k"onnen wir einen neuen Vektorraum $V/U$ wie folgt definieren.
Die Element von $V/U$ sind Teilmengen der Form
\[
\tilde v = \{v+u\,|\, u\in U\}.
\]
Offenbar ist $\tilde v_1=\tilde v_2$ wenn $v_1-v_2\in U$.
Wir m"ussen ausserdem die Vektorraum-Operationen definieren.
Es ist naheliegend
\begin{align*}
\tilde v_1+\tilde v_2
&=
\{ v_1+v_2+u\,|\, u\in U\}
\\
\lambda \tilde v_1
&=
\{\lambda v_1+u\,|\, u\in U\}.
\end{align*}
zu setzen, doch wir m"ussen sicherstellen dass eine andere Wahl
der Repr"asentanten $v_i$ auf die gleichen Mengen f"uhrt.
Seien also $v_1'$ und $v_2'$ andere Repr"asentanten, es gilt also
$v_1'-v_1\in U$ and $v_2'-v_2\in U$.
Dann gilt
\begin{align*}
\{v_1 + v_2 + u\,|\, u\in U\}
&=
\{v_1' + \underbrace{v_1-v_1'}_{\displaystyle\in U}
+ v_2' + \underbrace{v_2-v_2'}_{\displaystyle\in U}
+ u\,|\, u\in U\} = \{v_1'+v_2'+u\,|\, u\in U\}
\\
\{\lambda v_1 + u\,|\, u\in U\}
&=
\{\lambda v_1' + \underbrace{\lambda(v_1-v_1')}_{\displaystyle \in U}+u
\,|\, u\in U\}
=
\{\lambda v_1' + u\,|\, u\in U\}
\end{align*}
Damit ist gezeigt, dass die Vektorraumoperationen in $V/U$ wohldefiniert
sind.

Jetzt muss aber auch noch gezeigt werden, dass die Axiome f"ur einen
Vektorraum erf"ullt sind.
Die Rechenregeln wie das Assoziativgesetz oder das Distributivgesetz
folgen jetzt unmittelbar daraus, dass die Rechenoperationen mit Hilfe
von Repr"asentationen definiert sind, und f"ur diese gelten die
genannten Gesetze.
Das gilt auch f"ur den Nullvektor, der Repr"asentant von $U$ ist.

Die Darstellung von Elementen von $V/U$ mit Hilfe von Repr"asentanten
bedeutet auch, dass es eine Abbildung $V\to U$ gibt, die definiert
ist durch
\[
\pi \colon V\to U: v\mapsto \tilde v=\{v+u\,|\, u\in U\}\in V/U.
\]
Auch hier folgt aus den Eigenschaften des Rechnens mit Repr"asentaten,
dass $\pi$ eine lineare Abbildung von Vektorr"aumen ist.
Ausserdem ist das Bild von $\pi$ die ganzge Menge,
$\operatorname{im}\pi = V/U$.

\begin{definition}
Der Vektorraum $V/U$ heisst der Quotient von $V$ nach dem Unterraum $U$,
$\pi$ heisst die (kanonische) Projektion.
\end{definition}

Geometrisch kann man sich den Quotientenraum so vorstellen:
Die Mengen $v+U$ mit $v\in V$ sind parallel verschobene Kopien des
Unterraumes $U$.
Im Quotienten werden alle $v$ miteinander identifiziert, die sich nur
durch ein Element in $U$ unterschieden.
Der Quotient besteht also aus den Richtungen ``quer'' zu $U$, w"ahrend 
$U$ zu einem einzigen Punkt kollabiert.

\section{Homologie}
Ein Komplex ist eine Folge von Vektorräumen $C_k$ und linearen Abbildungen
$\partial_k$
\[
\xymatrix{
*+\txt{} \ar[r]^{\partial}
	&C_{n+2} \ar[r]^{\partial_{n+2}}
		&C_{n+1} \ar[r]^{\partial_{n+1}}
			&C_{n} \ar[r]^{\partial_n}
				&C_{n-1} \ar[r]^{\partial_{n-1}}
					&C_{n-2} \ar[r]^{\partial_{n-2}}
						&*+\txt{}
}
\]
Ausserdem wird verlangt, dass $\partial_{k} \circ \partial_{k+1}=0$ ist.
Dies bedeutet, dass
$\operatorname{im}\partial_{k+1}\subset \operatorname{ker}\partial_{k}$.
Wir k"onnen daher die Vektorräume
\[
H_k = \operatorname{ker}\partial_k/\operatorname{im}\partial_{k+1}
\]
definieren.
$H_k$ heisst die $k$-te Homologiegruppe des Komplexes $(C_k,\partial_k)$.

Zu einem Polyeder kann man in natürlicher Weise einen Komplex konstruieren.
Der Vektorraum $C_0$ hat einen Basis-Vektor für jede Ecke eines Polyeders.
Für jede orientierte Kante des Polyeders fügen wir dem Vektorraum $C_1$
einen Basisvektor hinzu. 
Weiter enthält $C_2$ für jede orientiert Seitenfläche des Polyeders
einen Basisvektor, dem ganzen Polyeder selbst wird der eindimensionale
Vektorraum $C_3$ zugeordnet.

\section{Der Eulersche Polyedersatz}





