%
% homologie.tex
%
% (c) 2017 Prof Dr Andreas Müller, Hochschule Rapperswil
%
\chapter{Homologie}
\rhead{Homologie}
Im Skript wird gezeigt, wie man zu jedem Graphen die Menge der Zyklen
finden kann.
In den Anwendungen wird dies daf"ur gebraucht, eine linear unabh"angige
Menge von Maschen f"ur die Gleichungen von Kirchhoff zu bestimmen.
Es zeigt sich jedoch auch, dass der Begriff des Zyklus sehr viel
allgemeiner formuliert werden kann, und dass sich damit ein algebraisches
Objekt konstruieren l"asst, mit welchem sich topologische Eigenschaften
eines K"orpers beschreiben lassen.
Der Begriff der Homologie-Gruppen verallgemeinert eine grundlegende
Beobachtung, die schon Euler "uber Polyeder gemacht hat.

\section{Kern, Bild und Quotientenr"aume}
\section{Homologie}
\section{Der Eulersche Polyedersatz}





