%
% endlichekoerper.tex
%
% (c) 2017 Prof Dr Andreas Mueller, Hochschule Rapperswil
%
\chapter{Lineare Algebra in endlichen K"orpern}
\rhead{Endliche K"orper}

Die Konstruktionen in Kapitel~1 und 2 des Skripts verlangen nicht mehr
als die Grundoperationen.
Die vorgestellten Algorithmen sollten daher in jeder Zahlenmenge
durchf"uhrbar sein, sobald die arithmetischen Operationen zur
Verf"ugung stehen.
Zum Beispiel wird die Theorie zwar fast ausschliesslich f"ur reelle
Zahlen entwickelt, doch funktioniert sie genau gleich f"ur rationale
oder komplexe Zahlen.
Alle drei Zahlenmengen sind sogenannte K"orper, sie enthalten
unendlich viele Elemente.
Es zeigt sich, dass es auch K"orper gibt, die nur endliche viele
Elemente enthalten.
Die K"orper werden jeweils durch eine Primzahl $p$ charakterisiert.
In der Kryptographie haben vor allem die K"orper zur Primzahl $p=2$
eine besondere Bedeutung.
In diesem Kapitel wird gezeigt, wie man in diesen K"orpern rechnet
und es wird an Beispielen illustriert, wie die wohlbekannten Algorithmen
der linearen Algebra sich auf diese K"orper "ubertragen.

\section{Endliche K"orper}
\subsection{K"orper}
Der Begriff des K"orpers fasst die Eigenschaften zusammen, die f"ur
die Skalare in der linearen Algebra ben"otigt werden.

Eine {\em Gruppe} ist eine Menge $G$ mit einer Verkn"upfung, die zwei Elementen
$a,b\in G$ das Element $ab\in G$ zuordnet.
Ausserdem m"ussen folgen Axiome erf"ullt sein:
\begin{enumerate}[label={\bf G.\arabic*},itemsep=0mm]
\item
Die Verkn"upfung ist assoziativ, d.~h.~$(ab)c=a(bc)$ f"ur alle $a,b,c\in G$.
\item
Es gibt ein Element $e\in G$ mit der Eigenschaft $eg=ge=g$ f"ur alle $g\in G$,
genant das Neutralelement.
\item
F"ur jedes Element $g\in G$ gibt es ein Element $g^{-1}\in G$ welches
$gg^{-1}=g^{-1}g=e$.
\end{enumerate}
Eine Gruppe heisst {\em abelsch} wenn $ab=ba$ f"ur alle $a,b\in G$.

Ein {\em Ring} ist eine Menge $R$ mit zwei Verkn"upfungen, der Addition
und der Multiplikation, mit folgenden Eigenschaften:
\begin{enumerate}[label={\bf R.\arabic*},itemsep=0mm]
\item $R$ ist bez"uglich der Addition eine abelsche Gruppe.
\item Die Multiplikation in $R$ ist assoziativ und hat ein Einselement.
\item F"ur drei Element $x,y,z\in R$ gilt $(x+y)z=xz+yz$ und
$z(x+y)=zx+zy$.
\end{enumerate}

Die Menge $\mathbb Z$ der ganzen Zahlen tr"agt die Struktur eines Ringes.
Die Menge der $n\times n$-Matrizen mit Eintr"agen in $\mathbb Z$ ist ebenfalls
ein Ring.
Ist $R$ ein Ring, dann ist die Menge 
\[
R[X]=\{ a_0+a_1X +a_2X^2+\dots +a_nX^n\,|\,a_i\in R\}
\]
der Polynome in der Variablen $X$ ein Ring.

Ein {\em K"orper}  ist ein Ring, so dass die Menge der von $0$ verschiedenen
Elemente eine abelsche Gruppe bez"uglich der Multiplikation bilden.

Die Menge $\mathbb Q$ der rationalen Zahlen tr"agt die Struktur eines
K"orpers.

\subsection{Reste}

\subsection{Charakteristik}

\section{Gauss-Algorithmus in $\mathbb F_p$}

\subsection{Ein Gleichungssystem "uber $\mathbb F_2$}
Besonders einfach ist die Arithmetik im K"orper $\mathbb F_2$,
da es nur ein einziges Element gibt, welches von $0$ verschieden ist,
n"amlich $1$.
Ausserdem ist die Addition nichts anderes als die XOR-Verkn"upfung.

Man kann also das folgende Gleichungssystem "uber $\mathbb F_2$
\begin{equation}
\begin{linsys}{3}
x_1& &   &+&x_3&=&1\\
x_1&+&x_2& &   &=&1\\
   & &x_2& &   &=&1\\
\end{linsys}
\label{ffield:gleichung}
\end{equation}
mit dem Gauss-Algorithmus wie folgt l"osen:
\begin{align*}
\begin{tabular}{|>{$}c<{$}>{$}c<{$}>{$}c<{$}|>{$}c<{$}|}
\hline
1&0&1&1\\
1&1&0&1\\
0&1&0&1\\
\hline
\end{tabular}
&
\rightarrow
\begin{tabular}{|>{$}c<{$}>{$}c<{$}>{$}c<{$}|>{$}c<{$}|}
\hline
1&0&1&1\\
0&1&1&0\\
0&1&0&1\\
\hline
\end{tabular}
\rightarrow
\begin{tabular}{|>{$}c<{$}>{$}c<{$}>{$}c<{$}|>{$}c<{$}|}
\hline
1&0&1&1\\
0&1&1&0\\
0&0&1&1\\
\hline
\end{tabular}
\rightarrow
\begin{tabular}{|>{$}c<{$}>{$}c<{$}>{$}c<{$}|>{$}c<{$}|}
\hline
1&0&0&0\\
0&1&0&1\\
0&0&1&1\\
\hline
\end{tabular}
\end{align*}
Daraus kann man die L"osung
\[
\begin{pmatrix}x_1\\x_2\\x_3\end{pmatrix}
=
\begin{pmatrix}0\\1\\1\end{pmatrix}
\]
ablesen.

Nat"urlich kann man auch die inverse Matrix bestimmen:
\begin{align*}
\begin{tabular}{|>{$}c<{$}>{$}c<{$}>{$}c<{$}|>{$}c<{$}>{$}c<{$}>{$}c<{$}|}
\hline
1&0&1&1&0&0\\
1&1&0&0&1&0\\
0&1&0&0&0&1\\
\hline
\end{tabular}
&
\rightarrow
\begin{tabular}{|>{$}c<{$}>{$}c<{$}>{$}c<{$}|>{$}c<{$}>{$}c<{$}>{$}c<{$}|}
\hline
1&0&1&1&0&0\\
0&1&1&1&1&0\\
0&1&0&0&0&1\\
\hline
\end{tabular}
\rightarrow
\begin{tabular}{|>{$}c<{$}>{$}c<{$}>{$}c<{$}|>{$}c<{$}>{$}c<{$}>{$}c<{$}|}
\hline
1&0&1&1&0&0\\
0&1&1&1&1&0\\
0&0&1&1&1&1\\
\hline
\end{tabular}
\\
&
\rightarrow
\begin{tabular}{|>{$}c<{$}>{$}c<{$}>{$}c<{$}|>{$}c<{$}>{$}c<{$}>{$}c<{$}|}
\hline
1&0&0&0&1&1\\
0&1&0&0&0&1\\
0&0&1&1&1&1\\
\hline
\end{tabular}
\end{align*}
Daraus liest man ab
\[
\begin{pmatrix}
1&0&1\\
1&1&0\\
0&1&0
\end{pmatrix}^{-1}
=
\begin{pmatrix}
0&1&1\\
0&0&1\\
1&1&1\\
\end{pmatrix}.
\]
Auch die eben gefundene L"osung des Gleichungssystems~\eqref{ffield:gleichung}
kann jetzt mit der inversen Matrix bestimmt werden:
\[
\begin{pmatrix}x_1\\x_2\\x_3\end{pmatrix}
=
\begin{pmatrix}
0&1&1\\
0&0&1\\
1&1&1\\
\end{pmatrix}
\begin{pmatrix}1\\1\\1\end{pmatrix}
=
\begin{pmatrix}0\\1\\1\end{pmatrix}.
\]

\subsection{LU- und LR-Zerlegung in $\mathbb F_5$}
Die Operationen in $\mathbb F_p$ werden durch die folgenden Additions-
bzw.~Multiplikationstabellen beschrieben.
\begin{center}
\begin{tabular}{|>{$}c<{$}|>{$}c<{$}>{$}c<{$}>{$}c<{$}>{$}c<{$}>{$}c<{$}|}
\hline
+&0&1&2&3&4\\
\hline
0&0&1&2&3&4\\
1&1&2&3&4&0\\
2&2&3&4&0&1\\
3&3&4&0&1&2\\
4&4&0&1&2&3\\
\hline
\end{tabular}
\qquad
\begin{tabular}{|>{$}c<{$}|>{$}c<{$}>{$}c<{$}>{$}c<{$}>{$}c<{$}>{$}c<{$}|}
\hline
\cdot&0&1&2&3&4\\
\hline
   0 &0&0&0&0&0\\
   1 &0&1&2&3&4\\
   2 &0&2&4&1&3\\
   3 &0&3&1&4&2\\
   4 &0&4&3&2&1\\
\hline
\end{tabular}
\end{center}
Damit ist es jetzt einfach, den Algorithmus zur Bestimmung der LU- und
LR-Zerlegung durchzuf"uhren.
Wir suchen die LU- und die LR-Zerlegung der Matrix
\[
A=\begin{pmatrix}
2&2&4\\
2&3&2\\
0&3&3
\end{pmatrix}.
\]
Der Gauss-Algorithmus liefert
\begin{align*}
\begin{tabular}{|>{$}c<{$}>{$}c<{$}>{$}c<{$}|}
\hline
2&2&4\\
2&3&2\\
0&3&3\\
\hline
\end{tabular}
&
\rightarrow
\begin{tabular}{|>{$}c<{$}>{$}c<{$}>{$}c<{$}|}
\hline
1&1&2\\
0&1&3\\
0&3&3\\
\hline
\end{tabular}
\rightarrow
\begin{tabular}{|>{$}c<{$}>{$}c<{$}>{$}c<{$}|}
\hline
1&1&2\\
0&1&3\\
0&0&4\\
\hline
\end{tabular}
\end{align*}
Daraus liest man die LU-Zerlegung ab:
\[
L
=
\begin{pmatrix}
2&0&0\\
2&1&0\\
0&3&4
\end{pmatrix},
\qquad
U
=
\begin{pmatrix}
1&1&2\\
0&1&3\\
0&0&1
\end{pmatrix}
\qquad
\Rightarrow
\qquad
LU=
\begin{pmatrix}
2&0&0\\
2&1&0\\
0&3&4
\end{pmatrix}
\begin{pmatrix}
1&1&2\\
0&1&3\\
0&0&1
\end{pmatrix}
=
\begin{pmatrix}
2&2&4\\
2&3&2\\
0&3&3
\end{pmatrix}
\]
F"ur die LR-Zerlegung muss $U$ mit $\operatorname{diag}(2,1,4)$
multipliziert werden und $L$ mit der Inversen:
\begin{align*}
L'
&=
L\operatorname{diag}(2,1,4)^{-1}
=
\begin{pmatrix}
2&0&0\\
2&1&0\\
0&3&4
\end{pmatrix}
\begin{pmatrix}
3&0&0\\
0&1&0\\
0&0&4
\end{pmatrix}
=
\begin{pmatrix}
1&0&0\\
1&1&0\\
0&3&1
\end{pmatrix}
\\
R'
&=
\operatorname{diag}(2,1,4)
\begin{pmatrix}
1&1&2\\
0&1&3\\
0&0&1
\end{pmatrix}
=
\begin{pmatrix}
2&0&0\\
0&1&0\\
0&0&4
\end{pmatrix}
\begin{pmatrix}
1&1&2\\
0&1&3\\
0&0&1
\end{pmatrix}
=
\begin{pmatrix}
2&2&4\\
0&1&3\\
0&0&4
\end{pmatrix}
\end{align*}
Kontrolle:
\[
L'R'
=
\begin{pmatrix}
1&0&0\\
1&1&0\\
0&3&1
\end{pmatrix}
\begin{pmatrix}
2&2&4\\
0&1&3\\
0&0&4
\end{pmatrix}
=
\begin{pmatrix}
2&2&4\\
2&3&2\\
0&3&3
\end{pmatrix}
\]

\subsection{Inverse Matrix in $\mathbb F_7$}
Die Additions- und Multiplikationstabellen f"ur $\mathbb F_7$ sind
\begin{center}
\begin{tabular}{|>{$}c<{$}|>{$}c<{$}>{$}c<{$}>{$}c<{$}>{$}c<{$}>{$}c<{$}>{$}c<{$}>{$}c<{$}|}
\hline
+&0&1&2&3&4&5&6\\
\hline
0&0&1&2&3&4&5&6\\
1&1&2&3&4&5&6&0\\
2&2&3&4&5&6&0&1\\
3&3&4&5&6&0&1&2\\
4&4&5&6&0&1&2&3\\
5&5&6&0&1&2&3&4\\
6&6&0&1&2&3&4&5\\
\hline
\end{tabular}
\qquad
\begin{tabular}{|>{$}c<{$}|>{$}c<{$}>{$}c<{$}>{$}c<{$}>{$}c<{$}>{$}c<{$}>{$}c<{$}>{$}c<{$}|}
\hline
\cdot&0&1&2&3&4&5&6\\
\hline
  0  &0&0&0&0&0&0&0\\
  1  &0&1&2&3&4&5&6\\
  2  &0&2&4&6&1&3&5\\
  3  &0&3&6&2&5&1&4\\
  4  &0&4&1&5&2&6&3\\
  5  &0&5&3&1&6&4&2\\
  6  &0&6&5&4&3&2&1\\
\hline
\end{tabular}
\end{center}
Damit k"onnen wir die inverse Matrix von
\[
A
=
\begin{pmatrix}
3&6&5\\
3&1&0\\
0&6&1
\end{pmatrix}
\]
mit dem Gauss-Algorithmus bestimmen:
\begin{align*}
\begin{tabular}{|ccc|ccc|}
\hline
3&6&5&1&0&0\\
3&1&0&0&1&0\\
0&6&1&0&0&1\\
\hline
\end{tabular}
&
\rightarrow
\begin{tabular}{|ccc|ccc|}
\hline
1&2&4&5&0&0\\
0&2&2&6&1&0\\
0&6&1&0&0&1\\
\hline
\end{tabular}
\rightarrow
\begin{tabular}{|ccc|ccc|}
\hline
1&2&4&5&0&0\\
0&1&1&3&4&0\\
0&0&2&3&4&1\\
\hline
\end{tabular}
\\
&
\rightarrow
\begin{tabular}{|ccc|ccc|}
\hline
1&2&0&6&6&5\\
0&1&0&5&2&3\\
0&0&1&5&2&4\\
\hline
\end{tabular}
\rightarrow
\begin{tabular}{|ccc|ccc|}
\hline
1&0&0&3&2&6\\
0&1&0&5&2&3\\
0&0&1&5&2&4\\
\hline
\end{tabular}
\end{align*}
Daraus liest man ab:
\[
A^{-1}
=
\begin{pmatrix}
3&2&6\\
5&2&3\\
5&2&4
\end{pmatrix}
\qquad
\Rightarrow
\qquad
AA^{-1}
=
\begin{pmatrix}
3&6&5\\
3&1&0\\
0&6&1
\end{pmatrix}
\begin{pmatrix}
3&2&6\\
5&2&3\\
5&2&4
\end{pmatrix}
=
\begin{pmatrix}
64&28&56\\
14& 8&21\\
35&14&22
\end{pmatrix}
=
\begin{pmatrix}
1&0&0\\
0&1&0\\
0&0&1
\end{pmatrix}.
\]
Alternativ k"onnen wir daf"ur auch Minoren verwenden.
Dazu brauchen wir zun"achst die Determinante, die wir mit der Sarrus-Formel
berechnen k"onnen:
\begin{align*}
\det(A)
&
=
\left|
\begin{matrix}
3&6&5\\
3&1&0\\
0&6&1
\end{matrix}
\right|
=
3+5\cdot3\cdot6-1\cdot3\cdot 6
=
3+6-4=5.
\end{align*}
Damit wird die inverse Matrix
\begin{align*}
A^{-1}
&=
\frac1{\det(A)}
{
\def\arraystretch{2.2}
\begin{pmatrix}
\def\arraystretch{1}
\phantom{-}
\left|\begin{matrix} 1&0\\6&1 \end{matrix}\right|&
\def\arraystretch{1}
-
\left|\begin{matrix} 6&5\\6&1 \end{matrix}\right|&
\def\arraystretch{1}
\phantom{-}
\left|\begin{matrix} 6&5\\1&0 \end{matrix}\right|
\\
\def\arraystretch{1}
-
\left|\begin{matrix} 3&0\\0&1 \end{matrix}\right|&
\def\arraystretch{1}
\phantom{-}
\left|\begin{matrix} 3&5\\0&1 \end{matrix}\right|&
\def\arraystretch{1}
-
\left|\begin{matrix} 3&5\\3&0 \end{matrix}\right|
\\
\def\arraystretch{1}
\phantom{-}
\left|\begin{matrix} 3&1\\0&6 \end{matrix}\right|&
\def\arraystretch{1}
-
\left|\begin{matrix} 3&6\\0&6 \end{matrix}\right|&
\def\arraystretch{1}
\phantom{-}
\left|\begin{matrix} 3&6\\3&1 \end{matrix}\right|
\end{pmatrix}
}
=
3\cdot
\begin{pmatrix}
 1\cdot 1-0\cdot 6&-6\cdot 1+5\cdot 6& 6\cdot 0-5\cdot 1\\
-3\cdot 1+0\cdot 0& 3\cdot 1-5\cdot 0&-3\cdot 0+5\cdot 3\\
 3\cdot 6-1\cdot 0&-3\cdot 6+6\cdot 0& 3\cdot 1-6\cdot 3
\end{pmatrix}
\\
&=
3\cdot
\begin{pmatrix}
1&3&2\\
4&3&1\\
4&3&6
\end{pmatrix}
=
\begin{pmatrix}
3&2&6\\
5&2&3\\
5&2&4
\end{pmatrix},
\end{align*}
in "Ubereinstimmung mit der Rechnung mit dem Gauss-Algorithmus.

\section{Matrizengruppen in $\mathbb F_p$}

\section{Eigenwerte und Eigenvektoren}

\subsection{Eigenvektoren "uber $\mathbb F_7$}
Wir suchen Eigenwerte und Eigenvektoren der Matrix
\[
A
=
\begin{pmatrix}
0&1&4\\
3&5&6\\
3&4&4
\end{pmatrix}
\]
"uber $\mathbb F_7$.

Wir berechnen zuerst das charakteristische Polynom:
\begin{align*}
\det(A-\lambda E)
&
=
\left|\begin{matrix}
-\lambda&1        &4\\
3       &5-\lambda&6\\
3       &4        &4-\lambda
\end{matrix}\right|
\\
&=
-\lambda(5-\lambda)(4-\lambda)
+1\cdot6\cdot3
+4\cdot3\cdot 4
-3\cdot(5-\lambda)\cdot 4
+4\cdot6\cdot\lambda
-(4-\lambda)\cdot3\cdot 1
\\
&=
-\lambda(6-2\lambda+\lambda^2)
\\
&=
-6\lambda+2\lambda^2-\lambda^3
+4
+6
-4+5\lambda
+3\lambda
-5+3\lambda
\\
&=
-\lambda^3+2\lambda^2+5\lambda+1.
\end{align*}
Um die Eigenwerte zu finden, m"ussen wir also die L"osungen der
Gleichung
\[
-\chi_A(\lambda)
=
\lambda^3
-2\lambda^2
-5\lambda
-1
=
\lambda^3
+5\lambda^2
+2\lambda
+6
=
0
\]
finden.
Da $\mathbb F_7$ nur $7$ Elemente hat, kann man alle Werte durchprobieren,
und findet $1$, $3$ und $5$ als Nullstellen.
Zur Kontrolle berechnen wir das Produkt
\begin{align*}
(\lambda -1)(\lambda-3)(\lambda-5)
&
=
(\lambda^2-4\lambda+3)(\lambda-5)
\\
&=
\lambda^3-4\lambda^2+3\lambda
-5\lambda^2+6\lambda-1
\\
&=
\lambda^3+5\lambda^2+2\lambda+6
\\
&=
-\chi_A(\lambda),
\end{align*}
wir haben also alle Eigenwerte gefunden.

Zu jedem Eigenwert m"ussen jetzt noch ein Eigenvektor gefunden werde.
F"ur $\lambda=1$ verwenden wir 
\begin{align*}
\begin{tabular}{|>{$}c<{$}>{$}c<{$}>{$}c<{$}|}
\hline
0-1&1&4\\
3&5-1&6\\
3&4&4-1\\
\hline
\end{tabular}
&=
\begin{tabular}{|>{$}c<{$}>{$}c<{$}>{$}c<{$}|}
\hline
6&1&4\\
3&4&6\\
3&4&3\\
\hline
\end{tabular}
\rightarrow
\begin{tabular}{|>{$}c<{$}>{$}c<{$}>{$}c<{$}|}
\hline
1&6&3\\
0&0&4\\
0&0&1\\
\hline
\end{tabular}
\rightarrow
\begin{tabular}{|>{$}c<{$}>{$}c<{$}>{$}c<{$}|}
\hline
1&6&3\\
0&0&1\\
0&0&0\\
\hline
\end{tabular}
\rightarrow
\begin{tabular}{|>{$}c<{$}>{$}c<{$}>{$}c<{$}|}
\hline
1&6&0\\
0&0&1\\
0&0&0\\
\hline
\end{tabular}
\end{align*}
Daraus liest man ab, dass die dritte Komponente verschwinden muss, und dass
die zweite Variable frei w"ahlbar ist.
W"ahlen wir sie als $1$, dann ist der Eigenvektor zum Eigenwert $\lambda=1$
\[
v_1
=
\begin{pmatrix}1\\1\\0\end{pmatrix}
\qquad\text{Kontrolle:}\qquad
Av_1
=
\begin{pmatrix}
0&1&4\\
3&5&6\\
3&4&4
\end{pmatrix}
\begin{pmatrix}1\\1\\0\end{pmatrix}
=
\begin{pmatrix}1\\1\\0\end{pmatrix}
\]

F"ur $\lambda = 3$
\begin{align*}
\begin{tabular}{|>{$}c<{$}>{$}c<{$}>{$}c<{$}|}
\hline
0-3&1&4\\
3&5-3&6\\
3&4&4-3\\
\hline
\end{tabular}
&=
\begin{tabular}{|>{$}c<{$}>{$}c<{$}>{$}c<{$}|}
\hline
4&1&4\\
3&2&6\\
3&4&1\\
\hline
\end{tabular}
\rightarrow
\begin{tabular}{|>{$}c<{$}>{$}c<{$}>{$}c<{$}|}
\hline
1&2&1\\
0&3&3\\
0&5&5\\
\hline
\end{tabular}
\rightarrow
\begin{tabular}{|>{$}c<{$}>{$}c<{$}>{$}c<{$}|}
\hline
1&2&1\\
0&1&1\\
0&0&0\\
\hline
\end{tabular}
\rightarrow
\begin{tabular}{|>{$}c<{$}>{$}c<{$}>{$}c<{$}|}
\hline
1&0&6\\
0&1&1\\
0&0&0\\
\hline
\end{tabular}
\end{align*}
Diesmal ist die dritte Komponente frei w"ahlbar, wir w"ahlen sie wieder als
$1$ und erhalten den Eigenvektor zum Eigenwert $\lambda=3$
\[
v_3=\begin{pmatrix}1\\6\\1\end{pmatrix}
\qquad\text{Kontrolle:}\qquad
Av_3
=
\begin{pmatrix}
0&1&4\\
3&5&6\\
3&4&4
\end{pmatrix}
\begin{pmatrix}1\\6\\1\end{pmatrix}
=
\begin{pmatrix}3\\4\\3\end{pmatrix}
=
3\begin{pmatrix}1\\6\\1\end{pmatrix}
=
3v_3.
\]

Schliesslich untersuchen wir den Eigenwert $\lambda=5$.
\begin{align*}
\begin{tabular}{|>{$}c<{$}>{$}c<{$}>{$}c<{$}|}
\hline
0-3&1&4\\
3&5-3&6\\
3&4&4-3\\
\hline
\end{tabular}
&=
\begin{tabular}{|>{$}c<{$}>{$}c<{$}>{$}c<{$}|}
\hline
2&1&4\\
3&0&6\\
3&4&6\\
\hline
\end{tabular}
\rightarrow
\begin{tabular}{|>{$}c<{$}>{$}c<{$}>{$}c<{$}|}
\hline
1&4&2\\
0&2&0\\
0&6&0\\
\hline
\end{tabular}
\rightarrow
\begin{tabular}{|>{$}c<{$}>{$}c<{$}>{$}c<{$}|}
\hline
1&0&2\\
0&1&0\\
0&0&0\\
\hline
\end{tabular}
\end{align*}
Die dritte Komponente ist $0$.
\[
v_5
=
\begin{pmatrix}5\\0\\1\end{pmatrix}
\qquad\text{Kontrolle:}\qquad
Av_5
=
\begin{pmatrix}
0&1&4\\
3&5&6\\
3&4&4
\end{pmatrix}
\begin{pmatrix}5\\0\\1\end{pmatrix}
=
\begin{pmatrix}4\\0\\5\end{pmatrix}
=
5\begin{pmatrix}5\\0\\1\end{pmatrix}
=
5v_5.
\]








