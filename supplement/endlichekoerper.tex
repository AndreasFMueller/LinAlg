%
% endlichekoerper.tex
%
% (c) 2017 Prof Dr Andreas Mueller, Hochschule Rapperswil
%
\chapter{Lineare Algebra in endlichen K"orpern}
\rhead{Endliche K"orper}

Die Konstruktionen in Kapitel~1 und 2 des Skripts verlangen nicht mehr
als die Grundoperationen.
Die vorgestellten Algorithmen sollten daher in jeder Zahlenmenge
durchf"uhrbar sein, sobald die arithmetischen Operationen zur
Verf"ugung stehen.
Zum Beispiel wird die Theorie zwar fast ausschliesslich f"ur reelle
Zahlen entwickelt, doch funktioniert sie genau gleich f"ur rationale
oder komplexe Zahlen.
Alle drei Zahlenmengen sind sogenannte K"orper, sie enthalten
unendlich viele Elemente.
Es zeigt sich, dass es auch K"orper gibt, die nur endliche viele
Elemente enthalten.
Die K"orper werden jeweils durch eine Primzahl $p$ charakterisiert.
In der Kryptographie haben vor allem die K"orper zur Primzahl $p=2$
eine besondere Bedeutung.
In diesem Kapitel wird gezeigt, wie man in diesen K"orpern rechnet
und es wird an Beispielen illustriert, wie die wohlbekannten Algorithmen
der linearen Algebra sich auf diese K"orper "ubertragen.

\section{Endliche K"orper}
\subsection{K"orper}
Der Begriff des K"orpers fasst die Eigenschaften zusammen, die f"ur
die Skalare in der linearen Algebra ben"otigt werden.

Eine {\em Gruppe} ist eine Menge $G$ mit einer Verkn"upfung, die zwei Elementen
$a,b\in G$ das Element $ab\in G$ zuordnet.
Ausserdem m"ussen folgen Axiome erf"ullt sein:
\begin{enumerate}[label={\bf G.\arabic*},itemsep=0mm]
\item
Die Verkn"upfung ist assoziativ, d.~h.~$(ab)c=a(bc)$ f"ur alle $a,b,c\in G$.
\item
Es gibt ein Element $e\in G$ mit der Eigenschaft $eg=ge=g$ f"ur alle $g\in G$,
genant das Neutralelement.
\item
F"ur jedes Element $g\in G$ gibt es ein Element $g^{-1}\in G$ welches
$gg^{-1}=g^{-1}g=e$.
\end{enumerate}
Eine Gruppe heisst {\em abelsch} wenn $ab=ba$ f"ur alle $a,b\in G$.

Ein {\em Ring} ist eine Menge $R$ mit zwei Verkn"upfungen, der Addition
und der Multiplikation, mit folgenden Eigenschaften:
\begin{enumerate}[label={\bf R.\arabic*},itemsep=0mm]
\item $R$ ist bez"uglich der Addition eine abelsche Gruppe.
\item Die Multiplikation in $R$ ist assoziativ und hat ein Einselement.
\item F"ur drei Element $x,y,z\in R$ gilt $(x+y)z=xz+yz$ und
$z(x+y)=zx+zy$.
\end{enumerate}

Die Menge $\mathbb Z$ der ganzen Zahlen tr"agt die Struktur eines Ringes.
Die Menge der $n\times n$-Matrizen mit Eintr"agen in $\mathbb Z$ ist ebenfalls
ein Ring.
Ist $R$ ein Ring, dann ist die Menge 
\[
R[X]=\{ a_0+a_1X +a_2X^2+\dots +a_nX^n\,|\,a_i\in R\}
\]
der Polynome in der Variablen $X$ ein Ring.

Ein {\em K"orper}  ist ein Ring, so dass die Menge der von $0$ verschiedenen
Elemente eine abelsche Gruppe bez"uglich der Multiplikation bilden.

Die Menge $\mathbb Q$ der rationalen Zahlen tr"agt die Struktur eines
K"orpers.

\subsection{Reste}

\subsection{Charakteristik}

\section{Gauss-Algorithmus in $\mathbb F_p$}

\subsection{Ein Beispiel "uber $\mathbb F_2$}
Besonders einfach ist die Arithmetik im K"orper $\mathbb F_2$,
da es nur ein einziges Element gibt, welches von $0$ verschieden ist,
n"amlich $1$.
Ausserdem ist die Addition nichts anderes als die XOR-Verkn"upfung.

Man kann also das folgende Gleichungssystem "uber $\mathbb F_2$
\begin{equation}
\begin{linsys}{3}
x_1& &   &+&x_3&=&1\\
x_1&+&x_2& &   &=&1\\
   & &x_2& &   &=&1\\
\end{linsys}
\label{ffield:gleichung}
\end{equation}
mit dem Gauss-Algorithmus wie folgt l"osen:
\begin{align*}
\begin{tabular}{|>{$}c<{$}>{$}c<{$}>{$}c<{$}|>{$}c<{$}|}
\hline
1&0&1&1\\
1&1&0&1\\
0&1&0&1\\
\hline
\end{tabular}
&
\rightarrow
\begin{tabular}{|>{$}c<{$}>{$}c<{$}>{$}c<{$}|>{$}c<{$}|}
\hline
1&0&1&1\\
0&1&1&0\\
0&1&0&1\\
\hline
\end{tabular}
\rightarrow
\begin{tabular}{|>{$}c<{$}>{$}c<{$}>{$}c<{$}|>{$}c<{$}|}
\hline
1&0&1&1\\
0&1&1&0\\
0&0&1&1\\
\hline
\end{tabular}
\rightarrow
\begin{tabular}{|>{$}c<{$}>{$}c<{$}>{$}c<{$}|>{$}c<{$}|}
\hline
1&0&0&0\\
0&1&0&1\\
0&0&1&1\\
\hline
\end{tabular}
\end{align*}
Daraus kann man die L"osung
\[
\begin{pmatrix}x_1\\x_2\\x_3\end{pmatrix}
=
\begin{pmatrix}0\\1\\1\end{pmatrix}
\]
ablesen.

Nat"urlich kann man auch die inverse Matrix bestimmen:
\begin{align*}
\begin{tabular}{|>{$}c<{$}>{$}c<{$}>{$}c<{$}|>{$}c<{$}>{$}c<{$}>{$}c<{$}|}
\hline
1&0&1&1&0&0\\
1&1&0&0&1&0\\
0&1&0&0&0&1\\
\hline
\end{tabular}
&
\rightarrow
\begin{tabular}{|>{$}c<{$}>{$}c<{$}>{$}c<{$}|>{$}c<{$}>{$}c<{$}>{$}c<{$}|}
\hline
1&0&1&1&0&0\\
0&1&1&1&1&0\\
0&1&0&0&0&1\\
\hline
\end{tabular}
\rightarrow
\begin{tabular}{|>{$}c<{$}>{$}c<{$}>{$}c<{$}|>{$}c<{$}>{$}c<{$}>{$}c<{$}|}
\hline
1&0&1&1&0&0\\
0&1&1&1&1&0\\
0&0&1&1&1&1\\
\hline
\end{tabular}
\\
&
\rightarrow
\begin{tabular}{|>{$}c<{$}>{$}c<{$}>{$}c<{$}|>{$}c<{$}>{$}c<{$}>{$}c<{$}|}
\hline
1&0&0&0&1&1\\
0&1&0&0&0&1\\
0&0&1&1&1&1\\
\hline
\end{tabular}
\end{align*}
Daraus liest man ab
\[
\begin{pmatrix}
1&0&1\\
1&1&0\\
0&1&0
\end{pmatrix}^{-1}
=
\begin{pmatrix}
0&1&1\\
0&0&1\\
1&1&1\\
\end{pmatrix}.
\]
Auch die eben gefundene L"osung des Gleichungssystems~\eqref{ffield:gleichung}
kann jetzt mit der inversen Matrix bestimmt werden:
\[
\begin{pmatrix}x_1\\x_2\\x_3\end{pmatrix}
=
\begin{pmatrix}
0&1&1\\
0&0&1\\
1&1&1\\
\end{pmatrix}
\begin{pmatrix}1\\1\\1\end{pmatrix}
=
\begin{pmatrix}0\\1\\1\end{pmatrix}.
\]

\section{Matrizengruppen in $\mathbb F_p$}

\section{Eigenwerte und Eigenvektoren}

