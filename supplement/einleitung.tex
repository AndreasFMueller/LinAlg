%
% einleitung.tex -- Einleitung zu Tabea's Supplement
%
% (c) 2017 Prof Dr Andreas Müller, Hochschule Rapperswil
%
\chapter*{Einleitung}
\lhead{Einleitung}
\rhead{}
Lineare Algebra gehört zu fast jedem Grundstudium der exakten
Naturwissenschaften und der Ingenieurwissenschaften.
Die meisten Studenten lernen lineare Algebra sehr früh in ihrer Ausbildung,
also zu einer Zeit, wo die möglichen Anwendungen noch beschränkt sind,
und auch die beschränkten Kenntnisse anderer mathematischer Teilgebiete,
insbesondere der Analsysis oder der Physik, erlauben nur beschränkte 
Anwendungen.
Doch selbst wenn solche Anwendungen dargestellt werden können, bleiben
die mathematischen Möglichkeiten trotzdem limitiert.
Meistens beschränkt sich die Anwendung auf die zweifellos elegante 
Matrixnotation.
Auf der Strecke bleiben die reichhaltigen Strukturen, die die
lineare Algebra verständlich machen kann.
Als Beispiel seien genannt:
\begin{itemize}
\item
In der Analysis lernt man das Konzept des Grenzwertes und gewöhnt
sich gleichzeitig eine Denkweise an, dass es zu jedem mathematischen Problem
eine approximative Lösung gibt, die mit entsprechendem Computer-Aufwand
genügend genau gemacht werden kann.
Dabei wird die Tatsache verwendet, dass die Menge der rationalen Zahlen
$\mathbb Q$ dicht liegt in $\mathbb R$.
Durch dieses Vorgehen wird die Unterscheidung zwischen $\mathbb Q$ und
$\mathbb R$ verwischt.
Noch gravierender ist allerdings ein anderer Unterschied.
Die in der Algebra formulierten Algorithmen bestimmen eine Lösung immer
in endlich vielen Schritten exakt, während die Analysis Lösungen in
einer grösser werdenden Anzahl Schritten immer genauer ermittelt.
Dabei wird die Anzahl der nötigen Schritte beliebig gross, wenn die
Genauigkeitsanforderungen gesteigert werden.
\item
Lineare Algebra ist nicht nur eine elegante Notation, sondern ein
Fachgebiet mit überaus reichhaltigen mathematischen Strukturen.
Zum Beispiel bilden die Unterräume eines Vektorraumes einen Verband,
ähnlich wie die Mengenlehre oder die Prädikatenlogik eine
Verbandsstruktur haben.
\item
Oberflächliche Betrachtung der Matrizenalgebra lässt vermuten, dass diese
auf Operationen beschränkt ist, die sich mit zwei Indizes beschreiben
lassen.
Die Indizes haben dabei nur die Bedeutung eines unvermeidbaren Übels.
Doch eine etwas sorgfältigere Untersuchung enthüllt, dass die Indexnotation
eine Vereinfachung ist, die den Blick auf die unterschiedliche Bedeutung
der beiden Indizes einer Matrix.
Die richtige Betrachtungsweise führt auf den Dualraum, das Tensorprodukt
von Vektorräumen und beliebige Tensoren.
Dabei betrachten wir Tensoren wieder nicht einfach als ein Objekt mit
vielen Indizes, wie das leider verschiedene Darstellungen der Tensorrechnungen
für Ingenieure tun.
Dadurch geht die ganze reichhaltige Struktur verloren.
\item
Das Skalarprodukt wird meistens im Rahmen der Vektorgeometrie 
eingeführt und seine Anwendungen bleiben oft auf die geometrische
Situation beschränkt.
Doch die Struktur eines Skalarproduktes ist viel allgemeiner.
Verfolgen wir sie etwas sorgfältiger, werden wir auf den Minkowski-Raum,
die Lorentz-Gruppe und mit etwas mehr Fleiss auch auf die allgemeine Theorie
der Hilberträume geführt.
\end{itemize}
Es fehlt eine Darstellung weiterführender Themen der linearen
Algebra, die einen Anwender von der Anfängervorlesung zur vielseitigen
Anwendung und den reichhaltigen, überall in der Mathematik und Physik
anwendbaren Strukturen führt.
Die abstrakteren Strukturen verlangen zwar, dass sich ein Student
etwas mehr anstrengt, doch das so gewonnene, viel tiefere Verständnis
ermögich gleichzeitig eine effizientere und erweiterungsfähige 
Gestaltung der Anwendung.

Das vorliegende Buch versucht, diese Lücke zu füllen.
Es entstand im Rahmen des Projektes, die junge Koautorin auf ihren
Lehrauftrag im Fache Lineare Algebra vorzubereiten.
Es sollte ihr ermöglichen, die moderne lineare Algebra  etwas mehr
aus der Vogelperspektive betrachten zu können.
Sie sollte in die Lage versetzt werden, den elementaren Stoff mit
einem ähnlich breiten Hintergrund unterrichten zu können, wie ein
ausgebildeter Mathematiker dies dank seiner langjährigen Erfahrung
kann.

\section*{Aufbau}
Die folgenden Kapitel sind so konzipiert, dass sie weitgehend voneinander
unabhängig gelesen werden können.
Sie beginnen jeweils mit einer Einleitung, in der die zentralen Ideen
und die Motivation hinter den in dem Kapitel aufgebauten Konstruktionen
zusammengefasst werden.
Es wird angestrebt, dem Leser einen ersten Eindruck zu geben, die 
ermöglichen soll zu entscheiden, ob die Ideen für sein Anwendungsproblem
möglicherweise nützlich sein könnten.
Um die Details zu verstehen, muss er sich dann jedoch auf den
Hauptteil des Kapitels einlassen.

\section*{Danksagung}






