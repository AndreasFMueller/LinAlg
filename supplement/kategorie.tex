%
% kategorie.tex -- Kategorien
%
% (c) 2017 Prof Dr Andreas Müller, Hochschule Rapperswil
%
\chapter{Kategorien%
\label{chapter:kategorien}}
\lhead{Kategorien}
\rhead{}
Im vorangegangenen Kapitel haben wir gezeigt, wie die aus der
elementaren Theorie bekannte Struktur der Spaltenvektoren und
Matrizen in eine allgemeiner Form zu fassen.
Die Vektorräume bilden in diesem Lichte die abstrakte algebraische Struktur,
in welcher offenbar die gesamte elementare lineare Algebra stattfinden
kann.
Doch was ist eigentlich eine ``mathematische Struktur''?
In diesem Kapitel sollen zunächst verschiedene mathematische Strukturen
so rekapituliert werden, dass ihre Gemeinsamkeiten erkennbar werden.
Daraus leiten wir dann die Begriffe Kategorie und Funktor ab, welche
wir in Beispielen in späteren Kapiteln weiter vertiefen werden.

\section{Beispiele von Strukturen}
\rhead{Beispiele}
In diesem Abschnitt wollen wir ein paar andere bereits bekannte
Arten von Strukturen zusammenstellen und auf eine Art beschreiben,
die die Gemeinsamkeit mit den Vektorräumen.

\subsection{Mengen und Abbildungen}
Mengen für sich genommen haben keine Struktur.
Das einzige, was uns ermöglich, sie miteinander zu verbinden, sind die
Abbildungen zwischen Mengen.
Zum Beispiel können wir mit Hilfe von bijektiven Abbildungen zwischen
endlichen Mengen herausfinden, welche Mengen gleich viele Elemente haben.

Zu zwei Mengen $A$ und $B$ können wir also die Menge 
\[
\operatorname{Hom}(A,B)
=
\{ f\colon A\to B\;|\; \text{$f$ ein Abbildung}\}
\]
der Abbildungen von $A$ nach $B$ bilden.
Falls $A=B$ ist, finden wir immer die identische Abbildung
$1_{A}\in \operatorname{Hom}(A,A)$, welche definiert ist durch
$1_A(x)=x$
für jedes $x\in A$.

Ausserdem kann man die Abbildungen
$\operatorname{Hom}(A,B)$
mit
$\operatorname{Hom}(B,C)$
zusammensetzen, man hat also eine Abbildung
\[
\circ
\colon
\operatorname{Hom}(B,C)
\times
\operatorname{Hom}(A,B)
:
(f,g)
\mapsto
f\circ g.
\]
Die identische Abbildung $1_A$ spielt dabei die spezielle Rolle eines
neutralen Elements,
es ist nämlich
\[
f\circ 1_A
=
f
\qquad\text{und}\qquad
1_B \circ f = f
\]
für $f\in\operatorname{Hom}(A,B)$.

Auf den ersten Blick sieht es so aus, als hätten wir mit dieser Fokusierung
auf Mengen und Abbildungen die ganze Information über die Elemente einer Menge
verloren.
Dem ist jedoch nicht so.
Dazu braucht man nur eine Menge mit einem einzigen Element, wir bezeichnen
sie mit $E=\{*\}$.
Eine Abbildung $f\colon E\to A$ ist festgelegt durch das Bild $f(*)$
des einzigen Elements von $E$.
Daher ist die Menge der Abbildungen von $E$ in die Menge $A$ g
\[
\operatorname{Hom}(E,A)
=
A,
\]
wir finden daher die Menge der Punkte $A$ wieder als eine Menge von
Abbildungen $\operatorname{Hom}(E,A)$.

\subsection{Körper und Homomorphismen}
Im Kapitel~\ref{chapter:vektorraum} haben wir den Begriff des Körpers
kennengelernt.
Wenn wir Körper miteinander vergleichen wollen, brauchen wir Abbildungen
zwischen zwei Körpern, die mit der Struktur des Körpers verträglich ist.
Seien also $K$ und $L$ Körper und $f\colon K\to L$ ein Abbildung.
Die Abbildung $f$ muss die Struktur erhalten, also muss gelten
\[
\begin{aligned}
f(0)&=0&f(1)&=1\\
f(a+b)&=f(a)+f(b)&f(ab)&=f(a)f(b)\\
f(-a)&=-f(a)&f(a^{-1})&=f(a)^{-1}
\end{aligned}
\]
Eine solche Abbildung heisst ein Körper-Homomorphismus.
Die identisch Abbildung $1_K\colon K\to K$ ist ein Körper-Homomorphismus.
Die Menge
\[
\operatorname{Hom}_{\text{Körper}}(K,L)
=
\{f\colon K\to L\;|\;\text{$f$ ist ein Körper-Homomorphismus}\}
\]
mit der Verkettung von Abbildungen
bildet daher eine Menge mit ähnlichen Eigenschaften wie die
Mengen $\operatorname{Hom}(A,B)$ für Mengen $A$ und $B$ im
vorangegangenen Abschnitt.

\subsection{Gruppen und Homomorphismen}
Eine Gruppe ist eine etwas einfachere Struktur als ein Körper.

\begin{definition}
Eine Menge $G$ mit einer multiplikativ geschriebenen Verknüpfung heisst
eine Gruppe, wenn gilt:
\begin{compactenum}
\item Die Verknüpfung ist assoziativ.
\item Es gibt ein neutrales Element $e$, also $ge=g$ und $eg=g$ für alle 
$g\in G$.
\item Zu jedem Element $g\in G$ gibt es das sogenannte inverse
Element $g^{-1}\in G$ mit der Eigenschaft $gg^{-1}=g^{-1}g=e$.
\end{compactenum}
Ein Homomorphismus ist eine Abbildung $\varphi\colon G\to H$ zwischen
Gruppen mit der Eigenschaft
$\varphi(g_1g_2)=\varphi(g_1)\varphi(g_2).$
\end{definition}

Beispiel von Gruppen sind die Mengen 
\[
C_n = \{ k\;|\; \text{$k$ ist Rest bezüglich $n$}\}
\]
mit dem neutralen Element $e=0$.
Das Element $1\in C_n$ hat eine spezielle Bedeutung, denn durch
wiederholte Verknüpfung mit sich selbst entstehen alle Elemente von $C_n$.
$C_n$ heisst die zyklische Gruppe der Ordnung $n$.

Wie bei den Mengen verliert man kaum Information über die Gruppen.
Aus der Definition eines Homomorphismus folgt zunächst $\varphi(e)=e$.
Da alle Elemente von $C_n$ aus $1$ erzeugt werden können, ist 
ein Homomorphismus $\varphi\colon C_n\to G$ gegeben durch das 
Element $\varphi(1)\in G$.
Die Homomorphismen von $C_n$ nach $G$ charakterisieren also diejenigen
Elemente $g\in G$, die die Eigenschaft $g^n=e$ haben.

\section{Kategorien}
\rhead{Kategorien}
In allen genannten Beispielen haben wir mit Objekten und den Abbildungen
zwischen diesen Objekten zu tun.
Wir haben auch gesehen, dass wir durch Einschränkung der Perspektive auf
Objekte und Abbildungen nichts an Information über den ``Inhalt'' der
Objekte verlieren.
Wir können daher noch eine Schritt weiter gehen und den Begriff einer
Kategorie definieren.

\begin{definition}
Eine {\em Kategorie} $\cal C$ besteht aus einer Menge von {\em Objekten},
bezeichnet mit $\operatorname{obj}(\cal C)$, und einer Menge von
{\em Morphismen} für jedes Paar von $A,B\in\operatorname{obj}(\cal C)$ von
Objekten, bezeichnet mit
$\operatorname{Hom}_{\cal C}(A,B)$.
Die Menge der Morphismen $\operatorname{Hom}_{\cal C}(A,A)$
enthält den speziellen Morphismus $1_A$ und es gilt
\[
f\circ 1_A = f
\qquad\text{und}\qquad
1_A\circ g = g
\]
für
$f\in\operatorname{Hom}_{\cal C}(A,B)$
und
$g\in\operatorname{Hom}_{\cal C}(B,A)$.
\end{definition}

Eine Kategorie fokussiert sich also auf die Objekte und die Abbildungen 
zwischen den Objekten.
Damit ist es möglich, viele Strukturen in der Mathematik auf ähnliche
Weise zu behandeln.
Die nachfolgenden Beispiele sollen dies illustrieren.

\subsection{Beispiele von Kategorien}
\subsubsection{Mengen}
Die Objekte der Kategorie der Mengen sind die Mengen, die Morphismen
sind die Abbildungen zwischen Mengen.

\subsubsection{Endliche Mengen}
Beschränkt man sich in der Kategorie der Mengen auf Objekte mit
endlich vielen Elementen, erhält man die Kategorie der endlichen Mengen.
Dabei muss man nicht einmal wissen, was eine endliche Menge ist,
denn man kann unendliche Mengen allein mit Hilfe von Begriffen einer
Kategorie beschreiben.
Eine Menge ist endlich, wenn jede injektive Selbstabbildung von $A$ auch
surjektiv ist.
Wir müssen also die Begriffe injektiv und surjektiv von Abbildungen
durch Eigenschaften von Morphismen ausdrücken, die sich für jede beliebige
Kategorie definieren lassen.

\begin{definition}
Ein Morphismus $f\colon A\to B$ heisst {\em Monomorphismus}, wenn 
für zwei beliebige Morphismen $g_1,g_2\colon C\to A$ mit
$f\circ g_1=f\circ g_2$ folgt, dass $g_1=g_2$ ist.
Ein Morphismus $f\colon A\to B$ heisst {\em Epimorphismus}, wenn
für zwei beliebige Morphismen $g_1,g_2\colon B\to D$ mit
$g_1\circ f=g_2\circ f$ folgt, dass $g_1=g_2$ ist.
\end{definition}
\index{Monomorphismus}%
\index{Epimorphismus}%
In der Kategorie der Mengen sind die Monomorphismen die injektiven
Abbildungen und die Epimorphismen sind die surjektiven Abbildungen.
Eine Menge $A$ ist also endlich, wenn jeder Monomorphismus in
$\operatorname{Hom}_{\text{Mengen}}(A,A)$ auch ein Epimorphismus ist.

\subsubsection{Körper}

\subsubsection{Gruppen}
Die Kategorie der Gruppen hat als Objekte die Gruppen und als
Morphismen die Homomorphismen.

\subsubsection{Vektorräume}
Die Kategorie der $K$-Vektorräume hat als Morphismenmengen die linearen
Abbildungen zwischen Vektorräumen.

\section{Funktoren}
\rhead{Funktoren}
Die Kategorie der endlichen Mengen ist in der Kategorie der Mengen enthalten.
Es gibt also eine Zuordnung, die jeder endlichen Menge eine Menge in
der ganzen Mengenkategorie zuordnet (das ist natürlich immer noch die
gleiche Menge), und jedem Morphismus einen Morphisums in der ganzen
Kategorie.
Wir können diese Zuordnung aber auch allgemeiner definieren:

\begin{definition}
Sind $\cal C$ und $\cal D$ zwei Kategorien, dann heisst eine Abbildung
$F\colon\cal C\to \cal D$, einem Objekt $A\in \operatorname{obj}(\cal C)$
ein Objekt $F(A)\in\operatorname{obj}(\cal D)$ zuordnet und einem
Morphismus $f\in\operatorname{Hom}_{\cal C}(A,B)$ einen Morphismus
$F(f)\in\operatorname{Hom}_{\cal D}(F(A),F(B))$, ein {\em kovarianter Funktor},
wenn
\[
F(f\circ g)=F(f)\circ F(g)
\qquad\text{und}\qquad
F(1_A)=1_{F(A)}
\]
gilt.
\end{definition}

Die Einbettung der Kategorie der endlichen Mengen in die Kategorie der
Mengen ist ein Funktor.

\subsubsection{Vergissfunktor}
Wir können auch einen Funktor von der Kategorie der Vektorräume in die
Kategorie der Mengen konstruieren.
Dazu ordnen wir einem Vektorraum $V$ die Menge der Vektoren zu, 
wir vergessena lso einfach die Tatsache, dass die Vektoren addiert und
mit Skalaren multipliziert werden können.
Lineare Abbildungen zwischen Vektorräumen sind natürlich auch Abbildungen
zwischen den Mengen von Vektoren, wir vergessen einfach die Tatsache,
dass die Abbildungen linear sind.
Allerdings gibt es viel mehr Abbildungen, die nicht linear sind.
Der Funktor $V$ ist also definiert durch
\begin{align*}
V
&\colon
\operatorname{obj}(\text{Vektorraum})
\to
\operatorname{obj}(\text{Menge})
:
U\mapsto V(U)=U\\
&\colon
\operatorname{Hom}_{\text{Vektorraum}}(U,W)
\to
\operatorname{Hom}_{\text{Menge}}(U,W)
:
f\mapsto V(f)=f
\end{align*}
Der Funktor $V$ heisst der Vergissfunktor.
Analog zur Kategorie der Vektorräume kann für jede Kategorie von Objekten,
die Mengen sind, ein Vergissfunktor definiert werden, also zum Beispiel
für die Kategorie der Gruppen.

\subsubsection{Vektorraumkonstruktion}
Viele Konstruktionen in der Mathematik lassen sich als Funktoren
beschreiben.
Als Beispiel betrachten wir eine Konstruktion, die endlichen Mengen
einen $K$-Vektorraum zuordnet, und einer Abbildung von endlichen Mengen
eine lineare Abbildung zwischen Vektorräumen.

Sei also $A$ eine endliche Menge.
Wir konstruieren den Vektorraum der Funktionen
\[
S(A) = \{v\colon A\to K\;|\;\text{$v$ eine Abbildung von $A$ nach $K$}\}.
\]
Funktionen können addiert und mit Skalaren in $K$ multipliziert werden,
bilden also auf natürliche Weise einen Vektorraum.

Eine Abbildung $f\colon A\to B$ macht aus einer Funktion $v\colon B\to K$
durch Zusammensetzung eine Funktion $S(f)(v)=v\circ f\colon A\to K$.
Aus einem Morphismus $f\in\operatorname{Hom}_{\text{Menge}}$ wird also 
ein Morphismus
$S(f)\in\operatorname{Hom}_{\text{Vektorraum}}(S(B),S(A))$
Dies sieht genauso aus wie die Definition wie eines kovarianten Funktors,
ausser dass $S(f)$ die falsche Richtung hat.
Wir definieren daher:

\begin{definition}
Sind $\cal C$ und $\cal D$ zwei Kategorien, dann heisst eine Abbildung
$F\colon\cal C\to \cal D$, einem Objekt $A\in \operatorname{obj}(\cal C)$
ein Objekt $F(A)\in\operatorname{obj}(\cal D)$ zuordnet und einem
Morphismus $f\in\operatorname{Hom}_{\cal C}(A,B)$ einen Morphismus
$F(f)\in\operatorname{Hom}_{\cal D}(F(B),F(A))$, ein
{\em kontravarianter Funktor},
wenn
\[
F(f\circ g)=F(g)\circ F(f)
\qquad\text{und}\qquad
F(1_A)=1_{F(A)}
\]
gilt.
\end{definition}

Zu jeder endlichen Menge $A$ kann man also einen Vektorraum $S(A)$
konstruieren, so dass die Elemente $a\in A$ zu Basisvektoren von $S(A)$
werden.

\subsubsection{Weitere Beispiele von Funktoren}
Wir werden weitere Beispiele von Funktoren im Kapitel~\ref{chapter:homologie}
kennenlernen.
Dort wird ein Funktor von der Kategorie der Polyeder oder
genauer von simplizialen Komplexen in die Kategorie der Vektorräume 
konstruiert.
Es wird sich zeigen, dass der eulersche Polyedersatz in diesem Rahmen
ganz allgemein verstanden werden kann.
Dieser Funktor hat auch Anwendungen bei der Analyse neuronaler Netzwerke.

