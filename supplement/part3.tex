%
% part3.tex -- Berechnung
%
% (c) 2017 Prof Dr Andreas Müller, Hochschule Rapperswil
%
\part{Berechnung}

\chapter*{Numerische lineare Algebra}
\lhead{Numerische lineare Algebra}
\rhead{}
In einer Grundlagen-Vorlesung über lineare Algebra werden meistens nur
die einfachsten numerischen Methoden besprochen.
Natürlich gehört der Gauss-Algorithmus dazu, doch seine Komplexität
von $O(n^3)$ ist für viele Anwendungen viel zu schlecht.
Zum Beispiel treten bei Anwendungen mit finiten Elementen sehr grosse
Gleichungssysteme auf, die jedoch oft eine besondere Struktur haben.
Daher wurden besondere Algorithmen geschaffen, mit denen auch solche
Gleichungssysteme gelöst werden können.






