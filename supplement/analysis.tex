%
% analysis.tex
%
% (c) 2017 Prof Dr Andreas Müller, Hochschule Rapperswil
%
\chapter{Analysis}
Die grundlegenden Operationen der Analysis sind lineare Operatoren
zwischen Funktionenr"aumen.
Im Analysis-Unterricht für Studienanf"anger steht dieser Aspekt jedoch
praktisch immer im Hintergrund.
Vor allem in der Theorie der partiellen Differentialgleichungen 
kann man jedoch einen Nutzen daraus ziehen, die Funktionenr"aume
sehr viel genauer als Vektorr"aume zu betrachten, und die
Differentialoperatoren als lineare Operatoren auf diesen R"aumen.
Durch Wahl einer geeigneten Norm kann man auch die Fragen der
Regularit"at als Fragen an die Stetigkeit dieser Operatoren
formulieren.

In diesem Kapitel wird der Versuch unternommen, die linear algebraischen
Aspekte der Analysis in den Vordergrund zu stellen.
Die Ableitung wird dabei zu einer Derivation auf dem Raum der
differenzierbaren Funktionen, die Stammfunktion eine partielle
Inverse.
Bei der merhdimensionalen Analysis kommen weiter Aspekte hinzu, die
zweiten Ableitungen bilden zum Beispiel bereits eine Matrix,
und Matrizenrechnung ist n"otig f"ur die Formulierung des
$n$-dimensionalen Newton-Algorithmus.

Im abschliessenden Absatz "uber Sobolevr"aume wird gezeigt, wie
die Stetigkeit und Glattheit von L"osungen von Differentialgleichungen
in das Skalarprodukt des Raumes eingebaut werden kann.
Die L"osungen sind dann Funktionen im Kern des Differentialoperators,
und der Differentialoperator selbst hat bez"uglich des Skalarproduktes
besondere Eigenschaften, man erwartet, dass er selbstadjungiert ist.
In diesem Fall folgt wie bei symmetrischen Matrizen, dass die
Eigenfunktionen des Operators zu verschiedenen Eigenwerten orthogonal 
sind, aber auch dass die Eigenfunktionen glatt sind, wie das auch
beim eindimensionalen Spezialfall der trigonometrischen Funktionen
zutrifft.

\section{Ableitung und Integral}

\section{Mehrdimensionale Analysis}
\subsection{Gradient}
\subsection{Jacobi-Matrix}
\subsection{Taylor-Reihe und Tensoren}
\subsection{Newton-Algorithmus}

\section{Sobolev-R"aume}

