%
% nichtstandardanalysis.tex -- Nichstandard Analysis
%
% (c) 2017 Prof Dr Andreas Müller, Hochschule Rapperswil
%
\chapter{Nichtstandard Analysis}
Zur Zeit von Newton, Leibniz und Euler war es üblich, zur Begründung der
Formeln der Analysis nicht weiter definierte infinitesimale Grössen
zu postulieren.
Diese hat man sich als noch kleiner als die reellen Zahlen vorgestellt,
ohne dafür eine konkrete Konstruktion anzugeben.

Im 20.~Jahrhundert wurde im Rahmen der Revision der Grundlagen der Mathematik
klar, dass es Modelle der Analsysis gibt, in denen die Existenz solcher
infinitesimaler Grössen plötzlich möglich wurde.
Ziel dieses Kapitels ist ein einfaches Modell für infinitesimale 
Zahlen zu entwicklen, welches mit der Theorie der Vektorräume
verstanden werden kann.

In dieser Nichstandard-Analysis werden die bekannten reellen Zahlen
erweitert um nichtstandard reelle Zahlen.
Es gibt dann zu jeder standard reellen Zahl $x$ weitere reellen Zahlen $y$ 
mit der Eigenschaft
\[
|y - x| < \varepsilon\quad\forall \varepsilon > 0 \text{ und $\varepsilon$ ist standard}
\]
Jede standard reelle Zahl hat also eine infinitesimale Umgebung von 
nichtstandard reellen Zahlen, die näher an $x$ sind als jede beliebige
andere standard reelle Zahl.







