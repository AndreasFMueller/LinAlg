%
% vektorraum.tex
%
% (c) 2017 Prof Dr Andreas Müller, Hochschule Rapperswil
%
\chapter{Vektorraum}
Wenn angehende Ingenieure zum ersten Mal mit linearer Algebra in
Kontakt kommen, arbeiten sie meistens mit Spaltenvektoren von
reellen Zahlen.
Diese Vektoren können mit reellen Zahlen multipliziert werden,
Vektoren können addiert und subtrahiert werden.
Später kommen Matrizen hinzu, welche oft als eine rein formale
Erweiterung der Vektoren betrachtet werden können, als ``dicke''
Vektoren, für die es eine zusätzliche Verknüpfung mit Vektoren
gibt.
Entsprechend liegt das Schwergewicht oft auf Anwendungen in der
Vektorgeometrie.

Dabei gerät oft in Vergessenheit, dass auch ein viel grösseres
Anwendungsgebiet zum Beispiel in der Analysis oder der Kryptographie
haben.
Auch Funktionen können addiert, subtrahiert und mit reellen Zahlen
multipliziert werden.
Auch für Funktionen kann man ein Skalarprodukt definieren, und
damit die geometrisch Idee der orthonormierten Basis und der
einfachen Zerlegung bezüglich einer Basis für Funktionen nutzen,
dies ist die geometrische Betrachtungsweise der Fourier-Theorie.

Damit man die Erkenntnisse der linearen Algebra in dieser Form
nutzen kann, muss man sich erst von der Fixierung auf Spaltenvektoren
lösen.
Man muss also die grundlegende Theorie zuerst so abstrakt formulieren,
dass sie tatsächlich auf die genannten Situationen angewendet werden 
kann.


\section{Skalare -- Körper}

\section{Vektoren -- Vektorraum}

\section{Lineare Abbildungen}

\section{Basis}

\section{Anwendung: Funktionenräume}



