%
% anwendungen.tex -- Anwendungen
%
% (c) 2017 Prof Dr Andreas Müller, Hochschule Rapperswil
%
\section{Anwendungen%
\label{section:tensor:anwendungen}}
\rhead{Anwendungen}
In diesem Abschnitt sollen ein paar Anwendungen des allgemeinen
Tensorkalküls gezeigt werden, der in den vorangegangenen Abschnitten
entwickelt worden ist.

\subsection{Tangentialvektoren einer Mannigfaltigkeit%
\label{tensor:subsetion:tangentialvektoren}}
Wir betrachten eine im dreidimensionalen Raum eingebettete Fläche,
die wir mit Hilfe eine Abbildung
\[
f\colon \mathbb R^2 \to \mathbb R^3: (u,v) \mapsto f(u,v)
\]
beschreiben können.
Die Abbildung gibt jedem Punkt auf der Fläche zugehörige Koordinaten $(u,v)$.
Die Ableitung nach den Koordinaten $u$ und $v$ liefert zwei linear
Tangentialvektoren im dreidimensionalen Raum:
\[
f_u = \frac{\partial f}{\partial u}
\qquad\text{und}\qquad
f_v = \frac{\partial f}{\partial v}.
\]
Diese Betrachtungsweise der Tangentialvektoren ist aber nicht
verallgemeinerungsfähig auf die Untersuchung beleibiger Räume
unabhängig von deren Einbettung in einen grösseren Raum, wie
man sie für die Beschreibung des Universums in der allgemeinen
Relativitätsttheorie braucht.

Ein Vektor $v$ im dreidimensionalen Raum wirkt an einem Punkt $p$
auf Funktionen $g$ von drei Variablen via der Richtungsableitung
\[
X_v\cdot g
=
\frac{d}{dt}g(p+vt)\bigg|_{t=0}.
\]
In der mehrdimensionalen Analysis lernt man, dass man die Richtungsableitung
auch mit dem Gradienten beschreiben kann, es gilt
\[
X_v\cdot g
=
\operatorname{grad}g\cdot v.
\]
Man liest ab, dass der Gradient eine Linearform ist, die jedem
Tangentialvektor $v$ eine Zahl zuordnet.

Übertragen wir dies auf das Beispiel der zweidimensionalen Fläche im
dreidimensionalen Raum.
Zu Koordinate $u$ gehört der Tangentialvektor $f_u$, wir wollen die 
zugehörige Richtungsableitung einer Funktion $g$ im Punkt $f(u_0,v_0)$
auf der Fläche berechnen.
Dazu brauchen wir die Ableitung
\[
X_{f_u}\cdot g(u_0,v_0)
=
\frac{\partial}{\partial u}g(f(u,v_0)) \bigg|_{u=u_0}
=
\frac{\partial}{\partial u}(g\circ f)(u_0,v_0).
\]
Dem Vektor $f_u$ entspricht also der Differentialoperator
\[
\partial_u
=
\frac{\partial}{\partial u},
\]
wir können diese Operatoren direkt als Basis der (verallgemeinerten) 
Tangentialvektoren betrachten.

Der Vorteil dieser Betrachtungsweise ist, dass sie erlaubt, sich
der Vorstellung der Einbettung vollständig zu lösen.
Für einen $n$-dimensionalen Raum mit Koordinatensystem mit den Koordinaten
$(x^1,\dots,x^n)$ betrachten wir die Ableitungsoperatoren 
\[
X_i = \frac{\partial}{\partial x^i}
\]
als die Tangentialvektoren.
Ein beliebiger Tangentialvektor ist Linearkombination der $X_i$, also
von der Form $X = \xi^i X_i$.
Die Richtungsableitung des Vektors $X$ auf einer Funktion $g$ ist
\[
X\cdot g(x^1,\dots,x^n)
=
\xi^i \frac{\partial g}{\partial x^i}.
\]

Diese Betrachtungsweise der Tangentialvektors ist auch mit
Koordinatenwechseln verträglich.
Ist $\tilde x_i$ ein anderes Koordinatensystem, gibt es Umrechnungsfunktionen
\[
x^i
=
x^i(\tilde x^1,\dots,\tilde x^n)
\qquad
\text{und}
\qquad
\tilde x^i
=
\tilde x^i(x^1,\dots,x^n).
\]
Die Umrechnung des Differentialoperators $\partial/\partial x^i$ in 
$\partial/\partial\tilde x^i$ ist durch die Jacobi-Matrix 
\[
\frac{\partial g}{\partial x^i}
=
\frac{\partial x^j}{\partial x^i}
\frac{\partial g}{\partial\tilde x^j}
\qquad\Rightarrow\qquad
X_i
=
\frac{\partial}{\partial x^i}
=
\frac{\partial\tilde x^j}{\partial x^i}
\frac{\partial}{\partial\tilde x^j}
=
\frac{\partial\tilde x^j}{\partial x^i} \tilde X_j.
\]
gegeben.

\subsection{Metrik auf einer Mannigfaltigkeit}
In einem $n$-dimensionalen Raum sei ein Kurve durch die Parametrisierung
\[
t
\mapsto
x^i(t)
\]
gegeben.
Der Tangentialvektor an der Stelle $t_0$ ist
\begin{equation}
X(t_0)
=
\dot x^i(t_0)
\frac{\partial}{\partial x^i}.
\label{tensor:kurve:tangentialvektor}
\end{equation}

Um die Länge der Kurve zwischen den Parameterwerten $a$ und $b$ zu bestimmen,
muss die Summe
\[
l = \sum_{k=0}^{N-} d(x^i(t_k),x^i(t_{k+1}))
\]
für beliebig feine Unterteilung $a=t_0<t_1<\dots <t_{N-1}<t_N=b$ des Intervals,
berechnet werden.
Darin ist $d(x^i(t_k), x^i(t_{k+1}))$ die Entfernung zwischen den Punkten
$x^i(t_k)$ und $x^i(t_{k+1})$.
Für kleine Unterschiede zwischen $t_k$ und $t_{k+1}$ ist die Distanz
näherungsweise durch die Ableitungen gegeben:
\[
d(x^i(t_k), x^i(t_{k+1}))
\simeq
\sqrt{g_{ij}(x^s(t_k))\,\dot x^i(t_k) \dot x^j(t_k)}.
\]
Die Summe
\eqref{tensor:kurve:tangentialvektor}
für die Kurvenlänge $l$ wir damit zu einem Integral
\begin{equation}
l=\int_a^b \sqrt{g_{ij}(x^s(t))\, \dot x^i(t) \dot x^j(t)}\,dt.
\label{tensor:kurve:kurvenlaenge}
\end{equation}
Die Kurvenlänge wird daher durch einen kovarianten Tensor $g_{ij}$ 
zweiter Stufe beschrieben, den sogenannten metrischen Tensor.

\subsection{Kovariante Ableitung, Geodäten und Krümmung}

\subsection{Differentialformen}
Die partiellen Ableitungsoperatoren haben wir in
Abschnitt~\ref{tensor:subsetion:tangentialvektoren}
als eine einbettungsunabhängige Beschreibung der Tangentialvektoren
erkannt.
Die Tangentialvektoren bilden einen $n$-dimensionalen Vektorraum
mit Basisvektoren $\partial/\partial x^i$.

\subsubsection{$1$-Formen}
Die Linearformen auf den Tangentialvektoren bilden ebenfalls einen
$n$-dimensionalen Vektorraum, den zu $\partial/\partial x^i$ duale Basisvektor
bezeichnen wir mit $dx^i$, es gilt also
\[
dx^i \cdot X_j = dx^i \cdot \frac{\partial}{\partial x^j}=\delta^i_j
\]

Die Notation $dx^i$ ist einigermassen willkürlich, lässt sich aber
wie folgt rechtfertigen.
Wir berechnen die Änderung der Koordinate $x^i$ entlang 
der Koordinatenlinie $t\mapsto (x^1,\dots,t,\dots x^n)$, wobei $t$
als $i$-te Koordinate verwendet wird, wenn $t$ von $a$ auf $b$ anwächst.
Entlang der Kurve ändert sich $x^i$ zwischen den zwei Punkten
$x^i(t_k)$ und $x^i(t_{k+1})$ näherungsweise um 
\[
\Delta x^i
=
\dot x^i(t) \cdot (t_{k+1}-t_k)
=
dx^i \cdot \dot x^j\, X_j.
\]
Wir können auch schreiben
\[
b-a
=
\int_a^b dx^i
\]
schreiben, in diesem Sinne ist $dx^i$ eine sinnvolle Bezeichnung für die
zum Vektor $\partial/\partial x^i$ duale Linearform.

\subsubsection{Integration einer $1$-Form}
Die Integration entlang einer Kurve hängt jetzt nicht mehr von der
Parametrisierung der Kurve ab.
Ist $\omega = \omega_i\,dx^i$ eine $1$-Form und $x^i(t)$ eine
Parametrisierung, dann ist das Integral der $1$Form $\omega$
entlang der Kurve
\[
\int_{\gamma} \omega_i dx^i
=
\int_a^b \omega_i(x^i(t))\dot x^i(t)\,dt.
\]
Eine andere Parametrisierung $t=t(s)$ liefert das Integral
\[
\int_a^b \omega^i(x^i(t)) \dot x^i(t)\,dt
=
\int_a^b \omega^i(x^i(s(t))) \frac{x^i(t)}{dt} \frac{dt}{ds}\,ds
=
\int_c^d \omega^i \frac{dx^i(s)}{ds}\,ds.
\]
Das Integral
\[
\int_\gamma \omega
\]
kann daher auf parametrisierungsunabhängige Art definiert werden.

\subsubsection{Metrik}
Der metrische Tensor, der in~\eqref{tensor:kurve:kurvenlaenge}
für die Berechnung der Kurvenlänge verwendet wird, kann jetzt als
Tensorprodukt geschrieben werden:
\[
l = \int_\gamma \sqrt{g_{ij}dx^i\otimes dx^j}.
\]

\subsubsection{$2$-Formen}
Wir betrachten jetzt eine Fläche $S$ mit Parametern $u^1$ und $u^2$, die
durch die Abbildungen $x^i(u^1,u^2)$ beschrieben wird.
Ausserdem sei eine bilinear-Form $a_{ij}dx^i\otimes dx^j$ gegeben.
Wir möchten wieder ein Integral von $\omega$ über die durch die
Parametrisierung beschrieben Fläche definieren und versuchen
\begin{equation}
\int_S \omega
=
\int_S a_{ij}dx^i\otimes dx^j
=
\int a_{ij}
\frac{\partial x^i}{\partial u^1}
\frac{\partial x^j}{\partial u^2}
\,
du^1\,du^2.
\label{tensor:2form}
\end{equation}
Bei einem Koordinatenwechsel $u^i(v^1, v^2)$ ändert der
Integrand in \eqref{tensor:2form}, von den Koordinaten unabhängig kann
das Integral nur sein, wenn sich das Integral wie ein Flächenintegral 
transformiert.
Dazu muss bei der Koordinatenumrechnung das Flächenelement mit der 
Determinante 
\begin{align*}
a_{ij}
\frac{\partial x^i}{\partial u^1}
\frac{\partial x^j}{\partial u^2}
du^1\,du^2
&=
a_{ij}
\frac{\partial x^i}{\partial u^1}
\frac{\partial x^j}{\partial u^2}
\biggl(
\frac{\partial u^1}{\partial v^1}
\frac{\partial u^2}{\partial v^2}
-
\frac{\partial u^1}{\partial v^2}
\frac{\partial u^2}{\partial v^1}
\biggr)
dv^1\,dv^2
\\
&=
\frac12
(
a_{ij}
-
a_{ji}
)
\frac{\partial x^i}{\partial u^k}
\frac{\partial x^j}{\partial u^l}
\biggl(
\frac{\partial u^k}{\partial v^1}
\frac{\partial u^l}{\partial v^2}
-
\frac{\partial u^k}{\partial v^2}
\frac{\partial u^l}{\partial v^1}
\biggr)
dv^1\,dv^2
\end{align*}
multipliziert werden.
Daraus folgt, dass das Integral nur definiert werden kann, wenn
\[
a_{ij}=-a_{ji},
\]
also nur dann, wenn $a_{ij}$ antisymmetrisch ist.
Eine antisymmetrische Bilinearform heisst daher auch einen $2$-Form.

Eine Basis von Zweiformen ist durch die Bilinearformen
\[
dx^i\wedge dx^j = \frac12 (dx^i\otimes dx^j - dx^j \otimes dx^i)
\]
gegeben.

\subsubsection{$p$-Formen}


\subsection{Äussere Ableitung und Maxwell-Gleichungen}

\subsection{Symbol eines Differentialoperators als Tensor}

