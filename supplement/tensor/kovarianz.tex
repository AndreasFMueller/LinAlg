%
% kovarianz.tex -- Kovarianz
%
% (c) 2017 Prof Dr Andreas Müller, Hochschule Rapperswil
%
\section{Kovarianz und Kontravarianz}
\rhead{Kovarianz und Kontravarianz}
Bisher sind wir davon ausgegangen, dass die Stellung eines Index die
einzige wesentliche Eigenschaft ist wenn es darum geht zu beschreiben,
was für ein Objekt $u^i$ oder $v_j$ ist.
Dies ist jedoch nicht ganz zutreffend.
Die Zahlen $u^i$ und $v_j$ beschreiben einen Vektor, doch dafür ist
die Wahl einer Basis Voraussetzung.
Dies bedeutet auch, dass die Wahl einer anderen Basis zu anderen
Koeffizienten $u^i$ oder $v_j$ führt, die immer noch den gleichen
Vektor beschreibt.
Wir müssen daher bei jedem Tensor auch beschreiben, wie die Komponenten
sich ändern, wenn man eine andere Basis wählt.

\subsection{Basistransformation}
Wir betrachten jetzt zwei verschiedenen Basen
\[
B=\{b_1,\dots,b_n\}
\qquad\text{und}\qquad
B'=\{b'_1,\dots,b_n'\}.
\]
eins Vektorraumes $V$.
Da beide Mengen $B$ und $B'$ jeden beliebigen Vektor $v\in V$ auszudrücken
gestatten, gibt es Zahlen $v^i$ und $v^{\prime i}$, mit denen sich der Vektor
$v$ beschreiben lässt:
\begin{equation}
v = v^ib_i = v^{\prime i}b'_i,
\label{tensor:basistransformation1}
\end{equation}
wir stellen uns die Frage, wie $v^{\prime i}$ aus $v^i$ berechnet
werden kann.

Da die Vektoren $b_i\in V$ sind, können wir auch diese Vektoren durch
die Vektoren $b'_i$ beschreiben.
Es muss Zahlen $t^k_i$ geben, so dass
\begin{equation}
b_i = t_i^k b'_k
\label{tensor:basistransformation2}
\end{equation}
gilt.
Daraus können wir aber auch ablesen, wie die Koeffizienten $v^{\prime i}$
zu berechnen sind.
Setzen wir \eqref{tensor:basistransformation2} in
\eqref{tensor:basistransformation1} ein, erhalten wir
\begin{equation}
v = v^i b_i = v^i t_i^kb'_k.
\end{equation}
Durch Vergleich mit dem zweiten Teil von \eqref{tensor:basistransformation1}
erkennen wir, dass 
\begin{equation}
v^i t_i^k
=
t_i^k v^i
=
v^{\prime k}.
\label{tensor:basistransformation3}
\end{equation}
Die Matrix $t_i^k$ beschreibt also auch die Transformation der Koordinaten
$v^{\prime k}$.

Die inverse Matrix $\bar t_j^i$ erfüllt die Bedingung
$\bar t_j^i t_i^k=\delta_i^k$.
Multiplizieren wir \eqref{tensor:basistransformation1} mit $\bar t_j^i$,
erhalten wir
\begin{equation}
\bar t_j^i b_i
=
\bar t_j^i t_i^k b'_k
=
\delta_j^k b'_k
=
b'_j,
\end{equation}
die inverse Matrix vermittelt also die Transformation in der umgekehrten
Richtung.
Wendet man die inverse Matrix auf \eqref{tensor:basistransformation3} an,
erhält man eine entsprechende Formel für die Umrechnung 
\[
\bar t_k^j
t_i^k v^i
=
\bar t_k^j
v^{\prime k}
=
v^j.
\]
Die Formel \eqref{tensor:basistransformation3} liefert als in jedem Fall
die Transformation zwischen Komponenten eines Vektors ausgedrückt
in verschiedenen Basen.

\subsection{Kovarianz und Kontravarianz}
Die Formel \eqref{tensor:basistransformation3} beschreibt die Transformation
zwischen kontravarianten Vektorkomponenten, es fehlt uns eine entsprechende
Formel für kovariante Komponenten.
Wir verwenden dazu, dass ein kovarianter Index den Zweck hat, mit einem
kontravarianten Index zusammen summiert zu werden.
Ist $u_i$ ein kovarianter Vektor und $v^i$ ein kontravarianter Vektor,
dann ist $u_iv^i$ eine Zahl.
Wenn $u_iv^i$ ein Objekt beschreiben soll, welches von der Wahl der
Basis unabhängig ist, dann darf es sich nicht ändern, wenn wir zu einer
anderen Basis übergehen.
Wir fordern daher, dass 
\[
\begin{aligned}
u_iv^i
=
u_i \bar t^i_k v^{\prime k}
=
u'_k v^{\prime k}
\qquad\Rightarrow\qquad
u'_k
&=
u_i \bar t^i_k
\\
t^k_i
u'_k
&=
u_i
\end{aligned}
\]
gilt.
Dies ist eine Transformationsformel für die kovarianten Komponenten
$u_i$ eines Vektors.
Stellt man die Transformationsformeln für $v^i$ und $u_i$ gegenüber
\[
v^{\prime k}
=
t_i^k v^i
\qquad\text{und}\qquad
u_i
=
t^k_i
u'_k,
\]
erkennt man, dass die Transformation für die kovarianten Komponenten
vom gestrichenen Koordinatensystem zum ungestrichenen die gleiche Matrix
verwendet wie \eqref{tensor:basistransformation1}, wenn die 
ungestrichenen Basisvektoren durch gestrichene ausgedrückt werden sollen.
Die Transformation der kontravarianten Komponten dagegen verwendet ebenfalls
die gleiche Matrix aber für die Transformation von ungestrichenen
zu gestrichenen Koordinaten, also in der umgekehrten Richtung.
Dies erklärt auch die Namen kovariant und kontravariant.

Wie soll ein Tensor höherer Stufe transformiert werden?
Die einfachsten Tensoren höherer Stufe sind Produkte von
ko- oder kontravarianten Tensoren erster Stufe.
Der Tensor $a^{ij}=u^iv^j$ hat offensichtlich das Transformationsgesetz
\[
a^{\prime ij}
=
u^{\prime i}v^{\prime j}
=
t^i_k u^k
t^j_l v^l
=
t^i_k
t^j_l
a^{kl}.
\]
Dir können daraus ablesen, dass jeder kovariante Index
vom ungestrichenen zum gestrichenen Koordinatensystem mit der Matrix
$t^i_k$ transformiert werden muss.
Analog zeigt der Tensor $a_{ij}=u_iv_j$ und die Rechnung
\[
a_{ij}
=
u_iv_j
=
t_i^k u'_k
t_j^l v'_l
=
t_i^k
t_j^l a'_{kl},
\]
dass die Matrix $t_i^k$ für die Transformation jedes kontravarianten 
Index vom gestrichenen zum ungestrichenen verwendet werden muss.

\begin{definition}
Eine $p$-fach kovarianter und $q$-fach kontravarianter Tensor ist
eine Grösse
\[
a_{i_1\dots i_p}^{j_1\dots j_q},
\]
die bei Basiswechsel zur gestrichenen Basis gemäss den äquivalenten Formeln
\begin{align*}
a_{i_1\cdots i_p}^{j_1\cdots j_q}
\,
t_{j_1}^{k_1}
\cdots
t_{j_p}^{k_q}
&=
a_{i_1\cdots i_p}^{\prime l_1\cdots l_q}
\,
t_{l_1}^{k_1}
\cdots
t_{l_q}^{k_q}
\\
a_{i_1\cdots i_p}^{j_1\cdots j_q}
&=
a_{k_1\cdots k_p}^{\prime l_1\cdots l_q}
\,
t_{l_1}^{j_1}\cdots t_{l_q}^{j_q}
\,
\bar t_{i_1}^{k_1}\cdots \bar t_{i_p}^{k_p}
\\
a_{i_1\cdots i_p}^{j_1\cdots j_q}
\,
\bar t_{k_1}^{i_1}\cdots \bar t_{k_p}^{i_p}
\,
t_{j_1}^{l_1}\cdots t_{j_q}^{l_q}
&=
a_{k_1\cdots k_p}^{\prime l_1\cdots l_q}
\end{align*}
transformiert wird.
\end{definition}

Ein den meisten Fällen sind die Transformationsregeln eine automatische
Folge der Definition der Objekte und müssen nicht weiter verifiziert werden.

\subsection{Koordinatenfreie Darstellung}

