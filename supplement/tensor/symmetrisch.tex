%
% symmetrisch.tex
%
% (c) 2017 Prof Dr Andreas Müller, Hochschule Rapperswil
%
\section{Symmetrische und antisymmetrische Tensoren%
\label{tensor:symmetrisch}}

\subsection{Symmetrieeigenschaften}
Sei $V=\mathbb R^n$ ein $n$-dimensionaler Vektorraum mit Basisvektoren $e_i$.
Die Tensorprodukte $e_i\otimes e_j$ bilden eine Basis der Tensoren
zweiter Stufe, Tensoren der Stufe $p$ haben als Basis die Tensorprodukte
\[
e_{i_1}\otimes e_{i_2}\otimes\dots\otimes e_{i_p}.
\]
In vielen Anwendungen spielen diese ganz allgemeine Tensoren nur eine
untergeordnete Rolle.
Sehr oft hat man mit Tensoren zu tun, die symmetrisch oder antisymmetrisch
sind. 
Ein Skalarprodukt hängt symmetrisch von den beiden Vektoren ab, ist
also eine lineare Funktion, die auf $e_i\otimes e_j$ und $e_j\otimes e_i$
die gleichen Werte annimmt.

Das Vektorprodukt zweier dreidimensionaler Vektoren ist eine
bilineare Funktion $\mathbb R^3\times\mathbb R^3\to\mathbb R^3$.
Weil sie bilinear ist, muss man sie auch als lineare Abbildung
\[
\mathbb R^3\otimes \mathbb R^3 \to \mathbb R^3
\]
beschreiben können.
Jede Komponente des Vektorproduktes hängt auf antisymmetrische
Weise von Faktoren ab.
Die Werte auf
\[
e_i\otimes e_j
\qquad\text{und}\qquad
e_j\otimes e_i
\]
müssen entgegengesetztes Vorzeichen haben.
Zur Beschreibung des Vektorproduktes brauchen wir daher nicht die Werte
auf beliebigen Tensorprodukten festzulegen, sondern nur auf ganz
besonderen, antisymmetrischen Tensoren.

\subsection{Antisymmetrische Tensoren}
Sei $V$ ein $n$-dimensionaler Vektorraum mit Basis $e_i$, $i\le i\le n$.
Eine antisymmetrische bilineare Funktion nimmt auf
$e_i\otimes e_j$ und $e_j\otimes e_i$
Werte mit entgegengesetztem Vorzeichen an.
Es reicht daher, die Werte auf
\[
e_i\wedge e_j
=
\frac12(
e_i\otimes e_j
-
e_j\otimes e_i
)
\]
zu bestimmen.
Die Vektoren $e_i\wedge e_j$ mit $i<j$ bilden eine Basis des Vektorraums
$V\wedge V$ der antisymmetrischen Tensoren.

\subsubsection{Vektorprodukt}
Jede Komponente des Vektorprodukt $w=u\times v$ ist eine antisymmetrische
bilineare Funktion der Vektoren $u$ und $v$.
Zum Beispiel ist die erste Komponente 
\[
w^1
=
u^2v^3-u^3v^2
\]

\subsubsection{$p$-Vektoren}
Eine multilineare Abbildung von $p$ Vektoren, die in jedem Paar von
Argumenten antisymmetrisch ist, erfüllt
\[
f(\dots,u,\dots,v,\dots)
=
-f(\dots,v,\dots,u,\dots)
\]
für jedes Paar von beliebigen Argumenten.
$f$ ist eine lineare Funktion, die auf den Tensorprodukten
\[
\dots\otimes e_i\otimes\dots\otimes e_j\otimes\dots
\qquad\text{und}\qquad
\dots\otimes e_j\otimes\dots\otimes e_i\otimes\dots
\]
entgegengesetzte Werte annimmt.
Es genügt, die Funktionswerte auf den vollständig antisymmetrischen
Tensorprodukte 
\[
e_{i_1}\wedge e_{i_2}\wedge\dots\wedge e_{i_p}
=
\frac1{p!}
\sum_{\sigma\in S_p}
e_{i_{\sigma(1)}}\otimes e_{i_{\sigma(2)}}\otimes\dots\otimes e_{i_{\sigma(p)}}
\]
festzulegen.
Diese bilden eine Basis eines Vektorraumes $\bigwedge^p V$.

\subsubsection{Determinante}

\subsection{Symmetrische Tensoren und Skalarprodukt}






