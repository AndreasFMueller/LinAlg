%
% beispiele.tex -- Weitere Beispiele für die Randoperator-Idee
%
% (c) 2017 Prof Dr Andreas Müller, Hochschule Rapperswil
%
\section{Randoperatoren}
Eine von Faradays überraschendsten Beobachtungen in seinen
Studien der Elektrodynamik, die später von Maxwell formalisiert
worden ist, ist die Tatsache, dass die Spannung in einer Drahtschleife
berechnet werden kann, indem man den Fluss des Magnetfelds durch die
von der Drahtschleife berandete Fläche berechnet.
Offenbar gibt es einen Zusammenhang zwischen dem, was auf dem Rand
passiert und dem Fluss durch die Fläche.

\subsection{Integration}
Schon bei der einfachen Integrationsformel mit Hilfe der Stammfunktion
können wir ein Beispiel eines Randoperators kennenlernen.
Sei $f(x)$ eine auf dem Interval $[a,b]$ definierte stetige Funktion,
und sei $F(x)$ eine Stammfunktion von $f(x)$.
Dann gilt nach dem Hauptsatz der Analysis
\begin{equation}
\int_a^b f(x)\,dx = F(b)-F(a).
\label{homologie:hauptsatz}
\end{equation}
Wir führen jetzt eine Notation ein, die etwas offensichtlicher werden
lässt, was diese Formel bedeutet.
Wir schreiben das Interval $I=[a,b]$, der Rand des Intervals 
$\partial I$ besteht nur aus zwei Punkten $a$ und $b$.
Wir müssen aber etwas genauer sein, denn die beiden Punkte haben
ja verschiedene, Bedeutung, $a$ ist der Anfangspunkt, $b$ der Endpunkt
des Intervals.
Um dies anzudeuten markieren wir $a$ mit $\mu(a)=-1$ (dort wo des Interval
``weggeht'') und $b$ mit $\mu(b)=1$.
$\mu$ beschreib das Gewicht, mit dem wir die Punkte $a$ und $b$
berücksichtigen müssen.
Damit können wir die rechte Seite von \eqref{homologie:hauptsatz}
jetzt schreiben als
\[
F(b)-F(a)
=
\sum_{x\in \partial I} F(x)\,\mu(x).
\]
Wir könnten diese noch etwas intuitiver als eine Art Integral über
den Rand von $I$ mit den Gewichten $\mu$ schreiben
\[
F(b)-F(a) = \int_{\partial I} F(x)\,d\mu(x).
\]
Die linke Seite von \eqref{homologie:hauptsatz} ist ein Integral von
$f(x)$ über das ganze Interval, aber es ist $f(x)=F'(x)$, 
so dass die Formel~\eqref{chapter:quotient} zu
\[
\int_I F'(x)\,dx = \int_{\partial I} F(x)\,d\mu(x)
\]
wird.
Auf der linken Seite steht ein Integral über einer Ableitung 
über das ganze Interval, auf der rechten das Integral der
ursprünglichen Funktion über den Rand.
Der Übergang von der Funktion zu ihrer Ableitung kann kompensiert
werden durch den Übergang vom Rand zum Inneren des Intervals.

\subsection{Integralsätze der mehrdimensionalen Analysis}
Der Satz von Gauss besagt, dass der Fluss eines Vektorfeldes $\vec v$
durch eine geschlossene Fläche gleich gross ist wie das Integral
der Divergenz des Vektorfeldes über das innere der Fläche.
Etwas formaler: sei $V$ ein dreidimensionales Volumen und $\partial V$
der Rand mit nach aussen orientierter Normale.
Dann gilt
\begin{equation}
\int_V\operatorname{div} \vec v\,dx
=
\int_{\partial V} \vec v\cdot d\vec n.
\label{homologie:gauss}
\end{equation}
Auf der linken Seite steht ein Integral über ganz $V$, auf der rechten
Seite nur über den Rand.
Dafür steht auf der linken Seite eine Art von Ableitung von $\vec v$,
auf der rechten Seite steht das unveränderte Vektorfeld.

Der Satz von Stokes zeigt ein ähnliches Phänomen.
Ist $S$ ein Flächenstück im dreidimensionalen Raum mit Flächennormale
$\vec n$.
Dann gilt
\[
\oint_{\partial S} \vec v\cdot d\vec s
=
\int_{S} \operatorname{rot}\vec v\cdot d\vec n.
\label{homologie:stokes}
\]
Auf der linken Seite steht das Wegintegral des Vektorveldes $\vec v$
entlang der Randkurve $\partial S$.
Auf der rechten Seite steht der Fluss des Vektorfeldes
$\operatorname{rot}\vec v$ durch die Fläche $S$.
Wieder muss beim Übergang vom Rand zum Inneren der Fläche ein
Differentialoperator (die Rotation) auf das Vektorfeld angewendet
werden.

Diese Beispiele zeigen, dass hier wohl ein grundsätzlicheres Phänomen
am Werk ist.
Tatsächlich gibt es für beliebige Mannigfaltigkeiten eine Theorie
der Differentialformen, welche Vektorfelder und Funktionen verallgemeinern.
Eine Differentialform $\omega$ kann über eine Untermannigfaltigkeit $S$
passender Dimension integriert werden.
Eine Differentialform kann auch abgeleitet werden, die sogenannte
äussere Ableitung von $\omega$ ist $d\omega$. 
Der verallgemeinerte Satz von Stokes besagt, dass das Integral von
$d\omega$ über die Untermannigfaltigkeit $S$ das gleiche gibt wie
das Integral von $\omega$ über den Rand:
\begin{equation}
\int_{\partial S}\omega = \int_{S}d\omega.
\label{homologie:stokesallg}
\end{equation}
Ist $S$ eine zweidimensionale Fläche in einem dreidimensionalen Raum,
dann brauchen wir eine eindimensionale Differentialform $\omega$,
also ein Vektorfeld.
Ihre äussere Ableitung ist exakt die Rotation des Vektorfeldes,
der verallgemeinerte Satz von Stokes \eqref{homologie:stokesallg}
enthält also die bekannte Formel \eqref{homologie:stokes} von Stokes
als Spezialfall.


\subsection{Simpliziale Komplexe}

