%
% algtopo.tex -- Intro in die algebraische Topologie
%
% (c) 2017 Prof Dr Andreas Müller, Hochschule Rapperswil
%
\section{Algebraische Topologie}
\rhead{Algebraische Topologie}
Die in diesem Kapitel gezeigte elementare Homologietheorie ist ein
erster Schritt in das weite Gebiet der algebraischen Topologie.
Diese befasst sich mit den Eigenschaften geometrischer Objekte, die
sich unter Deformationen nicht ändern.
Das Grundprinzip ist, einem geometrischen Objekt ein algebraisches Objekt
zuzuordnen.
Zum Beispiel haben wir gesehen, wie man einem Polyeder oder allgemeiner
einem simplizialen Komplex $X$ eine Reihe von Vektorräumen $H_0(X,\mathbb R)$,
$H_1(X,\mathbb R),\dots$ zuordnet.
An verschiedenen Dimensionen dieser Vektorräume können wir bereits 
Unterschiede erkennen.

Die Dimension des Homologie-Vektorraumes $H_0(X,\mathbb R)$ gibt die Anzahl
der Zusammenhangskomponenten an.
Die Dimension von $H_1(X,\mathbb R)$ zählt die Zyklen, die 




