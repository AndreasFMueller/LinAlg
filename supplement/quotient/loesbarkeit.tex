%
% loesbarkeit.tex -- Loesbarkeit von Gleichungssystemen als Motivation
%
% (c) 2017 Prof Dr Andreas Müller, Hochschule Rapperswil
%
\section{Lösbarkeit von Gleichungssystemen}
\rhead{Lösbarkeit von Gleichungssystemen}
Ein lineares $n\times m$-Gleichungssystem der Form $Ax=b$
ist genau dann lösbar, wenn der Vektor $b$ im Bildraum von $A$ ist,
der definiert ist als
\[
\operatorname{im} A = \{ Ax\;|\; x\in K^n\}.
\]
Wenn $x_1$ und $x_2$ Lösungen der Gleichung sind, dann folgt
\[
Ax_1=b\;\wedge\; Ax_2=b
\quad\Rightarrow\quad
A(x_1-x_2)=0,
\]
oder
\[
x_1-x_2\in\operatorname{ker} A = \{ x\in K^n\;|\; Ax=0\}.
\]
Die Lösung eines Gleichungssystem ist also genau dann lösbar, wenn
$b\in \operatorname{im}A$, sie ist eindeutig, wenn $\operatorname{ker}A=0$.

Die Frage nach der Lösbarkeit von Gleichungssystemen hat uns also die
Mengen $\operatorname{im}A$ und $\operatorname{ker} A$ geführt.
Deren Definition ist aber unabhängig von der Tatsache, dass die
Vektorräume endlichdimensional sind.
In diesem Kapitel sollen die Begriffe Kern und Bild verallgemeinert
werden zu den Begriffen Unterraum und Quotientenraum.



