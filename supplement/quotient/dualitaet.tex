%
% dualitaet.tex -- bildraum und quotientenraum
%
% (c) 2017 Prof Dr Andreas Müller, Hochschule Rapperswil
%
\section{Dualität}
\rhead{Dualität}
Wir betrachten wieder einen endlichdimensionalen Unterraum $U\subset V$
und die kanonische Projektion $\pi\colon V\to V/U$.
Die dualen Abbildungen führt auf die Abbildungen
\[
\begin{tikzcd}
(V/U)^* \ar[r,"\pi^*"] & V^* \ar[r, "i^*"] & U^*.
\end{tikzcd}
\]
Wir wollen zeigen, dass $(V/U)^*$ als Unterraum von $V^*$
betrachtet werden kann, und $U^*$ als der zugehörige Quotientenraum.

Sei zunächst $l$ eine Linearform $V/U$.
Dies bedeutet, dass $l$ eine Linearform auf $V$ ist mit der Eigenschaft,
dass $l(u)=0$ für alle $u\in U$.
Zwei verschiedene Linearformen $l_1$ und $l_2$ auf $V/U$ haben 
beide diese Eigenschaft, sie müssen sich also auf irgend einem
Vektor von $V$ unterschieden, unterscheiden sich daher auch wenn
betrachtet als Linearformen auf $V$.
Damit ist gezeigt, dass $(V/U)^*$ ein Unterraum von $V^*$ ist.

Die Abbildung $i^*\colon V^* \to U^*$ bildet eine Linearform $l$
auf die Linearform $l\circ i$ auf $U$ ab.
Eine Linearform auf $(V/U)^*$ ist eine Linearform, welche auf $U$
verschwindet, also ist $i^*\circ \pi^*$.
Aber es gilt auch umgekehrt: Wenn die Linearformo $l$ durch $i^*$ auf
die Nullform abgebildet wird, dann bedeutet dies genau, dass $l(u)=0$
für alle $u\in U$, d.~h.~die Linearform war eigentlich eine Linearform
auf $V/U$.

Die Dualität macht also aus einer Einbettung $U\hookrightarrow V$ und der
kanonischen Projektion $\pi\colon V\twoheadrightarrow U/V$ eine Einbettung
$(V/U)^* \hookrightarrow V^*$ und eine Projektion $V^* \twoheadrightarrow U^*$.

Für endlichdimensionale Vektorräume folgt
\[
\dim V^* = \dim U^* + \dim (V/U)^*.
\]

