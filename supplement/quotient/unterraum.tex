%
% unterraum.tex -- Konzept unterraum und quotient
%
% (c) 2017 Prof Dr Andreas Müller, Hochschule Rapperswil
%
\section{Unterraum und Quotiententraum}
\rhead{Unterraum und Quotientenraum}
Sei $V$ ein $K$-Vektorraum und $U\subset V$ ein in $V$ enthaltener
Vektorraum.
Man nennt $U$ einen Unterraum von $V$.
Ist $V$ endlichdimensional, dann kann man $V$ mit Spaltenvektoren $K^n$
identifizieren.
Offensichtliche Unterräume sind dann die Vektorräume
\[
V_I
=
\{ v\in K^n\;|\; v_i = 0\;\forall i\in I\},
\]
die man für jede Teilmenge $I$ der ganzen Zahlen zwischen 0 und $n$ 
bilden kann.

\subsection{Unterraum als Bildraum}
Wie kann man einen Unterraum $U$ von $V$ beschreiben?
Eine Basis von $V$ wird im Allgemeinen nicht so beschaffen sein,
dass ein Teil der Basisvektoren eine Basis des Unterraums bildet.
Wenn wir also den Unterraum $U$ mit Hilfe einer Basis beschreiben
wollen, dann müssen wir darin erst eine Basis finden.
Für einen endlichdimensionalen Vektorraum ist dies immer möglich,
wie der folgende Satz zeigt:

\begin{satz}
\label{quotient:basiserweiterung}
Sei $U$ ein Unterraum eines endlichdimensionalen $K$-Vektorraumes $V$
der Dimension $n$,
dann gibt es eine Basis $B=\{b_1,\dots,b_n\}$ von $V$ derart, dass
die ersten $m$ Vektoren $B_m=\{b_1,\dots,b_m\}$ eine Basis des Unterraums
$U$ bilden.
\end{satz}

\begin{proof}[Beweis]
Wir konstruieren die Basis iterativ.
Zunächst konstruieren wir eine Basis von $U$ und erweitern diese dann
zu einer Basis von $V$.
Dabei dürfen wir nicht einfach annehmen, wir hätten eine Basis von $U$,
denn die Voraussetzungen versprechen nicht, dass der Unterraum $U$
endlichdimensional ist.
Dieses Resultat müssen wir im Laufe der Konstruktion gewinnen.

Falls $U=0$ ist, dann gibt es keine linear unabhängigen Vektoren in $U$,
der Unterraum $U$ ist $0$-dimensional.
Wir setzen $m=0$ und $B_m=\emptyset$ und müssen daraus nur noch eine Basis
von $V$ konstruieren.

Falls $U\ne 0$ ist, dann gibt es einen Vektor $b_1\in U\setminus\{0\}$,
wir konstruieren also $B_1=\{b_1\}$.
Falls der von $B_1$ aufgespannte Unterraum von $U$ nicht ganz $U$ ist, gibt
es einen weiteren Vektor $b_2\in U\setminus\langle B_1\rangle$.
Notwendigerweise sind die Vektoren $B_2=\{b_1,b_2\}$ linear unabhängig.
Diese Prozedur wiederholen wir, solange $U\ne \langle B_m\rangle$ ist.
Da die Vektoren in $B_m$ eine linear unabhängig Menge von Vektoren in $V$
sind, muss $m\le n$ sein, der Prozess muss also irgendwann abbrechen,
womit $m=\operatorname{dim}U$ und $B_m$ betimmt sind.

Die Basis $B_m$ muss jetzt noch zu einer Basis von $V$ erweitert werden.
Falls $\langle B_k\rangle\ne V$ gibt es einen weiteren
Vektor $b_{k+1} in V\setminus\langle B_k\rangle$ so, dass 
$B_{k+1}=B_k\cup\{b_{k+1}\}$.
Da $V$ endlichdimensional ist, muss auch dieser Prozess terminieren,
d.~h.~wir werden eine Basis $B_n\subset V$ finden.
\end{proof}

Eine Ebene durch den Nullpunkt ist ein zweidimensionaler Unterraum 
des dreidimensionalen Raumes.
Der Satz~\ref{quotient:basiserweiterung} besagt dann, dass es eine
Basis $B=\{b_1,b_2,b_3\}$ gibt, so dass die Ebene von den Vektoren
$b_1$ und $b_2$ aufgespannt wird,
dass also der Unterraum $U$ als
\[
U=\{s_1b_1+s_2b_2\;|\; s_1,s_2\in K\} \subset V
\]
geschrieben werden kann.

Satz~\ref{quotient:basiserweiterung} zeigt, dass immer eine Basis eines
Unterraums gefunden werden kann.
Betrachten wir $V$ wieder als Raum von $n$-dimensionalen Spaltenvektoren,
dann können wir die Vektoren der Basis $B$ mit einer Matrix beschreiben.
Der Unterraum $U$ ist der Raum aufgespannt von den ersten $m$ Spalten
der Matrix.

Wir können diese Eigenschaft auch mit Hilfe einer linearen Abbildung
$f\colon K^n\to V$
mit dem Diagramm
\[
 \begin{tikzcd} {K^m} \arrow[r,"f"] \arrow[d,hook] & U \arrow[d,hook] \\
  {K^n} \arrow[r,"f"]              & V \\
 \end{tikzcd}
\]
beschreiben.
Die Matrix der Abbildung $K^m \to K^n$ besteht aus den Spalten aus den
Vektoren $e_1,\dots,e_m$.

\subsection{Kern}
Ein Unterraum kann aber auch als Lösungsmenge eines Gleichungssystems
beschrieben.
Eine Ebene im dreidimensionalen Raum kann als Lösungsmenge einer einzigen
linearen Gleichung mit drei Unbekannten beschrieben werden.
Im allgemeinen Fall brauchen wir eine lineare Abbildung $f\colon V\to W$,
wobei ziemlich egal ist, was $W$ für ein Vektorraum ist.
Die Menge
\[
U=\operatorname{ker}f = \{ v\in V\;|\; f(v)=0\}
\]
heisst der {\em Kern} der Abbildung $f$.
Er ist ein Vektorraum, denn es gilt
\begin{align*}
u_1,u_2\in\operatorname{ker} f,\quad \lambda_1,\lambda_2\in K
\quad
&\Rightarrow
\quad
f(u_1)=0\wedge f(u_2)=0
\\
&\Rightarrow
\quad
f(\lambda_1u_1+\lambda_2u_2)
=
\lambda_1 f(u_1) + \lambda_2 f(u_2)
=
0
\\
&\Rightarrow
\quad
\lambda_1 u_1 + \lambda_2 u_2\in \operatorname{ker}f.
\end{align*}

\subsection{Quotientenraum}
Sei jetzt wieder $U\subset V$ ein Unterraum des endlichdimensionalen
Vektorraumes.
Gemäss Satz~\ref{quotient:basiserweiterung} können wir eine Basis so
wählen, dass die ersten $m$ Basisvektoren eine Basis von $U$ sind.
Die nachfolgenden Basisvektoren $b_{m+1},\dots,b_n$ sind linear unabhängig
von $U$.
Wir können also die Vektoren von $V$ aufteilen in solche, die in $U$
sind, und solche, die ``quer'' zu $U$ verlaufen.
Natürlich ist dies keine eindeutige Charakterisierung, denn wenn ein
Vektor $v\in V\setminus U$ ist, dann ist auch $v+u\in V\setminus U$
für jeden beliebigen Vektor $u\in U$.
Die Vektoren ``quer'' zu $U$ also nur bis auf einen Summanden in $U$
bestimmt.
Die folgende Definition entfernt diesen Einfluss

\begin{definition}
Für jeden Vektor $v\in V$ definieren wir 
\[
M(v) = \{ w\in V\;|\; v-w\in U\}.
\]
Die $M(v)$ können addiert und mit skalaren multipliziert werden
\[
\begin{aligned}
M(v_1+v_2)&=M(v_1) + M(v_2),&
M(\lambda v)&=\lambda M(v)
\end{aligned}
\]
und bilden daher einen Vektorraum $V/U$ genannt der Quotientenraum.
Die lineare Abbildung
\[
\pi \colon V \to V/U : v \mapsto M(v)
\]
heisst die {\em kanonische Projektion}.
\end{definition}

Aus der Definition ist klar, dass die kanonische Projektion als
Kern die Vektoren $U$ hat.
Man kann die drei Vektorräume in dem einzigen Diagramm
\[
\begin{tikzcd}
U\ar[r,hook] & V \ar[r,two heads,"\pi"] & V/U
\end{tikzcd}
\]
zusammenfassen.
Unter Verwendung einer Basis wie vorhin finden wir, dass $V/U$ die
Vektoren $M(v_{m+1}),\dots,M(v_n)$ als Basis hat.
Daraus können wir ableiten, dass
\[
\operatorname{dim} V
=
\operatorname{dim} U
+
\operatorname{dim} V/U
=
\operatorname{ker}\pi
+
\operatorname{coker}\pi,
\]
wobei $\operatorname{coker}\pi=V/U$ einfach nur ein anderer Name für
den Quotientenraum ist.

\subsection{Universelle Eigenschaft des Quotienten}
Der Quotientenraum kann auch wie folgt beschrieben werden.
Sei $f\colon V\to W$ eine lineare Abbildung mit der Eigenschaft
dass $f(u)=0$ für $u\in U$.
Unterscheiden sich die Vektoren $v_1$ und $v_2$ nur um einen Vektor
in $U$, dann ist $v_1-v_2\in U$ und daher $f(v_1-v_2)=0$, also
$f(u_1)=f(u_2)$.
Damit können wir eine lineare Abbildung $f^*\colon V/U\to W$
mittels
\[
f^*(M(v)) = f(v)
\]
definieren.
Dies ist wohldefiniert, weil sich verschiedene Vektoren von $M(v)$ nur
um einen Vektor in $U$ unterscheiden.
Ausserdem ist
$f^*$ auf diese Art eindeutig bestimmt derart, das Diagramm
\[
\begin{tikzcd}
U \ar[r,hook] \ar[d,"0"]
		& V \ar[d, "f"] \ar[r,two heads,"\pi"]
				& V/U \ar[dl,dashed,"f^*"]
\\
0 \ar[r,hook]
		& W
\end{tikzcd}
\]
kommutativ ist, d.~h.~zwei verschiedene Pfade durch das Diagramm
bilden Elemente auf die gleichen Bilder ab.

Umgekehrt könnte man den Quotientenraum auch mit Hilfe der folgenden 
universellen Eigenschaft definiert werden.
Der Quotientenraum ist derjenige lineare Vektorraum $Q$ für den
das Diagramm
\[
\begin{tikzcd}
U \ar[r,hook] \ar[d,"0"]
		& V \ar[d, "f"] \ar[r,two heads,"\pi"]
				& Q \ar[dl,dashed,"g"]
\\
0 \ar[r,hook]
		& W
\end{tikzcd}
\]
auf genau eine Art durch die lineare Abbildung $g$ zu einem 
kommutativen Diagramm ergänzt werden kann.

