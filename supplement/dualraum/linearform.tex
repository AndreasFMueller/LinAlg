%
% linearform.tex -- Abschnitt über Linearformen
%
% (c) 2017 Prof Dr Andreas Müller, Hochschule Rapperswil
%

\section{Linearformen}
Sei $V$ ein Vektorraum über $K$, wobei wir zunächst einschränkend
annehmen, dass $V$ endlichdimensional ist.
Für die nachfolgende Diskussion kännen wir uns vorstellen, in $V$ eine Basis
$(b_i)_{1\le i\le n}$ gewählt zu haben, so dass wir $V$ mit $K^n$
identifizieren können.
Ein Vektor in $V$ kann als Spaltenvektor geschrieben werden.
Die Basisvektoren werden dann mit den Standardbasisvektoren $e_i$
identifiziert.

Wir versuchen jetzt die Bedeutung der Zeilenvektoren zu ergründen.
Die Matrizenrechnung lehrt uns, dass wir einen $n$-dimensionalen
Zeilenvektor $l$ mit einem $n$-dimensionalne Spaltenvektor $v\in K^n$
nach der Regel
\[
l\cdot v
=
\begin{pmatrix}
l_1&l_2&\dots&l_n
\end{pmatrix}
\begin{pmatrix}v_1\\v_2\\\vdots\\v_n\end{pmatrix}
=
l_1v_1+l_2v_2+\dots+l_nv_n
\]
multiplizieren können.
Linearformen in einem $n$-dimensionalen Raum können also selbst
als den $n$-dimensionalen Vektorraum Zeilenvektoren der Länge $n$
betrachtet werden.

Dass die Linearformen selbst einen Vektorraum bilden, wussten wir
natürlich schon länger.
In Kapitel~\ref{chapter:vektorraum} haben wir bereits gezeigt,
dass die Menge lineare Abbildungen ein Vektorraum ist.
Da $K$ selbst auch ein eindimensionaler Vektorraum ist, ist
$V^*=\operatorname{Hom}(V,K)$ ebenfalls ein Vektorraum, er heisst
der {\em Dualraum}.
\index{Dualraum}%

Seien jetzt $U,V$ zwei $K$-Vektorräume und $f\colon U\to V$ eine lineare
Abbildung zwischen den beiden Vektorräumen.
Aus $f$ können wir eine lineare Abbildung zwischen den Dulräumen
$U^*$ und $V^*$ wie folgt konstruieren.
Ist $l\in V^*$ eine Linearform, dann ist die Abbildung
\[
f^*(l)\colon U\to K:u\mapsto f^*(l)(u)=l(f(u)) = l\circ f(u)
\]
linear, also eine Linearform in $U^*$.
Die Abbildung $f^*$ ist also eine lineare Abbildung $f^*\colon V^*\to U^*$
zwischen den Dualräumen.

Es wird etwas klarer, was hier vorgeht, wenn wir die Konstruktion
des Dualraum etwas umständlicher als $\operatorname{Dual}(U)=U^*$
schreiben.
Die von $f$ induzierte Abbildung der Dualräume wird dann mit
\[
f^*
=
\operatorname{Dual}(f)
\colon 
\operatorname{Dual}(V)
\to
\operatorname{Dual}(U)
\]
bezeichnet.
In der Terminologie von Kapitel~\ref{chapter:kategorien} ist
$\operatorname{Dual}$ ein kontravarianter Funktor in der Kategorie
der $K$-Vektorräume.

