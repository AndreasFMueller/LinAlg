%
% linearform.tex -- Abschnitt über Linearformen
%
% (c) 2017 Prof Dr Andreas Müller, Hochschule Rapperswil
%

\section{Linearformen}
\rhead{Linearformen}
Der Begriff der Linearform ist offensichtlich nicht auf endlichdimensionale
Vektorräume beschränkt.
Wir wollen in diesem Abschnitt die Beobachtungen, die wir für Zeilen-
und Spaltenvektoren gemacht haben, auf beliebige Vektoren verallgemeinern.

\subsection{Notation}
Sei $V$ ein $K$-Vektorraum.
Der Vektorraum $\operatorname{Hom}(V,K)$, ebenfalls ein $K$-Vektorraum,
besteht aus linearen Abbildungen von $V$ mach $K$.
Normalerweise schreibt man $l(v)$ für den Wert einer Linearform
$l\in\operatorname{Hom}(V,K)$ auf dem Vektor $v\in V$.
Diese Notation passt aber nicht gut zur Matrizenschreibweise aus
Abschnitt~\ref{section:dualraum:zeilenspalten}.
Wir schreiben daher 
\[
\langle l,v\rangle = l(v),
\]
dies ähnelt viel mehr der Produktnotation für Zeilen- und Spaltenvektoren.

\subsection{Der Dualraum}
Den Vektorraum $\operatorname{Hom}(V,K)$ haben wir
als Spezialfall des Vektorraums $\operatorname{Hom}(V,U)$
bereits im
Kapitel~\ref{chapter:vektorraum} kennengelernt.
Der Spezialfall ist aber wichtig genug, dass wir ihm einen eigenen Namen
geben.

\begin{definition}
Der Vektorraum der Linearformen auf einem $K$-Vektorraum $V$ heisst
der Dualraum und wird auch mit
\[
\operatorname{Hom}(V,K) = V^* = \operatorname{Dual}(V)
\]
bezeichnet.
\end{definition}

\index{Dualraum}%

\subsection{Induzierte Lineare Abbildung}
Seien jetzt $U,V$ zwei $K$-Vektorräume und $f\colon U\to V$ eine lineare
Abbildung zwischen den beiden Vektorräumen.
Aus $f$ können wir eine lineare Abbildung zwischen den Dulräumen
$U^*$ und $V^*$ wie folgt konstruieren.
Ist $l\in V^*$ eine Linearform, dann ist die Abbildung
\[
f^*(l)\colon U\to K:u\mapsto f^*(l)(u)=l(f(u)) = l\circ f(u)
\]
linear, also eine Linearform in $U^*$.
Die Abbildung $f^*$ ist also eine lineare Abbildung $f^*\colon V^*\to U^*$
zwischen den Dualräumen.

Es wird etwas klarer, was hier vorgeht, wenn wir die Konstruktion
des Dualraum etwas umständlicher als $\operatorname{Dual}(U)=U^*$
schreiben.
Die von $f$ induzierte Abbildung der Dualräume wird dann mit
\[
f^*
=
\operatorname{Dual}(f)
\colon 
\operatorname{Dual}(V)
\to
\operatorname{Dual}(U)
\]
bezeichnet.
In der Terminologie von Kapitel~\ref{chapter:kategorien} ist
$\operatorname{Dual}$ ein kontravarianter Funktor in der Kategorie
der $K$-Vektorräume.

