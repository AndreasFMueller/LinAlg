%
% skalarprodukt.tex -- Abschnitt über Skalarprodukte
%
% (c) 2017 Prof Dr Andreas Müller, Hochschule Rapperswil
%
\section{Dualität und Skalarprodukt}
\rhead{Dualität und Skalarprodukt}
In der elementaren Vektorgeometrie verwendet man als Skalarprodukt
zweier $n$-dimensionaler Spaltenvektoren $u,v\in K^n$ die Grösse
\[
u\cdot v
=
u^t v
=
u_1v_1+\dots+u_nv_n.
\]
Im Lichte der Betrachtungen von Abschnitt~\ref{dualraum:section:transposition}
erzeugt die Transposition zu jedem Vektor $u\in K^n$ eine Linearform
$u^t\in K^{n*}$.
Jede Linearform ensteht durch Transposition aus einem Vektor.
Das Skalarprodukt in einem endlichdimensionalen Vektorraum kann 
daher verstanden werden als der Ausdruck der Tatsache, dass die
Transposition den Vektorraum mit seinem Dualraum identifizieren kann.

In Abschnitt~\ref{section:dualraum:unendlich} wird gezeigt, dass
eine solche Identifikation bei unendlichdimensionalen Vektorräumen
nicht mehr möglich ist.
Unendlichdimensionale Vektorräume haben einen sehr viel grösseren Dualraum.

Um diesen Unterschied zwischen Skalarprodukt und Auswertung von Linearformen
deutlicher zu machen, lohnt es sich, eine Notation zu verwenden, welche
die Gemeinsamkeiten und Unterschiede deutlicher macht.
Für einen Vektor $u\in V$ und eine Linearform $v\in  V^*$ schreiben
wir für die Auswertung der Linearform $v$ auf $u$
\[
\langle v,u\rangle = v(u).
\]
Für das Skalarprodukt zweier Vektoren $v$ und $u$ in $V$  schreiben wir
\[
(v,u) = u\cdot v.
\]
Damit kann man für $v,u\in K^n$ schreiben
\[
\langle v^t,u\rangle = (v,u),
\]
bzw.~für $v\in K^{n*}$ und $u\in K^n$
\[
\langle v, u\rangle
=
(v^t,u).
\]
In dieser Form wird es in Kapitel~\ref{chapter:hilbertspaces}
möglich sein, den Übergang zwischen Dualraum
und Skalarprodukt auch in einem geeignet angereicherten,
unendlichdimensionalen Vektorraum
zu ermöglichen, selbst wenn die Darstellung der Vektoren als
Spaltenvektoren nicht möglich ist.

