%
% zeilenspalten.tex -- das Verhältnis von Zeilen und Spalten
%
% (c) 2017 Prof Dr Andreas Müller, Hochschule Rapperswil
%
\section{Zeilen- und Spaltenvektoren%
\label{section:dualraum:zeilenspalten}}
\rhead{Zeilen- und Spaltenvektoren}
In der elementaren Theorie arbeitet man meistens ausschliesslich mit
Spaltenvektoren, die Zeilenvektoren erhalten nur eine stiefmütterliche
Behandlung.
In diesem Abschnitt soll die besondere Beziehung zwischen Zeilen- und
Spaltenvektoren genauer beleuchtet werden.
Insbesondere sind wir an einer Verallgemeinerung interessiert, die
für beliebige Vektorräume konstruiert werden kann.

\subsection{Linearformen}
In der elementaren linearen Algebra lernt man als eines der einfachsten
Beispiele das Matrizenprodukt eines Zeilenvektors mit einem Spaltenvektor
kennen.
Dieses Produkt ordnet einem Spaltenvektor $v$ auf lineare Weise einen Skalar zu.
Die Abbildung
\[
l\colon \mathbb R^n\to \mathbb R:
v\mapsto
l(v)
=
\begin{pmatrix}l_1&l_2&\dots&l_n\end{pmatrix}
\begin{pmatrix}v_1\\v_2\\\vdots\\v_n\end{pmatrix}
\]
ist ein lineare Abbildung vom $n$-dimensionalen Vektorraum $V=\mathbb R^n$
in den eindimensionalen Vektorraum $\mathbb R$, also ein Element von
$\operatorname{Hom}(\mathbb R^n, \mathbb R)$.
Die linearen Abbildung $V\to\mathbb R$ oder allgemeiner $V\to K$ für
einen $K$-Vektorraum heissen auch {\em Linearformen}.
\index{Linearform}

\subsection{Dualbasis}
Umgekehrt lässt sich mit Hilfe einer Basis in einem endlichdimensionalen
Vektorraum $V$ jede Linearform auf diese Art schreiben.
Die Schreibweise suggeriert auch, dass der Vektorraum der Linearformen
eine naheliegende Basis hat, ihre Basisvektoren sind die durch die
Zeilenvektoren
\begin{align*}
e_1^*&=\begin{pmatrix}1&0&\dots&0\end{pmatrix}\\
e_2^*&=\begin{pmatrix}0&1&\dots&0\end{pmatrix}\\
&\vdots\\
e_n^*&=\begin{pmatrix}0&0&\dots&1\end{pmatrix}
\end{align*}
beschriebenen Linearformen.
Diese Basis ``passt'' zur Standardbasis des Vektorraums der Spaltenvektoren
in dem Sinne, dass
\[
e_i^* e_j = \delta_{ij}
\]
gilt.

\subsection{Abbildungen}
Ist $A$ eine $n\times m$ Matrix, die eine lineare Abbildung von
$\mathbb R^m\to\mathbb R^n$
beschreibt, dann können wir aus einer Linearform $l$ auf $\mathbb R^n$ eine
neue Linearform $\tilde l$ auf $\mathbb R^m$ konstruieren, indem wir
\[
\tilde l(u) = lAu
\]
setzen.
$Au$ ist ein $m$-dimensionaler Spaltenvektor.
Die Zuordnung $l\mapsto \tilde l$ ist natürlich wieder eine lineare
Abbildung, die zugehörigen Zeilenvektoren werden gemäss der Regel
\[
l\mapsto \tilde l = lA
\]
abgebildet.
Man beachte aber, dass diese Abbildung in ``umgekehrter Richtung''
geht, also 
\[
\mathbb R^m\to\mathbb R^n
\qquad\text{führt auf}\qquad
\operatorname{Hom}(\mathbb R^n,\mathbb R)
\to
\operatorname{Hom}(\mathbb R^m,\mathbb R).
\]
Die Enstehung der lineare Abbildung $\tilde l$ kann man als Zusammensetzung
in dem Diagramm 
\[
\begin{tikzcd}
\mathbb R^m \ar[d,"A"] \ar[r,"\tilde l"]
	& \mathbb R \ar[d,equal]
\\
\mathbb R^n \ar[r,"l"]
	& \mathbb R
\end{tikzcd}
\qquad
\text{ergibt Abbildung der $\operatorname{Hom}$-Vektorräume}
\qquad
\begin{tikzcd}
\tilde l \ar[r,hook]
	&\operatorname{Hom}(\mathbb R^m,\mathbb R)
\\
l \ar[u, maps to] \ar[r,hook]
	&\operatorname{Hom}(\mathbb R^n,\mathbb R) \ar[u,"A^*"]
\end{tikzcd}
\]
visualisieren.

\subsection{Transposition}
Üblicherweise verwendet man eine Basis $B=\{b_1,\dots,b_n\}$
eines Vektorraums $V$ dazu, einem beliebigen Vektor $v\in V$ einen
Spaltenvektor mit Komponenten $v_i$ zuzuordnen derart, dass
\[
v=v_1b_1+\dots+v_nb_n
\]
Wenden wir dieses Vorgehen auf die Standardbasis $\{e_1^*, \dots e_n^*\}$
des Vektorraums $\operatorname{Hom}(\mathbb R^n, \mathbb R)$ an, bekommen
wir für die Linearform $l$ die Darstellung
\[
l = l_1e_1^* + l_2e_2^* + \dots + l_ne_n^*.
\]
In diesem Vorgehen wäre der Linearform $l$ also der Spaltenvektor
\[
\begin{pmatrix}
l_1\\
l_2\\
\vdots\\
l_n
\end{pmatrix}
=
\begin{pmatrix}l_1&l_2&\dots&l_n \end{pmatrix}^t
\]
zuzuordnen.

Sei $A$ wieder eine $m\times n$-Matrix wie oben.
Der Linearform $A^*l$, die wir oben durch Zusammensetzung konstruiert haben, 
wird der Spaltenvektor
\[
(lA)^t = A^tl^t
\]
zugeordnet.
Die Abbildung $A^*$, die auf Linearformen wirkt, wird also durch die
transponierte Matrix wiedergegeben.



