%
% unendlich.tex -- Abschnitt über Dualraum unendlichdimensionaler Vektorräume
%
% (c) 2017 Prof Dr Andreas Müller, Hochschule Rapperswil
%
\section{Dualität und unendlichdimensionale Vektorräume%
\label{section:dualraum:unendlich}}
\rhead{Dualität}
Für endlichdimensionale Vektorräume haben wir herausgefunden, dass
der Dualraum die gleiche Dimension hat, indem wir zu einer Basis die
duale Basis konstruiert haben.
Ein endlichdimensionaler Vektorraum und sein Dualraum haben daher die
gleiche Dimension.

Für unendlichdimensionale Vektorräume ist die Situation komplizierter.
Wir betrachten dazu einen Vektorraum $V$ mit einer Basis $B$, die jedoch
unendlich viele Elemente enthält.
Um die Situation nicht zu kompliziert werden zu lassen, nehmen wir an,
dass die Basis abzählbar ist, dass heisst dass die Basisvektoren 
numeriert werden können.
Wir schreiben sie
\[
B= \{ b_1,b_2,b_e,\dots\}.
\]
Natürlich können wir weiterhin die Linearformen $b_i^*$ konstruieren.
Das Lemma~\ref{dualraum:lemmaunabh} hat nicht vorausgesetzt, dass der
Vektorraum endlichdimensional muss, daher sind die dualen Linearformen
$b_i^*$ sicher linear unabhängig.

Trotzdem ist die Menge $B^*=\{b_i^*\;|\; i=\mathbb N\}$ keine Basis.
Wir können nämlich eine Linearform $l$ konstruieren, die sich nicht als 
Linearkombination der $b_i^*$ schreiben lässt.
Um den Wert von $l$ auf einem Vektor $v$ festzulegen, schreiben wir
zunächst
\[
v=v_0b_0+v_1b_1+v_2b_2+v_3b_3+\dots,
\]
wobei nur endlich viele der Koeffizienten $v_i$ von $0$ verschieden sind.
Als Wert von $l$ auf $v$ legen wir daher fest
\begin{equation}
l(v) = \sum_{i\in\mathbb N} v_i.
\label{dualraum:linf}
\end{equation}
Da nur endlich viele der $v_i$ von Null verschieden sind, ist die Summe
in \eqref{dualraum:linf} wohldefiniert.
In einem beliebigen Körper $K$ gibt es nämlich im Gegensatz zu den
Körpern $\mathbb R$ und $\mathbb C$ kein Konzept des Grenzwertes und
damit auch keine Möglichkeit, eine unendliche Summe zu berechnen.

Wir müssen jetzt noch einsehen, dass sich $l$ nicht als Linearkombination
von Linearformen $b_i^*$ geschreiben werden kann.
Wir zeigen dies dadurch, dass wir die Annahme, $l$ lasse sich als
Linearkombination der $b_i^*$ schreiben, zu einem Widerspruch führen.
Nehmen wir also an, dass 
\begin{equation}
l=l_0b_0^* + l_1b_1^*+l_2b_2^*+\dots+l_nb_n^*.
\label{dualraum:linf2}
\end{equation}
Wir haben eine endliche Summe geschrieben, weil in einem $K$-Vektorraum
unendliche Summen nicht definiert sind.
Eine Linearkombination kann daher immer nur endlich viele Summanden
enhalten.

Wir werten jetzt $l$ auf dem Basisvektor $b_{n+1}$ aus.
Aus der Definition wissen wir, dass $l(b_{n+1})=1$.
Setzen wir $b_{n+1}$ in die Darstellung 
\eqref{dualraum:linf2} von $l$ ein, erhalten wir
\[
1
=
l(b_{n+1})
=
l_0\underbrace{b_0^*(b_{n+1})}_{\displaystyle=0}
+
l_1\underbrace{b_1^*(b_{n+1})}_{\displaystyle=0}
+
l_2\underbrace{b_2^*(b_{n+1})}_{\displaystyle=0}
+\dots+
l_n\underbrace{b_n^*(b_{n+1})}_{\displaystyle=0}
=
0.
\]
Damit haben wir einen Widerspruch erhalten.
Es ist also nicht möglich, die Linearform $l$ als Linearkombination
von $b_i^*$ zu schreiben.

Dieses Beispiel zeigt, dass der Dualraum eines unendlichdimensionalen
Vektorraumes viele weitere Linearformen enthält, die sich nicht
als Linearkombinationen von $b_i^*$ schreiben lassen.
Man kann sogar zeigen, dass die es überabzählbar viele weitere, linear
unabhängige Linearformen gibt.
Ist nämlich $I\subset\mathbb N$ eine beliebige Teilmenge der natürlichen
Zahlen, dann können wir die Linearform $l_I$ definieren, die auf dem
Vektor $v$ den Wert
\[
l_I(v) = \sum_{i\in I}v_i
\]
haben soll.
Da nur endlich viele der Komponenten $v_i$ von $0$ verschieden sind, 
ist die Summe wohldefiniert.
Die oben beschriebene Linearform $l$ ist nichts anderers als $l_{\mathbb N}$.
Für jede unendliche Teilmenge $I$ ist $l_I$ eine Linearform, die nicht
durch die $b_i^*$ dargestellt werden kann.
Man kann sogar zeigen, dass es unter den $l_I$ überabzählbar viele 
linear unabhängige Linearformen gefunden werden können.
Eine Basis des Dualraumes $V^*$ ist daher viel grösser als eine Basis
von $V$.

Man kann diese Beobachtung auch als Indiz darauf betrachten, dass die
der Begriff des Vektorraumes für unendlichdimensionale Anwendungen noch
etwas zu wenig restriktiv ist.
Tatsächlich zeigt es sich, dass durch Hinzufügen eines Skalarproduktes
und von Grenzwerten eine sehr viel geeignetere Struktur entsteht,
der Hilbert-Raum.
Mehr dazu im Kapitel~\ref{chapter:hilbertspaces}.




