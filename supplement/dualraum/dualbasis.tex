%
% dualbasis.tex -- Abschnitt über die Dualbasis
%
% (c) 2017 Prof Dr Andreas Müller, Hochschule Rapperswil
%
\section{Dualbasis}
\rhead{Dualbasis}
Bei endlichdimensionalen Vektorräumen konnten wir mit Hilfe einer Basis
die abstrakte Struktur des Vektorraumes auf die viel einfachere und
übersichtlichere Situation der $n$-dimensionalen Spaltenvektoren 
reduzieren.
Wir sind also bestrebt, aus einer Basis des Vektorraums eine passende
Basis des Dualraumes zu konstruieren, und damit auch den
Dualraum auf einen Raum von Spaltenvektoren zu reduzieren.

Sei daher jetzt $V$ ein $K$-Vektorraum und $B$ eine Basis von $V$.
Wir suchen eine Basis des Dualraumes $V^*$.
Da sich jeder Vektor in $V$ als Linearkombination von Basisvektoren
schreiben lässt, ist eine Linearform in $V^*$ festgelegt durch die Werte
auf den Basisvektoren.
Wir konstruieren daher zu jedem Vektor $b\in B$ die {\em duale Linearform}
mittels der Definition
\begin{equation}
b^*\colon V\to K:b'\mapsto
\begin{cases}
1&\qquad b'=b\\
0&\qquad b'\ne b,\; b'\in B
\end{cases}
\label{dualraum:dualeform}
\end{equation}
Die duale Linearform $b^*$ hat also den Wert $1$ auf dem Basisvektor $b$
und den Wert $0$ auf allen anderen Basisvektoren.
Ist $v=v_1b_1+\dots+v_nb_n$, dann ist
$b_i^*(v)=v_i$, d.~h.~die dualen Lineareformen der Basisvektoren können
dazu verwendet werden, die Komponenten eines Vektors in der Basis $B$
zu berechnen.

Sei jetzt zusätzlich $V$ ein $n$-dimensionaler Vektorraum sein mit
der Basis $B=\{b_1,\dots,b_n\}$.
Vektoren in $V$ können dann als Linearkombinationen 
\[
v=v_1b_1+\dots+v_nb_n
\]
von Basisvektoren schreiben.
Die zu $b_i$ duale Linearform $b_i^*$ hat auf $v$ den Wert
\[
b_i^*(v)
=
v_1\underbrace{b_i^*(b_1)}_{\displaystyle=0}
+
\dots
+
v_i\underbrace{b_i^*(b_i)}_{\displaystyle=1}
+
\dots
+
v_n\underbrace{b_i^*(b_n)}_{\displaystyle=0}
=
v_i
\]
Sei weiter $l$ eine Linearform auf $V$.
Wir wollen $l$ als Linearkombination von dualen Linearformen $b^*$
mit $b\in B$ schreiben.
Wir suchen also Zahlen $l_1,\dots,l_n$ derart, dass
\[
l = l_1b_1^* + \dots + l_nb_n^*.
\]
Durch Einsetzen der Basisvektoren folgt
\[
l(b_i)
=
l_1b_1^*(b_i) + \dots + l_nb_n^*(b_i)
=
l_i,
\]
damit habe wir die $l_i$ bereits gefunden.
Damit ist nachgewiesen, dass die $b_i^*$ den Dualraum $V^*$ erzeugt.

Wir möchten jetzt auch noch zeigen, dass die dualen Linearformen linear
unabhängig sind.
Wir formulieren dies wieder etwas abstrakter und geben einen formalen
Beweis.

\begin{lemma}
\label{dualraum:lemmaunabh}
Sie $V$ ein $K$-Vektorraum mit Basis $B$, dann ist die
Menge der dualen Linearformen $B^* = \{b^*\;|\;b\in B\}$ linear
unabhängig im $K$-Vektorraum $V^*$.
\end{lemma}

\begin{proof}[Beweis]
Wir müssen zeigen, dass eine verschwindende Linearkombination 
\[
l
=
\lambda_1 b_1^* + \dots +\lambda_n b_n^* = 0
\]
nur für $\lambda_1=\dots=\lambda_n=0$ möglich ist.
Dazu wenden wir $l$ auf die Basisvektoren $b_i$ an und erhalten
\[
0
=
l(b_i)
=
\lambda_1b_1^*(b_i) + \dots + \lambda_nb_n^*(b_i)
=
\lambda_i.
\]
Damit haben wir nachgerechnet, dass $\lambda_i=0$ gilt, die einzige
verschwindende Linearkombiantion von Basisformen $b^*\in B^*$ ist
daher die Nullform.
\end{proof}

Damit haben wir insgesamt den folgenden Satz bewiesen.

\begin{satz}
Ist $V$ ein $n$-dimensionaler $K$-Vektorraum mit Basis $B=\{b_1,\dots,b_n\}$,
dann ist $V^*$ ein $n$-dimensionaler $K$-Vektorraum mit 
Basis $B^*=\{b_1^*,\dots,b_n^*\}$.
\end{satz}

Die Basis $B^*$ von $V^*$ heisst die Dualbasis.
In den folgenden Abschnitten betrachten wir nur endlichdimensionale
Vektorräume und ihre Dualräume, die ebenfalls endlichdimensional sind,
und kehren erst in
Abschnitt~\ref{section:dualraum:unendlich} zu den unendlichdimensionalen
und der speziellen Struktur ihrer Dualräume zurück.

