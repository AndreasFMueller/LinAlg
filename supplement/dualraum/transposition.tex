%
% transposition.tex -- Abschnitt über die Transposition
%
% (c) 2017 Prof Dr Andreas Müller, Hochschule Rapperswil
%
\section{Transposition -- das oberflächliche Markenzeichen der Dualität%
\label{dualraum:section:transposition}}
Einen $n$-dimensionalen Vektorraum $V$ können wir mit Hilfe einer
Basis immer sofort in den Vektorraum der $n$-dimensionalen
Spaltenvektoren $K^n$ umwandeln.
Als Basisvektoren können wir die Standardbasisvektoren $e_i$ verwenden,
die genau in Zeile $i$ eine $1$ haben und sonst überall $0$.

In Abschnitt~\ref{section:vektorraum:linabb} wurde gezeigt, wie zu
einer linearen Abbildung eine Matrix gehört.
Im Falle einer Linearform ist der Bildraum eindimensional, sie muss
also als $1\times n$-Matrix dargestellt werden können.
Es stellt sich daher die Frage, welche $1\times n$-Matrizen die
dualen Linearformen $e_i^*$ darstellen.

Nach Definition \eqref{dualraum:dualeform} der dualen Linearform
muss gelten
\[
e_i^* e_j
=
\delta_{ij}
=
\begin{cases}
1&\qquad i=j\\
0&\qquad \text{sonst.}
\end{cases}
\]
Dies ist nur möglich, wenn $e_i^*$ die Matrix
\[
e_i^*
=
\begin{pmatrix}
0&\dots&1&\dots&0
\end{pmatrix}
\]
mit einer $1$ an der $i$-ten Stelle und $0$ sonst.
Man kann also schreiben $e_i^*=e_i^t$.

Wir betrachten jetzt eine lineare Abbildung $f\colon U=K^n\to V=K^m$.
Aus Abschnitt~\ref{section:vektorraum:linabb} ist bekannt, dass
$f$ durch eine $m\times n$-Matrix $A$ beschreiben werden kann:
\[
\begin{pmatrix}
v_1\\\vdots\\v_m
\end{pmatrix}
=
\begin{pmatrix}
a_{11}&\dots &a_{1n}\\
\vdots&\ddots&\vdots\\
a_{m1}&\dots &a_{mn}
\end{pmatrix}
\begin{pmatrix}
u_1\\\vdots\\u_n
\end{pmatrix}.
\]
Aus einer Linearform $l$ in $V^*$ mit der Matrix
\[
\begin{pmatrix}l_1&\dots&l_n\end{pmatrix}
\]
wird durch Anwendung von $f^*$ die Linearform $f^*(l)=l\circ f$, wir
wollen die Matrix von $f^*(l)$ ermitteln.
Dazu genügt es, die Werte von $f^*(l)$ auf einem Vektor $u\in K^n$
zu bestimmen.
Wir berechnen also
\[
f^*(l)(u)
=
l(f(u))
=
\begin{pmatrix}l_1&\dots&l_n\end{pmatrix}
\begin{pmatrix}
a_{11}&\dots &a_{1n}\\
\vdots&\ddots&\vdots\\
a_{m1}&\dots &a_{mn}
\end{pmatrix}
\begin{pmatrix}
u_1\\\vdots\\u_n
\end{pmatrix}
=
\sum_{j=1}^nl_j \sum_{i=1}^m a_{ji}u_i
=
\sum_{i=1}^m
\biggl(
\sum_{j=1}^nl_j a_{ji}
\biggr)
u_i.
\]
Die Linearform $f^*(l)$ hat daher die $1\times m$-Matrix
\begin{equation}
\begin{pmatrix}
\sum_{j=1}^nl_j a_{j1}
&\dots&
\sum_{j=1}^nl_j a_{jm}
\end{pmatrix}
=
lA.
\label{dualraum:transp}
\end{equation}
Die duale Abbildung $f^*$ wird also durch Rechtsmultiplikation der
Zeilenvektoren mit $A$ vermittelt.

Verwendet man die duale Basis dazu, den Dualraum $V^*$ von $V$
mit Spaltenvektoren $K^n$ zu identifizieren dann ist der 
zu $l$ gehörende Spaltenvektor $l^t$.
Die duale Abbildung $f^*\colon V^*\to U^*$ kann mit Hilfe der
Dualbasis ebenfalls als $n\times m$-Matrix beschrieben werden.
Aus \eqref{dualraum:transp} kann man ablesen, dass diese Matrix
$A^t$ sein muss.
Die Transposition ist also nichts anderes als ein Ausdruck des Übergangs
zu einer Darstellung in der dualen Basis.

