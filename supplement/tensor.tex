%
% tensor.tex -- Tensoralgebra (Indexgymnastik)
%
% (c) 2017 Prof Dr Andreas Müller, Hochschule Rapperswil
%
\chapter{Tensoren}
\rhead{Tensoren}
Matrizen und Vektoren sind ein- und zweidimensionale Anordnungen von
Zahlen.
Matrizen operieren linear auf Vektoren und erzeugen neue Vektoren.
Kann dieses Konzept verallgemeinert werden auf Operationen, die
auf mehreren Vektoren oder sogar Matrizen wirken?
Tensoren beantworten diese Frage.


\section{Vektor, Matrix, Tensor}
\subsection{Operationen --- Summationskonvention}
\subsection{Tensoren beliebiger Stufe}
\subsection{Symmetrische und Antisymmetrische Tensoren}
\subsection{Skalarprodukt und Metrik}

\section{Kovarianz und Kontravarianz}
\subsection{Basistransformation}
\subsection{Kovarianz und Kontravarianz}

\section{Anwendungen}
\subsection{Tangentialvektoren einer Mannigfaltigkeit}
\subsection{Metrik auf einer Mannigfaltigkeit}
\subsection{Kovariante Ableitung, Geodäten und Krümmung}
\subsection{Symbol eines Differentialoperators als Tensor}

