%
% basis.tex
%
% (c) 2017 Prof Dr Andreas Müller, Hochschule Rapperswil
%
\section{Basis}
Der $K$-Vektorraum $K[X]$ hat die Eigenschaft, dass jeder Vektor darin,
also jedes Polynom mit Koeffizienten $K$ als Linearkombination einer
kleinen Menge von Vektoren geschrieben werden kann.
Die Monome
\[
\begin{aligned}
p_0(X)&=1,
&
p_1(X)&=X,
&
p_2(X)&=X^2,
&&\dots&
p_n(X)&=X^n,
&&\dots
\end{aligned}
\]
Das Polynom $p(X)=a_0+a_1X+a_2X^2+\dots+a_nX^n$ ist die Linearkombination
\[
p(X)
=
a_0p_0(X)
+
a_1p_1(X)
+
a_2p_2(X)
+\dots
+
a_np_n(X).
\]
Nach dem Prinzip des Koeffizientenvergleichs
ist ausserdem die Darstellung von $p(X)$ als Linearkombination
der Polynome $p_0,p_1,\dots,p_n,\dots$ eindeutig.
Die Menge
\[
B=\{p_0,p_1,\dots,p_n,\dots\}
\]
hat eine spezielle Bedeutung für den Vektorraum $K[X]$, jeder Vektor
von $K[X]$ lässt sich auf genau eine Weise als Linearkombination der
Vektoren aus $B$ darstellen.
Da die Darstellung immer eindeutig ist, sind die Vektoren in $B$
linear unabhängig.
Da sich jeder Vektor als Linearkombination darstellen lässt, ist
$K[X]=\langle B\rangle$.

\begin{definition}
Eine Teilmenge $B\subset V$ von Vektoren eines $K$-Vektorraums heisst
eine {\em Basis} von $V$, wenn jeder Vektor $v\in V$ auf genau eine Art
als Linearkombination
\[
v=\lambda_1b_1+\dots+\lambda_nb_n
\]
von Vektoren $b_1,\dots,b_n\in B$ dargestellt werden kann.
\end{definition}

Man beachte, dass nicht gefordert wurde, dass die Menge $B$ endlich sein
muss.
Es ist aber aus der Definition klar, dass die Vektoren in $B$ linear
unabhängig sind und zusammen den ganzen Vektorraum erzeugen,
also $V=\langle B\rangle$.

\begin{definition}
Ein Vektorraum $V$ heisst {\em endlichdimensional}, wenn er eine Basis $B$
mit endlich vielen Vektoren besteht.
Die {\em Dimension} eines endlichdimensionalen Vektorraums $V$ ist
die Anzahl der Basisvektoren $\operatorname{dim} V = |B|$.
\end{definition}

Diese Definition ist nur sinnvoll, wenn alle möglichen Basen von $V$ die
gleiche Anzahl Basisvektoren haben, dies ist aber nicht unmittelbar klar.
Wir fassen das in die Form eines Lemmas und formulieren einen abstrakten
formalen Beweis.

\begin{lemma}
Sei $V$ ein endlichdimensionaler $K$-Vektorraum, dann hat jede Basis
von $V$ die gleiche Anzahl Basisvektoren.
\end{lemma}

\begin{proof}[Beweis]
Seien $B$ und $\tilde B$ zwei verschiedenen Basen mit Basisvektoren
$b_1,\dots,b_n\in B$ und $\tilde b_1,\dots,\tilde b_m\in \tilde B$,
wobei wir ohne Einschränkung der Allgmeinheit annehmen dürfen, dass $n>m$
ist.
Da sowohl $B$ als auch $\tilde B$ Basen sind, lässt sich jeder Basisvektor
aus den jeweils anderen Basisvektoren linear kombinieren.
Es gibt also eindeutig bestimmte Zahlen $a_{ij}\in K$ und
$\tilde a_{ij}\in K$ mit
\[
\tilde b_j = \sum_{i=1}^n a_{ji} b_i
\qquad\text{und}\qquad
b_i = \sum_{j=1}^m \tilde a_{ij}\tilde b_j.
\]
Wir behaupten, dass die Vektoren $b_1,\dots,b_n$ linear abhängig sein 
müssen, dass also $B$ gar keine Basis sein kann.

Wir behaupten also, dass es Zahlen $\lambda_1,\dots,\lambda_n\in K$
gibt mit der Eigenschaft
\[
\lambda_1 b_1+\dots+\lambda_n b_n = 0,
\]
wir müssen zeigen, wie man die Zahlen $\lambda_i$ finden kann.
Setzt man die Darstellung durch die Vektoren $\tilde b_j$ ein, erhält man
\[
\sum_{i=1}^n
\lambda_i
\sum_{j=1}^m
\tilde a_{ij}\tilde b_j
=
\sum_{j=1}^m
\biggl(
\sum_{i=1}^n
\lambda_i
\tilde a_{ij}
\biggr)
\tilde b_j
=
0.
\]
Da die Vektoren $\tilde b_j$ linear unabhängig sind, muss in dieser
Summe jede Klammer verschwinden:
\[
\sum_{i=1}^n \tilde a_{ij}\lambda_i =0 
\]
Dies ist ein lineares Gleichungssystem mit $m$ Gleichungen ($j=1,\dots,m$)
für $n$ Unbekannte ($i=1,\dots,n$).
Aus der elementaren Theorie wissen wir, dass ein solches Gleichungssystem
eine nichttriviale Lösung haben muss.
Damit ist gezeigt, dass die Vektoren $b\in B$ nicht linear unabhängig sein
können, $B$ kann also gar keine Basis sein.
Folglich müssen zwei Basen eines endlichdimensionalen Vektorraums
die gleiche Anzahl.
\end{proof}

Wir haben diesen formalen Beweis noch aus einem anderen Grund im
Detail durchgeführt.
Mit Hilfe der Basis haben wir das Problem auf eine Frage über ein
Gleichungssystem reduziert.
Dies ist ein allgemeines Prinzip.
Sei $B=\{b_1,\dots,b_n\}$ eine Basis des Vektorraums $V$.
dann lässt sich jeder Vektor $v\in V$  auf eindeutige Art als
Linearkombination von Vektoren von $B$ darstellen, es gibt
also Zahlen $v_i\in K$ mit der Eigenschaft
\[
v=v_1b_1+\dots+v_nb_n.
\]
Zu jedem Vektor $v\in K$ gibt es also einen Spaltenvektor mit Komponenten
$v_i$, die Abbildung
\[
\varphi\colon
V\to K^n
:
v\mapsto \begin{pmatrix}v_1\\\vdots\\v_n\end{pmatrix}
\]
ist eine lineare Abbildung.
Da jeder Vektor als Linearkombination von Vektoren aus $B$ geschrieben
werden kann, ist $\varphi$ surjektiv.
Weil die Darstellung als Linearkombination eindeutig ist, ist $\varphi$
auch injektiv.
$\varphi$ ist also ein bijektive lineare Abbildung $V\to K^n$.
Eine Basis ermöglich also, jeden beliebigen Vektorraum $V$ mit einem
Vektorraum $K^n$ von Spaltenvektoren zu identifizieren.




