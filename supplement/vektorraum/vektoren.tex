%
% vektoren.tex -- Vektoren
%
% (c) 2017 Prof Dr Andreas Müller, Hochschule Rapperswil
%
\section{Vektoren}
Die zweite Zutat zu der Struktur, die wir in diesem Kapitel
konstruieren wollen, sind die Vektoren.
Wir gehen wieder wie bei den Skalaren vor un formulieren nur
die unbedingt nötigen Anforderungen in Form von Axiomen.
Dann zeigen wir neue Beispiele von Vektorräumen, was illustrieren soll,
dass die abstrakte Theorie erlaubt, auf vereinheitlichte Weise über
verschiedene Arten von Vektoren Erkenntnisse zu gewinnen.

\subsection{Axiome eines Vektorraumes}
Die Vektoren bilden eine Menge, in der man Addition und Subraktion 
ausführen kann.
Ausserdem kann man Vektoren mit Skalaren aus einem Körper $K$ multiplizieren.

\begin{definition}
Eine Vektorraum über dem Körper $K$ oder $K$-Vektorraum
ist eine Menge $V$, genannt die Vektoren,
mit einer kommutativen und assoziativen Operation $+$ und einer Multiplikation
von Skalaren mit Vektoren, mit folgenden Eigenschaften
\begin{compactenum}
\item Die Operationen sind assoziativ: $(\lambda\mu)v=\lambda(\mu v)$ für
$\lambda,\mu\in K$ und $v\in V$.
\item Es gibt neutrales Element der Addition $0\in V$, es gilt also
$0+v=v$ für $v\in V$.
\item Jede Gleichung $x+u=v$ mit $u,v\in V$ hat eine eindeutige Lösung
$x=v-u\in V$.
\item Die Operationen sind miteinander verträglich im Sinne der 
Distributivgesetze:
\begin{align*}
(\lambda + \mu)v&=\lambda v + \mu v
\\
\lambda(u+v)&=\lambda u + \lambda v
\end{align*}
für $u,v\in V$ und $\lambda,\mu\in K$.
\end{compactenum}
\end{definition}

\subsection{Beispiele}
Das offensichtliche Beispiel der Zeilen- oder Spalten-Vektoren wollen wir
hier nicht erneut besprechen.

\subsubsection{Polynome}

\subsubsection{Stetige Funktionen}

\subsection{Lineare Abhängigkeit}






