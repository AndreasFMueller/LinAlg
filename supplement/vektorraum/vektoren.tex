%
% vektoren.tex -- Vektoren
%
% (c) 2017 Prof Dr Andreas Müller, Hochschule Rapperswil
%
\section{Vektoren}
Die zweite Zutat zu der Struktur, die wir in diesem Kapitel
konstruieren wollen, sind die Vektoren.
Wir gehen wieder wie bei den Skalaren vor un formulieren nur
die unbedingt nötigen Anforderungen in Form von Axiomen.
Dann zeigen wir neue Beispiele von Vektorräumen, was illustrieren soll,
dass die abstrakte Theorie erlaubt, auf vereinheitlichte Weise über
verschiedene Arten von Vektoren Erkenntnisse zu gewinnen.

\subsection{Axiome eines Vektorraumes}
Die Vektoren bilden eine Menge, in der man Addition und Subraktion 
ausführen kann.
Ausserdem kann man Vektoren mit Skalaren aus einem Körper $K$ multiplizieren.

\begin{definition}
Eine Vektorraum über dem Körper $K$ oder $K$-Vektorraum
ist eine Menge $V$, genannt die Vektoren,
mit einer kommutativen und assoziativen Operation $+$ und einer Multiplikation
von Skalaren mit Vektoren, mit folgenden Eigenschaften
\begin{compactenum}
\item Die Operationen sind assoziativ: $(\lambda\mu)v=\lambda(\mu v)$ für
$\lambda,\mu\in K$ und $v\in V$.
\item Es gibt neutrales Element der Addition $0\in V$, es gilt also
$0+v=v$ für $v\in V$.
\item Jede Gleichung $x+u=v$ mit $u,v\in V$ hat eine eindeutige Lösung
$x=v-u\in V$.
\item Für die Multiplikation gilt $1\cdot v=v$ für alle $v\in V$.
\item Die Operationen sind miteinander verträglich im Sinne der 
Distributivgesetze:
\begin{align*}
(\lambda + \mu)v&=\lambda v + \mu v
\\
\lambda(u+v)&=\lambda u + \lambda v
\end{align*}
für $u,v\in V$ und $\lambda,\mu\in K$.
\end{compactenum}
\end{definition}

\subsection{Beispiele}
Das offensichtliche Beispiel der Zeilen- oder Spalten-Vektoren wollen wir
hier nicht erneut besprechen.

\subsubsection{Polynome}
Ist $K$ ein Körper, dann sei
\[
K[X]
=
\{
a_0+a_1X+\dots a_nX^n\,|\, a_0,\dots,a_n\in K
\}
\]
die Menge aller Polynome mit Koeffizienten in $K$.
$K[X]$ ist ein Vektorraum, denn man kann offenbar Polynome addieren und
mit Skalaren aus $K$ multiplizieren.
Die Addition ist kommutativ und das Polynom $0\in K[X]$ ist ihr neutrales
Element.
Die Vektorraumaxiome sind nichts anders als die üblichen Rechenregeln
für Polynome.

Man bemerkt, dass $K[X]$ sogar eine kommutative Multiplikation hat
mit dem neutralen Element $1$.
$K[X]$ ist also ein Ring und sogar ein Integritätsbereich, es lässt sich
also der Quotientenkörper $K(X)$ der rationalen Funktionen mit Koeffizienten
in $K$ konstruieren.

\subsubsection{Stetige Funktionen}
Die Menge
\[
C([a,b])
=
\{ f\colon [a,b]\to \mathbb R\;|\; \text{$f$ ist stetig}\},
\]
ist ein $\mathbb R$-Vektorraum.
Funktionen können punktweise addiert und mit reellen Skalaren multipliziert
werden, wenn man definiert
\[
\begin{aligned}
(f+g)(x)&=f(x)+g(x)
&&\text{und}
&
(\lambda f)(x)&=\lambda f(x)
\end{aligned}
\]
für $f,g\in C([a,b])$ und $\lambda\in\mathbb R$.

Auch in diesem Fall ist $C([a,b])$ ein Ring mit der konstanten Funktion $1$
als neutrales Element der Multiplikation.
Allerdings ist dieser Ring nicht ein Integritätsbereich, da das Produkt der
beiden stetigen Funktionen
\[
f(x)
=
\begin{cases}
0&\qquad x<\displaystyle\frac{a+b}2\\
\displaystyle x-\frac{a+b}2&\qquad x\ge \displaystyle\frac{a+b}2
\end{cases}
\qquad\text{und}\qquad
g(x)
=
\begin{cases}
\displaystyle\frac{a+b}2-x&\qquad x<\displaystyle\frac{a+b}2\\
0&\qquad x\ge \displaystyle\frac{a+b}2
\end{cases}
\]
verschwindget: $(fg)(x)=f(x)g(x)=0$.
Damit ist die Konstruktion des Quotientenkörpers nicht möglich.
Dies illustriert, dass die algebraische Struktur der Polynome viel
rigider ist als die der stetigen Funktionen.

Die Menge $C([a,b])$ trägt aber noch zusätzliche Struktur, welche
wir erst später untersuchen können.
Es ist möglich, auf $C([a,b])$ eine Norm zu definieren, und damit
Cauchy-Folgen und Grenzwerte zu definieren.
Dies ist möglich auf eine Art, dass Grenzwerte von stetigen Funtionen
existieren und wieder stetige Funktionen sind.
So etwas ist für Polynome nicht möglich, da man jede beliebige stetige
Funktion beliebig genau mit Polynomen approximieren kann.
Wir können den Polynomring daher als einen Teilvektorraunm
$\mathbb R[X]\subset C([a,b])$ betrachten.

\subsection{Lineare Abhängigkeit}
Sind $v_i,b\in V$ Vektoren, dann ist
\begin{equation}
x_1 v_1 + \dots + x_n v_n = b
\label{vektorraum:lingl}
\end{equation}
ein lineares Gleichungssystem für die Zahlen $x_1,\dots,x_n\in K$.
Die linke Seite von \eqref{vektorraum:lingl} heisst eine
{\em Linearkombination} der Vektoren $v_1,\dots,v_n$.
In diesem Moment interessiert uns nicht die Frage, wie man dieses
Gleichungsssystem löst, sondern die Frage, ob das Gleichungssystem
übrerhaupt lösbar ist.

Je nach rechter Seite $b$ könnte es gar keine Lösung der Gleichung
\eqref{vektorraum:lingl} geben.
Dieses Phänomen ist bekannt aus der elementaren Theorie, nur wenn
\[
b\in \{ x_1v_1+\dots+ x_nv_n\,|x_i\in K\}
=
\operatorname{span}(v_1,\dots,v_n)
=
\langle v_1,\dots,v_n\rangle,
\]
gilt, wird das Gleichungssystem eine Lösung haben.
Die Menge auf der rechten Seite ist ein Unterraum von $V$ und
heisst das {\em Erzeugnis} der Vektoren $v_1,\dots,v_n$ oder
\index{Erzeugnis}%
der von $v_1,\dots,v_n$ {\em aufgespannte} Vektorraum.
\index{aufgespannter Vektorraum}%
Die Bedingung ist aber auf jeden Fall erfüllt für $b=0$, 
eine Lösung kann man dann auch unmittelbar angeben:
$(x_1,\dots,x_n)=(0,\dots,0)$.

Es bleibt die Frage, ob die Lösung eindeutig bestimmt ist.
Nehmen wir an, dass es zwei Lösungen $(x_1,\dots,x_n)$ und
$(x_1',\dots,x_n')$ des Gleichungssystems \eqref{vektorraum:lingl}
gibt.
Für beide Lösung gilt die Gleichung:
\begin{align*}
x_1v_1+\dots+x_nv_n&=b\\
x_1'v_1+\dots+x_n'v_n&=b
\end{align*}
Die Differenz ist dann
\begin{equation}
(x_1-x_1')v_1+\dots + (x_n-x_h') v_n = 0
\label{vektorraum:diffgl}
\end{equation}
Die Gleichung
\eqref{vektorraum:lingl} ist also genau dann eindeutig lösbar
wenn die Gleichung \eqref{vektorraum:diffgl} nur mit den Werten
\[
\begin{aligned}
x_1-x_1'&=0,&&\dots,&
x_n-x_n'&=0
\end{aligned}
\]
befriedigt werden kann.
Wir fassen dies zusammen im Begriff der linearen Unabhängigkeit.

\begin{definition}
Die Vektoren $v_1,\dots,v_n$ heissen {\em linear unabhängig}, wenn die
Gleichung
\[
\lambda_1v_1 + \dots + \lambda_nv_n=0
\]
nur die Lösung $\lambda_1=\dots=\lambda_n=0$ hat.
\end{definition}
\index{linear abhängig}

Ein lineares Gleichungssystem der Form \eqref{vektorraum:lingl}
kann als nur dann eindeutig lösbar sein, wenn die Vektoren
$v_1,\dots,v_n$ linear unabhängig sind.
Eine Lösung wird nur existieren, wenn die Vektoren
$v_1,\dots,v_n,b$ linear abhängig sind.


