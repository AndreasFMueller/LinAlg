%
% lineareabbildung.tex
%
% (c) 2017 Prof Dr Andreas Müller, Hochschule Rapperswil
%
\section{Lineare Abbildungen%
\label{section:vektorraum:linabb}}
\rhead{Lineare Abbildungen}
In diesem Abschnitt betrachten wir zwei Vektorräume $U$ und $V$ und
eine Abbildungen $f\colon U\to V$, welche mit der Vektorraumstruktur
verträglich sein soll. 
Dazu muss für zwei Vektoren $u_1,u_2\in U$ und für $\lambda\in K$ gelten
\begin{equation}
f(u_1+u_2)=f(u_1)+f(u_2)
\qquad\text{und}\qquad
f(\lambda u_1)=\lambda f(u_1).
\label{vektorraum:linear}
\end{equation}
Wir können diese Eigenschaft in einer einzigen Formel zusammenfassen:

\begin{definition}
Eine Abbildung $f\colon U\to V$ heisst linear, wenn gilt
\[
f(\lambda_1 u_1+\lambda_2 u_2)=\lambda_1 f(u_1) + \lambda_2 f(u_2)
\]
für alle $u_1,u_2\in U$ und $\lambda_1,\lambda_2\in K$.
\end{definition}

Die Formeln~\eqref{vektorraum:linear} sind Spezialfälle der Definition.
Die Menge der linearen Abbildungen ist selbst wieder ein Vektorraum.

\begin{definition}
Seien $U$ und $V$ Vektorräume über $K$, dann ist
\[
L(U,V)
=
\operatorname{Hom}(U,V)
=
\{f\colon U\to V\;|\; \text{$f$ ist linear}\}
\]
der Vektorraum der linearen Abbildungen von $U$ nach $V$.
\end{definition}

Wir müssen nachprüfen, dass lineare Abbildungen tatsächlich einen
Vektorraum bilden. 
Dazu muss zunächst überprüft werden, ob die Summe linearer Abbildungen
wieder linear ist.
Tatsächlich ist
\begin{align*}
(f_1+f_2)(\lambda_1u_1+\lambda_2 u_2)
&=
f_1(\lambda_1u_1+\lambda_2 u_2)
+
f_2(\lambda_1u_1+\lambda_2 u_2)
\\
&=
\lambda_1f_1(u_1)+\lambda_2f_1(u_2)
+
\lambda_1f_2(u_1)+\lambda_2 f_2(u_2)
\\
&=
\lambda_1(f_1 + f_2)(u_1)+\lambda_2(f_1+f_2)(u_2),
\\
(\lambda f)(\lambda_1 u_1 + \lambda_2 u_2)
&=
\lambda (f(\lambda_1 u_1 + \lambda_2 u_2))
\\
&=
\lambda\lambda_1 f(u_1)
+
\lambda\lambda_2 f(u_2)
\\
&=
\lambda_1 (\lambda f)(u_1)
+
\lambda_2 (\lambda f)(u_2)
\end{align*}
also sind $f_1+f_2$ und $\lambda f$ wieder lineare Abbildungen.
Neutrales Element ist die lineare Abbildung $0:\colon U\to V:v\mapsto 0$,
die jeden Vektor auf den Nullvektor $0$ abbildet.
Ausserdem gelten natürlich die üblichen Rechenregeln, die sich als
Distributivgesetz äussern.


