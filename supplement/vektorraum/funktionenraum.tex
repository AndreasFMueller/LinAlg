%
% funktionenraum.tex -- Funktionenraum als Beispiel 
%
% (c) 2017 Prof Dr Andreas Müller, Hochschule Rapperswil
%
\section{Funktionenräume}
Funktionenräume sind die einfachsten Beispiele reichhaltiger
unendlichdimensionaler Vektorräume.
In diesem Abschnitt soll daher ein paar Beispiele von solchen
Funktionenräumen etwas genauer unter die Lupe genommen werden.

\subsection{Der Vektorraum der stetigen Funktionen}
Für ein Interval $I=[a,b]$ betrachten wir die Menge
\[
C(I) = \{f\colon I\to\mathbb R\;|\; \text{$f$ ist stetig}\}
\]
der stetigen Funktionen.

\subsubsection{Linearformen auf $C(I)$}
Für jeden Punkt $x\in I$ kann man die Abbildung
\[
\varepsilon_x\colon C(I)\to\mathbb R: f\mapsto f(x)
\]
definieren.
Offenbar ist $\varepsilon_x$ linear, denn es gilt
\[
\varepsilon(f+g) = (f+g)(x) = f(x) + g(x) = \varepsilon_x f + \varepsilon_x g
\qquad\text{und}\qquad
\varepsilon_x (\lambda f) = (\lambda f)(x)= \lambda f(x)
=\lambda\varepsilon_x(f).
\]
Eine weitere Möglichkeit, eine Linearform zu definieren, ergibt sich mit
Hilfe einer stetigen Funktion $g$.
Dazu setzen wir
\[
I_g(f) =  \int_I g(x)f(x)\,dx.
\]
Da auf einem abgeschlossenen und beschränkten Interval stetige Funktionen
beschränkt sein müssen, ist dieses Integral immer definiert.
Die Abbildung ist tatsächlich linear, denn die Rechenregeln für das
Integral besagen
\begin{align*}
I_g(f+h)
&=
\int_Ig(x)(f(x)+h(x))\,dx
=
\int_Ig(x)f(x)\,dx \int_I g(x)h(x))\,dx
=
I_g(f) + I_g(h)
\\
I_g(\lambda f)
&=
\int_Ig(x)\lambda f(x)\,dx
=
\lambda\int_Ig(x)f(x)\,dx
=
\lambda I_g(f),
\end{align*}
was gleichbedeutend mit Linearität ist.

Es stellt sich heraus, dass die beiden vorgestellten Arten von Linearformen
nicht disjunkt sind.
Die Linearform $\varepsilon_x$ kann nämlich nicht in der Form $I_g$ dargestellt
werden.
Dazu würde eine Funktion $\delta_x$ benötigt mit der Eigenschaft
\[
I_{\delta_x}(f)= \int_I\delta_x(\xi)f(\xi)\,d\xi = f(x),
\]
also die Dirac-$\delta$-``Funktion''.
Keine stetige Funktion hat diese Eigenschaft, also ist $\varepsilon_x$
nicht in der Form $I_g$ darstellbar.

\subsubsection{Integraloperatoren}
Ist $K(x,y)$ eine stetige Funktion auf $I\times I$, dann definiert $K$
eine lineare Abbildung $C(I)\to C(I)$ durch die Formel
\[
(Kf)(x) = \int_I K(x,\xi)f(\xi)\,d\xi.
\]
Man nennt $K$ einen Integral-Operator mit Kernfunktion $K(x,y)$.
Die Greensche Funktion, mit welcher elliptische Randwertprobleme
gelöst werden können, ist ein Beispiel eines solchen Integraloperators.

\subsection{Integrierbare Funktionen}

\subsection{Differenzierbare Funktionen}
Sei $I$ ein Interval und 
\[
C^k(I)
=
\{
f\colon I\to\mathbb B\;|\;
\text{$f$ ist $k$-mal stetig differenzierbar}
\}
\]
die Menge der $k$-fach stetig differenzierbaren Funktionen.
Die bekannten Rechenregeln für die Ableitung besagen, dass
dass $C^k(I)$ ein Vektorraum ist.

\subsubsection{Ableitungen}
Für die Ableitung von Funktionen einer Variablen gelten die
Rechenregeln
\[
(f+g)'=f'+g'
\qquad\text{und}\qquad
(\lambda f)' = \lambda f',
\]
daraus schliessen wir, dass die Ableitung eine lineare Abbildung
\[
\frac{d}{dx} \colon C^k(I)\to C^{k-1}(I)
\]
ist.
Da $f'$ eventuell nur noch $(k-1)$-mal stetig differenzierbar ist, wenn
$f\in C^k(I)$ war, ist klar, dass die Ableitung nicht in den gleichen
Vektorraum zurückführt.
Dasselbe gilt für Ableitungen beliebiger Ordnung.

Die verschiedenen Vektorräume $C^k(I)$ sind eineinander
geschachtelt:
\[
C(I) \supset C^1(I) \supset C^2(I)\supset \dots \supset
C^k(I) \supset \dots \supset C^\infty(I)
\]
Darin ist $C^\infty(I)$ der Vektorraum der beleibig oft stetig
differenzierbaren Funktionen, die auch {\em glatt} heissen.
\index{glatte Funktion}%
Nur in diesem letzten Vektorraum sind Ableitungen beliebiger Ordnung
definiert, und die Ableitungen führen in den gleichen Vektorraum zurück,
also
\[
\begin{tikzcd}
C^k(I) \ar[r,"\displaystyle\frac{d^s}{dx^s}"]
	& C^{k-s}(I)
\\
C^\infty(I) \ar[r,"\displaystyle\frac{d^s}{dx^s}"] \ar[u,hook]
	&C^\infty(I) \ar[u,hook]
\end{tikzcd}
\]
Man kann auch den Kern dieser linearen Ableitungen verstehen.
Verschwindet die Ableitung einer Funktion, muss sie auf dem Interval
$I$ konstant sein.
Der Kern von $d/dx$ ist also der Unterraum der konstanten Funktionen.
Verschwindet die $s$-te Ableitung, muss die Funktion ein Polynom
vom Grade $s-1$ sein.
Der Kern von $d^s/dx^s$ besteht also aus den Polynomen vom Grad $s-1$,
oder etwas formeller
\[
\operatorname{ker}\frac{d^s}{dx^s} = \{ p\in \mathbb R[X]\;|\; \deg p < s\}.
\]
Das Zeichen $\deg p$ bedeutet den Grad des Polynoms $p$.

\subsubsection{Stammfunktion}
In der Analysis lernt man auch, zu jeder stetigen Funktion auf dem
Interval $I$ die Stammfunktion zu finden. 
Nach dem Hauptsatz ist die Stammfunktion immer einmal mehr stetig
differenzierbar als die ursprüngliche Funktion es war.
Das finden der Stammfunktion ist aber nicht eindeutig, die Stammfunktion
ist nur bis auf eine Konstante festgelegt.
Man kann diese zum Beispiel dadurch tun, dass man für $I=[a,b]$ die
Stammfunktion als
\[
Sf(x) = \int_a^x f(\xi)\,d\xi
\]
definiert.
Man kann als untere Grenze aber auch jeden beliebigen anderen Punkt
des Intervals nehmen.

Die Abbildung $S$ ist linear, denn es gilt nach den bekannten
Rechenregeln für das Integral
\begin{align*}
S(f+g)(x)
&=
\int_a^x f(\xi) + g(\xi)\,d\xi
=
\int_a^x f(\xi)\,d\xi + \int_a^xg(\xi)\,d\xi
=
(Sf)(x) + (Sg)(x)
\\
S(\lambda f)(x)
&=
\int_a^x \lambda f(\xi)\,d\xi
=
\lambda \int_a^x f(\xi)\,d\xi
=
\lambda (Sf)(x)
\end{align*}
oder
\begin{align*}
S(f+g)&=Sf+Sg
\\
S(\lambda f)&=\lambda Sf,
\end{align*}
die Abbildung $S$ ist also eine lineare Abbildung
\[
C^k(I)\to C^{k+1}(I).
\]

Der Hauptsatz besagt, dass  die Verkettung der Ableitung und der Integration
$S$ aus einer Funktion $f$ bis auf eine Konstante die gleiche Funktion
macht, allerdings muss die neue Funktion im Punkt $a$ den Wert $0$ annehmen.
Es gilt also $ (S(f'))(a) = 0 $.
Die Funktion $f(x)-f(a)$ ist eine Funktion mit dieser Eigenschaft, man findet
also
\[
\biggl(S\circ \frac{d}{dx} f\biggr) (x)
=
f(x)-f(a).
\]
Integriert man zuerst, und leitet erst dann ab, dann lässt die Ableitung
die Integrationskonstante wieder verschwinden, d.~h.~man findet
\[
\frac{d}{dx}\circ S f = f.
\]
Schreibt man die Ableitung etwas kompakter als $D=d/dx$, dann findet man
also, dass
$DS=E$ und $SD$ zwei verschiedene lineare Abbildungen sind.
Man kann sogar noch etwas mehr herausholen.
Führt man $SD$ zweimal aus, erhält man
\[
SDSD = S(DS)D=SED=SD.
\]
Man nennt die Operation $SD$ idempotent, sie zweimal auszuführend ändert
nichts mehr am Resultat.
Dies ist auch anschaulich klar: die Operation $SD$ subtrahiert von einer
Funktion den Wert im Punkt $a$, danach ist der Wert in $a$ gleich Null,
nochmals dieselbe Operation ändert die Funktion nicht mehr.

Eine itempotente lineare Abbildung in einem Vektorraum heisst auch
{\em Projektion},
\index{Projektion}%
die orthogonalprojektion in einem dreidimensionalen Vektorraum hat
ebenfalls diese Eigenschaft.

\subsubsection{Grenzwerte}
Die lineare Algebra ist nicht in der Lage, die analytischen Eigenschaften
dieses Funktionenraums abzubilden.
Man weiss zum Beispiel, dass Grenzwerte von differenzierbaren Funktionen
nicht mehr differenzierbar sein müssen.
Die Funktionen 
\[
f_k(x) = \sqrt[2k+1]{\sin x}
\]
bilden eine Folge von stetigen Funktionen auf $\mathbb R$, welche
punkteweise gegen die ``Rechteck''-Funktion
\[
f(x)=\begin{cases}
 0&\qquad x\in \pi\mathbb Z \\
 1&\qquad x\in (2l\pi, (2l+1)\pi), l\in\mathbb Z\\
-1&\qquad x\in ((2l-1)\pi, 2l\pi), l\in\mathbb Z
\end{cases}
\]
konvergiert.
Letztere ist aber nicht einmal stetig, geschweige denn differenzierbar.

In den Vektorräumen $C^k(I)$ wie bisher defiert ist aber noch nicht
einmal der Begriff des Grenzwertes wohldefiniert, dazu muss der
Vektorraum erst mit einer analytischen Struktur ausgestattet werden,
zum Beispiel mit einer Norm.
Dies soll im Kapitel~XXX untersucht werden.

