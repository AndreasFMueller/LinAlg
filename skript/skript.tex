%
% skript.tex -- Skript zur Vorlesung Lineare Algebra
%               gehalten an der Hochschule Rapperswil im Wintersemester 09
%
% (c) 2009 Prof. Dr. Andreas Mueller, HSR
% $Id: skript.tex,v 1.34 2008/11/02 22:46:16 afm Exp $
%
\documentclass[a4paper,12pt]{book}
\usepackage{german}
\usepackage{times}

\usepackage{amsmath}
\usepackage{amssymb}
\usepackage{amsfonts}
\usepackage{amsthm}
\usepackage{graphicx}
\usepackage{fancyhdr}
\usepackage{textcomp}
\usepackage[all]{xy}
\usepackage{txfonts}
\usepackage{array}
\usepackage{makeidx}
\usepackage{verbatim}
\usepackage{pdflscape}
\usepackage{paralist}
\usepackage{epic}
\usepackage[colorlinks=true]{hyperref}
\usepackage{geometry}
\geometry{papersize={210mm,297mm},total={160mm,240mm},top=31mm,bindingoffset=15mm}
\setlength{\unitlength}{1pt}
\usepackage{color}
\makeindex
\setlength{\headheight}{15pt}
\begin{document}
\pagestyle{fancy}
\lhead{Lineare Algebra}
\frontmatter
\newcommand\HRule{\noindent\rule{\linewidth}{1.5pt}}
\begin{titlepage}
\vspace*{\stretch{1}}
\HRule
\vspace*{10pt}
\begin{flushright}
{\Huge
Lineare Algebra}
\end{flushright}
\HRule
\begin{flushright}
\vspace{30pt}
\LARGE
Andreas M"uller
\end{flushright}
\vspace*{\stretch{2}}
\begin{center}
Hochschule f"ur Technik, Rapperswil, 2009-2016
\end{center}
\end{titlepage}
\hypersetup{
    linktoc=all,
    linkcolor=blue
}
\tableofcontents
\newtheorem{satz}{Satz}[chapter]
\newtheorem{hilfssatz}[satz]{Hilfssatz}
\newtheorem{definition}[satz]{Definition}
\newtheorem{annahme}[satz]{Annahme}
\newtheorem{aufgabe}[satz]{Aufgabe}
% Beispiel
\newenvironment{beispiel}[1][Beispiel]{%
\begin{proof}[\bf #1]%
\renewcommand{\qedsymbol}{$\bigcirc$}%
}{\end{proof}}
\mainmatter
\input linsys.tex
\input einleitung.tex
\input lingl.tex
\section{Anwendungen}
\input applications/kirchhoff.tex
\input applications/sniping.tex
\input determinanten.tex
\section{Anwendungen}
\input applications/spanning.tex
\input vektorgeometrie.tex
%\section{Anwendungen}
%\input applications/raytracer.tex
\input geometrie.tex
\section{Anwendungen}
\input applications/color.tex
\input applications/fitting.tex
\input applications/traegheit.tex
\input applications/wavelets.tex
\input applications/registration.tex
\input applications/tracking.tex
\input zerlegung.tex
%\input wavelets.tex
%\input vektoren.tex
%\input polynome.tex
%\input abbildungen.tex
\input eigen.tex
\section{Anwendungen}
\input applications/google.tex
\input skript.ind
\appendix
\end{document}
