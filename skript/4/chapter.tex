%
% chapter.tex 
%
% (c) 2018 Prof Dr Andreas Müller, Hochschule Rapperswil
%
\chapter{Orthogonalität\label{chapter:orthogonalitaet}}
\lhead{Orthogonalität}
\rhead{}
In der bisher entwickelten Vektorgeometrie haben Längen und Winkel keine 
Rolle gespielt.
Wir haben akzeptiert, dass lineare Abbildungen aus der Elementargeometrie
bekannte Eigenschaften wie Rechtwinkligkeit von Geraden oder Gleichseitigkeit
von Dreiecken zerstören können.
Wir hatten keine Wahl, weil wir kein Werkzeug zur Verfügung hatten, 
um Längen und Winkel zu messen.
Das Skalarprodukt ist dieses zusätzliche Werkzeug.
Wir führen es in Abschnitt~\ref{section:ortho-skalar} ein.

Mit dem Skalarprodukt eröffnet sich ein ganze Menge neuer Anwendungen.
Wir können Geraden und Ebenen jetzt noch etwas prägnanter mit
der Normalenform beschreiben (Abschnitt~\ref{section:normalform}),
wir können besonders gut geeignete Basen konstruieren, in denen die
Zerlegung in Komponenten speziell einfach ist
(Orthonormalbasis, Abschnitt~\ref{section:orthonormalbasis}),
wir können Kreise und Kugeln
beschreiben (Abschnitt~\ref{section:kreisundkugel}) und wir können eine
allgemeine Technik für die Lösung von überbestimmten Gleichungssystemen
entwickeln (Least Squares, Abschnitt~\ref{section:ueberbestimmt}).
In Abschnitt~\ref{section:orthabb} charakterisieren wir Abbildungen,
die das Skalarprodukt nicht verändern, sie beschreiben Bewegungen
des Raumes.

%\begin{verbatim}
5.1 Determinante und Orientierung
- Orientierter Flächeninhalt
- Orientiertes Volumen
5.2 Vektorprodukt
- Definition und Eigenschaften
- Normalenvektoren
- Abstandsformeln
5.3 Abbildungen
- Determinante und Volumenänderung
- Spezielle lineare Gruppe
- Spezielle orthogonale Gruppe
\end{verbatim}



%
% skalar.tex -- Skalarprodukt
%
% (c) 2018 Prof Dr Andreas Müller, Hochschule Rapperswil
%
\section{Orthogonale Projektion und Skalarprodukt\label{section:ortho-skalar}}
\rhead{Orthogonale Projektion und Skalarprodukt}
\index{Skalarprodukt}
Abstand und Winkel spielen in der euklidischen Geometrie eine fundamentale
Rolle, die bisher eingeführten Elemente der Vektorgeometrie erlauben
jedoch noch nicht, Abstände oder Winkel zu berechnen.
Aus der elementaren Trigonometrie ist bekannt, dass der Schlüssel dazu
das Verständnis rechtwinkliger Dreiecke ist.
Der Kosinus eines Winkels ist das Verhältnis von Ankathete zu Hypothenuse.
De Ankathete ist aber auch die orthogonale Projektion der Hypothenuse
auf die Richtung der Ankathete.
Das Skalarprodukt soll daher aus der orthogonalen Projektion entwickelt
werden.

%
% Orthgonale Projektion
%
\subsection{Orthogonale Projektion\label{subsection:orthoproj}}
\index{orthogonale Projektion}
\index{Projektion!orthogonale|see{orthogonale Projektion}}
Zunächst möchten wir zeigen, dass sich Längen und Winkel berechnen
lassen, wenn man in der Lage ist, die Länge der orthogonalen Projektion
eines Vektors $\vec{v}$ auf jeden beliebigen anderen Vektor $\vec{u}$
zu berechnen.
\begin{figure}
\begin{center}
\includegraphics{4/images/projektion.pdf}
\end{center}
\caption{Orthogonale Projektion\label{orthproj}}
\end{figure}

Seien also $\vec u$, $\vec v$ zwei beliebige Vektoren wie in Abbildung~\ref{orthproj}, und $p_{\vec u}(\vec v)$
die Länge der Projektion des Vektors $\vec v$ auf $\vec u$.
Wir versehen diese Länge mit einem Vorzeichen, zeigt der auf $\vec u$
projizierte Vektor $\vec v$ in die gleiche Richtung wie $\vec u$
nehmen wir die Länge positiv, zeigt der projizierte Vektor in die
Gegenrichtung, ist $p_{\vec u}(\vec v)$ negativ.

Die Länge von $\vec v$ ist $p_{\vec v}(\vec v)$, und für den Winkel
$\alpha$ zwischen den beiden Vektoren ist
\begin{equation}
\cos \alpha = \frac{p_{\vec u}(\vec v)}{p_{\vec v}{\vec v}}.
\label{zwischenwinkel}
\end{equation}
Offenbar ist die Länge der Projektion die grundlegendere Grösse,
aus der man die anderen Konzepte ableiten kann.
Etwas ungünstig ist an dieser Projektion nur, dass die beiden Vektoren nicht
symmetrisch eingehen.
\begin{figure}
\centering
\includegraphics{4/images/linearitaet.pdf}
\caption{Die Projektionsabbilung
$\vec{v}\mapsto p_{\vec{u}}(\vec{v})$
ist linear. Die linke Graphik zeigt
$p_{\vec{u}}(\vec{v}_1+\vec{v}_2)
=
p_{\vec{u}}(\vec{v}_1)+p_{\vec{u}}(\vec{v}_2)$,
die rechte ist der Strahlensatz und zeigt
$p_{\vec{u}}(\lambda\vec{v})=\lambda p_{\vec{u}}(\vec{v})$.
\label{projektionlinearitaet}}
\end{figure}
Immerhin ist $p_{\vec u}(\vec v)$ linear in $\vec v$, wie man
sich mit der Abbildung~\ref{projektionlinearitaet}
sofort überzeugen kann, es ist also
\begin{align*}
p_{\vec u}(\vec v_1+\vec v_2)&=p_{\vec u}(\vec v_1)+p_{\vec u}(\vec v_2)\\
p_{\vec u}(\lambda \vec v)&=\lambda p_{\vec u}(\vec v).
\end{align*}

%
% Skalarprodukt
%
\subsection{Skalarprodukt}
\index{Skalarprodukt}
\begin{figure}
\begin{center}
\includegraphics{4/images/skalarprodukt.pdf}
\end{center}
\caption{Skalarprodukt $\vec u\cdot \vec v$ des Einheitsvektors $\vec u$
und des Vektors $\vec v$ mit Zwischenwinkel
$\alpha$.\label{image-skalarprodukt}}
\end{figure}
Gesucht ist daher eine Konstruktion, welche immer noch linear ist,
aber auch symmetrisch in $\vec u$ und $\vec v$.
Die Formel (\ref{zwischenwinkel}) deutet auch an, wie dies erreicht
werden kann.
Der Zwischenwinkel kann natürlich auch berechnet werden,
indem die beiden Vektoren vertauscht werden:
\[
\cos \alpha
=
\frac{p_{\vec u}(\vec v)}{p_{\vec v}(\vec v)}
=
\frac{p_{\vec v}(\vec u)}{p_{\vec u}(\vec u)}
\]
Multipliziert man diese Gleichung mit
$
p_{\vec u}(\vec u)
p_{\vec v}(\vec v)
$, erhält man
\[
\vec u\cdot\vec v
=
p_{\vec u}(\vec u)
p_{\vec v}(\vec v)
\cos\alpha =
p_{\vec u}(\vec u)p_{\vec u}(\vec v)
=
p_{\vec v}(\vec v)p_{\vec v}(\vec u),
\]
was offenbar symmetrisch in $\vec u$ und $\vec v$ ist.

\begin{definition}Das Skalarprodukt zweier Vektoren $\vec u$ und
$\vec v$ ist
\[
\vec u\cdot\vec v
=
p_{\vec u}(\vec u)
p_{\vec v}(\vec v)
\cos\alpha.
\]
\end{definition}
Diese Grösse ist linear in $\vec u$ und linear in $\vec v$, und man kann
daraus $p_{\vec u}(\vec v)$ mittels
\[
p_{\vec u}(\vec v)
=
\frac{p_{\vec u}(\vec v)p_{\vec u}(\vec u)}{p_{\vec u}(\vec u)}
=
\frac{\vec u\cdot\vec v}{\sqrt{p_{\vec u}(\vec u)^2}}
=
\frac{\vec u\cdot\vec v}{\sqrt{\vec u\cdot \vec u}}
\]
wieder zurückgewinnen.

\begin{satz}
Seien $\vec u$ und $\vec v$ zwei Vektoren, dann ist
\[
|\vec u|=p_{\vec u}(\vec u)=\sqrt{\vec u\cdot\vec u}
\]
die Länge des Vektors, und für den Zwischenwinkel $\alpha$ gilt
\[
|\vec u|\,|\vec v|\cos\alpha=\vec u\cdot\vec v
\]
Zwei vom Nullvektor verschiedene Vektoren  stehen genau dann senkrecht
aufeinander, wenn $\vec u\cdot\vec v=0$.
Die Projektion $\vec v_{\|}$ von $\vec v$ auf $\vec u$ ist
\[
\vec v_{\|}=\frac{\vec v\cdot\vec u}{\vec u\cdot\vec u}\vec u.
\]
Ist $\vec u$ ein Einheitsvektor, dann ist $\vec v_{\|}=(\vec v\cdot \vec u)\vec u$.
\end{satz}
\index{Zwischenwinkel}

%
% Skalarprodukt und Standardbasis
%
\subsection{Skalarprodukt und Standardbasis}
Zur praktischen Berechnung des Skalarproduktes benötigen wir
eine Formel, die das Skalarprodukt aus den Vektorkomponenten
berechnet.
Schreibt man
\[
\vec u=\begin{pmatrix}u_1\\u_2\\u_3\end{pmatrix}
=u_1\vec e_1+u_2\vec e_2+u_3\vec e_3
,
\qquad
\vec v=\begin{pmatrix}v_1\\v_2\\v_3\end{pmatrix}
=v_1\vec e_1+v_2\vec e_2+v_3\vec e_3
\]
dann kann das Skalarprodukt mit der Linearität berechnet werden:
\begin{align*}
\vec u\cdot\vec v
&=
(u_1\vec e_1+u_2\vec e_2+u_3\vec e_3)\cdot
(v_1\vec e_1+v_2\vec e_2+v_3\vec e_3)
\\
&=
u_1v_1\vec e_1\cdot\vec e_1+
u_1v_2\vec e_1\cdot\vec e_2+
u_1v_3\vec e_1\cdot\vec e_3\\
&\qquad +
u_2v_1\vec e_2\cdot\vec e_1+
u_2v_2\vec e_2\cdot\vec e_2+
u_2v_3\vec e_2\cdot\vec e_3\\
&\qquad+
u_3v_1\vec e_3\cdot\vec e_1+
u_3v_2\vec e_3\cdot\vec e_2+
u_3v_3\vec e_3\cdot\vec e_3
\end{align*}
Die Skalarprodukte von aufeinander senkrecht stehenden Vektoren
verschwinden, es bleiben nur die Termen mit $\vec e_i\cdot\vec e_i$,
das Skalarprodukt eines Vektors mit sich selbst ist das Quadrat
der Länge, also $\vec e_i\cdot \vec e_i=1$ und so erhalten wir den
Satz
\begin{satz}
Das Skalarprodukt zweier Vektoren
\[
\vec u=\begin{pmatrix}u_1\\u_2\\u_3\end{pmatrix},
\qquad
\vec v=\begin{pmatrix}v_1\\v_2\\v_3\end{pmatrix}
\]
ist
\[
\vec u\cdot\vec v
=
u_1v_1+u_2v_2+u_3v_3.
\]
\end{satz}

\begin{beispiel}
Berechne die Länge und den Zwischenwinkel der Vektoren
\begin{align*}
\vec a&= \begin{pmatrix} 3\\12\\ 4 \end{pmatrix},
&
\vec b&= \begin{pmatrix}2\\3\\6\end{pmatrix}.
\end{align*}

\smallskip

{\parindent 0pt Die} Länge der Vektoren ist
\begin{align*}
|\vec a|
&
=\sqrt{\vec a\cdot \vec a}
&
|\vec b|
&
=\sqrt{\vec b\cdot \vec b}
\\
&=\sqrt{9+144+16}
&
&=\sqrt{4+9+36}
\\
&=\sqrt{169}=13
&
&
=\sqrt{49}=7
\end{align*}
Damit kann man jetzt auch den Zwischenwinkel berechnen
\begin{align*}
\cos\alpha&= \frac{\vec a\cdot \vec b}{|\vec a|\;|\vec b|}
=
\frac{6+36+24}{13\cdot 7}=\frac{66}{91}=0.72527
\\
\alpha&=43.51^\circ
\end{align*}
\end{beispiel}
Häufig braucht man zu einem Vektor einen Vektor mit gleicher Richtung,
aber Einheitslänge.
Wir verwenden die Schreibweise
\[
\vec{v}^0 = \frac{\vec{v}}{|\vec{v}|}
\]
für den zum Vektor $\vec{v}$ gehörigen Einheitsvektor.


%
% anwendungen.tex -- Anwendungen des Skalarproduktes
%
% (c) 2018 Prof Dr Andreas Müller, Hochschule Rapperswil
%
\section{Erste Anwendungen des Skalarproduktes\label{section:normalform}}
\rhead{Erste Anwendungen des Skalarproduktes}
Aus den Eigenschaften des Skalarproduktes ergeben sich unmittelbar
erste Anwendungen.

%
% Normalenform von Ebene und Gerade
%
\subsection{Normalenform von Ebene und Gerade}
\begin{figure}
\begin{center}
\includegraphics{4/images/normalenform.pdf}
\end{center}
\caption{Ebene in Normalenform\label{image-normalenform}}
\end{figure}
Das Skalarprodukt gibt uns eine neue Möglichkeit, Ebenen zu
beschreiben.
Eine Ebene durch den Punkt $P$ senkrecht auf den Vektor
$\vec n$ besteht genau aus jenen Punkten $Q$, für die der Vektor
$\overset{\rightarrow}{PQ}$ auf $\vec n$ senkrecht steht
(Abbildung~\ref{image-normalenform}).
Mit dem Skalarprodukt
ausgedrückt: Die Menge der Ortsvektoren der Punkte einer Ebene durch $P$ mit
\index{Normale}
Normale $\vec n$ ist
\[
\{\vec r\;|\;(\vec r-\vec p)\cdot \vec n=0\}
\]
Multipliziert man die Gleichung aus, erhält man
\begin{align*}
\left(
\begin{pmatrix}x\\y\\z\end{pmatrix}
-\begin{pmatrix}p_1\\p_2\\p_3\end{pmatrix}\right)\cdot
\begin{pmatrix}n_1\\n_2\\n_3\end{pmatrix}&=0
\\
(x-p_1)n_1+(y-p_2)n_2+(z-p_3)n_3&=0
\\
n_1x+n_2y+n_3z&=p_1n_1+p_2n_2+p_3n_3
\end{align*}
Diese Form der Ebenengleichung, in der $\vec n$ ein Einheitsnormalenvektor ist,
heisst auch {\em Hessesche Normalform}.
\index{Normalenform}
\index{Hessesche Normalform}

\begin{satz}
Ist $\vec n$ ein Einheitsvektor, dann ist
\[
d=(\vec r-\vec p_0)\cdot \vec n
\]
der Abstand des Punktes mit dem Ortsvektor $\vec r$ von der Ebene durch
den Punkt mit Ortsvektor $\vec p_0$ und Normalen $\vec n$.
In Koordinaten:
\[
d=n_xx+n_yy+n_zz-\vec p_0\cdot\vec n
\]
\end{satz}
\begin{proof}[Beweis]
$(\vec r-\vec p)\cdot \vec n$ ist die Länge der Projektion des Vektors
$\vec r -\vec p$ auf den Normalenvektor $\vec n$, also genau der behauptete
Abstand.
\end{proof}

\begin{beispiel}
Man finde die Normalenform der Ebenengleichung (\ref{beispielebene}) auf
Seite~\pageref{beispielebene},
und berechne den Abstand des Punktes $(1,1,1)$ von der Ebene.

\medskip

{\parindent 0pt Die} Lösung vollzieht sich in folgenden Schritten:
\begin{compactenum}
\item Bestimme die Normale der Ebene.
\item Schreibe die Gleichung der Ebene in Normalenform.
\item Bringe die Normalenform in Hessesche Normalform.
\item Berechne den Abstand des Punktes $(1,1,1)$.
\end{compactenum}
Gesucht ist ein Vektor $\vec n$, der auf beiden
Richtungsvektoren senkrecht steht, also
\begin{equation}
\begin{pmatrix}n_1\\n_2\\n_3\end{pmatrix}
\cdot
\begin{pmatrix}2\\2\\-2\end{pmatrix}
=0,
\qquad
\begin{pmatrix}n_1\\n_2\\n_3\end{pmatrix}
\cdot
\begin{pmatrix}3\\-3\\-1\end{pmatrix}
=0
\label{gleichungen-fuer-normale}
\end{equation}
Dies ist gleichbedeutend mit dem Gleichungssystem
\[
\begin{pmatrix}
2&2&-2\\
3&-3&-1
\end{pmatrix}
\begin{pmatrix}n_1\\n_2\\n_3\end{pmatrix}
=\begin{pmatrix}0\\0 \end{pmatrix}
\]
Der Gauss-Algorithmus liefert
\begin{align*}
\begin{tabular}{|>{$}c<{$}>{$}c<{$}>{$}c<{$}|}
\hline
2%
\begin{picture}(0,0)
\color{red}\put(-3,4){\circle{12}}
\end{picture}%
&2&-2\\
3%
\begin{picture}(0,0)%
\color{blue}\drawline(-8,-2)(-8,10)(2,10)(2,-2)
\end{picture}%
&-3&-1\\
\hline
\end{tabular}
&\rightarrow
\begin{tabular}{|>{$}c<{$}>{$}c<{$}>{$}c<{$}|}
\hline
1&1&-1\\
0&-6%
\begin{picture}(0,0)%
\color{red}\put(-7,4){\circle{15}}
\end{picture}%
&2\\
\hline
\end{tabular}
\rightarrow
\begin{tabular}{|>{$}c<{$}>{$}c<{$}>{$}c<{$}|}
\hline
1&1%
\begin{picture}(0,0)
\color{blue}\drawline(-8,10)(-8,-2)(2,-2)(2,10)
\end{picture}%
&-1\\
0&1&-\frac13\\
\hline
\end{tabular}
\rightarrow
\begin{tabular}{|>{$}c<{$}>{$}c<{$}>{$}c<{$}|}
\hline
1&0&-\frac23\\
0&1&-\frac13\\
\hline
\end{tabular}
\end{align*}
Die Komponente $n_3$ ist frei wählbar, wir setzen $n_3=3$, und bekommen
$n_1=2$ und $n_2=1$.
Tatsächlich ist
\begin{align*}
\begin{pmatrix}2\\1\\3\end{pmatrix}
\cdot
\begin{pmatrix}2\\2\\-2\end{pmatrix}
&=4+2-6=0
&
\begin{pmatrix}2\\1\\3\end{pmatrix}
\cdot
\begin{pmatrix}3\\-3\\-1\end{pmatrix}
&=6-3-3=0.
\end{align*}
Damit ist die Normalenform der Ebenengleichung
\begin{align}
\begin{pmatrix}2\\1\\3\end{pmatrix}\cdot\left(
\begin{pmatrix}x\\y\\z\end{pmatrix} - \begin{pmatrix}1\\2\\1\end{pmatrix}
\right)&=0\notag\\
\Rightarrow\qquad
2x+y+3z&=7\label{normalenform}
\end{align}
Diese Form ist zwar eine Normalenform, aber noch nicht die Hessesche
Normalform, da man für diesen Zweck einen Einheitsvektor als
Normalenvektor verwenden muss.
Unser Normalenvektor hat aber die Länge $|\vec n|=\sqrt{14}$.
Dividieren wir die Gleichung (\ref{normalenform})
durch $\sqrt{14}$, erhalten wir die Hessesche Normalform:
\begin{equation}
d=\frac{2}{\sqrt{14}}x+\frac{1}{\sqrt{14}}y+\frac{3}{\sqrt{14}}z-\frac{7}{\sqrt{14}}.
\label{hnf}
\end{equation}
Die Hessesche Normalform berechnet den Abstand eines Punktes von der
Ebene.
Man muss jetzt also nur noch den Punkt $(1,1,1)$ in die  Gleichung
(\ref{hnf}) einsetzen:
\[
d = (2+1+3-7)/\sqrt{14}=-1/\sqrt{14}=-0.26726,
\]
der gesuchte Abstand ist also $d=0.26726$.
\end{beispiel}

%
% Spiegelungen an einer Geraden oder Ebenen
%
\subsection{Spiegelung an einer Geraden oder Ebenen\label{spiegelung}}
\begin{figure}
\begin{center}
\includegraphics{4/images/spiegelung.pdf}
\end{center}
\caption{Spiegelung eines Vektors $\vec v$ an der Ebene senkrecht auf $\vec n$.
\label{image-spiegelung}}
\end{figure}
Da man mit dem Skalarprodukt senkrechte Projektionen berechnen kann,
muss es auch möglich sein, die Spiegelung eines Vektors $\vec v$
an einer Ebene mit Normale $\vec n$ zu berechnen ($|\vec n|=1$).
Dazu zerlegt man den Vektor $\vec v$ in eine Komponente $\vec v_{\|}$
parallel zur Ebene und eine Komponenten $\vec v_{\perp}$ senkrecht dazu,
also $\vec v=\vec v_{\|}+\vec v_{\perp}$ (Abbildung~\ref{image-spiegelung}).
Die senkrechte Komponente
ist im wesentlichen die Projektion von $\vec v$ auf $\vec n$:
\[
\vec v_{\perp}=
(\vec v\cdot\vec n)\vec n
.
\]
Die parallele Komponente ist der Rest:
\[
\vec v_{\|}=\vec v -\vec v_{\perp}=
\vec v-(\vec v\cdot\vec n)\vec n
,
\]
Beim gespiegelten Vektor zeigt die senkrechte Komponente in die
entgegengesetzte Richtung:
\begin{equation}
\vec v_{\text{gespiegelt}}=
\vec v_{\|}-\vec v_{\perp}
=
\vec v-(\vec v\cdot\vec n)\vec n
-
(\vec v\cdot\vec n)\vec n
=\vec v-2(\vec v\cdot\vec n)\vec n.
\label{equation:spiegelung}
\end{equation}

\begin{beispiel}
Man spiegle den Vektor $\vec a$ an der Ebene mit der Normalen $\vec n$,
\[
\vec a=\begin{pmatrix}1\\2\\3\end{pmatrix},
\qquad
\vec n=\begin{pmatrix}1\\1\\1\end{pmatrix}
\]

\smallskip

{\parindent 0pt Zunächst stellen wir fest,} dass $\vec n$ noch
kein Einheitsvektor ist, dass wir stattdessen $\vec n_0=\vec n/\sqrt{3}$
verwenden müssen.
Damit kann $\vec a$ jetzt die parallelen und orthogonalen
Komponenten zerlegt werden:
\[
\vec a_{\perp}=(\vec a\cdot\vec n_0)\vec n_0
=\frac1{\sqrt{3}} (1+2+3)\frac1{\sqrt{3}}\begin{pmatrix}1\\1\\1\end{pmatrix}
=\begin{pmatrix}2\\2\\2\end{pmatrix},
\quad
\vec a_{\|}=\begin{pmatrix} -1\\0\\1 \end{pmatrix}.
\]
Nach Formel (\ref{equation:spiegelung}) ist
\[
\vec a'=\vec a_{\|}-\vec a_{\perp}
=
\begin{pmatrix}-1\\0\\1\end{pmatrix}-\begin{pmatrix}2\\2\\2\end{pmatrix}
=\begin{pmatrix}-3\\-2\\-1\end{pmatrix}.
\]
der gespiegelt Vektor.
\end{beispiel}


%
% orthonrmalbasis.tex -- Orthonormierte Basis und Gram Schmidt
%
% (c) 2018 Prof Dr Andreas Müller, Hochschule Rapperswil
%
\section{Skalarprodukt und Basis\label{section:orthonormalbasis}}
Bei der Berechnung des Skalarproduktes in Komponenten in der Standardbasis
hat sich gezeigt, dass eine Basis aus Einheitsvektoren, die zusätzlich
aufeinander senkrecht stehen, besonders gut für die Arbeit mit dem
Skalarprodukt geeignet ist.
Nicht immer hat man allerdings eine solch bequeme Basis.
In diesemAbschnitt sollen die Vorzüge einer solchen Basis nochmals
herausgearbeitet werden und es soll gezeigt werden, wie man aus
einer beliebigen Basis immer eine passende Basis aus orthogonalen
Einheitsvektoren machen kann.
Schliesslich wird gezeigt, wie sich das Skalarprodukt in einer beliebigen
Basis schreiben lässt.

\subsection{Orthonormalbasis}
Die Vektoren $\vec e_i$ stehen senkrecht aufeinander und haben
Länge $1$.
Der Vektor
\[
\vec v
=
\begin{pmatrix}v_1\\v_2\\v_3\end{pmatrix}
\]
lässt sich mit Hilfe des Skalarproduktes als Summe von Vielfachen
der Vektoren $\vec e_i$ schreiben.
Es ist nämlich $v_i=\vec v\cdot\vec e_i$, also
\[
\vec v
=
\begin{pmatrix}v_1\\v_2\\v_3\end{pmatrix}
=
v_1\vec e_1
+
v_2\vec e_2
+
v_3\vec e_3
=
(\vec v\cdot \vec e_1)\vec e_1
+
(\vec v\cdot \vec e_2)\vec e_2
+
(\vec v\cdot \vec e_3)\vec e_3
\]
Dies funktioniert aber nicht nur für die Vektoren $\vec e_i$.
Seien $\vec b_1$, $\vec b_2$ und $\vec b_3$ drei aufeinander senkrecht
stehende Vektoren der Länge $1$.
Mit dem Skalarprodukt kann man dies durch
\[
\vec b_i\cdot\vec b_j=\begin{cases}
0&\qquad i\ne j\\
1&\qquad i=j
\end{cases}
\]
ausdrücken.
Versucht man den den Vektor $\vec v$ als Linearkombination
der Vektoren $\vec b_i$ zu schreiben, also
\[
\vec v
=
v_1'\vec b_1
+
v_2'\vec b_2
+
v_3'\vec b_3
\]
Berechnet man jetzt das Skalarprodukt von $\vec v$ mit $\vec b_i$,
findet man
\begin{align*}
\vec v\cdot \vec b_i
&=
(
v_1'\vec b_1
+
v_2'\vec b_2
+
v_3'\vec b_3
)\cdot
\vec b_i
\\
&=
v_1'\vec b_1\cdot\vec b_i
+
v_2'\vec b_2\cdot\vec b_i
+
v_3'\vec b_3\cdot\vec b_i
\\
&=v_i'
\end{align*}
weil alle Skalarprodukte verschwinden ausser zwischen
zwei gleichen Vektoren.

\begin{definition} Die Koeffizienten der Einheitsmatrix
\[
\delta_{ij}=
\begin{cases}
0&\qquad i\ne j\\
1&\qquad i=j
\end{cases}
\]
heisst {\em Kronecker-Delta}.
\end{definition}

\begin{definition}
$n$ Vektoren $\vec b_i$ heissen orthonormiert, wenn gilt
\[
\vec b_i\cdot\vec b_j=\delta_{ij}.
\]
\end{definition}

\begin{satz}
Sind die Vektoren $\vec b_i$ orthonormiert, dann kann man jeden
Vektor $\vec v$ als Linearkombination der Vektoren $\vec b_i$
\[
\vec v=
(\vec v\cdot\vec b_1)\vec b_1
+
(\vec v\cdot\vec b_2)\vec b_2
+
(\vec v\cdot\vec b_3)\vec b_3
\]
schreiben.
Diese Darstellung ist eindeutig.
\end{satz}

\begin{proof}[Beweis]
Es ist nur noch zu beweisen, dass es nur eine solche Darstellung als
Linearkombination gibt.
Gäbe es zwei Darstellungen, also
\begin{align*}
\vec v
&=
v_1'\vec b_1+
v_2'\vec b_2+
v_3'\vec b_3\\
&=
v_1''\vec b_1+
v_2''\vec b_2+
v_3''\vec b_3,
\end{align*}
können wir die Differenz bilden:
\[
0=
(v_1'-v_1'')\vec b_1
+
(v_2'-v_2'')\vec b_2
+
(v_3'-v_3'')\vec b_3
\]
Das Skalarprodukt mit $b_i$ ergibt dann
\[
0=(v_i'-v_i'')\quad\Rightarrow\quad v_i'=v_i''.
\]
Da wir $i$ beliebig wählen können folgt, dass die
Koeffizienten $v_i'$ und $v_i''$ übereinstimmen.
\end{proof}

%\subsection{Verallgemeinertes Skalarprodukt}

%\begin{definition}
%Ein symmetrische Matrix $g_{ij}$ definiert ein allgemeines Skalarprodukt
%$$g(x,y)=\sum_{i,j=1}^ng_{ij}x_iy_i$$
%der Vektoren
%$$x=\begin{pmatrix}x_1\\\vdots\\x_n\end{pmatrix}\qquad\text{und}
%\qquad y=\begin{pmatrix}y_1\\\vdots\\y_n\end{pmatrix}.$$
%\end{definition}
%Das zu Beginn dieses Abschnitts definiert Skalarprodukt ist ein
%verallgemeinertes Skalarprodukt mit der Matrix $I$.

\subsection{Orthonormalisierung}
Für orthonormierte Vektoren ist es besonders einfach, eine Darstellung
eines beliebigen Vektors als Linearkombination zu finden.
Es ist daher sicher nützlich, aus einer Menge von Vektoren
$\{\vec a_1,\vec a_2,\vec a_3\}$
eine neue Menge von Vektoren zu konstruieren, die sich von der gegeben
möglichst wenig
unterscheidet, aber dennoch aus orthonormierten Vektoren besteht.

\begin{satz}[Gram-Schmidt]
\label{satz-gram-schmidt}
Seien $\{\vec a_1,\vec a_2,\vec a_3\}$ linear unabhängige Vektoren.
Dann gibt es orthonormierte Vektoren $\{\vec b_1,\vec b_2,\vec b_3\}$ so,
dass $b_k$ aus $a_1,\dots,a_k$ linear kombiniert werden kann, für jedes $k$.
Die $b_i$ lassen sich wie folgt berechnen
\begin{align*}
\vec b_1&=\frac1{|\vec a_1|}a_1\\
\vec b_2&=
\frac{
\vec a_2-(\vec a_2\cdot \vec b_1)\vec b_1
}{
|\vec a_2-(\vec a_2\cdot \vec b_1)\vec b_1|
}
\\
\vec b_3
&=
\frac{
\vec a_3-(\vec a_3\cdot \vec b_1)\vec b_1-(\vec a_3\cdot\vec b_2)\vec b_2
}{
|
\vec a_3-(\vec a_3\cdot \vec b_1)\vec b_1-(\vec a_3\cdot\vec b_2)\vec b_2
|
}
\\
&\phantom{=}\vdots\\
\vec b_k&=\frac{\vec a_k-(\vec a_k\cdot \vec b_1)\vec b_1-(\vec a_k\cdot \vec b_2)\vec b_2-\dots-(\vec a_k\cdot \vec b_{k-1})\vec b_{k-1}}{|\vec a_k-(\vec a_k\cdot \vec b_1)\vec b_1-(\vec a_k\cdot \vec b_2)\vec b_2-\dots-(\vec a_k\cdot \vec b_{k-1})\vec b_{k-1}|}
\end{align*}
Das Verfahren lässt sich offenbar auf eine beliebige Zahl linear
unabhängiger $n$-dimensionaler Vektoren verallgemeinern.
\end{satz}

Auf die Reihenfolge der Vektoren kommt es entscheidend an, wie die
folgenden zwei Beispiele zeigen
\begin{beispiel}
Die Vektoren
\[
\vec a_1=\begin{pmatrix}1\\0\\0\end{pmatrix},\qquad
\vec a_2=\begin{pmatrix}1\\1\\0\end{pmatrix},\qquad
\vec a_3=\begin{pmatrix}1\\1\\1\end{pmatrix}
\]
sind zu orthonormieren.

Die Formeln aus Satz~\ref{satz-gram-schmidt} liefern folgende Vektoren:
\begin{align*}
\vec b_1&=\frac{\vec a_1}{|\vec a_1|}=\begin{pmatrix}1\\0\\0\end{pmatrix}\\
\vec b_2&=
\frac{
\vec a_2-(\vec a_2\cdot \vec b_1)\vec b_1
}{
|\vec a_2-(\vec a_2\cdot \vec b_1)\vec b_1|
}
=
\frac{
\begin{pmatrix}1\\1\\0\end{pmatrix}-1\cdot\begin{pmatrix}1\\0\\0\end{pmatrix}
}{\dots}=\begin{pmatrix}0\\1\\0\end{pmatrix}\\
\\
\vec b_3&=
\frac{\vec a_3 -(\vec a_3\cdot \vec b_1)\vec b_1-(\vec a_3\cdot\vec b_2)\vec b_2}{\dots}
=\frac{\begin{pmatrix}1\\1\\1\end{pmatrix}-1\cdot \begin{pmatrix}1\\0\\0\end{pmatrix}-1\cdot\begin{pmatrix}0\\1\\0\end{pmatrix}
}{\dots}=\begin{pmatrix}0\\0\\1\end{pmatrix}
\end{align*}
Man findet also genau die Vektoren der Standardbasis.
\end{beispiel}

\begin{beispiel}
Die Vektoren
\[
\vec a_1=\begin{pmatrix}1\\0\\0\end{pmatrix},\qquad
\vec a_2=\begin{pmatrix}1\\1\\1\end{pmatrix},\qquad
\vec a_3=\begin{pmatrix}1\\1\\0\end{pmatrix}
\]
sind zu orthonormieren.

Dieses Beispiel unterscheidet sich vom vorangegangenen nur
durch die Reihenfolge der Vektoren.
Wieder können die Formeln von Satz~\ref{satz-gram-schmidt} angewandt werden:
\begin{align*}
\vec b_1&=\frac{\vec a_1}{|\vec a_1|}=\begin{pmatrix}1\\0\\0\end{pmatrix}
\\
\vec b_2
&=
\frac{\vec a_2-(\vec a_2\cdot \vec b_1)\vec b_1}{\dots}
=
\frac{\begin{pmatrix}1\\1\\1\end{pmatrix}-1\cdot\begin{pmatrix}1\\0\\0\end{pmatrix}}{\dots}=\frac1{\sqrt{2}}\begin{pmatrix}0\\1\\1\end{pmatrix}
\\
\vec b_3
&=
\frac{\vec a_3-(\vec a_3\cdot\vec b_1)\vec b_1-(\vec a_3\cdot\vec b_2)\vec b_2}{\dots}
=\frac{\displaystyle\begin{pmatrix}1\\1\\0\end{pmatrix}-1\cdot\begin{pmatrix}1\\0\\0\end{pmatrix}-\frac1{\sqrt{2}}\cdot\frac1{\sqrt{2}}\begin{pmatrix}0\\1\\1\end{pmatrix} }{\cdots}
=\frac{1}{\sqrt{2}}\begin{pmatrix}0\\1\\-1\end{pmatrix}.
\end{align*}
Die gefundenen Vektoren sind völlig verschiedenen von den Vektoren
im vorangegangenen Beispiel.
\end{beispiel}


%
% kreis.tex 
%
% (c) 2018 Prof Dr Andreas Müller, Hochschule Rapperswil
%
\section{Kreis und Kugel\label{section:kreisundkugel}}
\rhead{Kreis und Kugel}
Da das Skalarprodukt die Länge eines Vektors berechnet, kann man
jetzt auch die Menge der Punkte eines Kreises in der Ebene
oder einer Kugel im Raum vektoriell beschreiben, und damit Standardaufgaben
lösen.

%
% Gleichungen von Kreis und Kugel
%
\subsection{Gleichungen von Kreis und Kugel}
\index{Kreis}
\index{Kugel}
Die Kugel $K(M,r)$ ist die Menge aller Punkte, die von einem festen Punkt, dem
Mittelpunkt oder Zentrum, alle den gleichen Abstand $r$ haben.
In vektorieller
Form heisst dies, dass die Ortsvektoren von Punkten mit dem Ortsvektor des
Mittelpunktes eine Differenz konstanter Länge haben.
Schreiben wir $\vec m$
für den Ortsvektor des Mittelpunktes, dann besteht die Kugel aus den
Punkten mit den Ortsvektoren
\[
K(M,r)
=
\{\vec p\;| \;|\vec p-\vec m|=r\}
=
\{\vec p\;| \;(\vec p-\vec m)\cdot(\vec p-\vec m)=r^2\}.
\]
Die zweite Form ist für die Lösung konkreter Problem oft nützlicher.

%
% Durchstosspunkt einer Geraden mit einer Kugel
%
\subsection{Durchstosspunkt einer Geraden mit einer Kugel
\label{durchstosspunktkugel}}
Gesucht ist der Durchstosspunkt der Geraden
\[
\vec p=\vec p_0+t\vec r
\]
durch die Kugel, also die Menge
\[
\{\vec p\;| \;|(\vec p_0+t\vec r)-\vec m|=r\}
\]
In der zweiten Form der Kugelgleichung haben wir
\begin{align*}
((\vec p_0+t\vec r)-\vec m)
\cdot
((\vec p_0+t\vec r)-\vec m)&=r^2
\\
(\vec p_0+t\vec r)
\cdot
(\vec p_0+t\vec r)
-2
(\vec p_0+t\vec r)\cdot \vec m
+\vec m\cdot\vec m&=r^2
\\
\vec p_0\cdot\vec p_0
+2t\vec p_0\cdot\vec r
+t^2\vec r\cdot\vec r
-2\vec p_0\cdot\vec m
-2t\vec r\cdot\vec m
+\vec m\cdot\vec m&=r^2
\\
t^2|\vec r|^2
+t(2\vec p_0\cdot\vec r-2\vec m\cdot\vec r)
+(\vec p_0\cdot\vec p_0-2\vec p_0\cdot\vec m+\vec m\cdot\vec m)&=r^2
\end{align*}
So erhalten wir die quadratische Gleichung für den Parameter $t$
des Durchstosspunktes:
\[
|\vec r|^2t^2
+2\vec r\cdot(\vec p_0-\vec m)t
+|\vec p_0-\vec m|^2-r^2 =0.
\]
Indem wir die Diskriminante berechnen,  können wir eine Kriterium
dafür ableiten, ob die Gerade die Kugel berührt.
Die Diskriminante
der quadratischen Gleichung $at^2+bt+c=0$ ist $\Delta = b^2-4ac$.
In unserem
Fall wird daraus
\[
\Delta
=
4(\vec p_0\cdot(\vec p_0-\vec m))^2-
4|\vec r|^2(
|\vec p_0-\vec m|^2-r^2
)
\]
Die Diskriminante ist zum Beispiel immer positiv, wenn der Stützpunkt
$\vec p_0$
der Gerade innerhalb der Kugel ist, also $|\vec p_0-\vec m|<r$

%
% Thaleskreis
%
\subsection{Thaleskreis}
\begin{figure}
\centering
\includegraphics{4/images/thales.pdf}
\caption{Die Punkte $P$, von denen aus die Strecke $AB$ unter einem
rechten Winkel erscheint, bilden den Thaleskreis.
\label{thales-graphik}}
\end{figure}
Mit dem Skalarprodukt kann man ausdrücken, von welchen Punkten aus
eine Strecke $AB$ unter einem rechten Winkel gesehen wird, es ist
dies die Menge der Punkte $P$ mit der Eigenschaft
\[
\overrightarrow{AP}\cdot\overrightarrow{BP}=0
\]
(Abbildung~\ref{thales-graphik}).
Unter Verwendung der Konvention, dass kleine Buchstaben für die
Ortsvektoren der Punkte mit entsprechenden Grossbuchstaben stehen, ist
dies gleichbedeutend mit
\[
(\vec p-\vec a)\cdot(\vec p-\vec b)=0
\]
Ausmultiplizieren und quadratisch ergänzen ergibt
\begin{align*}
\vec p^2-\vec p\cdot\vec b-\vec a\cdot\vec p+\vec a\cdot\vec b&=0
\\
\vec p^2-(\vec a+\vec b)\cdot \vec p+\vec a\cdot\vec b&=0
\\
\vec p^2-2\left(\frac{\vec a+\vec b}{2}\right)\cdot \vec p
+\left(\frac{\vec a+\vec b}{2}\right)^2
-\left(\frac{\vec a+\vec b}{2}\right)^2
+\vec a\cdot\vec b&=0
\\
\left(\vec p
-\frac{\vec a+\vec b}{2}\right)^2&=
\left(\frac{\vec a+\vec b}{2}\right)^2-\vec a\cdot \vec b
\\
&=
\frac{\vec a^2+2\vec a\cdot\vec b+\vec b^2-4\vec a\cdot \vec b}{4}
\\
&=
\frac{\vec a^2-2\vec a\cdot\vec b+\vec b^2}{4}
\\
\left(\vec p
-\frac{\vec a+\vec b}{2}\right)^2&=
\left(\frac{\vec a-\vec b}{2}\right)^2
\end{align*}
Diese Gleichung beschreibt einen Kreis um den Punkt $(\vec a+\vec b)/2$
mit dem Radius $|\vec a-\vec b|/2$.
Dies ist der Thaleskreis.

%
% Tangente in einem Punkt
%
\subsection{Tangente oder Tangentialebene in einem Punkt}
\begin{figure}
\centering
\includegraphics{4/images/tangente.pdf}
\caption{Tangente im Punkt $P_0$ an den Kreis um $M$ mit Radius $r$.
Der Vektor $\vec{p}-\vec{p}_0$ muss auf dem
Normalenvektor $\vec{n}$ senkrecht stehen, letzterer hat die Richtung
von $
%\overrightarrow{MP_0}=
\vec{p}_0-\vec{m}$.
\label{tangente-graphik}}
\end{figure}
Die Normale auf einen Kreis ist immer parallel zum Radiusvektor
(Abbildung~\ref{tangente-graphik}),
dasselbe gilt für eine Kugel.
Daher kann
man die Gleichung der Tangente in einem Punkt $P_0$ an einen Kreis
oder der Tangentialebene in einem Punkt der Kugel sofort angeben:
\begin{satz}\label{kugeltangentialebene}
Die Tangentialebene an eine Kugel mit Mittelpunktsortsvektor
$\vec m$ und Radius  $r$ im Punkt mit Ortsektor $\vec p_0$ ist
\begin{equation}
\{\vec p\;|\;
(\vec p-\vec p_0)\cdot(\vec p_0-\vec m)=0
\}
\label{eqn-kugeltangente}
\end{equation}
\end{satz}
Diese Form (\ref{eqn-kugeltangente}) ist noch nicht optimal, man kann den
Radius nicht direkt ablesen.
Addieren wir jedoch noch die Gleichung
der Kugel für den Vektor $\vec p_0$, erhalten wir
\begin{align*}
(\vec p-\vec p_0)\cdot(\vec p_0-\vec m)&=0\\
(\vec p_0-\vec m)\cdot(\vec p_0-\vec m)&=r^2\\
\Rightarrow\qquad
(\vec p-\vec m)\cdot(\vec p_0-\vec m)&=r^2
\end{align*}
Die Tantentengleichung ist also die Kreisgleichung, in der man
eine Kopie von $\vec p$ durch den Berührpunkt $\vec p_0$ ersetzt
hat.

%
% Tangente/Tangentialebene von einem Punkt aus
%
\subsection{Tangente von einem Punkt an einen Kreis}
Es sind die Gleichungen der Tangenten an einen Kreis $K(M,r)$ zu finden,
welche durch den Punkt $P$ gehen.
Die klassische Konstruktion der
Elementargeometrie verlangt, dass über der Strecke $MP$ der
Thaleskreis gezeichnet wird, die Schnittpunkte des Thaleskreis mit
$K(M,r)$ sind die Berührpunkte der Tangenten.

Die Tangenten sollen jetzt aber auf rein algebraische Art gefunden
werden.
Das Problem wäre im wesentlich gelöst, wenn der Berührpunkt $P_0$
mit Ortsvektor
$\vec p_0$ bekannt wäre.
Dieser muss natürlich auf dem Kreis liegen,
also
\[
(\vec p_0-\vec m)^2=r^2.
\]
Ausserdem muss der Punkt $P$ auf der Tangente im Punkt $P_0$ liegen,
also muss $MP_0$ senkrecht auf $PP_0$ stehen:
\begin{equation}
(\vec p_0-\vec m)\cdot(\vec p-\vec p_0)=0
\label{thalesbedingung}
\end{equation}
Damit haben wir zwei Gleichungen für die beiden unbekannten Koordinaten
von $P_0$ gefunden, im Allgemeinen werden sie die Bestimmung von $\vec p_0$
ermöglichen (Ausnahmen: $P$ im Inneren des Kreises).

Zur Lösung multiplizieren wir die zweite Gleichung noch aus
\begin{align*}
\vec p_0^2-\vec m\cdot\vec p_0-\vec p\cdot\vec p_0+\vec m\cdot\vec p&=0
\\
\vec p_0^2-\vec p_0\cdot (\vec m+\vec p)+\vec m\cdot\vec p&=0
\\
\vec p_0^2-2\vec p_0\cdot \left(\frac{\vec m+\vec p}{2}\right)+\vec m\cdot\vec p&=0
\\
\vec p_0^2-2\vec p_0\cdot \left(\frac{\vec m+\vec p}{2}\right)+
\left(\frac{\vec m+\vec p}2\right)^2
-\left(\frac{\vec m+\vec p}2\right)^2
+\vec m\cdot\vec p&=0
\\
\left(\vec p_0- \frac{\vec m+\vec p}{2}\right)^2
&=
\left(\frac{\vec m+\vec p}2\right)^2
-\vec m\cdot\vec p
\\
\left(\vec p_0- \frac{\vec m+\vec p}{2}\right)^2
&=
\left(\frac{\vec m-\vec p}2\right)^2
\end{align*}
Dies ist wieder ein Kreis mit Mittelpunkt $(\vec m+\vec p)/2$ und
Radius $|\vec m-\vec p|/2$, also wieder ein Thaleskreis.
Das ist natürlich nicht überraschend, die Bedingung (\ref{thalesbedingung})
ist ja nichts anderes als die Definition des Thaleskreises.
Die algebraische Rechnung macht also nichts anderes als die klassische
Konstruktion.

%
%  Reflexion eines Lichtstrahls und Ray-Tracing
%
\subsection{Reflexion eines Lichtstrahls}
\begin{figure}
\begin{center}
\includegraphics[width=1\hsize]{graphics/raytracing}
\end{center}
\caption{Mit Ray Tracing erzeugtes computergeneriertes Bild\label{raytracing}}
\end{figure}
Computergraphik-Effekte sind aus modernen Filmen nicht mehr wegzudenken.
Ganze Spielfilme wurden schon vollständig im Computer erzeugt.
Wie können die Bilder so realistisch wirken?

In Abschnitt \ref{spiegelung} haben wir gelernt, wie ein Vektor
gespiegelt wird.
Wir können also den Lichtstrahl, der in das Auge
des Beobachters einer Szene fällt, zurückverfolgen und seine Reflektion
an jeder beliebigen reflektierenden Fläche der Szene berechnen, bis wir bei
einer Lichtquelle oder einem nicht reflektierenden Objekt ankommen.
Die an diesem Punkt abgestrahlte Farbe ist dan jene, die der Beobachter
wahrnimmt.
Dieses Verfahren nennt man Ray Tracing.
Offenbar ist es sehr
aufwendig, denn Lichtstrahlen können nicht nur reflektiert, sondern auch
gestreut werden, und sie können zum Beispiel durch Nebel abgeschwächt
werden.
Die Berechnung hochauflösender Szenen ist daher sehr aufwendig,
die Herstellung von CG-Filmen in Spielfilm-Länge, wie Pixar sie beispielsweise
produziert, benötigt die Rechenleistung grosser Computer-Cluster.

Die Berechnung der Reflexion an einer Kugel wie in Abbildung \ref{raytracing}
erfolgt nach folgendem Algorithmus:
\begin{enumerate}
\item Berechne den Durchstosspunkt des Lichtstrahles mit der reflektierenden
Kugeloberfläche wie in \ref{durchstosspunktkugel} beschrieben.
\item Berechnet die Normale im Durchstosspunkt gemäss Satz \ref{kugeltangentialebene}.
\item Berechne die gespiegelte Gerade gemäss Abschnitt \ref{spiegelung}.
\end{enumerate}
Dieses Verfahren wird jedoch nicht nur für die Computergraphik verwendet,
sondern auch in der Optik.
Spiegelteleskope bestehen aus gekrümmten Spiegeln,
durch Berechnung der Strahlen kann man erfahren, welche Abbildungsqualität
man von dem Teleskop erwarten kann.


%
% abbildungen.tex -- Lineare Abbildungen
%
% (c) 2018 Prof Dr Andreas Müller, Hochschule Rapperswil
%
\section{Lineare Abbildungen%
\label{skript:section:lineare abbildungen}}
Wie im vorangegangenen Abschnitt verwenden wir im folgenden eine Basis
$\mathcal{B}=\{\vec{b}_1,\dots,\vec{b}_n\}$, die Vektoren $\vec{v}_i$
sind linear unabhängig, und jeder Vektor lässt sich als Linearkombination
dieser Basisvektoren schreiben.

%
% Affine Abbildungen
%
\subsection{Affine und lineare Abbildungen}
Wenn man von einem Problem einen Plan machen, dann darf die Perspektive
keine Rolle spielen.
Der Plan darf keine für das Problem wesentlichen Eigenschaften stören.
Auf ein mathematisches Problem übertragen müssen wir in der Lage sein,
Abbildungen auf das Problem anzuwenden, welche helfen können, das
Problem zu lösen.
Diese Abbildungen dürfen die wesentlichen Eigenschaften der untersuchten
Objekte nicht verändern.

Die dem Abschnitt~\ref{skript:koordinaten} vorangestellten Axiome
sprechen von Geraden, Ebenen und Parallelen als den primären geometrischen
Objekten.
Zulässige Abbildungen dürfen also Ebenen und Geraden nicht zerstören.
Wenn aber Geraden auf Geraden abgebildet werden, dann werden Geraden,
die sich nicht schneiden, auf Geraden abgebildet, die sich nicht schneiden,
Parallelität ist also automatisch erhalten.

\begin{definition}
Eine Abbildung, die Ebene, Geraden und Parallelität erhält,
heisst {\em affine Abbildung}
\end{definition}
\index{affine Abbildung}

Aus den Axiomen haben wir die algebraischen Eigenschaften von Ortsvektoren
konstruiert.
Affine Abbildungen müssen also verträglich sein mit den algebraischen
Operationen.
Um die Geometrie mit Vektoren auszudrücken, mussten wir ausserdem
einen ausgezeichneten Punkt $O$ haben.
Dieser muss natürlich unter Abbildungen, die uns interessieren
ebenfalls erhalten bleiben.
Eine affine Abbildung $\varphi$ muss also folgende Regeln erfüllen:
\begin{align*}
\varphi(O)&=O
\\
\varphi(\lambda\vec{p})&=\lambda\varphi(\vec{p})
\\
\varphi(\vec{u}+\vec{v})&=\varphi(\vec{u}) + \varphi(\vec{v})
\end{align*}

\begin{definition}
Eine Abbildung $\varphi\colon \mathbb R^n \to \mathbb R^m$ mit den
Eigenschaften
\begin{align*}
\varphi(\lambda\vec{p})&=\lambda\varphi(\vec{p})
&&\text{und}&
\varphi(\vec{u}+\vec{v})&=\varphi(\vec{u}) + \varphi(\vec{v})
\end{align*}
heisst {\em linear}.
\end{definition}

%
% Beschreibung linearer Abbildungen mit Matrizen
%
\subsection{Beschreibung linearer Abbildungen mit Matrizen}
Wir wollen jetzt lineare Abbildungen der Ebene und des dreidimensionalen
Raumes mit Hilfe einer Basis genauer beschreiben.
Sei also eine Basis $\mathcal{B}=\{\vec{b}_1,\dots,\vec{b}_n\}$ 
gegeben und eine lineare Abbildung $\varphi$.

\subsubsection{Matrix einer linearen Abbildung}
Jeder beliebige Vektor $\vec{x}$ kann als Linearkombination
\[
\vec{x}
=
x_1\vec{b}_1
+\dots+
x_n\vec{b}_n
\]
der Vektoren $\vec{b}_i$ geschrieben werden.
Der Bildvektor $\varphi(\vec{x})$ kann mit den Linearitätseigenschaften
vereinfacht werden:
\[
\varphi(\vec{x})
=
\varphi(
x_1\vec{b}_1
+\dots+
x_n\vec{b}_n
)
=
x_1\varphi(\vec{b}_1)
+\dots+
x_n\varphi(\vec{b}_n).
\]
Die lineare Abbildung ist also vollständig durch die Bilder der Basisvektoren
$\varphi(\vec{b}_i)$ festgelegt.

In der Basis werden die Vektoren $\vec{b}_i$ durch die Standardbasisvektoren
\[
\vec{e}_1 = \begin{pmatrix}1\\0\\\vdots\\0\end{pmatrix},\quad
\vec{e}_2 = \begin{pmatrix}0\\1\\\vdots\\0\end{pmatrix},
\quad\dots,\quad
\vec{e}_n = \begin{pmatrix}0\\0\\\vdots\\1\end{pmatrix}
\]
dargestellt.
Auch die Bildvektoren können in der Basis $\mathcal{B}$ ausgedrückt werden,
wir schreiben die Bilder als Spaltenvektoren
\[
\vec{a}_i
=
\varphi(\vec{b}_i)
=
\begin{pmatrix}
a_{1i}\\a_{2i}\\\vdots\\a_{ni}
\end{pmatrix}.
\]
Der Bildvektor $\varphi(\vec{x})$ ist daher die Linearkombination
\[
\varphi(\vec{x})
=
x_1
\begin{pmatrix}
a_{11}\\a_{21}\\\vdots\\a_{n1}
\end{pmatrix}
+
x_2
\begin{pmatrix}
a_{12}\\a_{22}\\\vdots\\a_{n2}
\end{pmatrix}
+
\dots
+
x_n
\begin{pmatrix}
a_{1n}\\a_{2n}\\\vdots\\a_{nn}
\end{pmatrix}
=
\begin{pmatrix}
a_{11}&a_{12}&\dots &a_{1n}\\
a_{21}&a_{22}&\dots &a_{2n}\\
\vdots&\vdots&\ddots&\vdots\\
a_{21}&a_{22}&\dots &a_{2n}
\end{pmatrix}
\begin{pmatrix}
x_1\\x_2\\\vdots\\x_n
\end{pmatrix}
\]
Die Spalten der Matrix $A$ sind die Koordinatenvektoren der Bilder
$\varphi(\vec{b}_i)$.
Wir fassen diese Resultate wie folgt zusammen.

\begin{satz}
\label{satz:affin:bilderderstandardbasisvektoren}
Eine lineare Abbildung wird durch die Matrix $A$ vollständig beschrieben,
die Spalten enthalten die Bilder der Standardbasisvektoren.
\end{satz}

\subsubsection{Beispiel: Vertauschung der Achsen}
Wir suchen die Matrix der linearen Abbildung, die die beiden Achsrichtungen
$\vec{e}_1$ und $\vec{e}_2$ verauscht.
In den Spalten von $A$ stehen die Bilder der Standardbasisvektoren,
daher muss in der ersten Spalte das Bild von $\vec{e}_1$ stehen, also
der Vektor $\vec{e}_2$.
In der zweiten Spalte muss dagegen $\vec{e}_1$ stehen.
Die gesuchte Matrix ist daher
\[
A=\begin{pmatrix}0&1\\1&0\end{pmatrix}.
\]

\subsubsection{Beispiel: Matrix einer Spiegelung}
\begin{figure}
\centering
\includegraphics{3/images/spiegelung.pdf}
\caption{Die Spiegelung an der Geraden $g$ bildet den Vektor
$\vec{e}_1$ auf $\vec{e}_2$ ab und umgekehrt.
\label{skript:affin:spiegelung}}
\end{figure}
Wir suchen die Matrix einer Spiegelung der Ebene an der Geraden durch
die Punkte $O$ und $(2,2)$.
Nach Satz~\label{satz:affin:bilderderstandardbasisvektoren} müssen wir 
die Bilder der Standardbasisvektoren bestimmen.
Aus Abbildung~\ref{skript:affin:spiegelung} kann man ablesen, dass die
Spiegelung den Vektor $\vec{e}_1$ auf $\vec{e}_2$ abbildet und umgekehrt.
Daraus kann man jetzt die Matrix $S$ der Spiegelung zusammensetzen,
sie enthält die Bilder der Standardbasisvektoren:
\[
\vec{e}_1\leftrightarrow\vec{e}_2
\qquad\Leftrightarrow\qquad
\begin{pmatrix}1\\0\end{pmatrix}
\leftrightarrow
\begin{pmatrix}0\\1\end{pmatrix}
\qquad\Leftrightarrow\qquad
S
=
\begin{pmatrix}0&1\\1&0\end{pmatrix}.
\]
Wir kontrollieren dieses Resultate, indem wir berechnen wie ein Vektor
auf der Geraden und senkrecht dazu abgebildet wird:
\begin{align*}
S
\begin{pmatrix}1\\1\end{pmatrix}
&=
\begin{pmatrix}0&1\\1&0\end{pmatrix}
\begin{pmatrix}1\\1\end{pmatrix}
=
\begin{pmatrix}1\\1\end{pmatrix}
\\
S
\begin{pmatrix}1\\-1\end{pmatrix}
&=
\begin{pmatrix}0&1\\1&0\end{pmatrix}
\begin{pmatrix}1\\-1\end{pmatrix}
=
\begin{pmatrix}-1\\1\end{pmatrix}
=
-
\begin{pmatrix}1\\-1\end{pmatrix}.
\end{align*}
Der Vektor in der zweiten Zeile steht senkrecht auf der Geraden $g$ 
und wird durch die Spiegelung mit $-1$ multipliziert.

Natürlich ist dies genau die gleiche Matrix wie im vorangegangenen
Beispiel, denn die Spiegelung vertauscht natürlich die beiden
Basisvektoren.

\subsubsection{Beispiel: Matrix einer Drehung}
\begin{figure}
\centering
\includegraphics{3/images/drehung.pdf}
\caption{Drehung der Ebene um den Winkel $\alpha$.
Die Drehmatrix $R$ besteht aus den Bildern der Standardbasisvektoren.
\label{skript:affin:drehung}}
\end{figure}
Gesucht ist die Matrix $R$ einer Drehung um den Winkel $\alpha$.
Aus Abbildung~\ref{skript:affin:drehung} liest man die Bilder der
Standardbasisvektoren ab:
\[
R\colon \vec{e}_1 \mapsto \begin{pmatrix}\cos\alpha\\\sin\alpha\end{pmatrix},
\qquad
R\colon \vec{e}_2 \mapsto \begin{pmatrix}-\sin\alpha\\\cos\alpha\end{pmatrix}.
\]
Daraus kann man die Drehmatrix
\[
R=\begin{pmatrix}\cos\alpha&-\sin\alpha\\\sin\alpha&\cos\alpha\end{pmatrix}
\]
zusammensetzen.

%
% Komposition linearer Abbildungen
%
\subsection{Zusammensetzung linearer Abbildungen}
Was für eine Matrix erhält man, wenn man zwei lineare Abbildungen,
je beschrieben durch Matrizen $A$ und $B$ nacheinander ausführt?
Diese Situation kann schematisch dargestellt werden durch das Diagramm
\begin{equation}
\definecolor{darkgreen}{rgb}{0,0.6,0}
\xymatrix{
{\color{red}\mathbb R^n } \ar[r]^{A} \ar@/_20pt/[rr]_{C}
&{\color{darkgreen}\mathbb R^m} \ar[r]^{B}
&{{\color{blue}\mathbb R^l}.}
}
\label{skript:affin:komposition:diagramm}
\end{equation}
Um die Matrix dieser Zusammensetzung zu finden muss man herausfinden,
auf welche Vektoren die Standardbasisvektoren abgebildet werden.
Die erste Abbildung mit Matrix $A$ bildet $\color{red}\vec{e}_i$,
$i=1,\dots,n$, auf die $i$-te Spalte von $A$ ab, die wir mit
\begin{equation}
\definecolor{darkgreen}{rgb}{0,0.6,0}
{\color{darkgreen}\vec{a}_i}
=\begin{pmatrix}
{\color{darkgreen}a_{1i}}\\\vdots\\{\color{darkgreen}a_{mi}}
\end{pmatrix}
=
{ \color{darkgreen} a_{1i} \vec{e}_1}
+\dots +
{ \color{darkgreen} a_{mi} \vec{e}_m}
\label{skript:affin:komposition:a}
\end{equation}
bezeichnen.
Die zweite Abbildung bildet die Standardbasisvektoren in
$\definecolor{darkgreen}{rgb}{0,0.6,0}
\color{darkgreen}\mathbb R^m$ auf die Spalten
von $B$ ab.
Die Zusammensetzung \eqref{skript:affin:komposition:diagramm}
bildet den Vektor 
$\definecolor{darkgreen}{rgb}{0,0.6,0}
\color{darkgreen}
\vec{a}_i$ in \eqref{skript:affin:komposition:a} 
ab auf 
{%
\definecolor{darkgreen}{rgb}{0,0.6,0}%
\begin{align*}
B{\color{darkgreen}\vec{a_i}}
&=
{\color{blue}\vec{b}_1} {\color{darkgreen}a_{1i}}+\dots
+{\color{blue}\vec{b}_m}{\color{darkgreen}a_{mi}}
=
{\color{blue}
\begin{pmatrix}b_{11}\\\vdots\\b_{l1}\end{pmatrix}} {\color{darkgreen}a_{1i}}
+\dots+
{\color{blue}
\begin{pmatrix}b_{1m}\\\vdots\\b_{lm}\end{pmatrix}} {\color{darkgreen}a_{mi}}
=
{\color{blue}
\begin{pmatrix}
\color{black}
{\color{blue}b_{11}}{\color{darkgreen}a_{1i}}+\dots+{\color{blue}b_{1m}}{\color{darkgreen}a_{mi}}\\
\color{black} \vdots\\
\color{black}
{\color{blue}b_{l1}}{\color{darkgreen}a_{1i}}+\dots+{\color{blue}b_{lm}}{\color{darkgreen}a_{mi}}\\
\end{pmatrix}
}.
\end{align*}}
Dies ist aber auch die $i$-te Spalte der Matrix $C$, bestehend aus den
Komponenten $c_{1i}$ bis $c_{li}$.
Man liest ab
{
\definecolor{darkgreen}{rgb}{0,0.6,0}
\begin{align*}
c_{ji}
&=
{\color{darkgreen}b_{j1}}{\color{blue}a_{1i}} + \dots
	+ {\color{darkgreen}b_{jm}}{\color{blue}a_{mi}}
\\
\begin{pmatrix}
\qquad&\quad&\qquad\\
\qquad&c_{ji}&\qquad\\
\qquad&\quad&\qquad\\
\end{pmatrix}
&=
\begin{pmatrix}
&\dots&\\
\color{darkgreen}b_{j1}&\color{darkgreen}\dots&\color{darkgreen}b_{jm}\\
&\dots&
\end{pmatrix}
\begin{pmatrix}
&\qquad&\color{blue}a_{1i}&\qquad&\\
&\vdots&\color{blue}\vdots&\vdots&\\
&     &\color{blue}a_{mi}&&
\end{pmatrix}
\end{align*}
}
Dies ist genau die Definition des Matrizen-Produktes aus
Abschnitt~\label{skript:subsection:matrizen}.

\begin{satz}
Die Matrix $C$ der Abbildung zusmmengesetzt aus der Abbildung
mit Matrix $A$ gefolgt von der Abbildung mit Matrix $B$ ist
$C=BA$.
\end{satz}

Man beachte die scheinbar `verkehrte' Reihenfolge der Faktoren.
Man erinnere sich aber daran, dass die die Vektoren, auf die die Matrizen
wirken, rechts von der Matrix hingeschrieben werden.
Die Wirkung der Matrix $C$ auf einen Vektor $v$ wird $Cv$ geschrieben.
Dies soll das gleiche sein, wie wenn zuerst $A$ wirkt, was den Bildvektor
$Av$ ergibt, und auf diesen Vektor wirkt jetzt $B$, geschrieben $B(Av)=BAv$.

%
% Basiswechsel
%
\subsection{Basiswechsel}
Eine lineare Abbildung wird in einer Basis $\color{red}\mathcal{B}$ durch
eine Matrix $\color{red}A$ beschrieben, in den Spalten stehen die Bilder
der Standardbasisvektoren.
Sowohl die Standardbasisvektoren wie auch die Komponenten in den
Spalten sind von der Basis abhängig, die Abbildungsmatrix $\color{red}A$
ist also ebenfalls basisabhängig.
Damit stellt sich die Frage, wie sich die Matrix ändert, wenn
man die Basis wechselt.

Wir möchten jetzt eine neue Basis $\color{blue}\mathcal{C}$ verwenden, die
Basistransformationsmatrix $T$ sei gegeben.
Die Matrix $T$ rechnet die Koordinaten eines Vektors in der Basis
$\color{red}\mathcal{B}$ um in Koordinaten in der Basis
$\color{blue}\mathcal{C}$.
Um das Bild eines Vektors $\vec{u}$ in der Basis $\color{blue}\mathcal{C}$ 
zu berechnen, muss man ihn erst in $\color{red}\mathcal{B}$ umrechnen, was
mit der inversen Matrix $T^{-1}$ geschehen kann.
Dann erst kann man $A$ anwenden, erhält dann aber einen Vektor
in der Basis $\color{red}\mathcal{B}$, man muss ihn also erst wieder
mit $T$ in die Basis $\color{blue}\mathcal{C}$ umrechnen.
Alles zusammen ist in der Basis $\mathcal{C}$ der Bildvektor 
$T{\color{red}A}T^{-1}u$.
Wir fassen das Resultat zusammen im folgenden Satz.

\begin{satz}
\label{skript:affin:basiswechsel:satz}
Sei $T$ die Transformationsmatrix, die Koordinaten von der Basis
$\color{red}\mathcal{B}$ in die Basis $\color{blue}\mathcal{C}$ umrechnet.
Die lineare Abbildung, die in der Basis $\color{red}\mathcal{B}$ durch die
Matrix $A$ beschrieben wird, wird in der Basis $\color{blue}\mathcal{C}$ durch
die Matrix
\[
{\color{blue}A'}=T{\color{red}A}T^{-1}
\]
beschrieben.
\end{satz}

Diese Situation kann auch im folgenden Diagramm
\begin{center}
\begin{tikzpicture}[>=latex,thick]
\fill[color=white] (-7.5,-1) rectangle (7.5,1);
\fill[color=red!10]  (-7.3,0.3) rectangle (0,1.7);
\fill[color=red!20]  (-1.7, 0.3) rectangle (1.7, 1.7);
\fill[color=blue!10]  (-7.3,-0.3) rectangle (0,-1.7);
\fill[color=blue!20] (-1.7,-0.3) rectangle (1.7,-1.7);
\node[color=blue] at (-1,-1) {$\mathbb R^n$};
\node[color=blue] at ( 1,-1) {$\mathbb R^n$};
\node[color=red] at (-1, 1) {$\mathbb R^n$};
\node[color=red] at ( 1, 1) {$\mathbb R^n$};
\node at (-5, 1) {Basis ${\color{red}\mathcal{B}}=\{\vec{b}_1,\dots,\vec{b}_n\}$:};
\node at (-5,-1) {Basis ${\color{red}\mathcal{C}}=\{\vec{c}_1,\dots,\vec{c}_n\}$:};
\draw[->,color=red] (-0.7,1)--(0.7,1);
\node[color=red] at (0,1) [above] {$A$};
\draw[->,color=blue] (-0.7,-1)--(0.7,-1);
\node[color=blue] at (0,-1) [above] {$A'$};
\draw[->] (-1,0.7)--(-1,-0.7);
\draw[->] ( 1,0.7)--( 1,-0.7);
\node at ( 1,0) [right] {$T$};
\node at (-1,0) [right] {$T$};
\draw[<-] (-1.3, 0.8) arc (150:210:1.6);
\node at (-1.5,0) [left] {$T^{-1}$};
\end{tikzpicture}
\end{center}
illustriert werden.
Die Abbildung $\color{blue}A'$ führt vom Vektorraum unten links zum
Vektorraum unten rechts.
Dieser Weg ist gleichbedeutend mit dem Umweg über die beiden Vektorräume
in der oberen Zeile.
Um von unten links nach oben links zu kommen, muss man die
Transformationsmatrix $T^{-1}$ verwenden.
Zusammengesetzt wird der Umweg durch $T{\color{red}A}T^{-1}$ beschrieben,
woraus wieder die Aussage des Satzes folgt.

\begin{beispiel}
In einem früheren Beispiel haben wir die Spiegelung an der $45^\circ$-Geraden
mit Hilfe der Matrix 
\[
S=\begin{pmatrix}0&1\\1&0\end{pmatrix}
\]
beschrieben.
Jetzt möchten wir ein Koordinatensystem verwenden, welches gegenüber
dem ursprünglichen um $45^\circ$ verdreht ist.
Wir möchten also die Basisvektoren
\[
\vec{c}_1 = \frac{1}{\sqrt{2}} \begin{pmatrix}1\\1\end{pmatrix}
\qquad\text{und}\qquad
\vec{c}_2 = \frac{1}{\sqrt{2}} \begin{pmatrix}-1\\1\end{pmatrix}
\]
verwenden und müssen daher die Transformationsmatrix $T$ für den
Basiswechsel vonder Standardbasis in die neue Basis
$\mathcal{C}=\{\vec{c}_1,\vec{c}_2\}$ ermitteln.
Wir verwenden zur Bestimmung von $T$ das
Tableau~\eqref{skript:affin:basistransformation-tableau}, also
\[
\begin{tabular}{|>{$}c<{$}>{$}c<{$}|>{$}c<{$}>{$}c<{$}|}
\hline
\frac1{\sqrt{2}}&-\frac1{\sqrt{2}}&1&0\\
\frac1{\sqrt{2}}&\frac1{\sqrt{2}}&0&1\\
\hline
\end{tabular}
\quad
\rightarrow
\quad
\begin{tabular}{|>{$}c<{$}>{$}c<{$}|>{$}c<{$}>{$}c<{$}|}
\hline
1&0&\frac1{\sqrt{2}}&\frac1{\sqrt{2}}\\
0&1&-\frac1{\sqrt{2}}&\frac1{\sqrt{2}}\\
\hline
\end{tabular}
\]
die Transformationsmatrix ist damit
\[
T
=
\frac1{\sqrt{2}}\begin{pmatrix}
1&1\\
-1&1
\end{pmatrix}
\quad\text{und}\quad
T^{-1}
=
\frac1{\sqrt{2}}\begin{pmatrix}
1&-1\\
1&1
\end{pmatrix}.
\]
Damit können wir jetzt die Abbildungsmatrix im neuen Koordinatensystem
berechnen:
\[
S'
=
TST^{-1}
=
\frac1{\sqrt{2}}\begin{pmatrix}
1&1\\
-1&1
\end{pmatrix}
\begin{pmatrix}
0&1\\1&0
\end{pmatrix}
\frac1{\sqrt{2}}\begin{pmatrix}
1&-1\\
1&1
\end{pmatrix}
=
\frac12
\begin{pmatrix}
1&1\\1&-1
\end{pmatrix}
\begin{pmatrix}
1&-1\\1&1
\end{pmatrix}
=
\frac12
\begin{pmatrix}
2&0\\0&-2
\end{pmatrix}
=
\begin{pmatrix}
1&0\\0&-1
\end{pmatrix}
\]
Im neuen Koordinatensystem wird die Abbildung durch die Matrix $S'$
beschrieben.
Diese Matrix besagt, dass der erste Basisvektor nicht verändert wird,
während der zweite mit $-1$ multipliziert wird.
Dies beschreibt genau die Spiegelung an der $45^\circ$-Geraden:
Ein Vektor auf der Geraden bleibt unverändert, während ein Vektor
senkrecht dazu mit $-1$ multipliziert wird.
\end{beispiel}

Der Satz~\ref{skript:affin:basiswechsel:satz} und das nachfolgende 
Beispiel zeigen ein weiteres Mal, dass es auf die Reihenfolge der
Faktoren im Matrizenprodukt ankommt.
Könnte man $A$ und $T$ vertauschen, wäre $TAT^{-1}=ATT^{-1}=AI=A$,
die Abbildungsmatrix würde als gar nicht von der Basis abhängen.

%%
%% Bildraum und Kern
%%
%\subsection{Bildraum und Kern\label{subsection:affin:bildraumundkern}}
%In Kapitel~\ref{chapter-lingl} wurde die Lösungsmenge eines linearen
%Gleichungssystems mit der Matrix $A$ untersucht.
%Inzwischen haben wir einer Matrix $A$ auch eine geometrische Bedeutung
%als Abbildungsmatrix gegeben.
%Die Frage der Lösbarkeit von Gleichungssystemen und der Begriff des
%Ranges sollten daher auch eine geometrische Aussage übersetzt
%werden können, was wir in diesem Abschnitt tun wollen.
%
%\subsubsection{Bildraum}
%Die Matrix $A$ einer linearen Abbildung $\varphi$ enthält in den Spalten
%die Bilder der Standardbasisvektoren.
%Vektoren, die als Bilder der Abbildung $\varphi$ auftreten können,
%müssen Linearekombinationen der Spaltenvektoren von $A$ sein.
%\begin{definition}
%Die {\em Bildmenge} $\operatorname{im}\varphi$ einer linearen Abbildung
%$\varphi$ ist der Vektorraum
%\[
%\operatorname{im}\varphi
%=
%\{\varphi(v)\;|\;v\in\mathbb R^n\}.
%\]
%\end{definition}
%
%Die Bildmenge $\operatorname{im}\varphi$ der Abbildung $\varphi$ ist
%daher nichts anderes als die Menge aller möglichen Linearkombinationen
%von Spaltenvektoren.
%Diese haben wir Seite \pageref{skript:affin:koordinaten:aufgespannt}
%als den erzeugten Raum der Spaltenvektoren kennengelernt.
%Wir fassen zusammen:
%\begin{satz}
%Die Bildmenge einer linearen Abbildung $\varphi$ ist der von den
%Spaltenvektoren der zugehörigen Abbildungsmatrix $A$ aufgespannte Raum:
%\[
%\operatorname{im}\varphi
%=
%\langle\vec{a}_1,\dots,\vec{a}_n\rangle
%=
%\langle\mathcal{A}\rangle,
%\]
%wobei $\mathcal{A}$ für die Menge der Spaltenvektoren von $A$ geschrieben
%haben.
%\end{satz}
%
%\subsubsection{Lösbarkeit von Gleichungssystemen}
%Ein Gleichungssystem mit Koeffizientenmatrix $A$ und rechter Seite $b$
%ist lösbar genau dann, wenn es einen Spaltenvektor $x$ gibt, derart 
%dass $Ax=b$.
%Das Gleichungssystem ist also genau dann lösbar, wenn die rechte Seite
%im Bildraum von $A$ liegt.
%
%\begin{beispiel}
%Ist der Vektor
%\[
%b=\begin{pmatrix}4\\2\\4\end{pmatrix}
%\qquad\text{im Bildraum der Abbildung mit Matrix}\qquad
%A=
%\begin{pmatrix}
%  -43& -26&  56\\
%  -22& -12&  28\\
%  -44& -26&  57
%\end{pmatrix}\text{?}
%\]
%Die Frage ist gleichbedeutend damit, ob das Gleichungsystem mit
%Koeffizientenmatrix $A$ und rechter Seite $b$ lösbar ist.
%Das Gauss-Tableau
%\[
%\begin{tabular}{|>{$}c<{$}>{$}c<{$}>{$}c<{$}|>{$}c<{$}|}
%\hline
%  -43& -26&  56&4\\
%  -22& -12&  28&2\\
%  -44& -26&  57&4\\
%\hline
%\end{tabular}
%\quad\rightarrow\quad
%\begin{tabular}{|>{$}c<{$}>{$}c<{$}>{$}c<{$}|>{$}c<{$}|}
%\hline
%   1&  0& -1&  0\\
%   0&  1& -\frac12&   0\\
%\hdashline
%   0&  0&  0&  1\\
%\hline
%\end{tabular}
%\]
%zeigt, dass dies wegen der $1$ in der rechten unteren Ecke nicht möglich
%ist.
%\end{beispiel}
%
%\subsubsection{Dimension des Bildraumes}
%Der Bildraum ist der von den Spaltenvektoren der Abbildungsmatrix
%aufgespannte Raum.
%Es kann durchaus sein, dass nicht alle Spaltenvektoren linear unabhängig
%sind, die Dimension des Bildraumes kann also auch kleiner sein also die
%Anzahl der Spaltenvektoren.
%
%\begin{aufgabe}
%Wie gross ist die Dimension des Bildraumes einer linearen Abbildung
%$\varphi$ mit Abbildungsmatrix $A$?
%\end{aufgabe}
%Die Dimension eines Vektorraumes ist die maximale Anzahl linear
%unabhängiger Vektoren einer Basis.
%Da die Spaltenvektoren von $A$ alles erzeugen, müssen wir nur noch
%ein paar Vektoren weglassen, welche nicht linear unabhängig sind von
%den anderen.
%In Kapitel~\ref{chapter-lingl} haben wir gelernt, dass der Rang der
%Matrix $A$ genau die Anzahl der linear unabhängigen Spalten ist.
%er ist daher auch die Dimension des Bildraumes.
%
%\begin{satz}
%Ist $A$ die Abbildungsmatrix einer linearen Abbildung $\varphi$, dann
%gilt
%\[
%\dim\operatorname{im}\varphi = \operatorname{Rang}A.
%\]
%\end{satz}
%
%\subsubsection{Nullmenge}
%\begin{definition}
%Sei $\varphi$ eine lineare Abbildung.
%Die Menge
%\[
%\operatorname{ker}\varphi = \{\vec{v}\;|\; \varphi(\vec{v}) = 0\}
%\]
%heisst {\em Nullmenge}, {\em Nullraum} oder {\em Kern} der Abbildung 
%$\varphi$.
%\end{definition}
%
%Der Kern von $\varphi$ ist ein Vektorraum, denn wenn zwei Vektoren
%$\vec{u},\vec{v}\in\ker\varphi$ im Kern sind, dann gilt dies
%auch für die Summe und für die Vielfachen:
%\[
%\begin{aligned}
%\varphi(\vec{u}+\vec{v})
%&=
%\varphi(\vec{u})+\varphi(\vec{v})
%=
%0+0=0
%&&\Rightarrow&
%\vec{u}+\vec{v}&\in\ker\varphi
%\\
%\varphi(\lambda\vec{u})
%&=
%\lambda\varphi(\vec{u}) = \lambda\cdot 0=0
%&&\Rightarrow&
%\lambda\vec{u}&\in\ker\varphi
%\end{aligned}
%\]
%Damit stellt sich automatisch die folgende Aufgabe.
%
%\begin{aufgabe}
%Sei $\varphi$ eine lineare Abbildung mit Abbildungsmatrix $A$.
%Man finde eine Basis von $\ker\varphi$.
%\end{aufgabe}
%
%Der Kern besteht aus denjenigen Vektoren $\vec{x}$, die $A\vec{x}=0$
%erfüllen.
%Der Kern von $\varphi$ ist daher nichts anderes als die Lösungsmenge
%des Gleichungssystems $Ax=0$, welche man mit dem Gauss-Algorithmus
%bestimmen kann, wie in Abschnitt~\ref{section:loesungsmenge} gezeigt
%wurde.
%Dort wurde auch gezeigt, dass die Anzahl der frei wählbaren Variablen
%auch die Anzahl der Basisvektoren der Lösungsmenge ist.
%Wenn $A$ eine $m\times n$-Matrix ist mit $\operatorname{Rang}A=r$,
%dann ist die Anzahl der frei wählbaren Variablen $n-r$, es gilt
%also
%\[
%\dim\ker \varphi = n -\operatorname{Rang}A.
%\]
%
%\begin{beispiel}
%Man finde eine Basis des Kernes der linearen Abbildung mit Matrix
%\[
%A=\begin{pmatrix}
%    9& -31& -26&   0\\
%    2& -10& -12& -28\\
%    8& -29& -26& -13
%\end{pmatrix}
%\]
%\smallskip
%
%{\parindent0pt Das} zugehörige Tableau ist
%\begin{equation}
%\begin{tabular}{|>{$}c<{$}>{$}c<{$}>{$}c<{$}>{$}c<{$}|>{$}c<{$}|}
%\hline
%    9& -31& -26&   0&0\\
%    2& -10& -12& -28&0\\
%    8& -29& -26& -13&0\\
%\hline
%\end{tabular}
%\quad\rightarrow\quad
%\begin{tabular}{|>{$}c<{$}>{$}c<{$}>{$}c<{$}>{$}c<{$}|>{$}c<{$}|}
%\hline
%    1&   0&   4&  31&0\\
%    0&   1&   2&   9&0\\
%\hdashline
%    0&   0&   0&   0&0\\
%\hline
%\end{tabular}
%\label{skript:affin:kern:tableau}
%\end{equation}
%Die letzten zwei Variablen sind frei wählbar, wir nennen sie $x_3$ und $x_4$,
%dann wird die Lösungsmenge
%\[
%\ker\varphi
%=
%\mathbb L
%=
%\left\{
%\left.
%x_3
%\begin{pmatrix}
%-4\\-2\\1\\0
%\end{pmatrix}
%+
%x_4
%\begin{pmatrix}
%-31\\-9\\0\\1
%\end{pmatrix}
%\;
%\right|
%\;
%x_3,x_4\in\mathbb R
%\right\}.
%\]
%Der Kern von $\varphi$ ist also ein zweidimensionaler Raum mit der Basis
%\[
%\mathcal{B} = \left\{
%\begin{pmatrix}
%-4\\-2\\1\\0
%\end{pmatrix},
%\begin{pmatrix}
%-31\\-9\\0\\1
%\end{pmatrix}
%\right\}.
%\]
%Das Tableau \eqref{skript:affin:kern:tableau} zeigt auch, dass
%$\operatorname{Rang}A=2$,
%passend zur Dimension $\dim\ker\varphi = n - \operatorname{Rang}A = 4-2=2$ von
%$\ker\varphi$.
%\end{beispiel}






%
% ueberbestimmt.tex
%
% (c) 2018 Prof Dr Andreas Müller, Hochschule Rapperswil
%
\section{Überbestimmte Gleichungssysteme -- ``Least Squares''%
\label{section:ueberbestimmt}}
\index{Gleichungssystem!ueberbestimmtes@\überbestimmtes}
Bei einem überbestimmten Gleichungssystem, also einem Gleichungssystem
mit mehr Gleichungen als Unbekannten, kann man im Allgemeinen nicht davon
ausgehen, dass es überhaupt eine Lösung gibt.
Ein solches Gleichungssystem hat die Form
\[
A v= b,
\]
wobei $A$ eine Matrix ist, die mehr Zeilen als Spalten hat.

Betrachten wir als Beispiel das Gleichungssystem
\[
\begin{pmatrix}
 6&14\\
12&28\\
 9&21
\end{pmatrix}
\begin{pmatrix}v_1\\v_2\end{pmatrix}
=
\begin{pmatrix}2\\4\\4\end{pmatrix}.
\]
Die Bildmenge der Abbildungsmatrix $A$ beschreibt eine Ebene durch
den Nullpunkt mit den Parametern $v_1$ und $v_2$.
Das Gleichungssystem sucht also nach den Parametern $v_1$ und $v_2$
des Punktes auf der Ebene mit den Koordinaten $(2,4,4)$.
Solche Parameterwerte müssen aber gar nicht existieren, es ist
ja nicht klar, dass der Punkt überhaupt auf der Ebene liegt.

Das Beispiel zeigt aber noch eine weitere Schwierigkeit.
Die beiden Spaltenvektoren von $A$ sind nämlich linear abhängig,
die zweite Spalte ist das $\frac{7}{3}$-fache der ersten Spalte.
Der Rang der Matrix ist daher $1$, die Bildmenge ist sogar nur
eine Gerade.
Man erkennt zum Beispiel, dass der Punkt $(2,4,3)$ auf der Geraden
liegt.
Der gegebene Punkt $(2,4,4)$ kann aber nicht auf der Geraden liegen.
Das Gleichungssystem ist nicht lösbar.

\rhead{Überbestimmte Gleichungssysteme}
%
% Lösungen im Sinne des kleinsten Abstandes
%
\subsection{Lösung im Sinne des kleinsten Abstandes}
Das beste, was man erwarten kann, ist ein Vektor $v_0$ so, dass
der Abstand des Punktes $ b$ von der Ebene (Gerade) bestehend
aus allen $Av$ für $v=v_0$ am kleinsten wird.
Dies geschieht natürlich genau dann, wenn der Differenzvektor $b-Av_0$ auf
allen Vektoren von $Av$ senkrecht steht.

Die Menge der Vektoren der Form $Av$ wird von den Spalten von $A$
aufgespannt.
Es genügt also zu testen, ob $b-Av_0$ auf diesen
Vektoren senkrecht steht.
Dazu müssen die Skalarprodukte von
Spalten von $A$ mit dem Vektor $b-Av_0$ verschwinden, oder
\[
A^t(b-Av_0)=0
\quad
\Rightarrow
\quad
A^tAv_0=A^tb
\]
wir haben also ein Gleichungssystem gefunden mit Matrix $A^tA$ und
rechter Seite $A^tb$, welches als Lösung den gesuchten Vektor
$v_0$ hat.
$A^t$ hat so viele Zeilen wie $v$ Komponenten hat, also
handelt es sich um ein Gleichungssystem mit gleich vielen Gleichungen
wie Unbekannten.
Es wird im Allgemeinen eine eindeutig bestimmte Lösung haben.

\begin{satz} Sei $A$ eine $n\times m$ Matrix und $b$ ein $n$-dimensionaler
Vektor.
Eine Lösung im Sinne minimaler quadrierter Abstände
$
(Av-b)\cdot(Av-b)
$
ist Lösung des Gleichungssystems
\[
A^tAv=A^tb
\]
mit $m$ Gleichungen und $m$ Unbekannten.
\end{satz}

\begin{beispiel}
Man finden den Fusspunkt des Lotes vom Punkt $P=(9,10,7)$ auf die Ebene
durch $O$, $A=(8,10,10)$ und $B=(9,13,12)$.

Der Fusspunkt des Lotes ist der Punkt der Ebene, der den geringsten
Abstand zu $P$ hat.
Die Ebenengleichung ist
\[
A\begin{pmatrix}s\\t\end{pmatrix}=
\begin{pmatrix}
 8& 9\\
10&13\\
10&12
\end{pmatrix}
\begin{pmatrix}s\\t\end{pmatrix}.
\]
Gesucht wird die ``beste Lösung'' von
\[
A\begin{pmatrix}s\\t\end{pmatrix}=\begin{pmatrix}9\\10\\7\end{pmatrix}=b
\]
Dazu muss zunächst die Matrix $A^tA$ und der Vektor $A^tb$
berechnet werden.
\[
A^tA=\begin{pmatrix}
264&322\\
322&394
\end{pmatrix}
,\qquad
A^tb=\begin{pmatrix}
242\\295
\end{pmatrix}.
\]
Daraus findet man die Lösung für $s$ und $t$ numerisch zu
\[
\begin{pmatrix}s\\t \end{pmatrix}
=
\begin{pmatrix}
   1.07831\\
  -0.13253
\end{pmatrix}
\]
und durch Einsetzen in die Ebenengleichung den Ortsvektor des Fusspunktes
\[
\vec f = \begin{pmatrix}
   7.4337\\
   9.0602\\
   9.1928
\end{pmatrix}.
\]
Man kann dieses Resultat dadurch kontrollieren, dass man nachrechnet, ob
$\vec p-\vec f$ senkrecht auf beiden Richtungsvektoren der Ebene
steht:
\[
(\vec p-\vec f)^tA=\begin{pmatrix}
  -1.7451\cdot10^{-11}&  -2.1316\cdot 10^{-11}
\end{pmatrix},
\]
im Rahmen der Rechengenauigkeit steht die Differenz also tatsächlich auf
den Richtungsvektoren senkrecht.
\end{beispiel}

%
% Alles in einem Schritt
%
\subsection{``Alles in einem Schritt''}
In Kapitel~\ref{chapter:affin} wurde gezeigt, wie man Probleme der
affinen Geometrie mit nur einer Durchführung des Gauss-Algorithmus
lösen konnte.
Bei der soeben vorgestellten Lösung des Fusspunkt-Problems
musste man aber wieder in mehrere Schritten vorgehen.
Zuerst wurden $s$ und $t$ bestimmt,
erst in einem zweiten Schritt konnte der Fusspunkt des Lotes
berechnet werden.

Natürlich ist das auch für dieses Problem möglich, man behandelt
$x$, $y$ und $z$ einfach als zusätzliche Variablen, die als Koordinaten
von $Av$ berechnet werden.
Das Gauss-Tableau dazu ist
\[
\begin{tabular}{|>{$}c<{$}>{$}c<{$}>{$}c<{$}>{$}c<{$}|>{$}c<{$}|}
\hline
x&y&z&s\quad t&\\
\hline
1&0&0&        &0\\
0&1&0&-A   &0\\
0&0&1&        &0\\
\hline
 & & &A^tA    &A^tb\\
\hline
\end{tabular}
\]
\begin{beispiel}
Für das Zahlenbeispiel lautet dieses Gauss-Tableau:
\[
\begin{tabular}{|>{$}c<{$}>{$}c<{$}>{$}c<{$}>{$}c<{$}>{$}c<{$}|>{$}c<{$}|}
\hline
x&y&z&s  &  t&\\
\hline
1&0&0& -8& -9&0\\
0&1&0&-10&-13&0\\
0&0&1&-10&-12&0\\
\hline
0&0&0&264&322&242\\
0&0&0&322&394&295\\
\hline
\end{tabular}
\rightarrow
\begin{tabular}{|>{$}c<{$}>{$}c<{$}>{$}c<{$}>{$}c<{$}>{$}c<{$}|>{$}r<{$}|}
\hline
x&y&z&s  &  t&\\
\hline
1&0&0&0&0&7.4337\\
0&1&0&0&0&9.0602\\
0&0&1&0&0&9.1928\\
\hline
0&0&0&1&0&1.0783\\
0&0&0&0&1&-0.1325\\
\hline
\end{tabular}
\]
Man erhält also genau die bereits früher gefundenen Lösungen.
\end{beispiel}

%
% Der allgemein Fall
%
\subsection{Der allgemeine Fall}
Überbestimmte Gleichungssysteme sind Gleichungssysteme der Form
$Ax=b$ mit einer $m\times n$-Matrix mit $m>n$, also mehr Gleichungen
als Unbekannten.
Im Allgemeinen sind sie nicht lösbar, weil $b$ nicht
im Bild von $A$ enthalten ist: $b\not\in \operatorname{im}A$.

Statt einer exakten Lösung könnte man daher eine approximative
Lösung suchen, welche die Gleichung möglichst gut erfüllt.
Der Vektor $Ax-b$ sollte also möglichst kurz sein.
Geometrisch
geht es also darum, das Lot vom Punkt $b$ auf den von den Spaltenvektoren
von $A$ aufgespannten Unterraum zu fällen.
Wir suchen also
einen Vektor, der auf allen Spaltenvektoren von $A$ senkrecht steht.
Das Skalarprodukt von Spaltenvektoren von $A$ mit $Ax-b$ ist aber
$A^t(Ax-b)$, wir müssen also das Gleichungssystem
\[A^t(Ax-b)=0\]
lösen.
Nach Ausmultiplizieren bekommen wir
\begin{equation}
A^tAx-A^tb=0\quad\Rightarrow\quad A^tAx=A^tb\quad\Rightarrow\quad
x=(A^tA)^{-1}A^tb.
\label{uberbestimmt}
\end{equation}
Man beachte, dass $A$ nicht quadratisch ist, und dass man daher
nicht mit $(A^tA)^{-1}A^t=A^{-1}(A^t)^{-1}A^t=A^{-1}$ vereinfachen
kann.

\begin{beispiel}
Sei 
\[
A=\begin{pmatrix}1\\1\\1\\\end{pmatrix},\quad b=\begin{pmatrix}1\\2\\3\end{pmatrix}.
\]
Offenbar ist $b\not\in\operatorname{im}A$.
Nach der Formel (\ref{uberbestimmt}) muss man zunächst $A^tA$ ausrechnen:
\[
A^tA=\begin{pmatrix}1&1&1\end{pmatrix}\begin{pmatrix}1\\1\\1\end{pmatrix}=3.
\]
Damit kann man jetzt nach (\ref{uberbestimmt}) die bestmögliche
approximative Lösung finden:
\[
x=\frac13\cdot\begin{pmatrix}1&1&1\end{pmatrix}
\begin{pmatrix}1\\2\\3\end{pmatrix}=2.
\]
Der von $b$ am wenigsten weit entfernte Punkt der Geraden mit
Richtung $A$ ist also der Punkt $(2,2,2)$.
\end{beispiel}

%
% Anwendungen der Methode der kleinsten Quadrate
%
\subsection{Anwendungen der Methode der kleinsten Quadrate}
Die Bedeutung der Methode der kleinsten Qudarate besteht darin, dass 
man in der Praxis sehr oft die Situation hat, dass man deutlich mehr
Daten hat als nötig, um die Parameter eines Problems zu bestimmen.
Zum Beispiel genügt es, drei Punkte eines Kreises zu kennen, um
Mittelpunkt und Radius exakt bestimmen zu können.
Oder zwei Punkte bestimmen eine Gerade eindeutig.
In der Praxis misst man meistens mehr Punkte.
Allerdings sind die Messungen mit Messfehlern behaftet.
Die Methode der kleinsten Quadrate erlaubt dann, die bestmöglichen
Werte für die Parameter zu finden.
Dies soll an zwei Beispielen illustriert werden.

\subsubsection{Gerade durch Punkte}
\begin{figure}
\centering
\includegraphics{4/images/linreg.pdf}
\caption{Bestimmung einer Geraden, die eine Menge von Punkten am besten
approximiert.
Die blauen Punkte sind entstanden, indem Punkte der grünen Gerade
mit einer zufälligen Abweichungen gestört wurden.
Die rot eingezeichnete Regeressionsgerade approximiert die ursprüngliche
Gerade mit grosser Genauigkeit.
\label{skript:linreg-abb}}
\end{figure}
Gegeben sind Punkte $(x_i,y_i)$ mit $1\le i\le n$, die ungefähr auf
einer Geraden $y={\color{red}a}x+{\color{red}b}$ liegen.
Gesucht sind die Werte von $\color{red}a$ und $\color{red}b$
(Abbildung~\ref{skript:linreg-abb}).
Zur Verdeutlichung heben wir im
Folgenden die Unbekannten rot hervor.
Wären die Punkte exakt auf der Geraden, müsste jeder die Geradengleichung
erfüllen, wir erhielten also die Gleichungen
\begin{equation}
\begin{linsys}{2}
{\color{red}a}x_1&+     &{\color{red}b}&=&y_1\\
{\color{red}a}x_2&+     &{\color{red}b}&=&y_2\\
                 &\vdots&              & &\vdots\hspace*{1mm}\\
{\color{red}a}x_n&+     &{\color{red}b}&=&y_n
\end{linsys}
\label{leastsquares:gerade}
\end{equation}
Dies ist ein Gleichungssystem für die Unbekannten ${\color{red}a}$ und
${\color{red}b}$ mit $n$
Gleichungen, im Allgemeinen ist es also überbestimmt.

Die Gleichung \eqref{leastsquares:gerade} kann mit dem Standardverfahren
gelöst werden.
Dazu schreiben wir zunächst die Matrix $A$ und den Vektor $b$ auf:
\[
A
=
\begin{pmatrix}
x_1   &1     \\
x_2   &1     \\
\vdots&\vdots\\
x_n   &1
\end{pmatrix}
,\qquad
x=
\begin{pmatrix}
\color{red}a\\
\color{red}b
\end{pmatrix},
\qquad
b
=
\begin{pmatrix}
y_1\\
y_2\\
\vdots\\
y_n
\end{pmatrix}
\]
Für die Lösung müssen wir $A^tA$ und $A^tb$ berechnen:
\begin{align*}
A^tA
&=
\begin{pmatrix}
x_1&x_2&\dots&x_n\\
 1 & 1 &\dots& 1
\end{pmatrix}
\begin{pmatrix}
x_1   &1     \\
x_2   &1     \\
\vdots&\vdots\\
x_n   &1
\end{pmatrix}
=
\begin{pmatrix}
\displaystyle\sum_{i=1}^n x_i^2 & \displaystyle\sum_{i=1}^n x_i \\
\displaystyle\sum_{i=1}^n x_i   &       n
\end{pmatrix}
\\
A^tb
&=
\begin{pmatrix}
x_1&x_2&\dots&x_n\\
 1 & 1 &\dots& 1
\end{pmatrix}
\begin{pmatrix}
y_1\\
y_2\\
\vdots\\
y_n
\end{pmatrix}
=
\begin{pmatrix}
\displaystyle \sum_{i=1}^n x_iy_i\\
\displaystyle \sum_{i=1}^n y_i
\end{pmatrix}.
\end{align*}
Die Determinante von $A^tA$ ist
\[
\det(A^tA)
=
n\sum_{i=1}^n x_i^2 -\biggl(\sum_{i=1}^n x_i\biggr)^2
.
\]
Dieses Gleichungssystem kann man jetzt mit der Kramerschen Regel für die
Unbekannte ${\color{red}a}$ lösen:
\begin{align*}
{\color{red}a}
&=
\frac{\displaystyle
n\sum_{i=1}^n x_iy_i-\sum_{i=1}^nx_i\sum_{i=1}^ny_i
}{\displaystyle
n\sum_{i=1}^n x_i^2 -\biggl(\sum_{i=1}^n x_i\biggr)^2
}
=
\frac{\displaystyle
\frac1n\sum_{i=1}^n x_iy_i-\frac1n\sum_{i=1}^nx_i\cdot \frac1n\sum_{i=1}^ny_i
}{\displaystyle
\frac1n\sum_{i=1}^n x_i^2 -\biggl(\frac1n\sum_{i=1}^n x_i\biggr)^2
}.
\end{align*}
Den Achsenabschnitt $\color{red}b$ könnte man natürlich auch so finden,
es geht aber auch direkter.
Bildet man die Summe der Gleichungen 
\ref{leastsquares:gerade}, erhält man
\[
{\color{red}a}
\sum_{i=1}^n x_i
+
n{\color{red}b}
=
\sum_{i=1}^n y_i.
\]
Auflösen nach $\color{red}b$ ergibt
\begin{align*}
{\color{red}b}
&=
\frac1n\sum_{i=1}^n y_i
-{\color{red}a}
\cdot
\frac1n\sum_{i=1}^n x_i.
\end{align*}
Die so gefundene Gerade  mit der Gleichung
$y={\color{red}a}x+{\color{red}b}$
heisst auch {\em Regressionsgerade}.
\index{Regressionsgerade}

\subsubsection{Kreis durch Punkte}
\begin{figure}
\centering
\includegraphics{4/images/kreis.pdf}
\caption{Bestimmung eines Kreises, der eine Menge von Punkten am besten
approximiert.
Die blauen Punkte sind entstanden, indem Punkte des grünen Kreises
mit einer zufälligen Abweichungen gestört wurden.
Der rot eingezeichnete Kreis approximiert den ursprünglichen
Kreis mit grosser Genauigkeit.
\label{skript:linreg-abb}}
\end{figure}
Gegeben sind die Punkte $(x_i,y_i)$ mit $1\le i\le n$, die ungefähr
auf einem Kreis liegen.
Gesucht sind Mittelpunkt $M=(m_x,m_y)$ und Radius $r$ dieses Kreises.
Wieder heben wir zur Verdeutlichung im Folgenden die Unbekannten farbig
hervor.
Zunächst müssen wir Gleichungen für die gesuchten Variablen aufstellen.
Die Gleichung eines Kreises ist
\[
(x_i - {\color{red}m_x})^2 + (y_i - {\color{red}m_y})^2 = {\color{red}r}^2.
\]
Diese Gleichungen sind allerdings nicht linear.
Das Standardverfahren ist also nicht anwendbar.
Wir multiplizieren daher aus, und erhalten
\[
x_i^2 - 2x_i {\color{red}m_x} + {\color{red}m_x}^2
+
y_i^2 - 2y_i {\color{red}m_y} + {\color{red}m_y}^2
=
{\color{red}r}^2.
\]
Indem wir die Quadrate der Variablen zu einer neuen Variable 
\[
{\color{red}c}
=
{\color{red}r}^2
-
{\color{red}m_x}^2
-
{\color{red}m_y}^2
\]
zusammenfassen, können wir die Gleichungen in lineare Form bringen:
\begin{equation}
2x_i{\color{red}m_x}
+
2y_i{\color{red}m_y}
+
{\color{red}c}
=
x_i^2+y_i^2,\qquad 1\le i\le n.
\end{equation}
In dieser Form lässt sich das Standardverfahren anwenden, die Matrix
$A$ und die Vektoren $x$ und $b$ sind
\begin{align*}
A
&=
\begin{pmatrix}
2x_1  & 2y_1 & 1    \\
2x_2  & 2y_2 & 1    \\
\vdots&\vdots&\vdots\\
2x_n  & 2y_n & 1    \\
\end{pmatrix},
&
x&=\begin{pmatrix}
{\color{red}m_x}\\
{\color{red}m_y}\\
{\color{red}c}
\end{pmatrix}
&&\text{und}&
b
&=
\begin{pmatrix}
x_1^2+y_1^2\\
x_2^2+y_2^2\\
\vdots\\
x_n^2+y_n^2
\end{pmatrix}.
\end{align*}
Aus der Lösung kann dann der Radius als 
\[
{\color{red}r}= 
\sqrt{
{\color{red}c}
+
{\color{red}m_x}^2
+
{\color{red}m_y}^2
}
\]
bestimmt werden.

\subsubsection{Drehung und Translation finden}
\begin{figure}
\centering
\includegraphics{4/images/bewegung.pdf}
\caption{Bestimmung von Drehwinkel und Translation einer Bewegung.
Die ursprüngliche Bewegung bewegt das schwarze Koordinatenkreuz auf
die grünen Achsen.
Die blauen Punkte werden mit dieser Bewegung abgebildet aber zusätzlich
noch mit einem zufälligen Fehler überlagert, so entstehen die grünen Punkte.
Die kurzen grünen Linien beginnen beim exakten Bild eines blauen Punktes.
Das im Text beschriebene Verfahren liefert den Drehwinkel und die
Verschiebung, die das Koordinatensystem auf die roten Achsen bewegt.
Damit werden die blauen Punkte auf die roten Punkte abgebildet.
Es zeigt sich, dass die grünen Linien ziemlich genau dort beginnen, wo
die gefundene Bewegung die blauen Punkte hinbewegt, das Verfahren kann
sowohl die Bewegung als auch die Fehler (die grünen Linien)
recht genau ermitteln.
\label{skript:bewegung-leastsquares}}
\end{figure}
Das Regstrierungsproblem in der Bildverarbeitung verlangt, zwei Bilder
der gleichen Szene mit Hilfe einer Drehung und Translation zur Deckung
zu bringen.
In der Astrophotographie kann man zum Beispiel in zwei Bildern die
Positionen der
Sterne $(x_i,y_i)$ und $(x_i',y_i')$ in jedem Bild finden und dann
die Transformation suchen, die die Punkte möglichst genau zur Deckung bringt.
Diese Transformation kann in Matrixform als
\begin{align*}
\begin{pmatrix}
x_i'\\y_i'
\end{pmatrix}
=
\begin{pmatrix}
\cos{\color{red}\alpha}&-\sin{\color{red}\alpha}& \color{red}t_x\\
\sin{\color{red}\alpha}&\phantom{-} \cos{\color{red}\alpha}& \color{red}t_y
\end{pmatrix}
\begin{pmatrix}
x_i\\y_i\\1
\end{pmatrix}
\end{align*}
geschrieben werden.
Der Winkel ${\color{red}\alpha}$ tritt in den Gleichungen nicht
linear auf.
Stattdessen verwenden wir
${\color{red}c}=\cos{\color{red}\alpha}$
und
${\color{red}s}=\sin{\color{red}\alpha}$
als Unbekannte.
Die Gleichungen werden 
\begin{equation}
\begin{linsys}{6}
x_i{\color{red}c} &-& y_i{\color{red}s} &+& {\color{red}t_x} & &                &=&x_i'\\
y_i{\color{red}c} &+& x_i{\color{red}s} & &                  &+&{\color{red}t_y}&=&y_i'\\
\end{linsys}
\end{equation}
Dies ist ein überbestimmtes lineares Gleichungssystem von
$2n$-Gleichungen für die vier Unbekannten
${\color{red}c}$,
${\color{red}s}$,
${\color{red}t_x}$ und
${\color{red}t_y}$.
Die Matrix $A$ und die Vektor $x$ und $b$ sind
\begin{align*}
A&=
\begin{pmatrix}
x_1&-y_1&1&0\\
y_1& x_1&0&1\\
x_2&-y_2&1&0\\
y_2& x_2&0&1\\
\vdots&\vdots&\vdots&\vdots\\
x_n&-y_n&1&0\\
y_n& x_n&0&1\\
\end{pmatrix},
&
x
&=
\begin{pmatrix}
{\color{red}c}\\
{\color{red}s}\\
{\color{red}t_x}\\
{\color{red}t_y}
\end{pmatrix}
&&\text{und}&
b
&=
\begin{pmatrix}
x_1'\\
y_1'\\
x_2'\\
y_2'\\
\vdots\\
x_n'\\
y_n'\\
\end{pmatrix}
\end{align*}
Es ist nicht garantiert, dass die Lösung
\[
\lambda^2
=
{\color{red}c}^2
+
{\color{red}s}^2
=1
\]
erfüllen wird, wie das für $\cos\alpha$ und $\sin\alpha$ der Fall wäre.
Die Bedeutung von $\lambda$ ist die eines zusätzlichen Streckungsfaktors,
dem das Bild für optimale Deckung auch noch unterworfen werden sollte.
Abbildung~\ref{skript:bewegung-leastsquares} zeigt an einem Beispiel, wie
das Verfahren die Bewegung aus wenigen Punkten ermitteln kann.





