%
% chapter.tex 
%
% (c) 2018 Prof Dr Andreas Müller, Hochschule Rapperswil
%
\chapter{Orthogonalität\label{chapter:orthogonalitaet}}
In der bisher entwickelten Vektorgeometrie haben Längen und Winkel keine 
Rolle gespielt.
Wir haben akzeptiert, dass lineare Abbildungen aus der Elementargeometrie
bekannte Eigenschaften wie Rechtwinkligkeit von Geraden oder Gleichseitigkeit
von Dreiecken zerstören können.
Wir hatten keine Wahl, weil wir kein Werkzeug zur Verfügung hatten, 
um Längen und Winkel zu messen.
Das Skalarprodukt ist dieses zusätzliche Werkzeug, wir führen es in
Abschnitt~\ref{section:skalarprodukt} ein.

Mit dem Skalarprodukt eröffnet sich ein ganze Menge neuer Anwendungen.
Wir können Geraden und Ebenen jetzt noch etwas prägnanter mit
der Normalenform beschreiben (Abschnitt~\ref{section:normalform}),
wir können besonders gut geeignete Basen konstruieren, in denen die
Zerlegung in Komponenten speziell einfach ist
(Orthonormalbasis, Abschnitt~\ref{section:orthonormalbasis}),
wir können Kreise und Kugeln
beschreiben (Abschnitt~\ref{section:kreisundkugel}) und wir können eine
allgemeine Technik für die Lösung von Überbestimmten Gleichungssystemen
entwickeln (Least Squares, Abschnitt~\ref{section:leastsquares})


