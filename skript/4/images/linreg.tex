%
% linreg.tex
%
% (c) 2018 Prof Dr Andreas Müller, Hochschule Rapperswil
%
\documentclass[tikz,12pt]{standalone}
\usepackage{times}
\usepackage{amsmath}
\usepackage{txfonts}
\usepackage[utf8]{inputenc}
\usepackage{graphics}
\usepackage{color}
\usepackage{pifont}
\usetikzlibrary{arrows,intersections,math,calc}
\begin{document}

\def\punkt#1{
        \fill[color=white] #1 circle[radius=0.08];
        \draw #1 circle[radius=0.08];
}

\def\p#1{
	\fill[color=blue] #1 circle[radius=0.07];
}

\definecolor{darkgreen}{rgb}{0,0.6,0}

\begin{tikzpicture}[>=latex,thick]

\coordinate (O) at (0,0);

%
% linreg.tex
%
% (c) 2018 Prof Dr Andreas Müller, Hochschule Rapperswil
%
\documentclass[tikz,12pt]{standalone}
\usepackage{times}
\usepackage{amsmath}
\usepackage{txfonts}
\usepackage[utf8]{inputenc}
\usepackage{graphics}
\usepackage{color}
\usepackage{pifont}
\usetikzlibrary{arrows,intersections,math,calc}
\begin{document}

\def\punkt#1{
        \fill[color=white] #1 circle[radius=0.08];
        \draw #1 circle[radius=0.08];
}

\def\p#1{
	\fill[color=blue] #1 circle[radius=0.07];
}

\definecolor{darkgreen}{rgb}{0,0.6,0}

\begin{tikzpicture}[>=latex,thick]

\coordinate (O) at (0,0);

%
% linreg.tex
%
% (c) 2018 Prof Dr Andreas Müller, Hochschule Rapperswil
%
\documentclass[tikz,12pt]{standalone}
\usepackage{times}
\usepackage{amsmath}
\usepackage{txfonts}
\usepackage[utf8]{inputenc}
\usepackage{graphics}
\usepackage{color}
\usepackage{pifont}
\usetikzlibrary{arrows,intersections,math,calc}
\begin{document}

\def\punkt#1{
        \fill[color=white] #1 circle[radius=0.08];
        \draw #1 circle[radius=0.08];
}

\def\p#1{
	\fill[color=blue] #1 circle[radius=0.07];
}

\definecolor{darkgreen}{rgb}{0,0.6,0}

\begin{tikzpicture}[>=latex,thick]

\coordinate (O) at (0,0);

%
% linreg.tex
%
% (c) 2018 Prof Dr Andreas Müller, Hochschule Rapperswil
%
\documentclass[tikz,12pt]{standalone}
\usepackage{times}
\usepackage{amsmath}
\usepackage{txfonts}
\usepackage[utf8]{inputenc}
\usepackage{graphics}
\usepackage{color}
\usepackage{pifont}
\usetikzlibrary{arrows,intersections,math,calc}
\begin{document}

\def\punkt#1{
        \fill[color=white] #1 circle[radius=0.08];
        \draw #1 circle[radius=0.08];
}

\def\p#1{
	\fill[color=blue] #1 circle[radius=0.07];
}

\definecolor{darkgreen}{rgb}{0,0.6,0}

\begin{tikzpicture}[>=latex,thick]

\coordinate (O) at (0,0);

\input{linreg.pkt}

\draw[->,line width=1] (-0.7, 0  )--(12,0  ) coordinate[label={$x$}];
\draw[->,line width=1] ( 0  ,-0.5)--( 0,7.5) coordinate[label={right:$y$}];

\foreach \x in {1,...,11}{
	\draw ({\x},-0.06)--({\x},0.06);
}
\foreach \y in {1,...,7}{
	\draw (-0.06,{\y})--(0.06,{\y});
}

\end{tikzpicture}

\end{document}



\draw[->,line width=1] (-0.7, 0  )--(12,0  ) coordinate[label={$x$}];
\draw[->,line width=1] ( 0  ,-0.5)--( 0,7.5) coordinate[label={right:$y$}];

\foreach \x in {1,...,11}{
	\draw ({\x},-0.06)--({\x},0.06);
}
\foreach \y in {1,...,7}{
	\draw (-0.06,{\y})--(0.06,{\y});
}

\end{tikzpicture}

\end{document}



\draw[->,line width=1] (-0.7, 0  )--(12,0  ) coordinate[label={$x$}];
\draw[->,line width=1] ( 0  ,-0.5)--( 0,7.5) coordinate[label={right:$y$}];

\foreach \x in {1,...,11}{
	\draw ({\x},-0.06)--({\x},0.06);
}
\foreach \y in {1,...,7}{
	\draw (-0.06,{\y})--(0.06,{\y});
}

\end{tikzpicture}

\end{document}



\draw[->,line width=1] (-0.7, 0  )--(12,0  ) coordinate[label={$x$}];
\draw[->,line width=1] ( 0  ,-0.5)--( 0,7.5) coordinate[label={right:$y$}];

\foreach \x in {1,...,11}{
	\draw ({\x},-0.06)--({\x},0.06);
}
\foreach \y in {1,...,7}{
	\draw (-0.06,{\y})--(0.06,{\y});
}

\end{tikzpicture}

\end{document}

