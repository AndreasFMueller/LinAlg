%
% bewegung.tex
%
% (c) 2018 Prof Dr Andreas Müller, Hochschule Rapperswil
%
\documentclass[tikz,12pt]{standalone}
\usepackage{times}
\usepackage{amsmath}
\usepackage{txfonts}
\usepackage[utf8]{inputenc}
\usepackage{graphics}
\usepackage{color}
\usepackage{pifont}
\usetikzlibrary{arrows,intersections,math,calc}
\begin{document}

\def\punkt#1{
        \fill[color=white] #1 circle[radius=0.08];
        \draw #1 circle[radius=0.08];
}

\definecolor{darkgreen}{rgb}{0,0.6,0}

\def\pa#1{
        \fill[color=blue] #1 circle[radius=0.07];
}
\def\pb#1{
        \fill[color=darkgreen] #1 circle[radius=0.07];
}
\def\pc#1{
        \fill[color=red] #1 circle[radius=0.07];
}

\begin{tikzpicture}[>=latex,thick]

\draw[->,line width=1] (-0.3, 0  )--(11.5,0  ) coordinate[label={$x$}];
\draw[->,line width=1] ( 0  ,-0.3)--( 0  ,7.5) coordinate[label={right:$y$}];

%
% bewegung.tex
%
% (c) 2018 Prof Dr Andreas Müller, Hochschule Rapperswil
%
\documentclass[tikz,12pt]{standalone}
\usepackage{times}
\usepackage{amsmath}
\usepackage{txfonts}
\usepackage[utf8]{inputenc}
\usepackage{graphics}
\usepackage{color}
\usepackage{pifont}
\usetikzlibrary{arrows,intersections,math,calc}
\begin{document}

\def\punkt#1{
        \fill[color=white] #1 circle[radius=0.08];
        \draw #1 circle[radius=0.08];
}

\definecolor{darkgreen}{rgb}{0,0.6,0}

\def\pa#1{
        \fill[color=blue] #1 circle[radius=0.07];
}
\def\pb#1{
        \fill[color=darkgreen] #1 circle[radius=0.07];
}
\def\pc#1{
        \fill[color=red] #1 circle[radius=0.07];
}

\begin{tikzpicture}[>=latex,thick]

\draw[->,line width=1] (-0.3, 0  )--(11.5,0  ) coordinate[label={$x$}];
\draw[->,line width=1] ( 0  ,-0.3)--( 0  ,7.5) coordinate[label={right:$y$}];

%
% bewegung.tex
%
% (c) 2018 Prof Dr Andreas Müller, Hochschule Rapperswil
%
\documentclass[tikz,12pt]{standalone}
\usepackage{times}
\usepackage{amsmath}
\usepackage{txfonts}
\usepackage[utf8]{inputenc}
\usepackage{graphics}
\usepackage{color}
\usepackage{pifont}
\usetikzlibrary{arrows,intersections,math,calc}
\begin{document}

\def\punkt#1{
        \fill[color=white] #1 circle[radius=0.08];
        \draw #1 circle[radius=0.08];
}

\definecolor{darkgreen}{rgb}{0,0.6,0}

\def\pa#1{
        \fill[color=blue] #1 circle[radius=0.07];
}
\def\pb#1{
        \fill[color=darkgreen] #1 circle[radius=0.07];
}
\def\pc#1{
        \fill[color=red] #1 circle[radius=0.07];
}

\begin{tikzpicture}[>=latex,thick]

\draw[->,line width=1] (-0.3, 0  )--(11.5,0  ) coordinate[label={$x$}];
\draw[->,line width=1] ( 0  ,-0.3)--( 0  ,7.5) coordinate[label={right:$y$}];

%
% bewegung.tex
%
% (c) 2018 Prof Dr Andreas Müller, Hochschule Rapperswil
%
\documentclass[tikz,12pt]{standalone}
\usepackage{times}
\usepackage{amsmath}
\usepackage{txfonts}
\usepackage[utf8]{inputenc}
\usepackage{graphics}
\usepackage{color}
\usepackage{pifont}
\usetikzlibrary{arrows,intersections,math,calc}
\begin{document}

\def\punkt#1{
        \fill[color=white] #1 circle[radius=0.08];
        \draw #1 circle[radius=0.08];
}

\definecolor{darkgreen}{rgb}{0,0.6,0}

\def\pa#1{
        \fill[color=blue] #1 circle[radius=0.07];
}
\def\pb#1{
        \fill[color=darkgreen] #1 circle[radius=0.07];
}
\def\pc#1{
        \fill[color=red] #1 circle[radius=0.07];
}

\begin{tikzpicture}[>=latex,thick]

\draw[->,line width=1] (-0.3, 0  )--(11.5,0  ) coordinate[label={$x$}];
\draw[->,line width=1] ( 0  ,-0.3)--( 0  ,7.5) coordinate[label={right:$y$}];

\input{bewegung.pkt}

\end{tikzpicture}

\end{document}



\end{tikzpicture}

\end{document}



\end{tikzpicture}

\end{document}



\end{tikzpicture}

\end{document}

