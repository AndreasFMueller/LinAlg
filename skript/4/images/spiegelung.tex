%
% spiegelung.tex
%
% (c) 2018 Prof Dr Andreas Müller, Hochschule Rapperswil
%
\documentclass[tikz]{standalone}
\usepackage{times}
\usepackage{amsmath}
\usepackage{txfonts}
\usepackage[utf8]{inputenc}
\usepackage{graphics}
\usepackage{color}
\usetikzlibrary{arrows,intersections,math}
\begin{document}

\definecolor{darkgreen}{rgb}{0,0.6,0}
\definecolor{rot}{rgb}{0.8,0,0}

\begin{tikzpicture}[>=latex,thick]

\def\x{3}
\def\y{2}

% Vektor v
\draw[->,line width=1.2pt] (0,0)--(\x,\y);
\node at ({\x/2},{\y/2}) [above left] {$\vec{v}$};

% Vektor v'
\draw[->,line width=1.2pt] (0,0)--(-\x,\y);
\node at ({-\x/2},{\y/2}) [above right] {$\vec{v}'$};

% parallele Komponente
\draw[->,line width=1.2pt,color=blue] (\x,0)--(\x,\y);
\draw[->,line width=1.2pt,color=blue] (-\x,0)--(-\x,\y);
\node[color=blue] at (\x,{\y/2}) [right] {$\vec{v}_{\|}$};
\node[color=blue] at (-\x,{\y/2}) [left] {$\vec{v}_{\|}$};

% orthogonale Komponente
\draw[->,line width=1.2pt,color=rot] (0,0)--(\x,0);
\draw[->,line width=1.2pt,color=rot] (0,0)--(-\x,0);
\node[color=rot] at ({\x/2},0) [below] {$\vec{v}_{\perp}$};
\node[color=rot] at ({-\x/2},0) [below] {$-\vec{v}_{\perp}$};

% Nullpunkt
\fill (0,0) circle[radius=0.08];

% Normalevektor
\draw[->,line width=1.2pt,color=darkgreen] (0,-1.5)--(2,-1.5);
\node at (2,-1.5) [right] {$\vec{n}$};

% Ebene
\draw[line width=0.7pt] (0,{-\y-0.5})--(0,{\y+0.5});
\node at (0,{\y}) [above right] {$\sigma$};

\end{tikzpicture}

\end{document}

