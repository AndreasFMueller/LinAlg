%
% tangente.tex
%
% (c) 2018 Prof Dr Andreas Müller, Hochschule Rapperswil
%
\documentclass[tikz,12pt]{standalone}
\usepackage{times}
\usepackage{amsmath}
\usepackage{txfonts}
\usepackage[utf8]{inputenc}
\usepackage{graphics}
\usepackage{color}
\usetikzlibrary{arrows,intersections,math,calc}
\begin{document}

\def\a{25}
\def\b{110}
\def\r{3}
\def\l{1.3}

\def\punkt#1{
	\fill[color=white] #1 circle[radius=0.07];
	\draw #1 circle[radius=0.07];
}

\begin{tikzpicture}[>=latex,thick]

\begin{scope}
	\clip (-3.5,-4) rectangle ({1.7*\r},{1.2*\r});
	\draw ({\r*(cos(\a)-2*sin(\a))},{\r*(sin(\a)+2*cos(\a))})
		--({\r*(cos(\a)+2*sin(\a))},{\r*(sin(\a)-2*cos(\a))});
\end{scope}

\coordinate (O) at (-3.5,-4);
\coordinate (M) at (0,0);
\coordinate (P0) at ({\r*cos(\a)},{\r*sin(\a)});
\coordinate (P) at ({\r*(cos(\a)+\l*sin(\a))},{\r*(sin(\a)-\l*cos(\a))});

% Nullpunkt
\node at (-3.5,-4) [left] {$O$};

% radius
\draw (0,0)--({\r*cos(\b)},{\r*sin(\b)});
\node at ({\r*cos(\b)/2},{\r*sin(\b)/2}) [left] {$r$};

% normale
\draw[->,line width=1.4pt] (P0)--($1.6*(P0)$);
\node at ($1.6*(P0)$) [above] {$\vec{n}$};
\draw[->,line width=1.0pt] (M)--(P0);

% rechter winkel
\draw ({(\r+0.5)*cos(\a)},{(\r+0.5)*sin(\a)}) arc (\a:{\a+90}:0.5);
\fill ({\r*cos(\a)+0.28*cos(45+\a)},{\r*sin(\a)+0.28*sin(45+\a)}) circle[radius=0.07];

% Punkt P0
\node at ($(P0)+(0.1,-0.1)$) [right] {$P_0$};
\draw[->,line width=1.4pt,color=gray] (O)--(P0);
\node at ($0.5*(O)+0.5*(P0)$) [below right] {$\vec{p}_0$};

% Punkt P
\draw[->,line width=1.4pt] ({\r*cos(\a)},{\r*sin(\a)})
		--({\r*(cos(\a)+\l*sin(\a))},{\r*(sin(\a)-\l*cos(\a))});

\draw[->,line width=1.4pt,color=gray] (O)--(P);
\node at ($0.5*(O)+0.5*(P)$) [below] {$\vec{p}$};
\node at (P) [right] {$P$};

% Vektor P_0P
\node at ($0.5*(P0)+0.5*(P)$) [right] {$\vec{p}-\vec{p}_0$};

% vektor m
\node at (0,0) [left] {$M$};
\draw[->,line width=1.4pt,color=gray] (-3.5,-4)--(0,0);
\node at ({-3.5*0.3},{-4*0.3}) [above left] {$\vec{m}$};

% Label für Vektor MP_0
\node at ($0.5*(P0)$) [above] {$\vec{r}$};

% Kreis
\draw (0,0) circle[radius=\r];

% Punkte
\punkt{(0,0)}                                                 % M
\punkt{({\r*(cos(\a)+\l*sin(\a))},{\r*(sin(\a)-\l*cos(\a))})} % P
\punkt{(-3.5,-4)}                                             % O
\punkt{({\r*cos(\a)},{\r*sin(\a)})}                           % P0

\end{tikzpicture}

\end{document}

