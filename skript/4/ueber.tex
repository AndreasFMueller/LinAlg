%
% ueber.tex -- Überbestimmte Gleichungen
%
% (c) 2018 Prof Dr Andreas Müller, Hochschule Rapperswil
%
\subsection{Überbestimmte Gleichungssysteme -- ``Least Squares''\label{section-least-squares}}
Bei einem überbestimmten Gleichungssystem, als einem Gleichungssystem
mit mehr Gleichungen als Unbekannten, kann man im allgemeinen nicht davon
ausgehen, dass es überhaupt eine Lösung gibt.
Ein solches Gleichungssystem hat die Form
\[
A v= b,
\]
wobei $A$ eine Matrix ist, die mehr Zeilen als Spalten hat.
In drei Dimensionen hat $v$ also zwei oder sogar nur eine Komponenten,
die Menge aller möglichen Vektoren $Av$ ist also eine Ebene oder
Gerade.

\subsubsection{Lösung im Sinne kleinsten Abstandes}
Das beste, was man erwarten kann, ist ein Vektor $v_0$ so, dass
der Abstand des Punktes $ b$ von der Ebene (Gerade) bestehende
aus allen $Av$ für $v=v_0$ am kleinsten wird.
Dies geschieht
natürlich genau dann, wenn der Differenzvektor $b-Av_0$ auf
allen Vektoren von $Av$ senkrecht steht.

Die Menge der Vektoren der Form $Av$ wird von den Spalten von $A$
aufgespannt, es genügt also zu testen, ob $b-Av_0$ auf diesen
Vektoren senkrecht steht.
Dazu müssen die Skalarprodukte von
Spalten von $A$ mit dem Vektor $b-Av_0$ verschwinden, oder
\[
A^t(b-Av_0)=0
\quad
\Rightarrow
\quad
A^tAv_0=A^tb
\]
wir haben also ein Gleichungssystem gefunden mit Matrix $A^tA$ und
rechter Seite $A^tb$, welches als Lösung den gesuchten Vektor
$v_0$ hat.
$A^t$ hat so viele Zeilen wie $v$ Komponenten hat, also
handelt es sich um ein Gleichungssystem mit gleich vielen Gleichungen
wie Unbekannten, es wird im Allgemeinen eine eindeutig bestimmte
Lösung haben.

\begin{satz} Sei $A$ eine $n\times m$ Matrix und $b$ ein $n$-dimensionaler
Vektor.
Eine Lösung im Sinne minimaler quadrierter Abstände
$
(Av-b)\cdot(Av-b)
$
ist Lösung des Gleichungssystems
\[
A^tAv=A^tb
\]
mit $m$ Gleichungen und $m$ Unbekannten.
\end{satz}

\begin{beispiel}
Man finden den Fusspunkt des Lotes vom Punkt $P=(9,10,7)$ auf die Ebene
durch $O$, $A=(8,10,10)$ und $B=(9,13,12)$.

Der Fusspunkt des Lotes ist der Punkt der Ebene, der den geringsten
Abstand zu $P$ hat.
Die Ebenengleichung ist
\[
A\begin{pmatrix}s\\t\end{pmatrix}=
\begin{pmatrix}
 8& 9\\
10&13\\
10&12
\end{pmatrix}
\begin{pmatrix}s\\t\end{pmatrix}.
\]
Gesucht wird die ``beste Lösung'' von
\[
A\begin{pmatrix}s\\t\end{pmatrix}=\begin{pmatrix}9\\10\\7\end{pmatrix}=b
\]
Dazu muss zunächst die Matrix $A^tA$ und der Vektor $A^tb$
berechnet werden.
\[
A^tA=\begin{pmatrix}
264&322\\
322&394
\end{pmatrix}
,\qquad
A^tb=\begin{pmatrix}
242\\295
\end{pmatrix}.
\]
Daraus findet man die Lösung für $s$ und $t$ numerisch zu
\[
\begin{pmatrix}s\\t \end{pmatrix}
=
\begin{pmatrix}
   1.07831\\
  -0.13253
\end{pmatrix}
\]
und durch Einsetzen in die Ebenengleichung den Ortsvektor des Fusspunktes
\[
\vec f = \begin{pmatrix}
   7.4337\\
   9.0602\\
   9.1928
\end{pmatrix}
\]
Man kann dieses Resultat dadurch kontrollieren, dass man nachrechnet, ob
$\vec p-\vec f$ senkrecht auf beiden Richtungsvektoren der Ebene
steht:
\[
(\vec p-\vec f)^tA=\begin{pmatrix}
  -1.7451\cdot10^{-11}&  -2.1316\cdot 10^{-11}
\end{pmatrix},
\]
im Rahmen der Rechengenauigkeit steht die Differenz also tatsächlich auf
den Richtungsvektoren senkrecht.
\end{beispiel}

\subsubsection{``Alles in einem Schritt''}
In Abschnitt \ref{section-vereinheitlichtes-verfahren} wurde gezeigt,
wie man Schnittproblem mit nur einer Durchführung des Gauss-Algorithmus
lösen konnte.
Im soeben gelösten Problem musste man aber wieder
in mehrere Schritten vorgehen.
Zuerst wurden $s$ und $t$ bestimmt,
und erst in einem zweiten Schritt konnte der Fusspunkt des Lotes
berechnet werden.

Natürlich ist das auch für dieses Problem möglich, man behandelt
$x$, $y$ und $z$ einfach als zusätzliche Variablen, die als Koordinaten
von $Av$ berechnet werden.
Das Gauss-Tableau dazu ist
\[
\begin{tabular}{|>{$}c<{$}>{$}c<{$}>{$}c<{$}>{$}c<{$}|>{$}c<{$}|}
\hline
x&y&z&s\quad t&\\
\hline
1&0&0&        &0\\
0&1&0&-A   &0\\
0&0&1&        &0\\
\hline
 & & &A^tA    &A^tb\\
\hline
\end{tabular}
\]
\begin{beispiel}
Für das Zahlenbeispiel lautet dieses Gauss-Tableau:
\[
\begin{tabular}{|>{$}c<{$}>{$}c<{$}>{$}c<{$}>{$}c<{$}>{$}c<{$}|>{$}c<{$}|}
\hline
x&y&z&s  &  t&\\
\hline
1&0&0& -8& -9&0\\
0&1&0&-10&-13&0\\
0&0&1&-10&-12&0\\
\hline
0&0&0&264&322&242\\
0&0&0&322&394&295\\
\hline
\end{tabular}
\rightarrow
\begin{tabular}{|>{$}c<{$}>{$}c<{$}>{$}c<{$}>{$}c<{$}>{$}c<{$}|>{$}r<{$}|}
\hline
x&y&z&s  &  t&\\
\hline
1&0&0&0&0&7.4337\\
0&1&0&0&0&9.0602\\
0&0&1&0&0&9.1928\\
\hline
0&0&0&1&0&1.0783\\
0&0&0&0&1&-0.1325\\
\hline
\end{tabular}
\]
Man erhält also genau die bereits früher gefundenen Lösungen.
\end{beispiel}

