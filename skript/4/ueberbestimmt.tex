%
% ueberbestimmt.tex
%
% (c) 2018 Prof Dr Andreas Müller, Hochschule Rapperswil
%
\section{Überbestimmte Gleichungssysteme\label{section:ueberbestimmt}}
\index{Gleichungssystem!ueberbestimmtes@\überbestimmtes}
Überbestimmte Gleichungssysteme sind Gleichungssysteme der Form
$Ax=b$ mit einer $m\times n$-Matrix mit $m>n$, also mehr Gleichungen
als Unbekannten.
Im allgemeinen sind sie nicht lösbar, weil $b$ nicht
im Bild von $A$ enthalten ist: $b\not\in \operatorname{im}A$.

Statt einer exakten Lösung könnte man daher eine approximative
Lösung suchen, welche die Gleichung möglichst gut erfüllt,
der Vektor $Ax-b$ sollte also möglichst kurz sein.
Geometrisch
geht es also darum, das Lot vom Punkt $b$ auf den von den Spaltenvektoren
von $A$ aufgespannten Unterraum zu fällen.
Wir suchen also
einen Vektor, der auf allen Spaltenvektoren von $A$ senkrecht steht.
Das Skalarprodukt von Spaltenvektoren von $A$ mit $Ax-b$ ist aber
$A^t(Ax-b)$, wir müssen also das Gleichungssystem
\[A^t(Ax-b)=0\]
lösen.
Nach Ausmultiplizieren bekommen wir
\begin{equation}
A^tAx-A^tb=0\quad\Rightarrow\quad A^tAx=A^tb\quad\Rightarrow\quad
x=(A^tA)^{-1}A^tb.
\label{uberbestimmt}
\end{equation}
Man beachte, dass $A$ nicht quadratisch ist, und dass man daher
nicht mit $(A^tA)^{-1}A^t=A^{-1}(A^t)^{-1}A^t=A^{-1}$ vereinfachen
kann.

\begin{beispiel}
Sei 
\[
A=\begin{pmatrix}1\\1\\1\\\end{pmatrix},\quad b=\begin{pmatrix}1\\2\\3\end{pmatrix}.
\]
Offenbar ist $b\not\in\operatorname{im}A$.
Nach der Formel (\ref{uberbestimmt}) muss man zunächst $A^tA$ ausrechnen:
\[
A^tA=\begin{pmatrix}1&1&1\end{pmatrix}\begin{pmatrix}1\\1\\1\end{pmatrix}=3.
\]
Damit kann man jetzt nach (\ref{uberbestimmt}) die bestmögliche
approximative Lösung finden:
\[
x=\frac13\cdot\begin{pmatrix}1&1&1\end{pmatrix}
\begin{pmatrix}1\\2\\3\end{pmatrix}=2.
\]
Der von $b$ am wenigsten weit entfernte Punkt der Geraden mit
Richtung $A$ ist also der Punkt $(2,2,2)$.
\end{beispiel}
