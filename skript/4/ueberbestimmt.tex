%
% ueberbestimmt.tex
%
% (c) 2018 Prof Dr Andreas Müller, Hochschule Rapperswil
%
\section{Überbestimmte Gleichungssysteme -- ``Least Squares''%
\label{section:ueberbestimmt}}
\index{Gleichungssystem!ueberbestimmtes@\überbestimmtes}
Bei einem überbestimmten Gleichungssystem, als einem Gleichungssystem
mit mehr Gleichungen als Unbekannten, kann man im allgemeinen nicht davon
ausgehen, dass es überhaupt eine Lösung gibt.
Ein solches Gleichungssystem hat die Form
\[
A v= b,
\]
wobei $A$ eine Matrix ist, die mehr Zeilen als Spalten hat.
In drei Dimensionen hat $v$ also zwei oder sogar nur eine Komponenten,
die Menge aller möglichen Vektoren $Av$ ist also eine Ebene oder
Gerade.

\rhead{Überbestimmte Gleichungssysteme}
%
% Lösungen im Sinne des kleinsten Abstandes
%
\subsection{Lösung im Sinne des kleinsten Abstandes}
Das beste, was man erwarten kann, ist ein Vektor $v_0$ so, dass
der Abstand des Punktes $ b$ von der Ebene (Gerade) bestehende
aus allen $Av$ für $v=v_0$ am kleinsten wird.
Dies geschieht
natürlich genau dann, wenn der Differenzvektor $b-Av_0$ auf
allen Vektoren von $Av$ senkrecht steht.

Die Menge der Vektoren der Form $Av$ wird von den Spalten von $A$
aufgespannt, es genügt also zu testen, ob $b-Av_0$ auf diesen
Vektoren senkrecht steht.
Dazu müssen die Skalarprodukte von
Spalten von $A$ mit dem Vektor $b-Av_0$ verschwinden, oder
\[
A^t(b-Av_0)=0
\quad
\Rightarrow
\quad
A^tAv_0=A^tb
\]
wir haben also ein Gleichungssystem gefunden mit Matrix $A^tA$ und
rechter Seite $A^tb$, welches als Lösung den gesuchten Vektor
$v_0$ hat.
$A^t$ hat so viele Zeilen wie $v$ Komponenten hat, also
handelt es sich um ein Gleichungssystem mit gleich vielen Gleichungen
wie Unbekannten, es wird im Allgemeinen eine eindeutig bestimmte
Lösung haben.

\begin{satz} Sei $A$ eine $n\times m$ Matrix und $b$ ein $n$-dimensionaler
Vektor.
Eine Lösung im Sinne minimaler quadrierter Abstände
$
(Av-b)\cdot(Av-b)
$
ist Lösung des Gleichungssystems
\[
A^tAv=A^tb
\]
mit $m$ Gleichungen und $m$ Unbekannten.
\end{satz}

\begin{beispiel}
Man finden den Fusspunkt des Lotes vom Punkt $P=(9,10,7)$ auf die Ebene
durch $O$, $A=(8,10,10)$ und $B=(9,13,12)$.

Der Fusspunkt des Lotes ist der Punkt der Ebene, der den geringsten
Abstand zu $P$ hat.
Die Ebenengleichung ist
\[
A\begin{pmatrix}s\\t\end{pmatrix}=
\begin{pmatrix}
 8& 9\\
10&13\\
10&12
\end{pmatrix}
\begin{pmatrix}s\\t\end{pmatrix}.
\]
Gesucht wird die ``beste Lösung'' von
\[
A\begin{pmatrix}s\\t\end{pmatrix}=\begin{pmatrix}9\\10\\7\end{pmatrix}=b
\]
Dazu muss zunächst die Matrix $A^tA$ und der Vektor $A^tb$
berechnet werden.
\[
A^tA=\begin{pmatrix}
264&322\\
322&394
\end{pmatrix}
,\qquad
A^tb=\begin{pmatrix}
242\\295
\end{pmatrix}.
\]
Daraus findet man die Lösung für $s$ und $t$ numerisch zu
\[
\begin{pmatrix}s\\t \end{pmatrix}
=
\begin{pmatrix}
   1.07831\\
  -0.13253
\end{pmatrix}
\]
und durch Einsetzen in die Ebenengleichung den Ortsvektor des Fusspunktes
\[
\vec f = \begin{pmatrix}
   7.4337\\
   9.0602\\
   9.1928
\end{pmatrix}
\]
Man kann dieses Resultat dadurch kontrollieren, dass man nachrechnet, ob
$\vec p-\vec f$ senkrecht auf beiden Richtungsvektoren der Ebene
steht:
\[
(\vec p-\vec f)^tA=\begin{pmatrix}
  -1.7451\cdot10^{-11}&  -2.1316\cdot 10^{-11}
\end{pmatrix},
\]
im Rahmen der Rechengenauigkeit steht die Differenz also tatsächlich auf
den Richtungsvektoren senkrecht.
\end{beispiel}

%
% Alles in einem Schritt
%
\subsection{``Alles in einem Schritt''}
In Abschnitt \ref{section-vereinheitlichtes-verfahren} wurde gezeigt,
wie man Schnittproblem mit nur einer Durchführung des Gauss-Algorithmus
lösen konnte.
Im soeben gelösten Problem musste man aber wieder
in mehrere Schritten vorgehen.
Zuerst wurden $s$ und $t$ bestimmt,
und erst in einem zweiten Schritt konnte der Fusspunkt des Lotes
berechnet werden.

Natürlich ist das auch für dieses Problem möglich, man behandelt
$x$, $y$ und $z$ einfach als zusätzliche Variablen, die als Koordinaten
von $Av$ berechnet werden.
Das Gauss-Tableau dazu ist
\[
\begin{tabular}{|>{$}c<{$}>{$}c<{$}>{$}c<{$}>{$}c<{$}|>{$}c<{$}|}
\hline
x&y&z&s\quad t&\\
\hline
1&0&0&        &0\\
0&1&0&-A   &0\\
0&0&1&        &0\\
\hline
 & & &A^tA    &A^tb\\
\hline
\end{tabular}
\]
\begin{beispiel}
Für das Zahlenbeispiel lautet dieses Gauss-Tableau:
\[
\begin{tabular}{|>{$}c<{$}>{$}c<{$}>{$}c<{$}>{$}c<{$}>{$}c<{$}|>{$}c<{$}|}
\hline
x&y&z&s  &  t&\\
\hline
1&0&0& -8& -9&0\\
0&1&0&-10&-13&0\\
0&0&1&-10&-12&0\\
\hline
0&0&0&264&322&242\\
0&0&0&322&394&295\\
\hline
\end{tabular}
\rightarrow
\begin{tabular}{|>{$}c<{$}>{$}c<{$}>{$}c<{$}>{$}c<{$}>{$}c<{$}|>{$}r<{$}|}
\hline
x&y&z&s  &  t&\\
\hline
1&0&0&0&0&7.4337\\
0&1&0&0&0&9.0602\\
0&0&1&0&0&9.1928\\
\hline
0&0&0&1&0&1.0783\\
0&0&0&0&1&-0.1325\\
\hline
\end{tabular}
\]
Man erhält also genau die bereits früher gefundenen Lösungen.
\end{beispiel}

%
% Der allgemein Fall
%
\subsection{Der allgemeine Fall}
Überbestimmte Gleichungssysteme sind Gleichungssysteme der Form
$Ax=b$ mit einer $m\times n$-Matrix mit $m>n$, also mehr Gleichungen
als Unbekannten.
Im allgemeinen sind sie nicht lösbar, weil $b$ nicht
im Bild von $A$ enthalten ist: $b\not\in \operatorname{im}A$.

Statt einer exakten Lösung könnte man daher eine approximative
Lösung suchen, welche die Gleichung möglichst gut erfüllt,
der Vektor $Ax-b$ sollte also möglichst kurz sein.
Geometrisch
geht es also darum, das Lot vom Punkt $b$ auf den von den Spaltenvektoren
von $A$ aufgespannten Unterraum zu fällen.
Wir suchen also
einen Vektor, der auf allen Spaltenvektoren von $A$ senkrecht steht.
Das Skalarprodukt von Spaltenvektoren von $A$ mit $Ax-b$ ist aber
$A^t(Ax-b)$, wir müssen also das Gleichungssystem
\[A^t(Ax-b)=0\]
lösen.
Nach Ausmultiplizieren bekommen wir
\begin{equation}
A^tAx-A^tb=0\quad\Rightarrow\quad A^tAx=A^tb\quad\Rightarrow\quad
x=(A^tA)^{-1}A^tb.
\label{uberbestimmt}
\end{equation}
Man beachte, dass $A$ nicht quadratisch ist, und dass man daher
nicht mit $(A^tA)^{-1}A^t=A^{-1}(A^t)^{-1}A^t=A^{-1}$ vereinfachen
kann.

\begin{beispiel}
Sei 
\[
A=\begin{pmatrix}1\\1\\1\\\end{pmatrix},\quad b=\begin{pmatrix}1\\2\\3\end{pmatrix}.
\]
Offenbar ist $b\not\in\operatorname{im}A$.
Nach der Formel (\ref{uberbestimmt}) muss man zunächst $A^tA$ ausrechnen:
\[
A^tA=\begin{pmatrix}1&1&1\end{pmatrix}\begin{pmatrix}1\\1\\1\end{pmatrix}=3.
\]
Damit kann man jetzt nach (\ref{uberbestimmt}) die bestmögliche
approximative Lösung finden:
\[
x=\frac13\cdot\begin{pmatrix}1&1&1\end{pmatrix}
\begin{pmatrix}1\\2\\3\end{pmatrix}=2.
\]
Der von $b$ am wenigsten weit entfernte Punkt der Geraden mit
Richtung $A$ ist also der Punkt $(2,2,2)$.
\end{beispiel}

%
% Anwendungen der Methode der kleinsten Quadrate
%
\subsection{Anwendungen der Methode der kleinsten Quadrate}
Die Bedeutung der Methode der kleinsten Qudarate besteht darin, dass 
man in der Praxis sehr oft die Situation hat, dass man deutlich mehr
Daten hat als nötig, um die Parameter eines Problems zu bestimmen.
Zum Beispiel genügt es, drei Punkte eines Kreises zu kennen, um
Mittelpunkt und Radius exakt bestimmen zu können.
Oder zwei Punkte bestimmen eine Gerade eindeutig.
In der Praxis misst man meistens mehr Punkte, allerdings sind die Messungen
mit Messfehlern behaftet.
Die Methode der kleinsten Quadrate erlaubt dann, die bestmöglichen
Werte für die Parameter zu finden.
Dies soll an zwei Beispielen illustriert werden.

\subsubsection{Gerade durch Punkte}
Gegeben sind Punkte $(x_i,y_i)$ mit $1\le i\le n$, die ungefähr auf
einer Geraden $y=ax+b$ liegen.
Gesucht sind die Werte von $a$ und $b$, zur Verdeutlichung heben wir im
Folgenden die Unbekannten rot hervor.
Wären die Punkte exakt auf der Geraden, würden die Gleichungen
\begin{equation}
\begin{linsys}{2}
{\color{red}a}x_1&+     &{\color{red}b}&=&y_1\\
{\color{red}a}x_2&+     &{\color{red}b}&=&y_2\\
                 &\vdots&              & &\vdots\hspace*{1mm}\\
{\color{red}a}x_n&+     &{\color{red}b}&=&y_n
\end{linsys}
\label{leastsquares:gerade}
\end{equation}
Dies ist ein Gleichungssystem für die Unbekannten ${\color{red}a}$ und
${\color{red}b}$ mit $n$
Gleichungen, im allgemeinen ist es also überbestimmt.

Die Gleichung \eqref{leastsquares:gerade} kann mit dem Standardverfahren
gelöst werden.
Dazu schreiben wir zunächst die Matrix $A$ und den Vektor $b$ auf:
\[
A
=
\begin{pmatrix}
x_1   &1     \\
x_2   &1     \\
\vdots&\vdots\\
x_n   &1
\end{pmatrix}
,\qquad
x=
\begin{pmatrix}
\color{red}a\\
\color{red}b
\end{pmatrix},
\qquad
b
=
\begin{pmatrix}
y_1\\
y_2\\
\vdots\\
y_n
\end{pmatrix}
\]
Für die Lösung müssen wir $A^tA$ und $A^tb$ berechnen:
\begin{align*}
A^tA
&=
\begin{pmatrix}
x_1&x_2&\dots&x_n\\
 1 & 1 &\dots& 1
\end{pmatrix}
\begin{pmatrix}
x_1   &1     \\
x_2   &1     \\
\vdots&\vdots\\
x_n   &1
\end{pmatrix}
=
\begin{pmatrix}
\displaystyle\sum_{i=1}^n x_i^2 & \displaystyle\sum_{i=1}^n x_i \\
\displaystyle\sum_{i=1}^n x_i   &       n
\end{pmatrix}
\\
A^tb
&=
\begin{pmatrix}
x_1&x_2&\dots&x_n\\
 1 & 1 &\dots& 1
\end{pmatrix}
\begin{pmatrix}
y_1\\
y_2\\
\vdots\\
y_n
\end{pmatrix}
=
\begin{pmatrix}
\displaystyle \sum_{i=1}^n x_iy_i\\
\displaystyle \sum_{i=1}^n y_i
\end{pmatrix}.
\end{align*}
Die Determinante von $A^tA$ ist
\[
\det(A^tA)
=
n\sum_{i=1}^n x_i^2 -\biggl(\sum_{i=1}^n x_i\biggr)^2
.
\]
Dieses Gleichungssystem kann man jetzt mit der Kramerschen Regel für die
Unbekannte ${\color{red}a}$ lösen:
\begin{align*}
{\color{red}a}
&=
\frac{\displaystyle
n\sum_{i=1}^n x_iy_i-\sum_{i=1}^nx_i\sum_{i=1}^ny_i
}{\displaystyle
n\sum_{i=1}^n x_i^2 -\biggl(\sum_{i=1}^n x_i\biggr)^2
}
=
\frac{\displaystyle
\frac1n\sum_{i=1}^n x_iy_i-\frac1n\sum_{i=1}^nx_i\cdot \frac1n\sum_{i=1}^ny_i
}{\displaystyle
\frac1n\sum_{i=1}^n x_i^2 -\biggl(\frac1n\sum_{i=1}^n x_i\biggr)^2
}
\\
\intertext{Das ursprüngliche Gleichungssystem impliziert, dass die Mittelwerte
von $x_i$ und $y_i$ die Beziehung $y={\color{red}a}x+{\color{red}b}$ erfüllen,
können wir auch nach ${\color{red}b}$ auflösen und erhalten.}
{\color{red}b}
&=
\frac1n\sum_{i=1}^n y_i
-{\color{red}a}
\cdot
\frac1n\sum_{i=1}^n x_i
\end{align*}
Die so gefundene Gerade heisst auch {\em Regressionsgerade}.
\index{Regressionsgerade}

\subsubsection{Kreis durch Punkte}
Gegeben sind die Punkte $(x_i,y_i)$ mit $1\le i\le n$, die ungefähr
auf einem Kreis liegen.
Gesucht sind Mittelpunkt $M=(m_x,m_y)$ und Radius $r$ dieses Kreises.
Wieder heben wir zur Verdeutlichung im Folgenden die Unbekannten farbig
hervor.
Zunächst müssen wir Gleichungen für die gesuchten Variablen aufstellen.
Die Gleichung eines Kreises ist
\[
(x_i - {\color{red}m_x})^2 + (y_i - {\color{red}m_y})^2 = {\color{red}r}^2.
\]
Diese Gleichungen sind allerdings nicht linear, das Standardverfahren 
ist also nicht anwendbar.
Wir multiplizieren daher aus, und erhalten
\[
x_i^2 - 2x_i {\color{red}m_x} + {\color{red}m_x}^2
+
y_i^2 - 2y_i {\color{red}m_y} + {\color{red}m_y}^2
=
{\color{red}r}^2.
\]
Indem wir die Quadrate der Variablen zu einer neuen Variable 
\[
{\color{red}c}
=
{\color{red}r}^2
-
{\color{red}m_x}^2
-
{\color{red}m_y}^2
\]
zusammenfassen, können wir die Gleichungen in lineare Form bringen:
\begin{equation}
2x_i{\color{red}m_x}
+
2y_i{\color{red}m_x}
+
{\color{red}c}
=
x_i^2+y_i^2,\qquad 1\le i\le n.
\end{equation}
In dieser Form lässt sich das Standardverfahren anwenden, die Matrix
$A$ und die Vektoren $x$ und $b$ sind
\begin{align*}
A
&=
\begin{pmatrix}
2x_1  & 2y_1 & 1    \\
2x_2  & 2y_2 & 1    \\
\vdots&\vdots&\vdots\\
2x_n  & 2y_n & 1    \\
\end{pmatrix},
&
x&=\begin{pmatrix}
{\color{red}m_x}\\
{\color{red}m_y}\\
{\color{red}c}
\end{pmatrix}
&&\text{und}&
b
&=
\begin{pmatrix}
x_1^2+y_1^2\\
x_2^2+y_2^2\\
\vdots\\
x_n^2+y_n^2
\end{pmatrix}.
\end{align*}
Aus der Lösung kann dann der Radius als 
\[
{\color{red}r}= 
\sqrt{
{\color{red}c}
+
{\color{red}m_x}^2
+
{\color{red}m_y}^2
}
\]
bestimmt werden.

\subsubsection{Drehung und Translation finden}
Das Regstrierungsproblem in der Bildverarbeitung verlangt, zwei Bilder
der gleichen Szene mit Hilfe einer Drehung und Translation zur Deckung
zu bringen.
In der Astrophotographie kann man zum Beispiel in zwei Bildern die
Positionen der
Sterne $(x_i,y_i)$ und $(x_i',y_i')$ in jedem Bild finden und dann
die Transformation suchen, die die Punkte möglichst genau zur Deckung bringt.
Diese Transformation kann in Matrixform als
\begin{align*}
\begin{pmatrix}
x_i'\\y_i'
\end{pmatrix}
=
\begin{pmatrix}
\cos{\color{red}\alpha}&-\sin{\color{red}\alpha}& \color{red}t_x\\
\sin{\color{red}\alpha}&\phantom{-} \cos{\color{red}\alpha}& \color{red}t_y
\end{pmatrix}
\begin{pmatrix}
x_i\\y_i\\1
\end{pmatrix}
\end{align*}
geschrieben werden.
Der Winkel ${\color{red}\alpha}$ tritt in den Gleichungen nicht
linear auf.
Stattdessen verwenden wir
${\color{red}c}=\cos{\color{red}\alpha}$
und
${\color{red}s}=\sin{\color{red}\alpha}$
als Unbekannte.
Die Gleichungen werden 
\begin{equation}
\begin{linsys}{6}
x_i{\color{red}c} &-& y_i{\color{red}s} &+& {\color{red}t_x} & &                &=&x_i'\\
y_i{\color{red}c} &+& x_i{\color{red}s} & &                  &+&{\color{red}t_y}&=&y_i'\\
\end{linsys}
\end{equation}
Dies ist ein überbestimmtes lineares Gleichungssystem von
$2n$-Gleichungen für die vier Unbekannten
${\color{red}c}$,
${\color{red}s}$,
${\color{red}t_x}$ und
${\color{red}t_y}$.
Die Matrix $A$ und die Vektor $x$ und $b$ sind
\begin{align*}
A&=
\begin{pmatrix}
x_1&-y_1&1&0\\
y_1& x_1&0&1\\
x_2&-y_2&1&0\\
y_2& x_2&0&1\\
\vdots&\vdots&\vdots&\vdots\\
x_n&-y_n&1&0\\
y_n& x_n&0&1\\
\end{pmatrix},
&
x
&=
\begin{pmatrix}
{\color{red}c}\\
{\color{red}s}\\
{\color{red}t_x}\\
{\color{red}t_y}
\end{pmatrix}
&&\text{und}&
b
&=
\begin{pmatrix}
x_1'\\
y_1'\\
x_2'\\
y_2'\\
\vdots\\
x_n'\\
y_n'\\
\end{pmatrix}
\end{align*}
Es ist nicht garantiert, dass die Lösung
\[
\lambda^2
=
{\color{red}c}^2
+
{\color{red}s}^2
=1
\]
erfüllen wird, wie das für $\cos\alpha$ und $\sin\alpha$ der Fall wäre.
Die Bedeutung von $\lambda$ ist die eines zusätzlichen Streckungsfaktors,
dem das Bild für optimale Deckung auch noch unterworfen werden sollte.



