%
% abbildungen.tex
%
% (c) 2018 Prof Dr Andreas Müller, Hochschule Rapperswil
%
\section{Orthogonale Abbildungen\label{section:orthabb}}
\rhead{Orthogonale Abbildungen}
Wenn eine affine Abbildung die Längen nicht verändert, wenn sie also
längentreu ist, dann bleiben in einem Dreieck auch alle Winkel bei
der Abbildung unverändert, also ist auch das Skalarprodukt unverändert.
In diesem Abschnitt untersuchen wir die Eigenschaften von linearen
Abbildungen, die das Skalarprodukt nicht verändern.

\subsection{Definition}
Lineare Abbildungen, welche Längen und Winkel nicht ändern, beschreiben
sogenannte {\em Bewegungen} des Raumes.
Damit zwei Vektoren $u$ und $v$ sollen nach Abbildung
mit der Matrix $A$ immer noch das gleiche Skalarprodukt haben, muss
\[
u^tv=(Au)^tAv=u^tA^tAv
\]
gelten.
Dies ist jedoch nur dann für alle Vektoren $u$ und $v$ möglich,
wenn $A^tA$ die Einheitsmatrix ist, also $A^tA=E$.
Das ist aber gleichbedeutend damit, dass $A^t=A^{-1}$.
Lineare Abbildungen, die Längen und
Winkel erhalten, haben also besonders einfach zu invertieren Matrizen.

\index{Matrix!orthognale}
\begin{definition}
\index{orthogonal}
Eine Matrix $A$ heisst orthogonal, wenn $A^tA=AA^t=E$.
\end{definition}

\begin{beispiel}
Die Matrix 
\[
A=
\begin{pmatrix}
\frac{\sqrt{2}}2& \frac{\sqrt{2}}2\\
-\frac{\sqrt{2}}2& \frac{\sqrt{2}}2
\end{pmatrix}
\]
ist orthogonal:
\[
A^tA=
\begin{pmatrix}
\frac{\sqrt{2}}2&-\frac{\sqrt{2}}2\\
\frac{\sqrt{2}}2& \frac{\sqrt{2}}2
\end{pmatrix}
\begin{pmatrix}
\frac{\sqrt{2}}2& \frac{\sqrt{2}}2\\
-\frac{\sqrt{2}}2& \frac{\sqrt{2}}2
\end{pmatrix}
=
\begin{pmatrix}
\frac24+\frac24&\frac24-\frac12\\
\frac24-\frac24&\frac24+\frac12
\end{pmatrix}
=E.
\qedhere
\]
\end{beispiel}

\subsection{Drehungen im zweidimensionalen Raum}
Die Matrix
\[
D_\alpha
=
\begin{pmatrix}
\cos\alpha&-\sin\alpha\\
\sin\alpha& \cos\alpha
\end{pmatrix}
\]
ist orthogonal:
\[
D_\alpha^tD_\alpha
=
\begin{pmatrix}
 \cos\alpha&\sin\alpha\\
-\sin\alpha&\cos\alpha
\end{pmatrix}
\begin{pmatrix}
\cos\alpha&-\sin\alpha\\
\sin\alpha& \cos\alpha
\end{pmatrix}
=
\begin{pmatrix}
\cos^2\alpha+\sin^2\alpha&0\\
0&\sin^2\alpha+\cos^2\alpha
\end{pmatrix}
=E.
\]
Diese Matrix beschreibt eine Drehung um den Winkel $\alpha$
in zwei Dimensionen.

\subsection{Spiegelungen}
\index{Spiegelung}
Die Matrix einer Spiegelung an einer Geraden ist orthogonal.

\smallskip

{\parindent 0pt
Sei} $v$ ein zweidimensionaler Vektor der Länge $1$.
Dann ist die Abbildung 
\[
u\mapsto u-2v(v\cdot u)
\]
die Spiegelung an der Geraden mit der Normalen $u$.
Ist nämlich
ein Vektor $u\perp v$, ist das Skalarprodukt $0$, und $u\mapsto u$
bleibt unverändert.
Andererseits wird der Vektor $v$ auf
$v-2v(v\cdot v)=v-2v=-v$ abgebildet.
Vektoren senkrecht auf $v$
werden also belassen, solche parallel zu $v$ werden umgekehrt.

Die Matrix dieser linearen Abbildung ist
\[
S_v=\begin{pmatrix}1&0\\0&1\end{pmatrix}
-
2\begin{pmatrix}v_1\\v_2\end{pmatrix}
\begin{pmatrix}v_1&v_2\end{pmatrix}
=
\begin{pmatrix}
1-2v_1^2&-2v_1v_2\\
-2v_1v_2&1-2v_2^2
\end{pmatrix},
\]
wir wollen nachrechnen, dass sie orthogonal ist.
Dabei beachten
wir, dass $1-v_1^2=v_2^2$ ist, weil ja $|v|=1$, also kann man $S_v$
auch schreiben
\[
S_v=\begin{pmatrix}
v_2^2-v_1^2&-2v_1v_2\\
-2v_1v_2&v_1^2-v_2^2
\end{pmatrix}
\]

\begin{align*}
S_v^tS_v
&=
\begin{pmatrix}
v_2^2-v_1^2&-2v_1v_2\\
-2v_1v_2&v_1^2-v_2^2
\end{pmatrix}
\begin{pmatrix}
v_2^2-v_1^2&-2v_1v_2\\
-2v_1v_2&v_1^2-v_2^2
\end{pmatrix}
\\
&=
\begin{pmatrix}
v_2^4-2v_1^2v_2^2+v_1^4+4v_1^2v_2^2&
-2(v_2^2-v_1^2)v_1v_2
+
-2(v_1^2-v_2^2)v_1v_2
\\
-2(v_1^2-v_2^2)v_1v_2
+
-2(v_2^2-v_1^2)v_1v_2
&
v_2^4-2v_1^2v_2^2+v_1^4+4v_1^2v_2^2
\end{pmatrix}
\\
&=
\begin{pmatrix}
v_1^4+2v_1^2v_2^2+v_2^4&0\\
0&v_1^4+2v_1^2v_2^2+v_2^4\\
\end{pmatrix}
\\
&=
\begin{pmatrix}
(v_1^2+v_2^2)^2&0\\
0&(v_1^2+v_2^2)^2
\end{pmatrix}=E
\end{align*}
Die Matrix $S_v$ ist also tatsächlich immer orthogonal.

Analog gilt in beliebig vielen Dimensionen, dass die Matrix
$S_v=E-2v v^t$  für einen Vektor $v$ der Länge $1$ eine orthogonale
Matrix ist, die die Spiegelung an einer Ebene mit Normale $v$
beschreibt.

\subsection{Eigenschaften orthogonaler Matrizen}
Eine besondere Eigenschaft orthogonaler Matrize ist, dass sich die
inverse Matrix ganz besonders leicht berechnen lässt, wie der folgende
Satz zeigt.
\begin{satz}
Die Spalten einer orthogonale Matrix $A$ sind orthonormiert, sie sind
orthogonale Vektoren der Länge 1.
Ausserdem ist $A^{-1}=A^t$.
\end{satz}

\begin{proof}[Beweis]
Multipliziert man $A^tA=E$ von rechts mit $A^{-1}$, bekommt man
$A^tAA^{-1}=A^t=A^{-1}$.

Wir interpretieren die Bedingung $A^tA=E$.
Für ein beliebiges $i$
bedeutet sie, dass Zeile $i$ von $A^t$ mal Spalte $i$ von $A$ $1$ ergibt.
Dies ist aber das Skalarprodukt von Spalte $i$ von $A$ mit Spalte $i$
von $A$, also die Länge vom Spaltenvektor mit der Nummer $i$.

Seien jetzt $i\ne j$ zwei verschiedene Indizes.
Dann bedeutet $A^tA=E$,
dass Zeile $i$ von $A^t$ mal Spalte $j$ von $A$ $0$ ergibt.
Oder das
Skalarprodukt von Spalte $i$ von $A$ mit Spalte $j$ von $A$ ist $0$.
Dies wiederum heisst, dass die Spaltenvektoren senkrecht stehen.
\end{proof}

%
% Vertauschung der Koordinatenachsen
%
\subsection{Vertauschung der Koordinatenachsen also orthogonale Abbildung}
Für Transformation, die die Koordinatenachsen vertauscht, ist
die Transformationsmatrix ebenfalls leicht zu ermitteln.
In der Basis
\[
\left\{
b_1'=\begin{pmatrix}0\\0\\1\end{pmatrix},
b_2'=\begin{pmatrix}0\\1\\0\end{pmatrix},
b_3'=\begin{pmatrix}1\\0\\0\end{pmatrix},
\right\}
\]
sind die $x$- und die $z$-Achse vertauscht worden.
die zugehörige Transformationsmatrix ist
\begin{equation}
T=\begin{pmatrix}
0&0&1\\
0&1&0\\
1&0&0
\end{pmatrix}^{-1}
=
\begin{pmatrix}
0&0&1\\
0&1&0\\
1&0&0
\end{pmatrix}.
\label{transformation-vertauschung}
\end{equation}

%
% Drehung des dreidimensionalen Raumes
%
\subsection{Drehungen des dreidimensionalen Raumes}
Die Drehungen des Raumes sind etwas komplizierter als die Drehungen der
Ebene, da jede beliebige Richtung als Drehachse auftreten kann.
Wir beginnen daher damit, Drehungen um die Koordinatenachsen zu beschreiben
und leiten daraus eine Methode her, den Drehwinkel zu bestimmen.
Die Bestimmung der Drehachse werden wir im Zusammenhang mit dem
Eigenwertproblem in Kapitel~\ref{chapter-eigen} genauer untersuchen.

\subsubsection{Drehungen um die Koordinatenachsen}
Drehungen um die Koordinatenachsen können mit Hilfe der zweidimensionalen
Drehmatrix $D_\alpha$ gefunden werden.
Die Drehung um die $x$-Achse um den
Winkel $\alpha$ ändert die $x$-Koordinate nicht, hat also die
Matrix
\[
D_{x,\alpha}=
\begin{pmatrix}
1&0&0\\
0&\cos\alpha&-\sin\alpha\\
0&\sin\alpha&\cos\alpha
\end{pmatrix}
\]
Analog können Drehungen um die anderen zwei Achsen formuliert werden:
\[
D_{y,\alpha}=\begin{pmatrix}
\cos\alpha&0&-\sin\alpha\\
0&1&0\\
\sin\alpha&0&\cos\alpha
\end{pmatrix}
,\qquad
D_{z,\alpha}=\begin{pmatrix}
\cos\alpha&-\sin\alpha&0\\
\sin\alpha&\cos\alpha&0\\
0&0&1
\end{pmatrix}
\]
Man kann diese Matrizen aus $D_{x,\alpha}$ aber auch mit Hilfe einer
Koordinatentransformation bekommen.
Die Transformationsmatrix
haben wir in (\ref{transformation-vertauschung}) gefunden:
\[
T=
\begin{pmatrix}
0&0&1\\
0&1&0\\
1&0&0
\end{pmatrix}
=T^{-1}
\]
Mit der Basiswechsel-Formel (\ref{abbildung-basiswechsel}) kann
man jetzt die Drehmatrix um die $z$-Achse aus der Drehmatrix um die
$x$-Achse berechnen:
\begin{align*}
D_{z,-\alpha}=TD_{x,\alpha}T^{-1}
&=
\begin{pmatrix}
0&0&1\\
0&1&0\\
1&0&0
\end{pmatrix}
\begin{pmatrix}
1&0&0\\
0&\cos\alpha&-\sin\alpha\\
0&\sin\alpha&\cos\alpha
\end{pmatrix}
\begin{pmatrix}
0&0&1\\
0&1&0\\
1&0&0
\end{pmatrix}
\\
&=
\begin{pmatrix}
0&\sin\alpha&\cos\alpha\\
0&\cos\alpha&-\sin\alpha\\
1&0&0
\end{pmatrix}
\begin{pmatrix}
0&0&1\\
0&1&0\\
1&0&0
\end{pmatrix}
=
\begin{pmatrix}
\cos\alpha&\sin\alpha&0\\
-\sin\alpha&\cos\alpha&0\\
0&0&1
\end{pmatrix}.
\end{align*}
\subsubsection{Drehwinkel ermitteln}
Für die Standard-Drehmatrizen $D_{x,\alpha}$ ist der Drehwinkel 
leicht mit Hilfe der Spur zu ermitteln.
Die Spur ist die Summe der
Diagonalelemente einer Matrix:
\[
\operatorname{Spur}
\begin{pmatrix}
a_{11}&\dots &a_{1n}\\
\vdots&\ddots&\vdots\\
a_{n1}&\dots &a_{nn}\\
\end{pmatrix}
=a_{11}+a_{22}+\dots+a_{nn}
\quad
\Rightarrow
\quad
\operatorname{Spur}D_{x,\alpha}=1+2\cos\alpha.
\]
Die Spur hat aber interessante algebraische Eigenschaften, welche
die Berechnung des Drehwinkels auch für beliebige Drehmatrizen
erlauben:
\begin{satz}
\label{spursatz}
Seinen $A$, $B$ und $C$ beliebige $n\times n$-Matrizen.
Dann gilt
\begin{compactenum}
\item $\operatorname{Spur}(ABC)=\operatorname{Spur}(BCA)=\operatorname{Spur}(CAB)$
\item $\operatorname{Spur}(AB)=\operatorname{Spur}(BA)$
\end{compactenum}
\end{satz}
Wir zeigen zunächst, wie man damit den Drehwinkel einer beliebigen
Drehung $D$ des dreidimensionalen Raumes bestimmen kann.
Sei $T$ eine Koordinatentransformation, welche die Drehachse der Drehung in die
$x$-Achse transformiert.
Dann hat die transformierte Drehmatrix die Form $D_{x,\alpha}$:
\[
TDT^{-1}= D_{x,\alpha}
\]
Für die Spur gilt dann
\begin{align*}
\operatorname{Spur}(TDT^{-1})
&=
\operatorname{Spur}(DT^{-1}T)
=
\operatorname{Spur}(D)
\\
\operatorname{Spur}(TDT^{-1})
&=
\operatorname{Spur}(D_{x,\alpha})=1+2\cos\alpha
\end{align*}
Aufgelöst nach $\cos\alpha$:
\begin{satz}\label{drehwinkelsatz}
Der Drehwinkel einer Drehung des $\mathbb R^3$ mit Matrix $D$ ist
\[
\cos\alpha =\frac{\operatorname{Spur}(D) -1 }2.
\]
\end{satz}

\begin{proof}[Beweis des Satzes \ref{spursatz}]
Die zweite Aussage folgt aus der ersten, indem man $C=E$ setzt:
\[
\operatorname{Spur}(AB)=\operatorname{Spur}(ABE)=\operatorname{Spur}(BEA)=
\operatorname{Spur}(BA).
\]
Die zyklische Vertauschung der Faktoren folgt aus den Formeln für die
Matrix-Multiplikation.
Das Element $ik$ der Matrix $AB$ ist
\[
\sum_{j=1}^na_{ij}b_{jk}
\]
und entsprechend ist das Matrixelement $il$ von $ABC$ 
\[
\sum_{j=1, k=1}^na_{ij}b_{jk}c_{kl}
\]
Die Diagonalelemente sind jene mit $i=l$, die Spur ist deren Summe:
\[
\operatorname{Spur}(ABC)
=\sum_{i,j,k=1}^na_{ij}b_{jk}c_{ki}
=\sum_{i,j,k=1}^nc_{ki}a_{ij}b_{jk}
=\sum_{j,k,i=1}^nc_{ij}a_{jk}b_{ki}
=\operatorname{Spur}(CAB)
\qedhere
\]
\end{proof}

\subsubsection{Euler-Winkel}
Um die Lage eines Körpers im Raum festzulegen wurde schon früh
die folgende nach Euler benannte  Parametrisierung verwendet.
Die Lage wird durch drei aufeinanderfolgende Drehungen um die $z$-,
die $x$- und dann nochmals die $z$-Achse herbeigeführt.
Die drei
Drehungen sind
\begin{compactenum}
\item eine Drehung um die $z$-Achse um den Winkel $\alpha$
\item eine Drehung um die $x$-Achse um den Winkel $\beta$
\item eine Drehung um die $z$-Achse um den Winkel $\gamma$
\end{compactenum}
Mit der Matrizen-Darstellung der einzelnen Drehungen kann auch
die Matrix der gesamten Drehung berechnet werden.
Wir schreiben
$D_{x,\alpha}$ für eine Drehung um die $x$-Achse um den Winkel
$\alpha$, und sinngemäss für die anderen Achsen.
Es ist
\begin{align*}
O(\alpha,\beta,\gamma)
&=
D_{z,\gamma}
D_{x,\beta}
D_{z,\alpha}
\\
&=
\begin{pmatrix}
\cos\gamma&-\sin\gamma&0\\
\sin\gamma&\cos\gamma&0\\
0&0&1
\end{pmatrix}
\begin{pmatrix}
1&0&0\\
0&\cos\beta&-\sin\beta\\
0&\sin\beta&\cos\beta
\end{pmatrix}
\begin{pmatrix}
\cos\alpha&-\sin\alpha&0\\
\sin\alpha&\cos\alpha&0\\
0&0&1
\end{pmatrix}
\\
&=
\begin{pmatrix}
\cos\gamma&-\sin\gamma&0\\
\sin\gamma&\cos\gamma&0\\
0&0&1
\end{pmatrix}
\begin{pmatrix}
\cos\alpha&-\sin\alpha&0\\
\sin\alpha\cos\beta&\cos\alpha\cos\beta&-\sin\beta\\
\sin\alpha\sin\beta&\cos\alpha\sin\beta&\cos\beta
\end{pmatrix}
\\
&=
\begin{pmatrix}
\cos\alpha\cos\gamma+\sin\alpha\cos\beta\sin\gamma
        &-\sin\alpha\cos\gamma-\cos\alpha\cos\beta\sin\gamma
                &\sin\beta\sin\gamma\\
\cos\alpha\sin\gamma+\sin\alpha\cos\beta\cos\gamma
        &-\sin\alpha\sin\gamma+\cos\alpha\cos\beta\cos\gamma
                &-\sin\beta\cos\gamma\\
\sin\alpha\sin\beta&\cos\alpha\sin\beta&\cos\beta
\end{pmatrix}
\end{align*}
In der Astronomie und der Raumfahrt sind die Winkel sehr gebräuchlich.
Der Winkel $\beta$ ist die Neigung der Bahn gegenüber der Referenzebene,
bei Satellitenbahnen um die Erde ist dies die Äquatorebene,
bei interplanetaren Missionen die Ebene der Erdbahn.
Der Winkel $\gamma$ ist der Winkel zwischen einer Referenzrichtung
und der Schnittgeraden der Bahnebene mit der Referenzebene.
Diese Schnittgerade heisst auch die Knotenlinie.
Bei Planetenbahnen ist die Referenzrichtung die Richtung des Frühlingspunktes.
Der Winkel $\alpha$ ist der Winkel zwischen der Knotenlinie und dem
erd- oder sonnennächsten Punkt.

%
% Die orthogonale Gruppe
%
\subsection{Die orthogonale Gruppe\label{subsection:orthogonale gruppe}}
Die orthogonalen Matrizen bilden eine wichtige Teilmenge der invertierbaren
Matrizen, die sogenante orthogonale Gruppe.

\begin{definition}
Die Menge aller orthogonalen Matrizen
\[
\operatorname{O}(n) = \{ A\in M_n(\mathbb R)\;|\; AA^t = E \}
\subset \operatorname{GL}_n(\mathbb R)
\]
heisst die orthogonale Gruppe.
\end{definition}
Die orthogonale Gruppe enthält einerseits alle Drehmatrizen, aber auch
alle Spiegelungen, da diese ebenfalls das Skalarprodukt erhalten.
Des Weiteren sind Produkte von Drehmatrizen und Spiegelungsmatrizen
ebenfalls in $\operatorname{O}(n)$.
Wir verfügen im Moment nicht nicht über die notwendigen Hilfsmittel,
Drehmatrizen von Abbildungen mit einer Spiegelungskomponente zu 
unterscheiden, dies wird im nächsten Kapitel anchgeholt.





