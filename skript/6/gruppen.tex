%
% gruppen.tex
%
% (c) 2018 Prof Dr Andreas Müller, Hochschule Rapperswil
%
\section{Gruppen\label{section:gruppen}}
\rhead{Gruppen}
Gleichungen lösen beruht darauf, dass sich Rechenoperationen rückgängig
machen lassen.
Zum Beispiel kann man in den reellen Zahlen Additionen und Multiplikationen
rückgängig machen, wenigstens wenn der Faktor nicht $0$ ist.
Im Vektorraum $\mathbb R^n$ können Additionen und
Multiplikationen mit reellen Zahlen rückgängig gemacht werden.
Das Vektorprodukt in $\mathbb R^3$ kann jedoch nicht rückgängig gemacht
werden, da verschiedene Faktoren auf das gleichen Produkt führen können.
Es gilt zum Beispiel
\[
\vec{a} \times (\vec{b} + t\vec{a})
=
\vec{a} \times \vec{b} + t(\underbrace{\vec{a}\times\vec{a}}_{\displaystyle=0})
=
\vec{a} \times \vec{b},
\]
für beliebige Werte von $t$ entsteht also immer das gleiche Produkt.
Daher ist es nicht möglich, aus dem Produkt mit $\vec{a}$ den
zweiten Faktor zurückzugewinnen.
Eine Gruppe ist eine Menge von mathematischen Objekten mit einer
Operation, die sich immer rückgängig machen lässt.

In diesem Abschnitt beschreiben wir die Eigenschaften der linearen
Abbildungen, die wir in früheren Kapiteln kennengelernt haben und
zeigen, dass sie alle Gruppen bilden.

%
% Was ist eine Gruppe
%
\subsection{Was ist eine Gruppe?}
\begin{definition}
Eine Gruppe $G$ ist eine Menge mit einer Verknüpfung, die meistens
multiplikativ geschrieben wird, und die folgende Eigenschaften hat:
\begin{enumerate}
\item Die Verknüpfung ist assoziativ: $a(bc)=(ab)c$ für alle $a,b,c\in G$.
\item Es gibt ein Element $e\in G$, genannt das neutrale Element,
mit der Eigenschaft $ae=ea=a$ für alle
Elemente $a\in G$.
\item Zu jedem Element $a$ gibt es ein Element $a^{-1}\in G$, genannt
das inverse Element von $a$, mit der
Eigenschaft, dass $a^{-1}a=e$.
\end{enumerate}
\end{definition}

Schreibt man die Verknüpfung additiv, dann können die Axiome für
eine Gruppe wie folgt geschrieben werden.
\begin{enumerate}
\item Die Verknüpfung ist assoziativ:
$a+(b+c)=(a+b)+c$ für $a,b,c\in G$.
\item Es gibt ein Element $0\in G$, genannt das neutrale Element oder
das Nullelement, mit der Eigenschaft $0+a=a+0=a$ für alle Elemente $a\in G$.
\item Zu jedem Element $a\in G$ gibt es ein Element $-a\in G$, genant
das negative Element von $a$, mit der Eigenschaft, dass $-a + a=0$.
\end{enumerate}

\begin{beispiel}
Die nicht verschwindenden reellen Zahlen
$\mathbb R^*=\{x\in\mathbb R\;|\; x\ne 0\}$
ist eine Gruppe mit der Multiplikation als Verknüpfung.

\smallskip

{\parindent0pt Das neutrale} Element ist die Zahl $1$, das inverse Element
von $a\in\mathbb R^*$ ist $1/a$.
\end{beispiel}

\begin{beispiel}
Die reellen Zahlen $\mathbb R$ mit der Addition als Verknüpfung ist eine
Gruppe.
\end{beispiel}

\begin{beispiel}
Ein Vektorraum $V$ ist eine Gruppe bezüglich der Verknüpfung der Addition
von Vektoren.
\end{beispiel}

%
% die allgemeine lineare Gruppe
%
\subsection{Die allgemeine lineare Gruppe $\operatorname{GL}_n(\mathbb R)$}
Die Menge der Matrizen $M_{mn}(\mathbb R)$ bildet keine Gruppe.
Zunächst ist die Multiplikation von Matrizen in $M_{mn}(\mathbb R)$
nicht definiert, ausser wenn $m=n$ gilt.
Für $m=n$ ist ein neutrales Element leicht zu finden.
Die Einheitsmatrix $E$ hat die Eigenschaft $AE=EA=A$.
Um eine Gruppe zu erhalten, muss man sich zusätzlich auf die 
Matrizen beschränken, welche invertierbar sind.

\begin{definition}
Die allgemeine lineare Gruppe ist die Menge
\[
\operatorname{GL}_n(\mathbb R)
=
\{ A\in M_n(\mathbb R)\;|\; \text{$A$ ist invertierbar}\}.
\]
\end{definition}

Für $n=1$ ist $\operatorname{GL}_1(\mathbb R) = \mathbb R^*$.
Um zu entscheiden, ob eine Matrix invertierbar ist, kann die Determinante
verwendet werden.
Die Determinante ist eine Abbildung
\[
\det
\colon
M_n(\mathbb R) \to \mathbb R
:
A \mapsto \det(A).
\]
Die allgemeine lineare Gruppe besteht aus denjenigen Matrizen,
die eine von $0$ verschiedene Determinante haben.
Das sind genau die Matrizen, die von der Determinante
auf einen Wert in $\mathbb R^*$ abgebildet werden.
Die Determinante ist daher eine surjektive Abbildung
\[
\det
\colon
\operatorname{GL}_n(\mathbb R)
\to
\mathbb R^*.
\]

Die Produktregel für die Determinante besagt, dass 
\begin{align*}
\det(AB)&=\det(A)\det(B),
\\
\det(A^{-1})&=\det(A)^{-1}.
\end{align*}
Die Determinante ist daher verträglich mit den Gruppenoperationen
von $\operatorname{GL}_n(\mathbb R)$ und $\mathbb R^*$.

%
% orthogonale Gruppe
%
\subsection{Die orthogonale Gruppe $\operatorname{O}(n)$
\label{subsection:orthogonale gruppe}}
Orthogonale Matrizen sind immer invertierbar, denn $A^t$ ist die 
inverse Matrix zu einer orthogonalen Matrix $A$.
Die orthogonalen Matrizen bilden daher eine Teilmenge der invertierbaren
Matrizen, die sogenante orthogonale Gruppe.

\begin{definition}
Die Menge aller orthogonalen Matrizen
\[
\operatorname{O}(n) = \{ A\in M_n(\mathbb R)\;|\; AA^t = E \}
\subset \operatorname{GL}_n(\mathbb R)
\]
heisst die orthogonale Gruppe.
\end{definition}
Die orthogonale Gruppe enthält einerseits alle Drehmatrizen, aber auch
alle Spiegelungen, da diese ebenfalls das Skalarprodukt erhalten.
Des Weiteren sind Produkte von Drehmatrizen und Spiegelungsmatrizen
ebenfalls in $\operatorname{O}(n)$.

Aus der Produktformel für die Determinante folgt
\[
A^tA=E
\quad\Rightarrow\quad
1
=
\det(E)
=
\det(A^tA)
=
\det(A^t)\det(A)
=
\det(A)^2
\quad\Rightarrow\quad
\det(A)=\pm 1.
\]
Orthogonale Matrizen haben also immer Determinante $\pm 1$.
Sie erhalten
also auf jeden Fall das Volumen, aber nicht unbedingt das orientierte
Volumen.
Spiegelungen kehren offensichtlich das Volumen um, sie haben Determinante
$-1$.

%
% spezielle lineare Gruppe
%
\subsection{Die spezielle lineare Gruppe $\operatorname{SL}_n(\mathbb R)$}
Matrizen in der allgemeinen linearen Gruppe $\operatorname{GL}_n(\mathbb R)$
können das Volumn verändern oder die Orientierung umkehren.
Die Orientierung oder das orientiert Volumen werden nur dann erhalten
sein, wenn die Determinante $1$ ist.

\begin{definition}
Die spezielle lineare Gruppe ist
\[
\operatorname{SL}_n(\mathbb R)
=
\{ A\in\operatorname{GL}_n(\mathbb R) \;|\; \det (A)=1\}.
\]
\end{definition}

Tatsächlich ist diese Menge eine Gruppe.
Dazu müssen wir überprüfen, dass das Produkt von Matrizen in
$\operatorname{SL}_n(\mathbb R)$
wieder in
$\operatorname{SL}_n(\mathbb R)$
ist.
Dies folgt aber aus der Produkteigenschaft der Determinante:
\[
\det(AB) 
=
\det(A) \det(B)
=
1\cdot 1
=
1
\]
für $A,B\in\operatorname{SL}_n(R)$.
Analog folgt auch, dass $A^{-1}\in\operatorname{SL}_n(\mathbb R)$ wenn
$A\in\operatorname{SL}_n(\mathbb R)$.

%
% Gruppe der Drehmatrizen
%
\subsection{Die Drehgruppe $\operatorname{SO}(n)$}
Drehungen des Raumes sind lineare Abbildungen, die 
das Skalarprodukt und die Orientierung erhalten.
Sie bilden die {\em spezielle orthogonale Gruppe}
\[
\operatorname{SO}(n) 
=
\operatorname{O}(n) \cap \operatorname{SL}_n(\mathbb R)
=
\{
A\in M_n(\mathbb R)\;|\; A^tA=E\wedge \det(A) = 1
\}.
\]

Die Differenz $\operatorname{O}(n) \setminus \operatorname{SO}(n)$
besteht aus Matrizen, die zwar das Skalarprodukt erhalten, aber die
Orientierung umkehren.
Ist $S_x$ die Spiegelung an der Ebene senkrecht zur $x$-Achse und ist
$A\in \operatorname{O}(n) \setminus \operatorname{SO}(n)$, dann ist
\[
\det(S_xA)= \det(S_x)\det(A) = (-1)\cdot(-1)=1
\qquad\Rightarrow\qquad
S_xA\in\operatorname{SO}(n).
\]
Dies bedeutet, dass die Multiplikation mit $S_x$ eine bijektive Abbildung
\[
\operatorname{SO}(n) \to \operatorname{O}(n)\setminus\operatorname{SO}(n)
:
A\mapsto S_xA
\]
definiert.
Die Gruppe $\operatorname{O}(n)$ besteht also aus zwei gleichartig gebauten
Teilmengen, $\operatorname{SO}(n)$  und
$\operatorname{O}(n)\setminus\operatorname{SO}(n)$.

Man kann die Gruppe $\operatorname{SO}(3)$ die Gruppe der Robotik nennen,
denn bei allen Bewegungen, die ein Roboter im Raum ausführen kann,
bleiben die Abmessungen erhalten (er soll ja nicht deformiert werden)
und die Orientierung kann nicht umkehren.
Die Bewegung eines Roboters besteht daher immer aus einer
Translationskomponente und einer Drehung in $\operatorname{SO}(3)$.
Die Herausforderung dabei ist natürlich, dass die Gruppe
$\operatorname{SO}(3)$ ziemlich kompliziert und vor allem nicht
kommutativ ist.

Für die Beschreibung einer Drohne ist es unabdingbar, in der
Gruppe $\operatorname{SO}(3)$ arbeiten zu können.
Wenn man die Bewegung eines Roboters aber weiter einschränken kann,
zum Beispiel indem man nur Bewegungen in der Ebene und Drehungen
um eine vertikale Achse zulässt, wie dies zum Beispiel bei
Eurobot gemacht wird, dann reicht es mit der Gruppe der Verschiebungen
und zweidimensionalen Drehungen um $\operatorname{SO}(2)$ zu arbeiten.
Diese letzte Gruppe ist viel einfacher, da sich Drehungen in der Ebene
durch den Drehwinkel beschreiben lassen und die Gruppe daher kommutativ ist.

