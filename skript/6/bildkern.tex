%
% bildkern.tex
%
% (c) 2018 Prof Dr Andreas Müller, Hochschule Rapperswil
%
\section{Bild und Kern\label{section:bildundkern}}
\rhead{Bild und Kern}
In Kapitel~\ref{chapter-lingl} wurde die Lösungsmenge eines linearen
Gleichungssystems mit der Matrix $A$ untersucht.
In Kapitel~\ref{chapter:affin} 
haben wir einer Matrix $A$ auch eine geometrische Bedeutung
als Abbildungsmatrix gegeben.
Die Frage der Lösbarkeit von Gleichungssystemen und der Begriff des
Ranges sollten daher auch eine geometrische Aussage übersetzt
werden können, was wir in diesem Abschnitt tun wollen.

\subsection{Bildraum}
Die Matrix $A$ einer linearen Abbildung $\varphi$ enthält in den Spalten
die Bilder der Standardbasisvektoren.
Vektoren, die als Bilder der Abbildung $\varphi$ auftreten können,
müssen Linearekombinationen der Spaltenvektoren von $A$ sein.
\begin{definition}
Die {\em Bildmenge} $\operatorname{im}\varphi$ einer linearen Abbildung
$\varphi$ ist der Vektorraum
\[
\operatorname{im}\varphi
=
\{\varphi(v)\;|\;v\in\mathbb R^n\}.
\]
\end{definition}

Die Bildmenge $\operatorname{im}\varphi$ der Abbildung $\varphi$ ist
daher nichts anderes als die Menge aller möglichen Linearkombinationen
von Spaltenvektoren.
Diese haben wir Seite \pageref{skript:affin:koordinaten:aufgespannt}
als den erzeugten Raum der Spaltenvektoren kennengelernt.
Wir fassen zusammen:
\begin{satz}
Die Bildmenge einer linearen Abbildung $\varphi$ ist der von den
Spaltenvektoren der zugehörigen Abbildungsmatrix $A$ aufgespannte Raum:
\[
\operatorname{im}\varphi
=
\langle\vec{a}_1,\dots,\vec{a}_n\rangle
=
\langle\mathcal{A}\rangle,
\]
wobei $\mathcal{A}$ für die Menge der Spaltenvektoren von $A$ geschrieben
haben.
\end{satz}

\subsection{Lösbarkeit von Gleichungssystemen}
Ein Gleichungssystem mit Koeffizientenmatrix $A$ und rechter Seite $b$
ist lösbar genau dann, wenn es einen Spaltenvektor $x$ gibt, derart
dass $Ax=b$.
Das Gleichungssystem ist also genau dann lösbar, wenn die rechte Seite
im Bildraum von $A$ liegt.

\begin{beispiel}
Ist der Vektor
\[
b=\begin{pmatrix}4\\2\\4\end{pmatrix}
\qquad\text{im Bildraum der Abbildung mit Matrix}\qquad
A=
\begin{pmatrix}
  -43& -26&  56\\
  -22& -12&  28\\
  -44& -26&  57
\end{pmatrix}\text{?}
\]
Die Frage ist gleichbedeutend damit, ob das Gleichungsystem mit
Koeffizientenmatrix $A$ und rechter Seite $b$ lösbar ist.
Das Gauss-Tableau
\[
\begin{tabular}{|>{$}c<{$}>{$}c<{$}>{$}c<{$}|>{$}c<{$}|}
\hline
  -43& -26&  56&4\\
  -22& -12&  28&2\\
  -44& -26&  57&4\\
\hline
\end{tabular}
\quad\rightarrow\quad
\begin{tabular}{|>{$}c<{$}>{$}c<{$}>{$}c<{$}|>{$}c<{$}|}
\hline
   1&  0& -1&  0\\
   0&  1& -\frac12&   0\\
\hdashline
   0&  0&  0&  1\\
\hline
\end{tabular}
\]
zeigt, dass dies wegen der $1$ in der rechten unteren Ecke nicht möglich
ist.
\end{beispiel}

\subsection{Dimension des Bildraumes}
Der Bildraum ist der von den Spaltenvektoren der Abbildungsmatrix
aufgespannte Raum.
Es kann durchaus sein, dass nicht alle Spaltenvektoren linear unabhängig
sind, die Dimension des Bildraumes kann also auch kleiner sein also die
Anzahl der Spaltenvektoren.

\begin{aufgabe}
Wie gross ist die Dimension des Bildraumes einer linearen Abbildung
$\varphi$ mit Abbildungsmatrix $A$?
\end{aufgabe}
Die Dimension eines Vektorraumes ist die maximale Anzahl linear
unabhängiger Vektoren einer Basis.
Da die Spaltenvektoren von $A$ alles erzeugen, müssen wir nur noch
ein paar Vektoren weglassen, welche nicht linear unabhängig sind von
den anderen.
In Kapitel~\ref{chapter-lingl} haben wir gelernt, dass der Rang der
Matrix $A$ genau die Anzahl der linear unabhängigen Spalten ist.
er ist daher auch die Dimension des Bildraumes.

\begin{satz}
Ist $A$ die Abbildungsmatrix einer linearen Abbildung $\varphi$, dann
gilt
\[
\dim\operatorname{im}\varphi = \operatorname{Rang}A.
\]
\end{satz}

\subsection{Nullmenge}
\begin{definition}
Sei $\varphi$ eine lineare Abbildung.
Die Menge
\[
\operatorname{ker}\varphi = \{\vec{v}\;|\; \varphi(\vec{v}) = 0\}
\]
heisst {\em Nullmenge}, {\em Nullraum} oder {\em Kern} der Abbildung
$\varphi$.
\end{definition}

Der Kern von $\varphi$ ist ein Vektorraum, denn wenn zwei Vektoren
$\vec{u},\vec{v}\in\ker\varphi$ im Kern sind, dann gilt dies
auch für die Summe und für die Vielfachen:
\[
\begin{aligned}
\varphi(\vec{u}+\vec{v})
&=
\varphi(\vec{u})+\varphi(\vec{v})
=
0+0=0
&&\Rightarrow&
\vec{u}+\vec{v}&\in\ker\varphi
\\
\varphi(\lambda\vec{u})
&=
\lambda\varphi(\vec{u}) = \lambda\cdot 0=0
&&\Rightarrow&
\lambda\vec{u}&\in\ker\varphi
\end{aligned}
\]
Damit stellt sich automatisch die folgende Aufgabe.

\begin{aufgabe}
Sei $\varphi$ eine lineare Abbildung mit Abbildungsmatrix $A$.
Man finde eine Basis von $\ker\varphi$.
\end{aufgabe}

Der Kern besteht aus denjenigen Vektoren $\vec{x}$, die $A\vec{x}=0$
erfüllen.
Der Kern von $\varphi$ ist daher nichts anderes als die Lösungsmenge
des Gleichungssystems $Ax=0$, welche man mit dem Gauss-Algorithmus
bestimmen kann, wie in Abschnitt~\ref{section:loesungsmenge} gezeigt
wurde.
Dort wurde auch gezeigt, dass die Anzahl der frei wählbaren Variablen
auch die Anzahl der Basisvektoren der Lösungsmenge ist.
Wenn $A$ eine $m\times n$-Matrix ist mit $\operatorname{Rang}A=r$,
dann ist die Anzahl der frei wählbaren Variablen $n-r$, es gilt
also
\[
\dim\ker \varphi = n -\operatorname{Rang}A.
\]

\begin{beispiel}
Man finde eine Basis des Kernes der linearen Abbildung mit Matrix
\[
A=\begin{pmatrix}
    9& -31& -26&   0\\
    2& -10& -12& -28\\
    8& -29& -26& -13
\end{pmatrix}
\]
\smallskip

{\parindent0pt Das} zugehörige Tableau ist
\begin{equation}
\begin{tabular}{|>{$}c<{$}>{$}c<{$}>{$}c<{$}>{$}c<{$}|>{$}c<{$}|}
\hline
    9& -31& -26&   0&0\\
    2& -10& -12& -28&0\\
    8& -29& -26& -13&0\\
\hline
\end{tabular}
\quad\rightarrow\quad
\begin{tabular}{|>{$}c<{$}>{$}c<{$}>{$}c<{$}>{$}c<{$}|>{$}c<{$}|}
\hline
    1&   0&   4&  31&0\\
    0&   1&   2&   9&0\\
\hdashline
    0&   0&   0&   0&0\\
\hline
\end{tabular}
\label{skript:affin:kern:tableau}
\end{equation}
Die letzten zwei Variablen sind frei wählbar, wir nennen sie $x_3$ und $x_4$,
dann wird die Lösungsmenge
\[
\ker\varphi
=
\mathbb L
=
\left\{
\left.
x_3
\begin{pmatrix}
-4\\-2\\1\\0
\end{pmatrix}
+
x_4
\begin{pmatrix}
-31\\-9\\0\\1
\end{pmatrix}
\;
\right|
\;
x_3,x_4\in\mathbb R
\right\}.
\]
Der Kern von $\varphi$ ist also ein zweidimensionaler Raum mit der Basis
\[
\mathcal{B} = \left\{
\begin{pmatrix}
-4\\-2\\1\\0
\end{pmatrix},
\begin{pmatrix}
-31\\-9\\0\\1
\end{pmatrix}
\right\}.
\]
Das Tableau \eqref{skript:affin:kern:tableau} zeigt auch, dass
$\operatorname{Rang}A=2$,
passend zur Dimension $\dim\ker\varphi = n - \operatorname{Rang}A = 4-2=2$ von
$\ker\varphi$.
\end{beispiel}


