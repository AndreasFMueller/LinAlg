%
% linabb.tex
%
% (c) 2018 Prof Dr Andreas Müller, Hochschule Rapperswil
%
\section{Lineare Abbildungen}
\rhead{Lineare Abbildungen}
Wir erinnern daran, dass eine lineare Abbildung $\varphi$ eine
Abbildung von Vektoren mit den Eigenschaften
\begin{equation}
\begin{aligned}
\varphi(u+v)&=\varphi(u)+\varphi(v)   &&\forall\;u,v\in\mathbb R^n
\\
\varphi(\lambda v)&=\lambda\varphi(v) &&\forall\;v\in\mathbb R^n,\;\lambda\in\mathbb R
\end{aligned}
\label{skript:linabb:def}
\end{equation}
ist.
Die Beschränkung auf Vektoren in $\mathbb R^n$ erscheint aber etwas
künstlich, denn die Eigenschaften
\eqref{skript:linabb:def}
werden zum Beispiel auch von der Ableitung von Funktionen erfüllt:
\begin{equation}
\begin{aligned}
\frac{d}{dx} ( f(x) + g(x)) &= \frac{d}{dx}f(x) + \frac{d}{dx}g(x)
\\
\frac{d}{dx}(\lambda f(x)) &= \lambda\frac{d}{dx}f(x)
\end{aligned}
\label{skript:linabb:abl}
\end{equation}
für differenzierbare Funktionen $f(x)$ und $g(x)$ und jede relle Zahl
$\lambda\in\mathbb R$.

%
% Vektorraum
%
\subsection{Abstrakter Vektorraum}
Wir erweitern daher den Begriff des Vektorraumes wie folgt:

\begin{definition}
Ein Vektorraum über den rellen Zahlen ist eine Menge $V$, genannt
Vektoren, mit zwei Operationen,
der Addition von Vektoren und der Multiplikation mit Zahlen aus $\mathbb R$,
die folgende Regeln erfüllen:
\begin{enumerate}
\item Die Addition ist kommutativ: $u+v=v+u$ für alle $u,v\in V$.
\item Die Addition ist assoziativ: $(u+v)+w=u+(v+w)$ für alle $u,v,w\in V$.
\item Es gibt einen speziellen Vektor $0\in V$ mit $0+v=v$ und $v+0=v$
für alle $v\in V$, genannt der Nullvektor.
\item Zu jedem Vektor $u$ gibt es einen Vektor $-u$ mit der Eigenschaft
$u+(-u)=0$.
\item Die Multiplikation ist assoziativ: $\lambda(\mu v) = (\lambda\mu)v$
für alle $v\in V$.
\item Es gelten die Distributivgesetze:
\begin{align*}
\lambda(u+v) &= \lambda u + \lambda v
\\
(\lambda + \mu)u&=\lambda u + \mu u
\end{align*}
für $u,v\in V$ und $\lambda,\mu\in\mathbb R$.
\end{enumerate}
\end{definition}

%
% Beispiele von Vektorraume
%
\subsection{Beispiele von Vektorräumen}
Wir zeigen einige Beispiele von Vektorräumen.

\subsubsection{Signale}
Ein abgetastetes Signal ist eine Folge von Abtast-Werten 
\[
\bm{s} = (s_0,s_1,s_2,s_3,\dots)
\]
mit $s_i\in\mathbb R$.
Ein Verstärker mit Verstärkungsfaktor $\lambda\in\mathbb R$ erzeugt daraus
das Signal
\[
\lambda\bm{s} = (\lambda s_0,\lambda s_1,\lambda s_2,\lambda s_3\dots).
\]
Mit einem Operationsverstärker kann man eine Addierschaltung realisieren,
die aus zwei Signalen $\boldsymbol{s}$ und $\boldsymbol{r}$ das Signal
\[
\bm{s} + \bm{r}
=
(s_0+r_0,
s_1+r_1,
s_2+r_2,
s_3+r_3,\dots)
\]
erzeugt.
Die Menge der Signale ist daher auf natürliche Art und Weise ein Vektorraum.

Lineare Schaltungen zeichnen sich dadurch aus, dass sie auf Signalen
wie lineare Abbildungen operieren.
Ein linearer Filter $F$ zum Beispiel wirkt auf Signalen
$\bm{s}$ und $\bm{r}$ nach den Regeln
\begin{align*}
F(\bm{s} + \bm{r})&=F(\bm{s}) + F(\bm{r})
\\
F(\lambda \bm{s})&=\lambda F(\bm{s}).
\end{align*}
Mit Operationsverstärkern und passiven Komponenten implementierte
Schaltungen funktionieren genau so.

\subsubsection{Bilder}
Digitale Bilder bestehen aus einem Pixelwert für jeden Punkt eines
rechteckigen Gitters.
Sie können also durch eine Matrix beschrieben werden.
Das Matrixelement $b_{ij}$ entspricht dem Wert an Pixelposition $(i,j)$.
Die Menge
\[
M_{mn}(\mathbb R)
=
\{
A\;|\;
\text{$A$ ist eine reelle $m\times n$ Matrix}
\}
\]
ist ein Vektorraum.
Man kann auch eine Basis für $M_{mn}(\mathbb R)$ angeben.
Die Matrizen $E_{ij}$, die eine $1$ in Zeile $i$ und Spalte $j$
enthalten und sonst lauter Nullen, bilden eine Basis.

Viele Operationen der Bildverarbeitung sind linear.
Die von vielen Bildverarbeitungsprogrammen implementierten
Glättungsalgorithmen, mit denen man zum Beispiel das Rauschen reduzieren
kann oder Scanning-Artefakte reduzieren kann, arbeiten linear.

Auch Bilder auf einer Kugeloberfläche bilden einen Vektorraum.
Die Méndez-Transformation ist eine lineare Abbildung, mit deren Hilfe
das sogenannte Registrierungsproblem für Bilder auf der Kugeloberfläche
gelöst werden kann: Es verlangt, die Drehung zu finden, die zwei Bilder
zur Deckung bringt.

\subsubsection{Polynome mit reellen Koeffizienten}
Die Menge der Polynome mit reellen Koeffizienten
\[
\mathbb R[X]
=
\{
a_0 + a_1X + a_2X^2 + \dots + a_nX^n
\;|\;
a_i\in\mathbb R
\}
\]
ist ein Vektorraum.
Man beachte, dass $n$ beleibig gross werden kann.
Polynome werden durch
eine beliebig grosse Anzahl von Koeffizienten $a_i$ beschrieben.
Die Polynome $\mathbb R[X]$ können daher nicht ein endlichdimensionaler
Vektorraum sein.
Die unendlich vielen Monome $1$, $X$, $X^2$,\dots sind linear unabhängig
und jedes Polynom kann auf eindeutige Weise als Linearkombination dieser
Monome geschrieben werden.
Die Monome bilden daher eine Basis
\[
\mathbb R[X] = \langle 1,X,X^2,X^3,\dots\rangle.
\]

In der Menge $\mathbb R[X]$ gibt es auch noch eine Multiplikation.
Man kann Polynome miteinander multiplizeren.
Eine solche Operation ist in einem Vektorraum nicht nötig, man spricht
von einer {\em Algebra}.
\index{Algebra}%

In $\mathbb R[X]$ kann man Ableiten.
Der Ableitungsoperator
\[
D\colon \mathbb R[X] \to \mathbb R[X]:
a_0+a_1X+a_2X^2+\dots +a_nX^n
\mapsto
a_1+2a_2X+3a_3X^2\dots +na_nX^{n-1}
\]
ist linear.
Die konstanten Polynome werden von $D$ auf $0$ abgebildet.

Man kann ausserdem sehen, dass jedes beliebige Polynom als Resultat
einer Ableitung entstehen kann.
Dazu konstruiert man die lineare Abbildung $I$, welche den Basisvektor
$X^k$ auf
\[
I\colon X^k \to \frac{1}{k+1} X^{k+1}
\]
abbildet.
Wegen
\[
DIX^k = D\frac{1}{k+1}X^{k+1}=X^k
\]
kann man folgern, dass $DIp(X)=p(X)$ für jedes Polynom
$p(X)\in\mathbb R[X]$ gilt.
Insbesondere ist das Polynom $p(X)$ die Ableitung des Polynoms $Ip(X)$.
Die lineare Abbildung $I$ bestimmt natürlich eine Stammfunktion von $p(X)$.


\subsubsection{Stetige Funktionen}
Die Menge
\[
C_{\mathbb R}([a,b]) = \{f:[a,b]\to\mathbb R\;|\; \text{$f(x)$ ist stetig}\}
\]
der stetigen Funktionen auf einem Interval $[a,b]$ bilden einen
Vektorraum.
Die Stammfunktion
\[
I
\colon
C_{\mathbb R}([a,b]) \to C_{\mathbb R}([a,b])
:
f(x) \mapsto If(x) = \int_a^x f(\xi)\,d\xi
\]
ist eine lineare Abbildung, denn es gilt
\[
\int f(x)+g(x)\,dx = \int f(x)\,dx + \int g(x)\,dx,
\qquad
\int \lambda f(x)\,dx = \lambda\int f(x)\,dx.
\]
Die Stammfunktionen sind differenzierbar, man könnte sie also wieder ableiten.
Es gibt aber auch stetige Funktionen in $C_{\mathbb R}([a,b])$, die nicht
differenzierbar sind.
Es ist also nicht möglich einen Ableitungsoperator in $C_{\mathbb R}([a,b])$
zu definieren.

Man beachte auch, dass der Operator $I$ aus dem Beispiel über Polynome
$\mathbb R[X]$ nicht mit dem $I$ für stetige Funktionen übereinstimmt,
ausser wenn $a=0$ ist.

\subsubsection{Glatte Funktionen}
Die Menge
\[
C^{\infty}_{\mathbb R}([a,b])
=
\{ f\colon [a,b]\to\mathbb R\;|\;
\text{$f$ ist beliebig oft stetig differenzierbar}\}
\]
der glatten, reellwertigen Funktionen auf dem Interval $[a,b]$
ist ein Vektorraum, denn die Summe glatter Funktionen ist wieder
eine glatte Funktion.

Die Polynome sind natürlich glatte Funktionen, also
$\mathbb R[X]\subset C^{\infty}_{\mathbb R}([a,b])$.
Aber es gibt auch glatte Funktionen, die sich nicht als Polynome
darstellen lassen, zum Beispiel die Funktionen
$\sin x$, $\cos x$ und $e^x$.

In $C^{\infty}_{\mathbb R}([a,b])$ kann man beliebig integrieren und
differenzieren.
Es gibt daher zwei lineare Operatoren $D$ und $I$ mit den Eigenschaften
\[
Df(x) = f'(x),
\qquad
DIf(x) = f(x).
\]
$IDf(x)$ ist jedoch nicht das selbe wie $f(x)$.
Aus der Analysis ist bekannt, dass eine Stammfunktion nur bis auf
eine Konstante definiert ist.
Addiert man eine Konstante $c$ zur Funktion $f(x)$ hinzu, folgt
\[
ID(f(x)+c) = IDf(x) + \underbrace{IDc}_{\displaystyle=0} =  IDf(x).
\]
Funktionen, die sich um eine Konstante unterschieden, werden von $ID$ auf
die gleiche Funktion abgebildet.
Man kann also schliessen, dass $D$ zwar surjektiv aber nicht injektiv ist,
dass aber $I$ injektiv aber nicht surjektiv ist.

Hier sieht man einen wesentlichen Unterschied zu endlichdimensionalen
Vektorräumen.
Eine lineare Abbildung $\mathbb R^n\to\mathbb R^n$, die injektiv ist, ist
auch surjektiv und damit invertierbar.
In einem unendlichdimensionalen Vektorraum braucht das nicht mehr zu
gelten.


%
% Lineare Abbildungen
%
\subsection{Lineare Abbildungen}
Der Vollständigkeit halber erwähnen wir noch die abstrakte Definition
einer linearen Abbildung.

\begin{definition}
Es seien zwei Vektorräume $U$ und $V$ gegeben.
Eine lineare Abbildung ist eine Abbildung
$\varphi\colon U\to V$ die zusätzlich
\begin{equation}
\varphi(\lambda u+\mu v) = \lambda \varphi(u) + \mu \varphi(v)
\label{skript:linabb:def:linkomb}
\end{equation}
für alle $u,v\in U$ und $\lambda,\mu\in\mathbb R$ erfüllt.
\end{definition}

Man kann die Bedingung der Definition interpretieren als die Forderung,
dass $\varphi$ mit Linearkombinationen verträglich sein soll.
Diese einzige Bedingung kann man auch als zwei separate Bedingungen
formulieren:
\begin{align}
\varphi(u+v)&=\varphi(u)+\varphi(v)
\label{skript:linabb:def:summe}
\\
\varphi(\lambda u)&=\lambda \varphi(u)
\label{skript:linabb:def:produkt}
\end{align}
Die Bedingung~\eqref{skript:linabb:def:summe} besagt, dass die Abbildung
$\varphi$ mit Additionen verträglich ist, und 
\eqref{skript:linabb:def:produkt} besagt, dass $\varphi$ mit skalaren
Multiplikationen verträglich ist.
Die beiden Bedingungen sind natürlich äquivalent.
Zum Beispiel erhält man aus \eqref{skript:linabb:def:linkomb}
mit $\lambda=1$ und $\mu=0$ die Bedingung \eqref{skript:linabb:def:summe},
und mit $\mu =0$ erhält man
\eqref{skript:linabb:def:produkt}.
Umgekehrt berechnet man mit Hilfe von
\eqref{skript:linabb:def:summe}
und
\eqref{skript:linabb:def:produkt}
den Wert von $\varphi$ auf einer Linearkombination:
\[
\varphi(\lambda u + \mu v)
\overset{\eqref{skript:linabb:def:summe}}{=}
\varphi(\lambda u) + \varphi(\mu v)
\overset{\eqref{skript:linabb:def:produkt}}{=}
\lambda \varphi( u) + \mu \varphi( v).
\]







