%
% linabb.tex
%
% (c) 2018 Prof Dr Andreas Müller, Hochschule Rapperswil
%
\section{Lineare Abbildungen}
\rhead{Lineare Abbildungen}
\subsection{Definition}
Eine $m\times n$-Matrix $A$ berechnet aus einem gegeben Vektor
$v\in\mathbb R^n$ mit Hilfe des
Matrizenproduktes einen neuen Vektor $Av\in\mathbb R^m$.
$A$ definiert also auch eine Abbildung
\[
A\colon\mathbb R^n\to\mathbb R^m:v\mapsto Av
\]
mit den Eigenschaften
\[
\left.
\begin{aligned}
A(u+v)&=Au+Av\\
A(\lambda u)&=\lambda Au
\end{aligned}\right\}\quad
\Rightarrow\quad
A(\lambda u+\mu v)=\lambda Au+\mu Av.
\]
Eine solche Abbildung heisst linear.
\index{lineare Abbildung}

\subsection{Matrix einer linearen Abbildung}
Umgekehrt definiert jede lineare Abbildung $V\to V$ mit Hilfe einer
Basis auch eine Matrix.
Sei $\varphi\colon V\to V$ eine Abbildung
mit $\varphi(\lambda u+\mu v)=\lambda \varphi(u)+\mu\varphi(v)$,
und $B=\{b_1,\dots,b_n\}$ eine Basis.
Dann genügt die Kenntnis
von $\varphi(b_i)$ für alle $i$, um die lineare Abbildung für
jeden beliebigen Vektor berechnen zu können.

Man kann nämlich jeden Vektor mit Hilfe seiner Koordinanten aus
den Basisvektoren linear kombinieren, also gilt auch
\[
\varphi(v)
=
\varphi(\xi_1b_1+\dots+\xi_nb_n)
=
\xi_1\varphi(b_1)+\dots+\xi_n\varphi(b_n).
\]
Ausserdem kann man die Koordinaten von $\varphi(b_i)$ in der Basis
$B$ ermitteln, und diese Koordinatenvektoren in eine Matrix $A$
schreiben.
Dann ist $A\xi$ der Koordinatenvektor von $\varphi(v)$.
Die Matrix $A$ enthält also in den Spalten die Bilder der Basisvektoren.

\subsection{Basiswechsel}
Wie sieht die Matrix einer linearen Abbildung aus, wenn man statt
der Basis $B$ die Basis $B'$ verwendet? Sei $T$ die Transformationsmatrix,
die Koordinaten bezüglich $B$ in Koordinaten bezüglich $B'$ umrechnet.
Die lineare Abbildung hat bezüglich der Basis $B$ die Matrix $A$, wir
suchen aber die Matrix $A'$, die die lineare Abbildung bezüglich der
Basis $B'$ beschreibt:
\[
\xymatrix{
\mathbb R^n\ar[r]^{A} \ar[d]_{T}
	&\mathbb R^n \ar[d]^{T}
\\
\mathbb R^n\ar[r]^{A'}
	&\mathbb R^n
}
\]
Aus dem Diagramm können wir ablesen, dass $A'$ gleichbedeutend
ist damit, die Koordinaten zuerst mit $T^{-1}$ von der Basis $B'$
auf die Basis $B$ umzurechnen, dort die Matrix $A$ anzuwenden, und
dann erneut mit $T$ in die Basis $B'$ zurückzukehren:
\begin{equation}
A'=TAT^{-1}.
\label{abbildung-basiswechsel}
\end{equation}

