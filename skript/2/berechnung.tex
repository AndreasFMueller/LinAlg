%
% berechnung.tex
%
% (c) 2018 Prof Dr Andreas Müller, Hochschule Rapperswil
%
\section{Berechnung der Determinanten}
\subsection{Berechnung mit Gauss-Verfahren}
Das Beispiel der $2\times 2$-Determinante aus dem ersten Abschnitt
lässt sich auch für beliebig grosse Determinanten verallgemeinern,
woraus sich ein effizientes Berechnungsverfahren für die Determinante
ergibt.

Im Laufe des Gauss-Verfahrens ändert sich die Determinante offenbar
immer dann, wenn eine {\bf I}-Operation ausgeführt wird.
In solchen Schritten wird durch das Pivot-Element dividiert.
Am Ende des 
Verfahrens bleibt die Einheitsmatrix stehen, welche die Determinante
$1$ hat.
Durch fortgesetztes Dividieren durch die Pivot-Elemente wird
aus der $\det A$ also $1$:
\[
\frac{\det A}{\prod_{\text{$p$ Pivot-Element} }p}=1\qquad\Rightarrow
\qquad
\det A=\prod_{\text{$p$ Pivot-Element}} p%
\begin{picture}(0,0)
\color{red}\put(-4,2){\circle{13}}
\end{picture}%
\;.
\]
Es folgt
\begin{satz}
\label{detprodpivot}
Die Determinante von $A$ ist das Produkt der Pivot-Elemente,
die im Laufe des Gauss-Verfahrens auftreten.
\end{satz}

