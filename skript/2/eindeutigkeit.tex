%
% eindeutigkeit.tex
%
% (c) 2018 Prof Dr Andreas Müller, Hochschule Rapperswil
%
\section{Eindeutigkeit der Determinante\label{deteindeutig}}
In diesem Abschnitt wollen wir zeigen, dass die Determinante
eindeutig ist.
Wir gehen dazu wie folgt vor.
Zunächst zeigen wir,
dass sich jede Determinantenberechnung auf die Berechnung der Determinanten
von sehr speziellen $A$s zurückführen lässt, nämlich solchen,
die genau eine $1$ in jeder Zeile und Spalte haben, und sonst nur
Nullen enthalten.
Im zweiten Schritt zeigen wir dann, dass diese
speziellen Determinanten in jeder Definition dasselbe geben.

Die $k$-te Spalte $a$ von $A$ kann man als lineare Kombination von
speziellen Spalten schreiben:
\[
a_k=a_{1k}
\begin{pmatrix}
1\\0\\\vdots\\0
\end{pmatrix}
+a_{2k}
\begin{pmatrix}
0\\1\\\vdots\\0
\end{pmatrix}
+\dots+
+a_{nk}
\begin{pmatrix}
0\\0\\\vdots\\1
\end{pmatrix}
\]
Setzt man dies als $k$-te Spalte in die Determinante ein (wir hatten früher
dafür die Bezeichnung $\Delta_k(a_k)$ eingeführt), kann man mit
der Linearität  die Determinante schreiben als
\[
\det A=a_{1k}\Delta_k\left(
\begin{matrix}
1\\0\\\vdots\\0
\end{matrix}
\right)
+a_{2k}
\Delta_k\left(
\begin{matrix}
0\\1\\\vdots\\0
\end{matrix}
\right)+\dots+
\Delta_k\left(
\begin{matrix}
0\\0\\\vdots\\1
\end{matrix}
\right).
\]
Die $\Delta_k$ auf der rechten Seite sind Determinanten, in denen 
eine Spalte ersetzt worden ist durch eine Spalte aus Nullen und genau
einer Eins.

Diese Zerlegung kann man jetzt noch $(n-1)$-mal wiederholen,
jedes mal kommt ein anderer Faktor $a_{ik}$ vor die Determinante.
Dabei werden auch Determinanten entsteht die die gleichen Spalten
haben, die also die Eins an der selben Stelle haben.
Diese Determinanten verschwinden alle, man kann sie ignorieren.
Der Vorfaktor ist
jeweils $a_{ik}$, wobei $i$ die Zeile der Eins bezeichnet, und $k$
die Nummer der Spalte, die man ersetzt hat.
Die Indizes des
Vorfaktors $a_{ik}$ sind gerade die Koordinaten der Eins, die durch
die Ersetzung in der Determinante stehen bleibt.

Am Ende bleibt eine Summe von Termen stehen, in der alle Determinanten
in jeder Zeile und Spalte genau eine $1$ und sonst lauter Nullen
enthalten.
Die Vorfaktoren sind die Produkte der Koeffizienten in $A$,
die an Stelle der $1$ in der ursprünglichen Matrizen standen.

Wir bezeichnen die speziellen Matrizen aus lauter Nullen und Einsen
mit $\sigma$, und die Menge all dieser Matrizen mit $S_n$.
Aus einem
Index $i$ kann man die zugehörige Spalte $k$ herausfinden, indem
man die einzige Spalte in $\sigma$ sucht, welche an der $i$-ten
Stelle eine $1$ enthält.
Wir schreiben $\sigma(i)$ für dieses $k$.
Mit dieser Schreibweise ist 
\[
\det A=\sum_{\sigma\in S_n}
a_{1\sigma(1)}
a_{2\sigma(2)}
\dots
a_{n\sigma(n)}
\det \sigma.
\]
Aus dieser Formel ist jetzt klar, dass die Determinante nicht vom
Vorgehen abhängt.

Dieselben Auflösungen kann man auch mit Hilfe der Linearität der
Determinante als Funktion der Spalten vornehmen, dabei kommt die 
gleiche Formel heraus.
Wenn man jetzt noch zeigen kann, dass
die Determinanten $\det\sigma$ nicht davon abhängen, ob man von
Zeilen oder von Spalten ausgeht, dann ist klar, dass es nur eine
Determinante gibt.

Die Determinante von $\sigma$ muss $1$ oder $-1$ sein, denn durch
Zeilenvertauschungen kann sie auf die Form $E$ gebracht werden.
Alternativ kann dies mit Spaltenvertauschungen geschehen.
Jede Vertauschung trägt einen Faktor $-1$ zum Wert der Determinante bei.
Es kommt also nur darauf an, ob die Zahl der nötigen
Vertauschungen gerade oder ungerade ist.
