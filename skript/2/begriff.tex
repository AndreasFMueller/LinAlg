%
% begriff.tex -- Begriff der Determinanten
%
% (c) 2018 Prof Dr Andreas Müller, Hochschule Rapperswil
%
\section{Begriff der Determinanten}
\rhead{Begriff der Determinanten}
Die Determinante soll eine Kennzahl liefern, an der sich ablesen
lässt, ob eine Matrix regulär oder singulär ist.
Zur Definition kann man sich dabei auf den Gauss-Algorithmus stützen.
Für das
Arbeiten mit Determinanten ist das nicht praktisch, dazu werden
Rechenregeln (Formeln) benötigt.
Daraus ergibt sich dann mit
dem Entwicklungssatz ein praktisch durchführbarer Algorithmus.
Schliesslich kann man Determinanten sogar dazu verwenden, Gleichungssysteme
zu lösen oder Matrizen zu invertieren.
\subsection{Definition}
\index{Determinante}
Zu einer Matrix $A=(a_{ij})$ suchen wir eine Grösse
$\det A$, welche folgende Eigenschaften hat:
\begin{compactenum}
\item $\det A$ ändert sich nicht unter der Operation $E$.\label{invarianzE}
\item Wird eine Zeile von $A$ mit $\lambda$ multipliziert,\label{skalarI}
wird auch $\det A$ mit $\lambda$ multipliziert.
\item $\det E=1$\label{normierung}
\end{compactenum}
Im trivialen Fall des $1\times1$-Schemas folgt daraus bereits
\[
\det(a)\overset{\text{Eigenschaft \ref{skalarI}}}=a\det(1)\overset{\text{Eigenschaft \ref{normierung}}}=a\cdot 1=a.
\]

Aber auch für den Fall $2\times 2$ können wir die Lösung
bereits bestimmen.
Dazu gehen wir wie folgt vor:
\begin{align*}
\begin{pmatrix}
a%
\begin{picture}(0,0)
\color{red}\put(-2,3){\circle{12}}
\end{picture}%
&b\\
c&d%
\end{pmatrix}
&\rightarrow
\begin{pmatrix}1&\frac{b}{a}\\
c%
\begin{picture}(0,0)
\color{blue}\drawline(-6,-1)(-6,10)(1,10)(1,-1)
\end{picture}%
&d\end{pmatrix}
&\det \begin{pmatrix}a&b\\c&d\end{pmatrix}
&=a%
\begin{picture}(0,0)
\color{red}\put(-3,3){\circle{10}}
\end{picture}%
\;
\det
\begin{pmatrix}1&\frac{b}{a}\\c&d\end{pmatrix}
\\
&\rightarrow
\begin{pmatrix}1&\frac{b}{a}\\0&d-\frac{bc}{a}%
\begin{picture}(0,0)
\color{red}\put(-15,3){\circle{28}}
\end{picture}%
\end{pmatrix}
&
&=a%
\begin{picture}(0,0)
\color{red}\put(-3,3){\circle{10}}
\end{picture}%
\;\det\begin{pmatrix}1&\frac{b}{a}\\0&d-\frac{bc}{a}\end{pmatrix}
\\
&\rightarrow
\begin{pmatrix}1&\frac{b}{a}%
\begin{picture}(0,0)
\color{blue}\drawline(-7,10)(-7,-5)(0,-5)(0,10)
\end{picture}%
\\0&1\end{pmatrix}
&
&=a%
\begin{picture}(0,0)
\color{red}\put(-3,3){\circle{10}}
\end{picture}%
\;
\left(d-\frac{bc}a\right)%
\begin{picture}(0,0)
\color{red}\put(-22.5,4){\circle{35}}
\end{picture}%
\det\begin{pmatrix}1&\frac{b}{a}\\0&1\end{pmatrix}
\\
&\rightarrow
\begin{pmatrix}1&0\\0&1\end{pmatrix}
&
&=(ad-bc)
\det \begin{pmatrix}1&0\\0&1\end{pmatrix}
\\
&&&=ad-bc
\end{align*}
Die Eigenschaften sind also offenbar bereits stark genug, um die
Determinante festzulegen.
Für die Determinante hat sich auch die
Bezeichnung mit zwei vertikalen Linien links und rechts des
Koeffizientenschemas eingebürgert:
\[
\det\begin{pmatrix}a&b\\c&d\end{pmatrix}
=
\left|\,\begin{matrix}a&b\\c&d\end{matrix}\,\right|
=
ad-bc.
\]

\subsection{Rechenregeln für Determinanten}
Nach ähnlichem Muster lassen sich auch aus den Eigenschaften auch
noch weitere praktische Rechenregeln ableiten:

\begin{hilfssatz}
Besteht eine Zeile in einer Matrix $A=(a_{ij})$ aus
lauter Nullen, ist auch $\det A=0$.
\label{nullzeile}
\end{hilfssatz}

\begin{proof}[Beweis]
Multipliziert man die aus Nullen bestehende Zeilen mit $\lambda$
wird $\det A$ nach Eigenschaft \ref{skalarI}  mit $\lambda$ multipliziert.
Durch die Multiplikation ändern sich die Nullen in der Zeile allerdings
nicht, es gilt also
\[
\lambda \det A=\det A
\]
für jedes beliebige $\lambda$.
Die einzige Zahl, die sich nicht ändert,
wenn man sie mit beliebigen Zahlen multipliziert, ist die Null.
Also $\det A=0$.
\end{proof}

\begin{hilfssatz}
Sind zwei Zeilen in einer Matrix $A=(a_{ij})$
gleich, dann ist $\det A=0$.
\end{hilfssatz}
\begin{proof}[Beweis]
Subtrahiert man die erste der beiden gleichen Zeilen von der
zweiten, ändert sich die Determinante nicht, denn dies ist eine
{\bf E}-Operation.
In der zweiten Zeile stehen nach dieser Operation
aber lauter Nullen, die Determinante muss nach Hilfssatz \ref{nullzeile}
also $0$ sein.
\end{proof}

\begin{hilfssatz}
Vertauscht man in einer Matrix $A=(a_{ij})$ zwei
Zeilen, dann ändert $\det A$ das Vorzeichen.
\end{hilfssatz}
\begin{proof}[Beweis]
Die Zeilenvertauschung kann man in folgenden Schritten durchführen:
\begin{compactenum}
\item Erste Zeile von der zweiten subtrahieren:
\[
\begin{pmatrix}
      &\dots&      &\dots\\
a_{i1}&\dots&a_{ik}&\dots\\
      &\dots&      &\dots\\
a_{j1}&\dots&a_{jk}&\dots\\
      &\dots&      &\dots\\
\end{pmatrix}
\rightarrow
\begin{pmatrix}
             &\dots&             &\dots\\
a_{i1}       &\dots&a_{ik}       &\dots\\
             &\dots&             &\dots\\
a_{j1}-a_{i1}&\dots&a_{jk}-a_{ik}&\dots\\
             &\dots&             &\dots\\
\end{pmatrix}
\]
\item Zweite Zeile zur ersten addieren:
\[
\begin{pmatrix}
             &\dots&             &\dots\\
a_{i1}       &\dots&a_{ik}       &\dots\\
             &\dots&             &\dots\\
a_{j1}-a_{i1}&\dots&a_{jk}-a_{ik}&\dots\\
             &\dots&             &\dots\\
\end{pmatrix}
\rightarrow
\begin{pmatrix}
             &\dots&             &\dots\\
a_{j1}       &\dots&a_{jk}       &\dots\\
             &\dots&             &\dots\\
a_{j1}-a_{i1}&\dots&a_{jk}-a_{ik}&\dots\\
             &\dots&             &\dots\\
\end{pmatrix}
\]
\item  Erste Zeile von der zweiten Subtrahieren:
\[
\begin{pmatrix}
             &\dots&             &\dots\\
a_{j1}       &\dots&a_{jk}       &\dots\\
             &\dots&             &\dots\\
a_{j1}-a_{i1}&\dots&a_{jk}-a_{ik}&\dots\\
             &\dots&             &\dots\\
\end{pmatrix}
\rightarrow
\begin{pmatrix}
       &\dots&       &\dots\\
a_{j1} &\dots&a_{jk} &\dots\\
       &\dots&       &\dots\\
-a_{i1}&\dots&-a_{ik}&\dots\\
       &\dots&       &\dots\\
\end{pmatrix}
\]
\item Zweite Zeile mit $-1$ multiplizieren.
\[
\begin{pmatrix}
       &\dots&       &\dots\\
a_{j1} &\dots&a_{jk} &\dots\\
       &\dots&       &\dots\\
-a_{i1}&\dots&-a_{ik}&\dots\\
       &\dots&       &\dots\\
\end{pmatrix}
\rightarrow
\begin{pmatrix}
      &\dots&      &\dots\\
a_{j1}&\dots&a_{jk} &\dots\\
      &\dots&      &\dots\\
a_{i1}&\dots&a_{ik}&\dots\\
      &\dots&      &\dots\\
\end{pmatrix}
\]
\end{compactenum}
Da die ersten drei Schritte {\bf E}-Operationen sind, ändert sich nach
Eigenschaft \ref{invarianzE} die Determinante dabei nicht.
Die letzte
Operation ist eine {\bf I}-Operation, nach Eigenschaft \ref{skalarI}
wird die Determinante dabei mit $-1$ multipliziert.
\end{proof}

\begin{hilfssatz}
\label{detlinabh}
Sind die Gleichungen eines Gleichungssystems linear abhängig, dann
gilt für die Koeffizienten $\det A=0$.
\end{hilfssatz}
\begin{proof}[Beweis]
Wenn die Zeilen linear abhängig sind, dann wissen wir bereits, dass
im Gauss-Verfahren, bei welchem nur die Operationen {\bf I} und {\bf E}
angewendet werden, am Ende mindestens eine Zeile aus lauter
Nullen auftauchen muss.
Am Ende des Prozesses ist die Determinante also $0$.

Da die Operation {\bf E} die Determinante gar nicht,
und {\bf I} nur um einen nicht verschwindenden Faktor ändert, muss
die Determinante also auch schon am Anfang des Prozesses $0$ gewesen
sein.
\end{proof}

\subsection{Determinante als lineare Funktion der Zeilen\label{detlinfun}}
Die bisher verwendeten Eigenschaften der Determinante hatten den
Vorteil, einen direkten Bezug zu den Operationen zu haben, mit 
denen wir Gleichungssysteme gelöst haben.
Sie hatten den Nachteil, etwas willkürlich zu sein.
Daher verwendet man oft die folgenden Eigenschaften, aus denen
die bisherigen Eigenschaften folgen.

\begin{compactenum}
\item[$1'$.] $\det A$ ist eine lineare Funktion der Zeilen, d.h.
\[
\left|
\;
\begin{matrix}
a_{11}&\dots&a_{1n}\\
\vdots&\ddots&\vdots\\
\lambda a'_{k1}+\mu a''_{k1}&\dots&\lambda a'_{kn}+\mu a''_{kn}\\
\vdots&\ddots&\vdots\\
a_{n1}&\dots&a_{nn}
\end{matrix}
\;
\right|
=
\lambda
\left|
\;
\begin{matrix}
a_{11}&\dots&a_{1n}\\
\vdots&\ddots&\vdots\\
a'_{k1}&\dots&a'_{kn}\\
\vdots&\ddots&\vdots\\
a_{n1}&\dots&a_{nn}
\end{matrix}
\;
\right|
+
\mu
\left|
\;
\begin{matrix}
a_{11}&\dots&a_{1n}\\
\vdots&\ddots&\vdots\\
a''_{k1}&\dots&a''_{kn}\\
\vdots&\ddots&\vdots\\
a_{n1}&\dots&a_{nn}
\end{matrix}
\;
\right|
\]
\item[$2'$.] Sind zwei Zeilen von $A$ gleich, ist $\det A=0$
\end{compactenum}

Aus diesen Eigenschaften lassen sich die Eigenschaften \ref{invarianzE}.~und
\ref{skalarI}.~ableiten.

\begin{hilfssatz} Aus den Eigenschaften $1'$.~und $2'$.~folgen die Eigenschaften
1.~und 2.~der Determinante.
\end{hilfssatz}
\begin{proof}[Beweis]
Setzt man 
$a'_{ki}=a_{ki}$, $a''_{ki}=a_{li}$, $\lambda=1$ entsteht auf der linken Seite
die Determinante nach dem hinzuaddieren des $\mu$-fachen der $l$-ten Zeile
zur $k$-ten Zeile, also das Resultat einer {\bf E}-Operation.
Auf der rechten Seite steht als erster Term die ursprüngliche Determinante,
der zweite Term ist aber eine Determinante, in der die $k$-te und die
$l$-te Zeile übereinstimmen, diese verschwindet also nach $2'$.
Damit ist die Eigenschaft \ref{invarianzE}.~bewiesen.

Setzt man $a'_{ki}=a_{ki}$ und $\mu=0$, steht auf der linken Seite
die Determinante, in der die $k$-te Zeile mit $\lambda$ multipliziert
wurde.
Auf der rechten Seite steht die mit $\lambda$ multiplizierte
Determinante, wegen $\mu=0$ fällt der zweite Term weg.
Somit ist auch die Eigenschaft \ref{skalarI}.~bewiesen.
\end{proof}

Wenn die Eigenschaften $1'$., $2'$.~und 3.~erfüllt sind, sind also immer
noch alle Schlussfolgerungen gültig, die wir aus den Eigenschaften
1.~bis 3.~gezogen hatten.

Der Vorteil dieser Eigenschaften gegenüber den bisher verwendeten
besteht darin, dass sich daraus eine Formel ableiten lässt, aus
der weiter Eigenschaften der Determinante leichter abgeleitet
werden können. 

Man kann zeigen, dass die Eigenschaften der Determinante diese
eindeutig bestimmen.
Es kann also nicht passieren, dass eine andere
Reihenfolge der Pivot-Elemente zu einem anderen Resultat führt.
Da diese Eigenschaft auf eine Art beweisen wird, die uns keine
anderen nützlichen Resultate liefert, verbannen wir den Beweis
in den letzten Abschnitt \ref{deteindeutig} dieses Kapitels.
Wir halten hier nur das Resultat fest.

\begin{satz}
\label{detcharacterisation}
Die Determinante
$\det A$
ist die einzige lineare Funktion der Zeilen von $A$, die folgende zwei
Eigenschaften hat:
\begin{compactenum}
\item Falls $A$ zwei gleiche Zeilen enthält ist $\det A=0$.
\item $\det E = 1$.
\end{compactenum}
\end{satz}

\subsection{Spalten statt Zeilen}
Die Definition verlangt explizit nach Zeilen-Operationen {\bf I}
und {\bf E}.
Wir könnten aber auch entsprechende Spalten-Operationen
{\bf I$\mathstrut^t$}
und
{\bf E$\mathstrut^t$}
verwenden, um die Determinante zu definieren.
Das vorgehen zur Berechnung der Determinante ist genau das selbe.
Man wendet die Operationen an, bis nur noch die Einheitsmatrix stehen bleibt.

Analog zum Vorgehen in Abschnitt \ref{detlinfun} können wir auch
fordern, dass die Determinante eine lineare Funktion der Spalten
(statt der Zeilen) sein soll, und würden dieselbe Determinante
erhalten wie mit der Definition durch die Operationen 
{\bf I$\mathstrut^t$}
und
{\bf E$\mathstrut^t$}.
Auch hier halten wir das Resultat fest:
\begin{satz}
Die Determinante
$\det A$ ist die einzige lineare Funktion der Spalten von $A$, die folgende
zwei Eigenschaften hat:
\begin{compactenum}
\item Falls $A$ zwei gleiche Spalten enthält ist $\det A=0$.
\item $\det E = 1$.
\end{compactenum}
\end{satz}
Im Prinzip könnte diese über die Spalten definierte Determinante
von der über die Zeilen definierten Determinante verschieden sein.
Im Abschnitt \ref{deteindeutig} wird jedoch eine Formel für die
Determinante abgeleitet, aus der hervorgeht, dass beide Definitionen
auf den selben Wert für die Determinante führen.

