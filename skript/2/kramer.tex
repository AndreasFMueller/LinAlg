%
% kramer.tex -- Lösung von Gleichungssystemen
%
% (c) 2018 Prof Dr Andreas Müller, Hochschule Rapeprswil
%
\section{Lösen von Gleichungssystemen}
\rhead{Cramersche Regel}
Aus dem Hilfssatz \ref{detlinabh} folgt sofort, dass ein Gleichungssystem
genau dann eindeutig lösbar ist, wenn die Determinante der Koeffizienten
nicht $0$ ist.

\begin{satz}
Das Gleichungssystem mit $n$ Unbekannten und $n$ Gleichungen
mit den Koeffizienten $a_{ij}$ ist genau dann eindeutig lösbar,
wenn
$\det A\ne 0$
\end{satz}

Wir möchten jetzt die in der Einleitung versprochen Formel für die Lösung
ableiten, mit der man die Lösung des Gleichungssystems aus lauter
Determinanten berechnen kann.

\index{Cramersche Regel}
\begin{satz}[Cramer]
Das Gleichungssystem mit $n$ Unbekannten und $n$ Gleichungen
mit den Koeffizienten $a_{ij}$ 
mit $\det A\ne 0$ hat die Lösungen
\[
x_1=\frac{
\left|\,\begin{matrix}
b_1&a_{12}&\dots&a_{1n}\\
\vdots&\vdots&\ddots&\vdots\\
b_n&a_{n2}&\dots&a_{nn}\\
\end{matrix}\,\right|
}{
\left|\,\begin{matrix}
a_{11}&a_{12}&\dots&a_{1n}\\
\vdots&\vdots&\ddots&\vdots\\
a_{n1}&a_{n2}&\dots&a_{nn}\\
\end{matrix}\,\right|
}
,\qquad\dots,\qquad
x_n=\frac{
\left|\,\begin{matrix}
a_{11}&\dots&a_{1,n-1}&b_1\\
\vdots&\ddots&\vdots&\vdots\\
a_{n1}&\dots&a_{n,n-1}&b_n\\
\end{matrix}\,\right|
}{
\left|\,\begin{matrix}
a_{11}&\dots&a_{1,n-1}&a_{1n}\\
\vdots&\ddots&\vdots&\vdots\\
a_{n1}&\dots&a_{n,n-1}&a_{nn}\\
\end{matrix}\,\right|
}
\]
Die Unbekannte $x_k$ berechnet man also als Quotient der Determinante
von $A$, in der man die $k$-te Spalte durch die rechten Seiten $b_i$
ersetzt hat, und der Determinanten von $A$.
\end{satz}
\begin{proof}[Beweis]
Schreiben wir $a_1,\dots,a_n$ für die Spalten von $A$ und $b$ für die 
Spalte der rechten Seite, dann bedeutet das Gleichungssystem, dass die
Spalte $b$ geschrieben werden kann als eine Linearkombination der
Spalten von $A$:
\begin{equation}
x_1a_1+\dots +x_na_n=b
\label{glinspalten}
\end{equation}
Die Determinante von $A$ ist eine lineare Funktion jeder einzelnen Spalte.
Wir betrachten für den Moment nur die Abhängigkeit von der $k$-ten
Spalte und schreiben dafür
\[
\Delta_k(u)=\left|
\,\begin{matrix}
a_{11}&\dots&u_1&\dots&a_{1n}\\
\vdots&\ddots&\vdots&\ddots&\vdots\\
a_{n1}&\dots&u_n&\dots&a_{nn}
\end{matrix}
\,\right|.
\]
Setzen wir beide Seiten von (\ref{glinspalten}) in $\Delta_k$ ein,
erhalten wir unter Ausnützung der Linearität:
\[
x_1\Delta_k(a_1)+\dots+x_k\Delta_k(a_k)+\dots+x_n\Delta_k(a_n)=\Delta_k(b).
\]
Auf der linken Seite enthalten alle Terme ausser dem $k$-ten die
Spalte $a_i$ zweimal, einmal am Platz $i$, und einmal neu eingefügt am
Platz $k$.
Da die Determinante verschwindet, wenn zwei Spalten übereinstimmen,
bleibt nur der $k$-te Term stehen:
\[
x_k\Delta_k(a_k)=\Delta_k(b)
\]
Auf der linken Seite setzt man die $k$-te Spalte als $k$-te Spalte
in $A$ ein und berechnet die Determinante, dies ist also nichts
anderes als die Determinante von $A$.
Auf der rechten Seite ersetze man die $k$-Spalte von $A$ durch $b$, also
\[
x_k
\left|\,\begin{matrix}
a_{11}&\dots&a_{1n}\\
\vdots&\ddots&\vdots\\
a_{n1}&\dots&a_{nn}\\
\end{matrix}\,\right|
=
\left|\,\begin{matrix}
a_{11}&\dots&b_1&\dots&a_{1n}\\
\vdots&\ddots&\vdots&\ddots&\vdots\\
a_{n1}&\dots&b_n&\dots&a_{nn}\\
\end{matrix}\,\right|
\]
Die Behauptung folgt jetzt durch Auflösen nach $x_k$.
\end{proof}
Dieser Satz stellt zwar eine hübsche Formel zur Berechnung der Lösung
bereit, praktisch nützlich ist diese jedoch kaum.
Die Berechnung der
$n+1$ Determinanten ist bereits aufwendiger als die Durchführung des
Gauss-Verfahrens, welches die Lösung auch schon liefert.

\begin{beispiel}Man finde die Lösung des Gleichungssystems
\[
\begin{linsys}{4}
-x&-&3y&&&=&\color[rgb]{0,0.5,0}-7\\
2x&+&3y&-&2z&=&\color[rgb]{0,0.5,0}2\\
2x&+&y&-&3z&=&\color[rgb]{0,0.5,0}-5
\end{linsys}
\]
Die Koeffizientenmatrix und die rechte Seite sind
\[
A=\begin{pmatrix}
-1&-3&0\\
2&3&-2\\
2&1&-3
\end{pmatrix}
,\qquad
b=\begin{pmatrix}
\color[rgb]{0,0.5,0}-7\\\color[rgb]{0,0.5,0}2\\\color[rgb]{0,0.5,0}-5
\end{pmatrix}
\]
wobei wir früher bereits $\det(A)=1$ gefunden haben.
Für die Lösungen des Gleichungssystems bekommen wir damit
\begin{align*}
x&=\frac{
\left|\;
\begin{matrix}
\color[rgb]{0,0.5,0}-7&-3&0\\
\color[rgb]{0,0.5,0}2&3&-2\\
\color[rgb]{0,0.5,0}-5&1&-3
\end{matrix}
\;\right|
}{\det(A)}
=63-30+0-0-14-18=1,
\\
\\
y&=\frac{
\left|\;
\begin{matrix}
-1&\color[rgb]{0,0.5,0}-7&0\\
2&\color[rgb]{0,0.5,0}2&-2\\
2&\color[rgb]{0,0.5,0}-5&-3
\end{matrix}
\;\right|
}{\det(A)}
=6+28+0-0+10-42=2,
\\
\\
z&=\frac{
\left|\;
\begin{matrix}
-1&-3&\color[rgb]{0,0.5,0}-7\\
2&3&\color[rgb]{0,0.5,0}2\\
2&1&\color[rgb]{0,0.5,0}-5
\end{matrix}
\;\right|
}{\det(A)}
=15-12-14+42+2-30=3.
\end{align*}
\end{beispiel}

