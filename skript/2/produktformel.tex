%
% produktformel.tex
%
% (c) 2018 Prof Dr Andreas Müller, Hochschule Rapperswil
%
\section{Produktformel}
Wie verträgt sich die Determinante mit Matrizenprodukten?
Überraschenderweise gibt es hierauf eine sehr einfach Antwort,
die auch nicht schwierig zu verstehen ist.

\begin{satz}\label{detprodukt}
Sind $A$ und $B$ $n\times n$-Matrizen, dann ist
\[
\det(AB)=\det(A)\det(B).
\]
\end{satz}

\begin{proof}[Beweisidee]
Wir gehen den Beweis wie folgt an.
Wir betrachten zunächst nur die Abhängigkeit von $\det(AB)$ von $A$.
Dabei werden wir feststellen,
dass sich $\det(AB)$ fast wie die Determinante verhält, nur der
Wert für $A=E$ ist nicht der richtige, $\det(B)$ statt $1$.
Indem wir
$\det(AB)$ durch $\det(B)$ dividieren, erhalten wir aber eine Funktion,
die sich genau wie die Determinante verhält.
Weil die Eigenschaften der Determinante diese eindeutig bestimmen, muss
$\det(AB)/\det(B)=\det(A)$ sein, woraus wir die Behauptung folgern können.
\end{proof}

\begin{proof}[Beweis]
Wir betrachten die Funktion $d\colon A\mapsto d(A) = \det(AB)$.
Sie hat die folgenden Eigenschaften, die weiter unten noch
begründet werden müssen:
\begin{enumerate}
\item[1'.] $d$ ist eine lineare Funktion der Zeilen von $A$.
\item[2.']
Hat $A$ zwei gleiche Zeilen, dann auch $AB$ und damit ist $d(A)=0$.
\item[3.]
$d(E)=\det(B)$.
\end{enumerate}
Die Funktion $d$ verhält sich bis auf die letzte Eigenschaft
wie die Determinante.
Dividiert man die Funktion $d$ durch die Determinante
von $B$, ist auch die letzte Eigenschaft die einer Determinanten.
Die Funktion
\[
d':A\mapsto \frac{\det(AB)}{\det(B)}
\]
hat die Eigenschaften:
\begin{enumerate}
\item[1'.] $d'$ ist eine lineare Funktion der Zeilen von $A$.
\item[2'.] Enthält $A$ zwei gleiche Zeilen, dann ist $d'(A)=0$.
\item[3.] $d'(E)=1$.
\end{enumerate}
Nach Satz \ref{detcharacterisation} muss $d'$ die Determinante von $A$ sein:
\begin{align*}
d'(A)&=\det(A)\\
\frac{\det(AB)}{\det(B)}&=\det(A)\\
\det(AB)&=\det(A)\det(B).
\end{align*}
Das beweist die Produktformel, bis auf die oben behaupteten Eigenschaften
1' und 2'.

Die Zeile mit der Nummer $i$ in $AB$ wird erhalten,
indem man die Zeile $i$ von $A$
mit der Matrix $B$ multipliziert.
Wenn also $A$ zwei gleiche Zeilen enthält, dann sind auch die
entsprechenden Zeilen von $AB$ gleich.
was Eigenschaft 2 beweist.

Ist die Zeile $i$ von $A$ eine Linearekombination der Form
\[
\begin{pmatrix}
a_{i1}&\dots&a_{in}
\end{pmatrix}
=
\lambda
\begin{pmatrix}
a'_{i1} &\dots &a'_{in}
\end{pmatrix}
+
\mu
\begin{pmatrix}
a''_{i1} &\dots &a''_{in}
\end{pmatrix},
\]
dann ist die Zeile $i$ von $AB$ ebenfalls eine Linearekombination,
nämlich
\[
\begin{pmatrix}
a_{i1}&\dots&a_{in}
\end{pmatrix}B
=
\lambda
\begin{pmatrix}
a'_{i1} &\dots &a'_{in}
\end{pmatrix}B
+
\mu
\begin{pmatrix}
a''_{i1} &\dots &a''_{in}
\end{pmatrix}B.
\]
Schreiben wir $A'$ für die Matrix, die in Zeile $i$ die $a'_{ij}$
statt der $a_{ij}$ enthalten, und analog für $A''$, dann folgt, dass
\[
d(A) = \det(AB)=\lambda \det(A'B)+\mu\det(A''B)=\lambda d(A') + \mu d(A''),
\]
was genau die Eigenschaft 1 ist.
\end{proof}

Es gibt keine vergleichbare Formel für die Determinante der Summe
von zwei Matrizen.
Schon $2\times 2$-Diagonalmatrizen zeigen uns,
das wir eine solche Formel auch nicht erwarten können:
\begin{align*}
A&=\begin{pmatrix}a&0\\0&1\end{pmatrix}&
\det(A)&=\left|\;\begin{matrix}a  &0\\0&  1\end{matrix}\;\right|= a\\
B&=\begin{pmatrix}1&0\\0&b\end{pmatrix}&
\det(B)&=\left|\;\begin{matrix}  1&0\\0&b  \end{matrix}\;\right|= b\\
A+B&=\begin{pmatrix}a+1&0\\0&b+1\end{pmatrix}
&
\det(A+B)&=\left|\;\begin{matrix}a+1&0\\0&b+1\end{matrix}\;\right|\\
&&&=(a+1)(b+1)=ab+a+b+1.
\end{align*}
Offenbar gibt es keine einfache Formel, die $\det(A+B)$ mit $\det(A)$ und
$\det(B)$ verbindet.

