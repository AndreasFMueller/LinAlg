%
% inverse.tex
%
% (c) 2018 Prof Dr Andreas Müller, Hochschule Rapperswil
%
\section{Inverse Matrix}
\rhead{Inverse Matrix}
\index{inverse Matrix}
Mit den im letzten Abschnitt beschriebenen Minoren kann man jetzt auch
eine Formel für die Elemente der inversen Matrix finden.
Die inverse
Matrix von $A$ hat in ihren Spalten die Lösungen $x$ des Gleichungssystems 
$Ax=b$ für ganz spezielle rechte Seiten $b$.
Die $j$-te Spalte ist die Lösung zur rechten Seite $e_j$, dem Einheitsvektor,
der genau an der $j$-ten Stelle eine $1$ hat, und sonst aus lauter Nullen
besteht.
Nach der Cramerschen Regel kann man die $i$-te Unbekannte $x_i$
in $Ax=e_j$ mit Determinanten berechnen, nämlich
\[
x_i=(-1)^{i+j}\frac{\det(A_{ji})}{\det(A)}.
\]
Dies ist auch der Eintrag in Spalte $j$ und Zeile $i$ der inversen
Matrix $A^{-1}$:
\begin{equation}
A^{-1}
=
\frac{1}{\det(A)}
\begin{pmatrix}
\det(A_{11})&-\det(A_{21})&\det(A_{31})& \dots&(-1)^{1+n} \det(A_{n1})\\
-\det(A_{12})&\det(A_{22})&-\det(A_{32})& \dots&(-1)^{2+n} \det(A_{n2})\\
\det(A_{13})&-\det(A_{23})&\det(A_{33})& \dots&(-1)^{3+n} \det(A_{n3})\\
\vdots&\vdots&\vdots&\ddots&\vdots\\
(-1)^{n+1}\det(A_{1n})&(-1)^{n+2}\det(A_{2n})&(-1)^{n+3}\det(A_{3n})& \dots&(-1)^{n+n} \det(A_{nn})\\
\end{pmatrix}
\label{inversematrix}
\end{equation}
\begin{definition}
Die Terme in der Matrix in (\ref{inversematrix})
heissen Kofaktoren der Matrix $A$.
Sie bilden die Matrix der Kofaktoren
\begin{equation}
\operatorname{cof}(A)_{ij}=
(-1)^{i+j}\det(A_{ij})
\label{cofactor}
\end{equation}
Mit dieser Notation ist die inverse Matrix
\begin{equation}
A^{-1}=\frac1{\det(A)}\operatorname{cof}(A)^t
\label{inversecofactors}
\end{equation}
\end{definition}

Für $2\times2$-Matrizen führt dies auf die manchmal nützliche
Formel
\[
\begin{pmatrix}
a&b\\c&d
\end{pmatrix}^{-1}
=
\frac1{ad-bc}\begin{pmatrix}
d&-b\\-c&a
\end{pmatrix}
\]
Für die Berechnung der Inversen grösserer Matrizen ist die Formel
nur in ganz speziellen Fällen von Nutzen.
Die Berechnung der Inverse
mit Hilfe des Gauss-Algorithmus erfordert im allgemeinen deutlich weniger
Aufwand.
Hingegen ist die Formel für theoretische Überlegungen durchaus
interessant.
Es folgt aus ihr zum Beispiel, dass die Inverse einer
Matrix mit ganzzahligen Einträgen und Determinante $1$ wieder
lauter ganzzahlige Einträge hat.
Die Menge
\[
\operatorname{SL}_n(\mathbb Z)=\{A\in M_{n}(\mathbb Z)|\det(A)=1\},
\]
hat also die Eigenschaft, dass mit $A\in\operatorname{SL}_2(\mathbb Z)$ 
auch $A^{-1}\in\operatorname{SL}_2(\mathbb Z)$.

\begin{beispiel}
Man bestimme die Inverse der Matrix
\[
A=\begin{pmatrix}
-1&-3&1\\
2&3&-2\\
2&1&-3
\end{pmatrix}
\]
Die Determinante von $A$ kann mit der Sarrus-Formel berechnet, werden.
Sie ist
\begin{align*}
\det(A)
&=
\left|
\begin{matrix}
-1&-3& 1\\
 2& 3&-2\\
 2& 1&-3
\end{matrix}
\right|
\\
&=(-1)\cdot 3\cdot(-3) +  (-3)\cdot(-2)\cdot 2 + 1\cdot 2\cdot 1
\\&\qquad
-2\cdot 3\cdot 1-1\cdot(-2)\cdot (-1)-(-3)\cdot 2\cdot (-3)
\\
&=9+12+2 - 6-2-18=-3.
\end{align*}
Die Minoren sind
\begin{align*}
\det A_{11}&=\left|\,\begin{matrix} 3&-2\\ 1&-3\end{matrix}\,\right|
%=-9+2
=-7
&
\det A_{12}&=\left|\,\begin{matrix} 2&-2\\ 2&-3\end{matrix}\,\right|
%=-6+4
=-2
&
\det A_{13}&=\left|\,\begin{matrix} 2& 3\\ 2& 1\end{matrix}\,\right|
%=2-6
=-4\\
\det A_{21}&=\left|\,\begin{matrix}-3& 1\\ 1&-3\end{matrix}\,\right|
=8
&
\det A_{22}&=\left|\,\begin{matrix}-1& 1\\ 2&-3\end{matrix}\,\right|
=1
&
\det A_{23}&=\left|\,\begin{matrix}-1&-3\\ 2& 1\end{matrix}\,\right|
%=-1+6
=5\\
\det A_{31}&=\left|\,\begin{matrix}-3& 1\\ 3&-2\end{matrix}\,\right|
=3
&
\det A_{32}&=\left|\,\begin{matrix}-1& 1\\ 2&-2\end{matrix}\,\right|
=0
&
\det A_{33}&=\left|\,\begin{matrix}-1&-3\\ 2& 3\end{matrix}\,\right|
%=-3+6
=3
\end{align*}
Beim Hinschreiben der Inversen muss man jetzt aber beachten,
dass das Element in Zeile $i$ und Spalte $j$ der inversen Matrix
mit $A_{ji}$ gebildet wird:
\[
A^{-1}=\frac1{\det(A)}\begin{pmatrix}
+\det(A_{11})&-\det(A_{21})&+\det(A_{31})\\
-\det(A_{12})&+\det(A_{22})&-\det(A_{32})\\
+\det(A_{13})&-\det(A_{23})&+\det(A_{33})
\end{pmatrix}
=\frac1{-3}\begin{pmatrix}
-7&-8& 3\\
 2& 1& 0\\
-4&-5& 3
\end{pmatrix}
\]
Zur Kontrolle rechnen wir das Produkt nach:
\begin{align*}
\begin{pmatrix}
-7&-8& 3\\
 2& 1& 0\\
-4&-5& 3
\end{pmatrix}
\begin{pmatrix}
-1&-3& 1\\
 2& 3&-2\\
 2& 1&-3
\end{pmatrix}
&=
\begin{pmatrix}
 7-16+6&21-24+3&-7+16-9\\
-2+ 2+0&-6+ 3+0& 2- 2+0\\
 4-10+6&12-15+3&-4+10-9
\end{pmatrix}
\\
&=-3\begin{pmatrix}
1&0&0\\
0&1&0\\
0&0&1
\end{pmatrix}.
\end{align*}
Da wir in diesem Produkt die Determinanten $-3$ noch nicht berücksichtigt
haben, ist das das zu erwartende Resultat.
\end{beispiel}

