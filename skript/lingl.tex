%
% Lineare Gleichungen
%
% (c) 2009 Prof Dr Andreas Mueller, Hochschule Rapperswil
%
\chapter{Lineare Gleichungssysteme\label{chapter-lingl}}
\rhead{Lineare Gleichungssysteme}
Die Basis f"ur die gesamte Lineare Algebra ist die F"ahigkeit, lineare
Gleichungen l"osen zu k"onnen. Daher beginnen wir in diesem Kapitel
damit, Verfahren f"ur die L"osung linearer Gleichungssysteme
zusammenzustellen. Dabei interessiert uns nicht nur die die L"osung
eines solchen Gleichungssystems, sondern auch die Frage, ob es
"uberhaupt l"osbar ist, oder ob es vielleicht sogar mehr als eine
L"osung hat. Dabei wird sich herausstellen, dass dabei immer die
gleichen Alternativen auftreten:
\begin{compactenum}
\item Keine L"osung
\item Genau eine L"osung
\item Unendlich viele L"osungen
\end{compactenum}
Genaugenommen sind dies nicht drei gleichberechtigte F"alle,
sondern
\index{regulaer@regul\"ar}
\index{singulaer@singul\"ar}
\begin{compactenum}
\item {\em Regul"arer} Fall: Genau eine L"osung
\item {\em Singul"arer} Fall
\begin{compactenum}
\item Keine L"osung
\item Unendlich viele L"osungen
\end{compactenum}
\end{compactenum}
Neben den Methoden, eine L"osung tats"achlich zu berechnen, ist
diese Dreier-Alternative ein zentrales Thema. Es stellt sich heraus,
dass diese nicht nur f"ur Gleichungen mit reellen oder rationalen
Koeffizienten gilt, sondern auch f"ur komplexe Zahlen, oder
f"ur weitere Systeme von ``Zahlen'', sogenannte K"orper. Daher
m"ussen zusammen mit den Methoden zur L"osung der Gleichungen
auch Algorithmen zur Unterscheidung dieser F"alle bereitgestellt
werden.
\index{Koerper@K\"orper}

\section{Gleichungssysteme}
\subsection{Begriffe und Notation}
\index{Linearform}
\begin{definition}
Eine Linearform $l(x_1,\dots,x_2)$ "uber $\mathbb R$ in den $n$
Variablen $x_1,\dots,x_n$
ist eine Funktion der Form
\[
l(x_1,\dots,x_2)=a_1x_1+a_2x_2+\dots+a_nx_n,
\]
wobei $a_i\in\mathbb R$ f"ur alle $i$.
\end{definition}
Ist $(x_1',\dots,x_n')$ ein zweiter Satz Variablen, und $\lambda\in\mathbb R$,
dann gilt
\begin{equation}
\begin{aligned}
l(x_1+x_1',\dots,x_n+x_n')&=l(x_1,\dots, x_n)+l(x_1',\dots,x_n')\\
l(\lambda x_1, \dots ,\lambda x_n)&=\lambda l(x_1,\dots,x_n)
\end{aligned}
\label{linearitaet-linearformen}
\end{equation}
Man nennt diese Eigenschaft {\em Linearit"at}, die Linearformen sind
{\em lineare} Funktionen.
\index{linear}%
Etwas salopp ausgedr"uckt sind lineare Funktionen, solche, die sich
auf die einzelnen Summanden einer Summe verteilen lassen, und f"ur
die sich Faktoren aus den Argumenten vor die Funktion ziehen lassen.
Multiplikation aller Argumente mit einer Zahl und Additionen aller
Argumente lassen sich also ``aufl"osen''.

\index{Gleichung!lineare}
Aus einer Linearform und einer Konstanten kann man eine
lineare Gleichung bilden:
$$l(x_1,\dots,x_n)=a_1x_1+\dots +a_nx_n=b$$
Hat man nur zwei oder drei Unbekannte, kann man sie etwas "ubersichtlicher
mit $x$, $y$ und $z$ bezeichnen:
$$ax+by+cz=d$$
ist auch eine lineare Gleichung. Die L"osung einer solchen Gleichung
besteht aus einem Tripel von Zahlen $(x,y,z)$, die die Gleichung erf"ullen.
Statt als Tripel wird die L"osung oft auch als Spalte geschrieben:
\[
\begin{pmatrix}
x\\y\\z
\end{pmatrix}.
\]

\index{Gleichungssystem!lineares}
Ein wesentlicher Teil der linearen Algebra befasst sich mit linearen
Gleichungssystemen, also mehreren linearen Gleichungen:
\begin{align*}
l_1(x_1,\dots,x_n)&=b_1\\
\vdots\qquad\qquad&\quad\vdots\\
l_m(x_1,\dots,x_n)&=b_m
\end{align*}
Dabei lassen wir vorerst zu, dass auch weniger oder mehr Gleichungen
als Unbekannte gegeben sind. Sp"ater werden wir diskutieren, unter
welchen Bedingungen diese Gleichungssysteme L"osungen haben.

Vollst"andig ausgeschrieben lauten die Gleichungssyteme
\begin{align*}
a_{11}x_1+a_{12}x_2+\dots a_{1n}x_n&=b_1\\
a_{21}x_1+a_{22}x_2+\dots a_{2n}x_n&=b_2\\
\vdots\\
a_{m1}x_1+a_{m2}x_2+\dots a_{mn}x_n&=b_m
\end{align*}
Das Gleichungssystem ist durch die Angabe der Koeffizienten $a_{ij}$
und der rechten Seite $b_j$ gegeben, gesucht sind die Unbekannten $x_j$.
Es ist also nicht unbedingt n"otig, die Gleichungen auszuschreiben,
wenn man diese Angaben in anderer Form machen kann, z.~B.~indem man
Unbekannte und rechte Seite in Spaltenform und die Koeffizienten in
Matrixform schreibt:
\[
A=\begin{pmatrix}
a_{11}&a_{12}&\dots &a_{1n}\\
a_{21}&a_{22}&\ddots&a_{2n}\\
\vdots&\vdots&\ddots&\vdots\\
a_{m1}&a_{m2}&\dots&a_{mn}
\end{pmatrix}
,\quad
x=\begin{pmatrix}
x_1\\x_2\\\vdots\\x_n
\end{pmatrix},
\quad
b=\begin{pmatrix}
b_1\\b_2\\\vdots\\b_m
\end{pmatrix}.
\]

\subsection{Eine Unbekannte}
Der Fall einer Unbekannten ist besonders einfach, wir diskutieren die
Gleichungen
$$ax=b$$
mit $a,b\in\mathbb R$ und suchen eine L"osung $x\in\mathbb R$.
Offenbar kann man die L"osung finden, indem man durch $a$ teilt:
$$x=\frac{b}{a},$$
dies ist aber nur dann m"oglich, wenn $a\ne 0$ ist. Dies scheint
der ``regul"are'' Fall zu sein, in dem es genau eine L"osung gibt. 

Der singul"are Fall ist $a=0$, die Gleichung wird dann zu 
$$0=b.$$
Offenbar ist diese Gleichung "uberhaupt nicht erf"ullbar, wenn $b\ne 0$
ist. Falls jedoch $b=0$, dann ist die Gleichung $0\cdot x=0$, die
zutrifft was immer wir f"ur Werte f"ur $x$ einsetzen, in diesem
Fall haben wir also unendlich viele L"osungen. Zusammengefasst
bekommen wird
\begin{enumerate}
\item regul"arer Fall: $a\ne 0\Rightarrow x=\frac{b}{a}$
\item singul"arer Fall: $a=0$
\begin{enumerate}
\item $b=0\Rightarrow$ jedes beliebige $x\in\mathbb R$ ist L"osung
\item $b\ne0\Rightarrow$ keine L"osungen
\end{enumerate}
\end{enumerate}

\subsection{Zwei Unbekannte}
Eine einzelne Gleichung mit zwei Unbekannten 
$$ax+by=c$$
kann keine eindeutige L"osung haben. Falls $b\ne 0$ kann man
sie nach $y$ aufl"osen:
$$y=\frac{c-ax}b=-\frac{b}{a}x+\frac{c}b,$$
sie beschreibt also eine Gerade. Die Menge der Punkte 
$$g_1=\{(x,y)\,|\,ax+by=c\}$$
ist eine Gerade.

F"ugt man jetzt eine zweite Gleichun hinzu, erh"alt man
ein Gleichungssystem
$$
\begin{linsys}{2}
ax&+&by&=&e\\
cx&+&dy&=&f
\end{linsys}
$$
\index{Loesungsmenge@L\"osungsmenge}
Die L"osungsmenge jeder dieser Gleichungen entspricht einer
Geraden:
\begin{align*}
g&=\{(x,y)\,|\,ax+by=e\}\\
h&=\{(x,y)\,|\,cx+dy=f\}
\end{align*}
Wir interessieren uns jedoch f"ur diejenigen Paare $(x,y)$,
die beide Gleichungen erf"ullen, also f"ur die Schnittmenge
der Geraden $g$ und $h$.

\index{Gerade}
Ohne dass wir bereits ein Verfahren zur Berechnung einer
L"osung haben, k"onnen wir bereits herausfinden, wieviele
L"osungen das Gleichungssystem hat. Zwei Geraden in der Ebene
k"onnen h"ochstens in den drei folgenden Lagen angeordnet sein:
\begin{enumerate}
\item Die Geraden sind nicht parallel, dann schneiden sie sich in
genau einem Punkt, das Gleichungssystem hat eine einzige L"osung.
\item Die Geraden sind parallel, dann gibt es zwei m"ogliche Unterf"alle:
\begin{enumerate}
\item Die Geraden haben einen positiven Abstand, dann gibt es keine
gemeinsamen Punkte, also hat das Gleichungssystem auch keine L"osung
\item Der Abstand der Geraden ist $0$, dann sind die Geraden deckungsgleich,
jeder Punkt der beiden Geraden ist L"osung des Gleichungssystems, es hat
also unendlich viele L"osungen.
\end{enumerate}
\end{enumerate}
Hier drei Beispiele von Gleichungssystemen, die diese drei Situationen
wiederspiegeln:
\begin{beispiel}[Beispiel f"ur den Fall 1]
Das folgende Gleichungssystem hat genau eine L"osung.
\begin{align*}
x+y=2\\
x-y=0
\end{align*}
hat die L"osung $(1,1)$, wie man durch Einsetzen nachpr"ufen kann.
\end{beispiel}
\begin{beispiel}[Beispiel f"ur den Fall 2. (a)]
Das Gleichungssystem
$$
\begin{linsys}{2}
x&+&y&=&1\\
2x&+&2y&=&0
\end{linsys}
$$
kann keine L"osung haben. Multipliziert man die erste Gleichung mit $2$
und subtrahiert die zweite Gleichung erh"alt man $0=2$, dies kann nat"urlich
nie erf"ullt werden, insbesondere gibt es kein Zahlenpaar $(x,y)$, f"ur
welches die beiden Gleichungen wahr w"urden.
\end{beispiel}
\begin{beispiel}[Beispiel f"ur den Fall 2. (b)]
Das Gleichungssystem
$$
\begin{linsys}{2}
x&+&y&=&1\\
2x&+&2y&=&2
\end{linsys}
$$
hat alle Punkte der Geraden $y=1-x$ als L"osung.
\end{beispiel}

\section{Das Gauss-Verfahren}
\index{Gauss-Verfahren}
\index{Gauss-Elimination|see{Gauss-Verfahren}}
F"ur die Anwendung brauchen wir ein effizientes Verfahren zur
L"osung von linearen Gleichungssystemen. Dabei erwarten wir nat"urlich,
dass wir zur Bestimmung von $n$ Unbekannten auch $n$ Gleichungen ben"otigen
werden. Dass dies so ist, wird sich im Laufe des Verfahrens so ergeben,
wir werden daher zun"achst weiter von $m$ Gleichungen ausgehen.
\subsection{Ein Beispiel}
Als Beispiel wollen wir das Gleichungssystem
\[
\begin{linsys}{3}
3\;x%
\begin{picture}(0,0)
\color{red}\put(-13,4){\circle{12}}
\end{picture}%
&-&6y&=&9\\
2\;x&+&4y&=&-2
\end{linsys}
\]
l"osen. Wir f"uhren dabei genau Buch "uber die einzelnen Schritte,
damit wir daraus sp"ater ein allgemeines Verfahren ableiten k"onnen.
Dazu werden wir in jedem Schritt zun"achst einzeichnen, um welchen
Term wir uns als n"achstes k"ummern wollen.

\subsubsection{Schritt I: $x$ isolieren}
Dividieren wir die erste Gleichung durch $3$, steht $x$ ohne Koeffizient
auf der linken Seite:
\[
\begin{linsys}{3}
3\;x%
\begin{picture}(0,0)
\color{red}\put(-13,4){\circle{12}}
\end{picture}%
&-&6y&=&9\\
2\;x&+&4y&=&-2
\end{linsys}
\qquad\rightarrow\qquad
\begin{linsys}{3}
x%
\begin{picture}(0,0)
\color{red}\put(-13,4){\circle{12}}
\end{picture}%
&-&2y&=&3\\
2\;x&+&4y&=&-2
\end{linsys}
\]
Es wird damit einfacher, nach $x$ aufzul"osen,
und das gefundene Resultat dazu zu verwenden, die anderen $x$ zu
eliminieren
\subsubsection{Schritt F: $x$ aus Folgegleichungen elminieren}
Mit Hilfe der ersten Gleichung k"onnen wir jetzt auch das $x$ in
der zweiten Gleichung eliminieren. Dazu subtrahieren wir das Doppelte
der ersten Gleichung von der zweiten. Resultat:
\[
\begin{linsys}{3}
x&-&2y&=&3\\
2x%
\begin{picture}(0,0)
\color{blue}\drawline(-14,-3)(-14,10)(12,10)(12,-3)
\end{picture}%
&+&4y&=&-2
\end{linsys}
\qquad
\rightarrow
\qquad
\begin{linsys}{3}
x&-&2y&=&3\\
 %
\begin{picture}(0,0)
\color{blue}\drawline(-14,-3)(-14,10)(12,10)(12,-3)
\end{picture}%
& &8y&=-8
\end{linsys}
\]
\subsubsection{Schritt I: $y$ isolieren}
Durch dividieren der zweiten Gleichung durch $8$ erreichen wir, dass
$y$ auf der linken Seite alleine steht, wir haben also bereits
eine Unbekannte bestimmt:
\[
\begin{linsys}{3}
x-2\;y&=3\\
8\;y%
\begin{picture}(0,0)
\color{red}\put(-12,4){\circle{12}}
\end{picture}%
&=-8
\end{linsys}
\qquad
\rightarrow
\qquad
\begin{linsys}{3}
x&-&2\;y &=&3\\
&&y%
\begin{picture}(0,0)
\color{red}\put(-12,4){\circle{12}}
\end{picture}%
&=&-1
\end{linsys}
\]

\subsubsection{Schritt R: $y$ aus Vorg"angergleichungen eliminieren}
Mit der zweiten Gleichung k"onnen wir jetzt auch das $y$ aus der ersten
Gleichung eleminieren. Wir addieren dazu das Doppelte der zweiten
Gleichung zur ersten:
\[
\begin{linsys}{3}
x&-&2y%
\begin{picture}(0,0)
\color{blue}\drawline(-32,10)(-32,-4)(2,-4)(2,10)
\end{picture}%
&=&3\\
&&y&=&-1
\end{linsys}
\qquad
\rightarrow
\qquad
\begin{linsys}{3}
x&&%
\begin{picture}(0,0)
\color{blue}\drawline(-18,10)(-18,-4)(1,-4)(1,10)
\end{picture}%
&=&1\\
&&y&=&-1
\end{linsys}
\]
Damit ist das Gleichungssystem vollst"andig gel"ost.

Offenbar ist es m"oglich, mit IFR-Schritten das Gleichungssystem zu l"osen.
Die Variablen und Operationszeichen haben dabei nur Platzhalterfunktion
gehabt, wir k"onnten in allen Schritten auch nur die Koeffizienten
in einem sogenannten Tableau schreiben:
\index{Tableau}
\begin{gather*}
\begin{tabular}{|>{$}c<{$}>{$}c<{$}|>{$}c<{$}|}
\hline
\begin{picture}(0,0)
{\color{red}\put(2.5,4){\circle{12}}}
\end{picture}
3&-6&9\\
2&4&-2\\
\hline
\end{tabular}
\rightarrow
\begin{tabular}{|>{$}c<{$}>{$}c<{$}|>{$}c<{$}|}
\hline
{\color{red}1}&-2&3\\
2%
\begin{picture}(0,0)
\color{blue}\drawline(-8,-2)(-8,10)(1,10)(1,-2)
\end{picture}%
&4&-2\\
\hline
\end{tabular}
\rightarrow
\begin{tabular}{|>{$}c<{$}>{$}c<{$}|>{$}c<{$}|}
\hline
1&-2&3\\
{\color{blue}0}&\begin{picture}(0,0)
\color{red}\put(2.8,4){\circle{12}}
\end{picture}
8&-8\\
\hline
\end{tabular}
\rightarrow
\begin{tabular}{|>{$}c<{$}>{$}c<{$}|>{$}c<{$}|}
\hline
1&
\begin{picture}(0,0)
\color{blue}\drawline(1,9)(1,-2)(20,-2)(20,9)
\end{picture}%
-2&3\\
0&{\color{red}1}&-1\\
\hline
\end{tabular}
\rightarrow
\begin{tabular}{|>{$}c<{$}>{$}c<{$}|>{$}c<{$}|}
\hline
1&{\color{blue}0}&1\\
0&1&-1\\
\hline
\end{tabular}
\end{gather*}
Die drei Arten von Operationen k"onnen ausschliesslich "uber die
Koeffizientenschemata beschrieben werden:
\begin{itemize}
\item[\bf I:] Eine Variable isolieren bedeutet, die ganze
Zeile durch den Koeffizienten dieser Variable zu teilen.
\item[\bf F:] Eine Variable mit Hilfe einer anderen Zeile zu eliminieren
bedeutet, ein geeignetes Vielfaches dieser anderen Zeile zu subtrahieren,
so dass der Koeffizient zu $0$ wird.
\item[\bf R:] Eliminieren aus einer fr"uheren Gleichung funktioniert
genau gleich wie {\bf F}.
\end{itemize}
Da {\bf F} und {\bf R} offenbar identisch sind, fassen wir diese
Operationen unter dem Namen {\bf E} zusammen.

\subsection{Operationen I und E f"ur drei Unbekannte}
Als Beispiel f"uhren wir das Verfahren auch noch f"ur die drei Unbekannten
des Gleichungssystems
\begin{equation}
\begin{linsys}{3}
x&+&2y&+&3z&=&10\\
6x&+&5y&+&4z&=&32\\
x&-&y&+&z&=&2
\end{linsys}
\end{equation}
Die zugeh"orige Folge von Tableaux ist
\begin{align*}
\begin{tabular}{|>{$}c<{$}>{$}c<{$}>{$}c<{$}|>{$}c<{$}|}
\hline
\begin{picture}(0,0)
\color{red}\put(3,4){\circle{12}}
\end{picture}%
1&2&3&10\\
6&5&4&32\\
\begin{picture}(0,0)
\color{blue}\drawline(-3,-2)(-3,25)(9,25)(9,-2)
\end{picture}%
1&-1&1&2\\
\hline
\end{tabular}
&\overset{\text{\bf E}}\rightarrow
\begin{tabular}{|>{$}c<{$}>{$}c<{$}>{$}c<{$}|>{$}c<{$}|}
\hline
1&2&3&10\\
0&-7%
\begin{picture}(0,0)
\color{red}\put(-7,4){\circle{17}}
\end{picture}%
&-14&-28\\
0&-3%
\begin{picture}(0,0)
\color{blue}\drawline(-15,-3)(-15,10)(1,10)(1,-3)
\end{picture}%
&-2&-8\\
\hline
\end{tabular}
\\
&\overset{\text{\bf I,E}}\rightarrow
\begin{tabular}{|>{$}c<{$}>{$}c<{$}>{$}c<{$}|>{$}c<{$}|}
\hline
1&2&3&10\\
0&1&2&4\\
0&0&\begin{picture}(0,0)
\color{red}\put(3,4){\circle{12}}
\end{picture}%
4&4\\
\hline
\end{tabular}
\\
&\overset{\text{\bf I}}\rightarrow
\begin{tabular}{|>{$}c<{$}>{$}c<{$}>{$}c<{$}|>{$}c<{$}|}
\hline
1&2&3&10\\
0&1&\begin{picture}(0,0)
\color{blue}\drawline(-3,24)(-3,-3)(8,-3)(8,24)
\end{picture}%
2&4\\
0&0&1&1\\
\hline
\end{tabular}
\\
&\overset{\text{\bf R}}\rightarrow
\begin{tabular}{|>{$}c<{$}>{$}c<{$}>{$}c<{$}|>{$}c<{$}|}
\hline
1&\begin{picture}(0,0)
\color{blue}\drawline(-3,9)(-3,-3)(8,-3)(8,9)
\end{picture}%
2&0&7\\
0&1&0&2\\
0&0&1&1\\
\hline
\end{tabular}
\\
&\overset{\text{\bf R}}\rightarrow
\begin{tabular}{|>{$}c<{$}>{$}c<{$}>{$}c<{$}|>{$}c<{$}|}
\hline
1&0&0&3\\
0&1&0&2\\
0&0&1&1\\
\hline
\end{tabular}
\end{align*}
Daraus lesen wir die L"osung $x=3$, $y=2$ und $z=1$ ab.
Im ersten Schritt konnte man sich eine Operation {\bf I} sparen, weil
der Koeffizient von $x$ bereits $1$ ist.

\subsection{Der Normalfall}
Mit den Operationen {\bf I} und {\bf E} kann man das Gleichungssystem
also l"osen. Dabei geht man in zwei Phasen vor, die man wie folgt
visualisieren kann:
\begin{gather*}
\begin{tabular}{|>{$}c<{$}>{$}c<{$}>{$}c<{$}>{$}c<{$}|>{$}c<{$}|}
\hline
\begin{picture}(0,0)
\color{red}\put(2.7,3.1){\circle{12}}
\end{picture}%
*&*&\dots&*&*\\
*&*&\dots&*&*\\
\vdots& \vdots& \ddots& \vdots&\vdots\\
\begin{picture}(0,0)
\color{blue}\drawline(-3,-2)(-3,43)(8.5,43)(8.5,-2)
\end{picture}%
*&*&\dots&*&*\\
\hline
\end{tabular}
\rightarrow
\begin{tabular}{|>{$}c<{$}>{$}c<{$}>{$}c<{$}>{$}c<{$}|>{$}c<{$}|}
\hline
1&*&\dots&*&*\\
0&1&\dots&*&*\\
\vdots& \vdots& \ddots&\begin{picture}(0,0)
\color{blue}\drawline(-4,43)(-4,-3)(7,-3)(7,43)
\end{picture}%
\vdots&\vdots\\
0&0&\dots&1&*\\
\hline
\end{tabular}
\rightarrow
\begin{tabular}{|>{$}c<{$}>{$}c<{$}>{$}c<{$}>{$}c<{$}|>{$}c<{$}|}
\hline
1&0&\dots&0&*\\
0&0&\dots&0&*\\
\vdots& \vdots& \ddots& \vdots&\vdots\\
0&1&\dots&1&*\\
\hline
\end{tabular}
\end{gather*}
\index{Vorwaertsreduktion@Vorw\"artsreduktion}
\index{Rueckwaertseinsetzen@R\"uckw\"artseinsetzen}
Die erste Phase wird auch Vorw"artsreduktion genannt, die
zweite Phase R"uckw"artseinsetzen.
\subsubsection{Vorw"artsreduktion}
In der ersten Phase muss man also alle Elemente unterhalb der
Diagonalen zu Null machen. Mit Operationen {\bf I} und {\bf E}
geht man dazu zeilenweise vor.
In jeder Zeile macht man mit {\bf I}
zun"achst den vordersten nicht verschwindenen Koeffizienten, das sogenannte
Pivot-Element zu $1$.
\index{Pivot-Element}
Die Zeile und Spalte dieses Elements heisst deshalb
auch Pivot-Zeile bzw.~Pivot-Spalte.
\index{Pivot-Zeile}
\index{Pivot-Spalte}
Dann subtrahiert man mit der Operation {\bf E} geeignete Vielfache
der Pivot-Zeile, so dass in der Spalte
unter der eben entstandenen $1$ alle Koeffizienten verschwinden.
\begin{gather*}
\begin{tabular}{|>{$}c<{$}>{$}c<{$}>{$}c<{$}>{$}c<{$}>{$}c<{$}|>{$}c<{$}|}
\hline
1&*&*&\dots&*&*\\
0&\begin{picture}(0,0)
\color{red}\put(2.8,3.05){\circle{12}}
\end{picture}%
*&*&\dots&*&*\\
0&*&*&\dots&*&*\\
\vdots&\vdots& \vdots& \ddots& \vdots&\vdots\\
0&*&*&\dots&*&*\\
\hline
\end{tabular}
\overset{\text{\bf I}}\rightarrow
\begin{tabular}{|>{$}c<{$}>{$}c<{$}>{$}c<{$}>{$}c<{$}>{$}c<{$}|>{$}c<{$}|}
\hline
1&*&*&\dots&*&*\\
0&1&*&\dots&*&*\\
0&*&*&\dots&*&*\\
\vdots&\vdots& \vdots& \ddots& \vdots&\vdots\\
0&\begin{picture}(0,0)
\color{blue}\drawline(-2,-2)(-2,43)(8,43)(8,-2)
\end{picture}%
*&*&\dots&1&*\\
\hline
\end{tabular}
\overset{\text{\bf E}}\rightarrow
\begin{tabular}{|>{$}c<{$}>{$}c<{$}>{$}c<{$}>{$}c<{$}>{$}c<{$}|>{$}c<{$}|}
\hline
1&*&*&\dots&*&*\\
0&1&*&\dots&*&*\\
0&0&*&\dots&*&*\\
\vdots& \vdots&\vdots& \ddots& \vdots&\vdots\\
0&0&*&\dots&*&*\\
\hline
\end{tabular}
\end{gather*}
In Formeln wird im Schritt {\bf I} auf der Zeile $i$
durch $a_{ii}$ dividiert:
\begin{align*}
a_{ij}'&=\frac{a_{ij}}{a_{ii}}&& i < j \le m,\\
b_i'&=\frac{b_i}{a_{ii}}
\end{align*}
Die Operation {\bf E} auf der Zeile $k>i$ ist
\begin{align*}
a_{kj}'&=a_{kj}-a_{ki}a_{ij}&&i < k \le m, i \le j\le n\\
b_k'&=b_k-a_{ki}b_i&&i < k \le m
\end{align*}
F"uhrt man beide Operationen in einem Schritt durch, erh"alt man die
Gleichungen:
\begin{equation}
\begin{aligned}
a_{ij}'&=\frac{a_{ij}}{a_{ii}}&&&b_i'&=\frac{b_i}{a_{ii}}\\
a_{kj}'&=a_{kj}-a_{ki}\frac{a_{ij}}{a_{ii}}&
&\qquad&b_k'&=b_k-a_{ki}\frac{a_{ij}}{a_{ii}}
\end{aligned}
\label{vorwaertsreduktion}
\end{equation}
mit $i < k\le m$ und $i\le j\le n$. Dieser Schritt muss f"ur $i=1,\dots,m$
wiederholt werden.

\subsubsection{R"uckw"artseinsetzen}
Das R"uckw"artseinsetzen kommt ohne die {\bf I}-Operationen aus.
Es eliminiert die nicht verschwinden Element oberhalb der Diagonalen
Spaltenweise, beginnend bei der letzten. F"ur die dritte Spalte bespielsweise
bedeutet dies bildlich:
\begin{gather*}
\begin{tabular}{|>{$}c<{$}>{$}c<{$}>{$}c<{$}>{$}c<{$}>{$}c<{$}|>{$}c<{$}|}
\hline
1&*&*&\dots&0&*\\
0&1&\begin{picture}(0,0)
\color{blue}\drawline(-2,22)(-2,-2)(8,-2)(8,22)
\end{picture}%
*&\dots&0&*\\
0&0&1&\dots&0&*\\
\vdots&\vdots& \vdots& \ddots& \vdots&\vdots\\
0&0&0&\dots&1&*\\
\hline
\end{tabular}
\overset{\text{\bf E}}\rightarrow
\begin{tabular}{|>{$}c<{$}>{$}c<{$}>{$}c<{$}>{$}c<{$}>{$}c<{$}|>{$}c<{$}|}
\hline
1&*&0&\dots&0&*\\
0&1&0&\dots&0&*\\
0&0&1&\dots&0&*\\
\vdots& \vdots&\vdots& \ddots& \vdots&\vdots\\
0&0&0&\dots&1&*\\
\hline
\end{tabular}
\end{gather*}
In Formeln bedeutet dies f"ur die {\bf E}-Operation, bei der die
Spalte mit der Nummer $i$ ``leerger"aumt'' wird:
\begin{align}
a_{kj}'&=a_{kj}- a_{ki} a_{ij}
&
b_k'&=b_k- a_{ki}b_i
\label{rueckwaertseinsetzen}
\end{align}
Dieser Schritt muss f"ur $i=m,\dots,2$ wiederholt werden.
\subsection{Sonderf"alle}
Nicht immer kann das Verfahren so problemlos durchgef"uhrt werden.
Es kann vorkommen, dass eines der Pivot-Elemente $a_{ii}=0$ ist.
Da beim Vorw"artsreduzieren immer durch das Pivot-Element geteilt
werden muss, bricht das Verfahren in diesem Falle ab, m"oglicherweise
obwohl das Gleichungssystem l"osbar ist. Zwei Strategien
sind denkbar, mit dieser Situation umzugehen.

Nicht nur ein verschwindendes Pivot-Element f"uhrt zu Problemen,
auch die Division durch ein kleines Pivot-Element kann bereits die
Rechengenauigkeit un\-g"unstig beeinflussen. Daher kann es je nach
Gleichungssystem geraten sein, eine der folgenden Strategien anzuwenden,
um die Genauigkeit der Resultate zu verbessern.
\subsubsection{Gleichungen umordnen}
Die Reihenfolge der Gleichungen hat keinen Einfluss auf die L"osungsmenge.
Ist $a_{ii}=0$, kann man jede der Gleichungen $i+1$ bis $m$ an Stelle
der Gleichung Nummer $i$ verwenden.

\begin{beispiel}[\bf Beispiel] Im Gleichungssystem $y=1, x=2$ versagt
in der Standardreihenfolge von Gleichungen und Variablen bereits der
erste Schritt:
\begin{gather*}
\begin{tabular}{|>{$}c<{$}>{$}c<{$}|>{$}c<{$}|}
\hline
0%
\begin{picture}(0,0)
\color{red}\put(-3,4){\circle{12}}
\put(4,0){!}
\end{picture}%
&1&1\\
1&0&2\\
\hline
\end{tabular}
\end{gather*}
Vertauscht man aber die beiden Zeilen, wird daraus
\begin{gather*}
\begin{tabular}{|>{$}c<{$}>{$}c<{$}|>{$}c<{$}|}
\hline
\begin{picture}(0,0)
\color{red}\put(2.7,4){\circle{12}}
\end{picture}%
1&0&2\\
0&1&1\\
\hline
\end{tabular}
\end{gather*}
was nicht nur l"osbar, sondern bereits gel"ost ist.
\end{beispiel}

\subsubsection{Variablen umordnen}
Die Reihenfolge der Variablen hat keinen Einfluss auf die L"osungsmenge.
Ist $a_{ii}=0$ kann jede andere Spalte $i+1$ bis $n$ Anstelle der
Spalte mit der Nummer $i$ verwendet werden. Allerdings muss man "uber
die Vertauschungen von Spalten Buch f"uhren, zum Beispiel indem man in
einer zus"atzlichen Zeile die Bedeutung der Spalten mitschreibt.

\begin{beispiel}[\bf Beispiel]
Wir versuchen, das Gleichungssystem
$$
\begin{linsys}{3}
   &   & y & - & z & = & 2\\
   &   & y & + & z & = & 4\\
-x &   &   & + & z & = & 2
\end{linsys}
$$
zu l"osen. Zweimal w"ahrend der Durchf"uhrung des Gauss-Algorithmus,
gekennzeichnet durch {\color{red}!},
sind wir gezwungen%
, durch eine Spaltenvertauschung daf"ur zu
sorgen, dass das Pivot-Element nicht $0$ ist%
\footnote{Es w"are nat"urlich auch m"oglich,
mit Zeilenvertauschungen das gleiche Ziel zu erreichen, doch soll
dieses Beispiel zeigen, wie man Spaltenvertauschungen f"ur diesen
Zweck einsetzen kann.}%
.
\begin{align*}
\begin{tabular}{|>{$}c<{$}>{$}c<{$}>{$}c<{$}|>{$}c<{$}|}
\hline
x&y&z&\\
\hline
0%
\begin{picture}(0,0)
\color{red}\put(-3,4){\circle{12}}
\put(5,0){!}
\end{picture}%
&1&-1&2\\
0&1&1&4\\
-1&0&1&2\\
\hline
\end{tabular}
&\overset{\displaystyle *}{\rightarrow}
\begin{tabular}{|>{$}c<{$}>{$}c<{$}>{$}c<{$}|>{$}c<{$}|}
\hline
y&x&z&\\
\hline
\begin{picture}(0,0)
\color{red}\put(3,4){\circle{12}}
\end{picture}%
1&0&-1&2\\
1&0&1&4\\
\begin{picture}(0,0)
\color{blue}\drawline(-2,-2)(-2,25)(8,25)(8,-2)
\end{picture}%
0&-1&1&2\\
\hline
\end{tabular}
\rightarrow
\begin{tabular}{|>{$}c<{$}>{$}c<{$}>{$}c<{$}|>{$}c<{$}|}
\hline
y&x&z&\\
\hline
1&0&-1&2\\
0&0%
\begin{picture}(0,0)
\color{red}\put(-3,4){\circle{12}}
\put(5,0){!}
\end{picture}%
&2&2\\
0&-1&1&2\\
\hline
\end{tabular}
\\
&\overset{\displaystyle *}{\rightarrow}
\begin{tabular}{|>{$}c<{$}>{$}c<{$}>{$}c<{$}|>{$}c<{$}|}
\hline
y&z&x&\\
\hline
1&-1&0&2\\
0& 2%
\begin{picture}(0,0)
\color{red}\put(-3,3.5){\circle{12}}
\end{picture}%
&0&2\\
0&1\begin{picture}(0,0)
\color{blue}\drawline(-8,-2)(-8,10)(2,10)(2,-2)
\end{picture}%
&-1&2\\
\hline
\end{tabular}
\rightarrow
\begin{tabular}{|>{$}c<{$}>{$}c<{$}>{$}c<{$}|>{$}c<{$}|}
\hline
y&z&x&\\
\hline
1&-1&0&2\\
0&1& 0&1\\
0&0&-1%
\begin{picture}(0,0)
\color{red}\put(-7,4){\circle{15}}
\end{picture}%
&1\\
\hline
\end{tabular}
\\
&\rightarrow
\begin{tabular}{|>{$}c<{$}>{$}c<{$}>{$}c<{$}|>{$}c<{$}|}
\hline
y&z&x&\\
\hline
1&-1%
\begin{picture}(0,0)
\color{blue}\drawline(-15,10)(-15,-2)(1,-2)(1,10)
\end{picture}%
&0&2\\
0&1& 0
\begin{picture}(0,0)%
\color{blue}\drawline(-10,24)(-10,-2)(3,-2)(3,24)%
\end{picture}%
&1\\
0&0&1&-1\\
\hline
\end{tabular}
\rightarrow
\begin{tabular}{|>{$}c<{$}>{$}c<{$}>{$}c<{$}|>{$}c<{$}|}
\hline
y&z&x&\\
\hline
1&0&0&3\\
0&1&0&1\\
0&0&1&-1\\
\hline
\end{tabular}
\\
&\overset{\displaystyle *}{\rightarrow}
\begin{tabular}{|>{$}c<{$}>{$}c<{$}>{$}c<{$}|>{$}c<{$}|}
\hline
x&y&z&\\
\hline
1&0&0&-1\\
0&1&0&3\\
0&0&1&1\\
\hline
\end{tabular}
\end{align*}
Im ersten Schritt wurden die ersten beiden
Spalten vertauscht, im dritten Schritt die letzten
beiden. Im letzten Schritt wurden die Variablen
und Gleichungen wieder in die ``Standardreihenfolge'' gebracht.
Die L"osung ist $(x,y,z)=(-1, 3, 1)$.
\end{beispiel}

\section{Lineare Abh"angigkeit}
\index{abhaengig@abh\"angig, linear}
\index{linear abh\"angig}
Im Gleichungssystem
\begin{align*}
x+2y&=5\\
2x+4y&=10
\end{align*}
ist die zweite Gleichung das Doppelte der ersten. Sie ist
also immer genau dann erf"ullt, wenn die erste erf"ullt ist.
Insbesondere schr"ankt sie die Menge der m"oglichen L"osungen
nicht weiter ein, die L"osungsmenge ist
$$\mathbb L=\{(x,y)\,|\, x+2y=5\}.$$
Auch kompliziertere Beziehungen zwischen Gleichungen sind
m"oglich. Im Gleichungssystem
\begin{align*}
 x+4y+5z&=1\\
3x-6y-2z&=-1\\
4x-2y+3z&=0
\end{align*}
ist die letzte Gleichung die Summe der ersten beiden. Sie ist also
immer bereits erf"ullt, wenn die ersten beiden Gleichungen erf"ullt
ist. Damit schr"ankt auch sie die L"osungsmenge nicht ein.

Zur Definition der eindeutigen L"osung eines Gleichungssystems 
tragen offenbar nur diejenigen Gleichungen etwas bei, die unabh"angig
von den anderen Gleichungen erf"ullbar sind. Wenn eine Gleichung
durch andere Gleichungen ausgedr"uckt werden kann, kann man sie
eliminieren. L"asst sich die letzte Gleichung durch die
anderen ausdr"ucken, gibt es Koeffizienten $\lambda_i$ so, dass
\begin{equation*}
\begin{aligned}
l_m(x_1,\dots,x_n)&=\lambda_1 l_1(x_1,\dots,x_n)+\dots+\lambda_{m-1}l_{m-1}(x_1,\dots, x_n)\\
b_m&=\lambda_1b_1+\dots+\lambda_{m-1}b_{m-1}
\end{aligned}
\end{equation*}
Setzt man $\lambda_m=-1$ und bringt alle Terme auf eine Seite,
dann bedeuten diese Gleichungen, dass
\begin{equation}
\begin{aligned}
0&=\lambda_1l_1(x_1,\dots,x_n)+\dots+\lambda_ml_m(x_1,\dots,x_n)\\
0&=\lambda_1b_1+\dots+\lambda_mb_m
\end{aligned}
\label{linearabhaengigegleichungen}
\end{equation}
Gibt es keine solche Beziehung, sind die Gleichungen unabh"angig
voneinander.
Diese Form verwenden wir, um den Begriff der linearen Abh"angigkeit
zu definieren.
\begin{definition}
Die Linearformen $l_1,\dots,l_m$ heissen linear unabh"angig,
wenn die einzige lineare Beziehung
\begin{equation}
\lambda_1l_1(x_1,\dots,x_n)+\dots+\lambda_ml_m(x_1,\dots,x_n)=0
\label{linearabhaengig}
\end{equation}
Sie heissen linear abh"angig, wenn es eine Beziehung (\ref{linearabhaengig})
gibt, in der nicht alle Koeffizienten $\lambda_i$ verschwinden.

Die Gleichungen $l_i(x_1,\dots,x_n)=b_i,1\le i\le m$ heissen linear unabh"angig,
wenn  die einzige Beziehung der Form (\ref{linearabhaengigegleichungen})
die Koeffiziente $\lambda_i=0$ hat. Sie heissen linear abh"angig, wenn
es eine Beziehung der Form (\ref{linearabhaengigegleichungen}) gibt.
\end{definition}

Wenn Linearformen oder Gleichungen linear abh"angig sind, dann kann
man eine oder mehrere von ihnen eliminieren, da sie zur L"osung nichts
beitragen.
Da der Gauss-Algorithmus versucht, genau die minimal ben"otigten Gleichungen
zu ermitteln, wird er eine Zeile voller Nullen hervorbringen.

\begin{hilfssatz}
Die Linearformen $l_1,\dots,l_m$ sind genau dann linear abh"angig, wenn
die Anwendung des Gaussalgorithmus auf das Koeffizientenschema dieser
Linearformen eine Zeile voller Nullen hervorbringt. Insbesondere sind
mehr als $n$ Linearformen immer linear abh"angig. Ist eine der Linearformen
bereits $0$, dann sind die Linearformen auf jeden Fall linear abh"angig.
\end{hilfssatz}

\begin{proof}[Beweis]
Ist $m>n$, dann erzeugt der Gauss-Algorithmus im besten Fall $n$ Zeilen,
die ausser Nullen in der Zeile $i$ auch noch in der $i$-ten Spalte eine
$1$ enthalten. Alle nachfolgenden Zeilen bestehen jedoch ausschliesslich
aus Nullen, somit sind die Gleichungen linear abh"angig.

Ist zum Beispiel $l_i(x_1,\dots,x_n)=0$, dann ergibt sich mit den
Koeffizienten
$$\lambda_1=0,\dots,\lambda_i=1,\dots \lambda_m=0$$
eine lineare Beziehung
$$\lambda_1l_1+\dots+\lambda_il_i+\dots+\lambda_ml_m=l_i=0,$$
in der nicht alle Koeffizienten verschwinden.
\end{proof}

Mit dem Gauss-Algorithmus k"onnen wir sehr rasch entscheiden, ob Zeilen 
linear abh"angig sind. Wenn wir aber die $\lambda_i$ bestimmen wollen,
die gem"ass Definition existieren sollen, dann m"ussen wir noch etwas
mehr arbeiten.
Wir illustrieren dies an einem Beispiel.

\begin{beispiel}
Der Gauss-Algorithmus
erkennt die folgenden Zeilen als linear abh"angig:
\begin{align*}
\begin{tabular}{|>{$}c<{$}>{$}c<{$}>{$}c<{$}|}
\hline
1\begin{picture}(0,0)
\color{red}\put(-3,4){\circle{12}}
\end{picture}%
&2&3\\
6&5&4\\
7\begin{picture}(0,0)
\color{blue}\drawline(-8,-2)(-8,24)(2,24)(2,-2)
\end{picture}%
&8&9\\
\hline
\end{tabular}
&\rightarrow
\begin{tabular}{|>{$}c<{$}>{$}c<{$}>{$}c<{$}|}
\hline
1&2&3\\
0&-7\begin{picture}(0,0)
\color{red}\put(-6,4){\circle{16}}
\end{picture}%
&-14\\
0&-6&-12\\
\hline
\end{tabular}
\rightarrow
\begin{tabular}{|>{$}c<{$}>{$}c<{$}>{$}c<{$}|}
\hline
1&2&3\\
0&1&2\\
0&-6%
\begin{picture}(0,0)
\color{blue}\drawline(-14,-2)(-14,10)(1,10)(1,-2)
\end{picture}%
&-12\\
\hline
\end{tabular}
\rightarrow
\begin{tabular}{|>{$}c<{$}>{$}c<{$}>{$}c<{$}|}
\hline
1&2&3\\
0&1&2\\
0&0&0\\
\hline
\end{tabular}
\end{align*}
Es gibt also Zahlen $\lambda_1$, $\lambda_2$ und $\lambda_3$,
so dass die Summe der $\lambda_i$-fachen der $i$-ten Zeile zusammen
die Null-Zeile ergeben:
\[
\begin{tabular}{>{$}c<{$}>{$}c<{$}|>{$}c<{$}>{$}c<{$}>{$}c<{$}}
\lambda_1&\cdot&1&2&3\\
\lambda_2&\cdot&6&5&4\\
\lambda_3&\cdot&7&8&9\\
\hline
&&0&0&0
\end{tabular}
\]
F"ur die $\lambda_i$ finden wir daraus ein lineares Gleichungssystem:
$$
\begin{linsys}{3}
1\lambda_1&+&6\lambda_2&+&7\lambda_3&=&0\\
2\lambda_1&+&5\lambda_2&+&8\lambda_3&=&0\\
3\lambda_1&+&4\lambda_2&+&9\lambda_3&=&0
\end{linsys}.
$$
Man beachte, dass das Koeffizienten-Schema gegen"uber dem
gegen"uber dem urspr"unglichen an der Diagonalen
gespiegelt worden ist.

Das Gleichungssystem kann jetzt mit Hilfe des
Gauss-Algorithmus gel"ost werden:
\begin{align*}
\begin{tabular}{|>{$}c<{$}>{$}c<{$}>{$}c<{$}|}
\hline
1%
\begin{picture}(0,0)
\color{red}\put(-3,4){\circle{12}}
\end{picture}%
&6&7\\
2&5&8\\
3%
\begin{picture}(0,0)
\color{blue}\drawline(-9,-2)(-9,24)(2,24)(2,-2)
\end{picture}%
&4&9\\
\hline
\end{tabular}
&\rightarrow
\begin{tabular}{|>{$}c<{$}>{$}c<{$}>{$}c<{$}|}
\hline
1&6&7\\
0&-7%
\begin{picture}(0,0)
\color{red}\put(-6.5,4){\circle{15}}
\end{picture}%
&-6\\
0&-14&-12\\
\hline
\end{tabular}
\rightarrow
\begin{tabular}{|>{$}c<{$}>{$}c<{$}>{$}c<{$}|}
\hline
1&6&7\\
0&1&\frac67\\
0&-14%
\begin{picture}(0,0)
\color{blue}\drawline(-20,-2)(-20,10)(1,10)(1,-2)
\end{picture}%
&-12\\
\hline
\end{tabular}
\\
&\rightarrow
\begin{tabular}{|>{$}c<{$}>{$}c<{$}>{$}c<{$}|}
\hline
1&6%
\begin{picture}(0,0)
\color{blue}\drawline(-8,9)(-8,-2)(2,-2)(2,9)
\end{picture}%
&7\\
0&1&\frac67\\
0&0&0\\
\hline
\end{tabular}
\rightarrow
\begin{tabular}{|>{$}c<{$}>{$}c<{$}>{$}c<{$}|}
\hline
1&0&\frac{13}{7}\\
0&1&\frac67\\
0&0&0\\
\hline
\end{tabular}
\end{align*}
Die beiden nicht verschwindenden Zeilen bedeuten, dass $\lambda_1$ und
$\lambda_2$ aus $\lambda_3$ berechnet werden k"onnen:
\begin{align*}
\lambda_1&=-\frac{13}7\lambda_3&\lambda_2&=-\frac{6}7\lambda_3.
\end{align*}
Eine ganzzahlige L"osung ergibt sich zum Beispiel f"ur $\lambda_3=7$,
also
$$
(
\lambda_1,
\lambda_2,
\lambda_3
)
=(
-13,-6,7
),
$$
was man durch nachrechnen auch best"atigen kann. Da in diesem Gleichungssystem
die auf der rechten Seite nur Nullen stehen, k"onnen wir schliessen,
dass eine L"osung mit $\lambda_i\ne 0$ nur im singul"aren Fall gefunden
werden kann. Die gespiegelten Koeffizienten sind also immer dann singul"ar,
wenn auch die ungespiegelten Koeffizienten singul"ar sind.
\end{beispiel}

\section{L"osungsmenge und Rang}
Was passiert, wenn im Gauss-Algorithmus die Anzahl der Gleichungen
nicht mit der Anzahl der Unbekannten "ubereinstimmt? Oder wenn
pl"otzlich alle Folgezeilen nur noch Koeffizienten $0$ enthalten,
also auch unter Verwendung der Vertauschungsstrategien kein geeignetes
Pivot-Element mehr zur Verf"ugung steht? Eine eindeutige L"osung kann
es in diesem Fall nat"urlich nicht mehr geben. Die Alternative
``gar keine L"osung'' gegen ``unendlich viele L"osungen'' ist aber
noch offen.
Ausserdem ist die Zahl der Gleichungen, nach denen dieses Ph"anomen
auftritt, eine charakteristische Gr"osse des betrachteten Gleichungssystems,
der sogenannte Rang.

\subsection{Mehr Unbekannte als Gleichungen}
In diesem Fall kommt das Vorw"artsreduzieren bereits zu einem Ende,
wenn noch gar nicht alle Unbekannten verwendet wurden.
Das R"uckw"artseinsetzen kann dann auch nicht alle Spalten ``reinigen'',
so dasss rechts einige Spalten stehen bleiben:
\begin{gather}
\begin{array}{|ccccccc|c|}
\hline
1&\dots&0&*&\dots&*&&*\\
\vdots&\ddots&\vdots&\vdots&\ddots&\vdots&&\vdots\\
0&\dots&1&*&\dots&*&&*\\
\hline
\end{array}
\end{gather}
\index{waehlbar@w\"ahlbar}
Schreibt man dies wieder als System von Gleichungen kann man ablesen,
dass man die Unbekannten $x_{m+1}$ bis $x_n$ gar nicht bestimmen kann.
Diese k"onnen frei gew"ahlt werden, und bestimmen dann die Unbekannten
$x_1$ bis $x_m$. Konkreter: haben wir das Koeffizientenschema
\begin{gather}
\begin{array}{|ccccccc|c|}
\hline
1&\dots&0&a_{1,m+1}&\dots&a_{1,n}&&b_1\\
\vdots&\ddots&\vdots&\vdots&\ddots&\vdots&&\vdots\\
0&\dots&1&a_{m,m+1}&\dots&a_{m,n}&&b_m\\
\hline
\end{array}
\end{gather}
gefunden, bedeutet dies, dass die Unbekannten $x_1$ bis $x_m$ bestimmt
sind durch die Gleichungen
\begin{align*}
x_1&=b_1-a_{1,m+1}x_{m+1}-\dots-a_{1n}x_n\\
&\vdots\\
x_m&=b_m-a_{m,m+1}x_{m+1}-\dots-a_{mn}x_n
\end{align*}
sobald die Unbekannten $x_{m+1}$ bis $x_n$ vorgegeben sind.
\subsubsection{L"osungsmenge}
Im vorliegenden Fall hat man eine L"osungsmenge, die durch $n-m$
Parameter beschrieben ist, n"amlich durch die frei w"ahlbaren
Variablen $x_{m+1}$ bis $x_n$. Ein L"osungsvektor hat also die
Form 
\[
\begin{pmatrix}
x_1\\\vdots\\x_m\\x_{m+1}\\\vdots\\x_n
\end{pmatrix}
=
\begin{pmatrix}
b_1-a_{1,m+1}x_{m+1}-\dots-a_{1n}x_n\\
\vdots\\
b_m-a_{m,m+1}x_{m+1}-\dots-a_{mn}x_n\\
x_{m+1}\\
\vdots\\
x_n
\end{pmatrix}
\]
Man spricht von einem $n-m$-dimensionalen L"osungsraum. Die
L"osungsmenge kann also als
$$
\mathbb L
=
\left\{
\left.
\begin{pmatrix}
b_1-a_{1,m+1}x_{m+1}-\dots-a_{1n}x_n\\
\vdots\\
b_m-a_{m,m+1}x_{m+1}-\dots-a_{mn}x_n\\
x_{m+1}\\
\vdots\\
x_n
\end{pmatrix}
\right|
x_{m+1},\dots,x_n\in\mathbb R
\right\}
$$
geschrieben werden.
\subsubsection{Vektorform der L"osungsmenge}
Wir k"onnen allerdings auch die Parameter $x_{m+1}$ bis $x_n$ noch
etwas verdeutlichen, indem wir die L"osung als Linearkombination von
Vektoren mit Koeffizienten $x_{m+1}$ bis $x_n$ schreiben:
\[
\begin{pmatrix}
x_1\\\vdots\\x_m\\x_{m+1}\\\vdots\\x_n
\end{pmatrix}
=
\begin{pmatrix}
b_1\\\vdots\\b_m\\0\\\vdots\\0
\end{pmatrix}
+x_{m+1}\begin{pmatrix}
-a_{1,m+1}\\ \vdots\\-a_{m,m+1}\\1\\\vdots\\0
\end{pmatrix}
+\dots+
x_n
\begin{pmatrix}
-a_{1n}\\
\vdots\\
-a_{mn}\\
0\\
\vdots\\
1
\end{pmatrix}
\]
Die L"osungsmenge k"onnen wir jetzt etwas kompakter in Vektorform schreiben.
Dazu k"urzen wir ab
\[
\tilde b
=
\begin{pmatrix}
b_1\\\vdots\\b_m\\0\\\vdots\\0
\end{pmatrix}
,\qquad
\tilde a_{m+1}=
\begin{pmatrix}
-a_{1,m+1}\\ \vdots\\-a_{m,m+1}\\1\\\vdots\\0
\end{pmatrix}
,\qquad
\tilde a_n
=
\begin{pmatrix}
-a_{1n}\\
\vdots\\
-a_{mn}\\
0\\
\vdots\\
1
\end{pmatrix}
\]
Damit ist die L"osungsmenge
\[
\mathbb L = \{
\tilde b+x_{m+1}\tilde a_{m+1}+\dots+x_n\tilde a_n\;|\;x_{m+1},\dots,x_n\in\mathbb R\}.
\]

\begin{beispiel}
Man bestimme die L"osungsmenge des Gleichungssystems
%   1  -2   3  -0  -4   1
%   3  -1   4  -0   3  -2
%   3  -1   2  -4  -5   4
\[
\begin{linsys}{5}
 x_1&-&2x_2&+&3x_3& &    &-&4x_5&=& 1\phantom{.}\\
3x_1&-& x_2&+&4x_3& &    &+&3x_5&=&-2\phantom{.}\\
3x_1&-& x_2&+&2x_3&-&4x_4&-&5x_5&=& 4.\\
\end{linsys}
\]

\smallskip
{\parindent 0pt Der Gauss-Algorithmus liefert}
\begin{align*}
\begin{tabular}{|>{$}c<{$}>{$}c<{$}>{$}c<{$}>{$}c<{$}>{$}c<{$}|>{$}c<{$}|}
\hline
   1%
\begin{picture}(0,0)
\color{red}\put(-3,4){\circle{12}}
\end{picture}%
& -2&  3&  0& -4&  1\\
   3& -1&  4&  0&  3& -2\\
   3%
\begin{picture}(0,0)
\color{blue}\drawline(-9,-2)(-9,24)(2,24)(2,-2)
\end{picture}%
& -1&  2& -4& -5&  4\\
\hline
\end{tabular}
&
\rightarrow
\begin{tabular}{|>{$}c<{$}>{$}c<{$}>{$}c<{$}>{$}c<{$}>{$}c<{$}|>{$}c<{$}|}
\hline
   1& -2&  3&  0& -4&  1\\
   0&  5%
\begin{picture}(0,0)
\color{red}\put(-3,4){\circle{12}}
\end{picture}%
& -5&  0& 15& -5\\
   0&  5%
\begin{picture}(0,0)
\color{blue}\drawline(-8,-2)(-8,10)(2,10)(2,-2)
\end{picture}%
& -7& -4& -7&  1\\
\hline
\end{tabular}
\\
\rightarrow
\begin{tabular}{|>{$}c<{$}>{$}c<{$}>{$}c<{$}>{$}c<{$}>{$}c<{$}|>{$}c<{$}|}
\hline
   1& -2&  3&  0& -4&  1\\
   0&  1& -1%
\begin{picture}(0,0)
\color{blue}\drawline(-15,24)(-15,-2)(1,-2)(1,24)
\end{picture}%
&  0&  3& -1\\
   0&  0& -2%
\begin{picture}(0,0)
\color{red}\put(-7,4){\circle{15}}
\end{picture}%
& -4& -8&  6\\
\hline
\end{tabular}
&
\rightarrow
\begin{tabular}{|>{$}c<{$}>{$}c<{$}>{$}c<{$}>{$}c<{$}>{$}c<{$}|>{$}c<{$}|}
\hline
   1& -2%
\begin{picture}(0,0)
\color{blue}\drawline(-15,10)(-15,-2)(1,-2)(1,10)
\end{picture}%
&  0& -6&-16& 10\\
   0&  1&  0&  2&  7& -4\\
   0&  0&  1&  2&  4& -3\\
\hline
\end{tabular}
\\
&
\rightarrow
\begin{tabular}{|>{$}c<{$}>{$}c<{$}>{$}c<{$}>{$}c<{$}>{$}c<{$}|>{$}c<{$}|}
\hline
   1&  0&  0& -2& -2&  2\\
   0&  1&  0&  2&  7& -4\\
   0&  0&  1&  2&  4& -3\\
\hline
\end{tabular}\,.
\end{align*}
Die Variablen $x_4$ und $x_5$ sind frei w"ahlbar, also wird die 
L"osungsmenge:
\[
\mathbb L=
\left\{
\left.
\begin{pmatrix}2\\-4\\7\\0\\0\end{pmatrix}
+x_4\begin{pmatrix}2\\-2\\-2\\1\\0\end{pmatrix}
+x_5\begin{pmatrix}2\\-7\\-4\\0\\1\end{pmatrix}
\;
\right|
\;
x_4,x_5\in\mathbb R
\right\}.
\]
\end{beispiel}

\subsection{Mehr Gleichungen als Unbekannte}
Sind weniger Unbekannte als Gleichungen gegeben, dann wird das Gauss-Verfahren
das Gleichungssystem bestenfalls in die Form
$$
\begin{array}{|ccc|c|}
\hline
1&\dots&0&*\\
\vdots&\ddots&\vdots&\vdots\\
0&\dots&1&*\\
\hline
0&\dots&0&\color{red}*\\
\vdots&\ddots&\vdots&\color{red}\vdots\\
0&\dots&0&\color{red}*\\
\hline
\end{array}
$$
bringen. Die letzten $m-n$ Zeilen entsprechen Gleichungen, die auf
der linken Seite $0$ stehen haben. Sie sind nur dann erf"ullbar, wenn
auch auf der rechten Seite, an Stelle der Sterne, nur Nullen vorkommen.
Man hat also im Allgemeinen keine L"osung, sondern erst, wenn die Werte
auf der rechten Seite der $m-n$ letzten Gleichungen verschwinden,
also wenn Zusatzbedingungen erf"ullt sind.

\subsection{Mischformen}
Es kann auch passieren, dass das Gaussverfahren abbricht, bevor alle
Gleichungen aufgebraucht sind, das Schema sieht dann wie folgt aus:
\begin{equation}
\begin{array}{|cccccc|c|}
\hline
1&\dots&0 &*&\dots&*&*\\
\vdots&\ddots&\vdots &\vdots&\ddots&\vdots&\vdots\\
0&\dots&1 &*&\dots&*&*\\
\hline
0&\dots&0 &0&\dots&0&\color{red}*\\
\vdots&\ddots&\vdots &\vdots&\ddots&\vdots&\color{red}\vdots\\
0&\dots&0 &0&\dots&0&\color{red}*\\
\hline
\end{array}
\label{rangdef}
\end{equation}
In diesem Fall kann man auch durch vertauschen von Zeilen oder Spalten
das Verfahren nach $r$ Gleichungen nicht weiterf"uhren. 
Es existiert genau dann eine L"osung, wenn die letzten $m-r$ Gleichungen
mit Nullen auf der linken Seite auch Nullen auf der
rechten Seite haben.

Nehmen wir an, dass auf der rechten Seite der Nullzeilen auch
nur Nullzeilen stehen.
Offensichtlich k"onnen nur diejenigen Variablen bestimmt werden,
deren Spalten im Laufe des Verfahrens als Pivot-Spalten aufgetreten sind.
Alle anderen Variablen werden durch die vorhandenen Gleichungen nicht
bestimmt, und sind frei w"ahlbar. 

\begin{beispiel}
Man finde die L"osungsmenge des Gleichungssystems
\[
\begin{linsys}{4}
x&+&2y&+&3z&=&1\\
x&&&+&z&=&0\\
 &&2y&+&2z&=&1
\end{linsys}
\]
Der Gauss-Algorithmus liefert:
\begin{align*}
\begin{tabular}{|>{$}c<{$}>{$}c<{$}>{$}c<{$}|>{$}c<{$}|}
\hline
1%
\begin{picture}(0,0)
\color{red}\put(-3,4){\circle{12}}
\end{picture}%
&2&3&1\\
1&0&1&0\\
0%
\begin{picture}(0,0)
\color{blue}\drawline(-8,-2)(-8,24)(2,24)(2,-2)
\end{picture}%
&2&2&1\\
\hline
\end{tabular}
&\rightarrow
\begin{tabular}{|>{$}c<{$}>{$}c<{$}>{$}c<{$}|>{$}c<{$}|}
\hline
1&2&3&1\\
0&-2%
\begin{picture}(0,0)
\color{red}\put(-6,4){\circle{15}}
\end{picture}%
&-2&-1\\
0&2%
\begin{picture}(0,0)
\color{blue}\drawline(-8,-2)(-8,10)(2,10)(2,-2)
\end{picture}%
&2&1\\
\hline
\end{tabular}
\\
&\rightarrow
\begin{tabular}{|>{$}c<{$}>{$}c<{$}>{$}c<{$}|>{$}c<{$}|}
\hline
1&2%
\begin{picture}(0,0)
\color{blue}\drawline(-8,9)(-8,-2)(1,-2)(1,9)
\end{picture}%
&3&1\\
0&1&1&\frac12\\
0&0&0&\color{red}0\\
\hline
\end{tabular}
\rightarrow
\begin{tabular}{|>{$}c<{$}>{$}c<{$}>{$}c<{$}|>{$}c<{$}|}
\hline
1&0&1&0\\
0&1&1&\frac12\\
0&0&0&\color{red}0\\
\hline
\end{tabular}
\end{align*}
Offenbar wird die letzte Gleichung gar nicht ben"otigt, die
L"osungsmenge kann aus den ersten zwei Gleichungen
abgelesen werden:
\[
\mathbb L=
\left\{
\left.
\begin{pmatrix}
0\\\frac12\\0
\end{pmatrix}
+x_3\begin{pmatrix}
-1\\-1\\1
\end{pmatrix}
\;
\right|\;
x_3\in\mathbb R
\right\}.
\]
\end{beispiel}

\subsection{Definition des Rangs}
\index{Rang}
Offenbar spielt die Zahl $r$, die Anzahl der Einsen im Schema (\ref{rangdef}),
eine wesentliche Rolle, wir nennen diese Zahl den Rang der Matrix.

\begin{definition}
Der Rang einer Matrix $A$ ist die maximale Anzahl linear unabh"angiger Zeilen.
\end{definition}
\begin{satz}
Die maximale Anzahl linear unabh"angiger Spalten einer Matrix $A$ ist
$\operatorname{Rang}A$.
\end{satz}
Aus dem Rang l"asst sich also auch ermitteln, wieviele frei w"ahlbare
Parameter in die L"osungsmenge eingehen.
\begin{satz}
Ist $A$ eine $m\times n$-Matrix mit $r=\operatorname{Rang}A$ 
und $b$ ein $m$-Vektor, dann gibt es $n-r$ Vektoren $\tilde a_1$
bis $\tilde a_{n-r}$ und einen Vektor $\tilde b$ so, dass
\[
\mathbb L
=
\{
\tilde b+x_1\tilde a_1+\dots+x_{n-r}\tilde a_{n+r}\;|\;x_1,\dots,x_{n-r}\in\mathbb R
\}
\]
die L"osungsmenge des Gleichungssystems $Ax=b$ ist.
\end{satz}

\section{Mehrere Gleichungssysteme simultan l"osen\label{simultan}}
Mit dem Gauss-Algorithmus kann man gleichzeitig mehrere Gleichungssysteme
l"osen, die sich nur in den rechten Seiten unterscheiden. Die rechten
Seiten haben n"amlich auf den Gang der Rechnung keinen
Einfluss. So kann man die beiden Gleichungssysteme
\begin{align*}
\frac12x+\frac{\sqrt{3}}2y&=1
&
\frac12x+\frac{\sqrt{3}}2y&=0
\\
-\frac{\sqrt{3}}2x+\frac12y&=0
&
-\frac{\sqrt{3}}2x+\frac12y&=1
\end{align*}
in einem Durchgang l"osen:
\begin{align*}
\begin{tabular}{|>{$}c<{$}>{$}c<{$}|>{$}c<{$}>{$}c<{$}|}
\hline
\frac12%
\begin{picture}(0,0)
\color{red}\put(-3.5,3){\circle{15}}
\end{picture}%
&\frac{\sqrt{3}}2&1&0\\
-\frac{\sqrt{3}}2&\frac12&0&1\\
\hline
\end{tabular}
&\rightarrow
\begin{tabular}{|>{$}c<{$}>{$}c<{$}|>{$}c<{$}>{$}c<{$}|}
\hline
1&\sqrt{3}&2&0\\
-\frac{\sqrt{3}}2%
\begin{picture}(0,0)
\color{blue}\drawline(-22,-4)(-22,13)(1,13)(1,-4)
\end{picture}%
&\frac12&0&1\\
\hline
\end{tabular}
\rightarrow
\begin{tabular}{|>{$}c<{$}>{$}c<{$}|>{$}c<{$}>{$}c<{$}|}
\hline
1&\sqrt{3}&2&0\\
0&2%
\begin{picture}(0,0)
\color{red}\put(-3,4){\circle{12}}
\end{picture}%
&\sqrt{3}&1\\
\hline
\end{tabular}
\\
&\rightarrow
\begin{tabular}{|>{$}c<{$}>{$}c<{$}|>{$}c<{$}>{$}c<{$}|}
\hline
1&\sqrt{3}%
\begin{picture}(0,0)
\color{blue}\drawline(-16,11)(-16,-3)(1,-3)(1,11)
\end{picture}%
&2&0\\
0&1&\frac{\sqrt{3}}2&\frac12\\
\hline
\end{tabular}
\rightarrow
\begin{tabular}{|>{$}c<{$}>{$}c<{$}|>{$}c<{$}>{$}c<{$}|}
\hline
1&0&\frac12&-\frac{\sqrt{3}}2\\
0&1&\frac{\sqrt{3}}2&\frac12\\
\hline
\end{tabular}
\end{align*}
Das erste Gleichungssystem hat also die L"osung
$(x,y)=(\frac12,\frac{\sqrt{3}}2)$,
das zweite
$(x,y)=(-\frac{\sqrt{3}}2, \frac12)$.

Die spezielle Wahl der rechten Seiten erlaubt auch, aus den
beiden gefundenen L"osungen die L"osung f"ur jede beliebige rechte
Seite zusammenzusetzen. Wenn n"amlich auf der rechten
Seite die Zahlen $b_1$ und $b_2$ stehen sollen, dann
bedeutet dies offenbar, dass wir $b_1$ mal die erste L"osung
und $b_2$ mal die zweite L"osung zusammennehmen m"ussen, um
die verlangten rechten Seiten zu finden. Also
$$x=\frac12b_1-\frac{\sqrt{3}}2b_2,\qquad y=\frac{\sqrt{3}}2b_1+\frac12b_2.$$
Durch Einsetzen in die urspr"ungliche Gleichung kann man die
L"osung "uberpr"ufen.

Ein interessanter Spezialfall liegt vor, wenn man ein Gleichungssystem
f"ur die rechten Seiten 
\[
e_1=\begin{pmatrix}1\\0\\0\end{pmatrix},\qquad
e_2=\begin{pmatrix}0\\1\\0\end{pmatrix},\qquad
e_3=\begin{pmatrix}0\\0\\1\end{pmatrix}
\]
bereits gel"ost und die L"osungsvektoren
\[
c_1=\begin{pmatrix} c_{11}\\ c_{21}\\ c_{31} \end{pmatrix},\qquad
c_2=\begin{pmatrix} c_{12}\\ c_{22}\\ c_{32} \end{pmatrix},\qquad
c_3=\begin{pmatrix} c_{13}\\ c_{23}\\ c_{33} \end{pmatrix}
\]
gefunden hat.
Da die allgemeine rechte Seite
\[
b=\begin{pmatrix}
b_1\\b_2\\b_3
\end{pmatrix}
=
b_1 \begin{pmatrix}1\\0\\0\end{pmatrix}
+
b_2 \begin{pmatrix}0\\1\\0\end{pmatrix}
+
b_3 \begin{pmatrix}0\\0\\1\end{pmatrix}
=
b_1 e_1 + b_2 e_2 + b_3 e_3
\]
geschrieben werden kann, kann auch die L"osung aus den Teill"osungen
zusammengesetzt werden:
\[
x = b_1 c_1 + b_2 c_2 + b_3 c_3.
\]
Man kann also ein allgemeines L"osungsverfahren auch wie folgt konzipieren.
Zun"achst bildet man das Gauss-Tableau
\begin{center}
\begin{tabular}{|>{$}c<{$}>{$}c<{$}>{$}c<{$}|>{$}c<{$}>{$}c<{$}>{$}c<{$}|}
\hline
a_{11}&a_{12}&a_{13}&1&0&0\\
a_{21}&a_{22}&a_{23}&0&1&0\\
a_{31}&a_{32}&a_{33}&0&0&1\\
\hline
\end{tabular}
\end{center}
welches man anschliessend mit dem Gauss-Algorithmus in das Tableau
\begin{center}
\begin{tabular}{|>{$}c<{$}>{$}c<{$}>{$}c<{$}|>{$}c<{$}>{$}c<{$}>{$}c<{$}|}
\hline
1&0&0&c_{11}&c_{12}&c_{13}\\
0&1&0&c_{21}&c_{22}&c_{23}\\
0&0&1&c_{31}&c_{32}&c_{33}\\
\hline
\end{tabular}
\end{center}
umwandelt.
Dann baut man die L"osung aus den Zahlen $c_{ij}$ und der rechten
Seite $b$ zusammen. Die Berechnung der $c_{ij}$ muss nur einmal
erfolgen, danach kann die L"osung mit verschiedenen rechten Seiten
$b$ bestimmt werden.

\section{Bezeichnungen: Matrizen und Vektoren}
Viele der Operationen im L"osungsverfahren f"ur lineare Gleichungssysteme,
sind unabh"angig von ihrem Bezug zu einem Gleichungssystem sinnvoll.
In der Tabellendarstellung haben wir auch einen Notation verwendet,
in der die Unbekannten "uberhaupt nicht mehr vorkommen.
Dies kann jedoch nicht nur ein praktische Abk"urzung sein, es muss mehr
dahinter stecken.
In diesem Abschnitt definieren wir die Vektoren und Matrizen, die zu
einem Gleichungssystem geh"oren.

\subsection{Vektoren}
\subsubsection{Definition von Zeilen- und Spaltenvektoren}
Die Operationen im L"osungsverfahren erfolgten jeweils Zeilen- oder Spaltenweise,
es ist daher sinnvoll, die Zeilen oder Spalten als eigene vollwertige 
mathematische Objekte zu definieren.

\begin{definition}
Das Zahlenschema
$$
\begin{pmatrix}
a_1&a_2&\dots&a_n
\end{pmatrix}
$$
mit $a_i\in\mathbb R$ heisst ein $n$-dimensionaler Zeilenvektor.
Das Zahlenschema
$$
\begin{pmatrix}
b_1\\b_2\\\vdots\\b_m
\end{pmatrix}
$$
heisst $m$-dimensionaler Spaltenvektor.
\end{definition}
Die Linearformen entsprechen also Zeilenvektoren, die rechte
Seite des Gleichungssystems ist ein Spaltenvektor.

Zeilen oder Spaltenvektoren k"urzen wir ab als ein einzelnes
Zeichen, zum Beispiel
$$a=\begin{pmatrix}a_1&a_2&\dots&a_n\end{pmatrix},\qquad b=\begin{pmatrix}1\\2\\3\end{pmatrix}.$$

\subsubsection{Rechenoperationen mit Vektoren}
F"ur Zeilen- oder Spaltenvektoren k"onnen wir die Addition von
Vektoren und die Multiplikation mit reellen Zahlen definieren.
Damit das "uberhaupt gehen kann, m"ussen die Vektoren nat"urlich
die gleiche L"ange (Dimension) haben.
Die Operationen erfolgen elementweise:

\begin{definition}
Seien $a$ und $a'$ zwei $n$-dimensionale Zeilenvektoren,
\begin{align*}
a&=\begin{pmatrix}a_1&a_2&\dots&a_n\end{pmatrix}
\\
a'&=\begin{pmatrix}a'_1&a'_2&\dots&a'_n\end{pmatrix}
\end{align*}
und $\lambda\in\mathbb R$, dann sind die Summe $a+a'$ und das
Produkt $\lambda a$ wie folgt definiert:
\begin{align*}
a+a'&=\begin{pmatrix}a_1+a'_1&a_2+a'_2&\dots&a_n+a'_n\end{pmatrix}
\\
\lambda a&=\begin{pmatrix}\lambda a_1&\lambda a_2&\dots&\lambda a_n\end{pmatrix}
\end{align*}
Sind $b$ und $b'$ zwei $m$-dimensionale Spaltenvektoren
\begin{align*}
b&=\begin{pmatrix}b_1\\b_2\\\vdots\\b_m\end{pmatrix},
&
b'&=\begin{pmatrix}b'_1\\b'_2\\\vdots\\b'_m\end{pmatrix}
\end{align*}
dann, sind die Summe $b+b'$ und das Produkt $\lambda b$ wie folgt 
definiert:
\begin{align*}
b+b'&=\begin{pmatrix}
b_1+b'_1\\
b_2+b'_2\\
\vdots\\
b_m+b'_m\\
\end{pmatrix},
&
\lambda b&=\begin{pmatrix}
\lambda b'_1\\
\lambda b'_2\\
\vdots\\
\lambda b'_m
\end{pmatrix}
\end{align*}
\end{definition}
Die Operationen {\bf E} und {\bf I} k"onnen wir jetzt mit Vektoren
beschreiben. Die Zeile mit der Nummer $i$ ist offenbar ein
$n+1$-dimensionaler Zeilenvektoren $z_i$
$$z_i=\begin{pmatrix}a_{i1}&a_{i2}&\dots&a_{im}&b_i\end{pmatrix}.$$
Die Operation {\bf I} f"ur die $i$-te Zeile bedeutet jetzt, dass man
den Zeilenvektor $z_i$ ersetzt durch den neuen Zeilenvektor
$$\frac1{a_{ii}}z_i
=
\frac1{a_{ii}}
\begin{pmatrix}a_{i1}&a_{i2}&\dots&a_{im}&b_i\end{pmatrix}
=
\begin{pmatrix}\frac{a_{i1}}{a_{ii}}&\frac{a_{i2}}{a_{ii}}&\dots&\frac{a_{im}}{a_{ii}}&\frac{b_i}{a_{ii}}\end{pmatrix}
$$
Bei der Operation {\bf E} wird der Zeilevektor $z_j$
verringert um das $a_{ji}$-fache des Zeilenvektors $z_i$, also
$$z_j \leftarrow z_j-a_{ji}z_i.$$

\subsubsection{Gleichungssystem in Vektorschreibweise}
Auch das Gleichungssystem selbst k"onnen wir mit Hilfe von Spaltenvektoren
schreiben. Betrachten wir die Spalten mit der Nummer $k$ als Spaltenvektor $a_k$,
also
$$a_k=\begin{pmatrix}a_{1k}\\a_{2k}\\\dots\\a_{mk}\end{pmatrix},$$
und die rechten Seiten des Gleichungssystems wie fr"uher als Spaltenvektor $b$,
dann k"onnen wir das Gleichungssystem schreiben als
$$
x_1\begin{pmatrix}a_{11}\\a_{21}\\\vdots\\a_{m1}\end{pmatrix}
+
x_2\begin{pmatrix}a_{12}\\a_{22}\\\vdots\\a_{m2}\end{pmatrix}
+
\dots
+
x_n\begin{pmatrix}a_{1n}\\a_{2n}\\\vdots\\a_{mn}\end{pmatrix}
=
\begin{pmatrix}b_1\\b_2\\\vdots\\b_m\end{pmatrix}
$$
oder kurz
$$
a_1x_1+a_2x_2+\dots+a_nx_n=b.
$$

\subsubsection{Spezielle Vektoren\label{speziellevektoren}}
Die folgenden speziellen Vektoren sind oft n"utzlich. Der Nullvektor besteht
aus lauter Nullen, wir schreiben daf"ur
$$0=\begin{pmatrix}0\\\vdots\\0\end{pmatrix}.$$
Im Verfahren f"ur die simultane L"osung von Gleichungssystemen haben
wir spezielle rechte Seiten verwendet, welche bis auf eine Stelle aus lauter
Nullen bestehen. Diese Vektoren bezeichnen wir in Zukunft mit
$$e_1=\begin{pmatrix}1\\0\\\vdots\\0\end{pmatrix},
\qquad
e_2=\begin{pmatrix}0\\1\\\vdots\\0\end{pmatrix},
\qquad
e_n=\begin{pmatrix}0\\0\\\vdots\\1\end{pmatrix}.
$$

\subsection{Matrizen}
\subsubsection{Definition einer Matrix}
Die Formulierung mit Zeilen- und Spaltenvektoren ist immer noch etwas umst"andlich,
wir w"urden erwarten, dass auch das ganze Koeffizientenschema $(a_{ij})$
eine eigenst"andiges mathematisches Objekt ist, immerhin bestimmt es den
Gang des L"osungsverfahrens vollst"andig. 

\begin{definition}
Das rechteckige Zahlenschema
$$
A=
\begin{pmatrix}
a_{11}&a_{12}&\dots&a_{1n}\\
a_{21}&a_{22}&\dots&a_{2n}\\
\vdots&\vdots&\ddots&\vdots\\
a_{m1}&a_{m2}&\dots&a_{mn}\\
\end{pmatrix}
$$
heisst $m\times n$-Matrix.
\end{definition}

\subsubsection{Produkt einer Matrix mit einem Vektor}
Um das Gleichungssystem mit der Koeffizientenmatrix $A$ zu beschreiben
fehlt uns jetzt noch eine Verkn"ufpung zwischen Matrizen und
Vektoren. Die folgende Definition tut, was wir uns w"unschen.
\begin{definition}
Ist $A$ eine $m\times n$-Matrix und $x$ ein $n$-dimensionaler Spaltenvektor,
dann definieren wir das Produkt $Ax$ also den $m$-dimensionalen
Spaltenvektor
$$
\begin{pmatrix}
a_{11}x_1+a_{12}x_2+\dots+a_{1n}x_n\\
a_{21}x_1+a_{22}x_2+\dots+a_{2n}x_n\\
\vdots\\
a_{m1}x_1+a_{m2}x_2+\dots+a_{mn}x_n\\
\end{pmatrix}
$$
Das Element $b_i$ in der $i$-ten Zeile dieses Spaltenvektors ist
$$b_i=\sum_{j=1}^na_{ij}x_j.$$
\end{definition}
Die Multiplikation $Ax$ erfolgt also, indem die Zeilen von $A$
elementweise mit der Spalte $x$ multipliziert, und dann alle Produkte
aufaddiert werden (``Zeile $\strut \times\mathstrut$ Spalte'').

\begin{beispiel} Sei
\[
A=\begin{pmatrix}
6&1&2\\
5&4&0\\
4&1&9
\end{pmatrix}
,\qquad
v=
\begin{pmatrix}
1\\8\\4
\end{pmatrix},
\]
wir suchen das Produkt $Av$:
\begin{align*}
Av&=
\begin{pmatrix}
\color{red}6&\color{red}1&\color{red}2\\
5&4&0\\
4&1&9
\end{pmatrix}
\begin{pmatrix}
\color{blue}1\\\color{blue}8\\\color{blue}4
\end{pmatrix}
=
\begin{pmatrix}
{\color{red}6}\cdot{\color{blue}1}+
{\color{red}1}\cdot{\color{blue}8}+
{\color{red}2}\cdot{\color{blue}4}
\\
5\cdot{\color{blue}1}+
4\cdot{\color{blue}8}+
0\cdot{\color{blue}4}
\\
4\cdot{\color{blue}1}+
1\cdot{\color{blue}8}+
9\cdot{\color{blue}4}
\end{pmatrix}
=\begin{pmatrix}
22\\37\\48
\end{pmatrix}.
\end{align*}
\end{beispiel}

Mit diesen Definitionen ist das Gleichungssystem mit Koeffizientenmatrix $A$,
dem L"osungs-Spaltenvektor $x$ und der rechten Seite $b$, ebenfalls einem
Spaltenvektor, zu
$$Ax=b$$
geworden. Diese Schreibweise suggeriert, dass wir einfach auf beiden
Seiten ``durch $A$ teilen'' k"onnten, um die L"osung zu finden. Auf dieses
Ziel werden wir hinarbeiten, und eine Multiplikation und ``Division''
von Matrizen definieren, so dass wir tats"achlich die L"osung des Gleichungssystems
mit Hilfe einer Formel
$$
x=A^{-1}b
$$
finden k"onnen.

\subsubsection{Rechenregeln f"ur das Produkt Matrix $\times$ Vektor}
F"ur das Produkt gelten die Regeln, die man sich von der Algebra 
gewohnt ist:
\begin{equation}
\begin{aligned}
A(u+v)&=Au+Av&\qquad&(A+B)v=Av+Bv\\
A(\lambda v)&=\lambda Av
\end{aligned}
\label{linearitaet-matrixvektor}
\end{equation}
Die Gleichungen (\ref{linearitaet-matrixvektor}) sehen aus wie
(\ref{linearitaet-linearformen}),
man sagt, $A$ sei eine {\em lineare} Abbildung.

\subsubsection{Produkt zweier Matrizen}
Da Matrizen als ``dicke'' Vektoren aufgefasst werden k"onnen, lassen
sich jetzt auch Matrizen mit anderen Matrizen multiplizieren.
Die Multiplikation erfolgt wie bei Vektoren als
$\text{Zeilen}\times\text{Spalten}$, das Element in Zeile $i$
und Spalte $j$ der Produktmatrix entsteht aus Zeile $i$ des ersten
Faktors und Spalte $j$ des zweiten Faktors.
Die Zeilenl"ange der ersten Matrix muss zur Zeilenl"ange
der zweiten Matrix passen:
\begin{align*}
\left(\begin{tabular}{|c|c|c|c|c|}
\hline
&&&&\\
\hline
&&&&\\
\hline
&&&&\\
\hline
\end{tabular}\right)
\cdot
\left(\begin{tabular}{|c|c|}
\hline
&\\
\hline
&\\
\hline
&\\
\hline
&\\
\hline
&\\
\hline
\end{tabular}\right)
&=
\left(\begin{tabular}{|c|c|}
\hline
&\\
\hline
&\\
\hline
&\\
\hline
\end{tabular}\right)
\\
(m\times l)\cdot (l\times n)&=(m\times n)
\end{align*}
\begin{beispiel}
\begin{align*}
A&=\begin{pmatrix}
\color{red}1&\color{red}2&\color{red}4\\
0&4&-1
\end{pmatrix}
\\
B&=\begin{pmatrix}
\color{blue}-2&4\\
\color{blue}-4&-5\\
\color{blue}2&4
\end{pmatrix}
\\
AB&=
\begin{pmatrix}
{\color{red}1}\cdot({\color{blue}-2})+{\color{red}2}\cdot({\color{blue}-4})+{\color{red}4}\cdot{\color{blue}2} &
{\color{red}1}\cdot 4+{\color{red}2}\cdot(-5)+{\color{red}4}\cdot 4\\
0\cdot ({\color{blue}-2})+4\cdot({\color{blue}-4})+(-1)\cdot {\color{blue}2} &
0\cdot 4+4\cdot(-5)+(-1)\cdot 4
\end{pmatrix}
\\
&=
\begin{pmatrix}
-2&10\\-18&-24
\end{pmatrix}
\\
BA&=\begin{pmatrix}
(-2)\cdot1+4\cdot 0&
(-2)\cdot2+4\cdot 4&
(-2)\cdot4+4\cdot (-1) \\
(-4)\cdot1+(-5)\cdot 0&
(-4)\cdot2+(-5)\cdot 4&
(-4)\cdot4+(-5)\cdot (-1)\\
2\cdot1+ 4\cdot0&
2\cdot2+ 4\cdot4&
2\cdot4+ 4\cdot (-1)
\end{pmatrix}
\\
&=
\begin{pmatrix}
   -2&  12& -12\\
   -4& -28& -11\\
    2&  20&   4
\end{pmatrix}
\end{align*}
\end{beispiel}
Das Beispiel illustriert auch, dass es wesentlich auf die Reihenfolge
der Faktoren ankommt. Dies gilt selbst dann, wenn die beiden
Matrizen quadratisch sind, also $AB$ und $BA$ ebenfalls quadratische
Matrizen sind.
\begin{beispiel}
Dieses Beispiel zeigt, dass das Matrizenprodukt nicht kommutativ ist.
\begin{align*}
\begin{pmatrix} 0&1\\0&0 \end{pmatrix}
\begin{pmatrix} 0&0\\1&0 \end{pmatrix}
&=
\begin{pmatrix} 1&0\\0&0 \end{pmatrix}
\\
\begin{pmatrix} 0&0\\1&0 \end{pmatrix}
\begin{pmatrix} 0&1\\0&0 \end{pmatrix}
&=
\begin{pmatrix} 0&0\\0&1 \end{pmatrix}
\end{align*}
Die beiden Produkte sind offensichtlich verschieden.
\end{beispiel}

\subsubsection{Transponierte Matrix}
Hat eine Matrix linear abh"angige Zeilen, lassen sich die Zeilen
linear zu einer Null-Zeile kombinieren. Die Zeilen der Matrix
$$A=
\begin{pmatrix}
a_{11}&\dots&a_{1n}\\
\vdots&\ddots&\vdots\\
a_{m1}&\dots&a_{mn}
\end{pmatrix}
$$
sind also genau dann linear abh"angig, wenn das Gleichungssystem
$$
\begin{linsys}{4}
a_{11}\lambda_1&+&a_{21}\lambda_2&+&\dots&+&a_{m1}\lambda_m&=&0\\
a_{12}\lambda_1&+&a_{22}\lambda_2&+&\dots&+&a_{m2}\lambda_m&=&0\\
\vdots         & &\vdots         & &\ddots&&\vdots         & &\vdots\\
a_{1n}\lambda_1&+&a_{2n}\lambda_2&+&\dots&+&a_{mn}\lambda_m&=&0\\
\end{linsys}
$$
eine L"osung hat, die nicht aus lauter Nullen besteht. Die Koeffizientenmatrix
dieses Gleichungssystems entsteht dadurch, dass $A$ an der
Diagonalen $a_{11}\dots a_{22}\dots a_{33}\dots$ gespiegelt wird.
\begin{definition} Ist $A$ eine $m\times n$-Matrix mit Eintr"agen
$a_{ij}$ in der $i$-ten Zeile und $j$-ten Spalte, dann heisst die
gespiegelte Matrix
$$A^t=\begin{pmatrix}
a_{11}&a_{21}&\dots&a_{m1}\\
a_{12}&a_{22}&\dots&a_{m2}\\
\vdots&\vdots&\ddots&\vdots\\
a_{1n}&a_{2n}&\dots&a_{mn}
\end{pmatrix}
$$
die transponierte Matrix von $A$. $A^t$ ist eine $n\times m$-Matrix.
Eine Matrix heisst symmetrisch, wenn $A^t=A$.
\index{Matrix!symmetrische}
\end{definition}
Da das Transponieren einer Matrix ihre Zeilen und Spalten vertauscht,
vertauscht
sie auch die Reihenfolge der Faktoren in einem Matrizenprodukt
\begin{align*}
A^tB^t&=\text{Zeilen von $A^t$}\times\text{Spalten von $B^t$}\\
      &=\text{Spalten von $A$}\times\text{Zeilen von $B$}\\
      &=(\text{Zeilen von $B$}\times\text{Spalten von $A$})^t\\
      &=(BA)^t.
\end{align*}

\begin{beispiel}
Gegeben sei die Matrix
\[
A=\begin{pmatrix}
4&3\\
5&4
\end{pmatrix}.
\]
Man berechne $A^tA$ und $AA^t$.

\smallskip

{\parindent 0pt $A^t$ ist}
\[
A^t=\begin{pmatrix}
4&5\\
3&4
\end{pmatrix}.
\]
Das Produkt der beiden Matrizen ist
\begin{align*}
A^tA&=
\begin{pmatrix}
4&5\\
3&4
\end{pmatrix}
\begin{pmatrix}
4&3\\
5&4
\end{pmatrix}
=\begin{pmatrix}
16+25&12+20\\
12+20&9+16
\end{pmatrix}
=
\begin{pmatrix}
41&32\\
32&25
\end{pmatrix},
\\
AA^t&=
\begin{pmatrix}
4&3\\
5&4
\end{pmatrix}
\begin{pmatrix}
4&5\\
3&4
\end{pmatrix}
=
\begin{pmatrix}
16+9&20+12\\
20+12&25+16
\end{pmatrix}
=
\begin{pmatrix}
25&32\\
32&41
\end{pmatrix}.
\end{align*}
Man beachte, dass auch hier wieder $A^tA\ne AA^t$.
\end{beispiel}

\subsubsection{Rechenregeln f"ur Matrizen}
Wir haben bereits fr"uher die Rechenregeln f"ur das Produkt einer Matrix
mit einem Vektor aufgestellt, und dabei festgestellt, dass die
aus der Algebra vertrauten Regeln erhalten bleiben.
Mit dem
Produkt Matrix $\times$ Matrix kommen noch einige neue Regeln dazu,
die auch vertraut sind. Man kann also mit Matrizen genau so rechnen,
wie man das in der Algebra immer gemacht hat, man muss nur
aufpassen, dass man nie die Reihenfolge der Matrizen vertauscht.
\begin{align*}
    A(u+v)&=Au+Av     &\Rightarrow&&A(B+C)&=AB+AC,    &(A+B)C&=AC+BC\\
A\lambda v&=\lambda Av&\Rightarrow&&A\lambda B&=\lambda AB
\end{align*}

\section{Inverse Matrix}
\index{Matrix!inverse}
Mit den neuen Notationen k"onnen wir jetzt auch das Verfahren zur simultanen
L"osung mehrere Gleichungssysteme aus Abschnitt \ref{simultan} etwas
"ubersichtlicher schreiben. Dazu brauchen wir die spezielle Matrix, die auf
der Diagonalen Einsen hat und sonst nur Nullen. Diese Matrix heisst die
Einheitsmatrix, wir bezeichnen sie mit
$$
I=
\begin{pmatrix}1&\dots&0\\
\vdots&\ddots&\vdots\\
0&\dots&1
\end{pmatrix},
$$
unabh"angig von der Anzahl Zeilen und Spalten, diese ist aus dem Zusammenhang
zu entnehmen.

Das Verfahren begann damit, an die gegebene Koeffizientenmatrix die Matrix $I$
anzuh"angen. Anschliessend wurde der Gaussalgorithmus verwendet, bis in
der linken H"alfte der Tabelle eine Einheitsmatrix stand. Symbolisch k"onnen
wir dies durch
\begin{equation}
\begin{tabular}{|c|c|}\hline
$A$&$I$\\
\hline
\end{tabular}
\rightarrow
\begin{tabular}{|c|c|}\hline
$I$&$C$\\
\hline
\end{tabular}
\label{gaussinverse}
\end{equation}
darstellen. Die Matrix $C$ enth"alt in ihren Spalten die L"osungen des
Gleichungssystems f"ur die speziellen rechten Seiten, die nur aus Nullen
und einer Eins bestehen.

Eine beliebige rechte Seite $b$ l"asst sich jetzt aus den speziellen rechten
Seiten zusammensetzen:
$$b=\begin{pmatrix}b_1\\b_2\\\vdots\\b_n\end{pmatrix}
=
b_1\begin{pmatrix}1\\0\\\vdots\\0\end{pmatrix}
+
b_2\begin{pmatrix}0\\1\\\vdots\\0\end{pmatrix}
+\dots+
b_n\begin{pmatrix}0\\0\\\vdots\\1\end{pmatrix}
$$
wir versuchen daher, auch die L"osung aus den entsprechenden Spalten von $C$
aufzubauen:
\begin{equation}
x
=
b_1\begin{pmatrix}c_{11}\\c_{21}\\\vdots\\c_{n1}\end{pmatrix}
+
b_2\begin{pmatrix}c_{12}\\c_{22}\\\vdots\\c_{n2}\end{pmatrix}
+\dots+
b_n\begin{pmatrix}c_{1n}\\c_{2n}\\\vdots\\c_{nn}\end{pmatrix}.
\label{zusammensetzen}
\end{equation}
Tats"achlich ergibt sich, wenn wir die Matrix $A$ auf diesen Vektor
anwenden
$$
Ax
=
b_1A\begin{pmatrix}c_{11}\\c_{21}\\\vdots\\c_{n1}\end{pmatrix}
+
b_2A\begin{pmatrix}c_{12}\\c_{22}\\\vdots\\c_{n2}\end{pmatrix}
+\dots+
b_nA\begin{pmatrix}c_{1n}\\c_{2n}\\\vdots\\c_{nn}\end{pmatrix}
=
b_1\begin{pmatrix}1\\0\\\vdots\\0\end{pmatrix}
+
b_2\begin{pmatrix}0\\1\\\vdots\\0\end{pmatrix}
+\dots+
b_n\begin{pmatrix}0\\0\\\vdots\\1\end{pmatrix}=b,
$$
dieses $x$ ist also tats"achlich die gesuchte L"osung.

Das Element auf Zeile $i$ in 
\ref{zusammensetzen} ist
$$
x_i=c_{i1}b_1+c_{i2}b_2+\dots+c_{in}b_n,
$$
dies ist aber nichts anderes als die $i$-te Komponente des Produktes
$Cb$, die L"osung $x$ kann als mit Hilfe eines Matrizenproduktes gefunden
werden:
$$ x = Cb.$$
Die Matrix $C$ hat also offenbar eine ganz besondere Bedeutung f"ur die
L"osung linearer Gleichungssysteme, ist sie einmal bestimmt, kann man das
Gleichungssystem f"ur jede beliebige rechte Seite l"osen. Daher definieren
wir
\begin{definition}Die Matrix $C$ aus \ref{gaussinverse} heisst inverse
Matrix von $A$, geschrieben $A^{-1}=C$, sie existiert immer, wenn $A$ regul"ar ist.
\end{definition}
Den Algorithmus zur Bestimmung der inversen Matrix kann also symbolisch
als
\begin{equation}
\begin{tabular}{|>{$}c<{$}|>{$}c<{$}|}
\hline
A&I\\
\hline
\end{tabular}
\rightarrow
\begin{tabular}{|>{$}c<{$}|>{$}c<{$}|}\hline
I&A^{-1}\\
\hline
\end{tabular}
\label{gaussinverse2}
\end{equation}
notiert werden.
\begin{satz}Die L"osung eines Gleichungssystems
\[
Ax=b
\]
mit
regul"arer Koeffizientenmatrix $A$ ist
\[
x=A^{-1}b.
\]
\end{satz}

\begin{beispiel}
Man finde die L"osung des Gleichungssystems
\[
Ax=
\begin{pmatrix}
  -4& -4& -1\\
   5&  3&  1\\
   4& -5&  0
\end{pmatrix}
\begin{pmatrix}x_1\\x_2\\x_3\end{pmatrix}
=
\begin{pmatrix}2\\1\\3\end{pmatrix}.
\]
mit Hilfe der Inversen Matrix.

\smallskip
{\parindent 0pt Man muss die Inverse der Matrix $A$ bestimmen,
der Gauss-Algorithmus liefert}
\begin{align*}
\begin{tabular}{|>{$}c<{$}>{$}c<{$}>{$}c<{$}|>{$}c<{$}>{$}c<{$}>{$}c<{$}|}
\hline
  -4& -4& -1 & 1& 0& 0\\
   5&  3&  1 & 0& 1& 0\\
   4& -5&  0 & 0& 0& 1\\
\hline
\end{tabular}
&\rightarrow
\begin{tabular}{|>{$}c<{$}>{$}c<{$}>{$}c<{$}|>{$}c<{$}>{$}c<{$}>{$}c<{$}|}
\hline
   1&  1& \frac14 &-\frac14& 0& 0\\
   0& -2&-\frac14 & \frac54& 1& 0\\
   0& -9& -1      & 1      & 0& 1\\
\hline
\end{tabular}
\\
&\rightarrow
\begin{tabular}{|>{$}c<{$}>{$}c<{$}>{$}c<{$}|>{$}c<{$}>{$}c<{$}>{$}c<{$}|}
\hline
   1&  1& \frac14 &-\frac14   &       0& 0\\
   0&  1& \frac18 &-\frac58   &-\frac12& 0\\
   0&  0& \frac18 &-\frac{37}8&-\frac92& 1\\
\hline
\end{tabular}
\\
&\rightarrow
\begin{tabular}{|>{$}c<{$}>{$}c<{$}>{$}c<{$}|>{$}c<{$}>{$}c<{$}>{$}c<{$}|}
\hline
   1&  1&       0 &       9   &       9&-2\\
   0&  1&       0 &       4   &       4&-1\\
   0&  0&       1 &-37        &     -36& 8\\
\hline
\end{tabular}
\\
&\rightarrow
\begin{tabular}{|>{$}c<{$}>{$}c<{$}>{$}c<{$}|>{$}c<{$}>{$}c<{$}>{$}c<{$}|}
\hline
   1&  0&       0 &       5   &       5&-1\\
   0&  1&       0 &       4   &       4&-1\\
   0&  0&       1 &-37        &     -36& 8\\
\hline
\end{tabular}
\end{align*}
Die inverse Matrix ist also
\[
A^{-1}=
\begin{pmatrix}
5&5&-1\\
4&4&-1\\
-37&-36&8
\end{pmatrix}.
\]
Zur Kontrolle kann man $A^{-1}A$ ausrechnen, man sollte die Einheitsmatrix
erhalten:
\begin{align*}
A^{-1}A&=
\begin{pmatrix}
5&5&-1\\
4&4&-1\\
-37&-36&8
\end{pmatrix}
\begin{pmatrix}
  -4& -4& -1\\
   5&  3&  1\\
   4& -5&  0
\end{pmatrix}
\\
&
=
\begin{pmatrix}
-20+25-4 & -20+15+5 & -5+5+0\\
-16+20-4 & -16+12+5 & -4+4+0\\
148-180+32 & 148-108-40&37-36+0
\end{pmatrix}
=\begin{pmatrix}
1&0&0\\0&1&0\\0&0&1\end{pmatrix}.
\end{align*}
Mit der inversen Matrix kann man jetzt auch die L"osung
\[
\begin{pmatrix}x_1\\x_2\\x_3\end{pmatrix}
=A^{-1}
\begin{pmatrix}2\\1\\3\end{pmatrix}
=
\begin{pmatrix}
5&5&-1\\
4&4&-1\\
-37&-36&8
\end{pmatrix}
\begin{pmatrix}2\\1\\3\end{pmatrix}
=\begin{pmatrix}
12\\
9\\
-86
\end{pmatrix}
\]
erhalten.
\end{beispiel}

\section{Homogene und Inhomogene Gleichungssysteme}
\index{homogen}
\index{inhomogen}
Bei der Untersuchung der linearen Abh"angigkeit haben wir aus einem
Gleichungssystem mit beliebigen rechten Seiten ein neues Gleichungssystem
mit ausschliesslich Nullen auf der rechten Seite abgeleitet. Es gibt
also einen Zusammenhang zwischen der L"osbarkeit des Systems 
$Ax=b$ mit $b\ne 0$ und $Ax=0$, die wir jetzt genauer untersuchen
wollen.
\begin{definition}
Sei $A$ eine $m\times n$-Matrix, und $b\ne 0$ ein $m$-dimensionaler
Spaltenvektor. Das Gleichungssystem mit nicht verschwindender
rechten Seite
$$Ax=b\qquad\text{heisst inhomogen,}$$
das Gleichungssystem mit Nullen auf der rechten Seite 
$$Ax=0\qquad\text{heisst homogen.}$$
\end{definition}
\subsection{Regul"are Koeffizientenmatrix}
Ist $A$ regul"ar, dann hat das Gleichungssystem $Ax=b$ f"ur jede beliebige
rechte Seite genau eine L"osung, insbesondere auch die Gleichung $Ax=0$.
Die L"osung kann mit der inversen Matrix berechnet werden, also $x=A^{-1}b$,
im Falle des homogenen Gleichungssystems ist die einzige L"osung also
die Nulll"osung $x=0$.

\subsection{Singul"are Koeffizientenmatrix}
Ist $A$ singular, sind zwei Alternativen m"oglich: entweder hat das
Gleichungssystem gar keine L"osung, oder es hat unendlich viele.
\begin{align*}
Ax&=b&\qquad&\begin{cases}
\text{keine L"osung}\\
\text{unendlich viele L"osungen}
\end{cases}
\\
Ax&=0&\qquad&\begin{cases}
\text{keine L"osung}&\text{$\Rightarrow$ kann wegen $A0=0$ nicht eintreten!}\\
\text{unendlich viele L"osungen}&
\end{cases}
\end{align*}
Ein homogenes Gleichungssystem mit singul"arer Koeffizientenmatrix hat also
immer unendlich viele L"osungen. Wir bezeichnen die L"osungsmenge des
Systems $Ax=0$ mit $\mathbb L_h$:
$$
\mathbb L_h=\{x|Ax=0\}.
$$

Ist jetzt $x_p$ eine beliebige L"osung des inhomogenen Systems $Ax=b$, dann
k"onnen wir weiter L"osungen finden, indem wir L"osungen $x_h\in\mathbb L_h$
dazuaddieren:
$$A(x_p+x_h)=Ax_p+Ax_h=b+0=b,$$
also ist $x_p+x_h$ auch wieder eine L"osung. Umgekehrt ist die Differenz
zwischen zwei L"osungen $x_p$ und $x_p'$ der inhomogenen Gleichung in
$\mathbb L_h$:
$$A(x_p-x_p')=Ax_p-Ax_p'=b-b=0\quad\Rightarrow\quad (x_p-x_p')\in\mathbb L_h.$$
Damit haben wir folgendes Rezept zur Bestimmung der L"osungsmenge des
inhomogenen Gleichungssystems gefunden
\begin{satz}
\index{partikulaer@partikul\"ar}
Ist $x_p$ eine spezielle L"osung des inhomogenen Gleichungssystems $Ax=b$,
auch partikul"are L"osung genannt,
dann ist die L"osungsmenge dieses Gleichungssystems 
$$\mathbb L=\{x_p+x_h|x_h\in\mathbb L_h\},$$
wobei $\mathbb L_h$ die L"osungsmenge des homogenen Gleichungssystems ist.
\end{satz}
Der Satz leistet eine Aufteilung des Problems, alle L"osungen zu finden,
in die beiden Teilprobleme
\begin{enumerate}
\item Finden einer einzigen L"osung des inhomogenen Systems.
\item Finden der L"osungsmenge des homogenen Systems.
\end{enumerate}
Dieses Muster trifft man an vielen Orten in der Mathematik wieder, zum
Beispiel bei der L"osung gew"ohnlicher Differentialgleichungen.
Oft lassen sich die beiden Teilprobleme mit spezialisierten Techniken
viel einfacher l"osen.

Das Problem, die L"osungsmenge des homogenen Systems zu bestimmen wird
dadurch vereinfacht, dass man in dieser L"osungsmenge rechnen kann. 
Sind $x$ und $x'$  in $\mathbb L_h$, dann sind auch deren Summe und
die Vielfachen in $\mathbb L_h$:
\begin{align*}
A(x+x')&=Ax+Ax'=0+0=0&\Rightarrow&&x+x'&\in\mathbb L_h\\
A(\lambda x)&=\lambda Ax=\lambda 0=0&\Rightarrow&&\lambda x&\in \mathbb L_h
\end{align*}
Hat man also erst mal ein paar Vektoren in $\mathbb L_h$ gefunden, kann
man unendlich viele weitere konstruieren, indem man alle Vielfachen und Summen
bildet.

\section{Rechenaufwand}
\index{Rechenaufwand}
Wir wollen den Rechenaufwand f"ur die Durchf"uhrung des Gauss-Verfahrens
bestimmen. Diese Information wird zum Beispiel ben"otigt, wenn man mit
einem Microcontroller einen Kalman-Filter implementieren m"ochte, dort
muss auch eine Matrix invertiert werden.
\index{AVR}
Ein typischer 8bit-Microcontroller (AVR)
ben"otigt ca 200$\mu$s f"ur eine Multiplikation, und etwa 80$\mu$s f"ur
eine Addition\footnote{Zeitangaben aus der Microcontroller Performance
Comparison auf \url {http://www.freertos.org.}}.
\index{ARM}
Ein 32bit-$\mu$C wie der ARM (LPC2106) ist wesentlich schneller,
er schafft diese Operationen in etwa 10$\mu$s. Je nach Rechenaufwand kann
es also n"otig sein, von einen schnelleren Prozessor zu verwenden. 

F"ur den $k$-ten Schritt beim Vorw"artsreduzieren muss die $k$-te
Zeile mit einer Zahl multipliziert werden ($n$ Operationen) und dann
von $n-k$ Zeilen subtrahiert werden ($(n-k)n$ Operationen). Es sind also
$$
\sum_{k=1}^n n(n-k+1) = n^3 - n \sum_{k=1}^n k + n^2
=n^3+n^2-n\frac{n(n+1)}2=\frac{n^2(n+1)}2 =O(n^3)
$$
Operationen n"otig. Bei $n=12$ Unbekannten braucht man also bereits
$936$ Operationen, mehr als ungef"ahr sieben mal pro Sekunde kann man das mit
einem 8bit AVR also nicht machen. Mit einem ARM7 dagegen k"onnte
man dies "uber 100 Mal pro Sekunde durchf"uhren.

