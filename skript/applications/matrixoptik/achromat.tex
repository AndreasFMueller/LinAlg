%
% achromat.tex
%
% (c) 2019 Prof Dr Andreas Müller, Hochschule Rapperswil
%
\subsection{Achromat\label{mo:subsection:achromat}}
Die meisten Medien brechen Licht unterschiedlicher Wellenlängen
verschieden stark.
Bei einer einzelnen Linse führt dies dazu, dass die Farben
verschieden grosse Brennweite haben.
Dies führt zu unschönen Farbsäumen und unscharfer Abbildung.

Die Abhängigkeit der Brechkraft von der Wellenlänge heisst {\em Dispersion}.
\index{Dispersioin}
Es gibt zwar auch Gläser mit sehr geringer Dispersion, doch sind diese
leider sehr teuer in der Herstellung und zum Teil auch empfindlich
auf Umwelteinflüsse.
Calzium-Fluorit zum Beispiel hat fast verschwindende Dispersion, darf
aber keinen grossen Temperaturveränderungen ausgesetzt werden.
Es wird daher nur in Spezialanwendungen eingesetzt, zum Beispiel bei
der Masken-Belichtung in der Chip-Herstellung oder für hochwertige 
Astrographen.

Da es Gläser ganz unterschiedlicher Dispersion gibt, besteht die Hoffnung,
durch Kombination geeigneter Gläser in einem mehrlinsigen System zu
erreichen, dass die Farben rot und blau die gleiche Brennweite haben.
Die Farbe grün dazwischen kann dann auch nicht allzu weit weg sein.
Auf diese Art erhält man ein System mit deutlich schwächeren Farbrändern
und grösserer Bildschärfe.

Das einfachste solche System ist der Achromat, erfunden vom  englischen
Amateuroptiker Chester Moor Hall 1733.
Eine Sammellinse aus Kronglas wird mit einer Zerstreuungslinse aus
Flintglas zusammengefügt.
Es entsteht ein System mit drei gekrümmten Flächen, deren Abstand
in begrenztem Rahmen gewählt werden kann.
Es stehen also insgesamt fünf Parameter zur Verfügung, die so gewählt
werden müssen, dass rotes und grünes Licht die gleiche Brennweite
erhalten.

