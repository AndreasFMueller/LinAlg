%
% kamera.tex
%
% (c) 2018 Prof Dr Andreas Müller, Hochschule Rapperswil
%
\section{Anwendung: Kamera-Geometrie\label{section:kamera}}
Die rasante Entwicklung der Mikroelektronik und vor allem der
allgegenwärtigen Kameras in Mobiltelefonen hat dazu geführt, dass 
Bildsensoren heute klein, billig und mit wenig Software-Aufwand
einsetzbar sind.
Zwar muss für die Verarbeitung der ein etwas höherer Aufwand geleistet
werden, doch die ebenfalls gestiegene Prozessorleistung ermöglicht
Bildanalysen, die vor wenigen Jahren noch undenkbar waren.
Standard-Schnittstellen und Bibliotheken für die Bildverarbeitung
haben Kameras sind zu Allzweck-Sensoren geworden.
Mit der gleichen, massenproduzierbaren Hardware können mit
verschiedener Software die unterschiedlichsten Anwendungsprobleme
gelöst werden.

Kameras können zum Beispiel verwendet werden, um Objekte im Raum
zu lokalisieren.
Dazu muss die Software die Objekte im Bild erst finden.
Die klassische Bildverarbeitung hat auf diese Aufgabe spezialisierte
Methoden entwickelt.
Wir können also davon ausgehen, dass wir die Pixel-Koordinaten eines
Objektes aus einem Bild herauslesen können.
Die geometrische Aufgabe, die jetzt gelöst werden muss, ist, die
Raumkoordinaten des Objektes zu bestimmen.
Mit nur einer Kamera ist das natürlich nicht möglich, im besten Fall kann
eine Gerade bestimmt werden, auf der das Objekt zu finden ist.
Mit zusätzlichen Kameras kann dann mittels Triangulation eine mehr
oder weniger genaue Position bestimmt werden.

Das umgekehrte Problem ist etwas einfacher: Gegeben ein Punkt im Raum,
finde die Pixelkoordinaten, auf die der Punkt durch die Kamera
abgebildet wird.
Um diese Aufgabe zu lösen, muss man natürlich einiges über die Kamera
wissen.
Zum Beispiel sollte die Kamera ungefähr auf das Objekt ausgerichtet sein,
wir brauchen also Position und Orientierung der Kamera.
Die Brennweite hat einen Einfluss auf des Gesichtsfeld der Kamera.

In den folgenden Abschnitten wird schrittweise ein Formalismus
aufgebaut, mit dem sich alle diese Probleme mit den Werkzeugen der
linearen Algebra behandeln lassen.

%\subsection{applications/kamera/homogen.tex}
%\subsection{applications/kamera/kmatrix.tex}
%\subsection{applications/kamera/drehung.tex}
%\subsection{applications/kamera/triangulation.tex}




