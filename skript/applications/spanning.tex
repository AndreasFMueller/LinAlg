%
% Berechnung der Zahl der Spanning Trees eines Graphen mit Hilfe
% der Determinante, Anwendung zu Kapitel 2, Determinanten
%
Kirchhoff hat nicht nur gezeigt, wie die Gleichungen für Spannungen
und Ströme in einem Widerstandsnetzwerk aufzustellen sind, dies
haben wir in \ref{appkirchhoff} dargestellt, er hat
auch einiges über die Lösungen herausgefunden.
Ausserdem haben sich seine Ideen als fruchtbar in der Netzwerktheorie
erwiesen, als Beispiel zeigen wir, wie man mit Determinanten die
Zahl der Spannbäume in einem Netzwerk berechnen kann.

\subsection{Lösung der Kirchhoff-Gleichungen}
Gehen wir wieder aus von einem Netzwerk wie in
Abbildung~\ref{netzwerk-numeriert}, wir verwenden die selben 
Bezeichnungen wie in Abschnitt \ref{appkirchhoff}.
Die Gleichungen nach Kirchhoff sind dann
\begin{align*}
Z^tRI&=Z^te \tag{Maschengleichungen}\\
\partial I&=0 \tag{Knotengleichungen}
\end{align*}
Wir wissen bereits, dass ein Netzwerk mit $n$ Kanten und $m$
Knoten $n-m+1$ linear unabhängige Maschengleichungen und
$m-1$ linear unabhängige Knotengleichungen hat, mit denen man
die $n$ Ströme $I$ bestimmen kann.

Kirchhoff hat seine Gleichungen so geschrieben:
\begin{center}
\includegraphics[width=0.8\hsize]{graphics/kh3}
\end{center}
In unserer modernen Schreibweise mit den Notationen von
Abschnitt~\ref{appkirchhoff} sind sie
\[
\begin{linsys}{5}
z_{11}R_1I_1&+&z_{21}R_2I_2&+&\dots &+&z_{n1}R_nI_n&=&z_{11}e_1&+&z_{21}e_2+\dots+z_{n1}e_n=b_1\\
      \vdots& &\vdots      & &\ddots& &\vdots      & &         & & \\
z_{1s}R_1I_1&+&z_{2s}R_2I_2&+&\dots &+&z_{ns}R_nI_n&=&z_{1s}e_1&+&z_{2s}e_2+\dots+z_{ns}e_n=b_s\\
   \partial_{11}I_1&+&   \partial_{12}I_2&+&\dots &+&   \partial_{1n}I_n&=&0&&\\
             \vdots& &   \vdots          & &\ddots& &             \vdots& &\vdots&&\\
\partial_{m-1,1}I_1&+&\partial_{m-1,2}I_2&+&\dots &+&\partial_{m-1,n}I_n&=&0&&\\
\end{linsys}
\]
Die Koeffizienten $z_{ij}$ sind die Matrixelemente der Zyklen-Matrix $Z$.
Kirchhoff versucht dann, die Ströme $I_i$ mit der Cramerschen 
Regel, also mit Hilfe von Determinanten zu bestimmen.
Dazu muss die Determinante der Koeffizienten-Matrix bestimmt werden.
Der Nenner ist
\begin{equation}
N=\left|\;\begin{matrix}
z_{11}R_1       &z_{21}R_2       &\dots &z_{n1}R_n       \\
\vdots          &\vdots          &\ddots&\vdots          \\
z_{1s}R_1       &z_{2s}R_2       &\dots &z_{ns}R_n       \\
\partial_{11}   &\partial_{12}   &\dots &\partial_{1n}   \\
\vdots          &\vdots          &\ddots&\vdots          \\
\partial_{m-1,1}&\partial_{m-1,2}&\dots &\partial_{m-1,n}
\end{matrix}\;\right|
=
\left|\;
\begin{tabular}{>{$}c<{$}>{$}c<{$}>{$}c<{$}}
&          &\\
&Z^tR      &\\
&          &\\
%\hline
%&          &\\
&\partial_-&\\
&          &
\end{tabular}
\;\right|.
\label{Ndenominator}
\end{equation}
Dabei meinen wir mit $\partial_-$ die Matrix, die aus $\partial$
entsteht, indem man die letzte Zeile weglässt.

Sowohl die $z_{ij}$ wie auch die $\partial_{ij}$ sind aus $\{-1,0,+1\}$.
Bei der Entwicklung der Determinante entstehen also jeweils Terme
mit genau $s$ Faktoren $R_i$ und einem Vorzeichen.
Beispielsweise entsteht der Term mit $R_1R_2\dots R_s$ durch
Auswahl von Elementen aus $s\times s$-Block in der linken oberen
Ecke, und verbleibenden Elementen aus dem $(m-1)\times (m-1)$-Block
aus der rechten unteren Ecke, er ist
\[
R_1R_2\dots R_s
\left|\;
\begin{matrix}
z_{11}       &z_{21}       &\dots &z_{s1}       \\
\vdots       &\vdots       &\ddots&\vdots       \\
z_{1s}       &z_{2s}       &\dots &z_{ss}       
\end{matrix}
\;\right|
\cdot
\left|\;
\begin{matrix}
\partial_{1,s+1}   &\partial_{1,s+2}   &\dots &\partial_{1n}   \\
\vdots             &\vdots             &\ddots&\vdots          \\
\partial_{m-1,s+1} &\partial_{m-1,s+2} &\dots &\partial_{m-1,n}
\end{matrix}
\;\right|
\]
Wählt man beliebige Indizes $i_1,i_2,\dots,i_s$, und sind $j_1,\dots,j_{m-1}$
die übrigen Indizes, dann ist der
Term mit $R_{i_1}R_{i_2}\dots R_{i_s}$ entsprechend:
\[
\sigma(i_1,\dots,i_s)
R_{i_1}R_{i_2}\dots R_{i_s}
\left|\;
\begin{matrix}
z_{i_1,1}    &\dots &z_{i_s,1}\\
\vdots       &\ddots&\vdots   \\
z_{i_1,s}    &\dots &z_{i_s,s}       
\end{matrix}
\;\right|
\cdot
\left|\;
\begin{matrix}
\partial_{1,j_1}   &\dots &\partial_{1j_{m-1}}   \\
\vdots             &\ddots&\vdots          \\
\partial_{m-1,j_1} &\dots &\partial_{m-1,j_{m-1}}
\end{matrix}
\;\right|
\]
Darin ist $\sigma(i_1,\dots,i_n)$ ein Vorzeichenfaktor, der
daher rührt, dass man die Spalten $i_1,\dots,i_s$ zunächst
mit Spalten-Vertauschungen in Positionen $1,\dots,s$ bringen
will.

Zur Ermittlung des Stromes $I_l$ muss in dieser Determinante die
Spalte $l$ mit durch die rechte Seite ersetzt werden. Dadurch
entstehen bei der Entwicklung der Determinanten ähnliche Terme,
allerdings kommt genau ein $R$ weniger vor.

Es scheint auf den ersten Blick ziemlich aussichtslos, diese
Determinanten zu berechnen, nur gerade die Struktur von Zähler
und Nenner kann man ermitteln. Schreiben wir $B_l$ für den 
Zähler, der zur Berechnung von $I_l$ benötigt wird, dann ist
\begin{align*}
B_l&=\sum_{i_1,\dots,i_{s-1}}A^l_{i_1,\dots,i_{s-1}}R_{i_1}\dots R_{i_{s-1}}\\
N&=\sum_{i_1,\dots,i_s}a_{i_1,\dots,i_s}R_{i_1}\dots R_{i_s}\\
I_l&=\frac{B_l}{N}
\end{align*}
mit noch zu bestimmenden Koeffizienten $A_{i_1,\dots,i_{s-1}}^l$
und $a_{i_1,\dots,i_s}$.

\subsubsection{Kirchhoffs Idee}
Die Berechnung der Koeffizienten wird möglich dank einer brillanten
Idee Kirchhoffs. Lässt man einzelne Widerstände im Netzwerk gegen
$\infty$ gehen, werden sich die Ströme auf neue Werte $I_{l,\infty}$
einstellen. Anderseits könnte man diese Widerstände auch gleich entfernen.
Man erhielte dann ein neues Netzwerk und neue Koeffizienten
\[
A_{i_1,\dots,i_{s'-1}}^{\prime l}
\quad
\text{und}
\quad
a_{i_1,\dots,i_{s'}}
\]
Diese Koeffizienten gestatten die Ströme $I'_l$ des modifizierten Netzwerks
zu berechnen. Natürlich muss $I_{l,\infty}=I'_l$ gelten.

Wenden wir diese Operation auf die ersten $s-1$ Drähte des Netzwerkes
an. Die Widerstände $R_1,\dots,R_{s-1}$ werden also beliebig gross gemacht.
Sowohl im Zähler wie auch im Nenner von $I_l$ kommen Produkte von diesen
Widerständen vor. Am stärksten wachsen werden aber die Terme, die
alle $s-1$ Widerstände enthalten. Kürzen wir den ganzen Bruch durch
$R_1\dots R_{s-1}$, dann bleibt im Zähler
\[
\frac{B_l}{R_1\dots R_{s-1}}=
A_{1,\dots,s-1}^l+r_l(R_1,\dots,R_n)
\]
wobei in jedem Term des Restes $r_l(R_1,\dots,R_n)$ mindestens einer
der Widerstände $R_1,\dots,R_{s-1}$ im Nenner vorkommt, der Rest
strebt also gegen $0$, wenn man die Widerstände gross macht.
Im Nenner bleiben in ähnlicher Weise die Terme
\[
\frac{N}{R_1\dots R_{s-1}}=
a_{1,\dots,s-1,s}R_s+a_{1,\dots,s-1,s+1}R_{s+1}+\dots+a_{1,\dots,s-1,n}R_n
+q(R_1,\dots,R_n)
\]
stehen, wobei der Rest $q(R_1,\dots,R_n)$ für wachsende Widerstände
gegen $0$ strebt. Folglich sind die Grenzströme im Netzwerk mit den
gegen unendlich strebenden Widerständen
\begin{equation}
I_{l,\infty}=
\frac{A_{1,\dots,s-1}^l}{a_{1,\dots,s-1,s}R_s+a_{1,\dots,s-1,s+1}R_{s+1}+\dots+a_{1,\dots,s-1,n}R_n}.
\label{Iinfinity}
\end{equation}
Wenn $l$ einer der Widerstände ist, den man gegen $\infty$ hat gehen lassen,
dann ist $I_{l,\infty}=0$ und nach $I'_l$ zu fragen hat keinen Sinn, diese
Kante gibt es im modifizierten Netzwerk nicht mehr. Für alle anderen 
Kanten muss $I_{l,\infty}=I_l'$ gelten:
\begin{equation}
I_{l,\infty}=
\begin{cases}
0&\quad l< s\\
I'_l&\quad\text{sonst}.
\end{cases}
\label{Iequations}
\end{equation}
Damit haben wir eine ganze Menge Gleichungen, die erfüllt sein
müssen, und es besteht die Hoffnung, dass sich daraus die Koeffizienten
bestimmen lassen.

\subsubsection{Ein einzelner Zyklus}
Wenn nach der Entfernung von $s-1$ Drähten nur noch ein einzelner
Zyklus übrig bleibt, dann lässt sich die Lösung sofort angeben.
Innerhalb des Zyklus muss der Strom immer gleich sein, und alle
Kanten ausserhalb des Zyklus haben Strom $0$. Sind $l_1,\dots,l_k$
die Kanten des Zyklus, dann ist die einzige verbleibende Kirchhoffsche
Gleichung:
\begin{align}
R_{l_1}I_{l_1}'+R_{l_2}I_{l_2}'+R_{l_k}I_{l_k}'&=e_{l_1}+e_{l_1}+\dots+e_{l_k}
\notag
\\
I_{l_1}'=I_{l_2}'=\dots=I_{l_k}'&=\frac{e_{l_1}+e_{l_1}+\dots+e_{l_k}}{R_{l_1}+R_{l_2}+\dots+R_{l_k}}.
\label{singlecycleprime}
\end{align}
Es sieht also so aus, als hätten wir den Koeffizienten bereits
bestimmt, (\ref{Iinfinity}) sieht doch sehr ähnlich wie
(\ref{singlecycleprime}) aus. Leider reicht das nicht ganz, Der Bruch
$I_{l,\infty}$ könnte gegenüber $I'_l$ auch erweitert worden sein.

Doch mindestens können wir ablesen, dass $a_{1,\dots,s-1,i}=0$ ist,
wenn $i$ nicht zum Zyklus $l_1,\dots,l_k$ gehört. Im Nenner von
$I_{l,\infty}$ können also nur Terme vorkommen, deren letzter Index
zum einzigen verbleibenden Zyklus gehört.
Ausserdem müssen diese Koeffizienten alle gleich sein:
\begin{equation}
a_{1,\dots,s-1,l_1}=a_{1,\dots,s-1,l_2}=\dots= a_{1,\dots,s-1,l_k}
\label{cyclecoefficients}
\end{equation}
\begin{hilfssatz}
\label{gleichekoef-n-1}
Wenn das Entfernen der Kanten $i_1,\dots,i_{s-1}$ genau einen
Zyklus übrig lässt, dann sind alle Koeffizienten
$a_{i_1,\dots,i_{s-1},l}$ gleich, wenn $l$ eine Kante des letzten
verbleibenden Zyklus ist.
\end{hilfssatz}
Damit ergibt sich jetzt der Plan für die weitere Untersuchung:
\begin{compactenum}
\item In dem Fall, dass nach Entfernen der ersten $s-1$ Kanten mehr als $1$
Zyklus übrig bleibt, müssen wir zeigen, dass dann $R_1\dots R_{s-1}$
in Zähler und Nenner gar nicht vorkommt, dass die Koeffizienten
$A_{1,\dots,s-1}^l$ und $a_{1,\dots,s-1,i}$ alle verschwinden.
\item Nicht nur die Koeffizienten zu einem einzelnen Zyklus
wie in (\ref{cyclecoefficients}) sind untereinander gleich, es sind
sogar alle nicht verschwindenden Koeffizienten gleich.
\end{compactenum}

\subsubsection{Mehr als ein Zyklus}
Nehmen wir jetzt an, dass nach dem Entfernen der ersten $s-1$ Kanten
$s'>1$ Zyklen übrig bleiben.
Die Gleichungen (\ref{Iequations}) können auf verschiedene Arten
zustande kommen.
\begin{compactenum}
\item Der Bruch $I_l'$ kann durch einen gemeinsamen Faktor
$R_{j_1}\dots R_{j_{s'-1}}$ gekürzt werden.

In diesem Fall gibt es also einen gemeinsamen Faktor
$R_{j_1}\dots R_{j_{s'-1}}$ in Zähler und Nenner von $I_l'$.
Da nur Widerstände in den Nenner
eingehen, die in einem Zyklus vorkommen, gibt es einen Widerstand
$R_x$ welcher sowohl in $R_{j_1}\dots R_{j_{s'-1}}$ als auch in
einem Zyklus vorkommt.
Kürzt man jetzt $R_{j_1}\dots R_{j_{s'-1}}$, entstehen in Zähler und
Nenner Ausdrücke, in denen $R_x$ nicht mehr vorkommt. 
Insbesondere hängt $I_x'$ nicht mehr von $R_x$ ab, auch für sehr grosses
$R_x$ nicht. Daher muss $I_x'=0$ sein, für beliebige Werte der Widerstände
und elektromotorischen Kräfte.
Das kann aber nicht sein, denn wenn man so viele weitere Widerstände 
beliebig gross macht, dass nur noch eine Zyklus verbleibt, der $R_x$
enthält, dann kann man durch Anpassen der elektromotorischen Kräfte
einen Strom in diesem Zyklus zum fliessen bringen. Dieser Fall kann
also nicht eintreten.
\item Der Zähler in $I_{l,\infty}$ und $I_l'$ verschwindet.
Dieser Fall ermöglicht uns allerdings nicht, weiter Aussagen
über die $a$-Koeffizienten zu machen.
\item Zähler und Nenner in $I_{l,\infty}$ verschwinden.
Da dieser Fall die einzige verbleibende Möglichkeit ist, müssen
also alle Koeffizienten $a_{1,\dots,s,i}$ und $A^l_{1,\dots,s-1}$
verschwinden, wenn durch Entfernung der ersten $s-1$ Kanten mehr als
ein Zyklus verbleibt.
\end{compactenum}
Als Resultat haben wir, dass nur Koeffizienten $a_{i_1,\dots,i_{s-1}}$
und $A_{i_1,\dots,i_{s-1}}^l$ von $0$ verschieden sein können, 
wenn die Entfernung der Kanten $i_1,\dots,i_{s-1}$ genau einen
Zyklus übrig lässt.

\begin{hilfssatz}
Die Koeffizienten $a_{i_1,\dots,i_s}$ sind nur dann von $0$ verschieden,
wenn durch Entfernen der Kanten $i_1,\dots,i_s$ alle Zyklen zerstört
werden.
\end{hilfssatz}

\begin{proof}[Beweis]
Wenn $i_1,\dots,i_s$ alle Zyklen zerstört, dann lässt
$i_1,\dots,i_{s-1}$ genau einen Zyklus übrig, und $i_s$ 
gehört zu diesem verbleibenden Zyklus. Dann ist nach dem
eben gezeigten $a_{i_1,\dots,i_s}\ne 0$.
\end{proof}

\subsubsection{Gleichheit der Koeffizienten}
Bis jetzt wissen wir, dass alle Terme $R_{i_1}\dots R_{i_{s-1}}R_j$
im Nenner von $I_{l,\infty}$ den gleichen Koeffizienten haben,
wenn nach Entfernung der Kanten $i_1,\dots,i_{s-1}$ genau ein
Zyklus verbleibt, und $j$ in diesem Zyklus liegt. Wir wollen jetzt
einsehen, dass überhaupt alle Koeffizienten im Nenner gleich sind.

\begin{hilfssatz}
\label{gleichekoef-induktionsschritt}
Wenn die $a$-Koeffizienten gleich sind,
wenn mindestens $\nu$ Kanten übereinstimmen,
dann sind die Koeffizienten auch gleich,
wenn nur die ersten $\nu-1$ Kanten übereinstimmen.
\end{hilfssatz}

\begin{proof}[Beweis]
Seien
$i_1,\dots,i_\nu,i_{\nu+1},\dots,i_s$
und
$i_1,\dots,i_{\nu},i'_{\nu+1},\dots,i_s'$
Kantenmengen,
durch deren
Entfernung alle Zyklen zerstört werden,
und die in den ersten $\nu$ Kanten übereinstimmen.
Dann wird durch $i_\nu'$ ein Zyklus zerstört, der auch von einem
der Indizes $i_{k}$ mit $k\ge\nu$ zerstört wird. Ohne Einschränkung
der Allgemeinheit können wir annehmen, dass $i_{\nu}$ diese Kante ist.

Dann gibt es ausserdem Kanten $i_{\nu+1}'',\dots,i_s''$, durch deren
Entfernung die verbleibenden $s-\nu$ Zyklen zerstört werden. Die Kantenmengen
\begin{align*}
&i_1,\dots,i_{\nu-1},i_{\nu},i_{\nu+1},\dots,i_s\\
&i_1,\dots,i_{\nu-1},i_{\nu},i_{\nu+1}'',\dots,i_s''\\
&i_1,\dots,i_{\nu-1},i_{\nu}',i_{\nu+1}'',\dots,i_s''\\
&i_1,\dots,i_{\nu-1},i_{\nu}',i_{\nu+1}',\dots,i_s'
\end{align*}
zerstören alle Zyklen, und haben von Zeile zu Zeile mindestens
$\nu$ gemeinsame Kanten (im mittleren Schritt sogar $s-1$
gemeinsame Kanten). Daher sind die Koeffizienten gleich:
\[
a_{i_1,\dots,i_{\nu-1},i_{\nu},i_{\nu+1},\dots,i_s}
=
a_{i_1,\dots,i_{\nu-1},i_{\nu},i_{\nu+1}'',\dots,i_s''}
=
a_{i_1,\dots,i_{\nu-1},i_{\nu}',i_{\nu+1}'',\dots,i_s''}
=
a_{i_1,\dots,i_{\nu-1},i_{\nu}',i_{\nu+1}',\dots,i_s'},
\]
was die Behauptung beweist.
\end{proof}

\begin{satz}
Wenn $i_1,\dots,i_s$ und $j_1,\dots,j_s$ alle Zyklen zerstören,
dann ist 
\[
a_{i_1,\dots,i_s}=a_{j_1,\dots,j_s}.
\]
\end{satz}

\begin{proof}[Beweis]
Nach Hilfssatz~\ref{gleichekoef-n-1} gilt die Behauptung, wenn die Kantenmengen
$n-1$ gemeinsame Kanten haben.
Nach Hilfssatz~\ref{gleichekoef-induktionsschritt} folgt Gleichheit der
Koeffizienten auch, wenn weniger Kanten übereinstimmen.
\end{proof}

\subsubsection{Die Lösung der Kirchhoff-Gleichungen}
Da wir jetzt alle Nenner kennen, können wir auch die Lösung
der Gleichung unmittelbar angeben.
Zunächst wissen wir, dass alle Koeffizienten der Terme
$R_{i_1}\dots R_{i_s}$ alle identisch sind, wenn $i_1,\dots,i_s$
alle Zyklen des Netzwerks zerstört,
\[
N =\sum_{i_1,\dots,i_s}R_{i_1}\dots R_{i_s}.
\]
Die Koeffizienten im Zähler kann man durch Vergleich
mit (\ref{singlecycleprime}) bekommen. Wenn nach Entfernung
von $i_1,\dots,i_{s-1}$ noch ein Zyklus $l_1,\dots,l_k$ bleibt, dann
ist
\[
A_{i_1,\dots,i_{s-1}}^l=\begin{cases}
e_{l_1}+\dots+e_{l_k}&\quad l\in\{l_1,\dots,l_k\}\\
0&\quad \text{sonst.}
\end{cases}
\]
Damit lassen sich die Kirchhoffschen Gleichungen dadurch lösen,
dass man alle Kantenmengen untersucht, die alle Zyklen zerstören.

\subsection{Spannbäume}
\index{Spannbaum}
Ein Spannbaum eines Netzwerks ist eine zyklenfreie Teilmenge der Kanten,
welche alle Knoten miteinander verbindet.
Ein Spannbaum ist also genau das, was übrig bleibt, wenn man aus dem
Netzwerk $s$ Kanten entfernt hat, so dass keine Zyklen mehr übrig bleiben.
Setzt man alle Widerstände des Netzwerkes auf Wert $1$, also
$R_1=\dots=R_n=1$ oder $R=E$,
dann ist der Nenner der Lösung der Kirchhoff-Gleichungen
genau die Zahl der Spannbäume des Netzwerkes. Die Kirchhoff-Gleichungen
könnten also eine Möglichkeit liefern, die Zahl der Spannbäume
zu zählen. 
Leider erinnert uns die Formel (\ref{Ndenominator}) auch daran, dass  
wir den Nenner nur bis auf einen Faktor bestimmt haben. In der
Tat können wir die Zyklen in $Z^t$ mit beliebigen Faktoren
multiplizieren. Um die Zyklen zu zählen, brauchen wir also
ein Verfahren, welches $Z$ gar nicht braucht.

Die Matrix auf der rechten Seite von (\ref{Ndenominator}) gibt uns den
Schlüssel. Multiplizieren wir sie mit der Transponierten bekommen
wir
\begin{equation}
\begin{pmatrix}
Z^t\\
\partial_-
\end{pmatrix}
\begin{pmatrix}
Z&\partial_-^t
\end{pmatrix}
=\begin{pmatrix}
Z^tZ&Z^t\partial_-^t\\
\partial_-Z&\partial_-\partial_-^t
\end{pmatrix}
\end{equation}
Die Blöcke ausserhalb der Diagonalen verschwinden, $\partial_-Z=0$
und $Z^t\partial_-^t=(\partial_-Z)^t=0$, weil die Spalten von $Z$ Zyklen sind.
Übrig bleibt also die Determinante
\begin{equation}
\det
\begin{pmatrix}
Z^tZ&0\\
0&\partial_-\partial_-^t
\end{pmatrix}
=\det(Z^tZ)\det(\partial_-\partial_-^t)
\end{equation}
Der erste Term auf der rechten Seite enthält offenbar alles, was von
der speziellen Wahl der Zyklen abhängt. Nur der zweite Teil ist
unabhängig, von der Wahl der Zyklen. $\partial_-\partial_-^t$ ist derjenige
Teil des Laplace-Operators des Netzwerk, den man durch Weglassen der
letzten Zeile und Spalte bekommt.

Aus Kirchhoffs Untersuchung wissen
wir, dass es nicht darauf ankommt, welche Zeile man in $\partial$ 
weglässt, um $\partial_-$ zu bilden. Alle so gebildeten
Matrizen müssten also die selbe Determinante haben.

\index{Matrix-Baum-Satz}
\begin{satz}[Matrix-Baum-Satz von Kirchhoff] Die Zahl $\tau(G)$
\label{matrixtreetheorem}
der Spannbäume eines
Netzwerks $G$ ist gegeben durch den Wert der Kofaktoren des Laplace-Operators
$\Delta_G$.
Alle Kofaktoren sind gleich.
\end{satz}
Wir teilen den Beweis in mehrere Schritte auf, die für sich genommen
auch interessante Resultate sein können. 

\begin{hilfssatz}
Ist in einer $n\times n$-Matrix $A$ die Summe jeder Zeile und jeder
Spalte $0$, dann sind alle Kofaktoren von $A$ gleich.
\end{hilfssatz}
\begin{proof}[Beweis]
Wir zeigen, dass ein beliebiger Kofaktor
$\operatorname{cof}(A)_{ij}=\operatorname{cof}(A_{nn})$ ist.
wir schreiben wieder $a_k$ für den $k$-ten Spaltenvektor von $A$.

Weil die Zeilen- und Spaltensumme immer $0$ ist, sind sowohl die
Zeilen wie auch die Spalten linear abhängig. Insbesondere kann die Spalte
$n$ als Linearkombination der anderen Spalten geschrieben werden:
\[
a_n=-a_1-a_2-\dots-a_{n-1}
\]
Wir haben also die Determinante
\[
\operatorname{cof}(A)_{ij}
=
(-1)^{i+j}
\left|\;
\begin{matrix}
a_{11}&a_{12}&\dots &a_{1,n-1}&-a_{11}-a_{12}-\dots-a_{1,n-1}\\
\vdots&\vdots&\ddots&\vdots\\
a_{n-1,1}&a_{n-1,2}&\dots&a_{n-1,n-1}&-a_{n-1,1}-a_{n-1,n-1}-\dots-a_{n-1,n-1}\\
*&*&\dots&*&*
\end{matrix}
\;\right|
\]
zu berechnen, in der die Zeile $i$ und die Spalte $j$ fehlt. Anstelle
der Sterne in der letzten Zeile sind entsprechende Linearkombinationen
der vorangegangenen Zeilen einzusetzen, die sich daraus ergeben,
dass die Summe der Spalten jeweils $0$ ergibt.
Die Determinante ändert nicht, wenn man eine Spalte zur letzten Spalte
hinzuaddiert, oder eine Zeile zur letzten Zeile. Addiert man jede
andere Spalte zur letzten Spalte hinzu, bleibt dort nur $-a_i$.
Ebenso bleibt in der letzten Zeile nur die Zeile $-a_{j1}\dots-a_{jn}$,
wenn man jede Zeile zur letzten Zeile hinzuaddiert. Wir bekommen also
\[
\operatorname{cof}(A)_{ij}
=
(-1)^{i+j}
\left|\;
\begin{matrix}
a_{11}&a_{12}&\dots &a_{1,n-1}&-a_{1j}\\
a_{21}&a_{22}&\dots &a_{2,n-1}&-a_{2j}\\
\vdots&\vdots&\ddots&\vdots\\
a_{n-1,1}&a_{n-1,2}&\dots&a_{n-1,n-1}&-a_{n-1,j}\\
-a_{i1}&-a_{i2}&\dots&-a_{i,n-1}&a_{i,j}
\end{matrix}
\;\right|
\]
Die negativen Vorzeichen in der letzten Zeile und Spalte kann man aus
der Determinante herausnehmen, das ändert die Determinante nicht.
Ebenso kann man durch $n-i-1$ Zeilenvertauschungen und durch
$n-j-1$ Spaltenvertauschungen die letzte Zeile in die
Position $i$ bringen, und die letzte Spalte in die Position $j$.
Dabei ändert sich das Vorzeichen $2n-i-j-2$ mal, wir erhalten also
einen zusätzlichen Vorzeichenfaktor $(-1)^{i+1j}$
\[
\operatorname{cof}(A)_{ij}=
\left|\;
\begin{matrix}
a_{11}&a_{12}&\dots&a_{1,n-1}\\
a_{21}&a_{22}&\dots&a_{2,n-1}\\
\vdots&\vdots&\ddots&\vdots\\
a_{n-1,1}&a_{n-1,2}&\dots&a_{n-1,n-1}
\end{matrix}
\;\right|
=\operatorname{cof}(A)_{nn}.
\qedhere
\]
\end{proof}
\begin{beispiel}
Die Berechnung der Kofaktor-Matrix in Octave ist etwas mühsam, weil
die Sprache keine Funktion dafür bereitstellt.
Im Internet findet man die Lösung $\operatorname{cof}(A)=\det(A)A'^{-1}$,
welche natürlich nicht anwendbar ist, weil unsere Matrix $\Delta$
nicht invertierbar ist.
Eine Quick-and-Dirty-Lösung
ist die Funktion
\verbatiminput{applications/cofactor.m}
Wenden wir diese Funktion auf den Laplace-Operator des Beispiels
(\ref{samplelaplace}) an, erhalten wir
{\small
\begin{verbatim}
> cof(Delta)
ans =

  252.00  252.00  252.00  252.00  252.00  252.00  252.00  252.00
  252.00  252.00  252.00  252.00  252.00  252.00  252.00  252.00
  252.00  252.00  252.00  252.00  252.00  252.00  252.00  252.00
  252.00  252.00  252.00  252.00  252.00  252.00  252.00  252.00
  252.00  252.00  252.00  252.00  252.00  252.00  252.00  252.00
  252.00  252.00  252.00  252.00  252.00  252.00  252.00  252.00
  252.00  252.00  252.00  252.00  252.00  252.00  252.00  252.00
  252.00  252.00  252.00  252.00  252.00  252.00  252.00  252.00
\end{verbatim}
}
\end{beispiel}

\begin{hilfssatz}
\label{matrixtreetheorem1}
Für ein Netzwerk $G$ mit nur einer Kante ist
\[
\Delta=\begin{pmatrix}
1&-1\\-1&1
\end{pmatrix},
\qquad
\operatorname{cof}(A)=\begin{pmatrix}1&1\\1&1\end{pmatrix}
\]
und
$\tau(G)=1$.
\end{hilfssatz}

\begin{proof}[Beweis]
Ein Netzwerk mit nur einer Kante hat natürlich nur einen einzigen
Spannbaum, also ist $\tau(G)=1$. Der $\partial$-Operator ist
\[
\partial=\begin{pmatrix}-1\\1\end{pmatrix}
\]
und daher
\begin{align*}
\Delta&=
\partial\partial^t=
\begin{pmatrix}-1\\1\end{pmatrix}
\begin{pmatrix}-1&1\end{pmatrix}
=
\begin{pmatrix}
1&-1\\
-1&1
\end{pmatrix}
\\
\operatorname{cof}(\Delta)&=\begin{pmatrix}1&1\\1&1\end{pmatrix}.
\qedhere
\end{align*}
\end{proof}
Jetzt muss nur noch gezeigt, werden, dass die Eigenschaft, dass alle
Kofaktoren des Laplace-Operators die Zahl der Spannbäume angeben,
beim Hinzufügen einer Kante nicht zerstört wird.

\begin{hilfssatz}
\label{matrixtreetheoremstep}
Falls die Aussage von Satz~\ref{matrixtreetheorem}
für alle Graphen mit höchstens $n$ Kanten
gilt, dann gilt sie auch für alle Graphen mit $n+1$ Kanten.
\end{hilfssatz}

\begin{proof}[Beweis]
Wir können annehmen, dass die hinzugefügte Kante die Nummer $1$
hat, dass wir also den Laplace-Operator wie folgt modifizieren:
\[
\Delta_n
=
\left(\begin{tabular}{>{$}c<{$}>{$}c<{$}|>{$}c<{$}>{$}c<{$}}
d_1   &    0&u_1\\
    0 &d_2  &u_2\\
\hline
 u_1^t&u_2^t&A
\end{tabular}
\right)
\quad
\rightarrow
\quad
\Delta_{n+1}
=
\left(\begin{tabular}{>{$}c<{$}>{$}c<{$}|>{$}c<{$}>{$}c<{$}}
d_1+1 &   -1&u_1\\
   -1 &d_2+1&u_2\\
\hline
u_1^t &u_2^t&A
\end{tabular}
\right)
\]
Wir müssen jetzt einsehen, dass sich bei dieser Operation
der Wert von $\tau(G)$ analog zum Wert der Kofaktoren ändert.

Das Hinzufügen einer Kante macht aus dem Graphen $G$ einen neuen
Graphen $G'$. Die Spannbäume von $G$ sind auch Spannbäume von $G'$,
allerdings solche, die die neue Kante nicht verwenden. Zusätzlich
hat $G'$ auch noch Spannbäume, die die neue Kante nicht enthalten,
sie setzen sich zusammen aus der neuen Kante und aus einem beinahe-Spannbaum
von $G$, der einen der Endpunkte der neuen Kante nicht triff. 
Ziehen wir die beiden Endpunkte der neuen Kante zu einem einzigen
Punkt zusammen, entsteht ein Graph $G^*$, wir suchen Spannbäume in 
diesem zusammengezogenen Graphen:
\[
\tau(G')=\tau(G)+\tau(G^*)
\]
Für beide Graphen $G$ und $G^*$ gilt die Voraussetzung, man kann
$\tau(G)$ und $\tau(G^*)$ also mit Kofaktoren berechnen. Der Laplace-Operator
von $G^*$ ist
\[
\Delta^*
=
\left(\begin{tabular}{>{$}c<{$}|>{$}c<{$}>{$}c<{$}}
d_1+d_2&u_1+u_2\\
\hline
u_1^t +u_2^t&A
\end{tabular}
\right)
\]
Daraus lesen wir ab $\tau(G^*)=\det(A)$.
Die Behauptung ist also bewiesen, wenn
\[
\left|\;\begin{tabular}{>{$}c<{$}|>{$}c<{$}>{$}c<{$}}
d_2+1&u_2\\
\hline
u_2^t&A
\end{tabular}
\;
\right|
\overset{?}{=}
\left|\;\begin{tabular}{>{$}c<{$}|>{$}c<{$}>{$}c<{$}}
d_2  &u_2\\
\hline
u_2^t&A
\end{tabular}
\;
\right|
+\det(A)
\]
ist.
Entwickeln wir die beide Seiten nach der ersten Zeile, bekommen
wir 
\[
(d_2+1)\det(A)+(\text{Terme mit $u_2$})
\overset{?}{=}
d_2\det(A)+(\text{Terme mit $u_2$})+\det(A),
\]
was offensichtlich übereinstimmt.

Eine kleine Unsauberkeit ist jedoch noch zu entfernen.
Beim Zusammenziehen einer Kante kann es passieren, dass zwischen zwei
Knoten plötzlich zwei Verbindungen entstehen. Doch für die
Theorie ist dies kein Hindernis: im Laplace-Operator steht
auf der Diagonalen immer noch der Grad des Knoten, also die Anzahl
der dort zusammentreffenden Kanten. Im Ausserdiagonalelement $(i,j)$
steht $-\text{Anzahl}$ der Verbindungen zwischen Knoten $i$ und $j$.
\end{proof}

\begin{proof}[Beweis von Satz~\ref{matrixtreetheorem}]
Der Beweis des Satzes~\ref{matrixtreetheorem} kann jetzt mit vollständiger
Induktion erfolgen.
Hilfssatz~\ref{matrixtreetheorem1} ist die Induktionsverankerung,
Hilfssatz~\ref{matrixtreetheoremstep} ist der Induktionsschritt.
\end{proof}

Mit diesem Satz kann man jetzt auch die Zahl der Spannbäume in gewissen
Netzwerken mit sehr vielen Verbindungen zählen. 
\begin{definition}
Ein Graph mit $n$ Knoten heisst vollständig, wenn jeder Knoten mit
jedem anderen Knoten verbunden ist.
\end{definition}
Der Laplace-Operator eines vollständigen Graphen mit $n$ Knoten ist
\[
\Delta=\begin{pmatrix}
n-1   &  -1  & \dots & -1 \\
 -1   & n-1  & \dots & -1 \\
\vdots&\vdots&\ddots &\vdots\\
 -1   &  -1  &\dots  &n-1
\end{pmatrix}.
\]
Die Spannbäume in einem vollständigen Graphen kann man zählen, indem
man die Kofaktoren berechnet.

\begin{satz}[Cayley]
\index{Cayley-Theorem}
Die Zahl der Spannbäume in einem vollständigen Graphen mit $n$ 
Knoten ist $n^{n-2}$.
\end{satz}

\begin{proof}[Beweis]
Die gesuchte Zahl ist der Wert der $(n-1)\times(n-1)$-Determinante
\[
D_n
=
\left|\;
\begin{matrix}
n-1   &  -1  & \dots & -1 \\
 -1   & n-1  & \dots & -1 \\
\vdots&\vdots&\ddots &\vdots\\
 -1   &  -1  &\dots  &n-1
\end{matrix}
\;\right|.
\]
Zeilenoperationen ändern die Determinante nicht, wenn wir also die
erste Zeile von jeder anderen Zeile subtrahieren, bekommen wir
\[
D_n=
\left|\;
\begin{matrix}
n-1   &  -1  & \dots & -1 \\
 -n   & n    & \dots &  0 \\
\vdots&\vdots&\ddots &\vdots\\
 -n   &   0  &\dots  &n  
\end{matrix}
\;\right|.
\]
In den Zeilen $2$ bis $n-1$ kann man einen gemeinsamen Faktor $n$
aus der Determinante herausnehmen:
\[
D_n=
n^{n-2}
\left|\;
\begin{matrix}
n-1   &  -1  & \dots & -1 \\
 -1   & 1    & \dots &  0 \\
\vdots&\vdots&\ddots &\vdots\\
 -1   &   0  &\dots  &1  
\end{matrix}
\;\right|.
\]
Addieren wir jede Zeile ausser der ersten zur ersten, bekommen wir
\[
D_n=
n^{n-2}
\left|\;
\begin{matrix}
  1   &   0  & \dots &  0 \\
 -1   & 1    & \dots &  0 \\
\vdots&\vdots&\ddots &\vdots\\
 -1   &   0  &\dots  &1  
\end{matrix}
\;\right|=n^{n-2},
\]
da die letzte Matrix eine Dreiecksmatrix ist.
\end{proof}
