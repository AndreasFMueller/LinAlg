%
% qr-video.tex
%
% (c) 2020 Prof Dr Andreas Müller, Hochschule Rapperswil
%
\documentclass[handout,aspect=169]{beamer}
\usepackage[orientation=landscape,size=custom,scale=0.7,width=16,height=9,debug]{beamerposter}
\usepackage{amsmath}
\usepackage{mathtools}
\usepackage{color}
\usepackage{tikz}
\usepackage{tikz-cd}
\usetikzlibrary{shapes.geometric}
\usetikzlibrary{decorations.pathreplacing}
\usetikzlibrary{calc}
\usetikzlibrary{arrows}
\usetikzlibrary{3d}
\usetikzlibrary{arrows,shapes,math,decorations.text,automata}

\begin{document}
\definecolor{darkgreen}{rgb}{0,0.6,0}
\setbeamertemplate{frametitle}[default][right]

\definecolor{rot}{rgb}{0.6,0.2,0.4}
\definecolor{gruen}{rgb}{0.4,0.6,0.2}
\definecolor{blau}{rgb}{0.2,0.4,0.6}

\begin{frame}
\begin{center}
{\usebeamercolor[fg]{title}
\usebeamerfont{block title}
How can we find the QR decomposition of $A$?}
\end{center}
\vspace{20pt}
We are looking for a rotation $Q$ that turns $A$ into a triangular matrix $R$:
\vspace{20pt}
\[
QA = R
\]
\end{frame}

\begin{frame}
\begin{center}
{\usebeamercolor[fg]{title}
\usebeamerfont{block title}
This is the same as finding a rotation that maps}
\end{center}
%Find a rotation that maps
\begin{itemize}
\item {\color{rot}the first column to the $x$-axis}
\item {\color{gruen}the second column into the $x$-$y$-plane}
\end{itemize}
\begin{center}
\begin{tikzpicture}[>=latex,thick]

\begin{scope}[xshift=-2.43cm]
\fill[color=rot!50] (-1.25,-1.05) rectangle (-0.55,1.05);
\fill[color=gruen!50] (-0.35,-1.05) rectangle (0.35,1.05);
\fill[color=blau!50] (0.55,-1.05) rectangle (1.25,1.05);
\node[color=gray!80,opacity=0.7] at (0,0) [scale=4] {$A$};
\end{scope}

\begin{scope}[xshift=2.43cm]
\fill[color=rot!50] (-1.25,0.35) rectangle (-0.55,1.05);
\fill[color=gruen!50] (-0.35,-0.35) rectangle (0.35,1.05);
\fill[color=blau!50] (0.55,-1.05) rectangle (1.25,1.05);
\node[color=gray!80,opacity=0.7] at (0,0) [scale=4] {$R$};
\end{scope}

\node at (0,0) {$\displaystyle
\begin{pmatrix}
\mathstrut&\mathstrut&\mathstrut\\
\phantom{00}&\phantom{00}&\phantom{00}\\
\mathstrut&\mathstrut&\mathstrut
\end{pmatrix}
\overset{\textstyle Q}{\longrightarrow}
\begin{pmatrix}
\mathstrut&\mathstrut&\mathstrut\\
\phantom{00}&\phantom{00}&\phantom{00}\\
\mathstrut&\mathstrut&\mathstrut
\end{pmatrix}
$};
\end{tikzpicture}
\end{center}
\end{frame}

\begin{frame}
\begin{center}
Reflect the {\color{rot}first column} to the $x$-axis
\end{center}
\end{frame}

\begin{frame}
\begin{center}
Reflect the {\color{gruen}second column} to the $x$-$y$-plane
\end{center}
\end{frame}

\begin{frame}
\begin{center}
But keep the $x$-axis fixed!
\end{center}
\end{frame}

\begin{frame}
\begin{center}
The composition of these reflections is the rotation $Q$
\end{center}
\end{frame}

\end{document}
