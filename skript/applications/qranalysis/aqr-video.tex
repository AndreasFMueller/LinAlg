%
% qam-video.tex
%
% (c) 2020 Prof Dr Andreas Müller, Hochschule Rapperswil
%
\documentclass[handout,aspect=169]{beamer}
\usepackage[orientation=landscape,size=custom,scale=0.7,width=16,height=9,debug]{beamerposter}
\usepackage{amsmath}
\usepackage{mathtools}
\usepackage{color}
\usepackage{tikz}
\usepackage{tikz-cd}
\usetikzlibrary{shapes.geometric}
\usetikzlibrary{decorations.pathreplacing}
\usetikzlibrary{calc}
\usetikzlibrary{arrows}
\usetikzlibrary{3d}
\usetikzlibrary{arrows,shapes,math,decorations.text,automata}

\begin{document}
\definecolor{darkgreen}{rgb}{0,0.6,0}
\setbeamertemplate{frametitle}[default][right]

\begin{frame}
\begin{center}
What does this matrix do?
\end{center}

\[
A= \begin{pmatrix*}[r]
   1.218035& -0.054603& -0.494702\\
   0.527149&  1.176368& -0.233772\\
   0.445540&  0.642788&  0.497816
\end{pmatrix*}
\]
\end{frame}

\begin{frame}
\vspace{6.5cm}
\begin{center}
But can we understand that geometrically?
\end{center}
\end{frame}

\begin{frame}
\frametitle{QR decomposition}
The QR decomposition allows us to write $A$ as a product
\begin{center}
\begin{tikzpicture}[>=latex,thick]
\node at (-0.5,1) {$A = Q \; R$};
\draw (-0.4,0.7) -- (-2,-0.7);
\draw (0.4,0.7) -- (2,-0.7);
\node at (-2,-1) {orthogonal};
\node at (-2,-1.7) {some rotation};
\node at (2,-1) {triangular};
\node at (2,-1.7) {???};
\end{tikzpicture}
\end{center}
\end{frame}

\begin{frame}
\vspace{6.5cm}
\begin{center}
The last step is a rotation
\end{center}
\end{frame}

\begin{frame}
\begin{center}
The triangular matrix $R$ makes this shape
\vspace{6.5cm}

How does it do that?
\end{center}
\end{frame}

\begin{frame}
\begin{center}
\begin{tikzpicture}[>=latex,thick]
\node at (0,0) {$\begin{aligned}
R &=
\begin{pmatrix}
s_x &  *  &  *  \\
 0  & s_y &  *  \\
 0  &  0  & s_z
\end{pmatrix}
\\[40pt]
&=
\begin{pmatrix}
1&*&*\\
0&1&*\\
0&0&1
\end{pmatrix}
\underbrace{
\begin{pmatrix}
 1 & 0 & 0 \\
 0 & 1 & 0 \\
 0 & 0 &s_z
\end{pmatrix}}_{\displaystyle=S_z}
\underbrace{
\begin{pmatrix}
 1 & 0 & 0 \\
 0 &s_y& 0 \\
 0 & 0 & 1 
\end{pmatrix}}_{\displaystyle=S_y}
\underbrace{
\begin{pmatrix}
s_x& 0 & 0 \\
 0 & 1 & 0 \\
 0 & 0 & 1 
\end{pmatrix}}_{\displaystyle=S_x}
\end{aligned}
$};
\node at (-7,0.5) [right] {pull the $s_*$ out into separate axis rescaling matrices:};
\end{tikzpicture}
\end{center}
\end{frame}

\begin{frame}
\vspace{6.5cm}
\begin{center}
Rescale $x$-axis: $S_x$\strut
\end{center}
\end{frame}

\begin{frame}
\vspace{6.5cm}
\begin{center}
Rescale $y$-axis: $S_y$\strut
\end{center}
\end{frame}

\begin{frame}
\vspace{6.5cm}
\begin{center}
Rescale $z$-axis: $S_z$\strut
\end{center}
\end{frame}

\begin{frame}
\begin{center}
\begin{tikzpicture}[>=latex]

\node[color=red] at (-2,3) {\begin{minipage}{6cm}
\raggedright
move tip of $y$-axis vector in the $x$-direction
\end{minipage}};

\node[color=blue] at (5,3) {\begin{minipage}{6cm}
\raggedright
move tip of $z$-axis vector in the $x$-direction
\end{minipage}};

\node[color=darkgreen] at (5,-2) {\begin{minipage}{6cm}
\raggedright
move tip of $z$-axis vector in the $y$-direction
\end{minipage}};

\node at (-2,0) {$\displaystyle
R = \begin{pmatrix} 
  *  & {\color{red}r_1} & {\color{blue}r_2} \\
  0  &  *  & {\color{darkgreen}r_3} \\
  0  &  0  &  *
\end{pmatrix}$};

\draw[->,color=red,line width=1.4pt] (-2,2.3) -- (-1.6,1);
\draw[->,color=blue,line width=1.4pt] (1.9,2.3) -- (-0.5,1);
\draw[->,color=darkgreen,line width=1.4pt] (1.9,-1.4) -- (-0.4,-0.2);

\end{tikzpicture}
\end{center}
\end{frame}

\begin{frame}
\vspace{6.5cm}
\begin{center}
Shear tip of $y$-axis vector in the $x$ direction
\end{center}
\end{frame}

\begin{frame}
\vspace{6.5cm}
\begin{center}
Shear tip of $z$-axis vector in the $x$ direction
\end{center}
\end{frame}

\begin{frame}
\vspace{6.5cm}
\begin{center}
Shear tip of $z$-axis vector in the $y$ direction
\end{center}
\end{frame}

\begin{frame}
The QR decomposition helps to decompose an arbitrary matrix $A$
into
\[
A
=
\text{rotation}
\cdot
\text{three shears}
\cdot
\text{three rescalings}
\]
\vspace{1cm}

Enjoy them all one after the other
\end{frame}


\end{document}
