%
% qam.tex -- Quadratur-Amplituden-Modulation mit Matrizen und Vektorgeometrie
%
% (c) 2020 Prof Dr Andreas Müller, Hochschule Rapperswil
%
\section{Quadratur-Amplituden-Modulation
\label{section:quadratur-amplituden-modulation}}
\rhead{Quadratur-Amplituden-Modulation}
Um ein zeitabhängiges Signal drahtlos zu übertragen, muss es auf
ein hochfrequentes Trägersignal aufmoduliert werden.
Wie macht man das und wie kann man das ursprüngliche Signal
wiedergewinnen?
Eine besonders flexible Methode, die sogenannte
Quadratur-Amplituden-Modu\-lation, lässt sich mit Hilfe von Vektorgeometrie
und Drehmatrizen besonders leicht verstehen.


%
% amplitudenmodulation.tex
%
% (c) 2020 Prof Dr Andreas Müller, Hochschule Rapperswil
%
\subsection{Amplitudenmodulation
\label{subsection:amplitudenmodulation}}
Die Technik der Amplitudenmodulation geht auf die Anfangszeit des
Radios zurück.
Um ein zeitabhängiges Audio-Signal $I(t)$ mit Frequenzen von typischwerweise
wenigen hundert Herz oder wenigen Kilohertz drahtlos zu übertragen,
verändert man die Amplitude eines Trägersignals von mehreren hundert
Kilohertz oder mehr und leitet es zu einer Antenne.
Diese strahlt dann eine entsprechend oszillierendes elektromagnetisches
Feld ab, welches von einem Empfänger aufgefangen und wieder hörbar gemacht
werden kann.

\begin{figure}
\centering
\includegraphics{applications/qam/am.pdf}
\caption{Amplitudenmodulation eines Signals $I(t)$ auf eine
Trägerfrequenz $\cos\omega t$.
Die Amplitude des Trägers wird im Takt des Signals verändert.
\label{figure:qam:am}}
\end{figure}

In Abbildung~\ref{figure:qam:am} wird die Amplitude des Träger
$\cos\omega t$ mit der Kreisfrequenz $\omega$ wird im Takt des
Signals verändert,
konkret wird der Träger mit $1+I(t)$ multipliziert,
das übermittelte Signal ist also
\begin{equation}
(1+I(t)) \cos\omega t.
\end{equation}
Das modulierte Signal ist besonders leicht zu demodulieren.
Es reicht, das Signal gleichzurichten und verbleibenden Reste
der Trägerfrequenz sowie die konstente Komponten auszufiltern.
Die konstante Komponente (der Summand $1$) ist nicht wirklich
interessant und dient nur einer leichter verständlichen Darstellung
in Abbildung~\ref{figure:qam:am}, wir werden sie in Zukunft 
ignorieren und nur noch das modulierte Signal $I(t)\cos\omega t$
betrachten.

Mit dieser Methode kann man zu jeder Zeit $t$ einen einzelnen
Wert $I(t)$ übermitteln.
Für die Praxis ist das schon bei Audiosignalen oft ungenügend, man
möchte doch mindestens die beiden Stereokanäle eines qualitativ
hochwertigen Signals übertragen können.

Die zweite Schwierigkeit ist die Frage, warum wir $\cos\omega t$
dem ebenfalls möglichen Träger $\sin\omega t$ vorziehen sollen.
Die beiden unterscheiden sich nur um eine Phasenverschiebung
von $90^\circ$, die für die oben beschriebene Modulation und
Demodulation bedeutungslos ist.







%
% zweidimensionale.tex
%
% (c) 2020 Prof Dr Andreas Müller, Hochschule Rapperswil
%
\subsection{Zweidimensionale Signale
\label{subsection:qam:zweidimensional}}
Wir suchen also nach einem Verfahren, mit welchem wir nicht nur
ein einzelnes zeitabhängiges Signal $I(t)$ übertragen können, sondern
auch noch ein zweites Signal, welches wir mehr oder weniger aus
historischen Gründen $Q(t)$ nennen wollen\footnote{Das Signal $I(t)$
heisst auch die In-phase-Komponente und $Q(t)$ Quadratur-Komponente}.
Man kann sich die beiden Signale zum Beispiel also die beiden
Stereokanäle eines Audiosignals vorstellen.
Wir stellen uns die beiden Komponenten als untrennbar zusammengehörig
vor, es ist daher sinnvoll, sie als zweidimensionalen Vektor
\[
\vec{v}(t)
=
\begin{pmatrix}I(t)\\Q(t)\end{pmatrix}
\]
zu schreiben.
Dieser Vektor beschreibt zu jeder Zeit $t$ einen Punkt in der
$I$-$Q$-Ebene.
Zu verschiedenen Zeiten beschreibt $\vec{v}(t)$ eine Kurve
in der Ebene.
Mit einem Oszilloskop im X-Y-Modus kann man den Vektor
sichtbar machen.
Zum Beispiel führt das Signal
\begin{equation}
I(t) = \cos t,\quad
Q(t) = \sin 3t
\qquad
\Rightarrow
\qquad
\vec{v}(t)
=
\begin{pmatrix}
\cos t\\
\sin 3t
\end{pmatrix}
\label{eqn:qam:liss1}
\end{equation}
auf die in Abbildung~\ref{figure:qam:lissajous} dargestellte Kurve.
\begin{figure}
\centering
\includegraphics[width=0.48\hsize]{applications/qam/images/lissajous.pdf}
\includegraphics[width=0.48\hsize]{applications/qam/images/lissajous.jpg}
\caption{Lissajous-Figur des zweidimensionalen Signals
\eqref{eqn:qam:liss1} kann auf einem Oszilloskop im X-Y-Modus
sichtbar gemacht werden.
\label{figure:qam:lissajous}}
\end{figure}
Solche Kurven sind bekannt als Lissajous-Figuren.
Mit komplizierteren Funktion $I(t)$ und $Q(t)$ kann fast jede
Linienzeichnung auf den Schirm des Oszilloskops gezaubert werden,
wie zum Beispiel auch die Internet ``Kunstform'' der Oscilloscope
Music (\url{https://www.youtube.com/watch?v=qnL40CbuodU}) zeigt.
In Abbildung~\ref{figure:qam:pilze} werden zum Beispiel Pilze und
ein Schmetterling mit zwei geeigneten Funktionen $I(t)$ und $Q(t)$
gezeichnet.

\begin{figure}
\centering
\includegraphics[width=0.5\hsize]{applications/qam/images/pilze.png}
\caption{Pilze und ein Schmetterling gezeichnet von zwei Signalen
$I(t)$ und $Q(t)$ aus dem Video \url{https://youtu.be/rtR63-ecUNo}
\label{figure:qam:pilze}}
\end{figure}




%
% modulation.tex
%
% (c) 2020 Prof Dr Andreas Müller, Hochschule Rapperswil
%
\subsection{Modulation zweidimensionaler Signale
\label{subsection:modulation}}
Wir streben jetzt an, ein zweidimensionales Signal
\[
\vec{v}(t)
=
\begin{pmatrix}I(t)\\Q(t)\end{pmatrix}
\]
drahtlos zu übertragen und müssen zu diesem Zweck ein geeignetes
Modulationsverfahren finden.

\begin{figure}
\centering
\includegraphics{applications/qam/images/icos.pdf}
\caption{Modulation des Signals $I(t)$ als Drehung um den Winkel $\omega t$
auf einem Kreis mit Radius $I(t)$.
\label{qam:figure:icos}}
\end{figure}
Mit nur dem einen Signal $I(t)$ haben wir $I(t)\cos\omega t$ als moduliertes
Signal gewählt.
Geometrisch können wir das auf einem Kreis mit Radius $I(t)$ als Drehung
um den Winkel $\omega t$ mit anschliessender Projektion auf die horizontale
Achse verstehen (Abbildung~\ref{qam:figure:icos} links).

Es ist daher naheliegend, für die Modulation des zweidimensionalen Signals
ebenfalls eine Drehung um den Winkel $\omega t$ zu verwenden.
Der Vektor $\vec{v}(t)$ wird in der $I$-$Q$-Ebene gedreht wie in
Abbildung~\ref{qam:figure:icos} rechts gezeigt.
Dazu kann eine Drehmatrix $D_{\omega t}$ verwendet werden.
Wir berechnen
\begin{equation}
D_{\omega t}
\vec{v}(t)
=
\begin{pmatrix}
\cos\omega t & -\sin\omega t\\
\sin\omega t &\phantom{-}\cos\omega t
\end{pmatrix}
\begin{pmatrix}I(t)\\Q(t)\end{pmatrix}
=
\begin{pmatrix}
I(t)\cos\omega t - Q(t) \sin\omega t\\
I(t)\sin\omega t + Q(t) \cos\omega t
\end{pmatrix}
=
\begin{pmatrix}
s(t)\\c(t)
\end{pmatrix}
\label{qam:eqn:modulation}
\end{equation}
Das zweite Signal $Q(t)$ modulieren wir also statt mit $\cos\omega t$ 
mit der Funktion $-\sin\omega t$.
Natürlich können wir aus den beiden Funktionen $I(t)\cos\omega t$ und
$-Q(t)\sin\omega t$ auch die Funktionen $I(t)$ und $Q(t)$ zurückgewinnen,
das reicht aber nicht.
Dazu brauchen wir nämlich $I(t)\cos\omega t$ und $-Q(t)\sin\omega t$
unabhängig voneinander, wir brauchen also zwei unabhängig Übertragungskanäle
für die beiden Signale, was wir vermeiden wollen.

\begin{figure}
\centering
\includegraphics{applications/qam/images/sep.pdf}
\caption{Rekonstruktion der Signale $I(t)$ und $Q(t)$ (oberste zwei Graphen)
aus der Summe $s(t) = I(t)\cos\omega t - Q(t)\sin\omega t$ (Mitte).
Die Nullstellen von $\cos\omega t$ sind durch feine blaue Linien 
dargestellt, die Nullstellen von $\sin\omega t$ durch feine rote Linien.
Der Graph von $s(t)$ ist jeweils mit der Farbe eingefärbt, die den
dominanten Beitrag repräsentiert.
Blaue Segmente im Graphen von $s(t)$ bedeuten, dass vor allem der Wert
von $Q(t)$ zum Wert beiträgt, dies geschieht in der Umgebung von
Nullstellen von $\cos\omega t$, in den roten Segmenten ist es der Wert von
$I(t)$, welcher dominiert während $\sin\omega t$ eine Nullstelle durchläuft.
Fette Punkte auf dem Graphen von $s(t)$ markieren Punkte bei den
genannten Nullstellen.
Die leeren Punkte sind Werte von $s(t)$, die um das Vorzeichen des
Trägeres korrigiert wurden, sie liegen genau auf dem Graphen der
ursprünglichen Signale $I(t)$ und $Q(t)$.
Die untersten zwei Graphen zeigen die rekonstruierten Signale
$\hat{I}(t)=s(t) \cos\omega t$ und $\hat{Q}(t) = -s(t) \sin\omega t$, 
welche in Abschnitt~\ref{subsection:demodulation} erklärt werden.
\label{figure:qam:sep}}
\end{figure}

Einzelne Werte der Funktionen $I(t)$ und $Q(t)$ können aber aus der Summe
\[
s(t)
=
I(t)\cos\omega t - Q(t)\sin\omega t
\]
rekonstruiert werden.
An den Stellen $t = k\pi/\omega$ für $k\in\mathbb Z$ verschwindet
der Faktor $\sin\omega t$,
so dass an diesen Stellen der zweite Summand in $s(t)$ wegfällt.
Ebenso wird der erste Summand an den Stellen
$t = (k+\frac12)\pi/\omega$ verschwinden.
Dieser Sachverhalt ist in Abbildung~\ref{figure:qam:sep} dargestellt.
Es ist also
\[
s\biggl(k\cdot \frac{\pi}{2\omega}\biggr)
=
\begin{cases}
I(k\cdot \frac{\pi}{2\omega})\cdot(-1)^{\frac{k}2}
&\qquad \text{$k$ gerade,}\\[5pt]
Q(k\cdot \frac{\pi}{2\omega})\cdot(-1)^{\frac{k-1}2}
&\qquad \text{$k$ ungerade.}
\end{cases}
\]
Zu Zeitpunkten, die Vielfache von $\pi/2\omega$ sind, kann man also aus
$s(t)$ die Werte von $I(t)$ und $Q(t)$ ablesen.
Tatsächlich lernt man im Fach {\em Signale und Systeme}, dass man daraus
die Funktionen $I(t)$ und $Q(t)$ rekonstruieren kann, wenn sie keine
Frequenzkomponenten grösser als die Trägerfrequenz haben.
Die Summe $s(t)$ ist also etwas, was man potentiell drahtlos übermitteln
kann, und woraus man die Komponenten $I(t)$ und $Q(t)$ zurückgewinnen kann.
Man nennt dieses Modulationsverfahren {\em Quadratur-Amplituden-Modulation}.

Da wir über das nötige signaltheoretische Wissen noch nicht verfügen,
müssen wir eine alternative, geometrische Methode suchen, wie wir
aus $s(t)$ die Komponenten $I(t)$ und $Q(t)$ wiedergewinnen können.
Das modulierte Signal $s(t)$ ist also nichts anderes als eine
Komponente eines Vektors, der entsteht indem $\vec{v}(t)$ mit sehr grosser
Winkelgeschwindigkeit $\omega$ um den Ursprung gedreht wird.
Wir sind aber insofern nicht weiter, dass wir $I(t)$ und $Q(t)$ noch nicht
rekonstruieren können.




%
% demodulation.tex -- Demodulation von QAM
%
% (c) 2020 Prof Dr Andreas Müller, Hochschule Rapperswil
%
\subsection{Demodulation
\label{subsection:demodulation}}
Die modulierten Komponenten $s(t)$ und $c(t)$ entstehen gemäss
\eqref{eqn:qam:modulation}
durch eine sehr rasche Drehung $D_{\omega t}\vec{v}(t)$
des Vektor $\vec{v}(t)$.
Da die Drehung durch eine Matrix beschrieben wird, können wir
sie auch wieder rückgängig machen, indem wir mit der inversen
Matrix
\[
D^{-1}_{\omega t} = D_{-\omega t}
=
\begin{pmatrix}
\phantom{-}\cos\omega t & \sin\omega t \\
         - \sin\omega t & \cos\omega t
\end{pmatrix}
\]
multiplizieren.
So finden wir
\[
D_{\omega t}^{-1}
\begin{pmatrix}s(t)\\c(t)\end{pmatrix}
=
D_{\omega t}^{-1}
D_{\omega t}
\begin{pmatrix} I(t)\\Q(t)\end{pmatrix}
=
\begin{pmatrix}
I(t)\\
Q(t)
\end{pmatrix}.
\]
Es ist also klar, dass man aus $s(t)$ und $c(t)$ die ursprünglichen Signale
$I(t)$ und $Q(t)$ rekonstruieren kann.
Allerdings ist auch dies nicht wirklich eine Lösung des Problems.
Es ist immer noch notwendig, die beiden Funktionen $s(t)$ und $c(t)$
getrennt zu übertragen, um $I(t)$ und $Q(t)$ wiederzugewinnen.

Wir suchen ein Verfahren, mit dem wir $I(t)$ und $Q(t)$ allein aus
$s(t)$ zurückgewinnen können.
Auch für dieses Problem suchen wir eine geometrische Lösung.
Wir gehen dazu aus von der Gleichung
\[
\vec{v}(t)
=
D_{-\omega t}\underbrace{D_{\omega t}
\vec{v}(t)}_{=\begin{pmatrix}s(t)\\c(t)\end{pmatrix}}.
\]
Die Tatsache, dass wir $c(t)$ nicht übertragen wollen, können wir dadurch
abbilden, dass wir in der Gleichung eine Projektionsmatrix $P$
verwenden, um die Komponeten $c(t)$ zu unterdrücken:
\[
P=\begin{pmatrix}1&0\\0&0\end{pmatrix}
\qquad\Rightarrow\qquad
P\begin{pmatrix}s(t)\\c(t)\end{pmatrix}
=
\begin{pmatrix}s(t)\\0\end{pmatrix}.
\]
Durch diese Änderung wird man natürlich nicht mehr $I(t)$ und $Q(t)$ 
zurückgewinnen können, stattdessen man wird modifizierte Funktionen
$\hat{I}(t)$ und $\hat{Q}(t)$ erhalten.
Das ganze Übertragung\-system könnte daher mit dem Matrizenprodukt
\[
\begin{pmatrix}
\hat{I}(t)\\
\hat{Q}(t)
\end{pmatrix}
=
D_{-\omega t} \begin{pmatrix}s(t)\\0\end{pmatrix}
=
D_{-\omega t} P D_{\omega t}\vec{v}(t)
\]
Wegen
\[
D_{-\omega t}P
=
\begin{pmatrix}
\phantom{-}\cos\omega t & 0 \\
         - \sin\omega t & 0
\end{pmatrix}
\]
bedeutet das
\begin{align*}
\hat{I}(t) &= \phantom{-}\cos\omega t s(t),\\
\hat{Q}(t) &= -\sin\omega t s(t).
\end{align*}
Wir bezeichnen den Vektoren mit diesen Komponenten als
\[
\hat{v}(t) = \begin{pmatrix}\hat{I}(t)\\\hat{Q}(t)\end{pmatrix}.
\]
Der Vektor ist also, was von der Rekonstruktion nach dem Wegfallen
der Komponente $c(t)$ noch übrig bleibt.

Wie unterschieden sich $\hat{I}(t)$ und $\hat{Q}(t)$ von $I(t)$ und $Q(t)$?
Dazu berechnen wir
\begin{align*}
D_{-\omega t}PD_{\omega t}
&=
\begin{pmatrix}
\phantom{-}\cos\omega t & \sin\omega t \\
         - \sin\omega t & \cos\omega t
\end{pmatrix}
\begin{pmatrix} 1 & 0 \\ 0 & 0 \end{pmatrix}
\begin{pmatrix}
\cos\omega t &          - \sin\omega t \\
\sin\omega t & \phantom{-}\cos\omega t
\end{pmatrix}
\\
&=
\begin{pmatrix}
 \cos^2\omega t           & -\cos\omega t \sin\omega t \\
-\cos\omega t \sin\omega t&\sin^2\omega t
\end{pmatrix}
=
\frac12
\begin{pmatrix}
1+\cos 2\omega t &  -\sin 2\omega t \\
 -\sin 2\omega t & 1-\cos 2\omega t
\end{pmatrix}
\\
&=
\frac12 E + \frac12
\begin{pmatrix}
 \cos 2\omega t & -\sin 2\omega t \\
-\sin 2\omega t & -\cos 2\omega t
\end{pmatrix}.
\end{align*}
Nach der Rekonstruktion bleiben also zwei Komponenten
\[
\hat{v}(t)
=
\frac12\vec{v}(t)
+
\frac12
\begin{pmatrix}
\phantom{-}\cos 2\omega t & -\sin 2\omega t \\
         - \sin 2\omega t & -\cos 2\omega t
\end{pmatrix}\vec{v}(t).
\]
Die erste Komponente ist bis auf den Faktor $\frac12$ der gesuchte
Vektor $\vec{v}(t)$.
Doch was ist die zweite Komponente?
Die Matrix kann man auch schreiben als
\begin{align*}
\begin{pmatrix}
\phantom{-}\cos2\omega t&-\sin2\omega t\\
         - \sin2\omega t&-\cos2\omega t
\end{pmatrix}
&=
\underbrace{
\begin{pmatrix}
1& 0\\
0&-1
\end{pmatrix}}_{\displaystyle = S}
\begin{pmatrix}
\cos2\omega t &          - \sin2\omega t \\
\sin2\omega t & \phantom{-}\cos2\omega t
\end{pmatrix}
=
\begin{pmatrix}
1& 0\\
0&-1
\end{pmatrix}
D_{2\omega t},
\end{align*}
bis auf die Spiegelungsmatrix $S$ handelt es sich also wieder um eine
Drehung.
Zusammen finden wir
\begin{equation}
\hat{v}(t)
=
\frac12 \vec{v}(t)
+
\frac12 SD_{2\omega t}\vec{v}(t).
\label{eqn:qam:filter}
\end{equation}

Gehen wir davon aus, dass die Bewegung des Vektors $\vec{v}(t)$ in
der $I$-$Q$-Ebene sehr viel langsamer ist als die Drehung mit der
Winkelgeschwindigkeit $2\omega$, dann können wir den zweiten Term
näherungsweise eliminieren.
Um dies zu verstehen, nehmen wir an, dass $\vec{v}(t)$ während eines
Zeitintervalls der Länge $L=\pi/\omega$ konstant ist ist.
Wir bezeichnen Mittelwerte über das Intervall mit dem Buchstaben $M$.
Wir beachten dann, dass der zweite Term in \eqref{eqn:qam:filter}
für dieses Zeitintervall die gleichförmige Drehung des Vektors
um den Nullpunkt beschreibt.
Der Mittelwert der zweiten Komponente über das Intervall verschwindet daher:
\[
M\frac12SD_{2\omega t}\vec{v}(t)
=
0
\]
Die erste Komponente von \eqref{eqn:qam:filter} ist während des Intervalls
konstant, ihr Mittelwert ist daher
\[
M\frac12\vec{v}(t)
=
\frac12\vec{v}(t)
\]
Wir finden daher den Mittelwert
\[
M\hat{v}(t)
=
\frac12\vec{v}(t),
\]
bis auf den Faktor $\frac12$ also $\vec{v}(t)$ rekonstruiert.
Verändert sich $\vec{v}(t)$ während des Intervalls um einen Betrag
kleiner als $\varepsilon$, dann kommt ein Fehler hinzu, der ebenfalls
von der Grössenordnung $\varepsilon$ ist.

In praktischen Anwendungen ist die Frequenz des Trägers mehrere
Grössenordnungen grösser als die typischen Frequenzen in $I(t)$ 
und $Q(t)$, die Annahme, dass sich $\vec{v}(t)$ während einer halben
Trägerperiode nicht ändert, ist daher mit grosser Genauigkeit erfüllt.
Die Mittelwertbildung wird technisch mit Hilfe eines Tiefpassfilters
realisiert.


%
% beispiele.tex
%
% (c) 2020 Prof Dr Andreas Müller, Hochschule Rapperswil
%
\subsection{Beispiele
\label{subsection:qam:beispiele}}
Die Quadratur-Amplituden-Modulation ermöglicht, im Vergleich zur
Trägerfrequenz langsam veränderliche zweidimensionale Signale zu
übertragen und wieder zu rekonstruieren.
Der besondere Nutzen dieser Technik ist jedoch, dass sie viele
ältere Modulationsverfahren als Spezialfälle enthält, wie in
diesem Abschnitt gezeigt werden soll.

\subsubsection{Amplitudenmodulation}
Amplitudenmodulation konnten wir verstehen, bevor wir $Q(t)$ kannten,
sie ist der Spezialfall $Q(t)=0$.

\subsubsection{Frequenzmodulation}
Bei der Frequenzmodulation des UKW-Radios wird die Trägerfrequenz
im Takt des zu übertragenden Tonsignals verändert.
Lässt sich dies auch mit Hilfe der Signale $I(t)$ und $Q(t)$
beschreiben?
Das ausgestrahlte Signal $s(t)$ entsteht als erste Komponente
des Vektors $D_{\omega t}\vec{v}(t)$.
Für konstantes $\vec{v}(t)$ oszilliert es mit der Kreisfrequenz $\omega$.
Will man, dass es schneller oszilliert, dann muss man den Vektor
$\vec{v}(t)$ zusätzlich drehen, was man natürlich wieder mit einer
Drehmatrix machen kann.
Möchten wir die Frequenz um $\alpha$  steigern, dann müssen wir
statt eines konstanten Vektors $\vec{v}_0=(1,0)^t$ den gedrehten
Vektor $D_{\alpha t}\vec{v}_0$, was auf das modulierte Signal
\[
\begin{pmatrix}
s(t)\\c(t)
\end{pmatrix}
=
D_{\omega t}D_{\alpha t}\vec{v}_0
=
D_{(\omega+\alpha)t}\vec{v}_0
=
\begin{pmatrix}
\cos(\omega+\alpha)t 
\\
\dots
\end{pmatrix}
\]
führt.
Daraus liest man ab, dass für die Signale $I(t)$ und $Q(t)$
\begin{equation}
\begin{pmatrix}I(t)\\Q(t)\end{pmatrix}
=
D_{\alpha t}\vec{v}_0
\qquad\Rightarrow\qquad
\left\{
\quad
\begin{aligned}
I(t)&=\cos\alpha t\\
Q(t)&=\sin\alpha t
\end{aligned}
\right.
\end{equation}
gilt.
Insbesondere kann man auch die Frequenzmodulation mit der
Quadratur-Amplituden-Modulation realisieren.

\subsubsection{Analoges Farbfernsehen}
Die Entwicklung des analogen Farbfernsehens sah sich vor die Aufgabe 
gestellt, zusätzlich zur bereits im Schwarz-Weiss-Fernsehen übertragenen
Helligkeit (Luminanz, Y), zusätzlich die Farbinformation zu übermitteln.
Üblich ist dabei die Verwendung des YUV-Farbraumes, für den die zusätzlichen
Signale $U=R-Y$ und $V=B-Y$ benötigt werden, die Farbinformation codieren.
Für ein farbloses Bild sind $U=0$ und $V=0$.

Das Problem ist also, zusätzlich zum Luminanzbild, welches bereits
amplitudenmoduliert übertragen wird, den Farbvektor $(U,V)^t$ zu
übertragen.
Es liegt daher nahe, dafür die Quadratur-Amplituden-Modulation zu
verwenden.
Im in Europa üblichen PAL-System wurde für den Träger für das Farbsignal
die Frequenz 4.43361875\,MHz verwendet.
Da ein Phasenfehler im Empfänger zu einer Drehung des Farbvektors
und damit zu einer auffälligen Verschiebung der Farben auf dem Farbkreis
führen würde, muss der Sender dem Empfänger die genaue Phase mitteilen.
Am Anfang jeder Zeile wird daher eine etwa zehn Perioden langer ``PAL-Burst''
übermittelt, den der Empfänger dazu verwenden kann, die Phase des
Farbträgers zu bestimmen.

Zusätzlich invertiert das PAL-System die Phase des Farbträgers
aufeinanderfolgender Zeilen, so dass sich Farbfehler durch Phasenfehler
auf aufeinanderfolgenden Zeilen wegmitteln.
Im PAL-System steht also Farbinformation jeweils nur für Paare von Zeilen
zur Verfügung und nur mit einer Dichte, die durch die Frequenz des Farbträgers
begrenzt ist.
Die effektive Farbauflösung eines PAL-Farbfernsehbildes ist daher halb so
gross wie die Helligkeitsauflösung.
Da auch die Farbauflösung des menschlichen Auges kleiner ist als die
Helligkeitsauflösung, ist diese Einschränkung des Systems von Auge nicht 
erkennbar.

\subsubsection{FSK und PSK}
Für die digitale Signalübertragung braucht man minimal die Fähigkeit,
zwei Zustände zu übermitteln, die man aber exakt wiedererkennnen können muss.
Frequency-Shift-Keying (FSK) ist ein Verfahren, welches zwei digitale Zustände
durch verschiedene Frequenzen codiert, es ist also ein
Frequenzmodulationsverfahren, von dem im vorangegangenen Abschnitt
bereits gezeigt wurde, wie es mit der Quadratur-Amplituden-Modulation
realisierbar ist.

Phase-Shift-Keying (PSK) verwendet stattdessen eine Phasenverschiebung
des Tragersignals.
Eine Phasenverschiebung um den Winkel $\varphi$ kann realisiert werden,
indem man eine Drehung um den Winkel $\varphi$ vorschaltet, also die
Drehmatrix $D_{\varphi}$ einfügt.
Besonders einfach ist eine Phasenverschiebung um den Winkel
$\varphi=180^\circ$, 
\[
D_{\varphi}
=
\begin{pmatrix}
\cos180^\circ&          - \sin180^\circ \\
\sin180^\circ& \phantom{-}\cos180^\circ
\end{pmatrix}
=
-E.
\]
Diese Phasenverschiebung wird also realisiert dadurch, dass man das
Vorzeichen von $I$ und $Q$ ändert.
Verwendet man den Vektor $(1,0)^t$ zur Codierung einer logischen
$\texttt{0}$, dann codiert der Vektor $(-1,0)^t$ eine logische $\texttt{1}$.
Auch PSK ist also mit Quadratur-Amplituden-Modulation realisierbar.

\subsubsection{Quantisierte QAM}
Mit Quadratur-Amplituden-Modulation lässt sich ein beliebiger Vektor
in der $I$-$Q$-Ebene übertragen.
Bei PSK wurden nur die Punkte $(1,0)$  und $(-1,0)$ in der $I$-$Q$-Ebene
verwendet.
Nach der Demodulation erhält man Vektoren, die wegen Fehlern nicht
exakt mit den ursprünglichen Vektoren übereinstimmen.
Da man aber nur die beiden logischen Zustände unterscheiden können muss,
kann man alle Vektoren mit $I>0$ als logische \texttt{0} decodieren
und Vektoren mit $I<0$ als logische \texttt{1}.

Statt nur zwei Zustände \texttt{0} und \texttt{1} zu codieren, könnte man
ein grössere Zahl von Punkten in der $I$-$Q$-Ebene verwenden, wie in
Abbildung~\ref{figure:qam:konstellation} dargestellt.
Die Punkte werden auch {\em Symbole} genannt.
Ein empfangener Vektor wird wegen Übertragungsfehlern nicht exakt mit
dem ursprünglichen Vektor übereinstimmen.
Zur Decodierung suchen wir dasjenige Symbol, welches dem Vektor am
nächsten liegt.
Man teilt also die Ebene in Teilgebiete $T_{\vec{v}_k}\subset \mathbb R^2$
zu jedem Symbol $\vec{v}_k$ auf.
Fällt der empfangene Vektor $\hat{v}$ in das Teilgebiet des Symbols
$\vec{v}_k$, also $\hat{v}\in T_{\vec{v}_k}$, dann decodieren wir ihn
als das Symbol $\vec{v}_k$.

\begin{figure}
\centering
\includegraphics{applications/qam/images/konstellation.pdf}
\caption{Konstellationsdiagramm für quantisierte QAM mit 16 verschiedenen
Symbolen.
Mit jedem Symbol werden vier Bit codiert.
Zu jedem Symbol gehört ein quadratisches Gebiet gleicher Farbe.
Fällt der empfangene Vektor in eines dieser Gebiet, wird er als
das zugehörige Symbol decodiert.
\label{figure:qam:konstellation}}
\end{figure}

Im Beispiel der Abbildung~\ref{figure:qam:konstellation} können 16 
verschiedene Vektoren unterschieden werden, die man mit vierstelligen
Binärzahlen identifizieren kann.
Mit jedem Symbol werden also vier Bit übertragen.
Dieses Verfahren heisst auch 16-QAM und wird bei DVB-T verwendet.

Die Punkte-Menge $\vec{v}_k$ heisst auch die {\em Konstellation}
des Verfahrens.
Durch feinere Aufteilung können mehr Bits pro Symbol übertragen werden,
wie in Tabelle~\ref{table:qam:xqam} zusammengstellt.
Abbildung~\ref{figure:qam:analyzer} zeigt, wie sich die Messung eines 256-QAM 
Signals auf einem Vector Signal Analyzer darstellt.

\begin{table}
\centering
\begin{tabular}{rrcrl}
\hline
Bits&Symbole&Konstellation&Name&Anwendung\\
\hline
   2&      4& $2\times 2$ &   4-QAM&DVB-S       \\
   4&     16& $4\times 4$ &   8-QAM&V.29, DVB-T \\
   6&     64& $8\times 8$ &  16-QAM&DVB-C, DVB-T\\
   8&    256&$16\times 16$& 256-QAM&DVB-C       \\
  10&   1024&$32\times 32$&1024-QAM&            \\
  12&   4096&$64\times 64$&4096-QAM&DVB-C2, G.hn\\
\hline
\end{tabular}
\caption{Verschiedene Konstellationen für quantisierte QAM mit Anwendungen.
\label{table:qam:xqam}}
\end{table}

\begin{figure}
\centering
\includegraphics[width=1.0\hsize]{applications/qam/images/analyzer.png}
\caption{Messung des Konstellationsdiagramms eines 256-QAM Signals
mit einem Vector Signal Analyzer.
Man beachte die Beschriftung der Achsen mit \texttt{I} und \texttt{Q}.
(Ausschnitt aus dem Video \url{https://www.youtube.com/watch?v=uV3O3tpjmS8}
bei 26:36).
\label{figure:qam:analyzer}}
\end{figure}

\subsubsection{$n$-PSK}
Analog zum Vorgehen bei der quantisierten QAM kann auch PSK diskretisiert
werden.
Als Konstellationsdiagramm für $n$-PSK dienen $n$ Punkte auf einem Kreis,
die durch einen Winkel $2\pi/n$ getrennt sind.
In Abbildung~\ref{figure:qam:psk} ist das Konstellationsdiagramm für
$8$-PSK dargestellt.
\begin{figure}
\centering
\includegraphics{applications/qam/images/psk.pdf}
\caption{Konstellationsdiagramm für 8-PSK 
\label{figure:qam:psk}}
\end{figure}

\subsubsection{Software Defined Radio}
Die vorangegangenen Beispiele haben illustriert, dass die
Quadraturamplitudenmodulation jedes besprochene Modulationsverfahren
realisieren kann.
Es ist nur nötig, einen Sender zu bauen, der Inputs $I(t)$ und $Q(t)$
entgegennimmt, die Modulation mit der Matrix $D_{\omega t}$ vornimmt
und das resultierende Signal $s(t)$ aussendet.
Auf der Empfängerseite braucht man eine physikalische Realisierung
der Matrix $D_{\omega_r t}$ und des Tiefpasses, der die demodulierten
Signal $\hat{I}(t)$ und $\hat{Q}(t)$ ausgibt.
Die Decodierung zum Beispiel als amplitudenoduliertes Sprachsignal,
als frequenzmoduliertes Musiksignal oder als digitales 16-QAM-Signal
kann danach ausschliesslich in Software erfolgen.
Die Modulationsart eines solchen sogenannten {\em Software Defined Radio (SDR)}
wird also durch die Software definiert, welche die Signale $I(t)$ und $Q(t)$
erzeugt bzw.~die Signale $\hat{I}(t)$ und $\hat{Q}(t)$ analysiert.










