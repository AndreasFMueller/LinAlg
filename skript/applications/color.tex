%
% color.tex -- RGB Farben als Beispiel eines dreidimensionalen Vektorraumes
%
% (c) 2013 Prof Dr Andreas Mueller, Hochschule Rapperswi
%
\subsection{Farben}
\subsubsection{Farben als Vektorraum}
Die Netzhaut des menschlichen Auges enth"alt zwei Arten von Rezeptoren.
Die hochemfindlichen St"abchenzellen sind vor allem f"ur gr"unes Licht
empflindlich, werden aber nur benutzt, wenn nicht ausreichend Licht
zur Verf"ugung steht.
Sie kommen vor allem bei Dunkelheit zum Einsatz, es ist mit ihnen
nicht m"oglich, Farben zu unterscheiden, daher nehmen wir in der
Nacht alles als grau war.
Auch beim Blick durch ein Teleskop kann man eine ferne Galaxie nicht
farbig sehen, da nur die St"abchenzellen empfindlich genug sind, die
uns keinen Farbeindruck liefern k"onnen. Helligkeit l"asst sich durch
eine einzige Zahl charakterisieren.

Bei Tageslicht sind die St"abchenzellen
bereits geblendet, dann kommen die Zapfen-Zellen zum Einsatz.
Sie sind viel weniger empfindlich, daf"ur gibt es sie in drei Ausf"uhrungen,
die f"ur Rot, Gr"un oder Blau empfindlich sind.
Sie erm"oglichen unser Farbensehen.
Jede Farbinformation muss sich also durch die Intensit"at wiedergeben
lassen, die die drei Arten von Zapfenzellen bestimmen k"onnen.
Die Menge aller Farben ist daher ein dreidimensionaler Vektorraum
bestehend aus Vektoren
\[
c=\begin{pmatrix}r\\g\\b\end{pmatrix}
\in\mathbb R^3,
\]
wobei $r$ das Signal der roten Zapfen, $g$ das Signal der gr"unen Zapfen
und $b$ das Signal der blauen Zapfen bezeichnet.

\subsubsection{Luminanz}
Wie hell ist eine Farbe?
Von den St"abchenzellen erhalten wir keine
Farbinformation, sondern nur Helligkeitsinformation.
Dabei bekommt der gr"une Anteil des Lichtes offenbar das
gr"osste Gewicht, weil St"abchen dort am empfindlichsten sind.
F"ur rotes und blaues Licht sind die St"abchen nicht ganz unempfindlich,
die roten und blauen Komponeten tragen also auch zum Helligkeitseindruck
bei, wenn auch in geringerem Masse.

Bei gen"ugend Licht werden die Zapfen aktiv, die Farbinformation liefern.
Trotzdem ist es uns weiterhin m"oglich, verschiedene Helligkeit
zu unterscheiden.
Offenbar ist das Gehirn in der Lage, eine Abbildung zu konsturieren,
die aus einer Farbe eine Helligkeit ableitet:
\[
L\colon \begin{pmatrix}r\\g\\b\end{pmatrix}\mapsto l.
\]
Diese Abbildung erf"ullt einige nat"urliche Bedingungen:
\begin{enumerate}
\item Wenn man zwei verschiedene Farben mischt, dann ist die Helligkeit
die Summe der Helligkeiten der einzelnen Farben:
\[
L(c_1 + c_2)=L(c_1) + L(c_2).
\]
\item Wenn man in einer Farbe alle Farbkomponenten mit dem gleichen 
Faktor verst"arkt, dann wird die Helligkeit ebenfalls um diesen Faktor
verst"arkt:
\[
L(\lambda c)=\lambda L(c).
\]
\end{enumerate}
Die Helligkeit oder {\em Luminanz} ist also eine lineare Abbildung
aus dem dreidimensionalen Farbraum in den eindimensionalen Helligkeitsraum.

Die Abbildung $L$ wird zum Beispiel ben"otigt, wenn man aus einem Farbbild
eine scharz/weiss-Bild machen m"ochte.
Nicht alle Menschen sind gleich, und so verwenden 
verschiedene technische Aufnahmesysteme
auch verschiedene lineare Abbildungen.
\begin{align}
L&=0.2126 r+0.7152 g + 0.0722 b\tag{ITU-r}
\\
L&=0.299 r + 0.587 g + 0.114 b\tag{CCIR601}
\\
L&=0.33 r + 0.5 g + 0.16 b\notag
\\
L&=0.375 r + 0.5 g + 0.125 b\notag
\end{align}
Die letzten zwei Formeln stimmen nicht besonders gut mit der tats"achlichen
Empfindlichkeitsverteilung der St"abchenzellen "uberrein, aber sie lassen
sich extrem rasch berechnen, was f"ur die Hersteller von Video-Kameras 
interessant ist.

Wir weisen darauf hin, dass die oben aufgelisteten Luminanz-Funktionen
alle die Eigenschaft haben, dass die Summe der Koeffizienten $1$ ist.
Wir verwenden das als Definition:
\begin{definition} Eine {\em Luminanz-Funktion} $L$ ist eine lineare Abbildung
des Farbraums in den Helligkeitsraum der Form
\[
L(c)=\rho \cdot r + \gamma \cdot g + \beta \cdot b
\]
mit den Eigenschaften
\begin{enumerate}
\item alle Koeffizienten $\rho$, $\gamma$ und $\beta$ sind positiv,
\item die Summe der Koeffizienten ist $1$.
\end{enumerate}
\end{definition}
Die Luminanz ist ein gewichtetes Mittel der Farbkomponenten.

\subsubsection{Farbs"attigung}
In diesem Abschnitt gehen wir von einer festen Luminanzfunktion aus.
Die Menge der Farben, die gleich hell sind wie eine vorgegebene Farbe $c_1$,
ist 
\[
\{ c\in\mathbb R^3\,|\, L(c)=L(c_1)\}.
\]
Diese Menge ist eine Ebene im Farbraum.

Gibt es einen Grauton $c_0$, welcher gleich hell ist wie $c_1$?
Damit wir diese Farbe als Grauton wahrnehmen, m"ussen alle drei Farbkomponenten
gleich sein, wir schreiben f"ur diese drei identischen Komponenten $x$,
bekommen also
\[
c_0=\begin{pmatrix}x\\x\\x\end{pmatrix}.
\]
Die zugeh"orige Luminanz ist
\[
L(c_0)=\rho\cdot x + \gamma\cdot x + \beta \cdot x=
(\rho + \gamma +\beta)\cdot x=x
\]
Der zu einer Farbe $c$ geh"orige graue Farbe ist also
\[
\begin{pmatrix}L(c)\\L(c)\\L(c)\end{pmatrix}
=
L(c)\begin{pmatrix}1\\1\\1\end{pmatrix}
\]
Wir schreiben f"ur den Vektor aus lauter Einsen in Zukunft $w$ (f"ur weiss):
\[
w=\begin{pmatrix}1\\1\\1\end{pmatrix}, \qquad L(w)=1.
\]
Wir k"onnen also eine Farbe $c_1$ immer in zwei Komponenten zerlegen,
eine monochrome Komponente und eine Komponente parallel zur Ebene
$\{ c\,|\, L(c)=L(c_1)\}$:
\[
c_1 = \underbrace{L(c_1)w}_{\text{Helligkeit}} + \underbrace{(c_1 - L(c_1)w)}_{\text{Farbe}}.
\]
Die Farbinformation steckt ausschliesslich im zweiten Term, der erste
Term gibt die Helligkeit wieder.
Damit bekommen wir eine M"oglichkeit, bei gleichbleibender Helligkeit
die Intenit"at der Farben zu ver"andern. Schreiben wir
\[
c(t)=L(c_1)w+t(c_1-L(w))w,
\]
dann liefert der Wert $t=0$ eine Farbe, die ein Vielfaches von $w$ ist, also
ein Grauwert. F"ur $t=1$ erhalten wir die urspr"ungliche Farbe zur"uck.
Die Zwischenwerte $0 < t < 1$ liefern Farben mit zunehmender Farbintensit"at,
oder wie man sagt, {\em Farbs"attigung}. 
Wir sind sogar in der Lage, die Farbe noch intensiver zu machen, indem
wir $t>1$ w"ahlen.
In der Abbildung~\ref{color:saettigung}
ist ein Bild mit verschiedenen Werten von $t$ gezeigt.
\begin{figure}
\begin{center}
\begin{tabular}{cccc}
\includegraphics[width=0.22\hsize]{graphics/saettigung0.jpg}&%
\includegraphics[width=0.22\hsize]{graphics/saettigunghalb.jpg}&%
\includegraphics[width=0.22\hsize]{graphics/saettigung1.jpg}&%
\includegraphics[width=0.22\hsize]{graphics/saettigung2.jpg}\\
$t=0$&$t=\frac12$&$t=1$&$t=2$
\end{tabular}
\end{center}
\caption{S"attigung f"ur verschiedene Werte von $t$
\label{color:saettigung}}
\end{figure}

\subsubsection{Farbton}
Die Farben gleicher Helligkeit bilden eine Ebene. In der Ebene gibt es
den ausgezeichneten Punkt $L(c_1)w$, der dem gleich hellen Grauton
entspricht. Wir nennen diesen Punkt auch den {\tt Weisspunkt}.
Je weiter entfernt ein Punkt $c$ in der Ebene von diesem
Punkt ist, desto intensiver die Farben, desto gr"osser die Farbs"attigung.

Ein Punkt in der Ebene $H=\{c\,|\, L(c)=L(c_1)\}$ ist aber durch
den Abstand vom Weisspunkt alleine nicht bestimmt. Durch Wahl einer 
geeigneten Null-Richtung k"onnen wir in der Ebene $H$ ein
Polarkoordinaten-System w"ahlen. Der Winkel entspricht dabei dem
Farbton.
Die Menge aller vom menschlichen Auge wahrnehmbaren k"onnen daher
auf einem Kreis angeordnet werden, dem Farbkreis.

Damit haben wir ein alternatives Koordinatensystem f"ur den Farbraum
konstruiert mit den drei Koordinaten
\begin{center}
\begin{tabular}{l|l}
\hline
Hue&Farbton, Farbwinkel\\
Saturation&Farbs"attigung\\
Brightness&Luminanz\\
\hline
\end{tabular}
\end{center}
Der Vorteil dieses Koordinatensystems gegen"uber dem RGB-System ist,
dass sich Bildverarbeitungsoperationen wie ``Heller'' oder ``mehr Farbe''
darin leichter beschreiben lassen.
