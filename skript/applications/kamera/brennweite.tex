%
% brennweite.tex
%
% (c) 2018 Prof Dr Andreas Müller, Hochschule Rapperswil
%
\documentclass[tikz]{standalone}
\usepackage{times}
\usepackage{amsmath}
\usepackage{txfonts}
\usepackage[utf8]{inputenc}
\usepackage{graphics}
\usepackage{color}
\usetikzlibrary{arrows,intersections}
\begin{document}
\definecolor{darkgreen}{rgb}{0,0.6,0}
\begin{tikzpicture}[>=latex,thick]

\draw[->] (-4.5,0)--(4.5,0) coordinate[label=$z$];
\draw[->] (0,-3.0)--(0,4.5) coordinate[label={left:$x$}];

\draw[color=red,line width=0.4pt] (4,4)--(-2.5,-2.5);

\draw[color=blue,line width=1.1pt] (0,0)--(4,0);
\draw[color=blue,line width=1.1pt] (4,0)--(4,4);
\node[color=blue] at (2,0) [below] {$1\,[\text{m}]$};
\node[color=blue] at (4,2) [right] {$1\,[\text{m}]$};

\draw[color=darkgreen,line width=1.1pt] (0,0)--(-2.5,0);
\draw[color=darkgreen,line width=1.1pt] (-2.5,0)--(-2.5,-2.5);
\node[color=darkgreen] at (-1.25,0) [above] {$f$};
\node[color=darkgreen] at (-2.5,-1.25) [left] {$f$};

\end{tikzpicture}
\end{document}
