%
% homogenekoordinaten.tex
%
% (c) 2018 Prof Dr Andreas Müller, Hochschule Rapperswil
%
\documentclass[tikz,12pt]{standalone}
\usepackage{times}
\usepackage{amsmath}
\usepackage{txfonts}
\usepackage[utf8]{inputenc}
\usepackage{graphics}
\usepackage{color}
\usepackage{pifont}
\usetikzlibrary{arrows,intersections,math,calc}
\begin{document}

\def\punkt#1#2{
        \fill[color=white] #1 circle[radius=0.08];
        \draw[color=#2] #1 circle[radius=0.08];
}

\begin{tikzpicture}[>=latex,thick]


\draw[->,color=blue] (-4,2)--(7,2) coordinate[label=$x$];

\draw[->] (0,-3)--(0,5) coordinate[label={right:$x_2$}];
\draw[->] (-4,0)--(7,0) coordinate[label=$x_1$];

\begin{scope}
\clip (-4,-3) rectangle (7,5);
\draw[color=red] (-4.8,-4)--(7.2,6);
\def\step{0.4}
\foreach \t in {-7,-6,...,7}{
	\punkt{({2.4*\step*\t},{2*\step*\t})}{red}
}
\end{scope}

\punkt{(2.4,2)}{blue}
\node at (2.4,2) [below right] {${\color{red}x} = {\color{blue}\displaystyle\frac{x_1}{x_2}}$};

\end{tikzpicture}

\end{document}

