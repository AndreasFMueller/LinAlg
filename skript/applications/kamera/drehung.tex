%
% drehung.tex
%
% (c) 2018 Prof Dr Andreas Müller, Hochschule Rapperswil
%
\subsection{Kameraorientierung und Kamerazentrum\label{section:kameraorientierung}}
Bis jetzt war die Kamera im Nullpunkt des Koordinatensystems platziert
und die Achse der Kamera war parallel zur $z$-Achse, die $x$-Achse des Chips
horizontal.
Eine Kamera kann aber in jedem beliebigen Punkt $c$, dem Kamerazentrum,
des Raumes platziert sein und die Achsen der Kamera können beliebig gegenüber
dem Standard-Koordinatensystem verdreht sein.
Um das Abbildungsproblem zu lösen, müssen wir also die Weltpunkte erst
umrechnen in das Koordinatensystem der Kamera, bevor wir die Kameramatrix
darauf anwenden können.

\subsubsection{Kamerazentrum}
Die Kamera befindet sich im Punkt mit dem Ortsvektor $\vec{c}$ statt dem
Nullpunkt.
Statt die Kamera in den Nullpunkt zu verschieben, können wir auch die
Position des abzubildenden Weltpunktes $P$ um $\vec{c}$ korrigieren.
Für eine Kamera im Punkt $C$ sieht der Nullpunkt genau so aus wie
der Punkt $-\vec{c}$ für eine Kamera im Nullpunkt.
Wir müssen also die Translation $\vec{p}\mapsto \vec{p}-\vec{c}$ durchführen.
Leider ist dies keine lineare Abbildung.
Wir haben aber bereits bei der Diskussion der Kameramatrix $K_m$
gesehen, dass dies in homogenen Koordinaten viel einfacher geht.
Wir schreiben daher Weltpunkte mit inhomogenen Koordinaten
$(x,y,z)$ als Quadrupel von homogenen Koordinaten $(x,y,z,1)$.
Damit können wir die Translation schreiben als
\[
\vec{p} \mapsto \vec{p}-\vec{c} =
\begin{pmatrix}
1&0&0&-c_x\\
0&1&0&-c_y\\
0&0&1&-c_z
\end{pmatrix}
\begin{pmatrix}
x\\y\\z\\1
\end{pmatrix}
=
\begin{pmatrix}
& & &  \\
&E& &-\vec{c}\\
& & &  
\end{pmatrix}
\begin{pmatrix}
x\\y\\z\\1
\end{pmatrix}.
\]
Wir werden diese Matrix $\begin{pmatrix}E&-\vec{c}\end{pmatrix}$ schreiben.

\subsubsection{Drehung}
Die Kamera ist gegenüber dem Standardkoordintensystem verdreht.
Wir können die Orienterung zum Beispiel dadurch bescheiben, dass wir
die $x$-, $y$- und $z$-Achsen der Kamera als Einheitsvektoren im
Standardkoordinatensystem ausdrücken und als Spalten in die Matrix $R$
einfüllen.
Die Matrix $R$ bildet die Standardbasisvektoren auf die Basisvektoren
des Kamera-Koordinatensystems ab.
Wir brauchen die Umkehrabbildung $D=R^{-1}$, um einen Vektor im
Standardkoordinatensystem ins Kamerakoordinatensystem umzurechnen.

Wir können die Matrix $D$ auch wie folgt verstehen.
Als Input für die Kameramatrix müssen wir wissen, wie weit der Weltpunkt
$P$ von $C$ entfernt ist ist ($z$-Koordinate), wie stark er horizontal
von der Kameraachse abweicht ($x$-Koordinate) und wie stark er vertikal
von der Kameraachse abweicht ($y$-Koordinate).
Wenn die Achsen der Kamera parallel zu den Standardbasis-Vektoren sind,
kann man diese Abweichungen als Projektionen auf die Achsen verstehen,
die man mit Hilfe des Skalarproduktes des Ortsvektors von $P$ mit den
Standardbasisvektoren bekommen kann.
Das Skalarprodukt mit den Standardbasisvektoren kann man als Matrizenprodukt
schreiben, indem man die Standardbasisvektoren in die Zeilen einer Matrix
schreibt, so entsteht eine Einheitsmatrix.

Möchte man beliebige orthonormierte Vektoren $\vec{d}_1$ bis $\vec{d}_3$
anstelle der Standardbasisvektoren als Orientierung der Kamera verwenden,
dann muss man wieder das
Skalarprodukt der Strahlrichtung mit den Basisvektoren bilden.
Auch dies kann man als Matrizenprodukt schreiben, wenn man die Vektoren
in die Zeilen einer Matrix $D$ einfüllt.

Die transponierte Matrix $D^t$ der so gebildeten Matrix $D$ enthält als
Spalten die Vektoren $\vec{d}_1$ bis $\vec{d}_3$.
Sie ist also genau die oben konstruierte Matrix $R$: $R=D^t=D^{-1}$.

\subsubsection{Die Lösung des Abbildungsproblems}
Damit haben wir jetzt das Abbildungsproblem vollständig gelöst.

\begin{satz}
Eine Kamera mit Kameramatrix $K$ im Punkt $C$ mit Orientierung $D$ bildet
einen Weltpunkt $P$ auf den Bildpunkt
\[
\tilde b
=
K D \begin{pmatrix} E&-\vec{c}\end{pmatrix} \tilde p
\]
ab.
\end{satz}

Das umgekehrte Problem kann nun ebenfalls gelöst werden.
Ausgehend von einem Bildpunkt soll die Gerade bestimmt werden, auf der
sich der abgebildete Weltpunkt befindet.
Das Kamerazentrum ist natürlich ein Stützvektor für diese Gerade, es
braucht nur noch die Richtung bestimmt werden.
Die Richtung $\vec{r}$ wird von $D$ und $K$ auf den $\tilde b$ abgebildet.
Daraus liest man ab, dass die Richtung $r$ aus $\tilde b$
mit Hilfe von
\[
\vec{r} = (KD)^{-1} \tilde b
\]
gefunden werden kann.
Man beachte, dass die homogenen Koordinaten des Bildpunktes nur bis auf
einen skalaren Faktor bestimmt sind, dass also für ein beliebiges
$t\in\mathbb R$ mit $t\ne 0$ 
\[
\vec{r}' = (KD)^{-1} t\tilde b = t (KD)^{-1} \tilde b=t\vec{r}
\]
ein genauso gut geeigneter Richtungsvektor ist.
Die gesuchte Gerade hat daher die Parameterdarstellung
\[
\vec{p} = \vec{c} + t (KD)^{-1}\tilde b.
\]
Die inverse Matrix von $K$ kann einfach bestimmt werden, indem wir die
geometrischen Abbildungen rückgängig machen:
\[
K^{-1}
=
(K_mK_f)^{-1}
=
K_f^{-1}K_m^{-1}
=
\begin{pmatrix}
1/f& 0 & 0\\
0  &1/f& 0\\
0  & 0 & 1
\end{pmatrix}
\begin{pmatrix}
1&0&-m_x\\
0&1&-m_y\\
0&0&1
\end{pmatrix}
=
\begin{pmatrix}
1/f& 0 & -m_x/f\\
 0 &1/f& -m_y/f\\
 0 &   &  1
\end{pmatrix}.
\]
Die inverse Matrix von $D$ ist ebenfalls sehr einfach zu bestimmen.
Da $D$ eine orthogonale Matrix ist, ist $D^{-1}=D^t$.
Daraus kann man jetzt auch
\[
(KD)^{-1} = D^{-1}K^{-1}
\]
ausrechnen.

