%
% Anwendung: Google-Pagerank
%
% (c) 2012 Prof Dr Andreas Mueller
% $Id$
%
\subsection{Google Pagerank}
\index{Pagerank}
\index{Google}
\index{Suchbegriff}
\index{Suchmaschine}
Es gibt viel mehr Web-Seiten als es m"ogliche Suchw"orter gibt, jede Web-Suche
liefert notwendigerweise Tausende wenn nicht Millionen von Treffern.
Welche Seiten soll man dem Benutzer der Suchmaschine auf der ersten Seite 
anzeigen?

\subsubsection{Wahrscheinlichkeitsmodell des Internets}
\index{Internet}
\index{Link}
Googles Antwort auf dieses Problem ist: man fragt das Internet!
Ob eine Web-Seite interessant ist entscheiden die Internet-User dadurch,
dass sie auf ihren eigenen Seiten Links darauf setzen. Je mehr Links auf
eine Seite zeigen, desto bedeutender ist sie.
\begin{figure}
\[\UseTips
\entrymodifiers={++[o][F]}
\xymatrix @=1.5cm {
1 \ar[r] \ar[d]
	& 3 \ar[r] \ar[dl]
		& 5 \ar[d] \ar@/^/[r] \ar[rd]
			& 7 \ar@/^/[d]  \ar@/^/[l]
                                \ar `ul^l[lll]+/u6mm/`l^dl[lll] [lll]
\\
2 \ar@/^/[r]
	& 4 \ar@/^/[l] \ar[r] \ar[ur]
		& 6 \ar@/^/[r]
			& 8 \ar@/^/[l] \ar@/^/[u]
}
\]
\caption{Beispiel-Internet\label{google-sample}}
\end{figure}%

Die bedeutendste Seite ist diejenige, die von den meisten Benutzern
besucht wird. Nummerieren wir alle Internet-Seiten von $1$ bis $n$,
k"onnen wir die Besuchs\-h"aufigkeiten $p_i$ aller Seiten in einem
Vektor
\[
p=\begin{pmatrix}p_1\\\vdots\\p_n\end{pmatrix}
\]
zusammenfassen. In regelm"assigen Zeitabst"anden verlassen Benutzer
die Seiten, die sie gerade am Lesen sind, und klicken sich weiter zu
anderen Seiten. Die Benutzer verteilen sich neu, sie besuchen jetzt mit
H"aufigkeit $p_i'$ die Seite $i$.
Dabei folgen sie nat"urlich den Links.
Gibt es auf einer Seite $k$ Links, nehmen wir an, dass die Benutzer
zu gleichen Teilen diesen Links folgen.
Im Beispiel-Internet nach Abbildung~\ref{google-sample} gibt es drei
Links, die von Seite 5 abgehen.
Die Benutzer, die die Seite 5
besuchen, klicken sich weiter zu je einem Drittel auf die
Seiten 6, 7 und 8.
Daraus kann man jetzt die H"aufigkeit ausrechnen, mit der die
einzelnen Seiten nach dem Weiterklicken besucht werden.
Die Besucher, die nach dem Klicken auf Seite $8$ landen, rekrutieren
sich von den Seiten 5, 6 und 7:
\[
p_8'=\frac13p_5+p_6+\frac13p_7.
\]
Der Vektor $p'$ kann aus dem Vektor $p$ mit einer Matrix
\[
H=\begin{pmatrix}
%     1 2       3       4       5 6       7       8
      0&0&      0&      0&      0&0&\frac13&      0\\
\frac12&0&\frac12&\frac13&      0&0&      0&      0\\
\frac12&0&      0&      0&      0&0&      0&      0\\
      0&1&      0&      0&      0&0&      0&      0\\
      0&0&\frac12&\frac13&      0&0&\frac13&      0\\
      0&0&      0&\frac13&\frac13&0&      0&\frac12\\
      0&0&      0&      0&\frac13&0&      0&\frac12\\
      0&0&      0&      0&\frac13&1&\frac13&0\\
\end{pmatrix}
\]
nach der Formel $p'=Hp$ berechnet werden. Man beachte, dass sich die
Elemente in jeder Spalte von $H$ zu $1$ summieren.

Nat"urlich "andert sich die Beliebtheit einer Webseite viel langsamer
als die Klicks Zeit brauchen, also sollten sich die H"aufigkeiten
$p_i'$ von den H"aufigkeiten $p_i$ nicht unterscheiden:
\[
p=Hp.
\]
Insbesondere ist $p$ ein Eigenvektor zum Eigenwert $1$.

Im Beispiel-Internet findet man f"ur den Eigenvektor numerisch
\[
p=
\begin{pmatrix}
   0.077670\\
   0.087379\\
   0.038835\\
   0.087379\\
   0.247573\\
   0.150485\\
   0.233010\\
   0.077670\\
\end{pmatrix}.
\]
Seite  5 ist also die beliebteste, gefolgt von Seite 7 und 6.

\subsubsection{Freier Wille}
\index{freier Wille}
Das Modell, dem die Matrix $H$ zu Grunde liegt, ber"ucksichtigt
noch nicht, dass einige Benutzer eine Seite nicht dank der Links
finden, sondern weil sie von ihrem freien Willen gebrauch machen,
im URL-Feld des Browsers einen beliebigen URL einzugeben.
Ohne weitere Informationen m"ussen wir annehmen, dass auf diesem
Wege die Benutzer jede beliebige Seite anspringen k"onnen.
W"are dies der einzige Mechanismus, mit dem man andere Seiten erreichen
kann, dann w"are
\[
p'=\frac1n\begin{pmatrix}
1\\\vdots\\1
\end{pmatrix}.
\]
Wir m"ochten dies aber wieder in der Form $p'=Mp$ schreiben.
Da die $p_i$ sich zu $1$ summieren, gilt
\[
\begin{pmatrix}
     1&     1&\dots &     1\\
     1&     1&\dots &     1\\
\vdots&\vdots&\ddots&\vdots\\
     1&     1&\dots &     1
\end{pmatrix}
p=\begin{pmatrix}1\\1\\\vdots\\1\end{pmatrix}.
\]
Wir bezeichnen die Matrix aus lauter Einsen mit $A$, dann ist
\[
p'=\frac{1}{n}Ap.
\]

Wir f"ugen jetzt die Matrix $H$ mit dem freien Wille zusammen zur 
sogenannten Google-Matrix.
\begin{definition}
Die Matrix 
\[
G=(1-\alpha)H+\frac{\alpha}nA
\]
heisst Google-Matrix.
Der Eigenvektor zum Eigenwert $1$ heisst Google Pagerank.
\end{definition}
\index{Pagerank}
\index{Google-Matrix}
Je gr"osser der Wert $\alpha$ gew"ahlt wird, desto bedeutender wird die
Komponente des freien Willens. F"ur $\alpha=0$ ist $G=H$.

F"ur das Beispiel-Internet verschiebt sich die Beliebtheit jetzt ein bisschen,
f"ur $\alpha = 0.1$ zum Beispiel bekommen wir f"ur den Pagerank-Vektor
\[
p=\begin{pmatrix}
   0.077241\\
   0.099007\\
   0.047258\\
   0.101606\\
   0.233925\\
   0.147918\\
   0.215804\\
   0.077241
\end{pmatrix}.
\]
Die Seite 5 ist zwar immer noch die beliebteste, aber sie ist etwas zur"uckgefallen,
daf"ur ist jetzt Seite 7 etwas beliebter geworden.

\subsubsection{Bestimmung des Eigenvektors}
Die Google-Matrix ist nur dann praktisch anwendbar, wenn es eine
M"oglichkeit gibt, den Pagerank-Vektor auch dann zu berechnen, wenn $n$
sehr gross ist, also von der Gr"ossenordnung $10^{10}$.

Tats"achlich ist bekannt, dass jeder andere Eigenwert deutlich kleiner als $1$
ist.
Die Potenzmethode von Abschnitt \ref{section:dominant} ist anwendbar,
da der gesuchte Eigenvektor der Eigenvektor zum dominanten Eigenwert
ist.
Man muss also nur mit einer m"oglichst guten Approximation von $p_0$ beginnen,
und dann die Matrix $G$ mehrmals wiederholt darauf anwenden, 
\[
p_1=Gp_0,\quad
p_2=Gp_1,\quad
p_3=Gp_2,\quad\dots p_k=Gp_{k-1}=G^kp.
\]
Google verf"ugt nat"urlich immer "uber eine gute erste Approximation $p_0$:
den Pagerank-Vektor aus der letzten Berechnung.
Alle paar Tage oder Wochen berechnet Google den Pagerank-Vektor neu,
wobei sich die Reihenfolge der Suchresultate nat"urlich ver"andern
kann. Webmaster nennen dieses Ph"anomen den Google-Dance. 

