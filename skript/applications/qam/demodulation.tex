%
% demodulation.tex -- Demodulation von QAM
%
% (c) 2020 Prof Dr Andreas Müller, Hochschule Rapperswil
%
\subsection{Demodulation
\label{subsection:demodulation}}
Die modulierten Komponenten $s(t)$ und $c(t)$ entstehen gemäss
\eqref{eqn:qam:modulation}
durch eine sehr rasche Drehung $D_{\omega t}\vec{v}(t)$
des Vektor $\vec{v}(t)$.
Da die Drehung durch eine Matrix beschrieben wird, können wir
sie auch wieder rückgängig machen, indem wir mit der inversen
Matrix
\[
D^{-1}_{\omega t} = D_{-\omega t}
=
\begin{pmatrix}
\phantom{-}\cos\omega t & \sin\omega t \\
         - \sin\omega t & \cos\omega t
\end{pmatrix}
\]
multiplizieren.
So finden wir
\[
D_{\omega t}^{-1}
\begin{pmatrix}s(t)\\c(t)\end{pmatrix}
=
D_{\omega t}^{-1}
D_{\omega t}
\begin{pmatrix} I(t)\\Q(t)\end{pmatrix}
=
\begin{pmatrix}
I(t)\\
Q(t)
\end{pmatrix}.
\]
Es ist also klar, dass man aus $s(t)$ und $c(t)$ die ursprünglichen Signale
$I(t)$ und $Q(t)$ rekonstruieren kann.
Allerdings ist auch dies nicht wirklich eine Lösung des Problems.
Es ist immer noch notwendig, die beiden Funktionen $s(t)$ und $c(t)$
getrennt zu übertragen, um $I(t)$ und $Q(t)$ wiederzugewinnen.

Wir suchen ein Verfahren, mit dem wir $I(t)$ und $Q(t)$ allein aus
$s(t)$ zurückgewinnen können.
Auch für dieses Problem suchen wir eine geometrische Lösung.
Wir gehen dazu aus von der Gleichung
\[
\vec{v}(t)
=
D_{-\omega t}\underbrace{D_{\omega t}
\vec{v}(t)}_{=\begin{pmatrix}s(t)\\c(t)\end{pmatrix}}.
\]
Die Tatsache, dass wir $c(t)$ nicht übertragen wollen, können wir dadurch
abbilden, dass wir in der Gleichung eine Projektionsmatrix $P$
verwenden, um die Komponeten $c(t)$ zu unterdrücken:
\[
P=\begin{pmatrix}1&0\\0&0\end{pmatrix}
\qquad\Rightarrow\qquad
P\begin{pmatrix}s(t)\\c(t)\end{pmatrix}
=
\begin{pmatrix}s(t)\\0\end{pmatrix}.
\]
Durch diese Änderung wird man natürlich nicht mehr $I(t)$ und $Q(t)$ 
zurückgewinnen können, stattdessen man wird modifizierte Funktionen
$\hat{I}(t)$ und $\hat{Q}(t)$ erhalten.
Das ganze Übertragung\-system könnte daher mit dem Matrizenprodukt
\[
\begin{pmatrix}
\hat{I}(t)\\
\hat{Q}(t)
\end{pmatrix}
=
D_{-\omega t} \begin{pmatrix}s(t)\\0\end{pmatrix}
=
D_{-\omega t} P D_{\omega t}\vec{v}(t)
\]
Wegen
\[
D_{-\omega t}P
=
\begin{pmatrix}
\phantom{-}\cos\omega t & 0 \\
         - \sin\omega t & 0
\end{pmatrix}
\]
bedeutet das
\begin{align*}
\hat{I}(t) &= \phantom{-}\cos\omega t s(t),\\
\hat{Q}(t) &= -\sin\omega t s(t).
\end{align*}
Wir bezeichnen den Vektoren mit diesen Komponenten als
\[
\hat{v}(t) = \begin{pmatrix}\hat{I}(t)\\\hat{Q}(t)\end{pmatrix}.
\]
Der Vektor ist also, was von der Rekonstruktion nach dem Wegfallen
der Komponente $c(t)$ noch übrig bleibt.

Wie unterschieden sich $\hat{I}(t)$ und $\hat{Q}(t)$ von $I(t)$ und $Q(t)$?
Dazu berechnen wir
\begin{align*}
D_{-\omega t}PD_{\omega t}
&=
\begin{pmatrix}
\phantom{-}\cos\omega t & \sin\omega t \\
         - \sin\omega t & \cos\omega t
\end{pmatrix}
\begin{pmatrix} 1 & 0 \\ 0 & 0 \end{pmatrix}
\begin{pmatrix}
\cos\omega t &          - \sin\omega t \\
\sin\omega t & \phantom{-}\cos\omega t
\end{pmatrix}
\\
&=
\begin{pmatrix}
 \cos^2\omega t           & -\cos\omega t \sin\omega t \\
-\cos\omega t \sin\omega t&\sin^2\omega t
\end{pmatrix}
=
\frac12
\begin{pmatrix}
1+\cos 2\omega t &  -\sin 2\omega t \\
 -\sin 2\omega t & 1-\cos 2\omega t
\end{pmatrix}
\\
&=
\frac12 E + \frac12
\begin{pmatrix}
 \cos 2\omega t & -\sin 2\omega t \\
-\sin 2\omega t & -\cos 2\omega t
\end{pmatrix}.
\end{align*}
Nach der Rekonstruktion bleiben also zwei Komponenten
\[
\hat{v}(t)
=
\frac12\vec{v}(t)
+
\frac12
\begin{pmatrix}
\phantom{-}\cos 2\omega t & -\sin 2\omega t \\
         - \sin 2\omega t & -\cos 2\omega t
\end{pmatrix}\vec{v}(t).
\]
Die erste Komponente ist bis auf den Faktor $\frac12$ der gesuchte
Vektor $\vec{v}(t)$.
Doch was ist die zweite Komponente?
Die Matrix kann man auch schreiben als
\begin{align*}
\begin{pmatrix}
\phantom{-}\cos2\omega t&-\sin2\omega t\\
         - \sin2\omega t&-\cos2\omega t
\end{pmatrix}
&=
\underbrace{
\begin{pmatrix}
1& 0\\
0&-1
\end{pmatrix}}_{\displaystyle = S}
\begin{pmatrix}
\cos2\omega t &          - \sin2\omega t \\
\sin2\omega t & \phantom{-}\cos2\omega t
\end{pmatrix}
=
\begin{pmatrix}
1& 0\\
0&-1
\end{pmatrix}
D_{2\omega t},
\end{align*}
bis auf die Spiegelungsmatrix $S$ handelt es sich also wieder um eine
Drehung.
Zusammen finden wir
\begin{equation}
\hat{v}(t)
=
\frac12 \vec{v}(t)
+
\frac12 SD_{2\omega t}\vec{v}(t).
\label{eqn:qam:filter}
\end{equation}

Gehen wir davon aus, dass die Bewegung des Vektors $\vec{v}(t)$ in
der $I$-$Q$-Ebene sehr viel langsamer ist als die Drehung mit der
Winkelgeschwindigkeit $2\omega$, dann können wir den zweiten Term
näherungsweise eliminieren.
Um dies zu verstehen, nehmen wir an, dass $\vec{v}(t)$ während eines
Zeitintervalls der Länge $L=\pi/\omega$ konstant ist ist.
Wir bezeichnen Mittelwerte über das Intervall mit dem Buchstaben $M$.
Wir beachten dann, dass der zweite Term in \eqref{eqn:qam:filter}
für dieses Zeitintervall die gleichförmige Drehung des Vektors
um den Nullpunkt beschreibt.
Der Mittelwert der zweiten Komponente über das Intervall verschwindet daher:
\[
M\frac12SD_{2\omega t}\vec{v}(t)
=
0
\]
Die erste Komponente von \eqref{eqn:qam:filter} ist während des Intervalls
konstant, ihr Mittelwert ist daher
\[
M\frac12\vec{v}(t)
=
\frac12\vec{v}(t)
\]
Wir finden daher den Mittelwert
\[
M\hat{v}(t)
=
\frac12\vec{v}(t),
\]
bis auf den Faktor $\frac12$ also $\vec{v}(t)$ rekonstruiert.
Verändert sich $\vec{v}(t)$ während des Intervalls um einen Betrag
kleiner als $\varepsilon$, dann kommt ein Fehler hinzu, der ebenfalls
von der Grössenordnung $\varepsilon$ ist.

In praktischen Anwendungen ist die Frequenz des Trägers mehrere
Grössenordnungen grösser als die typischen Frequenzen in $I(t)$ 
und $Q(t)$, die Annahme, dass sich $\vec{v}(t)$ während einer halben
Trägerperiode nicht ändert, ist daher mit grosser Genauigkeit erfüllt.
Die Mittelwertbildung wird technisch mit Hilfe eines Tiefpassfilters
realisiert.

