%
% konstellation.tex -- Konstallationsdiagramm für 16-QAM
%
% (c) 2018 Prof Dr Andreas Müller, Hochschule Rapperswil
%
\documentclass[tikz,12pt]{standalone}
\usepackage{times}
\usepackage{amsmath}
\usepackage{txfonts}
\usepackage[utf8]{inputenc}
\usepackage{graphics}
\usepackage{color}
\usepackage{pifont}
\usetikzlibrary{arrows,intersections,math,calc}
\begin{document}

\def\s{0.07}
\def\a{16}
\def\b{32}

\def\drawblock#1#2#3#4{
	\pgfmathparse{#1}
	\xdef\xbegin{\pgfmathresult}
	\pgfmathparse{#2}
	\xdef\xend{\pgfmathresult}
	\pgfmathparse{#3}
	\xdef\ybegin{\pgfmathresult}
	\pgfmathparse{#4}
	\xdef\yend{\pgfmathresult}
	\foreach \x in {\xbegin,...,\xend}{
		\foreach \y in {\ybegin,...,\yend}{
			\fill[color=red] ({\s*(0.5+\x)},{\s*(0.5+\y)})
				circle[radius=0.02];
		}
	}
}

\def\crossconstellation#1#2{
	\fill[color=white,opacity=0.7]
		({-#2*\s},{-#2*\s}) rectangle ({#2*\s},{#2*\s});
	\drawblock{-#1}{#1-1}{-#2}{#2-1}
	\drawblock{-#2}{-#1-1}{-#1}{#1-1}
	\drawblock{#1}{#2-1}{-#1}{#1-1}
}
\def\constellation#1{
	\fill[color=white,opacity=0.7]
		({-#1*\s},{-#1*\s}) rectangle ({#1*\s},{#1*\s});
	\drawblock{-#1}{#1-1}{-#1}{#1-1}
}

\begin{tikzpicture}[>=latex,thick]

\begin{scope}[xshift=-5.2cm]
\draw[->] ({-\b*\s-0.1},0) -- ({\b*\s+0.3},0) coordinate[label={$I$}];
\draw[->] (0,{-\b*\s-0.1}) -- (0,{\b*\s+0.3}) coordinate[label={right:$Q$}];
\constellation{\a}
\node at (0,{-\b*\s-0.5}) {$2^{10}$-QAM};
\end{scope}

\draw[->] ({-\b*\s-0.1},0) -- ({\b*\s+0.3},0) coordinate[label={$I$}];
\draw[->] (0,{-\b*\s-0.1}) -- (0,{\b*\s+0.3}) coordinate[label={right:$Q$}];
\crossconstellation{\a}{(1.5*\a)}
\node at (0,{-\b*\s-0.5}) {$2^{11}$-QAM};

\begin{scope}[xshift=5.2cm]
\draw[->] ({-\b*\s-0.1},0) -- ({\b*\s+0.3},0) coordinate[label={$I$}];
\draw[->] (0,{-\b*\s-0.1}) -- (0,{\b*\s+0.3}) coordinate[label={right:$Q$}];
\constellation{\b}
\node at (0,{-\b*\s-0.5}) {$2^{12}$-QAM};
\end{scope}

\end{tikzpicture}

\end{document}

