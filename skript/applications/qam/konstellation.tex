%
% konstellation.tex -- Konstallationsdiagramm für 16-QAM
%
% (c) 2018 Prof Dr Andreas Müller, Hochschule Rapperswil
%
\documentclass[tikz,12pt]{standalone}
\usepackage{times}
\usepackage{amsmath}
\usepackage{txfonts}
\usepackage[utf8]{inputenc}
\usepackage{graphics}
\usepackage{color}
\usepackage{pifont}
\usetikzlibrary{arrows,intersections,math,calc}
\begin{document}

\def\s{1.8}

\def\kreis#1#2#3#4{
	\fill[color=farbe!#3] ({(#1)*\s/2},{(#2)*\s/2})
			circle[radius={#4*\s}];
}


\def\symbol#1#2#3{
	\pgfmathparse{(#1+3)/6+(#2+3)/6}
	\xdef\r{\pgfmathresult}
	\pgfmathparse{0.4+0.4*(3-#2)/6}
	\xdef\g{\pgfmathresult}
	\pgfmathparse{(3-#1)/6}
	\xdef\b{\pgfmathresult}
	\definecolor{farbe}{rgb}{\r,\g,\b}
	\begin{scope}
		\clip
		({(#1-1)*\s/2},{(#2-1)*\s/2})
		rectangle
		({(#1+1)*\s/2},{(#2+1)*\s/2});
		\kreis{#1}{#2}{05}{0.75}
		\kreis{#1}{#2}{10}{0.70}
		\kreis{#1}{#2}{15}{0.65}
		\kreis{#1}{#2}{20}{0.60}
		\kreis{#1}{#2}{25}{0.55}
		\kreis{#1}{#2}{30}{0.50}
		\kreis{#1}{#2}{35}{0.45}
		\kreis{#1}{#2}{40}{0.40}
		\kreis{#1}{#2}{45}{0.35}
		\kreis{#1}{#2}{50}{0.30}
		\kreis{#1}{#2}{55}{0.25}
		\kreis{#1}{#2}{60}{0.20}
		\kreis{#1}{#2}{65}{0.15}
		\kreis{#1}{#2}{70}{0.10}
		\kreis{#1}{#2}{75}{0.05}
		\kreis{#1}{#2}{80}{0.05}
		\kreis{#1}{#2}{85}{0.05}
		\kreis{#1}{#2}{90}{0.05}
		\kreis{#1}{#2}{100}{0.05}
	\end{scope}
	\node at ({(#1)*\s/2},{(#2)*\s/2}) [above] {\tiny\texttt{#3}};
%	\fill ({\s*(#1)/2},{\s*(#2)/2}) circle[radius=0.05];
	\draw ({\s*(#1)/2},{\s*(#2)/2}) circle[radius=0.08];
}

\begin{tikzpicture}[>=latex,thick]

\symbol{-3}{-3}{1111}
\symbol{-3}{-1}{1101}
\symbol{-3}{+1}{1010}
\symbol{-3}{+3}{1011}

\symbol{-1}{-3}{1110}
\symbol{-1}{-1}{1100}
\symbol{-1}{+1}{1000}
\symbol{-1}{+3}{1001}

\symbol{+1}{-3}{0101}
\symbol{+1}{-1}{0100}
\symbol{+1}{+1}{0000}
\symbol{+1}{+3}{0010}

\symbol{+3}{-3}{0111}
\symbol{+3}{-1}{0110}
\symbol{+3}{+1}{0001}
\symbol{+3}{+3}{0011}

\draw[->] ({-2*\s-0.1},0) -- ({2*\s+0.3},0) coordinate[label={$I$}];
\draw[->] (0,{-2*\s-0.1}) -- (0,{2*\s+0.3}) coordinate[label={right:$Q$}];

\end{tikzpicture}

\end{document}

