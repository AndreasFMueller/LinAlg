%
% modulation.tex
%
% (c) 2020 Prof Dr Andreas Müller, Hochschule Rapperswil
%
\subsection{Modulation zweidimensionaler Signale
\label{subsection:modulation}}
Wir streben jetzt an, ein zweidimensionales Signal
\[
\vec{v}(t)
=
\begin{pmatrix}I(t)\\Q(t)\end{pmatrix}
\]
drahtlos zu übertragen und müssen zu diesem Zweck ein geeignetes
Modulationsverfahren finden.

Mit nur dem einen Signal $I(t)$ würden wir wie vorher $I(t)\cos\omega t$
wählen.
Für das zweite Signal $Q(t)$ könnten wir statt des $\cos\omega t$ die
Funktion $-\sin\omega t$ wählen\footnote{Das Vorzeichen ist hier so 
gewählt, dass die Formeln, die später herauskommen werden, mit den
üblichen Konventionen in der Theorie der Quadraturamplitudenmodulation
übereinstimmen.}, also das modulierte Signal
$-Q(t)\sin\omega t$ verwenden.
Natürlich können wir aus den beiden Funktionen $I(t)\cos\omega t$ und
$-Q(t)\sin\omega t$ auch die Funktionen $I(t)$ und $Q(t)$ zurückgewinnen,
das reicht aber nicht.
Dazu brauchen wir nämlich $I(t)\cos\omega t$ und $-Q(t)\sin\omega t$
unabhängig voneinander, wir brauchen also zwei unabhängig Übertragungskanäle
für die beiden Signale, was wir vermeiden wollen.

\begin{figure}
\centering
\includegraphics{applications/qam/sep.pdf}
\caption{Rekonstruktion der Signal $I(t)$ (oben) und $Q(t)$ (Mitte)
aus der Summe $s(t) = I(t)\cos\omega t - Q(t)\sin\omega t$ (unten).
Die Nullstellen von $\cos\omega t$ sind durch feine blaue Linien 
dargestellt, die Nullstellen von $\sin\omega t$ durch feine rote Linien.
Der Graph von $s(t)$ ist jeweils mit der Farbe eingefärbt, die den
dominanten Beitrag repräsentiert.
Blaue Segmente im Graphen von $s(t)$ bedeuten, dass vor allem der Wert
von $Q(t)$ zum Wert beiträgt, dies geschieht in der Umgebung von
Nullstellen von $\cos\omega t$, in roten Segmenten ist es der Wert von
$I(t)$ in der Nähe von Nullstellen von $\sin\omega t$.
Fette Punkte auf dem Graphen von $s(t)$ markieren Punkte bei den
genannten Nullstellen.
Die leeren Punkte sind Werte von $s(t)$, die um das Vorzeichen des
Trägeres korrigiert wurden, sie liegen genau auf dem Graphen der
ursprünglichen Signale $I(t)$ und $Q(t)$.
Die untersten zwei Graphen zeigen die rekonstruierten Signale
$\hat{I}(t)=s(t) \cos\omega t$ und $\hat{Q}(t) = -s(t) \sin\omega t$, 
welche in Abschnitt~\ref{subsection:demodulation} erklärt werden.
\label{figure:qam:sep}}
\end{figure}

Einzelne Werte der Funktionen $I(t)$ und $Q(t)$ können aber aus der Summe
\[
s(t)
=
I(t)\cos\omega t - Q(t)\sin\omega t
\]
rekonstruiert werden.
An den Stellen $t = k\pi/\omega$ für $k\in\mathbb Z$ verschwindet
der Faktor $\sin\omega t$,
so dass an diesen Stellen der zweite Summand in $s(t)$ wegfällt.
Ebenso wird der erste Summand an den Stellen
$t = (k+\frac12)\pi/\omega$ verschwinden.
Dieser Sachverhalt ist in Abbildung~\ref{figure:qam:sep} dargestellt.
Es ist also
\[
s\biggl(k\cdot \frac{\pi}{2\omega}\biggr)
=
\begin{cases}
I(k\cdot \frac{\pi}{2\omega})\cdot(-1)^{\frac{k}2}
&\qquad \text{$k$ gerade,}\\[5pt]
Q(k\cdot \frac{\pi}{2\omega})\cdot(-1)^{\frac{k-1}2}
&\qquad \text{$k$ ungerade.}
\end{cases}
\]
Zu Zeitpunkten, die Vielfache von $\pi/2\omega$ sind, kann man also aus
$s(t)$ die Werte von $I(t)$ und $Q(t)$ ablesen.
Tatsächlich lernt man im Fach {\em Signale und Systeme}, dass man daraus
die Funktionen $I(t)$ und $Q(t)$ rekonstruieren kann, wenn sie keine
Frequenzkomponenten kleiner als die Trägerfrequenz haben.
Die Summe $s(t)$ ist also etwas, was man potentiell drahtlos übermitteln
kann, und woraus man die Komponenten $I(t)$ und $Q(t)$ zurückgewinnen kann.
Man nennt dieses Modulationsverfahren {\em Quadratur-Amplituden-Modulation}.

Da wir über das nötige signaltheoretische Wissen noch nicht verfügen,
müssen wir eine alternative, geometrische Methode suchen, wie wir
aus $s(t)$ die Komponenten $I(t)$ und $Q(t)$ wiedergewinnen zu können.
Dazu beachten wir zunächst, dass wir die Summe $s(t)$ als ein
Matrizenprodukt hätten bekommen können:
\[
s(t)
=
\begin{pmatrix}\cos\omega t&-\sin\omega t\end{pmatrix}
\begin{pmatrix}I(t)\\Q(t)\end{pmatrix}.
\]
Die Zeilenmatrix sieht aus wie die erste Ziele einer Drehmatrix
mit dem Drehwinkel $\omega t$.
Die Funktion $s(t)$ ist also nur die erste Komponenten eines
Vektors, der durch Drehung des Vektors $\vec{v}(t)$ um den
Winkel $\omega t$ gemäss
\begin{equation}
\begin{pmatrix} s(t)\\ c(t) \end{pmatrix}
=
\underbrace{
\begin{pmatrix}
\cos\omega t & -\sin\omega t\\
\sin\omega t & \phantom{-}\cos\omega t
\end{pmatrix}}_{\displaystyle D_{\omega t}}
\begin{pmatrix}I(t)\\Q(t)\end{pmatrix}
=
D_{\omega t} \vec{v}(t)
\label{eqn:qam:modulation}
\end{equation}
entsteht.
Das modulierte Signal $s(t)$ ist also nichts anderes als eine
Komponente eines Vektors der entsteht, indem $\vec{v}(t)$ mit sehr grosser
Winkelgeschwindigkeit $\omega$ um den Ursprung gedreht wird.
Wir sind aber insofern nicht weiter, dass wir $I(t)$ und $Q(t)$ noch nicht
rekonstruieren können.



