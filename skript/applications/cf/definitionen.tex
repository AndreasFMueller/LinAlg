%
% definitionen.tex
%
% (c) 2020 Prof Dr Andreas Müller, Hochschule Rapperswil
%
\subsection{Definitionen \label{subsection:cf:definition}}
Ein {\em regulärer} Kettenbruch ist ein Kettenbruch der Form
\index{Kettenbruch!regulär}%
\eqref{eqn:cf:kettenbruch}, in dem alle Teilzähler $b_k=1$ sind.
Ein solcher Kettenbruch kann etwas kompakter als
\[
[a_0;a_1,a_2,a_3]
=
a_0+\cfrac{1}{
a_1+\cfrac{1}{
a_2+\cfrac{1}{
a_3}}}
\]
geschrieben werden.

Ein unendlicher Kettenbruch ist ein Kettenbruch, der nicht aufhört.
Ein unendlicher regulärer Kettenbruch kann als
\[
[a_0;a_1,a_2,a_3,\dots]
=
a_0+\cfrac{1}{
a_1+\cfrac{1}{
a_2+\cfrac{1}{
a_3+\cfrac{1}{\dots}}}}
\]
geschrieben werden.
Eine rationale Zahl hat immer eine endliche Kettenbruchentwicklung,
irrationale Zahlen dagegen haben eine unendliche Kettenbruchentwicklung.
Kettenbruchentwicklungen können daher verwendet werden, 
um zu untersuchen, ob eine Zahl rational oder irrational ist.

Bricht man einen unendlichen oder unendlichen Kettenbruch früher ab,
erhält man einen Näherungsbruch.
Wenn nur die ersten $n$ Teilzähler und -nenner verwendet werden
sprechen wir vom $n$-ten Näherungsbruch
\[
\frac{p_n}{q_n}
=
a_0+\cfrac{b_1}{
a_1+\cfrac{b_2}{
a_2+\cfrac{b_3}{
a_3+\cfrac{b_4}{
\dots+\cfrac{b_{n-1}}{
a_{n-1}+\cfrac{b_n}{a_n}}}}}}
\]
Die Näherungsbrüche eines regulären Kettenbruchs kann man auch
\[
\frac{p_n}{q_n}
=
[a_0;a_1,a_2,\dots,a_{n-1},a_n]
\]
schreiben.
Der Wert eines unendlichen Kettenbruchs ist der Grenzwert 
\[
[a_0;a_1,a_2,a_3,\dots]
=
\lim_{n\to\infty} \frac{p_n}{q_n}.
\]

Ein regulärer Kettenbruch heisst periodisch mit Periode $k$,
wenn es eine Zahl $k$ gibt derart, dass $a_i=a_{i+k}$ für $i>N$.
Man kann zeigen, dass der Wert eines periodischen Kettenbruchs 
immer Nullstelle eines quadratischen Polynoms mit rationalen
Koeffizienten ist, eine sogenannte quadratische Irrationalität.

