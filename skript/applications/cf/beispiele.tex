%
% beispiele.tex
%
% (c) 2020 Prof Dr Andreas Müller, Hochschule Rapperswil
%
\subsection{Beispiele \label{subsection:cf:beispiele}}
Um zu Beispielen von Kettenbrüchen zu kommen, müssen wir untersuchen,
wie wir die einzelnen Zahlen $a_k$ in einem regulären Kettenbruch
für die Zahl  $x$ kommen können.
Wir schreiben 
\begin{align*}
x=
x_0
&=
a_0+\cfrac{1}{
a_1+\cfrac{1}{
a_2+\cfrac{1}{
a_3+\cfrac{1}{
a_4+\cfrac{1}{\dots}}}}} = a_0 + \frac{1}{x_1}
\\
x_1
&=
a_1+\cfrac{1}{
a_2+\cfrac{1}{
a_3+\cfrac{1}{
a_4+\cfrac{1}{
a_5+\cfrac{1}{\dots}}}}} = a_0 + \frac{1}{x_2}
\\
&\phantom{0}\vdots
\\
x_k
&=
a_k+\cfrac{1}{
a_{k+1}+\cfrac{1}{
a_{k+2}+\cfrac{1}{
a_{k+3}+\cfrac{1}{
a_{k+4}+\cfrac{1}{\dots}}}}} = a_k + \frac{1}{x_{k+1}}.
\end{align*}
Dabei ist $a_k$ der ganzzahlige Anteil von $x_k$, $x_k - a_k$ ist
eine Zahl zwischen $0$ und $1$.
Wir lesen daraus die Rekursionsformel
\[
x_{k+1} = \frac{1}{x_k-a_k},
\qquad
a_k = \lfloor x_k\rfloor
\]
ab.
Die Kettenbruchentwicklung bricht ab in dem Moment, wo $x_k$ eine ganze Zahl
ist, dann ist auch $a_k=x_k$.
Die Teilnenner $a_k$ können also ermittelt werden, indem wir das
Rechenschema
\[
\begin{aligned}
x=x_0&                   &&\Rightarrow & a_0&=\lfloor x_0\rfloor \\
  x_1&=\frac{1}{x_0-a_0} &&\Rightarrow & a_1&=\lfloor x_1\rfloor \\
  x_2&=\frac{1}{x_1-a_1} &&\Rightarrow & a_2&=\lfloor x_2\rfloor \\
  x_3&=\frac{1}{x_2-a_2} &&\Rightarrow & a_3&=\lfloor x_3\rfloor \\
     &\phantom{1}\vdots  &&            &    &\phantom{1}\vdots
\end{aligned}
\]
ausfüllen.

\begin{beispiel}
Der Bruch $x=\frac{314}{100}$ hat die Kettenbruchentwicklung
\[
\begin{aligned}
x=x_0&=3.14                                           &&\Rightarrow& a_0 &= 3 \\
  x_1&=\frac{1}{x_0-a_0}=\frac{100}{14}=7+\frac{2}{14}&&\Rightarrow& a_1 &= 7 \\
  x_2&=7                                              &&\Rightarrow& a_2 &= 7
\end{aligned}
\]
An dieser Stelle bricht die Kettenbruchentwicklung ab weil $x_2=a_2$.
Wir schliessen, dass
\[
3.14 
=
3+\cfrac{1}{
7+\cfrac{1}{7}}
\]
ist.
\end{beispiel}

\begin{beispiel}
In diesem Beispiel soll 
ein regulärer Kettenbruch für die Zahl $x=1.\overline{234}$ gefunden werden.
Wir füllen das Rechenschema aus:
\[
\begin{aligned}
x_0&=1.\overline{234}                         &&\Rightarrow& a_0&= 1 \\
x_1&=\frac{1}{x_0-a_0} = 4.2\overline{692307} &&\Rightarrow& a_1&= 4 \\
x_2&=\frac{1}{x_1-a_1} = 3.\overline{714285}  &&\Rightarrow& a_2&= 3 \\
x_3&=\frac{1}{x_2-a_2} = 1.4                  &&\Rightarrow& a_3&= 1 \\
x_4&=\frac{1}{x_3-a_3} = 2.5                  &&\Rightarrow& a_4&= 2 \\
x_5&=\frac{1}{x_4-a_4} = 2                    &&\Rightarrow& a_5&= 2 \\
\end{aligned}
\]
Damit haben wir die Kettenbruchdarstellung 
\[
x=1+\cfrac{1}{
4+\cfrac{1}{
3+\cfrac{1}{
1+\cfrac{1}{
2+\cfrac{1}{
2}}}}}
\]
gefunden.
\end{beispiel}

\begin{beispiel}
In diesem Beispiel soll die Kettenbruchentwicklung der Zahl
$x=\sqrt{2}$ gefunden werden.
Da $\sqrt{2}$ nicht rational ist, kann der zugehörige Kettenbruch
nicht nach endlich vielen Schritten abbrechen.
Wir verwenden den gleichen Prozess wie in der vorangegangenen Aufgabe:
\[
\begin{aligned}
x_0&=\sqrt{2}                                         &&\Rightarrow& a_0 &= 1 \\
x_1&=\frac{1}{x_0-a_0}=\frac{1}{\sqrt{2}-1}=\sqrt{2}+1&&\Rightarrow& a_1 &= 2 \\
x_2&=\frac{1}{x_1-a_1}=\frac{1}{\sqrt{2}-1}=\sqrt{2}+1&&\Rightarrow& a_2 &= 2 \\
   &\phantom{0}\vdots                                 &&           &     &\phantom{0}\vdots
\end{aligned}
\]
Daraus kann man bereits ablesen, dass sich die Kettenbruchentwicklung
wiederholen wird und dass $a_k=2$ für alle $k>0$ sein wird.
Der gesuchte Kettenbruch für $\sqrt{2}$ ist also
\[
\sqrt{2}
=
1+\cfrac{1}{
2+\cfrac{1}{
2+\cfrac{1}{
2+\cfrac{1}{
2+\cfrac{1}{\dots}}}}}
=
[1;\overline{2}].
\]
\end{beispiel}

\begin{beispiel}
Gegeben ist der Kettenbruch
\[
x=1+\cfrac{1}{
1+\cfrac{1}{
1+\cfrac{1}{
1+\cfrac{1}{\dots}}}}
=[1;\overline{1}],
\]
wie gross ist $x$?
Der Nenner des ersten Bruches ist wieder der ganze Kettenbruch, es gilt
also die Gleichung
\[
x = 1+\frac{1}{x}
\qquad\Rightarrow\qquad
x^2-x-1=0.
\]
Diese quadratische Gleichung hat die Nullstellen
\[
x = \frac{1\pm\sqrt{1+4}}2=\frac{1\pm\sqrt{5}}2.
\]
Das negative Zeichen führt auf einen negativen Wert für $x$, der Wert
des Kettenbruches muss aber ganz offensichtlich positiv sein.
Damit bleibt
\[
x=\frac{1+\sqrt{5}}2
\approx
1.61803398874989\ldots=\varphi,
\]
das Verhältniss des goldenen Schnittes.
\end{beispiel}

Man kann zeigen, dass jede quadratische Irrationalität, also jede Wurzel
einer quadratischen Gleichung mit rationalen Koeffizienten, eine periodisch
Kettenbruchentwicklung hat.

\begin{beispiel}
Wir verfizieren diese Behauptung für die Quadratwurzel $x=\sqrt{47}$.
Das Berechnungsschema liefert
\[
\begin{aligned}
x_0&=\sqrt{47}
	&&\Rightarrow&
		a_0 &= 6
\\
x_1&=\frac{1}{x_0-a_0} = \frac{\sqrt{47}+6}{11}
	&&\Rightarrow&
		a_1 &= 1
\\
x_2&=\frac{1}{x_1-a_1} = \frac{\sqrt{47}+5}{2}
	&&\Rightarrow&
		a_2 &= 5
\\
x_3&=\frac{1}{x_2-a_2} = \frac{\sqrt{47}+5}{11}
	&&\Rightarrow&
		a_3 &= 1
\\
x_4&=\frac{1}{x_3-a_3} = \sqrt{47}+6
	&&\Rightarrow&
		a_4 &= 12
\end{aligned}
\]
An dieser Stelle beginnt sich das Schema zu wiederholen.
Es ist nämlich $x_0-a_0 = \sqrt{47}-6 = (\sqrt{47}+6) - 12 = x_4-a_4$.
Also ist
\[
\sqrt{47}
=
6+\cfrac{1}{
1+\cfrac{1}{
5+\cfrac{1}{
1+\cfrac{1}{
12+\cfrac{1}{
1+\cfrac{1}{
5+\cfrac{1}{
1+\cfrac{1}{
12+\cfrac{1}{\dots}}}}}}}}}
=
[6;\overline{1,5,1,12}]
\]
\end{beispiel}



