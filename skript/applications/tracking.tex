%
% tracking.tex -- Verfolgen einer Sternposition und Kontrolle eines Teleskops
%
% (c) 2013 Prof Dr Andreas Mueller, Hochschule Rapperswil
%
\subsection{Nachführung und ihre Kalibrierung}
Wie verfolgt man ein Objekt mit einer Kamera?
Offenbar muss man dazu
die Bilder der Kamera analysieren, die Verschiebung des Objektes innerhalb
des Bildes feststellen, und die Halterung der Kamera so zu bewegen,
dass das Objekt erneut zentriert erscheint.
Damit dies möglich ist, muss man wissen, wie stark sich die Steuersignale
der Halterung auf die Verschiebung des Bildes in der Kamera auswirken.

In der Astronomie werden besonders hohe Anforderungen an die Präzision
einer solchen Steuerung gestellt, schliesslich will man da ein grosses
Teleskop während Stunden auf den gleichen Punkt am Himmel richten, um
zum Beispiel eine Aufnahme einer Galaxie entsprechend lange belichten
zu können.
Dabei darf die Abweichung nie grösser als Bruchteile eines
einer Winkelsekunde werden, da sonst das Bild unscharf würde.

Astronomische Teleskop werden daher oft auf einer sogenannten
äquatorialen Montierung eingesetzt, welche eine Achse parallel
zur Erdachse hat. Dreht man diese Achse mit genau der gleichen
Geschwindigkeit wie die Erde, aber in entgegengesetzter Richtung,
sollte sich die Drehung aufheben. In der Praxis verbleibt durch
Aufstellungsfehler und Atmosphäreneinflüsse trotzdem ein Restfehler,
der gemessen und korrigiert werden muss.

\subsubsection{Guiding}
Nachführung oder Guiding beschreibt den Prozess, durch geeignete
Korrektursignale an die Montierung des Teleskops, ein Objekt, meist
einen einzelnen, leicht zu identifizierenden Stern, ständig an der
gleichen Stelle auf dem CCD-Chip der Kamera zu halten. Dazu wird
eine Hilfskamera-verwendet, die entweder mit einem eigenen kleinen
Teleskop, dem Leitfernrohr, ausgestattet ist, oder die das Licht
des Sterns aus dem Strahlengang des Hauptteleskops erhält, wo
es mit Hilfe eines kleine Spiegels auf den CCD-Chip der Nachführkamera
ausgelenkt wird.

Die Montierung kann typischerweise in zwei senkrechten Richtungen
bewegt werden, zusätzlich zu der konstanten Drehung um die
Achse parallel zur Erdachse.
Wir nehmen an, dass wir die Geschwindigkeit
dieser zusätzlichen Bewegung im Intervall $[-1,1]$ angeben können.
In der Praxis kann man jedoch meistens die Extremgeschwindigkeiten
$\pm 1$ aktivieren. Um die mittlere Geschwindigkeit $0.2$ zu erhalten,
muss man also die Geschwindigkeit $+1$ während 200ms aktivieren.

Bei einer äquatorialen Montierung entsprechen die Bewegungsrichtungen,
die man mit diesen Steuersignalen erreichen kann, der geographischen
Länge und Breite auf der Himmelskugel. Die Astronomen kennen diese
Koordinaten als Rektaszension, die Bewegung um die Erdachse, und Deklination,
sie verwenden meist $\alpha$ und $\delta$ für die beiden Winkel.

Es ist klar, dass eine Bewegung in Deklination immer die gleiche Auswirkung
auf ein Sternbild in der Kamera hat.
Die Auswirkung einer Bewegung um Erdachse hängt jedoch
davon ab, ob ein Objekt nahe des Äquators oder nahe des Pols
betrachtet wird. Wird das Teleskop parallel zur Erdachse ausgerichtet
($\delta=90^\circ$), dann bewegt sich ein Stern im Zentrum des Bildfeldes
bei Drehung um die Erdachse überhaupt nicht.
Je nach Position des Objektes auf der Himmelskugel sind also andere
Korrektursignale erforderlich, um das Objekt im Zentrum des CCD-Chips
zu halten.

Die Hilfskamera liefert Bilder des Leitsterns, aus denen die genaue
Position des Sternes im Koordinatensystem des CCD-Chips gemessen
werden kann. Die Achsen dieses Koordinatensystems müssen dabei
überhaupt nichts mit den Richtungen für Rektaszension und Deklination
zu tun haben.

\subsubsection{Kalibrierung}
Ohne irgendwelche Korrektursignale wird der Leitstern ungefähr mit
gleichförmiger Geschwindigkeit $v$ über den CCD-Chip driften.
Wendet man Korrektursignale $(\alpha,\delta)$ an, kommt eine
weitere Bewegung hinzu. Für die Regelung brauchen wir den Zusammenhang
zwischen $(\alpha,\delta)$ und der zugehörigen Verschiebung $(x,y)$ auf dem
CCD-Chip. Der Zusammenhang muss linear sein:
\begin{equation}
\begin{linsys}{3}
x&=&a_{11}\alpha&+&a_{12}\delta&+&tv_1\\
y&=&a_{21}\alpha&+&a_{22}\delta&+&tv_2\\
\end{linsys}
\qquad
\Leftrightarrow
\qquad
\begin{pmatrix}x\\y\end{pmatrix}
=A\begin{pmatrix}\alpha\\\delta\end{pmatrix}+tv
\label{guiding:model}
\end{equation}
Wir müssen die Matrix $A$ und den Vektor $v$ ermitteln.

Um Daten für die Bestimmung von $A$ und $v$ zu bekommen, kann man wie
folgt vorgehen. Man fährt mit dem Teleskop nacheinander alle Gitterpunkte
eines $2\times 2$ Quadrates mit den Ecken
$(-1,-1)$ und $(1,1)$ im $(\alpha,\delta)$-Koordinatensystem an.
\begin{figure}
\begin{center}
\includegraphics[width=\hsize]{graphics/calibration.jpg}
\end{center}
\caption{Bewegung des Leitsterns während des Kalibrierungsprozesses
und dem anschliessenden Anfahren der Nachführposition,
Langzeitaufnahme mit der Hauptkamera.
Gut zu erkennen ist auch, dass nicht ein Quadrat entsteht. Das Bild
entstand bei der Kalibrierung für die Aufnahmen, aus denen 
Abbildung~\ref{andromeda-image} zusammengesetzt ist, also bei
einer Deklination von ca.~$40^\circ$.
\label{guiding:calibration}}
\end{figure}
Abbildung \ref{guiding:calibration} zeigt eine Langzeitbelichtung der
Hauptkamera während des Kalibrierungsprozesses.
Nach jeder Bewegung misst man die Position des Leitsterns neu, und
notiert auch die Zeit. Für jede Position $(\alpha_i,\delta_i)$
bekommt man eine Zeit $t_i$ und eine Leitstern-Position $(x_i,y_i)$.
Wenn die Transformation (\ref{guiding:model}) zutrifft, müssten die
beiden Gleichungen
\[
\begin{linsys}{6}
x_i&=&\alpha_ia_{11}&+&\delta_ia_{12}&+&t_iv_1& &              & &              & &      \\
y_i&=&              & &              & &      & &\alpha_ia_{21}&+&\delta_ia_{22}&+&t_iv_2\\
\end{linsys}
\]
erfüllt sein. Für die $n=9$ Positionen des Quadrates ergibt dies insgesamt
$2n=18$  Gleichungen für die sechs Unbekannten $A$ und $v$.
Das Gleichungssystem ist zwar überbestimmt, 
es lässt sich aber mit der Methode von Abschnitt \ref{section:ueberbestimmt}
effizient lösen. 

\subsubsection{Korrektur}
Wir jetzt eine Positions-Abweichung $(x,y)$ des Leitsterns gemessen,
kann aus $A$ und $v$ die Korrektur in Rektaszension und Deklination
ermittelt werden, mit der die Abweichung innert $t=1$ Sekunde wieder
zum Verschwinden gebracht werden kann. Dazu muss man nur
(\ref{guiding:model}) nach $(\alpha,\delta)$ auflösen:
\[
\begin{pmatrix}x\\y\end{pmatrix}
=
A\begin{pmatrix}\alpha\\\delta\end{pmatrix}+tv
\qquad
\Rightarrow
\qquad
\begin{pmatrix}\alpha\\\delta\end{pmatrix}=
A^{-1}\left(
\begin{pmatrix}
x\\y
\end{pmatrix}
-tv
\right).
\]
Die Kalibrationsdaten $A$ und $v$ hängen offenbar nur von der relativen
Orientierung der Leitkamera zu den Achsen des Himmelskoordinatensystems
ab, sowie von der Deklination des Zielobjektes.
Insbesondere muss die
Kalibration nur einmal durchgeführt werden, solange man nur Objekte verfolgen
will, deren Deklination nicht sehr verschieden ist, und die Leitkamera in ihrer
Fassung nicht gedreht wird.

\subsubsection{Qualitätsbeurteilung der Kalibration}
Wie erkennt man, ob die Kalibrationsdaten eine erfolgreiche Nachführung
ermöglichen werden. Montierungsunzulänglichkeiten wie übermässig 
viel Spiel des Antriebs könnte dem entgegenstehen.

Auf einer perfekten Montierung wird sich der Leitstern während er 
Kalibrierung exakt entlang des Gitters bewegen. Die beiden Spaltenvektoren
\[
a_1=\begin{pmatrix}a_{11}\\ a_{21}\end{pmatrix}
\quad\text{und}
\quad
a_2=\begin{pmatrix}a_{12}\\ a_{22}\end{pmatrix}
\]
sollten also senkrecht stehen. Dies ist sehr einfach zu prüfen:
das Produkt $AA^t$ sollte eine Diagonalmatrix sein. Die Grösse der
Ausserdiagonalelement der Matrix $AA^t$ gibt also an, wie gut die Kalibrierung
gelungen ist.
Als dimensionslose  Masszahl für die Kalibrierungsqualität
eignet sich der Quotient
\[
\frac{a_1\cdot a_2}{\det(A)}.
\]
Werte nahe bei $0$ deuten auf eine gute Kalibrierung hin.
