%
% rechenaufwand.tex -- Rechenaufwand zur Lösung linearer Gleichungssysteme
%
% (c) 2009 Prof Dr Andreas Mueller, Hochschule Rapperswil
%
\section{Rechenaufwand}
\index{Rechenaufwand}
Wir wollen den Rechenaufwand für die Durchführung des Gauss-Verfahrens
bestimmen.
Diese Information wird zum Beispiel benötigt, wenn man mit
einem Microcontroller einen Kalman-Filter implementieren möchte, dort
muss auch eine Matrix invertiert werden.
\index{AVR}
Ein typischer 8bit-Microcontroller (AVR)
benötigt ca 200$\mu$s für eine Multiplikation, und etwa 80$\mu$s für
eine Addition\footnote{Zeitangaben aus der Microcontroller Performance
Comparison auf \url {http://www.freertos.org.}}.
\index{ARM}
Ein 32bit-$\mu$C wie der ARM (LPC2106) ist wesentlich schneller,
er schafft diese Operationen in etwa 10$\mu$s.
Je nach Rechenaufwand kann
es also nötig sein, von einen schnelleren Prozessor zu verwenden.

Für den $k$-ten Schritt beim Vorwärtsreduzieren muss die $k$-te
Zeile mit einer Zahl multipliziert werden ($n$ Operationen) und dann
von $n-k$ Zeilen subtrahiert werden ($(n-k)n$ Operationen).
Es sind also
\[
\sum_{k=1}^n n(n-k+1) = n^3 - n \sum_{k=1}^n k + n^2
=n^3+n^2-n\frac{n(n+1)}2=\frac{n^2(n+1)}2 =O(n^3)
\]
Operationen nötig.
Bei $n=12$ Unbekannten braucht man also bereits
$936$ Operationen, mehr als ungefähr sieben mal pro Sekunde kann man das mit
einem 8bit AVR also nicht machen.
Mit einem ARM7 dagegen könnte
man dies über 100 Mal pro Sekunde durchführen.

