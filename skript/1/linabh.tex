%
% linabh.tex -- lineare Abhängigkeit
%
% (c) 2009 Prof Dr Andreas Mueller, Hochschule Rapperswil
%
\section{Lineare Abhängigkeit%
\label{skript:section:linabh}}
\index{abhängig, linear}
\index{linear abhängig}
Im Gleichungssystem
\begin{align*}
x+2y&=5\\
2x+4y&=10
\end{align*}
ist die zweite Gleichung das Doppelte der ersten.
Sie ist also immer genau dann erfüllt, wenn die erste erfüllt ist.
Insbesondere schränkt sie die Menge der möglichen Lösungen
nicht weiter ein, die Lösungsmenge ist
\[
\mathbb L=\{(x,y)\,|\, x+2y=5\}.
\]
Auch kompliziertere Beziehungen zwischen Gleichungen sind
möglich.
Im Gleichungssystem
\begin{align*}
 x+4y+5z&=1\\
3x-6y-2z&=-1\\
4x-2y+3z&=0
\end{align*}
ist die letzte Gleichung die Summe der ersten beiden.
Sie ist also
immer bereits erfüllt, wenn die ersten beiden Gleichungen erfüllt
ist.
Damit schränkt auch sie die Lösungsmenge nicht ein.

Zur Definition der eindeutigen Lösung eines Gleichungssystems 
tragen offenbar nur diejenigen Gleichungen etwas bei, die unabhängig
von den anderen Gleichungen erfüllbar sind.
Wenn eine Gleichung
durch andere Gleichungen ausgedrückt werden kann, kann man sie
eliminieren.
Lässt sich die letzte Gleichung durch die
anderen ausdrücken, gibt es Koeffizienten $\lambda_i$ so, dass
\begin{equation*}
\begin{aligned}
l_m(x_1,\dots,x_n)&=\lambda_1 l_1(x_1,\dots,x_n)+\dots+\lambda_{m-1}l_{m-1}(x_1,\dots, x_n)\\
b_m&=\lambda_1b_1+\dots+\lambda_{m-1}b_{m-1}
\end{aligned}
\end{equation*}
Setzt man $\lambda_m=-1$ und bringt alle Terme auf eine Seite,
dann bedeuten diese Gleichungen, dass
\begin{equation}
\begin{aligned}
0&=\lambda_1l_1(x_1,\dots,x_n)+\dots+\lambda_ml_m(x_1,\dots,x_n)\\
0&=\lambda_1b_1+\dots+\lambda_mb_m
\end{aligned}
\label{linearabhaengigegleichungen}
\end{equation}
Gibt es keine solche Beziehung, sind die Gleichungen unabhängig
voneinander.
Diese Form verwenden wir, um den Begriff der linearen Abhängigkeit
zu definieren.
\begin{definition}
Die Linearformen $l_1,\dots,l_m$ heissen linear unabhängig,
wenn die einzige lineare Beziehung
\begin{equation}
\lambda_1l_1(x_1,\dots,x_n)+\dots+\lambda_ml_m(x_1,\dots,x_n)=0
\label{linearabhaengig}
\end{equation}
jene mit $\lambda_i=0$ für $1\le i\le m$ ist.

Die $l_i$ heissen linear abhängig, wenn es eine Beziehung der Form
(\ref{linearabhaengig})
gibt, in der nicht alle Koeffizienten $\lambda_i$ verschwinden.

Die Gleichungen $l_i(x_1,\dots,x_n)=b_i,1\le i\le m$ heissen linear unabhängig,
wenn  die einzige Beziehung der Form (\ref{linearabhaengigegleichungen})
die Koeffizienten $\lambda_i=0$ hat.
Sie heissen linear abhängig, wenn
es eine Beziehung der Form (\ref{linearabhaengigegleichungen}) gibt.
\end{definition}

Wenn Linearformen oder Gleichungen linear abhängig sind, dann kann
man eine oder mehrere von ihnen eliminieren, da sie zur Lösung nichts
beitragen.
Da der Gauss-Algorithmus versucht, genau die minimal benötigten Gleichungen
zu ermitteln, wird er eine Zeile voller Nullen hervorbringen.

\begin{hilfssatz}
Die Linearformen $l_1,\dots,l_m$ sind genau dann linear abhängig, wenn
die Anwendung des Gauss-Algorithmus auf das Koeffizientenschema dieser
Linearformen eine Zeile voller Nullen hervorbringt.
Insbesondere sind mehr als $n$ Linearformen immer linear abhängig.
Ist eine der Linearformen
bereits $0$, dann sind die Linearformen auf jeden Fall linear abhängig.
\end{hilfssatz}

\begin{proof}[Beweis]
Ist $m>n$, dann erzeugt der Gauss-Algorithmus im besten Fall $n$ Zeilen,
die ausser Nullen in der Zeile $i$ auch noch in der $i$-ten Spalte eine
$1$ enthalten.
Alle nachfolgenden Zeilen bestehen jedoch ausschliesslich
aus Nullen, somit sind die Gleichungen linear abhängig.

Ist zum Beispiel $l_i(x_1,\dots,x_n)=0$, dann ergibt sich mit den
Koeffizienten
\[
\lambda_1=0,\dots,\lambda_i=1,\dots \lambda_m=0
\]
eine lineare Beziehung
\[
\lambda_1l_1+\dots+\lambda_il_i+\dots+\lambda_ml_m=l_i=0,
\]
in der nicht alle Koeffizienten verschwinden.
\end{proof}

Mit dem Gauss-Algorithmus können wir sehr rasch entscheiden, ob Zeilen 
linear abhängig sind.
Wenn wir aber die $\lambda_i$ bestimmen wollen,
die gemäss Definition existieren sollen, dann müssen wir noch etwas
mehr arbeiten.
Wir illustrieren dies an einem Beispiel.

\begin{beispiel}
Der Gauss-Algorithmus
erkennt die folgenden Zeilen als linear abhängig:
\begin{align}
\begin{tabular}{|>{$}c<{$}>{$}c<{$}>{$}c<{$}|}
\hline
x_1&x_2&x_3\\
\hline
1\begin{picture}(0,0)
\color{red}\put(-3,4){\circle{12}}
\end{picture}%
&2&3\\
6&5&4\\
7\begin{picture}(0,0)
\color{blue}\drawline(-8,-2)(-8,24)(2,24)(2,-2)
\end{picture}%
&8&9\\
\hline
\end{tabular}
&\rightarrow
\begin{tabular}{|>{$}c<{$}>{$}c<{$}>{$}c<{$}|}
\hline
x_1&x_2&x_3\\
\hline
1&2&3\\
0&-7\begin{picture}(0,0)
\color{red}\put(-6,4){\circle{16}}
\end{picture}%
&-14\\
0&-6&-12\\
\hline
\end{tabular}
\rightarrow
\begin{tabular}{|>{$}c<{$}>{$}c<{$}>{$}c<{$}|}
\hline
x_1&x_2&x_3\\
\hline
1&2&3\\
0&1&2\\
0&-6%
\begin{picture}(0,0)
\color{blue}\drawline(-14,-2)(-14,10)(1,10)(1,-2)
\end{picture}%
&-12\\
\hline
\end{tabular}
\rightarrow
\begin{tabular}{|>{$}c<{$}>{$}c<{$}>{$}c<{$}|}
\hline
x_1&x_2&x_3\\
\hline
1&2&3\\
0&1&2\\
0&0&0\\
\hline
\end{tabular}
\label{skript:linabh:xgauss}
\end{align}
Es gibt also Zahlen $\lambda_1$, $\lambda_2$ und $\lambda_3$,
so dass die Summe der $\lambda_i$-fachen der $i$-ten Zeile zusammen
die Null-Zeile ergeben:
\[
\begin{tabular}{>{$}c<{$}>{$}c<{$}|>{$}c<{$}>{$}c<{$}>{$}c<{$}}
\lambda_1&\cdot&1&2&3\\
\lambda_2&\cdot&6&5&4\\
\lambda_3&\cdot&7&8&9\\
\hline
&&0&0&0
\end{tabular}
\]
Für die $\lambda_i$ finden wir daraus ein lineares Gleichungssystem:
\begin{equation}
\begin{linsys}{3}
1\lambda_1&+&6\lambda_2&+&7\lambda_3&=&0\\
2\lambda_1&+&5\lambda_2&+&8\lambda_3&=&0\\
3\lambda_1&+&4\lambda_2&+&9\lambda_3&=&0
\end{linsys}.
\label{skript:linabh:lambdagl}
\end{equation}
Man beachte, dass das Koeffizientenschema 
gegenüber dem ursprünglichen an der Diagonalen
gespiegelt worden ist.

Das Gleichungssystem kann jetzt mit Hilfe des
Gauss-Algorithmus gelöst werden:
\begin{align*}
\begin{tabular}{|>{$}c<{$}>{$}c<{$}>{$}c<{$}|}
\hline
\lambda_1&\lambda_2&\lambda_3\\
\hline
1%
\begin{picture}(0,0)
\color{red}\put(-3,4){\circle{12}}
\end{picture}%
&6&7\\
2&5&8\\
3%
\begin{picture}(0,0)
\color{blue}\drawline(-9,-2)(-9,24)(2,24)(2,-2)
\end{picture}%
&4&9\\
\hline
\end{tabular}
&\rightarrow
\begin{tabular}{|>{$}c<{$}>{$}c<{$}>{$}c<{$}|}
\hline
\lambda_1&\lambda_2&\lambda_3\\
\hline
1&6&7\\
0&-7%
\begin{picture}(0,0)
\color{red}\put(-6.5,4){\circle{15}}
\end{picture}%
&-6\\
0&-14&-12\\
\hline
\end{tabular}
\rightarrow
\begin{tabular}{|>{$}c<{$}>{$}c<{$}>{$}c<{$}|}
\hline
\lambda_1&\lambda_2&\lambda_3\\
\hline
1&6&7\\
0&1&\frac67\\
0&-14%
\begin{picture}(0,0)
\color{blue}\drawline(-20,-2)(-20,10)(1,10)(1,-2)
\end{picture}%
&-12\\
\hline
\end{tabular}
\\
&\rightarrow
\begin{tabular}{|>{$}c<{$}>{$}c<{$}>{$}c<{$}|}
\hline
\lambda_1&\lambda_2&\lambda_3\\
\hline
1&6%
\begin{picture}(0,0)
\color{blue}\drawline(-8,9)(-8,-2)(2,-2)(2,9)
\end{picture}%
&7\\
0&1&\frac67\\
0&0&0\\
\hline
\end{tabular}
\rightarrow
\begin{tabular}{|>{$}c<{$}>{$}c<{$}>{$}c<{$}|}
\hline
\lambda_1&\lambda_2&\lambda_3\\
\hline
1&0&\frac{13}{7}\\
0&1&\frac67\\
0&0&0\\
\hline
\end{tabular}
\end{align*}
Die beiden nicht verschwindenden Zeilen bedeuten, dass $\lambda_1$ und
$\lambda_2$ aus $\lambda_3$ berechnet werden können:
\begin{align*}
\lambda_1&=-\frac{13}7\lambda_3&\lambda_2&=-\frac{6}7\lambda_3.
\end{align*}
Eine ganzzahlige Lösung ergibt sich zum Beispiel für $\lambda_3=7$,
also
\[
(
\lambda_1,
\lambda_2,
\lambda_3
)
=(
-13,-6,7
),
\]
was man durch nachrechnen auch bestätigen kann.
Da in diesem Gleichungssystem
auf der rechten Seite nur Nullen stehen, können wir schliessen,
dass eine Lösung mit $\lambda_i\ne 0$ nur im singulären Fall gefunden
werden kann.
Die gespiegelten Koeffizienten sind also immer dann singulär,
wenn auch die ungespiegelten Koeffizienten singulär sind.

Man kann diese $\lambda_i$ auch direkt aus der Durchführung der
des Gauss-Algorithmus in ableiten.
Wir schreiben zu diesem Zwecke hin, wie der Gauss-Algorithmus die
Linearformen $l_1$, $l_2$ und $l_3$ miteinander verrechnet.
\begin{align}
\begin{tabular}{|>{$}c<{$}|}
\hline
l_1\\
l_2\\
l_3\\
\hline
\end{tabular}
\rightarrow
\begin{tabular}{|>{$}c<{$}|}
\hline
l_1\\
l_2-6l_1\\
l_3-7l_1\\
\hline
\end{tabular}
\rightarrow
\begin{tabular}{|>{$}c<{$}|}
\hline
l_1\\
-\frac{1}{7}(l_2-6l_1)\\
l_3-7l_1\\
\hline
\end{tabular}
\rightarrow
\begin{tabular}{|>{$}c<{$}|}
\hline
l_1\\
-\frac{1}{7}(l_2-6l_1)\\
l_3-7l_1+6(-\frac{1}{7}(l_2-6l_1))\\
\hline
\end{tabular}
\label{skript:linabh:linformabl}
\end{align}
Der Gaussalgorithmus hat in der dritten Zeile eine Nullzeile geliefert,
die Linearkombination in der dritten Zeile von
\eqref{skript:linabh:linformabl} muss daher verschwinden.
Sie kann ausserdem vereinfacht werden zu
\[
0=l_3-7l_1-\frac{6}{7}l_2 +\frac{36}{7}l_1
=
-\frac{13}{7}l_1-\frac{6}{7}l_2+l_3.
\]
Multipliziert man mit 7 erhält man
\[
-13\lambda_1-6\lambda_2+7\lambda_3=0,
\]
genau wie vorher wesentlich eleganter mit dem Gleichungssystem
\eqref{skript:linabh:lambdagl} und dem Gauss-Al\-go\-rith\-mus
gefunden.
\end{beispiel}

