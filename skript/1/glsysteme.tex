%
% glsysteme.tex -- Gleichungssysteme
%
% (c) 2009 Prof Dr Andreas Mueller, Hochschule Rapperswil
%
\section{Gleichungssysteme}
\rhead{Gleichungssyteme}
\subsection{Begriffe und Notation}
\index{Linearform}
\begin{definition}
Eine Linearform $l(x_1,x_2,\dots,x_n)$ über $\mathbb R$ in den $n$
Variablen $x_1,\dots,x_n$
ist eine Funktion der Form
\[
l(x_1,x_2,\dots,x_n)=a_1x_1+a_2x_2+\dots+a_nx_n,
\]
wobei $a_i\in\mathbb R$ für alle $i$.
\end{definition}
Ist $(x_1',\dots,x_n')$ ein zweiter Satz Variablen und $\lambda\in\mathbb R$,
dann gilt
\begin{equation}
\begin{aligned}
l(x_1+x_1',\dots,x_n+x_n')&=l(x_1,\dots, x_n)+l(x_1',\dots,x_n')\\
l(\lambda x_1, \dots ,\lambda x_n)&=\lambda l(x_1,\dots,x_n)
\end{aligned}
\label{linearitaet-linearformen}
\end{equation}
Man nennt diese Eigenschaft {\em Linearität}.
Die Linearformen sind
{\em lineare} Funktionen.
\index{linear}%
Etwas salopp ausgedrückt sind lineare Funktionen solche, die sich
auf die einzelnen Summanden einer Summe verteilen lassen und für
die sich Faktoren aus den Argumenten vor die Funktion ziehen lassen.
Multiplikation aller Argumente mit einer Zahl und Additionen aller
Argumente lassen sich also ``auflösen''.

\index{Gleichung!lineare}
Aus einer Linearform und einer Konstanten kann man eine
lineare Gleichung bilden:
\[
l(x_1,\dots,x_n)=a_1x_1+\dots +a_nx_n=b
\]
Hat man nur zwei oder drei Unbekannte, kann man sie etwas übersichtlicher
mit $x$, $y$ und $z$ bezeichnen:
\[
ax+by+cz=d
\]
ist auch eine lineare Gleichung.
Die Lösung einer solchen Gleichung
besteht aus einem Tripel von Zahlen $(x,y,z)$, die die Gleichung erfüllen.
Statt als Tripel wird die Lösung oft auch als Spalte geschrieben:
\[
\begin{pmatrix}
x\\y\\z
\end{pmatrix}.
\]

\index{Gleichungssystem!lineares}
Ein wesentlicher Teil der linearen Algebra befasst sich mit linearen
Gleichungssystemen, also mehreren linearen Gleichungen:
\begin{align*}
l_1(x_1,\dots,x_n)&=b_1\\
\vdots\qquad\qquad&\quad\vdots\\
l_m(x_1,\dots,x_n)&=b_m
\end{align*}
Dabei lassen wir vorerst zu, dass auch weniger oder mehr Gleichungen
als Unbekannte gegeben sind.
Später werden wir diskutieren, unter
welchen Bedingungen diese Gleichungssysteme Lösungen haben.

Vollständig ausgeschrieben lauten die Gleichungssysteme
\begin{align*}
a_{11}x_1+a_{12}x_2+\dots a_{1n}x_n&=b_1\\
a_{21}x_1+a_{22}x_2+\dots a_{2n}x_n&=b_2\\
\vdots\\
a_{m1}x_1+a_{m2}x_2+\dots a_{mn}x_n&=b_m
\end{align*}
Das Gleichungssystem ist durch die Angabe der Koeffizienten $a_{ij}$
und der rechten Seite $b_i$ gegeben, gesucht sind die Unbekannten $x_j$.
Es ist also nicht unbedingt nötig, die Gleichungen auszuschreiben,
wenn man diese Angaben in anderer Form machen kann, z.~B.~indem man
Unbekannte und rechte Seite in Spaltenform und die Koeffizienten in
Matrixform schreibt:
\[
A=\begin{pmatrix}
a_{11}&a_{12}&\dots &a_{1n}\\
a_{21}&a_{22}&\ddots&a_{2n}\\
\vdots&\vdots&\ddots&\vdots\\
a_{m1}&a_{m2}&\dots&a_{mn}
\end{pmatrix}
,\quad
x=\begin{pmatrix}
x_1\\x_2\\\vdots\\x_n
\end{pmatrix},
\quad
b=\begin{pmatrix}
b_1\\b_2\\\vdots\\b_m
\end{pmatrix}.
\]

\subsection{Eine Unbekannte}
Der Fall einer Unbekannten ist besonders einfach.
Wir diskutieren die Gleichungen
\[
ax=b
\]
mit $a,b\in\mathbb R$ und suchen eine Lösung $x\in\mathbb R$.
Offenbar kann man die Lösung finden, indem man durch $a$ teilt:
\[
x=\frac{b}{a},
\]
dies ist aber nur dann möglich, wenn $a\ne 0$ ist.
Dies scheint
der ``reguläre'' Fall zu sein, in dem es genau eine Lösung gibt.

Der singuläre Fall ist $a=0$, die Gleichung wird dann zu 
\[
0=b.
\]
Offenbar ist diese Gleichung überhaupt nicht erfüllbar, wenn $b\ne 0$
ist.
Falls jedoch $b=0$, dann ist die Gleichung $0\cdot x=0$, die
zutrifft was immer wir für Werte für $x$ einsetzen.
In diesem Fall haben wir also unendlich viele Lösungen.
Zusammengefasst
bekommen wird
\begin{enumerate}
\item regulärer Fall: $a\ne 0\Rightarrow x=\frac{b}{a}$
\item singulärer Fall: $a=0$
\begin{enumerate}
\item $b=0\Rightarrow$ jedes beliebige $x\in\mathbb R$ ist Lösung
\item $b\ne0\Rightarrow$ keine Lösungen
\end{enumerate}
\end{enumerate}
Wichtig: die linke Seite (das $a$) entscheidet über regulär/singulär,
die rechte Seite entscheidet im singulären Fall über keine oder
undendlich viele Lösungen.

\subsection{Zwei Unbekannte}
Eine einzelne Gleichung mit zwei Unbekannten 
\[
ax+by=c
\]
kann keine eindeutige Lösung haben.
Falls $b\ne 0$ kann man
sie nach $y$ auflösen:
\[
y=\frac{c-ax}b=-\frac{a}{b}x+\frac{c}b,
\]
sie beschreibt also eine Gerade.
Die Menge der Punkte 
\[
g_1=\{(x,y)\,|\,ax+by=c\}
\]
ist eine Gerade.

Fügt man jetzt eine zweite Gleichung hinzu, erhält man
ein Gleichungssystem
\[
\begin{linsys}{2}
ax&+&by&=&e\\
cx&+&dy&=&f.
\end{linsys}
\]
\index{Lösungsmenge}
Die Lösungsmenge jeder dieser Gleichungen entspricht einer
Geraden:
\begin{align*}
g&=\{(x,y)\,|\,ax+by=e\}\\
h&=\{(x,y)\,|\,cx+dy=f\}
\end{align*}
Wir interessieren uns jedoch für diejenigen Paare $(x,y)$,
die beide Gleichungen erfüllen, also für die Schnittmenge
der Geraden $g$ und $h$.

\index{Gerade}
Ohne dass wir bereits ein Verfahren zur Berechnung einer
Lösung haben, können wir bereits herausfinden, wie viele
Lösungen das Gleichungssystem hat.
Zwei Geraden in der Ebene
können höchstens in den drei folgenden Lagen angeordnet sein:
\begin{enumerate}
\item Die Geraden sind nicht parallel, dann schneiden sie sich in
genau einem Punkt und das Gleichungssystem hat eine einzige Lösung.
\item Die Geraden sind parallel, dann gibt es zwei mögliche Unterfälle:
\begin{enumerate}
\item Die Geraden haben einen positiven Abstand, dann gibt es keine
gemeinsamen Punkte, also hat das Gleichungssystem auch keine Lösung.
\item Der Abstand der Geraden ist $0$, dann sind die Geraden deckungsgleich,
jeder Punkt der beiden Geraden ist Lösung des Gleichungssystems, es hat
also unendlich viele Lösungen.
\end{enumerate}
\end{enumerate}
Hier drei Beispiele von Gleichungssystemen, die diese drei Situationen
widerspiegeln:
\begin{beispiel}[Beispiel für den Fall 1]
Das folgende Gleichungssystem hat genau eine Lösung.
\begin{align*}
x+y=2\\
x-y=0
\end{align*}
hat die Lösung $(1,1)$, wie man durch Einsetzen nachprüfen kann.
\end{beispiel}
\begin{beispiel}[Beispiel für den Fall 2.~(a)]
Das Gleichungssystem
\[
\begin{linsys}{2}
x&+&y&=&1\\
2x&+&2y&=&0
\end{linsys}
\]
kann keine Lösung haben.
Multipliziert man die erste Gleichung mit $2$
und subtrahiert die zweite Gleichung erhält man $0=2$.
Dies kann natürlich
nie erfüllt werden, insbesondere gibt es kein Zahlenpaar $(x,y)$, für
welches die beiden Gleichungen wahr würden.
\end{beispiel}
\begin{beispiel}[Beispiel für den Fall 2.~(b)]
Das Gleichungssystem
\[
\begin{linsys}{2}
x&+&y&=&1\\
2x&+&2y&=&2
\end{linsys}
\]
hat alle Punkte der Geraden $y=1-x$ als Lösung.
\end{beispiel}

