%
% matrizen.tex -- Matrizen und Vektoren
%
% (c) 2009 Prof Dr Andreas Mueller, Hochschule Rapperswil
%
\section{Notation: Matrizen und Vektoren}
\rhead{Notation: Matrizen und Vektoren}
Viele der Operationen im Lösungsverfahren für lineare Gleichungssysteme,
sind unabhängig von ihrem Bezug zu einem Gleichungssystem sinnvoll.
In der Tableaudarstellung haben wir auch einen Notation verwendet,
in der die Unbekannten überhaupt nicht mehr vorkommen.
Dies kann jedoch nicht nur ein praktische Abkürzung sein, es muss mehr
dahinter stecken.
In diesem Abschnitt definieren wir die Vektoren und Matrizen, die zu
einem Gleichungssystem gehören.

\subsection{Vektoren}
\subsubsection{Definition von Zeilen- und Spaltenvektoren}
Die Operationen im Lösungsverfahren erfolgten jeweils Zeilen- oder Spaltenweise,
es ist daher sinnvoll, die Zeilen oder Spalten als eigene vollwertige 
mathematische Objekte zu definieren.

\begin{definition}
Das Zahlenschema
\[
\begin{pmatrix}
a_1&a_2&\dots&a_n
\end{pmatrix}
\]
mit $a_i\in\mathbb R$ heisst ein $n$-dimensionaler Zeilenvektor.
Das Zahlenschema
\[
\begin{pmatrix}
b_1\\b_2\\\vdots\\b_m
\end{pmatrix}
\]
heisst $m$-dimensionaler Spaltenvektor.
\end{definition}
Die Linearformen entsprechen also Zeilenvektoren, die rechte
Seite des Gleichungssystems ist ein Spaltenvektor.

Zeilen oder Spaltenvektoren kürzen wir ab als ein einzelnes
Zeichen, zum Beispiel
\[
a
=
\begin{pmatrix}
a_1&a_2&\dots&a_n
\end{pmatrix},\qquad
b=\begin{pmatrix}
1\\2\\3
\end{pmatrix}.
\]

\subsubsection{Rechenoperationen mit Vektoren}
Für Zeilen- oder Spaltenvektoren können wir die Addition von
Vektoren und die Multiplikation mit reellen Zahlen definieren.
Damit das überhaupt gehen kann, müssen die Vektoren natürlich
die gleiche Länge (Dimension) haben.
Die Operationen erfolgen elementweise:

\begin{definition}
Seien $a$ und $a'$ zwei $n$-dimensionale Zeilenvektoren,
\begin{align*}
a&=\begin{pmatrix}a_1&a_2&\dots&a_n\end{pmatrix}
\\
a'&=\begin{pmatrix}a'_1&a'_2&\dots&a'_n\end{pmatrix}
\end{align*}
und $\lambda\in\mathbb R$, dann sind die Summe $a+a'$ und das
Produkt $\lambda a$ wie folgt definiert:
\begin{align*}
a+a'&=\begin{pmatrix}a_1+a'_1&a_2+a'_2&\dots&a_n+a'_n\end{pmatrix}
\\
\lambda a&=\begin{pmatrix}\lambda a_1&\lambda a_2&\dots&\lambda a_n\end{pmatrix}
\end{align*}
Sind $b$ und $b'$ zwei $m$-dimensionale Spaltenvektoren
\begin{align*}
b&=\begin{pmatrix}b_1\\b_2\\\vdots\\b_m\end{pmatrix},
&
b'&=\begin{pmatrix}b'_1\\b'_2\\\vdots\\b'_m\end{pmatrix}
\end{align*}
dann, sind die Summe $b+b'$ und das Produkt $\lambda b$ wie folgt 
definiert:
\begin{align*}
b+b'&=\begin{pmatrix}
b_1+b'_1\\
b_2+b'_2\\
\vdots\\
b_m+b'_m\\
\end{pmatrix},
&
\lambda b&=\begin{pmatrix}
\lambda b'_1\\
\lambda b'_2\\
\vdots\\
\lambda b'_m
\end{pmatrix}
\end{align*}
\end{definition}
Die Operationen {\bf E} und {\bf I} können wir jetzt mit Vektoren
beschreiben.
Die Zeile mit der Nummer $i$ ist offenbar ein
$n+1$-dimensionaler Zeilenvektoren $z_i$
\[
z_i=\begin{pmatrix}a_{i1}&a_{i2}&\dots&a_{im}&b_i\end{pmatrix}.
\]
Die Operation {\bf I} für die $i$-te Zeile bedeutet jetzt, dass man
den Zeilenvektor $z_i$ ersetzt durch den neuen Zeilenvektor
\[
\frac1{a_{ii}}z_i
=
\frac1{a_{ii}}
\begin{pmatrix}a_{i1}&a_{i2}&\dots&a_{im}&b_i\end{pmatrix}
=
\begin{pmatrix}\frac{a_{i1}}{a_{ii}}&\frac{a_{i2}}{a_{ii}}&\dots&\frac{a_{im}}{a_{ii}}&\frac{b_i}{a_{ii}}\end{pmatrix}
\]
Bei der Operation {\bf E} wird der Zeilenvektor $z_j$
verringert um das $a_{ji}$-fache des Zeilenvektors $z_i$, also
\[
z_j \leftarrow z_j-a_{ji}z_i.
\]

\subsubsection{Gleichungssystem in Vektorschreibweise}
Auch das Gleichungssystem selbst können wir mit Hilfe von Spaltenvektoren
schreiben.
Betrachten wir die Spalten mit der Nummer $k$ als Spaltenvektor $a_k$,
also
\[
a_k=\begin{pmatrix}a_{1k}\\a_{2k}\\\dots\\a_{mk}\end{pmatrix},
\]
und die rechten Seiten des Gleichungssystems wie früher als Spaltenvektor $b$,
dann können wir das Gleichungssystem schreiben als
\[
x_1\begin{pmatrix}a_{11}\\a_{21}\\\vdots\\a_{m1}\end{pmatrix}
+
x_2\begin{pmatrix}a_{12}\\a_{22}\\\vdots\\a_{m2}\end{pmatrix}
+
\dots
+
x_n\begin{pmatrix}a_{1n}\\a_{2n}\\\vdots\\a_{mn}\end{pmatrix}
=
\begin{pmatrix}b_1\\b_2\\\vdots\\b_m\end{pmatrix}
\]
oder kurz
\[
a_1x_1+a_2x_2+\dots+a_nx_n=b.
\]

\subsubsection{Spezielle Vektoren\label{speziellevektoren}}
Die folgenden speziellen Vektoren sind oft nützlich.
Der Nullvektor besteht
aus lauter Nullen, wir schreiben dafür
\[
0=\begin{pmatrix}0\\\vdots\\0\end{pmatrix}.
\]
Im Verfahren für die simultane Lösung von Gleichungssystemen haben
wir spezielle rechte Seiten verwendet, welche bis auf eine Stelle aus lauter
Nullen bestehen.
Diese Vektoren bezeichnen wir in Zukunft mit
\[
e_1=\begin{pmatrix}1\\0\\\vdots\\0\end{pmatrix},
\qquad
e_2=\begin{pmatrix}0\\1\\\vdots\\0\end{pmatrix},
\qquad
e_n=\begin{pmatrix}0\\0\\\vdots\\1\end{pmatrix}.
\]

\subsection{Matrizen}
\subsubsection{Definition einer Matrix}
Die Formulierung mit Zeilen- und Spaltenvektoren ist immer noch etwas umständlich,
wir würden erwarten, dass auch das ganze Koeffizientenschema $(a_{ij})$
eine eigenständiges mathematisches Objekt ist, immerhin bestimmt es den
Gang des Lösungsverfahrens vollständig.

\begin{definition}
Das rechteckige Zahlenschema
\[
A=
\begin{pmatrix}
a_{11}&a_{12}&\dots&a_{1n}\\
a_{21}&a_{22}&\dots&a_{2n}\\
\vdots&\vdots&\ddots&\vdots\\
a_{m1}&a_{m2}&\dots&a_{mn}\\
\end{pmatrix}
\]
heisst $m\times n$-Matrix.
\end{definition}

\subsubsection{Produkt einer Matrix mit einem Vektor}
Um das Gleichungssystem mit der Koeffizientenmatrix $A$ zu beschreiben
fehlt uns jetzt noch eine Verknüpfung zwischen Matrizen und
Vektoren.
Die folgende Definition tut, was wir uns wünschen.
\begin{definition}
Ist $A$ eine $m\times n$-Matrix und $x$ ein $n$-dimensionaler Spaltenvektor,
dann definieren wir das Produkt $Ax$ also den $m$-dimensionalen
Spaltenvektor
\[
\begin{pmatrix}
a_{11}x_1+a_{12}x_2+\dots+a_{1n}x_n\\
a_{21}x_1+a_{22}x_2+\dots+a_{2n}x_n\\
\vdots\\
a_{m1}x_1+a_{m2}x_2+\dots+a_{mn}x_n\\
\end{pmatrix}
\]
Das Element $b_i$ in der $i$-ten Zeile dieses Spaltenvektors ist
\[
b_i=\sum_{j=1}^na_{ij}x_j.
\]
\end{definition}
Die Multiplikation $Ax$ erfolgt also, indem die Zeilen von $A$
elementweise mit der Spalte $x$ multipliziert, und dann alle Produkte
aufaddiert werden (``Zeile $\strut \times\mathstrut$ Spalte'').

\begin{beispiel} Sei
\[
A=\begin{pmatrix}
6&1&2\\
5&4&0\\
4&1&9
\end{pmatrix}
,\qquad
v=
\begin{pmatrix}
1\\8\\4
\end{pmatrix},
\]
wir suchen das Produkt $Av$:
\begin{align*}
Av&=
\begin{pmatrix}
\color{red}6&\color{red}1&\color{red}2\\
5&4&0\\
4&1&9
\end{pmatrix}
\begin{pmatrix}
\color{blue}1\\\color{blue}8\\\color{blue}4
\end{pmatrix}
=
\begin{pmatrix}
{\color{red}6}\cdot{\color{blue}1}+
{\color{red}1}\cdot{\color{blue}8}+
{\color{red}2}\cdot{\color{blue}4}
\\
5\cdot{\color{blue}1}+
4\cdot{\color{blue}8}+
0\cdot{\color{blue}4}
\\
4\cdot{\color{blue}1}+
1\cdot{\color{blue}8}+
9\cdot{\color{blue}4}
\end{pmatrix}
=\begin{pmatrix}
22\\37\\48
\end{pmatrix}.
\end{align*}
\end{beispiel}

Mit diesen Definitionen ist das Gleichungssystem mit Koeffizientenmatrix $A$,
dem Lösungs-Spaltenvektor $x$ und der rechten Seite $b$, ebenfalls einem
Spaltenvektor, zu
\[
Ax=b
\]
geworden.
Diese Schreibweise suggeriert, dass wir einfach auf beiden
Seiten ``durch $A$ teilen'' könnten, um die Lösung zu finden.
Auf dieses
Ziel werden wir hinarbeiten, und eine Multiplikation und ``Division''
von Matrizen definieren, so dass wir tatsächlich die Lösung des Gleichungssystems
mit Hilfe einer Formel
\[
x=A^{-1}b
\]
finden können.

\subsubsection{Rechenregeln für das Produkt Matrix $\times$ Vektor}
Für das Produkt gelten die Regeln, die man sich von der Algebra 
gewohnt ist:
\begin{equation}
\begin{aligned}
A(u+v)&=Au+Av&\qquad&(A+B)v=Av+Bv\\
A(\lambda v)&=\lambda Av
\end{aligned}
\label{linearitaet-matrixvektor}
\end{equation}
Die Gleichungen (\ref{linearitaet-matrixvektor}) sehen aus wie
(\ref{linearitaet-linearformen}),
man sagt, $A$ sei eine {\em lineare} Abbildung.

\subsubsection{Produkt zweier Matrizen}
Da Matrizen als ``dicke'' Vektoren aufgefasst werden können, lassen
sich jetzt auch Matrizen mit anderen Matrizen multiplizieren.
Die Multiplikation erfolgt wie bei Vektoren als
$\text{Zeilen}\times\text{Spalten}$, das Element in Zeile $i$
und Spalte $j$ der Produktmatrix entsteht aus Zeile $i$ des ersten
Faktors und Spalte $j$ des zweiten Faktors.
Die Zeilenlänge der ersten Matrix muss zur Spaltenlänge
der zweiten Matrix passen:
\begin{align*}
\left(\begin{tabular}{|c|c|c|c|c|}
\hline
&&&&\\
\hline
&&&&\\
\hline
&&&&\\
\hline
\end{tabular}\right)
\cdot
\left(\begin{tabular}{|c|c|}
\hline
&\\
\hline
&\\
\hline
&\\
\hline
&\\
\hline
&\\
\hline
\end{tabular}\right)
&=
\left(\begin{tabular}{|c|c|}
\hline
&\\
\hline
&\\
\hline
&\\
\hline
\end{tabular}\right)
\\
(m\times l)\cdot (l\times n)&=(m\times n)
\end{align*}
\begin{beispiel}
\begin{align*}
A&=\begin{pmatrix}
\color{red}1&\color{red}2&\color{red}4\\
0&4&-1
\end{pmatrix}
\\
B&=\begin{pmatrix}
\color{blue}-2&4\\
\color{blue}-4&-5\\
\color{blue}2&4
\end{pmatrix}
\\
AB&=
\begin{pmatrix}
{\color{red}1}\cdot({\color{blue}-2})+{\color{red}2}\cdot({\color{blue}-4})+{\color{red}4}\cdot{\color{blue}2} &
{\color{red}1}\cdot 4+{\color{red}2}\cdot(-5)+{\color{red}4}\cdot 4\\
0\cdot ({\color{blue}-2})+4\cdot({\color{blue}-4})+(-1)\cdot {\color{blue}2} &
0\cdot 4+4\cdot(-5)+(-1)\cdot 4
\end{pmatrix}
\\
&=
\begin{pmatrix}
-2&10\\-18&-24
\end{pmatrix}
\\
BA&=\begin{pmatrix}
(-2)\cdot1+4\cdot 0&
(-2)\cdot2+4\cdot 4&
(-2)\cdot4+4\cdot (-1) \\
(-4)\cdot1+(-5)\cdot 0&
(-4)\cdot2+(-5)\cdot 4&
(-4)\cdot4+(-5)\cdot (-1)\\
2\cdot1+ 4\cdot0&
2\cdot2+ 4\cdot4&
2\cdot4+ 4\cdot (-1)
\end{pmatrix}
\\
&=
\begin{pmatrix}
   -2&  12& -12\\
   -4& -28& -11\\
    2&  20&   4
\end{pmatrix}
\end{align*}
\end{beispiel}
Das Beispiel illustriert auch, dass es wesentlich auf die Reihenfolge
der Faktoren ankommt.
Dies gilt selbst dann, wenn die beiden
Matrizen quadratisch sind, also $AB$ und $BA$ ebenfalls quadratische
Matrizen sind.
\begin{beispiel}
Dieses Beispiel zeigt, dass das Matrizenprodukt nicht kommutativ ist.
\begin{align*}
\begin{pmatrix} 0&1\\0&0 \end{pmatrix}
\begin{pmatrix} 0&0\\1&0 \end{pmatrix}
&=
\begin{pmatrix} 1&0\\0&0 \end{pmatrix}
\\
\begin{pmatrix} 0&0\\1&0 \end{pmatrix}
\begin{pmatrix} 0&1\\0&0 \end{pmatrix}
&=
\begin{pmatrix} 0&0\\0&1 \end{pmatrix}
\end{align*}
Die beiden Produkte sind offensichtlich verschieden.
\end{beispiel}

\subsubsection{Transponierte Matrix}
Hat eine Matrix linear abhängige Zeilen, lassen sich die Zeilen
linear zu einer Null-Zeile kombinieren.
Die Zeilen der Matrix
\[
A=
\begin{pmatrix}
a_{11}&\dots&a_{1n}\\
\vdots&\ddots&\vdots\\
a_{m1}&\dots&a_{mn}
\end{pmatrix}
\]
sind also genau dann linear abhängig, wenn das Gleichungssystem
\[
\begin{linsys}{4}
a_{11}\lambda_1&+&a_{21}\lambda_2&+&\dots&+&a_{m1}\lambda_m&=&0\\
a_{12}\lambda_1&+&a_{22}\lambda_2&+&\dots&+&a_{m2}\lambda_m&=&0\\
\vdots         & &\vdots         & &\ddots&&\vdots         & &\vdots\\
a_{1n}\lambda_1&+&a_{2n}\lambda_2&+&\dots&+&a_{mn}\lambda_m&=&0\\
\end{linsys}
\]
eine Lösung hat, die nicht aus lauter Nullen besteht.
Die Koeffizientenmatrix
dieses Gleichungssystems entsteht dadurch, dass $A$ an der
Diagonalen $a_{11}\dots a_{22}\dots a_{33}\dots$ gespiegelt wird.
\begin{definition} Ist $A$ eine $m\times n$-Matrix mit Einträgen
$a_{ij}$ in der $i$-ten Zeile und $j$-ten Spalte, dann heisst die
gespiegelte Matrix
\[
A^t=\begin{pmatrix}
a_{11}&a_{21}&\dots&a_{m1}\\
a_{12}&a_{22}&\dots&a_{m2}\\
\vdots&\vdots&\ddots&\vdots\\
a_{1n}&a_{2n}&\dots&a_{mn}
\end{pmatrix}
\]
die transponierte Matrix von $A$.
$A^t$ ist eine $n\times m$-Matrix.
Eine Matrix heisst symmetrisch, wenn $A^t=A$.
\index{Matrix!symmetrische}
\end{definition}
Da das Transponieren einer Matrix ihre Zeilen und Spalten vertauscht,
vertauscht
sie auch die Reihenfolge der Faktoren in einem Matrizenprodukt
\begin{align*}
A^tB^t&=\text{Zeilen von $A^t$}\times\text{Spalten von $B^t$}\\
      &=\text{Spalten von $A$}\times\text{Zeilen von $B$}\\
      &=(\text{Zeilen von $B$}\times\text{Spalten von $A$})^t\\
      &=(BA)^t.
\end{align*}

\begin{beispiel}
Gegeben sei die Matrix
\[
A=\begin{pmatrix}
4&3\\
5&4
\end{pmatrix}.
\]
Man berechne $A^tA$ und $AA^t$.

\smallskip

{\parindent 0pt $A^t$ ist}
\[
A^t=\begin{pmatrix}
4&5\\
3&4
\end{pmatrix}.
\]
Das Produkt der beiden Matrizen ist
\begin{align*}
A^tA&=
\begin{pmatrix}
4&5\\
3&4
\end{pmatrix}
\begin{pmatrix}
4&3\\
5&4
\end{pmatrix}
=\begin{pmatrix}
16+25&12+20\\
12+20&9+16
\end{pmatrix}
=
\begin{pmatrix}
41&32\\
32&25
\end{pmatrix},
\\
AA^t&=
\begin{pmatrix}
4&3\\
5&4
\end{pmatrix}
\begin{pmatrix}
4&5\\
3&4
\end{pmatrix}
=
\begin{pmatrix}
16+9&20+12\\
20+12&25+16
\end{pmatrix}
=
\begin{pmatrix}
25&32\\
32&41
\end{pmatrix}.
\end{align*}
Man beachte, dass auch hier wieder $A^tA\ne AA^t$.
\end{beispiel}

\subsubsection{Rechenregeln für Matrizen}
Wir haben bereits früher die Rechenregeln für das Produkt einer Matrix
mit einem Vektor aufgestellt, und dabei festgestellt, dass die
aus der Algebra vertrauten Regeln erhalten bleiben.
Mit dem
Produkt Matrix $\times$ Matrix kommen noch einige neue Regeln dazu,
die auch vertraut sind.
Man kann also mit Matrizen genau so rechnen,
wie man das in der Algebra immer gemacht hat, man muss nur
aufpassen, dass man nie die Reihenfolge der Matrizen vertauscht.
\begin{align*}
    A(u+v)&=Au+Av     &\Rightarrow&&A(B+C)&=AB+AC,    &(A+B)C&=AC+BC\\
A\lambda v&=\lambda Av&\Rightarrow&&A\lambda B&=\lambda AB
\end{align*}

\subsubsection{Matrixnotation für Gleichungssyteme}
Lineare Gleichungssysteme können jetzt mit Hilfe der Matrixnotation
sehr kompakt geschrieben werden.
Das Gleichungssytem
\[
\begin{linsys}{3}
a_{11}x_1&+&\dots &+&a_{1n}x_n&=&b_1\\
\vdots\hspace*{1em}& &\ddots& &\vdots\hspace*{1em}& &\vdots\hspace*{1mm}\\
a_{m1}x_1&+&\dots &+&a_{mn}x_n&=&b_1\\
\end{linsys}
\]
wird geschrieben als
\[
\begin{aligned}
Ax&=b
&&\text{mit}&
A&=\begin{pmatrix}
a_{11}&\dots &a_{1n}\\
\vdots&\ddots&\vdots\\
a_{m1}&\dots &a_{mn}
\end{pmatrix},&
x&=\begin{pmatrix}x_1\\\vdots\\x_n\end{pmatrix}
&&\text{und}&
b&=\begin{pmatrix}b_1\\\vdots\\b_m\end{pmatrix}.
\end{aligned}
\]
$A$ ist eine $m\times n$-Matrix, $x$ ist ein $n$-dimensionaler Vektor
und $b$ ist $m$-dimensional.

\begin{beispiel}
Das Gleichungssystem
von Abschnitt~\label{skript:subsection:iedreiunbekannte}
kann geschrieben werden als
\[
\underbrace{
\begin{pmatrix}
1&2&3\\
6&5&4\\
1&-1&1
\end{pmatrix}
}_{\displaystyle=A}
x
=
\underbrace{
\begin{pmatrix}10\\32\\2\end{pmatrix}
}_{\displaystyle=b},
\]
die in Abschnitt~\label{skript:subsection:iedreiunbekannte} gefundene
Lösung war
\[
x=\begin{pmatrix}3\\2\\1\end{pmatrix}.
\qedhere
\]
\end{beispiel}


\begin{beispiel}
Das Gleichungssystem 
\[
\begin{linsys}{4}
3x_1&+&4x_2&=& 9y_1&+4y_2\\
5x_1&+&7x_2&=&20y_1&+9y_2
\end{linsys}
\]
kann geschrieben werden als
\[
Ax=By
\]
mit
\[
\begin{aligned}
A&=\begin{pmatrix}3&4\\5&7\end{pmatrix},&
x&=\begin{pmatrix}x_1\\x_2\end{pmatrix},&
B&=\begin{pmatrix}9&4\\20&9\end{pmatrix}&
 &\text{und}&
y&=\begin{pmatrix}y_1\\y_2\end{pmatrix}.
\end{aligned}
\]
In Abschnitt~\label{subsection:rechte seite linear} haben wir ausgerechnet,
dass die Lösung $x$ geschrieben werden kann als
\[
x
=
\begin{pmatrix}x_1\\x_2\end{pmatrix}
=
Cy
=
\begin{pmatrix}
-17&-8\\15&7
\end{pmatrix}
\begin{pmatrix}
y_1\\y_2
\end{pmatrix}.
\qedhere
\]
\end{beispiel}

