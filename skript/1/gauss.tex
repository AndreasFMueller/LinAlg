%
% gauss.tex -- Gauss-Elimination
%
% (c) 2009 Prof Dr Andreas Mueller, Hochschule Rapperswil
%
\section{Das Gauss-Verfahren}
\rhead{Gauss-Algorithmus}
\index{Gauss-Verfahren}
\index{Gauss-Elimination|see{Gauss-Verfahren}}
Für die Anwendung brauchen wir ein effizientes Verfahren zur
Lösung von linearen Gleichungssystemen.
Dabei erwarten wir natürlich,
dass wir zur Bestimmung von $n$ Unbekannten auch $n$ Gleichungen benötigen
werden.
Dass dies so ist, wird sich im Laufe des Verfahrens so ergeben,
wir werden daher zunächst weiter von $m$ Gleichungen ausgehen.
\subsection{Ein Beispiel}
Als Beispiel wollen wir das Gleichungssystem
\[
\begin{linsys}{3}
3\;x%
\begin{picture}(0,0)
\color{red}\put(-13,4){\circle{12}}
\end{picture}%
&-&6y&=&9\\
2\;x&+&4y&=&-2
\end{linsys}
\]
lösen.
Wir führen dabei genau Buch über die einzelnen Schritte,
damit wir daraus später ein allgemeines Verfahren ableiten können.
Dazu werden wir in jedem Schritt zunächst einzeichnen, um welchen
Term wir uns als nächstes kümmern wollen.

\subsubsection{Schritt I: $x$ isolieren}
Dividieren wir die erste Gleichung durch $3$, steht $x$ ohne Koeffizient
auf der linken Seite:
\[
\begin{linsys}{3}
3\;x%
\begin{picture}(0,0)
\color{red}\put(-13,4){\circle{12}}
\end{picture}%
&-&6y&=&9\\
2\;x&+&4y&=&-2
\end{linsys}
\qquad\rightarrow\qquad
\begin{linsys}{3}
x%
\begin{picture}(0,0)
\color{red}\put(-13,4){\circle{12}}
\end{picture}%
&-&2y&=&3\\
2\;x&+&4y&=&-2
\end{linsys}
\]
Es wird damit einfacher, nach $x$ aufzulösen
und das gefundene Resultat dazu zu verwenden, die anderen $x$ zu
eliminieren
\subsubsection{Schritt F: $x$ aus Folgegleichungen eliminieren}
Mit Hilfe der ersten Gleichung können wir jetzt auch das $x$ in
der zweiten Gleichung eliminieren.
Dazu subtrahieren wir das Doppelte der ersten Gleichung von der zweiten.
Resultat:
\[
\begin{linsys}{3}
x&-&2y&=&3\\
2x%
\begin{picture}(0,0)
\color{blue}\drawline(-14,-3)(-14,10)(12,10)(12,-3)
\end{picture}%
&+&4y&=&-2
\end{linsys}
\qquad
\rightarrow
\qquad
\begin{linsys}{3}
x&-&2y&=&3\\
 %
\begin{picture}(0,0)
\color{blue}\drawline(-14,-3)(-14,10)(12,10)(12,-3)
\end{picture}%
& &8y&=-8
\end{linsys}
\]
\subsubsection{Schritt I: $y$ isolieren}
Durch dividieren der zweiten Gleichung durch $8$ erreichen wir, dass
$y$ auf der linken Seite alleine steht.
Wir haben also bereits eine Unbekannte bestimmt:
\[
\begin{linsys}{3}
x-2\;y&=3\\
8\;y%
\begin{picture}(0,0)
\color{red}\put(-12,4){\circle{12}}
\end{picture}%
&=-8
\end{linsys}
\qquad
\rightarrow
\qquad
\begin{linsys}{3}
x&-&2\;y &=&3\\
&&y%
\begin{picture}(0,0)
\color{red}\put(-12,4){\circle{12}}
\end{picture}%
&=&-1
\end{linsys}
\]

\subsubsection{Schritt R: $y$ aus Vorgängergleichungen eliminieren%
\footnote{Der Buchstabe R kommt daher, dass diese Operationen später
Rückwärtseinsetzen genannt wird}}
Mit der zweiten Gleichung können wir jetzt auch das $y$ aus der ersten
Gleichung eliminieren.
Wir addieren dazu das Doppelte der zweiten
Gleichung zur ersten:
\[
\begin{linsys}{3}
x&-&2y%
\begin{picture}(0,0)
\color{blue}\drawline(-32,10)(-32,-4)(2,-4)(2,10)
\end{picture}%
&=&3\\
&&y&=&-1
\end{linsys}
\qquad
\rightarrow
\qquad
\begin{linsys}{3}
x&&%
\begin{picture}(0,0)
\color{blue}\drawline(-18,10)(-18,-4)(1,-4)(1,10)
\end{picture}%
&=&1\\
&&y&=&-1
\end{linsys}
\]
Damit ist das Gleichungssystem vollständig gelöst.

Offenbar ist es möglich, mit IFR-Schritten das Gleichungssystem zu lösen.
Die Variablen und Operationszeichen haben dabei nur Platzhalterfunktion
gehabt.
Wir könnten in allen Schritten auch nur die Koeffizienten
in einem sogenannten Tableau schreiben:
\index{Tableau}
\begin{gather*}
\begin{tabular}{|>{$}c<{$}>{$}c<{$}|>{$}c<{$}|}
\hline
\begin{picture}(0,0)
{\color{red}\put(2.5,4){\circle{12}}}
\end{picture}
3&-6&9\\
2&4&-2\\
\hline
\end{tabular}
\rightarrow
\begin{tabular}{|>{$}c<{$}>{$}c<{$}|>{$}c<{$}|}
\hline
{\color{red}1}&-2&3\\
2%
\begin{picture}(0,0)
\color{blue}\drawline(-8,-2)(-8,10)(1,10)(1,-2)
\end{picture}%
&4&-2\\
\hline
\end{tabular}
\rightarrow
\begin{tabular}{|>{$}c<{$}>{$}c<{$}|>{$}c<{$}|}
\hline
1&-2&3\\
{\color{blue}0}&\begin{picture}(0,0)
\color{red}\put(2.8,4){\circle{12}}
\end{picture}
8&-8\\
\hline
\end{tabular}
\rightarrow
\begin{tabular}{|>{$}c<{$}>{$}c<{$}|>{$}c<{$}|}
\hline
1&
\begin{picture}(0,0)
\color{blue}\drawline(1,9)(1,-2)(20,-2)(20,9)
\end{picture}%
-2&3\\
0&{\color{red}1}&-1\\
\hline
\end{tabular}
\rightarrow
\begin{tabular}{|>{$}c<{$}>{$}c<{$}|>{$}c<{$}|}
\hline
1&{\color{blue}0}&1\\
0&1&-1\\
\hline
\end{tabular}
\end{gather*}
Die drei Arten von Operationen können ausschliesslich über die
Koeffizientenschemata beschrieben werden:
\begin{itemize}
\item[\bf I:] Eine Variable isolieren bedeutet, die ganze
Zeile durch den Koeffizienten dieser Variable zu teilen.
\item[\bf F:] Eine Variable mit Hilfe einer anderen Zeile zu eliminieren
bedeutet, ein geeignetes Vielfaches dieser anderen Zeile zu subtrahieren,
so dass der Koeffizient zu $0$ wird.
\item[\bf R:] Eliminieren aus einer früheren Gleichung funktioniert
genau gleich wie {\bf F}.
\end{itemize}
Da {\bf F} und {\bf R} offenbar identisch sind, fassen wir diese
Operationen unter dem Namen {\bf E} zusammen.

\subsection{Operationen I und E für drei Unbekannte%
\label{skript:subsection:iedreiunbekannte}}
Als Beispiel führen wir das Verfahren auch noch für die drei Unbekannten
des Gleichungssystems
\begin{equation}
\begin{linsys}{3}
x&+&2y&+&3z&=&10\\
6x&+&5y&+&4z&=&32\\
x&-&y&+&z&=&2
\end{linsys}
\end{equation}
Die zugehörige Folge von Tableaus ist
\begin{align*}
\begin{tabular}{|>{$}c<{$}>{$}c<{$}>{$}c<{$}|>{$}c<{$}|}
\hline
\begin{picture}(0,0)
\color{red}\put(3,4){\circle{12}}
\end{picture}%
1&2&3&10\\
6&5&4&32\\
\begin{picture}(0,0)
\color{blue}\drawline(-3,-2)(-3,25)(9,25)(9,-2)
\end{picture}%
1&-1&1&2\\
\hline
\end{tabular}
&\overset{\text{\bf E}}\rightarrow
\begin{tabular}{|>{$}c<{$}>{$}c<{$}>{$}c<{$}|>{$}c<{$}|}
\hline
1&2&3&10\\
0&-7%
\begin{picture}(0,0)
\color{red}\put(-7,4){\circle{17}}
\end{picture}%
&-14&-28\\
0&-3%
\begin{picture}(0,0)
\color{blue}\drawline(-15,-3)(-15,10)(1,10)(1,-3)
\end{picture}%
&-2&-8\\
\hline
\end{tabular}
\\
&\overset{\text{\bf I,E}}\rightarrow
\begin{tabular}{|>{$}c<{$}>{$}c<{$}>{$}c<{$}|>{$}c<{$}|}
\hline
1&2&3&10\\
0&1&2&4\\
0&0&\begin{picture}(0,0)
\color{red}\put(3,4){\circle{12}}
\end{picture}%
4&4\\
\hline
\end{tabular}
\\
&\overset{\text{\bf I}}\rightarrow
\begin{tabular}{|>{$}c<{$}>{$}c<{$}>{$}c<{$}|>{$}c<{$}|}
\hline
1&2&3&10\\
0&1&\begin{picture}(0,0)
\color{blue}\drawline(-3,24)(-3,-3)(8,-3)(8,24)
\end{picture}%
2&4\\
0&0&1&1\\
\hline
\end{tabular}
\\
&\overset{\text{\bf R}}\rightarrow
\begin{tabular}{|>{$}c<{$}>{$}c<{$}>{$}c<{$}|>{$}c<{$}|}
\hline
1&\begin{picture}(0,0)
\color{blue}\drawline(-3,9)(-3,-3)(8,-3)(8,9)
\end{picture}%
2&0&7\\
0&1&0&2\\
0&0&1&1\\
\hline
\end{tabular}
\\
&\overset{\text{\bf R}}\rightarrow
\begin{tabular}{|>{$}c<{$}>{$}c<{$}>{$}c<{$}|>{$}c<{$}|}
\hline
1&0&0&3\\
0&1&0&2\\
0&0&1&1\\
\hline
\end{tabular}
\end{align*}
Daraus lesen wir die Lösung $x=3$, $y=2$ und $z=1$ ab.
Im ersten Schritt konnte man sich eine Operation {\bf I} sparen, weil
der Koeffizient von $x$ bereits $1$ ist.

\subsection{Der Normalfall}
Mit den Operationen {\bf I} und {\bf E} kann man das Gleichungssystem
also lösen.
Dabei geht man in zwei Phasen vor, die man wie folgt
visualisieren kann:
\begin{gather*}
\begin{tabular}{|>{$}c<{$}>{$}c<{$}>{$}c<{$}>{$}c<{$}|>{$}c<{$}|}
\hline
\begin{picture}(0,0)
\color{red}\put(2.7,3.1){\circle{12}}
\end{picture}%
*&*&\dots&*&*\\
*&*&\dots&*&*\\
\vdots& \vdots& \ddots& \vdots&\vdots\\
\begin{picture}(0,0)
\color{blue}\drawline(-3,-2)(-3,43)(8.5,43)(8.5,-2)
\end{picture}%
*&*&\dots&*&*\\
\hline
\end{tabular}
\rightarrow
\begin{tabular}{|>{$}c<{$}>{$}c<{$}>{$}c<{$}>{$}c<{$}|>{$}c<{$}|}
\hline
1&*&\dots&*&*\\
0&1&\dots&*&*\\
\vdots& \vdots& \ddots&\begin{picture}(0,0)
\color{blue}\drawline(-4,43)(-4,-3)(7,-3)(7,43)
\end{picture}%
\vdots&\vdots\\
0&0&\dots&1&*\\
\hline
\end{tabular}
\rightarrow
\begin{tabular}{|>{$}c<{$}>{$}c<{$}>{$}c<{$}>{$}c<{$}|>{$}c<{$}|}
\hline
1&0&\dots&0&*\\
0&1&\dots&0&*\\
\vdots& \vdots& \ddots& \vdots&\vdots\\
0&0&\dots&1&*\\
\hline
\end{tabular}
\end{gather*}
\index{Vorwaertsreduktion@Vorwärtsreduktion}
\index{Rueckwaertseinsetzen@Rückwärtseinsetzen}
Die erste Phase wird auch Vorwärtsreduktion genannt, die
zweite Phase Rückwärtseinsetzen.
\subsubsection{Vorwärtsreduktion}
In der ersten Phase muss man also alle Elemente unterhalb der
Diagonalen zu Null machen.
Mit Operationen {\bf I} und {\bf E} geht man dazu zeilenweise vor.
In jeder Zeile macht man mit {\bf I}
zunächst den vordersten nicht verschwindenden Koeffizienten, das sogenannte
Pivot-Element zu $1$.
\index{Pivot-Element}
Die Zeile und Spalte dieses Elements heisst deshalb
auch Pivot-Zeile bzw.~Pivot-Spalte.
\index{Pivot-Zeile}
\index{Pivot-Spalte}
Dann subtrahiert man mit der Operation {\bf E} geeignete Vielfache
der Pivot-Zeile, so dass in der Spalte
unter der eben entstandenen $1$ alle Koeffizienten verschwinden.
\begin{gather*}
\begin{tabular}{|>{$}c<{$}>{$}c<{$}>{$}c<{$}>{$}c<{$}>{$}c<{$}|>{$}c<{$}|}
\hline
1&*&*&\dots&*&*\\
0&\begin{picture}(0,0)
\color{red}\put(2.8,3.05){\circle{12}}
\end{picture}%
*&*&\dots&*&*\\
0&*&*&\dots&*&*\\
\vdots&\vdots& \vdots& \ddots& \vdots&\vdots\\
0&*&*&\dots&*&*\\
\hline
\end{tabular}
\overset{\text{\bf I}}\rightarrow
\begin{tabular}{|>{$}c<{$}>{$}c<{$}>{$}c<{$}>{$}c<{$}>{$}c<{$}|>{$}c<{$}|}
\hline
1&*&*&\dots&*&*\\
0&1&*&\dots&*&*\\
0&*&*&\dots&*&*\\
\vdots&\vdots& \vdots& \ddots& \vdots&\vdots\\
0&\begin{picture}(0,0)
\color{blue}\drawline(-2,-2)(-2,43)(8,43)(8,-2)
\end{picture}%
*&*&\dots&1&*\\
\hline
\end{tabular}
\overset{\text{\bf E}}\rightarrow
\begin{tabular}{|>{$}c<{$}>{$}c<{$}>{$}c<{$}>{$}c<{$}>{$}c<{$}|>{$}c<{$}|}
\hline
1&*&*&\dots&*&*\\
0&1&*&\dots&*&*\\
0&0&*&\dots&*&*\\
\vdots& \vdots&\vdots& \ddots& \vdots&\vdots\\
0&0&*&\dots&*&*\\
\hline
\end{tabular}
\end{gather*}
In Formeln wird im Schritt {\bf I} auf der Zeile $i$
durch $a_{ii}$ dividiert:
\begin{align*}
a_{ij}'&=\frac{a_{ij}}{a_{ii}}&& i < j \le m,\\
b_i'&=\frac{b_i}{a_{ii}}
\end{align*}
Die Operation {\bf E} auf der Zeile $k>i$ ist
\begin{align*}
a_{kj}'&=a_{kj}-a_{ki}a_{ij}&&i < k \le m, i \le j\le n\\
b_k'&=b_k-a_{ki}b_i&&i < k \le m.
\end{align*}
Führt man beide Operationen in einem Schritt durch, erhält man die
Gleichungen:
\begin{equation}
\begin{aligned}
a_{ij}'&=\frac{a_{ij}}{a_{ii}}&&&b_i'&=\frac{b_i}{a_{ii}}\\
a_{kj}'&=a_{kj}-a_{ki}\frac{a_{ij}}{a_{ii}}&
&\qquad&b_k'&=b_k-a_{ki}\frac{a_{ij}}{a_{ii}}
\end{aligned}
\label{vorwaertsreduktion}
\end{equation}
mit $i < k\le m$ und $i\le j\le n$. Dieser Schritt muss für $i=1,\dots,m$
wiederholt werden.

\subsubsection{Rückwärtseinsetzen}
Das Rückwärtseinsetzen kommt ohne die {\bf I}-Operationen aus.
Es eliminiert die nicht verschwinden Element oberhalb der Diagonalen
Spaltenweise, beginnend bei der letzten.
Für die dritte Spalte beispielsweise bedeutet dies bildlich:
\begin{gather*}
\begin{tabular}{|>{$}c<{$}>{$}c<{$}>{$}c<{$}>{$}c<{$}>{$}c<{$}|>{$}c<{$}|}
\hline
1&*&*&\dots&0&*\\
0&1&\begin{picture}(0,0)
\color{blue}\drawline(-2,22)(-2,-2)(8,-2)(8,22)
\end{picture}%
*&\dots&0&*\\
0&0&1&\dots&0&*\\
\vdots&\vdots& \vdots& \ddots& \vdots&\vdots\\
0&0&0&\dots&1&*\\
\hline
\end{tabular}
\overset{\text{\bf E}}\rightarrow
\begin{tabular}{|>{$}c<{$}>{$}c<{$}>{$}c<{$}>{$}c<{$}>{$}c<{$}|>{$}c<{$}|}
\hline
1&*&0&\dots&0&*\\
0&1&0&\dots&0&*\\
0&0&1&\dots&0&*\\
\vdots& \vdots&\vdots& \ddots& \vdots&\vdots\\
0&0&0&\dots&1&*\\
\hline
\end{tabular}
\end{gather*}
In Formeln bedeutet dies für die {\bf E}-Operation, bei der die
Spalte mit der Nummer $i$ ``leer geräumt'' wird:
\begin{align}
a_{kj}'&=a_{kj}- a_{ki} a_{ij}
&
b_k'&=b_k- a_{ki}b_i
\label{rueckwaertseinsetzen}
\end{align}
Dieser Schritt muss für $i=m,\dots,2$ wiederholt werden.
\subsection{Sonderfälle\label{subsection:sonderfaelle}}
Nicht immer kann das Verfahren so problemlos durchgeführt werden.
Es kann vorkommen, dass eines der Pivot-Elemente $a_{ii}=0$ ist.
Da beim Vorwärtsreduzieren immer durch das Pivot-Element geteilt
werden muss, bricht das Verfahren in diesem Falle ab, möglicherweise
obwohl das Gleichungssystem lösbar ist.
Zwei Strategien sind denkbar, mit dieser Situation umzugehen.

Nicht nur ein verschwindendes Pivot-Element führt zu Problemen,
auch die Division durch ein kleines Pivot-Element kann bereits die
Rechengenauigkeit un\-günstig beeinflussen.
Daher kann es je nach
Gleichungssystem geraten sein, eine der folgenden Strategien anzuwenden,
um die Genauigkeit der Resultate zu verbessern.
\subsubsection{Gleichungen umordnen}
Die Reihenfolge der Gleichungen hat keinen Einfluss auf die Lösungsmenge.
Ist $a_{ii}=0$, kann man jede der Gleichungen $i+1$ bis $m$ an Stelle
der Gleichung Nummer $i$ verwenden.

\begin{beispiel}[\bf Beispiel] Im Gleichungssystem $y=1, x=2$ versagt
in der Standardreihenfolge von Gleichungen und Variablen bereits der
erste Schritt:
\begin{gather*}
\begin{tabular}{|>{$}c<{$}>{$}c<{$}|>{$}c<{$}|}
\hline
0%
\begin{picture}(0,0)
\color{red}\put(-3,4){\circle{12}}
\put(4,0){!}
\end{picture}%
&1&1\\
1&0&2\\
\hline
\end{tabular}
\end{gather*}
Vertauscht man aber die beiden Zeilen, wird daraus
\begin{gather*}
\begin{tabular}{|>{$}c<{$}>{$}c<{$}|>{$}c<{$}|}
\hline
\begin{picture}(0,0)
\color{red}\put(2.7,4){\circle{12}}
\end{picture}%
1&0&2\\
0&1&1\\
\hline
\end{tabular}
\end{gather*}
was nicht nur lösbar, sondern bereits gelöst ist.
\end{beispiel}

\subsubsection{Variablen umordnen}
Die Reihenfolge der Variablen hat keinen Einfluss auf die Lösungsmenge.
Ist $a_{ii}=0$ kann jede andere Spalte $i+1$ bis $n$ Anstelle der
Spalte mit der Nummer $i$ verwendet werden.
Allerdings muss man über
die Vertauschungen von Spalten Buch führen, zum Beispiel indem man in
einer zusätzlichen Zeile die Bedeutung der Spalten mitschreibt.

\begin{beispiel}[\bf Beispiel]
Wir versuchen, das Gleichungssystem
\[
\begin{linsys}{3}
   &   & y & - & z & = & 2\\
   &   & y & + & z & = & 4\\
-x &   &   & + & z & = & 2
\end{linsys}
\]
zu lösen.
Zweimal während der Durchführung des Gauss-Algorithmus,
gekennzeichnet durch {\color{red}!},
sind wir gezwungen%
, durch eine Spaltenvertauschung dafür zu
sorgen, dass das Pivot-Element nicht $0$ ist%
\footnote{Es wäre natürlich auch möglich,
mit Zeilenvertauschungen das gleiche Ziel zu erreichen, doch soll
dieses Beispiel zeigen, wie man Spaltenvertauschungen für diesen
Zweck einsetzen kann.}%
.
\begin{align*}
\begin{tabular}{|>{$}c<{$}>{$}c<{$}>{$}c<{$}|>{$}c<{$}|}
\hline
x&y&z&\\
\hline
0%
\begin{picture}(0,0)
\color{red}\put(-3,4){\circle{12}}
\put(5,0){!}
\end{picture}%
&1&-1&2\\
0&1&1&4\\
-1&0&1&2\\
\hline
\end{tabular}
&\overset{\displaystyle *}{\rightarrow}
\begin{tabular}{|>{$}c<{$}>{$}c<{$}>{$}c<{$}|>{$}c<{$}|}
\hline
y&x&z&\\
\hline
\begin{picture}(0,0)
\color{red}\put(3,4){\circle{12}}
\end{picture}%
1&0&-1&2\\
1&0&1&4\\
\begin{picture}(0,0)
\color{blue}\drawline(-2,-2)(-2,25)(8,25)(8,-2)
\end{picture}%
0&-1&1&2\\
\hline
\end{tabular}
\rightarrow
\begin{tabular}{|>{$}c<{$}>{$}c<{$}>{$}c<{$}|>{$}c<{$}|}
\hline
y&x&z&\\
\hline
1&0&-1&2\\
0&0%
\begin{picture}(0,0)
\color{red}\put(-3,4){\circle{12}}
\put(5,0){!}
\end{picture}%
&2&2\\
0&-1&1&2\\
\hline
\end{tabular}
\\
&\overset{\displaystyle *}{\rightarrow}
\begin{tabular}{|>{$}c<{$}>{$}c<{$}>{$}c<{$}|>{$}c<{$}|}
\hline
y&z&x&\\
\hline
1&-1&0&2\\
0& 2%
\begin{picture}(0,0)
\color{red}\put(-3,3.5){\circle{12}}
\end{picture}%
&0&2\\
0&1\begin{picture}(0,0)
\color{blue}\drawline(-8,-2)(-8,10)(2,10)(2,-2)
\end{picture}%
&-1&2\\
\hline
\end{tabular}
\rightarrow
\begin{tabular}{|>{$}c<{$}>{$}c<{$}>{$}c<{$}|>{$}c<{$}|}
\hline
y&z&x&\\
\hline
1&-1&0&2\\
0&1& 0&1\\
0&0&-1%
\begin{picture}(0,0)
\color{red}\put(-7,4){\circle{15}}
\end{picture}%
&1\\
\hline
\end{tabular}
\\
&\rightarrow
\begin{tabular}{|>{$}c<{$}>{$}c<{$}>{$}c<{$}|>{$}c<{$}|}
\hline
y&z&x&\\
\hline
1&-1%
\begin{picture}(0,0)
\color{blue}\drawline(-15,10)(-15,-2)(1,-2)(1,10)
\end{picture}%
&0&2\\
0&1& 0
\begin{picture}(0,0)%
\color{blue}\drawline(-10,24)(-10,-2)(3,-2)(3,24)%
\end{picture}%
&1\\
0&0&1&-1\\
\hline
\end{tabular}
\rightarrow
\begin{tabular}{|>{$}c<{$}>{$}c<{$}>{$}c<{$}|>{$}c<{$}|}
\hline
y&z&x&\\
\hline
1&0&0&3\\
0&1&0&1\\
0&0&1&-1\\
\hline
\end{tabular}
\\
&\overset{\displaystyle *}{\rightarrow}
\begin{tabular}{|>{$}c<{$}>{$}c<{$}>{$}c<{$}|>{$}c<{$}|}
\hline
x&y&z&\\
\hline
1&0&0&-1\\
0&1&0&3\\
0&0&1&1\\
\hline
\end{tabular}
\end{align*}
Im ersten Schritt wurden die ersten beiden
Spalten vertauscht, im dritten Schritt die letzten
beiden.
Im letzten Schritt wurden die Variablen
und Gleichungen wieder in die ``Standardreihenfolge'' gebracht.
Die Lösung ist $(x,y,z)=(-1, 3, 1)$.
\end{beispiel}

