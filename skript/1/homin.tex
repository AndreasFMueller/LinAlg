%
% homin.tex -- homogene und inhomogene Gleichungssysteme
%
% (c) 2009 Prof Dr Andreas Mueller, Hochschule Rapperswil
%
\section{Homogene und Inhomogene Gleichungssysteme}
\index{homogen}
\index{inhomogen}
Bei der Untersuchung der linearen Abhängigkeit haben wir aus einem
Gleichungssystem mit beliebigen rechten Seiten ein neues Gleichungssystem
mit ausschliesslich Nullen auf der rechten Seite abgeleitet.
Es gibt also einen Zusammenhang zwischen der Lösbarkeit des Systems 
$Ax=b$ mit $b\ne 0$ und $Ax=0$, die wir jetzt genauer untersuchen
wollen.
\begin{definition}
Sei $A$ eine $m\times n$-Matrix, und $b\ne 0$ ein $m$-dimensionaler
Spaltenvektor.
Das Gleichungssystem mit nicht verschwindender rechten Seite
\[
Ax=b\qquad\text{heisst inhomogen,}
\]
das Gleichungssystem mit Nullen auf der rechten Seite 
\[
Ax=0\qquad\text{heisst homogen.}
\]
\end{definition}
\subsection{Reguläre Koeffizientenmatrix}
Ist $A$ regulär, dann hat das Gleichungssystem $Ax=b$ für jede beliebige
rechte Seite genau eine Lösung, insbesondere auch die Gleichung $Ax=0$.
Die Lösung kann mit der inversen Matrix berechnet werden, also $x=A^{-1}b$,
im Falle des homogenen Gleichungssystems ist die einzige Lösung also
die Nulllösung $x=0$.

\subsection{Singuläre Koeffizientenmatrix}
Ist $A$ singular, sind zwei Alternativen möglich: entweder hat das
Gleichungssystem gar keine Lösung, oder es hat unendlich viele.
\begin{align*}
Ax&=b&\qquad&\begin{cases}
\text{keine Lösung}\\
\text{unendlich viele Lösungen}
\end{cases}
\\
Ax&=0&\qquad&\begin{cases}
\text{keine Lösung}&\text{$\Rightarrow$ kann wegen $A0=0$ nicht eintreten!}\\
\text{unendlich viele Lösungen}&
\end{cases}
\end{align*}
Ein homogenes Gleichungssystem mit singulärer Koeffizientenmatrix hat also
immer unendlich viele Lösungen.
Wir bezeichnen die Lösungsmenge des Systems $Ax=0$ mit $\mathbb L_h$:
\[
\mathbb L_h=\{x|Ax=0\}.
\]

Ist jetzt $x_p$ eine beliebige Lösung des inhomogenen Systems $Ax=b$, dann
können wir weiter Lösungen finden, indem wir Lösungen $x_h\in\mathbb L_h$
dazuaddieren:
\[
A(x_p+x_h)=Ax_p+Ax_h=b+0=b,
\]
also ist $x_p+x_h$ auch wieder eine Lösung.
Umgekehrt ist die Differenz
zwischen zwei Lösungen $x_p$ und $x_p'$ der inhomogenen Gleichung in
$\mathbb L_h$:
\[
A(x_p-x_p')=Ax_p-Ax_p'=b-b=0\quad\Rightarrow\quad (x_p-x_p')\in\mathbb L_h.
\]
Damit haben wir folgendes Rezept zur Bestimmung der Lösungsmenge des
inhomogenen Gleichungssystems gefunden
\begin{satz}
\index{partikulaer@partikulär}
Ist $x_p$ eine spezielle Lösung des inhomogenen Gleichungssystems $Ax=b$,
auch partikuläre Lösung genannt,
dann ist die Lösungsmenge dieses Gleichungssystems 
\[
\mathbb L=\{x_p+x_h|x_h\in\mathbb L_h\},
\]
wobei $\mathbb L_h$ die Lösungsmenge des homogenen Gleichungssystems ist.
\end{satz}
Der Satz leistet eine Aufteilung des Problems, alle Lösungen zu finden,
in die beiden Teilprobleme
\begin{enumerate}
\item Finden einer einzigen Lösung des inhomogenen Systems.
\item Finden der Lösungsmenge des homogenen Systems.
\end{enumerate}
Dieses Muster trifft man an vielen Orten in der Mathematik wieder, zum
Beispiel bei der Lösung gewöhnlicher Differentialgleichungen.
Oft lassen sich die beiden Teilprobleme mit spezialisierten Techniken
viel einfacher lösen.

Das Problem, die Lösungsmenge des homogenen Systems zu bestimmen wird
dadurch vereinfacht, dass man in dieser Lösungsmenge rechnen kann.
Sind $x$ und $x'$  in $\mathbb L_h$, dann sind auch deren Summe und
die Vielfachen in $\mathbb L_h$:
\begin{align*}
A(x+x')&=Ax+Ax'=0+0=0&\Rightarrow&&x+x'&\in\mathbb L_h\\
A(\lambda x)&=\lambda Ax=\lambda 0=0&\Rightarrow&&\lambda x&\in \mathbb L_h
\end{align*}
Hat man also erst mal ein paar Vektoren in $\mathbb L_h$ gefunden, kann
man unendlich viele weitere konstruieren, indem man alle Vielfachen und Summen
bildet.

