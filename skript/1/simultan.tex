%
% simultan.tex -- mehrere Gleichungssysteme gleichzeitig lösen
%
% (c) 2009 Prof Dr Andreas Mueller, Hochschule Rapperswil
%
\section{Mehrere rechte Seiten\label{section:mehrere rechte seiten}}
\rhead{Simultane Lösung mit mehreren rechten Seiten}
Bis jetzt wurde der Gauss-Algorithmus immer nur auf Gleichungen mit
einer konstanten rechten Seite angewendet.
Der Gauss-Algorithmus kann noch wesentlich mehr.
Er kann in einem Durchgang ein Gleichungssysteme für mehrere verschiedene
rechte Seiten lösen, dies wird in Abschnitt~\ref{simultan} gezeigt.
Er kann aber auch Gleichungssysteme lösen, in denen die rechte
Seite eine Linearform ist (Abschnitt~\ref{subsection:rechte seite linear}).

%
% Mehrere Gleichungssysteme simultan lösen
%
\subsection{Mehrere Gleichungssysteme simultan lösen\label{simultan}}
Mit dem Gauss-Algorithmus kann man gleichzeitig mehrere Gleichungssysteme
lösen, die sich nur in den rechten Seiten unterscheiden.
Die rechten Seiten haben nämlich auf den Gang der Rechnung keinen
Einfluss.
So kann man die beiden Gleichungssysteme
\begin{align*}
\frac12x+\frac{\sqrt{3}}2y&=1
&
\frac12x+\frac{\sqrt{3}}2y&=0
\\
-\frac{\sqrt{3}}2x+\frac12y&=0
&
-\frac{\sqrt{3}}2x+\frac12y&=1
\end{align*}
in einem Durchgang lösen:
\begin{align*}
\begin{tabular}{|>{$}c<{$}>{$}c<{$}|>{$}c<{$}>{$}c<{$}|}
\hline
\frac12%
\begin{picture}(0,0)
\color{red}\put(-3.5,3){\circle{15}}
\end{picture}%
&\frac{\sqrt{3}}2&1&0\\
-\frac{\sqrt{3}}2&\frac12&0&1\\
\hline
\end{tabular}
&\rightarrow
\begin{tabular}{|>{$}c<{$}>{$}c<{$}|>{$}c<{$}>{$}c<{$}|}
\hline
1&\sqrt{3}&2&0\\
-\frac{\sqrt{3}}2%
\begin{picture}(0,0)
\color{blue}\drawline(-22,-4)(-22,13)(1,13)(1,-4)
\end{picture}%
&\frac12&0&1\\
\hline
\end{tabular}
\rightarrow
\begin{tabular}{|>{$}c<{$}>{$}c<{$}|>{$}c<{$}>{$}c<{$}|}
\hline
1&\sqrt{3}&2&0\\
0&2%
\begin{picture}(0,0)
\color{red}\put(-3,4){\circle{12}}
\end{picture}%
&\sqrt{3}&1\\
\hline
\end{tabular}
\\
&\rightarrow
\begin{tabular}{|>{$}c<{$}>{$}c<{$}|>{$}c<{$}>{$}c<{$}|}
\hline
1&\sqrt{3}%
\begin{picture}(0,0)
\color{blue}\drawline(-16,11)(-16,-3)(1,-3)(1,11)
\end{picture}%
&2&0\\
0&1&\frac{\sqrt{3}}2&\frac12\\
\hline
\end{tabular}
\rightarrow
\begin{tabular}{|>{$}c<{$}>{$}c<{$}|>{$}c<{$}>{$}c<{$}|}
\hline
1&0&\frac12&-\frac{\sqrt{3}}2\\
0&1&\frac{\sqrt{3}}2&\frac12\\
\hline
\end{tabular}
\end{align*}
Das erste Gleichungssystem hat also die Lösung
$(x,y)=(\frac12,\frac{\sqrt{3}}2)$,
das zweite
$(x,y)=(-\frac{\sqrt{3}}2, \frac12)$.

Die spezielle Wahl der rechten Seiten erlaubt auch, aus den
beiden gefundenen Lösungen die Lösung für jede beliebige rechte
Seite zusammenzusetzen. Wenn nämlich auf der rechten
Seite die Zahlen $b_1$ und $b_2$ stehen sollen, dann
bedeutet dies offenbar, dass wir $b_1$ mal die erste Lösung
und $b_2$ mal die zweite Lösung zusammennehmen müssen, um
die verlangten rechten Seiten zu finden.
Also
\[
x=\frac12b_1-\frac{\sqrt{3}}2b_2,\qquad y=\frac{\sqrt{3}}2b_1+\frac12b_2.
\]
Durch Einsetzen in die ursprüngliche Gleichung kann man die
Lösung überprüfen.

%
% Linearform als rechte Seite
%
\subsection{Linearform als rechte Seite\label{subsection:rechte seite linear}}
Das Gauss-Tableau ist nur eine abgekürzte Schreibweise für ein Gleichungssystem.
In Abschnitt~\ref{subsection:sonderfaelle} haben wir die Spalten mit
den Variablennamen beschriftet, so dass wir sogar die Variablennamen
aus dem Tableau wieder bekommen können.
Die Übersetzungsregel können wir schematisch als
\[
\begin{tabular}{|>{$}c<{$} >{$}c<{$} >{$}c<{$} >{$}c<{$}|}
\hline
x_1&x_2&\dots&x_n\\
\hline
a_{11}&a_{12}&\dots&a_{1n}\\
\vdots&\vdots&\ddots&\vdots\\
a_{m1}&a_{m2}&\dots&a_{mn}\\
\hline
\end{tabular}
\quad
\leftrightarrow
\quad
\begin{linsys}{3}
a_{11}x_1&+&a_{12}x_2&+&\dots &+&a_{1n}x_n\\
\vdots\hspace*{1em} & &\vdots\hspace*{1em} & &\ddots& &\vdots\hspace*{1em}\\
a_{m1}x_1&+&a_{m2}x_2&+&\dots &+&a_{mn}x_n\\
\end{linsys}
\]
darstellen.
Damit lässt sich aber auch ein Gleichungssystem in ein Tableau übersetzen,
welches Linearformen auf der rechten Seite hat.
Das Gleichheitszeichen wird zum Vertikalstrich wie früher und die
Linearformen auf der rechten Seite werden zu Spalten, die mit den
Variablen der Linearformen beschriftet sind.

\begin{beispiel}
\label{skript:lingl:simultan-beispiel}
Als Beispiel übersetzen wir
\begin{equation}
\begin{linsys}{4}
3x_1&+&4x_2&=& 9y_1&+4y_2\\
5x_1&+&7x_2&=&20y_1&+9y_2
\end{linsys}
\quad\leftrightarrow\quad
\begin{tabular}{| >{$}c<{$} >{$}c<{$}| >{$}c<{$} >{$}c<{$}|}
\hline
x_1&x_2&y_1&y_3\\
\hline
3&4&9&4\\
5&7&20&9\\
\hline
\end{tabular}
\label{skript:rechtslinearubers}
\end{equation}

Die Lösung des Gleichungssystems~\eqref{skript:rechtslinearubers} 
funktioniert mit dem Gaussalgorithmus wie bisher.
Wir führen den Gauss-Algorithmus durch:
\begin{align*}
\begin{tabular}{| >{$}c<{$} >{$}c<{$}| >{$}c<{$} >{$}c<{$}|}
\hline
x_1&x_2&y_1&y_3\\
\hline
3&4&9&4\\
5&7&20&9\\
\hline
\end{tabular}
&\rightarrow
\begin{tabular}{| >{$}c<{$} >{$}c<{$}| >{$}c<{$} >{$}c<{$}|}
\hline
x_1&x_2&y_1&y_3\\
\hline
1&\frac{4}{3}&3&\frac{4}{3}\\
0&\frac{1}{3}&5&\frac{7}{3}\\
\hline
\end{tabular}
\rightarrow
\begin{tabular}{| >{$}c<{$} >{$}c<{$}| >{$}c<{$} >{$}c<{$}|}
\hline
x_1&x_2&y_1&y_3\\
\hline
1&0&-17&-8\\
0&1& 15& 7\\
\hline
\end{tabular}
\end{align*}
Die Rückübersetzung davon ergibt
\[
\begin{linsys}{3}
x_1& &   &=&-17y_1&-&8y_2\\
   & &x_2&=& 15y_1&+&7y_2
\end{linsys}
\]
Setzen wir diese Linearformen für $x_1$ und $x_2$ in die ursprünglichen
Gleichungen~\eqref{skript:rechtslinearubers} ein, erhalten
wir
\begin{align*}
3x_1+4x_2
&=
3(-17y_1-8y_2)+4(15y_1+7y_2)
=
-51y_1-24y_2+60y_1+28y_2
=
9y_1+4y_2
\\
5x_1+7x_1
&=
5(-17y_1-8y_2)+7(15y_1+7y_2)
=
-85y_1-40y_2+105y_1+49y_2
=
20y_1+9y_2,
\end{align*}
also genau die ursprünglichen Gleichungen.
Der Gauss-Algorithmus hat also tatsächlich die Lösungen des
Gleichungssystems~\eqref{skript:rechtslinearubers} als
Linearformen in den Variablen $y_1$ und $y_2$ der rechten Seite
geliefert.
\end{beispiel}

Dies gilt natürlich ganz allgemein.
Auf der rechten Seite des Tableaus finden während des Gauss-Algorithmus
nur Multiplikationen mit Zahlen und Additionen statt.
Diese Operationen kann man getrennt für jeden einzelnen Term der
Form $b_{ij}y_j$ auf der rechten Seite durchführen.
In diesem Lichte sind die Spalten auf der rechten Seite des Tableaus
nur ein ``Buchhaltungshilfsmittel'', mit dem wir sicherstellen, dass
die Variablen auf der rechten Seite nicht durcheinander geraten.

Es gibt keinen Grund, warum die Anzahl der Variablen auf der rechten
Seite genau $n$ sein soll.
Ganz allgemein können wir ein Gleichungssystem mit $m$ Gleichungen
und $n$ Unbekannten sowie $l$ Variablen auf der  rechten Seite
in ein Tableau mit $m$ Zeilen und $n$ Spalten links und $l$
Spalten rechts vom Vertikalstrich übersetzen:
\[
\begin{linsys}{5}
a_{11}x_1&+&\dots&+&a_{1n}x_n&=&b_{11}y_1&+&\dots&+&b_{1l}y_l\\
\vdots*\hspace{1em}&&\ddots&&\vdots\hspace*{1em}&&\vdots\hspace*{1em}&&\ddots&&\vdots\hspace*{1em}\\
a_{m1}x_1&+&\dots&+&a_{mn}x_n&=&b_{m1}y_1&+&\dots&+&b_{ml}y_l\\
\end{linsys}
\quad\rightarrow\quad
\begin{tabular}{| >{$}c<{$}>{$}c<{$}>{$}c<{$}|>{$}c<{$}>{$}c<{$}>{$}c<{$}|}
\hline
x_1&\dots&x_n&y_1&\dots&y_l\\
\hline
a_{11}&\dots&a_{1n}&b_{11}&\dots&b_{1l}\\
\vdots&\ddots&\vdots&\vdots&\ddots&\vdots\\
a_{m1}&\dots&a_{mn}&b_{m1}&\dots&b_{ml}\\
\hline
\end{tabular}
\]
Das Schlusstableau nach Anwendung des Gauss-Algorithmus ist dann
\begin{equation}
\begin{tabular}{|>{$}c<{$}>{$}c<{$}>{$}c<{$}:>{$}c<{$}>{$}c<{$}>{$}c<{$}|
>{$}c<{$}>{$}c<{$}>{$}c<{$}|}
\hline
x_{i_1}&\dots&x_{i_r}
	&x_{i_{r+1}}&\dots&x_{i_n}
		&y_1&\dots&y_l\\
\hline
1&\dots &0     
	&\color{green}c_{1,r+1}&\color{green}\dots &\color{green}c_{1n}
		&d_{11}&\dots&d_{1l}\\
\vdots&\ddots&\vdots
	&\color{green}\vdots&\color{green}\ddots&\color{green}\vdots
		&\vdots&\ddots&\vdots\\
0&\dots &1
	&\color{green}c_{r,r+1}&\color{green}\dots &\color{green}c_{rn}
		&d_{r1}&\dots&d_{rl}\\
\hdashline
0&\dots &0     
	&0&\dots &0
		&\color{red}d_{r+1,1}&\color{red}\dots&\color{red}d_{r+1,l}\\
\vdots&\ddots&\vdots
	&\vdots&\ddots&\vdots
		&\color{red}\vdots&\color{red}\ddots&\color{red}\vdots\\
0&\dots &0
	&0&\dots &0
		&\color{red}d_{m1}&\color{red}\dots&\color{red}d_{ml}\\
\hline
\end{tabular}
\label{skript:simultan:schluss}
\end{equation}
Die Zahl $r$ ist der Rang der Matrix $A$.
Dabei ist zu beachten, dass Infolge möglicher Spaltenvertauschungen
die Variablen $x_i$ nicht mehr ihre ursprünglichen Plätze haben.
Dies wird durch die neuen Indizes $i_1,\dots,i_n$
ausgedrückt.

Die {\color{red}roten} Koeffizienten bestimmen, ob das Gleichungssystem
überhaupt eine Lösung hat.
Wenn eine der Linearformen gebildet aus den {\color{red}roten} Koeffizienten
einen Wert $\mathstrut \ne 0$ liefert für die gegebenen Werte der
$y$-Variablen, dann ist das Gleichungssystem nicht lösbar.
Wenn man also wissen will, für welche Werte von $y_i$ das Gleichungssystem
überhaupt lösbar ist, muss man das kleinere Gleichungssystem
\[
\begin{linsys}{3}
{\color{red}d_{r+1,1}}y_1&+&\dots&+&{\color{red}d_{r+1,l}}y_l&=&0\\
\vdots\hspace*{1em}&&\ddots&&\vdots\hspace*{1em}&&\vdots\hspace*{0.7mm}\\
{\color{red}d_{m1}}y_1&+&\dots&+&{\color{red}d_{ml}}y_l&=&0\\
\end{linsys}
\]
lösen, was natürlich wiederum mit dem Gauss-Algorithmus geschehen kann.
Insbesondere ist es für $l>m-r$ immer möglich, eine Kombination von $y_i$
zu finden, für die der {\color{red}rote} Teil des Gleichungssystems
verschwindet.

Wenn der {\color{red}rote} Teil verschwindet, hat das Gleichungssystem
mindestens eine Lösung.
Wenn $n>r$ ist, dann sind die {\color{green}grünen} Variablen $x_{i_{r+1}}$
bis $x_{i_n}$ frei wählbar, es gibt also unendlich viele Lösungen.
Die ersten $r$ Variablen $x_{i_1}$ bis $x_{i_r}$ sind dann durch die
{\color{green}frei wählbaren} Variablen eindeutig bestimmt.
Aus dem Tableau~\eqref{skript:simultan:schluss} kann man die Gleichungen
\[
\begin{linsys}{6}
x_{i_1}&=&d_{11}y_1&+&\dots&+&d_{1l}y_l&-&{\color{green}c_{1,r+1}}x_{i_{r+1}}&-&\dots&-&{\color{green}c_{1n}}x_{i_n}\\
\vdots\hspace*{0.5em} && \vdots\hspace*{1em} && \ddots && \vdots\hspace*{1em} && \vdots\hspace*{1em} && \ddots && \vdots\hspace*{1em}\\
x_{i_r}&=&d_{r1}y_1&+&\dots&+&d_{rl}y_l&-&{\color{green}c_{r,r+1}}x_{i_{r+1}}&-&\dots&-&{\color{green}c_{rn}}x_{i_n}\\
\end{linsys}
\]
für die Variablen $x_{i_1}$ bis $x_{i_r}$ ablesen.
Dies lässt sich natürlich auch in der Form einer Lösungsmenge schreiben.

%
% Standardbasisvektoren
%
\subsection{Standardbasisvektoren als rechte Seiten%
\label{skript:subsection:standardrs}}
Ein interessanter Spezialfall liegt vor, wenn man ein Gleichungssystem
für die rechten Seiten 
\[
e_1=\begin{pmatrix}1\\0\\0\end{pmatrix},\qquad
e_2=\begin{pmatrix}0\\1\\0\end{pmatrix},\qquad
e_3=\begin{pmatrix}0\\0\\1\end{pmatrix}
\]
bereits gelöst und die Lösungsvektoren
\[
c_1=\begin{pmatrix} c_{11}\\ c_{21}\\ c_{31} \end{pmatrix},\qquad
c_2=\begin{pmatrix} c_{12}\\ c_{22}\\ c_{32} \end{pmatrix},\qquad
c_3=\begin{pmatrix} c_{13}\\ c_{23}\\ c_{33} \end{pmatrix}
\]
gefunden hat.
Da die allgemeine rechte Seite
\[
b=\begin{pmatrix}
b_1\\b_2\\b_3
\end{pmatrix}
=
b_1 \begin{pmatrix}1\\0\\0\end{pmatrix}
+
b_2 \begin{pmatrix}0\\1\\0\end{pmatrix}
+
b_3 \begin{pmatrix}0\\0\\1\end{pmatrix}
=
b_1 e_1 + b_2 e_2 + b_3 e_3
\]
als Linearkombination der Standardbasisvektoren $e_i$ geschrieben
werden kann, kann man die Lösung des Gleichungssystems mit rechter
Seite $b$ auch als ein Beispiel für ein Gleichungssystem mit rechten
Seiten betrachten, die Linearformen in den `Variablen' $b_i$ sind.
Das Tableau dazu ist
\[
\begin{tabular}{|>{$}c<{$}>{$}c<{$}>{$}c<{$}|>{$}c<{$}|}
\hline
x_1&x_2&x_3&\\
\hline
a_{11}&a_{12}&a_{13}&b_1\\
a_{21}&a_{22}&a_{23}&b_2\\
a_{31}&a_{32}&a_{33}&b_3\\
\hline
\end{tabular}
\quad\rightarrow\quad
\begin{tabular}{|>{$}c<{$}>{$}c<{$}>{$}c<{$}|>{$}c<{$}>{$}c<{$}>{$}c<{$}|}
\hline
x_1&x_2&x_3&b_1&b_2&b_3\\
\hline
a_{11}&a_{12}&a_{13}&1&0&0\\
a_{21}&a_{22}&a_{23}&0&1&0\\
a_{31}&a_{32}&a_{33}&0&0&1\\
\hline
\end{tabular}
\]
Wendet man darauf den Gauss-Algorithmus an, bekommt man das
Tableau
\[
\begin{tabular}{|>{$}c<{$}>{$}c<{$}>{$}c<{$}|>{$}c<{$}>{$}c<{$}>{$}c<{$}|}
\hline
x_1&x_2&x_3&b_1&b_2&b_3\\
\hline
1&0&0&c_{11}&c_{12}&c_{13}\\
0&1&0&c_{21}&c_{22}&c_{23}\\
0&0&1&c_{31}&c_{32}&c_{33}\\
\hline
\end{tabular},
\]
welches mit den Gleichungen
\[
\begin{linsys}{3}
x_1&=&c_{11}b_1&+&c_{12}b_2&+&c_{13}b_3\\
x_2&=&c_{21}b_1&+&c_{22}b_2&+&c_{23}b_3\\
x_3&=&c_{31}b_1&+&c_{32}b_2&+&c_{33}b_3\\
\end{linsys}
\]
gleichbedeutend ist.
Diese kann man auch vektoriell als
\[
x = b_1 c_1 + b_2 c_2 + b_3 c_3
\]
schreiben.

Man kann also ein allgemeines Lösungsverfahren für lineare Gleichungssysteme
wie folgt konzipieren.
Zunächst bildet man das Gauss-Tableau
\begin{center}
\begin{tabular}{|>{$}c<{$}>{$}c<{$}>{$}c<{$}|>{$}c<{$}>{$}c<{$}>{$}c<{$}|}
\hline
a_{11}&a_{12}&a_{13}&1&0&0\\
a_{21}&a_{22}&a_{23}&0&1&0\\
a_{31}&a_{32}&a_{33}&0&0&1\\
\hline
\end{tabular}
\end{center}
welches man anschliessend mit dem Gauss-Algorithmus in das Tableau
\begin{center}
\begin{tabular}{|>{$}c<{$}>{$}c<{$}>{$}c<{$}|>{$}c<{$}>{$}c<{$}>{$}c<{$}|}
\hline
1&0&0&c_{11}&c_{12}&c_{13}\\
0&1&0&c_{21}&c_{22}&c_{23}\\
0&0&1&c_{31}&c_{32}&c_{33}\\
\hline
\end{tabular}
\end{center}
umwandelt.
Dann baut man die Lösung aus den Zahlen $c_{ij}$ und der rechten
Seite $b$ zusammen.
Die Berechnung der $c_{ij}$ muss nur einmal
erfolgen, danach kann die Lösung mit verschiedenen rechten Seiten
$b$ bestimmt werden.

