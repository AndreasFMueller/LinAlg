%
% simultan.tex -- mehrere Gleichungssysteme gleichzeitig lösen
%
% (c) 2009 Prof Dr Andreas Mueller, Hochschule Rapperswil
%
\section{Mehrere Gleichungssysteme simultan lösen\label{simultan}}
Mit dem Gauss-Algorithmus kann man gleichzeitig mehrere Gleichungssysteme
lösen, die sich nur in den rechten Seiten unterscheiden.
Die rechten Seiten haben nämlich auf den Gang der Rechnung keinen
Einfluss.
So kann man die beiden Gleichungssysteme
\begin{align*}
\frac12x+\frac{\sqrt{3}}2y&=1
&
\frac12x+\frac{\sqrt{3}}2y&=0
\\
-\frac{\sqrt{3}}2x+\frac12y&=0
&
-\frac{\sqrt{3}}2x+\frac12y&=1
\end{align*}
in einem Durchgang lösen:
\begin{align*}
\begin{tabular}{|>{$}c<{$}>{$}c<{$}|>{$}c<{$}>{$}c<{$}|}
\hline
\frac12%
\begin{picture}(0,0)
\color{red}\put(-3.5,3){\circle{15}}
\end{picture}%
&\frac{\sqrt{3}}2&1&0\\
-\frac{\sqrt{3}}2&\frac12&0&1\\
\hline
\end{tabular}
&\rightarrow
\begin{tabular}{|>{$}c<{$}>{$}c<{$}|>{$}c<{$}>{$}c<{$}|}
\hline
1&\sqrt{3}&2&0\\
-\frac{\sqrt{3}}2%
\begin{picture}(0,0)
\color{blue}\drawline(-22,-4)(-22,13)(1,13)(1,-4)
\end{picture}%
&\frac12&0&1\\
\hline
\end{tabular}
\rightarrow
\begin{tabular}{|>{$}c<{$}>{$}c<{$}|>{$}c<{$}>{$}c<{$}|}
\hline
1&\sqrt{3}&2&0\\
0&2%
\begin{picture}(0,0)
\color{red}\put(-3,4){\circle{12}}
\end{picture}%
&\sqrt{3}&1\\
\hline
\end{tabular}
\\
&\rightarrow
\begin{tabular}{|>{$}c<{$}>{$}c<{$}|>{$}c<{$}>{$}c<{$}|}
\hline
1&\sqrt{3}%
\begin{picture}(0,0)
\color{blue}\drawline(-16,11)(-16,-3)(1,-3)(1,11)
\end{picture}%
&2&0\\
0&1&\frac{\sqrt{3}}2&\frac12\\
\hline
\end{tabular}
\rightarrow
\begin{tabular}{|>{$}c<{$}>{$}c<{$}|>{$}c<{$}>{$}c<{$}|}
\hline
1&0&\frac12&-\frac{\sqrt{3}}2\\
0&1&\frac{\sqrt{3}}2&\frac12\\
\hline
\end{tabular}
\end{align*}
Das erste Gleichungssystem hat also die Lösung
$(x,y)=(\frac12,\frac{\sqrt{3}}2)$,
das zweite
$(x,y)=(-\frac{\sqrt{3}}2, \frac12)$.

Die spezielle Wahl der rechten Seiten erlaubt auch, aus den
beiden gefundenen Lösungen die Lösung für jede beliebige rechte
Seite zusammenzusetzen. Wenn nämlich auf der rechten
Seite die Zahlen $b_1$ und $b_2$ stehen sollen, dann
bedeutet dies offenbar, dass wir $b_1$ mal die erste Lösung
und $b_2$ mal die zweite Lösung zusammennehmen müssen, um
die verlangten rechten Seiten zu finden.
Also
\[
x=\frac12b_1-\frac{\sqrt{3}}2b_2,\qquad y=\frac{\sqrt{3}}2b_1+\frac12b_2.
\]
Durch Einsetzen in die ursprüngliche Gleichung kann man die
Lösung überprüfen.

Ein interessanter Spezialfall liegt vor, wenn man ein Gleichungssystem
für die rechten Seiten 
\[
e_1=\begin{pmatrix}1\\0\\0\end{pmatrix},\qquad
e_2=\begin{pmatrix}0\\1\\0\end{pmatrix},\qquad
e_3=\begin{pmatrix}0\\0\\1\end{pmatrix}
\]
bereits gelöst und die Lösungsvektoren
\[
c_1=\begin{pmatrix} c_{11}\\ c_{21}\\ c_{31} \end{pmatrix},\qquad
c_2=\begin{pmatrix} c_{12}\\ c_{22}\\ c_{32} \end{pmatrix},\qquad
c_3=\begin{pmatrix} c_{13}\\ c_{23}\\ c_{33} \end{pmatrix}
\]
gefunden hat.
Da die allgemeine rechte Seite
\[
b=\begin{pmatrix}
b_1\\b_2\\b_3
\end{pmatrix}
=
b_1 \begin{pmatrix}1\\0\\0\end{pmatrix}
+
b_2 \begin{pmatrix}0\\1\\0\end{pmatrix}
+
b_3 \begin{pmatrix}0\\0\\1\end{pmatrix}
=
b_1 e_1 + b_2 e_2 + b_3 e_3
\]
geschrieben werden kann, kann auch die Lösung aus den Teillösungen
zusammengesetzt werden:
\[
x = b_1 c_1 + b_2 c_2 + b_3 c_3.
\]
Man kann also ein allgemeines Lösungsverfahren auch wie folgt konzipieren.
Zunächst bildet man das Gauss-Tableau
\begin{center}
\begin{tabular}{|>{$}c<{$}>{$}c<{$}>{$}c<{$}|>{$}c<{$}>{$}c<{$}>{$}c<{$}|}
\hline
a_{11}&a_{12}&a_{13}&1&0&0\\
a_{21}&a_{22}&a_{23}&0&1&0\\
a_{31}&a_{32}&a_{33}&0&0&1\\
\hline
\end{tabular}
\end{center}
welches man anschliessend mit dem Gauss-Algorithmus in das Tableau
\begin{center}
\begin{tabular}{|>{$}c<{$}>{$}c<{$}>{$}c<{$}|>{$}c<{$}>{$}c<{$}>{$}c<{$}|}
\hline
1&0&0&c_{11}&c_{12}&c_{13}\\
0&1&0&c_{21}&c_{22}&c_{23}\\
0&0&1&c_{31}&c_{32}&c_{33}\\
\hline
\end{tabular}
\end{center}
umwandelt.
Dann baut man die Lösung aus den Zahlen $c_{ij}$ und der rechten
Seite $b$ zusammen.
Die Berechnung der $c_{ij}$ muss nur einmal
erfolgen, danach kann die Lösung mit verschiedenen rechten Seiten
$b$ bestimmt werden.


