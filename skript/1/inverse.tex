%
% inverse.tex -- Inverse Matrix
%
% (c) 2009 Prof Dr Andreas Mueller, Hochschule Rapperswil
%
\section{Inverse Matrix}
\rhead{Inverse Matrix}
\index{Matrix!inverse}
Mit den neuen Notationen können wir jetzt auch das Verfahren zur simultanen
Lösung mehrere Gleichungssysteme aus Abschnitt \ref{simultan} etwas
übersichtlicher schreiben.
Dazu brauchen wir die spezielle Matrix, die auf der Diagonalen Einsen hat
und sonst nur Nullen.
Diese Matrix heisst die {\em Einheitsmatrix}, wir bezeichnen sie mit
\index{Einheitsmatrix}
\[
I=
\begin{pmatrix}1&\dots&0\\
\vdots&\ddots&\vdots\\
0&\dots&1
\end{pmatrix},
\]
unabhängig von der Anzahl Zeilen und Spalten, diese ist aus dem Zusammenhang
zu entnehmen.

Das bereits in Abschnitt~\label{skript:subsection:standardrs}
skizzierte
Verfahren begann damit, an die gegebene Koeffizientenmatrix die Matrix $I$
anzuhängen.
Anschliessend wurde der Gauss-Algorithmus verwendet, bis in
der linken Hälfte der Tabelle eine Einheitsmatrix stand.
Symbolisch können wir dies durch
\begin{equation}
\begin{tabular}{|c|c|}\hline
$A$&$I$\\
\hline
\end{tabular}
\rightarrow
\begin{tabular}{|c|c|}\hline
$I$&$C$\\
\hline
\end{tabular}
\label{gaussinverse}
\end{equation}
darstellen.
Die Matrix $C$ enthält in ihren Spalten die Lösungen des
Gleichungssystems für die speziellen rechten Seiten, die nur aus Nullen
und einer Eins bestehen.

Eine beliebige rechte Seite $b$ lässt sich jetzt aus den speziellen rechten
Seiten zusammensetzen:
\[
b=\begin{pmatrix}b_1\\b_2\\\vdots\\b_n\end{pmatrix}
=
b_1\begin{pmatrix}1\\0\\\vdots\\0\end{pmatrix}
+
b_2\begin{pmatrix}0\\1\\\vdots\\0\end{pmatrix}
+\dots+
b_n\begin{pmatrix}0\\0\\\vdots\\1\end{pmatrix}
\]
wir versuchen daher, auch die Lösung aus den entsprechenden Spalten von $C$
aufzubauen:
\begin{equation}
x
=
b_1\begin{pmatrix}c_{11}\\c_{21}\\\vdots\\c_{n1}\end{pmatrix}
+
b_2\begin{pmatrix}c_{12}\\c_{22}\\\vdots\\c_{n2}\end{pmatrix}
+\dots+
b_n\begin{pmatrix}c_{1n}\\c_{2n}\\\vdots\\c_{nn}\end{pmatrix}.
\label{zusammensetzen}
\end{equation}
Tatsächlich ergibt sich, wenn wir die Matrix $A$ auf diesen Vektor
anwenden
\[
Ax
=
b_1A\begin{pmatrix}c_{11}\\c_{21}\\\vdots\\c_{n1}\end{pmatrix}
+
b_2A\begin{pmatrix}c_{12}\\c_{22}\\\vdots\\c_{n2}\end{pmatrix}
+\dots+
b_nA\begin{pmatrix}c_{1n}\\c_{2n}\\\vdots\\c_{nn}\end{pmatrix}
=
b_1\begin{pmatrix}1\\0\\\vdots\\0\end{pmatrix}
+
b_2\begin{pmatrix}0\\1\\\vdots\\0\end{pmatrix}
+\dots+
b_n\begin{pmatrix}0\\0\\\vdots\\1\end{pmatrix}=b,
\]
dieses $x$ ist also tatsächlich die gesuchte Lösung.

Das Element auf Zeile $i$ in 
\eqref{zusammensetzen} ist
\[
x_i=c_{i1}b_1+c_{i2}b_2+\dots+c_{in}b_n,
\]
dies ist aber nichts anderes als die $i$-te Komponente des Produktes
$Cb$, die Lösung $x$ kann als mit Hilfe eines Matrizenproduktes gefunden
werden:
\[
x = Cb.
\]
Die Matrix $C$ hat also offenbar eine ganz besondere Bedeutung für die
Lösung linearer Gleichungssysteme, ist sie einmal bestimmt, kann man das
Gleichungssystem für jede beliebige rechte Seite lösen.
Daher definieren wir
\begin{definition}Die Matrix $C$ aus \eqref{gaussinverse} heisst die
{\em inverse Matrix} von $A$, geschrieben $A^{-1}=C$.
Sie existiert immer, wenn $A$ regulär ist.
\end{definition}
Den Algorithmus zur Bestimmung der inversen Matrix kann also symbolisch
als
\begin{equation}
\begin{tabular}{|>{$}c<{$}|>{$}c<{$}|}
\hline
A&I\\
\hline
\end{tabular}
\rightarrow
\begin{tabular}{|>{$}c<{$}|>{$}c<{$}|}\hline
I&A^{-1}\\
\hline
\end{tabular}
\label{gaussinverse2}
\end{equation}
notiert werden.
\begin{satz}Die Lösung eines Gleichungssystems
\[
Ax=b
\]
mit
regulärer Koeffizientenmatrix $A$ ist
\[
x=A^{-1}b.
\]
\end{satz}

\begin{beispiel}
Man finde die Lösung des Gleichungssystems
\[
Ax=
\begin{pmatrix}
  -4& -4& -1\\
   5&  3&  1\\
   4& -5&  0
\end{pmatrix}
\begin{pmatrix}x_1\\x_2\\x_3\end{pmatrix}
=
\begin{pmatrix}2\\1\\3\end{pmatrix}.
\]
mit Hilfe der Inversen Matrix.

\smallskip
{\parindent 0pt Man muss die Inverse der Matrix $A$ bestimmen,
der Gauss-Algorithmus liefert}
\begin{align*}
\begin{tabular}{|>{$}c<{$}>{$}c<{$}>{$}c<{$}|>{$}c<{$}>{$}c<{$}>{$}c<{$}|}
\hline
  -4& -4& -1 & 1& 0& 0\\
   5&  3&  1 & 0& 1& 0\\
   4& -5&  0 & 0& 0& 1\\
\hline
\end{tabular}
&\rightarrow
\begin{tabular}{|>{$}c<{$}>{$}c<{$}>{$}c<{$}|>{$}c<{$}>{$}c<{$}>{$}c<{$}|}
\hline
   1&  1& \frac14 &-\frac14& 0& 0\\
   0& -2&-\frac14 & \frac54& 1& 0\\
   0& -9& -1      & 1      & 0& 1\\
\hline
\end{tabular}
\\
&\rightarrow
\begin{tabular}{|>{$}c<{$}>{$}c<{$}>{$}c<{$}|>{$}c<{$}>{$}c<{$}>{$}c<{$}|}
\hline
   1&  1& \frac14 &-\frac14   &       0& 0\\
   0&  1& \frac18 &-\frac58   &-\frac12& 0\\
   0&  0& \frac18 &-\frac{37}8&-\frac92& 1\\
\hline
\end{tabular}
\\
&\rightarrow
\begin{tabular}{|>{$}c<{$}>{$}c<{$}>{$}c<{$}|>{$}c<{$}>{$}c<{$}>{$}c<{$}|}
\hline
   1&  1&       0 &       9   &       9&-2\\
   0&  1&       0 &       4   &       4&-1\\
   0&  0&       1 &-37        &     -36& 8\\
\hline
\end{tabular}
\\
&\rightarrow
\begin{tabular}{|>{$}c<{$}>{$}c<{$}>{$}c<{$}|>{$}c<{$}>{$}c<{$}>{$}c<{$}|}
\hline
   1&  0&       0 &       5   &       5&-1\\
   0&  1&       0 &       4   &       4&-1\\
   0&  0&       1 &-37        &     -36& 8\\
\hline
\end{tabular}
\end{align*}
Die inverse Matrix ist also
\[
A^{-1}=
\begin{pmatrix}
5&5&-1\\
4&4&-1\\
-37&-36&8
\end{pmatrix}.
\]
Zur Kontrolle kann man $A^{-1}A$ ausrechnen, man sollte die Einheitsmatrix
erhalten:
\begin{align*}
A^{-1}A&=
\begin{pmatrix}
5&5&-1\\
4&4&-1\\
-37&-36&8
\end{pmatrix}
\begin{pmatrix}
  -4& -4& -1\\
   5&  3&  1\\
   4& -5&  0
\end{pmatrix}
\\
&
=
\begin{pmatrix}
-20+25-4 & -20+15+5 & -5+5+0\\
-16+20-4 & -16+12+5 & -4+4+0\\
148-180+32 & 148-108-40&37-36+0
\end{pmatrix}
=\begin{pmatrix}
1&0&0\\0&1&0\\0&0&1\end{pmatrix}.
\end{align*}
Mit der inversen Matrix kann man jetzt auch die Lösung
\[
\begin{pmatrix}x_1\\x_2\\x_3\end{pmatrix}
=A^{-1}
\begin{pmatrix}2\\1\\3\end{pmatrix}
=
\begin{pmatrix}
5&5&-1\\
4&4&-1\\
-37&-36&8
\end{pmatrix}
\begin{pmatrix}2\\1\\3\end{pmatrix}
=\begin{pmatrix}
12\\
9\\
-86
\end{pmatrix}
\]
erhalten.
\end{beispiel}

