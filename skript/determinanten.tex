\chapter{Determinanten\label{chapter-determinanten}}
\rhead{Determinanten}
Die L"osungen eines linearen Gleichungssystems "andern sich nicht,
wenn man die Operationen {\bf E} oder {\bf I} ausf"uhrt. Es sollte
also eine hoffentlich geringe Anzahl von ``Kennzahlen'' f"ur die
Koeffizienten $(a_{ij})$
und die rechten Seiten geben, die sich unter {\bf E} und {\bf I}
m"oglichst nicht "andern, und aus denen man die L"osung auch
schon berechnen kann. Ziel dieses Kapitels ist, diese Kennzahlen
zu finden.

Es stellt sich heraus, dass es zu einer Matrix
im wesentlichen nur eine solche Kennzahl gibt, die Determinante,
die sich mit dem Gaussverfahren einfach berechnen l"asst. Sie erweist
sich auch in anderen Anwendungen als praktisch. In sp"ateren Kapiteln
wird sie zum Beispiel f"ur folgendes verwendet:
\index{Vektorprodukt}
\index{Volumenberechnung}
\index{Eigenwert}
\index{Eigenvektor}
\begin{compactitem}
\item Kriterium f"ur L"osbarkeit eines Gleichungssystems
\item Vektorprodukt
\item Volumenberechnung
\item Eigenwerte, charakteristisches Polynom
\end{compactitem}
\section{Begriff der Determinanten}
Die Determinante soll eine Kennzahl liefern, an der sich ablesen
l"asst, ob eine Matrix regul"ar oder singul"ar ist. Zur Definition
kann man sich dabei auf den Gauss-Algorithmus st"utzen. F"ur das
Arbeiten mit Determinaten ist das nicht praktisch, dazu werden
Rechenregeln (Formeln) ben"otigt. Daraus ergibt sich dann mit
dem Entwicklungssatz ein praktisch durchf"uhrbarer Algorithmus.
Schliesslich kann man Determinanten sogar dazu verwenden, Gleichungssysteme
zu l"osen oder Matrizen zu invertieren.
\subsection{Definition}
\index{Determinante}
Zu einer Matrix $A=(a_{ij})$ suchen wir eine Gr"osse
$\det A$, welche folgende Eigenschaften hat:
\begin{compactenum}
\item $\det A$ "andert sich nicht unter der Operation $E$.\label{invarianzE}
\item Wird eine Zeile von $A$ mit $\lambda$ multipliziert,\label{skalarI}
wird auch $\det A$ mit $\lambda$ multipliziert.
\item $\det E=1$\label{normierung}
\end{compactenum}
Im trivialen Fall des $1\times1$-Schemas folgt daraus bereits
\[
\det(a)\overset{\text{Eigenschaft \ref{skalarI}}}=a\det(1)\overset{\text{Eigenschaft \ref{normierung}}}=a\cdot 1=a.
\]

Aber auch f"ur den Fall $2\times 2$ k"onnen wir die L"osung
bereits bestimmen. Dazu gehen wir wie folgt vor:
\begin{align*}
\begin{pmatrix}
a%
\begin{picture}(0,0)
\color{red}\put(-2,3){\circle{12}}
\end{picture}%
&b\\
c&d%
\end{pmatrix}
&\rightarrow
\begin{pmatrix}1&\frac{b}{a}\\
c%
\begin{picture}(0,0)
\color{blue}\drawline(-6,-1)(-6,10)(1,10)(1,-1)
\end{picture}%
&d\end{pmatrix}
&\det \begin{pmatrix}a&b\\c&d\end{pmatrix}
&=a%
\begin{picture}(0,0)
\color{red}\put(-3,3){\circle{10}}
\end{picture}%
\;
\det
\begin{pmatrix}1&\frac{b}{a}\\c&d\end{pmatrix}
\\
&\rightarrow
\begin{pmatrix}1&\frac{b}{a}\\0&d-\frac{bc}{a}%
\begin{picture}(0,0)
\color{red}\put(-15,3){\circle{28}}
\end{picture}%
\end{pmatrix}
&
&=a%
\begin{picture}(0,0)
\color{red}\put(-3,3){\circle{10}}
\end{picture}%
\;\det\begin{pmatrix}1&\frac{b}{a}\\0&d-\frac{bc}{a}\end{pmatrix}
\\
&\rightarrow
\begin{pmatrix}1&\frac{b}{a}%
\begin{picture}(0,0)
\color{blue}\drawline(-7,10)(-7,-5)(0,-5)(0,10)
\end{picture}%
\\0&1\end{pmatrix}
&
&=a%
\begin{picture}(0,0)
\color{red}\put(-3,3){\circle{10}}
\end{picture}%
\;
\left(d-\frac{bc}a\right)%
\begin{picture}(0,0)
\color{red}\put(-22.5,4){\circle{35}}
\end{picture}%
\det\begin{pmatrix}1&\frac{b}{a}\\0&1\end{pmatrix}
\\
&\rightarrow
\begin{pmatrix}1&0\\0&1\end{pmatrix}
&
&=(ad-bc)
\det \begin{pmatrix}1&0\\0&1\end{pmatrix}
\\
&&&=ad-bc
\end{align*}
Die Eigenschaften sind also offenbar bereits stark genug, um die
Determinante festzulegen. F"ur die Determinate hat sich auch die
Bezeichnung mit zwei vertikalen Linien links und rechts des
Koeffizientenschemas eingeb"urgert:
\[
\det\begin{pmatrix}a&b\\c&d\end{pmatrix}
=
\left|\,\begin{matrix}a&b\\c&d\end{matrix}\,\right|
=
ad-bc.
\]

\subsection{Rechenregeln f"ur Determinanten}
Nach "ahnlichem Muster lassen sich auch aus den Eigenschaften auch
noch weitere praktische Rechenregeln ableiten:

\begin{hilfssatz}
Besteht eine Zeile in einer Matrix $A=(a_{ij})$ aus
lauter Nullen, ist auch $\det A=0$.
\label{nullzeile}
\end{hilfssatz}

\begin{proof}[Beweis]
Multipliziert man die aus Nullen bestehende Zeilen mit $\lambda$
wird $\det A$ nach Eigenschaft \ref{skalarI}  mit $\lambda$ multipliziert.
Durch die Multiplikation "andern sich die Nullen in der Zeile allerdings
nicht, es gilt also
\[
\lambda \det A=\det A
\]
f"ur jedes beliebige $\lambda$. Die einzige Zahl, die sich nicht "andert,
wenn man sie mit beliebigen Zahlen multipliziert, ist die Null. Also
$\det A=0$.
\end{proof}

\begin{hilfssatz}
Sind zwei Zeilen in einer Matrix $A=(a_{ij})$
gleich, dann ist $\det A=0$.
\end{hilfssatz}
\begin{proof}[Beweis]
Subtrahiert man die erste der beiden gleichen Zeilen von der
zweiten, "andert sich die Determinate nicht, denn dies ist eine
{\bf E}-Operation. In der zweiten Zeile stehen nach dieser Operation
aber lauter Nullen, die Determinante muss nach Hilfssatz \ref{nullzeile}
also $0$ sein.
\end{proof}

\begin{hilfssatz}
Vertauscht man in einer Matrix $A=(a_{ij})$ zwei
Zeilen, dann "andert $\det A$ das Vorzeichen.
\end{hilfssatz}
\begin{proof}[Beweis]
Die Zeilenvertauschung kann man in folgenden Schritten durchf"uhren:
\begin{compactenum}
\item Erste Zeile von der zweiten subtrahieren:
\[
\begin{pmatrix}
      &\dots&      &\dots\\
a_{i1}&\dots&a_{ik}&\dots\\
      &\dots&      &\dots\\
a_{j1}&\dots&a_{jk}&\dots\\
      &\dots&      &\dots\\
\end{pmatrix}
\rightarrow
\begin{pmatrix}
             &\dots&             &\dots\\
a_{i1}       &\dots&a_{ik}       &\dots\\
             &\dots&             &\dots\\
a_{j1}-a_{i1}&\dots&a_{jk}-a_{ik}&\dots\\
             &\dots&             &\dots\\
\end{pmatrix}
\]
\item Zweite Zeile zur ersten addieren:
\[
\begin{pmatrix}
             &\dots&             &\dots\\
a_{i1}       &\dots&a_{ik}       &\dots\\
             &\dots&             &\dots\\
a_{j1}-a_{i1}&\dots&a_{jk}-a_{ik}&\dots\\
             &\dots&             &\dots\\
\end{pmatrix}
\rightarrow
\begin{pmatrix}
             &\dots&             &\dots\\
a_{j1}       &\dots&a_{jk}       &\dots\\
             &\dots&             &\dots\\
a_{j1}-a_{i1}&\dots&a_{jk}-a_{ik}&\dots\\
             &\dots&             &\dots\\
\end{pmatrix}
\]
\item  Erste Zeile von der zweiten Subtrahieren:
\[
\begin{pmatrix}
             &\dots&             &\dots\\
a_{j1}       &\dots&a_{jk}       &\dots\\
             &\dots&             &\dots\\
a_{j1}-a_{i1}&\dots&a_{jk}-a_{ik}&\dots\\
             &\dots&             &\dots\\
\end{pmatrix}
\rightarrow
\begin{pmatrix}
       &\dots&       &\dots\\
a_{j1} &\dots&a_{jk} &\dots\\
       &\dots&       &\dots\\
-a_{i1}&\dots&-a_{ik}&\dots\\
       &\dots&       &\dots\\
\end{pmatrix}
\]
\item Zweite Zeile mit $-1$ multiplizieren.
\[
\begin{pmatrix}
       &\dots&       &\dots\\
a_{j1} &\dots&a_{jk} &\dots\\
       &\dots&       &\dots\\
-a_{i1}&\dots&-a_{ik}&\dots\\
       &\dots&       &\dots\\
\end{pmatrix}
\rightarrow
\begin{pmatrix}
      &\dots&      &\dots\\
a_{j1}&\dots&a_{jk} &\dots\\
      &\dots&      &\dots\\
a_{i1}&\dots&a_{ik}&\dots\\
      &\dots&      &\dots\\
\end{pmatrix}
\]
\end{compactenum}
Da die ersten drei Schritte {\bf E}-Operationen sind, "andert sich nach
Eigenschaft \ref{invarianzE} die Determinante dabei nicht. Die letzte
Operation ist eine {\bf I}-Operation, nach Eigenschaft \ref{skalarI}
wird die Determinante dabei mit $-1$ multipliziert.
\end{proof}

\begin{hilfssatz}
\label{detlinabh}
Sind die Gleichungen eines Gleichungssystems linear abh"angig, dann
gilt f"ur die Koeffizienten $\det A=0$.
\end{hilfssatz}
\begin{proof}[Beweis]
Wenn die Zeilen linear abh"angig sind, dann wissen wir bereits, dass
im Gauss-Verfahren, bei welchem nur die Operationen {\bf I} und {\bf E}
angewendet werden, am Ende mindestens eine Zeile aus lauter
Nullen auftauchen muss. Am Ende des Prozesses ist die Determinante
also $0$.

Da die Operation {\bf E} die Determinante gar nicht,
und {\bf I} nur um einen nicht verschwindenden Faktor "andert, muss
die Determinante also auch schon am Anfang des Prozesses $0$ gewesen
sein.
\end{proof}

\subsection{Determinante als lineare Funktion der Zeilen\label{detlinfun}}
Die bisher verwendeten Eigenschaften der Determinante hatten den
Vorteil, einen direkten Bezug zu den Operationen zu haben, mit 
denen wir Gleichungssysteme gel"ost haben. Sie hatten den
Nachteil, etwas willk"urlich zu sein.
Daher verwendet man oft die folgenden Eigenschaften, aus denen
die bisherigen Eigenschaften folgen.

\begin{compactenum}
\item[$1'$.] $\det A$ ist eine lineare Funktion der Zeilen, d.h.
\[
\left|
\;
\begin{matrix}
a_{11}&\dots&a_{1n}\\
\vdots&\ddots&\vdots\\
\lambda a'_{k1}+\mu a''_{k1}&\dots&\lambda a'_{kn}+\mu a''_{kn}\\
\vdots&\ddots&\vdots\\
a_{n1}&\dots&a_{nn}
\end{matrix}
\;
\right|
=
\lambda
\left|
\;
\begin{matrix}
a_{11}&\dots&a_{1n}\\
\vdots&\ddots&\vdots\\
a'_{k1}&\dots&a'_{kn}\\
\vdots&\ddots&\vdots\\
a_{n1}&\dots&a_{nn}
\end{matrix}
\;
\right|
+
\mu
\left|
\;
\begin{matrix}
a_{11}&\dots&a_{1n}\\
\vdots&\ddots&\vdots\\
a''_{k1}&\dots&a''_{kn}\\
\vdots&\ddots&\vdots\\
a_{n1}&\dots&a_{nn}
\end{matrix}
\;
\right|
\]
\item[$2'$.] Sind zwei Zeilen von $A$ gleich, ist $\det A=0$
\end{compactenum}

Aus diesen Eigenschaften lassen sich die Eigenschaften \ref{invarianzE}.~und
\ref{skalarI}.~ableiten.

\begin{hilfssatz} Aus den Eigenschaften $1'$.~und $2'$.~folgen die Eigenschaften
1.~und 2.~der Determinante.
\end{hilfssatz}
\begin{proof}[Beweis]
Setzt man 
$a'_{ki}=a_{ki}$, $a''_{ki}=a_{li}$, $\lambda=1$ entsteht auf der linken Seite
die Determinante nach dem hinzuadieren des $\mu$-fachen der $l$-ten Zeile
zur $k$-ten Zeile, also das Resultat einer {\bf E}-Operation. Auf
der rechten Seite steht als erster Term die urspr"ungliche Determinante,
der zweite Term ist aber eine Determinante, in der die $k$-te und die
$l$-te Zeile "ubereinstimmen, diese verschwindet also nach $2'$. Damit
ist die Eigenschaft \ref{invarianzE}.~bewiesen.

Setzt man $a'_{ki}=a_{ki}$ und $\mu=0$, steht auf der linken Seite
die Determinante, in der die $k$-te Zeile mit $\lambda$ multipliziert
wurde. Auf der rechten Seite steht die mit $\lambda$ multilizierte
Determinante, wegen $\mu=0$ f"allt der zweite Term weg. Somit ist
auch die Eigenschaft \ref{skalarI}.~ bewiesen.
\end{proof}

Wenn die Eigenschaften $1'$., $2'$.~und 3.~erf"ullt sind, sind also immer
noch alle Schlussfolgerungen g"ultig, die wir aus den Eigenschaften
1.~bis 3.~gezogen hatten.

Der Vorteil dieser Eigenschaften gegen"uber den bisher verwendeten
besteht darin, dass sich daraus eine Formel ableiten l"asst, aus
der weiter Eigenschaften der Determinante leichter abgeleitet
werden k"onnen. 

Man kann zeigen, dass die Eigenschaften der Determinante diese
eindeutig bestimmen. Es kann also nicht passieren, dass eine andere
Reihenfolge der Pivotelemente zu einem anderen Resultat f"uhrt.
Da diese Eigenschaft auf eine Art beweisen wird, die uns keine
anderen n"utzlichen Resultate liefert, verbannen wir den Beweis
in den letzten Abschnitt \ref{deteindeutig} dieses Kapitels.
Wir halten hier nur das Resultat fest.

\begin{satz}
\label{detcharacterisation}
Die Determinante
$\det A$
ist die einzige lineare Funktion der Zeilen von $A$, die folgende zwei
Eigenschaften hat:
\begin{compactenum}
\item Falls $A$ zwei gleiche Zeilen enth"alt ist $\det A=0$.
\item $\det E = 1$.
\end{compactenum}
\end{satz}

\subsection{Spalten statt Zeilen}
Die Definition verlangt explizit nach Zeilen-Operationen {\bf I}
und {\bf E}. Wir k"onnten aber auch entsprechende Spalten-Operationen
{\bf I$\mathstrut^t$}
und
{\bf E$\mathstrut^t$}
verwenden, um die Determinante zu definieren. Das vorgehen zur Berechnung
der Determinante ist genau das selbe. Man wendet die Operationen an, bis
nur noch die Einheitsmatrix stehen bleibt.

Analog zum Vorgehen in Abschnitt \ref{detlinfun} k"onnen wir auch
fordern, dass die Deterninte eine lineare Funktion der Spalten
(statt der Zeilen) sein soll, und w"urden dieselbe Determinante
erhalten wie mit der Definition durch die Operationen 
{\bf I$\mathstrut^t$}
und
{\bf E$\mathstrut^t$}.
Auch hier halten wir das Resultat fest:
\begin{satz}
Die Determinante
$\det A$ ist die einzige lineare Funktion der Spalten von $A$, die folgende
zwei Eigenschaften hat:
\begin{compactenum}
\item Falls $A$ zwei gleiche Spalten enth"alt ist $\det A=0$.
\item $\det E = 1$.
\end{compactenum}
\end{satz}
Im Prinzip k"onnte diese "uber die Spalten definierte Determinante
von der "uber die Zeilen definierten Determinante verschieden sein.
Im Abschnitt \ref{deteindeutig} wird jedoch eine Formel f"ur die
Determinante abgeleitet, aus der hervorgeht, dass beide Definitionen
auf den selben Wert f"ur die Determinante f"uhren.

\section{Berechnung der Determinanten}
\subsection{Berechnung mit Gauss-Verfahren}
Das Beispiel der $2\times 2$-Determinante aus dem ersten Abschnitt
l"asst sich auch f"ur beliebig grosse Determinanten verallgemeinern,
woraus sich ein effizientes Berechnungsverfahren f"ur die Determinante
ergibt.

Im Laufe des Gauss-Verfahrens "andert sich die Determinate offenbar
imer dann, wenn eine {\bf I}-Operation ausgef"uhrt wird. In solchen
Schritten wird durch das Pivot-Element dividiert. Am Ende des 
Verfahrens bleibt die Einheitsmatrix stehen, welche die Determinante
$1$ hat. Durch fortgesetztes Dividieren durch die Pivot-Elemente wird
aus der $\det A$ also $1$:
\[
\frac{\det A}{\prod_{\text{$p$ Pivot-Element} }p}=1\qquad\Rightarrow
\qquad
\det A=\prod_{\text{$p$ Pivot-Element}} p%
\begin{picture}(0,0)
\color{red}\put(-4,2){\circle{13}}
\end{picture}%
\;.
\]
Es folgt
\begin{satz}
\label{detprodpivot}
Die Determinante von $A$ ist das Produkt der Pivot-Elemente,
die im Laufe des Gauss-Verfahrens auftreten.
\end{satz}

\subsection{Entwicklungssatz\label{entwicklungssatz}}
\index{Entwicklungssatz}
\index{rekursiv}
Der Entwicklungssatz erlaubt, Determinanten rekursiv zu berechnen.
F"ur die Berechnung einer $n\times n$-Determinante werden $n$
$(n-1)\times(n-1)$-Determinanten bestimmt und miteinander verkn"upft.
Die Entwicklung einer Determinante erfolgt einer Spalte oder Zeile.
Da wir durch Vertauschungen und Transposition diese Zeile oder Spalte
immer zur vordersten Spalte machen k"onnen, stellen wir nur die
Entwicklung der Determinante nach der ersten Spalte dar.

Seien also $a_{ij}$ die Eintr"age in einer $n\times n$-Determinante.
Die erste Spalte k"onnen wir als Summe von Vektoren betrachten,
die jeweils nur an einer Stelle eine Eins haben und sonst aus
Nullen bestehen:
\[
\begin{pmatrix}
a_{11}\\a_{21}\\\vdots\\a_{n1}
\end{pmatrix}
=
a_{11}\begin{pmatrix}1\\0\\\vdots\\0\end{pmatrix}
+
a_{21}\begin{pmatrix}0\\1\\\vdots\\0\end{pmatrix}
+
\dots
+
a_{n1}\begin{pmatrix}0\\0\\\vdots\\1\end{pmatrix}
\]
Da die Determinante eine lineare Funktion der Spalten ist,
k"onnen wir dies einsetzen:
\begin{align*}
\det(A)
%=
%\left|\;
%\begin{matrix}
%a_{11}&a_{12}&\dots&a_{1n}\\
%a_{21}&a_{22}&\dots&a_{2n}\\
%\vdots&\vdots&\ddots&\vdots\\
%a_{n1}&a_{n2}&\dots&a_{nn}
%\end{matrix}
%\;\right|
&=
a_{11}
\left|\;
\begin{matrix}
1&a_{12}&\dots&a_{1n}\\
0&a_{22}&\dots&a_{2n}\\
\vdots&\vdots&\ddots&\vdots\\
0&a_{n2}&\dots&a_{nn}
\end{matrix}
\;\right|
+a_{21}
\left|\;
\begin{matrix}
0&a_{12}&\dots&a_{1n}\\
1&a_{22}&\dots&a_{2n}\\
\vdots&\vdots&\ddots&\vdots\\
0&a_{n2}&\dots&a_{nn}
\end{matrix}
\;\right|
+\dots
\\
&\qquad
+a_{n1}
\left|\;
\begin{matrix}
0&a_{12}&\dots&a_{1n}\\
0&a_{22}&\dots&a_{2n}\\
\vdots&\vdots&\ddots&\vdots\\
1&a_{n2}&\dots&a_{nn}
\end{matrix}
\;\right|
\end{align*}
Durch Vertauschungen kann man die Zeile, die mit $1$ beginnt, in jeder
Determinante nach oben bringen, die einzelnen Terme erhalten dadurch
alternierende Vorzeichen
\begin{align*}
\det(A)
&=
a_{11}
\left|\;
\begin{matrix}
1&a_{12}&\dots&a_{1n}\\
0&a_{22}&\dots&a_{2n}\\
\vdots&\vdots&\ddots&\vdots\\
0&a_{n2}&\dots&a_{nn}
\end{matrix}
\;\right|
-a_{21}
\left|\;
\begin{matrix}
1&a_{22}&\dots&a_{2n}\\
0&a_{12}&\dots&a_{1n}\\
\vdots&\vdots&\ddots&\vdots\\
0&a_{n2}&\dots&a_{nn}
\end{matrix}
\;\right|
+\dots
\\
&\qquad
+(-1)^{n+1}a_{n1}
\left|\;
\begin{matrix}
1&a_{n2}&\dots&a_{nn}\\
0&a_{12}&\dots&a_{1n}\\
0&a_{22}&\dots&a_{2n}\\
\vdots&\vdots&\ddots&\vdots\\
0&a_{n-1,2}&\dots&a_{n-1,n}
\end{matrix}
\;\right|
\end{align*}
Um die einzelnen Determinanten zu berechnen, muss man jetzt 
den Gauss-Algorithmus anwenden. In der ersten Zeile und Spalte
gibt es nichts mehr zu tun, das dort stehende Pivot-Element ist
bereits $1$. Es bleibt also nur noch die $(n-1)\times(n-1)$-Matrix
im rechten unteren Teil, diese besteht aus den Zeilen und Spalten
von $A$ die "ubrig bleiben, wenn man die erste Spalte wegstreicht,
und im $i$-ten Summanden die $i$-te Zeile. Der Gauss-Algorithmus wird
die Determinante dieser $(n-1)\times(n-1)$ Matrix liefern.
\begin{align*}
\det(A)
&=
a_{11}
\left|\;
\begin{matrix}
a_{22}&\dots&a_{2n}\\
a_{32}&\dots&a_{3n}\\
\vdots&\ddots&\vdots\\
a_{n2}&\dots&a_{nn}
\end{matrix}
\;\right|
-a_{21}
\left|\;
\begin{matrix}
a_{12}&\dots&a_{1n}\\
a_{32}&\dots&a_{3n}\\
\vdots&\ddots&\vdots\\
a_{n2}&\dots&a_{nn}
\end{matrix}
\;\right|
+\dots
\\
&\qquad
+(-1)^{n+1}a_{n1}
\left|\;
\begin{matrix}
a_{12}&\dots&a_{1n}\\
a_{22}&\dots&a_{2n}\\
\vdots&\ddots&\vdots\\
a_{n-1,2}&\dots&a_{n-1,n}
\end{matrix}
\;\right|
\end{align*}
Wir verwenden folgende Notation, um diese Formel noch etwas kompakter
auszudr"ucken.
\index{Minor}
\begin{definition}Sei $A$ eine $n\times n$-Matrix mit Eintr"agen $a_{ij}$.
Dann heisst die Matrix $A_{ij}$, die sich aus $A$ durch wegstreichen
der $i$-ten Zeile und der $j$-ten Spalte ergibt, der $i$-$j$-Minor der
Matrix $A$.
\end{definition}
Damit kann der Determinantenentwicklungssatz nun wie folgt formuliert
werden:
\begin{satz}
Sei $A$ eine $n\times n$-Matrix. Dann ist
\[
\det(A)=
\sum_{i=1}^n(-1)^{i+1}a_{i1}\det(A_{i1})
=
\sum_{i=1}^n(-1)^{1+j}a_{1j}\det(A_{1j})
\]
die Entwicklung nach der ersten Spalte bzw.~Zeile, und 
\[
\det(A)=
\sum_{i=1}^n(-1)^{i+j}a_{ij}\det(A_{ij})
=
\sum_{j=1}^n(-1)^{i+j}a_{ij}\det(A_{ij})
\]
die Entwicklung nach der $j$-ten Spalte bzw.~$i$-ten Zeile. 
\end{satz}

\begin{beispiel}Die zu Beginn des Kapitels gefundene Formel f"ur
die Determinante einer $2\times 2$-Matrix ist ein Spezialfall
des Entwicklungssatzes
\[
\left|\;
\begin{matrix}
a&b\\c&d
\end{matrix}
\;\right|
=a\cdot\det(d)-b\cdot\det(c)|=ad-bc.
\]
\end{beispiel}

\begin{beispiel}
Man berechne die Determinante der Matrix
\[
A=\begin{pmatrix}
-1&-3&0\\
2&3&-2\\
2&1&-3
\end{pmatrix}
\]
Der Entwicklungssatz liefert
\begin{align*}
\det(A)&=
-1\det\begin{pmatrix}3&-2\\1&-3\end{pmatrix}
-2\det\begin{pmatrix}-3&0\\1&-3\end{pmatrix}
+2\det\begin{pmatrix}-3&0\\3&-2\end{pmatrix}
\\
&=
-1(-9+2)-2(9-0)+2(6-0)=7-18+12=1.
\end{align*}
\end{beispiel}

\subsection{Spezialfall: Dimension 3, die Sarrus Formel}
\index{Sarrus-Formel}
Mit diesem Satz l"asst sich jetzt auch die Sarrussche Formel einfach beweisen:
\begin{align*}
\left|\;
\begin{matrix}
a_{11}&a_{12}&a_{13}\\
a_{21}&a_{22}&a_{23}\\
a_{31}&a_{32}&a_{33}
\end{matrix}\;\right|
&=
a_{11}
\left|\;\begin{matrix}
a_{22}&a_{23}\\
a_{32}&a_{33}
\end{matrix}\;\right|
-
a_{21}
\left|\;\begin{matrix}
a_{12}&a_{13}\\
a_{32}&a_{33}
\end{matrix}\;\right|
+
a_{31}
\left|\;\begin{matrix}
a_{12}&a_{13}\\
a_{22}&a_{23}
\end{matrix}\;\right|
\\
&=a_{11}(a_{22}a_{33}-a_{23}a_{32})
-a_{21}(a_{12}a_{33}-a_{13}a_{32})
+a_{31}(a_{12}a_{23}-a_{13}a_{22})
\\
&=
a_{11}a_{22}a_{33}+a_{12}a_{23}a_{31}+a_{13}a_{21}a_{32}
-a_{31}a_{22}a_{13}-a_{32}a_{23}a_{11}-a_{33}a_{21}a_{12}
\end{align*}
Achtung: diese Formel kann nicht auf gr"ossere Determinanten
verallgemeinert werden.

\section{Produktformel}
Wie vertr"agt sich die Determinante mit Matrizenprodukten?
"Uberraschenderweise gibt es hierauf eine sehr einfach Antwort,
die auch nicht schwierig zu verstehen ist.

\begin{satz}\label{detprodukt}
Sind $A$ und $B$ $n\times n$-Matrizen, dann ist
\[
\det(AB)=\det(A)\det(B).
\]
\end{satz}

\begin{proof}[Beweisidee]
Wir gehen den Beweis wie folgt an. Wir betrachten zun"achst nur die
Abh"angigkeit von $\det(AB)$ von $A$. Dabei werden wir feststellen,
dass sich $\det(AB)$ fast wie die Determinante verh"alt, nur der
Wert f"ur $A=E$ ist nicht der richtige, $\det(B)$ statt $1$. Indem wir
$\det(AB)$ durch $\det(B)$ dividieren, erhalten wir aber eine Funktion,
die sich genau wie die Determinante verh"alt.
Weil die Eigenschaften der Determinante diese eindeutig bestimmen, muss
$\det(AB)/\det(B)=\det(A)$ sein, woraus wir die Behauptung folgern k"onnen.
\end{proof}

\begin{proof}[Beweis]
Wir betrachten die Funktion $d\colon A\mapsto d(A) = \det(AB)$.
Sie hat die folgenden Eigenschaften, die weiter unten noch
begr"undet werden m"ussen:
\begin{enumerate}
\item[1'.] $d$ ist eine lineare Funktion der Zeilen von $A$.
\item[2.']
Hat $A$ zwei gleiche Zeilen, dann auch $AB$ und damit ist $d(A)=0$.
\item[3.]
$d(E)=\det(B)$.
\end{enumerate}
Die Funktion $d$ verh"alt sich bis auf die letzte Eigenschaft
wie die Determinante. Dividiert man die Funktion $d$ durch die Determinante
von $B$, ist auch die letzte Eigenschaft die einer Determinanten.
Die Funktion
\[
d':A\mapsto \frac{\det(AB)}{\det(B)}
\]
hat die Eigenschaften:
\begin{enumerate}
\item[1'.] $d'$ ist eine lineare Funktion der Zeilen von $A$.
\item[2'.] Enth"alt $A$ zwei gleiche Zeilen, dann ist $d'(A)=0$.
\item[3.] $d'(E)=1$.
\end{enumerate}
Nach Satz \ref{detcharacterisation} muss $d'$ die Determinante von $A$ sein:
\begin{align*}
d'(A)&=\det(A)\\
\frac{\det(AB)}{\det(B)}&=\det(A)\\
\det(AB)&=\det(A)\det(B).
\end{align*}
Das beweist die Produktformel, bis auf die oben behaupteten Eigenschaften
1' und 2'.

Die Zeile mit der Nummer $i$ in $AB$ wird erhalten,
indem man die Zeile $i$ von $A$
mit der Matrix $B$ multipliziert. Wenn also $A$ zwei gleiche
Zeilen enth"alt, dann sind auch die entsprechenden Zeilen von $AB$
gleich. was Eigenschaft 2 beweist.

Ist die Zeile $i$ von $A$ eine Linearekombination der Form
\[
\begin{pmatrix}
a_{i1}&\dots&a_{in}
\end{pmatrix}
=
\lambda
\begin{pmatrix}
a'_{i1} &\dots &a'_{in}
\end{pmatrix}
+
\mu
\begin{pmatrix}
a''_{i1} &\dots &a''_{in}
\end{pmatrix},
\]
dann ist die Zeile $i$ von $AB$ ebenfalls eine Linearekombination,
n"amlich
\[
\begin{pmatrix}
a_{i1}&\dots&a_{in}
\end{pmatrix}B
=
\lambda
\begin{pmatrix}
a'_{i1} &\dots &a'_{in}
\end{pmatrix}B
+
\mu
\begin{pmatrix}
a''_{i1} &\dots &a''_{in}
\end{pmatrix}B.
\]
Schreiben wir $A'$ f"ur die Matrix, die in Zeile $i$ die $a'_{ij}$
statt der $a_{ij}$ enthalten, und analog f"ur $A''$, dann folgt, dass
\[
d(A) = \det(AB)=\lambda \det(A'B)+\mu\det(A''B)=\lambda d(A') + \mu d(A''),
\]
was genau die Eigenschaft 1 ist.
\end{proof}

Es gibt keine vergleichbare Formel f"ur die Determinante der Summe
von zwei Matrizen.
Schon $2\times 2$-Diagonalmatrizen zeigen uns,
das wir eine solche Formel auch nicht erwarten k"onnen:
\begin{align*}
A&=\begin{pmatrix}a&0\\0&1\end{pmatrix}&
\det(A)&=\left|\;\begin{matrix}a  &0\\0&  1\end{matrix}\;\right|= a\\
B&=\begin{pmatrix}1&0\\0&b\end{pmatrix}&
\det(B)&=\left|\;\begin{matrix}  1&0\\0&b  \end{matrix}\;\right|= b\\
A+B&=\begin{pmatrix}a+1&0\\0&b+1\end{pmatrix}
&
\det(A+B)&=\left|\;\begin{matrix}a+1&0\\0&b+1\end{matrix}\;\right|\\
&&&=(a+1)(b+1)=ab+a+b+1.
\end{align*}
Offenbar gibt es keine einfache Formel, die $\det(A+B)$ mit $\det(A)$ und
$\det(B)$ verbindet.

\section{L"osen von Gleichungssystemen}
Aus dem Hilfssatz \ref{detlinabh} folgt sofort, dass ein Gleichungssystem
genau dann eindeutig l"osbar ist, wenn die Determinante der Koeffizienten
nicht $0$ ist.

\begin{satz}
Das Gleichungssystem mit $n$ Unbekannten und $n$ Gleichungen
mit den Koeffizienten $a_{ij}$ ist genau dann eindeutig l"osbar,
wenn
$\det A\ne 0$
\end{satz}

Wir m"ochten jetzt die in der Einleitung versprochen Formel f"ur die L"osung
ableiten, mit der man die L"osung des Gleichungssystems aus lauter
Determinanten berechnen kann.

\index{Cramersche Regel}
\begin{satz}[Cramer]
Das Gleichungssystem mit $n$ Unbekannten und $n$ Gleichungen
mit den Koeffizienten $a_{ij}$ 
mit $\det A\ne 0$ hat die L"osungen
\[
x_1=\frac{
\left|\,\begin{matrix}
b_1&a_{12}&\dots&a_{1n}\\
\vdots&\vdots&\ddots&\vdots\\
b_n&a_{n2}&\dots&a_{nn}\\
\end{matrix}\,\right|
}{
\left|\,\begin{matrix}
a_{11}&a_{12}&\dots&a_{1n}\\
\vdots&\vdots&\ddots&\vdots\\
a_{n1}&a_{n2}&\dots&a_{nn}\\
\end{matrix}\,\right|
}
,\qquad\dots,\qquad
x_n=\frac{
\left|\,\begin{matrix}
a_{11}&\dots&a_{1,n-1}&b_1\\
\vdots&\ddots&\vdots&\vdots\\
a_{n1}&\dots&a_{n,n-1}&b_n\\
\end{matrix}\,\right|
}{
\left|\,\begin{matrix}
a_{11}&\dots&a_{1,n-1}&a_{1n}\\
\vdots&\ddots&\vdots&\vdots\\
a_{n1}&\dots&a_{n,n-1}&a_{nn}\\
\end{matrix}\,\right|
}
\]
Die Unbekannte $x_k$ berechnet man also als Quotient der Determinante
von $A$, in der man die $k$-te Spalte durch die rechten Seiten $b_i$
ersetzt hat, und der Determinanten von $A$.
\end{satz}
\begin{proof}[Beweis]
Schreiben wir $a_1,\dots,a_n$ f"ur die Spalten von $A$ und $b$ f"ur die 
Spalte der rechten Seite, dann bedeutet das Glechungssystem, dass die
Spalte $b$ geschrieben werden kann als eine Linearkombination der
Spalten von $A$:
\begin{equation}
x_1a_1+\dots +x_na_n=b
\label{glinspalten}
\end{equation}
Die Determinante von $A$ ist eine lineare Funktion jeder einzelnen Spalte.
Wir betrachten f"ur den Moment nur die Abh"angigkeit von der $k$-ten
Spalte und schreiben daf"ur
\[
\Delta_k(u)=\left|
\,\begin{matrix}
a_{11}&\dots&u_1&\dots&a_{1n}\\
\vdots&\ddots&\vdots&\ddots&\vdots\\
a_{n1}&\dots&u_n&\dots&a_{nn}
\end{matrix}
\,\right|.
\]
Setzen wir beide Seiten von (\ref{glinspalten}) in $\Delta_k$ ein,
erhalten wir unter Ausn"utzung der Linearit"at:
\[
x_1\Delta_k(a_1)+\dots+x_k\Delta_k(a_k)+\dots+x_n\Delta_k(a_n)=\Delta_k(b).
\]
Auf der linken Seite enthalten alle Terme ausser dem $k$-ten die
Spalte $a_i$ zweimal, einmal am Platz $i$, und einmal neu eingef"ugt am
Platz $k$. Da die Determinante verschwindet, wenn zwei Spalten "ubereinstimmen,
bleibt nur der $k$-te Term stehen:
\[
x_k\Delta_k(a_k)=\Delta_k(b)
\]
Auf der linken Seite setzt man die $k$-te Spalte als $k$-te Spalte
in $A$ ein und berechnet die Determinante, dies ist also nichts
anderes als die Determinante von $A$. Auf der rechten Seite ersetze man
die $k$-Spalte von $A$ durch $b$, also
\[
x_k
\left|\,\begin{matrix}
a_{11}&\dots&a_{1n}\\
\vdots&\ddots&\vdots\\
a_{n1}&\dots&a_{nn}\\
\end{matrix}\,\right|
=
\left|\,\begin{matrix}
a_{11}&\dots&b_1&\dots&a_{1n}\\
\vdots&\ddots&\vdots&\ddots&\vdots\\
a_{n1}&\dots&b_n&\dots&a_{nn}\\
\end{matrix}\,\right|
\]
Die Behauptung folgt jetzt durch Aufl"osen nach $x_k$.
\end{proof}
Dieser Satz stellt zwar eine h"ubsche Formel zur Berechnung der L"osung
bereit, praktisch n"utzlich ist diese jedoch kaum. Die Berechnung der
$n+1$ Determinanten ist bereits aufwendiger als die Durchf"uhrung des
Gaussverfahrens, welches die L"osung auch schon liefert.

\begin{beispiel}Man finde die L"osung des Gleichungssytems
\[
\begin{linsys}{4}
-x&-&3y&&&=&\color[rgb]{0,0.5,0}-7\\
2x&+&3y&-&2z&=&\color[rgb]{0,0.5,0}2\\
2x&+&y&-&3z&=&\color[rgb]{0,0.5,0}-5
\end{linsys}
\]
Die Koeffizientenmatrix und die rechte Seite sind
\[
A=\begin{pmatrix}
-1&-3&0\\
2&3&-2\\
2&1&-3
\end{pmatrix}
,\qquad
b=\begin{pmatrix}
\color[rgb]{0,0.5,0}-7\\\color[rgb]{0,0.5,0}2\\\color[rgb]{0,0.5,0}-5
\end{pmatrix}
\]
wobei wir fr"uher bereits $\det(A)=1$ gefunden haben. F"ur die L"osungen
des Gleichungssystems bekommen wir damit
\begin{align*}
x&=\frac{
\left|\;
\begin{matrix}
\color[rgb]{0,0.5,0}-7&-3&0\\
\color[rgb]{0,0.5,0}2&3&-2\\
\color[rgb]{0,0.5,0}-5&1&-3
\end{matrix}
\;\right|
}{\det(A)}
=63-30+0-0-14-18=1,
\\
\\
y&=\frac{
\left|\;
\begin{matrix}
-1&\color[rgb]{0,0.5,0}-7&0\\
2&\color[rgb]{0,0.5,0}2&-2\\
2&\color[rgb]{0,0.5,0}-5&-3
\end{matrix}
\;\right|
}{\det(A)}
=6+28+0-0+10-42=2,
\\
\\
z&=\frac{
\left|\;
\begin{matrix}
-1&-3&\color[rgb]{0,0.5,0}-7\\
2&3&\color[rgb]{0,0.5,0}2\\
2&1&\color[rgb]{0,0.5,0}-5
\end{matrix}
\;\right|
}{\det(A)}
=15-12-14+42+2-30=3.
\end{align*}
\end{beispiel}

\section{Inverse Matrix}
\index{inverse Matrix}
Mit den im letzten Abschnitt beschriebenen Minoren kann man jetzt auch
eine Formel f"ur die Elemente der inversen Matrix finden. Die inverse
Matrix von $A$ hat in ihren Spalten die L"osungen $x$ des Gleichungssystems 
$Ax=b$ f"ur ganz spezielle rechte Seiten $b$.
Die $j$-te Spalte ist die L"osung zur rechten Seite $e_j$, dem Einheitsvektor,
der genau an der $j$-ten Stelle eine $1$ hat, und sonst aus lauter Nullen
besteht. Nach der Cramerschen Regel kann man die $i$-te Unbekannte $x_i$
in $Ax=e_j$ mit Determinanten berechnen, n"amlich
\[
x_i=(-1)^{i+j}\frac{\det(A_{ji})}{\det(A)}.
\]
Dies ist auch der Eintrag in Spalte $j$ und Zeile $i$ der inversen
Matrix $A^{-1}$:
\begin{equation}
A^{-1}
=
\frac{1}{\det(A)}
\begin{pmatrix}
\det(A_{11})&-\det(A_{21})&\det(A_{31})& \dots&(-1)^{1+n} \det(A_{n1})\\
-\det(A_{12})&\det(A_{22})&-\det(A_{32})& \dots&(-1)^{2+n} \det(A_{n2})\\
\det(A_{13})&-\det(A_{23})&\det(A_{33})& \dots&(-1)^{3+n} \det(A_{n3})\\
\vdots&\vdots&\vdots&\ddots&\vdots\\
(-1)^{n+1}\det(A_{1n})&(-1)^{n+2}\det(A_{2n})&(-1)^{n+3}\det(A_{3n})& \dots&(-1)^{n+n} \det(A_{nn})\\
\end{pmatrix}
\label{inversematrix}
\end{equation}
\begin{definition}
Die Terme in der Matrix in (\ref{inversematrix})
heissen Kofaktoren der Matrix $A$.
Sie bilden die Matrix der Kofaktoren
\begin{equation}
\operatorname{cof}(A)_{ij}=
(-1)^{i+j}\det(A_{ij})
\label{cofactor}
\end{equation}
Mit dieser Notation ist die inverse Matrix
\begin{equation}
A^{-1}=\frac1{\det(A)}\operatorname{cof}(A)^t
\label{inversecofactors}
\end{equation}
\end{definition}

F"ur $2\times2$-Matrizen f"uhrt dies auf die manchmal n"utzliche
Formel
\[
\begin{pmatrix}
a&b\\c&d
\end{pmatrix}^{-1}
=
\frac1{ad-bc}\begin{pmatrix}
d&-b\\-c&a
\end{pmatrix}
\]
F"ur die Berechnung der Inversen gr"osserer Matrizen ist die Formel
nur in ganz speziellen F"allen von Nutzen. Die Berechnung der Inverse
mit Hilfe des Gauss-Algorithmus erfordert im allgemeinen deutlich weniger
Aufwand.
Hingegen ist die Formel f"ur theoretische "Uberlegungen durchaus
interessant. Es folgt aus ihr zum Beispiel, dass die Inverse einer
Matrix mit ganzzahligen Eintr"agen und Determinante $1$ wieder
lauter ganzzahlige Eintr"age hat. Die Menge
\[
\operatorname{SL}_n(\mathbb Z)=\{A\in M_{n}(\mathbb Z)|\det(A)=1\},
\]
hat also die Eigenschaft, dass mit $A\in\operatorname{SL}_2(\mathbb Z)$ 
auch $A^{-1}\in\operatorname{SL}_2(\mathbb Z)$.

\begin{beispiel}
Man bestimme die Inverse der Matrix
\[
A=\begin{pmatrix}
-1&-3&0\\
2&3&-2\\
2&1&-3
\end{pmatrix}
\]
Die Determinante von $A$ wurde fr"uher schon berechnet, $\det(A)=1$.
Die Minoren sind
\begin{align*}
\det A_{11}&=\left|\,\begin{matrix} 3&-2\\ 1&-3\end{matrix}\,\right|
%=-9+2
=-7
&
\det A_{12}&=\left|\,\begin{matrix} 2&-2\\ 2&-3\end{matrix}\,\right|
%=-6+4
=-2
&
\det A_{13}&=\left|\,\begin{matrix} 2& 3\\ 2& 1\end{matrix}\,\right|
%=2-6
=-4\\
\det A_{21}&=\left|\,\begin{matrix}-3& 0\\ 1&-3\end{matrix}\,\right|
=9
&
\det A_{22}&=\left|\,\begin{matrix}-1& 0\\ 2&-3\end{matrix}\,\right|
=3
&
\det A_{23}&=\left|\,\begin{matrix}-1&-3\\ 2& 1\end{matrix}\,\right|
%=-1+6
=5\\
\det A_{31}&=\left|\,\begin{matrix}-3& 0\\ 3&-2\end{matrix}\,\right|
=6
&
\det A_{32}&=\left|\,\begin{matrix}-1& 0\\ 2&-2\end{matrix}\,\right|
=2
&
\det A_{33}&=\left|\,\begin{matrix}-1&-3\\ 2& 3\end{matrix}\,\right|
%=-3+6
=3
\end{align*}
Beim Hinschreiben der Inversen muss man jetzt aber beachten,
dass das Element in Zeile $i$ und Spalte $j$ der inversen Matrix
mit $A_{ji}$ gebildet wird:
\[
A^{-1}=\frac1{\det(A)}\begin{pmatrix}
+\det(A_{11})&-\det(A_{21})&+\det(A_{31})\\
-\det(A_{12})&+\det(A_{22})&-\det(A_{32})\\
+\det(A_{13})&-\det(A_{23})&+\det(A_{33})
\end{pmatrix}
=\begin{pmatrix}
-7&-9&6\\
2&3&-2\\
-4&-5&3
\end{pmatrix}
\]
\end{beispiel}

\section{Eindeutigkeit der Determinante\label{deteindeutig}}
In diesem Abschnitt wollen wir zeigen, dass die Determinante
eindeutig ist. Wir gehen dazu wie folgt vor. Zun"achst zeigen wir,
dass sich jede Determinantenberechnung auf die Berechnung der Determinanten
von sehr speziellen $A$s zur"uckf"uhren l"asst, n"amlich solchen,
die genau eine $1$ in jeder Zeile und Spalte haben, und sonst nur
Nullen enthalten. Im zweiten Schritt zeigen wir dann, dass diese
speziellen Determinanten in jeder Definition dasselbe geben.

Die $k$-te Spalte $a$ von $A$ kann man als lineare Kombination von
speziellen Spalten schreiben:
\[
a_k=a_{1k}
\begin{pmatrix}
1\\0\\\vdots\\0
\end{pmatrix}
+a_{2k}
\begin{pmatrix}
0\\1\\\vdots\\0
\end{pmatrix}
+\dots+
+a_{nk}
\begin{pmatrix}
0\\0\\\vdots\\1
\end{pmatrix}
\]
Setzt man dies als $k$-te Spalte in die Determinante ein (wir hatten fr"uher
daf"ur die Bezeichnung $\Delta_k(a_k)$ eingef"uhrt), kann man mit
der Linearit"at  die Determinante schreiben als
\[
\det A=a_{1k}\Delta_k\left(
\begin{matrix}
1\\0\\\vdots\\0
\end{matrix}
\right)
+a_{2k}
\Delta_k\left(
\begin{matrix}
0\\1\\\vdots\\0
\end{matrix}
\right)+\dots+
\Delta_k\left(
\begin{matrix}
0\\0\\\vdots\\1
\end{matrix}
\right).
\]
Die $\Delta_k$ auf der rechten Seite sind Determinanten, in denen 
eine Spalte ersetzt worden ist durch eine Spalte aus Nullen und genau
einer Eins.

Diese Zerlegung kann man jetzt noch $(n-1)$-mal wiederholen,
jedesmal kommt ein anderer Faktor $a_{ik}$ vor die Determinante.
Dabei werden auch Determinanten ensteht die die gleichen Spalten
haben, die also die Eins an der selben Stelle haben. Diese Determinanten
verschwinden alle, man kann sie ignorieren. Der Vorfaktor ist
jeweils $a_{ik}$, wobei $i$ die Zeile der Eins bezeichnet, und $k$
die Nummer der Spalte, die man ersetzt hat. Die Indizes des
Vorfaktors $a_{ik}$ sind gerade die Koordinaten der Eins, die durch
die Ersetzung in der Determinate stehen bleibt.

Am Ende bleibt eine Summe von Termen stehen, in der alle Determinanten
in jeder Zeile und Spalte genau eine $1$ und sonst lauter Nullen
enthalten. Die Vorfaktoren sind die Produkte der Koeffizienten in $A$,
die an Stelle der $1$ in der urspr"unglichen Matrizen standen.

Wir bezeichnen die speziellen Matrizen aus lauter Nullen und Einsen
mit $\sigma$, und die Menge all dieser Matrizen mit $S_n$. Aus einem
Index $i$ kann man die zugeh"orige Spalte $k$ herausfinden, indem
man die einzige Spalte in $\sigma$ sucht, welche an der $i$-ten
Stelle eine $1$ enth"alt. Wir schreiben $\sigma(i)$ f"ur dieses $k$.
Mit dieser Schreibweise ist 
\[
\det A=\sum_{\sigma\in S_n}
a_{1\sigma(1)}
a_{2\sigma(2)}
\dots
a_{n\sigma(n)}
\det \sigma.
\]
Aus dieser Formel ist jetzt klar, dass die Determinante nicht vom
Vorgehen abh"angt.

Dieselben Aufl"osungen kann man auch mit Hilfe der Linearit"at der
Determinante als Funktion der Spalten vornehmen, dabei kommt die 
gleiche Formel heraus. Wenn man jetzt noch zeigen kann, dass
die Determinanten $\det\sigma$ nicht davon abh"angen, ob man von
Zeilen oder von Spalten ausgeht, dann ist klar, dass es nur eine
Determinante gibt.

Die Determinante von $\sigma$ muss $1$ oder $-1$ sein, denn durch
Zeilenvertauschungen kann sie auf die Form $E$ gebracht werden.
Alternativ kann dies mit Spaltenvertauschungen geschehen. Jede
Vertauschung tr"agt einen Faktor $-1$ zum Wert der Determinante
bei. Es kommt also nur darauf an, ob die Zahl der n"otigen
Vertauschungen gerade oder ungerade ist.
