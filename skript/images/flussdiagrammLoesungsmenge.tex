%
% flussdiagrammLoesungsmenge.tex -- Entscheidungsdiagram über Lösbarkeit
%
% (c) 2017 Tabea Méndez, Hochschule Rapperswil
%
\definecolor{greenT}	{RGB}{0,170,0}

\begin{tikzpicture}[>=latex, scale=1,
    connection/.style={draw=black,thick},
	processStep/.style={rectangle,inner sep=3pt,minimum height=1cm,minimum width=3cm,draw=black,semithick},	
	decision/.style={diamond,inner sep=-1pt,minimum height=1.5cm,minimum width=3cm,draw=black,semithick,aspect=2.5},	
	start/.style={circle,inner sep=0pt,minimum size=16.5pt, draw=black,fill=black,semithick},	
   connection/.style={draw=black,thick,->},
]	

	\def\step{2.3};
	\def\shift{5};



	\node[start](begin) at(0,-0.2*\step){ };
	\node[decision](s0) at(0,-1*\step){\textcolor{red}{$\ast=0\,?$}};
	\node[decision,inner sep=1pt](s1) at(0,-2*\step){\textcolor{greenT}{\parbox{3cm}{\centering Hat es frei wähl-\\ bare Variabeln?}}};

	\node[processStep](s2) at(\shift,-1*\step){Keine Lösung};
	\node[processStep](s3) at(\shift,-2*\step){Eine Lösung};
	\node[processStep](s4) at(0,-2.9*\step){$\infty$-viele Lösungen};


	\draw[connection](begin)--(s0);
	\draw[connection](s0)--node[right,yshift=2pt]{\footnotesize Ja}(s1);
	\draw[connection](s0)--node[above]{\footnotesize Nein}(s2);
	\draw[connection](s1)--node[above]{\footnotesize Nein}(s3);
	\draw[connection](s1)--node[right,yshift=2pt]{\footnotesize Ja}(s4);

\end{tikzpicture}



