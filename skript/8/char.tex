%
% char.tex
%
% (c) 2018 Prof Dr Andreas Müller, Hochschule Rapperswil
%
\section{Charakteristische Gleichung}
\rhead{Charakteristische Gleichung}
\index{charakteristische Gleichung}
Welche Eigenwerte und Eigenvektoren hat eine Matrix? Ein Eigenvektor 
ist ein von $0$ verschiedener Vektor $x$, der die Gleichung
$Ax=\lambda x$ erfüllt, oder wie wir früher gesehen haben,
eine nicht verschwindende Lösung des homogenen Gleichungssystems
$(A-\lambda E)x=0$.
Im Kapitel \ref{chapter-determinanten} wurde gezeigt, dass ein eine solche
Lösung nur dann existieren kann, wenn die Determinante des Gleichungssystems
verschwindet, also
\[
\det(A-\lambda E)=
\left|
\begin{matrix}
a_{11}-\lambda&a_{12}&\dots&a_{1n}\\
a_{21}&a_{22}-\lambda&\dots&a_{2n}\\
\vdots&\vdots&\ddots&\vdots\\
a_{n1}&a_{n2}&\dots&a_{nn}-\lambda
\end{matrix}
\right|
\]
Die Determinante ist ein Polynom $n$-ten Grades in $\lambda$, das Finden der
Eigenwerte läuft also darauf hinaus, Nullstellen eines Polynoms zu finden.
\begin{satz}
Die Eigenwerte einer $n\times n$ Matrix $A$ sind Nullstellen des
Polynoms
\[
\chi_A(\lambda)=\det(A-\lambda E)
\]
vom Grad $n$.
\end{satz}
\begin{definition}
\index{charakteristisches Polynom}
Das Polynom $\chi_A(\lambda)=\det(A-\lambda E)$ heisst
charakteristisches Polynom,
die Gleichung $\chi_A(\lambda)=0$ heisst 
charakteristische Gleichung.
\end{definition}

\begin{beispiel}
Die Matrix $A=\begin{pmatrix}0&1\\1&0\end{pmatrix}$ hat das 
charakteristische Polynom
\begin{align*}
\det(A-\lambda I)&=\left|\begin{matrix}-\lambda&1\\1&-\lambda\end{matrix}\right|\\
&=\lambda^2-1=(\lambda+1)(\lambda-1)
\end{align*}
mit den Lösungen $\lambda_\pm=\pm1$.
Um die Eigenvektoren zu finden, muss
man jetzt das Gleichungssystem $(A-\lambda E)x=0$ bestimmen.
Für die Eigenwerte $\pm1$ hat das Gleichungssystem die Matrizen
\[
\begin{pmatrix}
-1&1\\1&-1
\end{pmatrix}\quad \text{für $\lambda=1$},\qquad
\begin{pmatrix}
1&1\\1&1
\end{pmatrix}\quad\text{für $\lambda=-1$}
\]
mit den Lösungsvektoren 
\[
\vec v_+=\begin{pmatrix}1\\1\end{pmatrix},
\qquad
\vec v_-=\begin{pmatrix}1\\-1\end{pmatrix}.
\]
\end{beispiel}

