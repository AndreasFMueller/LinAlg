%
% problem.tex
%
% (c) 2018 Prof Dr Andreas Müller, Hochschule Rapperswil
%
\section{Problemstellung}
\rhead{Problemstellung}
In diesem Kapitel gehen wir immer von einer $n\times n$-Matrix $A$ aus.

\medskip
{\parindent0pt \bf Problem:} Finde Vektoren $x\in\mathbb R^n$ und
Zahlen $\lambda\in\mathbb R$ so, dass $Ax=\lambda x$.
\medskip

{\parindent0pt Diese Problemstellung ist nicht ganz klar:}
\begin{compactitem}
\item Der Vektor $x=0$ ist immer eine Lösung, das Problem ist also
nur dann interessant, wenn Vektoren $x\ne 0$ verlangt werden.
\item Der Vektor $x$ ist nicht eindeutig bestimmt.
Gilt $Ax=\lambda x$, dann gilt dasselbe auch für $y=\mu x$:
\[
Ay= A(\mu x)=\mu Ax=\mu\lambda x = \lambda (\mu x)=\lambda y.
\]
Der Vektor $x$ kann also bestenfalls bis auf ein Vielfaches bestimmt werden.
\end{compactitem}

\index{Eigenwert}
\index{Eigenvektor}
\begin{definition}
Ein Vektor $v \in \mathbb R^n$ heisst Eigenvektor der $n\times n$-Matrix $A$
zum Eigenwert $\lambda$, wenn $v\ne 0$ und $Av=\lambda v$ gilt.
\end{definition}

\begin{beispiel}
Die Matrix
\[
\begin{pmatrix}
3&0&0\\
0&5&0\\
0&0&7
\end{pmatrix}
\]
hat die Eigenwerte $3$, $5$ und $7$, mit den Eigenvektoren
\[
\begin{pmatrix}
1\\0\\0
\end{pmatrix}
,
\begin{pmatrix}
0\\1\\0
\end{pmatrix}
,
\begin{pmatrix}
0\\0\\1
\end{pmatrix}
\]
\end{beispiel}

\begin{hilfssatz}
\index{Diagonalmatrix}
Die Diagonalmatrix
\[
\operatorname{diag}(\lambda_1,\dots,\lambda_n)
=\begin{pmatrix}
\lambda_1&\dots&0\\
\vdots&\ddots&\vdots\\
0&\dots&\lambda_n
\end{pmatrix}
\]
hat die Eigenwerte $\lambda_1,\dots,\lambda_n$, die Standardbasisvektoren
$e_i$ sind Eigenvektoren zum Eigenwert $\lambda_i$.
\end{hilfssatz}
Offenbar ist dies ein besonders einfacher Fall, die Basisvektoren sind
alle Eigenvektoren.
Es gibt einige Klassen von Matrizen, bei denen es
immer möglich ist, eine Basis aus Eigenvektoren zu wählen.
Hat die
$n\times n$-Matrix $A$ $n$ linear unabhängig Eigenvektoren $v_1,\dots,v_n$
mit Eigenwerten $\lambda_1,\dots,\lambda_n$, dann kann man die Wirkung
der Matrix $A$ auf einen beliebigen Vektor $x=\alpha_1v_1+\dots+\alpha_nv_n$
sofort berechnen:
\[
Ax=\alpha_1Av_1+\dots+\alpha_nAv_n=\alpha_1\lambda_1v_1+\dots+\alpha_n\lambda_nv_n.
\]
In der Basis $\{v_1,\dots,v_n\}$ entspricht $A$ also der Matrix
\[
A=\operatorname{diag}(\lambda_1,\dots,\lambda_n)
=
\begin{pmatrix}
\lambda_1&\dots&0\\
\vdots&\ddots&\vdots\\
0&\dots&\lambda_n
\end{pmatrix}.
\]
\begin{definition}
\index{diagonalisierbar}
\index{diagonalisieren}
Eine Matrix $A$ heisst diagonalisierbar, wenn es eine Basis aus
Eigenvektoren gibt.
\end{definition}

Wir haben also die Gleichung $Ax=\lambda x$ zu lösen.
Wäre $\lambda$ bekannt, könnten wir das Problem in ein Standardproblem
umwandeln:
\[
Ax=\lambda x\qquad\Rightarrow\qquad Ax-\lambda Ex=(A-\lambda E)x=0.
\]
Gesucht ist also eine nicht verschwindende Lösung des homogenen
Gleichungssystems mit
Matrix $A-\lambda E$.
Nur kennen wir $\lambda$ nicht.
Allerdings
liefert gerade die Bedingung, dass wir ein nicht verschwindende
Lösung des homogenen Systems finden wollen, ein Kriterium zur
Bestimmung von $\lambda$: $A-\lambda E$ muss singulär sein.
Das Eigenwertproblem wird also in zwei Schritten gelöst:
\begin{compactenum}
\item Bestimmung der möglichen Eigenwerte: Für welche Werte von $\lambda$
ist $A-\lambda E$ singulär?
\item Bestimmung der Eigenvektoren zu gegebenem Eigenwert: Finde die
Lösungen von $(A-\lambda E)x=0$.
Hierzu kann der Gauss-Algorithmus
verwendet werden, oder in einfachen Fällen auch die Cramersche Regel.
\end{compactenum}
Die folgenden Einführungsbeispiele sollen zeigen, wie das Eigenwertproblem
in der Praxis auftaucht und gelöst werden kann.

