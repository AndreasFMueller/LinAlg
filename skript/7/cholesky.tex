%
% cholesky.tex -- Cholesky-Zerlegung
%
% (c) 2009 Prof Dr Andreas Mueller, Hochschule Rapperswil
%
\section{Cholesky-Zerlegung}
\index{Cholesky-Zerlegung}
\index{Zerlegung!Cholesky}
Kann man aus einer Matrix die Wurzel ziehen? Was wäre das überhaupt?
Aus einer Diagonalmatrix mit positiven Diagonalelementen kann man offenbar
die Wurzel ziehen, in dem man die Wurzel aus den Diagonalelementen zieht.
Negative Diagonalelemente machen dies unmöglich.
Man muss also auf jeden
Fall fordern, dass die Matrix in einem gewissen Sinn positiv ist.

\begin{definition}
Eine symmetrische Matrix $A$ heisst positiv definit, falls $v^tAv>0$ für
alle Vektoren $v\ne 0$.
\end{definition}

Im Abschnitt \ref{section-least-squares} haben wir eine Anwendung
kennengelernt, in der ein Gleichungssystem der Form
\[
A^tAx=A^t b
\]
gelöst werden musste.
Die Matrix $A^tA$ ist genau von der hier
beschriebenen Art: symmetrisch und positiv definit.
Es ist nämlich
\[
v^tA^tAv=(Av)^tAv=(Av)\cdot (Av)=|Av|^2> 0
\]
falls $A$ regulär, wie für positiv definite Matrizen gefordert.

Für positiv definite symmetrische Matrizen gibt es eine Quadratwurzel
im Sinne des folgenden Satzes:
\begin{satz}[Cholesky-Zerlegung]
\index{Cholesky-Zerlegung}
\index{Zerlegung!Cholesky}
Ist $A$ eine symmetrische positiv definite Matrix, dann gibt es genau eine
untere Dreiecksmatrix $L$ mit $A=LL^t$.
\end{satz}

Anstelle eines Beweises zeigen wir einen Algorithmus, mit dem man die Matrix
$L$ konstruieren kann.
Das Verfahren startet mit einer positiv definiten
symmetrischen $n\times n$-Matrix $A$.
Gesucht wird eine untere Dreiecksmatrix $L$
mit $A=LL^t$.
Die Matrix $L$ kann man symbolisch in der Form 
\[
L=\left(
\begin{tabular}{>{$}c<{$}|>{$}c<{$}>{$}c<{$}>{$}c<{$}}
l_{11}&&0&\\
\hline
&&&\\
l&&L'&\\
&&&
\end{tabular}
\right)
\]
Darin ist $l'$ ein $n-1$-dimensionaler Spaltenvektor und $L'$ eine
untere Dreiecksmatrix mit $n-1$ Zeilen und Spalten.
Wir wollen $LL^t$ mit $A$ vergleichen, und unterteilen deshalb auch die
Matrix $A$ nach dem gleichen Muster
\[
A=\left(
\begin{tabular}{>{$}c<{$}|>{$}c<{$}>{$}c<{$}>{$}c<{$}}
a_{11}&&a^t&\\
\hline
&&&\\
a&&A'&\\
&&&
\end{tabular}
\right).
\]
Dann bekommen wir für $LL^t$:
\[
LL^t=
\left(
\begin{tabular}{>{$}c<{$}|>{$}c<{$}>{$}c<{$}>{$}c<{$}}
l_{11}&&0&\\
\hline
&&&\\
l&&L'&\\
&&&
\end{tabular}
\right)
\left(
\begin{tabular}{>{$}c<{$}|>{$}c<{$}>{$}c<{$}>{$}c<{$}}
l_{11}&&l^t&\\
\hline
&&&\\
0&&L'^t&\\
&&&
\end{tabular}
\right)
=
\left(
\begin{tabular}{>{$}c<{$}|>{$}c<{$}>{$}c<{$}>{$}c<{$}}
l_{11}^2&&l_{11}l^t&\\
\hline
&&&\\
l_{11}l&&ll^t+L'L'^t&\\
&&&
\end{tabular}
\right)
\]
Daraus können wir ablesen, dass 
\begin{align*}
a_{11}&=l_{11}^2   &\Rightarrow&&l_{11}&=\sqrt{a_{11}}\\
     a&=l_{11}l    &\Rightarrow&&     l&=\frac1{\sqrt{a_{11}}}a\\
    A'&=ll^t+L'L'^t&\Rightarrow&&L'L'^t&=A'-ll^t
\end{align*}
Um $L$ zu finden, muss also nur noch eine Zerlegung von $A'-ll^t$
finden.
Damit wurde das $n$-dimensionale Problem auf ein $n-1$-dimensionales
Problem reduziert.
Durch Wiederholung dieses Schrittes gelangt man am Schluss
zu einem eindimensionalen Problem, also dem Problem, die Wurzel aus einer
positiven Zahl zu ziehen.

\begin{beispiel}
Man finde die Cholesky-Zerlegung der Matrix
\[
A=\begin{pmatrix}
9&3&3\\
3&17&21\\
3&21&107
\end{pmatrix}.
\]
Der erste Schritt leitet daraus ab, dass
\begin{align*}
l_{11}&=3,&
l&=\frac13\begin{pmatrix}3\\3\end{pmatrix}=\begin{pmatrix}1\\1\end{pmatrix},&
ll^t&=\begin{pmatrix}1&1\\1&1\end{pmatrix},&
A'-ll^t&=\begin{pmatrix}16&20\\20&106\end{pmatrix}
\end{align*}
Bis jetzt ist bekannt, dass
\[
L=\begin{pmatrix}
3&0&0\\
1&?&?\\
1&?&?
\end{pmatrix}
\]
Für den unbekannten Teil wird jetzt der gleiche Algorithmus auf
die Matrix $A'-ll^t$ angewendet.
Im zweiten Schritt muss jetzt die Matrix $A'-ll^t$ zerlegt werden.
Wir rezyklieren die Bezeichnungen aus dem ersten Schritt, und
bekommen
\begin{align*}
l_{11}&=4,&
l&=\frac14\begin{pmatrix}20\end{pmatrix}=\begin{pmatrix}5\end{pmatrix},&
ll^t&=\begin{pmatrix}25\end{pmatrix},&
A'-ll^t&=\begin{pmatrix}81\end{pmatrix}
\end{align*}
Damit ist jetzt von $L$ bekannt:
\[
L=\begin{pmatrix}
3&0&0\\
1&4&0\\
1&5&?
\end{pmatrix}
\]
Für das fehlende Element muss jetzt nur noch die Cholesky-Zerlegung
einer $1\times 1$-Matrix bestimmt werden, dies ist aber die 
Quadratwurzel.
Somit ist
\[
L=
\begin{pmatrix}
3&0&0\\
1&4&0\\
1&5&9
\end{pmatrix}
,\quad\text{Kontrolle:}\;
\begin{pmatrix}
3&0&0\\
1&4&0\\
1&5&9
\end{pmatrix}
\begin{pmatrix}
3&1&1\\
0&4&5\\
0&0&9
\end{pmatrix}
=\begin{pmatrix}
9&3&3\\
3&17&21\\
3&21&107
\end{pmatrix}=A.
\]
\end{beispiel}

{\small
Unsere Beschreibung des Algorithmus für die Cholesky-Zerlegung ist in
einem Punkt unvollständig.
Woher wissen wir, dass $A'-ll^t$ positiv
definit ist? Dies ist ja die Voraussetzung, dass der Algorithmus
überhaupt angewendet werden kann.

Wir wissen, dass $A$ und $A'$ positiv definit sind.
Für jeden $n-1$-dimensionalen Vektor $v$ können wir einen Vektor
\[
\tilde v=\left(
\begin{tabular}{c}
$\alpha$\\
\hline
$\mathstrut$\\
$v$\\
$\mathstrut$
\end{tabular}
\right)
\]
bilden.
Nach Voraussetzung ist $v^tA'v>0$ und $\tilde v^tA\tilde v>0$.
Wir möchten nachrechnen, dass $v^t(A'-ll^t)v>0$ gilt, das würde bedeuten,
dass $A'-ll^t$ positiv definit ist.
Es ist
\begin{align*}
0\ge
\tilde v^tA\tilde v
&=
\left(
\begin{tabular}{>{$}c<{$}|>{$}c<{$}>{$}c<{$}>{$}c<{$}}
\alpha&&v^t&
\end{tabular}
\right)
\left(
\begin{tabular}{>{$}c<{$}|>{$}c<{$}>{$}c<{$}>{$}c<{$}}
a_{11}&&a^t&\\
\hline
&&&\\
a&&A'&\\
&&&
\end{tabular}
\right)
\left(
\begin{tabular}{>{$}c<{$}}
\alpha\\
\hline
\mathstrut\\
v\\
\mathstrut
\end{tabular}
\right)
\\
&=
\left(
\begin{tabular}{>{$}c<{$}|>{$}c<{$}>{$}c<{$}>{$}c<{$}}
\alpha&&v^t&
\end{tabular}
\right)
\left(
\begin{tabular}{>{$}c<{$}}
a_{11}\alpha+a^tv\\
\hline
\\
\alpha a+A'v\\
\\
\end{tabular}
\right)
\\
&=\alpha^2a_{11}+\alpha a^tv+\alpha v^ta+v^tA'v
\\
&=\alpha^2a_{11}+2\alpha (a\cdot v)+v^tA'v
\end{align*}
Setzt man $\alpha=-(a\cdot v)/a_{11}$, wird daraus
\[
\frac{(a\cdot v)^2}{a_{11}^2}a_{11}-2\frac{(a\cdot v)^2}{a_{11}}+v^tAv
=v^tAv-(l\cdot v)^2\ge 0.
\]
Andererseits ist
\[
v^t(A-ll^t)v=v^tAv-v^tll^tv=v^tAv-(l\cdot v)^2\ge 0,
\]
also ist $A-ll^t$ positiv definit.
}

