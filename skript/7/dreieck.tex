%
% dreieck.tex
%
% (c) 2009 Prof Dr Andreas Mueller, Hochschule Rapperswil
%
\section{Zerlegung von Dreiecksmatrizen}
\rhead{Zerlegung von Dreiecksmatrizen}
\index{Dreiecksmatrix}
Als Vorbereitung untersuchen wir einige Beispiele von möglichen
Zerlegungen von Dreiecksmatrizen.
In diesem Abschnitt ist $L$ immer eine untere Dreiecksmatrix
\[
L=\begin{pmatrix}
l_{11}&0     &\dots &0     \\
l_{21}&l_{22}&\dots &0     \\
\vdots&\vdots&\ddots&\dots \\
l_{n1}&l_{n2}&\dots &l_{nn}
\end{pmatrix}
\]
\begin{hilfssatz}
Eine untere Dreiecksmatrix $L$ kann als Produkt einer Diagonalmatrix
mit einer unteren Dreiecksmatrix $L_0$ mit Einsen auf der Diagonalen
geschrieben werden, $L=\operatorname{diag}(l_{11},\dots,l_{nn}) L_0$.
\end{hilfssatz}

\begin{proof}[Beweis]
Für $L_0$ muss man die Matrix
\begin{align*}
L_0&=
\operatorname{diag}(l_{11},\dots,l_{nn})^{-1} L
=
\begin{pmatrix}
\frac1{l_{11}}&0             &\dots &0\\
0             &\frac1{l_{22}}&\dots &\vdots\\
\vdots        &\vdots        &\ddots&\vdots\\
0             &0             &\dots &\frac1{l_{nn}}
\end{pmatrix}
\begin{pmatrix}
l_{11}&0     &\dots &0     \\
l_{21}&l_{22}&\dots &0     \\
\vdots&\dots &\ddots&\dots \\
l_{n1}&l_{n2}&\dots &l_{nn}
\end{pmatrix}
\\
&=
\begin{pmatrix}
\frac{l_{11}}{l_{11}}&0                    &\dots &0     \\
\frac{l_{21}}{l_{22}}&\frac{l_{22}}{l_{22}}&\dots &0     \\
\vdots               &\vdots               &\ddots&\vdots\\
\frac{l_{n1}}{l_{nn}}&\frac{l_{n2}}{l_{nn}}&\dots &\frac{l_{nn}}{l_{nn}}
\end{pmatrix}
=
\begin{pmatrix}
1                    &0                    &\dots &0     \\
\frac{l_{21}}{l_{22}}&1                    &\dots &0     \\
\vdots               &\vdots               &\ddots&\vdots\\
\frac{l_{n1}}{l_{nn}}&\frac{l_{n2}}{l_{nn}}&\dots &1
\end{pmatrix}
\end{align*}
nehmen.
\end{proof}
Offenbar entsteht $L_0$ aus $L$, indem man jede Zeile durch das zugehörige
Diagonalelement teilt, ähnlich wie man das im Laufe des Gauss-Algorithmus
macht.

\begin{hilfssatz}
Sei $L$ eine unter Dreiecksmatrix mit $n$ Zeilen und Spalten, und $k<n$.
Dann kann man $L$ in zwei untere Dreiecksmatrizen aufteilen:
\begin{align*}
L&=
\begin{pmatrix}
l_{11}   &\dots &0        &0          &\dots &0     \\
\vdots   &\ddots&\vdots   &\vdots     &\ddots&\vdots\\
l_{k1}   &\dots &l_{kk}   &0          &\dots &0     \\
l_{k+1,1}&\dots &l_{k+1,k}&l_{k+1,k+1}&\dots &0     \\
\vdots   &\ddots&\vdots   &\vdots     &\ddots&\vdots\\
l_{nn}   &\dots &l_{nk}   &l_{n,k+1}  &\dots &l_{nn}
\end{pmatrix}
\\
&=
\begin{pmatrix}
l_{11}   &\dots &0        &0          &\dots &0     \\
\vdots   &\ddots&\vdots   &\vdots     &\ddots&\vdots\\
l_{k1}   &\dots &l_{kk}   &0          &\dots &0     \\
l_{k+1,1}&\dots &l_{k+1,k}&1          &\dots &0     \\
\vdots   &\ddots&\vdots   &\vdots     &\ddots&\vdots\\
l_{nn}   &\dots &l_{nk}   &0          &\dots &1
\end{pmatrix}
\begin{pmatrix}
1        &\dots &0        &0          &\dots &0     \\
\vdots   &\ddots&\vdots   &\vdots     &\ddots&\vdots\\
0        &\dots &1        &0          &\dots &0     \\
0        &\dots &0        &l_{k+1,k+1}&\dots &0     \\
\vdots   &\ddots&\vdots   &\vdots     &\ddots&\vdots\\
0        &\dots &0        &l_{n,k+1}  &\dots &l_{nn}
\end{pmatrix}
\end{align*}
\end{hilfssatz}

\begin{proof}[Beweis]
Durch Nachrechnen.
\end{proof}

Diese Zerlegung kann man natürlich iterieren, bis in jeder Teilmatrix
nur noch eine Spalte übrig bleibt.
Schreiben wir
\[
L_i=\begin{pmatrix}
1      &\dots &0     &\dots &0     \\
\vdots &\ddots&\vdots&\ddots&\vdots\\
0      &\dots &l_{ii}&\dots &0     \\
\vdots &\ddots&\vdots&\ddots&\vdots\\
0      &\dots &l_{ni}&\dots &1
\end{pmatrix},
\]
dann können wir $L$ schreiben als
\begin{equation}
L=L_1L_2\dots L_{n-1}L_n.
\label{lproductdecomposition}
\end{equation}

