%
% lr.tex -- Zerlegung in Dreiecksmatrizen
%
% (c) 2009 Prof Dr Andreas Mueller, Hochschule Rapperswil
%
\section{Zerlegung einer Matrix in Dreiecksmatrizen}
\rhead{Zerlegung einer Matrix in Dreiecksmatrizen}
\index{LU-Zerlegung}
\index{Zerlegung!LU}
Der Gauss-Algorithmus produziert mit bei der Vorwärtsreduktion eine
obere Dreiecksmatrix, die noch dazu lauter Einsen auf der Diagonalen
hat.
Es liegt also nahe, eine Zerlegung von $A$ mit dieser Dreiecksmatrix
als Faktor $U$ zu versuchen.

Um den zweiten Faktor zu finden, studieren wir nochmals den
Gauss-Algorithmus.
Im $k$-ten Schritt wird das Diagonalelement
$a_{kk}$ zu $1$ gemacht, und damit werden alle Elemente späteren
Elemente in der Spalte eliminiert.
Man weiss also bereits, was an
diesen Stellen in der Matrix stehen wird, es ist nicht nötig, diese
Einträge zu berechnen oder zu speichern.
Wir lassen daher diese Einträge stehen.
Das Gauss-Tableau sieht dann nach der Vorwärtsreduktion
so aus:
\[
\begin{tabular}{|>{$}c<{$}>{$}c<{$}>{$}c<{$}>{$}c<{$}>{$}c<{$}|}
\hline
a_{11}&a_{12}&a_{13}&\dots &a_{1n}\\
a_{21}&a_{22}&a_{23}&\dots &a_{2n}\\
a_{31}&a_{32}&a_{33}&\dots &a_{3n}\\
\vdots&\vdots&\vdots&\ddots&\vdots\\
a_{n1}&a_{n2}&a_{n3}&\dots &a_{nn}\\
\hline
\end{tabular}
\rightarrow
\begin{tabular}{|>{$}c<{$}>{$}c<{$}>{$}c<{$}>{$}c<{$}>{$}c<{$}|}
\hline
l_{11}&u_{12}&u_{13}&\dots &u_{1n}\\
l_{21}&l_{22}&u_{23}&\dots &u_{2n}\\
l_{31}&l_{32}&l_{33}&\dots &u_{3n}\\
\vdots&\vdots&\vdots&\ddots&\vdots\\
l_{n1}&l_{n2}&l_{n3}&\dots &l_{nn}\\
\hline
\end{tabular}
\]
Zu beachten ist ferner, dass der Gauss-Algorithmus im $k$-ten Schritt nur 
die Elemente in den Zeilen mit Indices $\ge k$ und Spalten mit Indices $>k$ 
modifiziert.
Die beiden Matrizen $L$ und $R$ sind dann
\[
L=\begin{pmatrix}
l_{11}&0     &0     &\dots &0\\
l_{21}&l_{22}&0     &\dots &0\\
l_{31}&l_{32}&l_{33}&\dots &0\\
\vdots&\vdots&\vdots&\ddots&\vdots\\
l_{n1}&l_{n2}&l_{n3}&\dots &l_{nn}
\end{pmatrix},
\qquad
U=
\begin{pmatrix}
1     &u_{12}&u_{13}&\dots &u_{1n}\\
0     &1     &u_{23}&\dots &u_{2n}\\
0     &0     &1     &\dots &u_{3n}\\
\vdots&\vdots&\vdots&\ddots&\vdots\\
0     &0     &0     &\dots &1
\end{pmatrix}.
\]
Wir wissen bereits aus (\ref{lproductdecomposition}), dass sich $L$
ein Produkt von Matrizen $L_i$ zerlegen lässt, die jeweils nur
die $i$-te Spalte von $L$ enthalten, im übrigen aber wie eine Einheitsmatrix
aussehen.

Wir behaupten jetzt, dass $L_i$ den $i$-ten Gauss-Vorwärtsredutionsschritt
rück\-gängig macht.
Da $U$ das Resultat der Vorwärtsreduktion ist, 
muss $LU$ die ursprüngliche Matrix $A$ sein.

Um den $i$-ten Gauss-Schritt rückgängig zu machen, muss die
$i$-Zeile mit dem Pivot-Element multipliziert werden.
Die Multiplikation
der Matrix $L_k$ mit einer Matrix $B$ liefert in Zeile $i$ aber genau das Produkt
von
\[
\begin{pmatrix}
0&\dots&l_{ii}&\dots &0
\end{pmatrix}
\]
mit allen Spalten von $B$, also jeweils das $i$-te Element dieser
Spalten multipliziert mit $l_{ii}$:
\begin{align*}
\begin{pmatrix}
0&\dots&l_{ii}&\dots &0
\end{pmatrix}
\begin{pmatrix}
b_{11}&\dots &b_{1n}\\
\vdots&\ddots&\vdots\\
b_{i1}&\dots &b_{in}\\
\vdots&\ddots&\vdots\\
b_{n1}&\dots &b_{nn}
\end{pmatrix}
=
\begin{pmatrix}
l_{ii}b_{i1}&\dots &l_{ii}b_{ii}&\dots &l_{ii}b_{in}
\end{pmatrix}
\end{align*}
Multiplikation mit $L_i$ macht
also die Division durch das Pivot-Element rückgängig.

Eine spätere Zeile $k>i$ von $L_i$ hat die Form
\[
\begin{pmatrix}
0&\dots&l_{ki}&\dots&1&\dots\\
\end{pmatrix}.
\]
Multipliziert man sie mit der Matrix $B$, bekommt man:
\begin{align*}
\begin{pmatrix}
0&\dots&l_{ki}&\dots&1&\dots\\
\end{pmatrix}
\begin{pmatrix}
b_{11}&\dots &b_{1n}\\
\vdots&\ddots&\vdots\\
b_{i1}&\dots &b_{in}\\
\vdots&\ddots&\vdots\\
b_{k1}&\dots&b_{kn}\\
\vdots&\ddots&\vdots\\
b_{n1}&\dots &b_{nn}
\end{pmatrix}
&=
\begin{pmatrix}
l_{ki}b_{i1}+b_{k1}
&\dots&
l_{ki}b_{in}+b_{kn}
\end{pmatrix}
\\
&=
l_{ki}
\begin{pmatrix}
b_{i1}
&\dots&
b_{in}
\end{pmatrix}
+
\begin{pmatrix}
b_{k1}
&\dots&
b_{kn}
\end{pmatrix}
\end{align*}
Zur Zeile $k$ wird also das $l_{ki}$-fache der Zeile
$i$ hinzuaddiert.
Das ist genau die Umkehrung dessen, was
die Gauss-Elimination macht: da wurde von der Zeile $k$
das $l_{ki}$-fache der $i$-ten Zeile subtrahiert.

Damit ist gezeigt, dass das $A=LU$, wir haben also den folgenden
Satz bewiesen.

\begin{satz}[LU-Zerlegung]
\index{LU-Zerlegung}
\index{Zerlegung!LU}
\label{ludecomposition}
Sei $A$ eine $m\times n$ Matrix mit Rang $r$.
Dann gibt es eine $m\times r$-Matrix $L$ und eine $r\times n$-Matrix
$U$ mit $A=LU$.
Ausserdem haben $L$ und $U$ die folgende Form:
\[
L=\begin{pmatrix}
l_{11}&0&\dots&0\\
l_{21}&l_{22}&\dots&0\\
\vdots&\vdots&\ddots&\vdots\\
l_{r1}&l_{r2}&\dots&l_{rr}\\
\vdots&\vdots& &\vdots\\
l_{m1}&l_{m2}&\dots&l_{mr}
\end{pmatrix},\qquad
U=\begin{pmatrix}
1&u_{12}&\dots&u_{1r}&u_{1,r+1}&\dots&u_{1n}\\
0&1     &\dots&u_{2r}&u_{2,r+1}&\dots&u_{2n}\\
\vdots&\vdots&\ddots&\vdots&\vdots\\
0&0&\dots&1&u_{r,r+1}&\dots&u_{rn}
\end{pmatrix}
\]
\end{satz}

Für quadratische $n\times n$-Matrizen mit Rang $n$ folgt insbesondere,
dass $L$ eine untere Dreiecksmatrix und $U$ eine obere Dreiecksmatrix
mit lauter $1$ auf der Diagonalen ist.
Damit wird die Berechnung der
Determinanten besonders einfach:

\begin{satz}
Sei $A$ eine $n\times n$-Matrix mit Rang $n$ und $A=LU$ die LU-Zerlegung.
Dann ist
\[
\det(A)=\det(L)=l_{11}l_{22}\dots l_{nn}.
\]
\end{satz}

Dieser Satz ist natürlich nichts anderes als Satz~\ref{detprodpivot}.
Die Pivots sind ja genau die $l_{ii}$, und 
das Produkt der Pivots ist das Produkt der Elemente
$l_{11}\dots l_{nn}$.

Natürlich können wir $L$ auch noch in ein Produkt aus einer
Dreiecksmatrix und einer Diagonalmatrix zerlegen:

\begin{satz}[LDU-Zerlegung]
\index{LDU-Zerlegung}
\index{Zerlegung!LDU}
\label{ldrdecomposition}
Sei $A$ eine $m\times n$ Matrix mit Rang $r$.
Dann gibt es eine $m\times r$ Matrix $L_0$, eine $r\times n$-Matrix $R$
und eine $r\times r$-Diagonalmatrix $D$
so, dass $A=L_0DR$.
Ausserdem hat $L_0$ die Form mit folgender Form:
\[
L_0=\begin{pmatrix}
1     &0&\dots&0\\
l_{21}&1     &\dots&0\\
\vdots&\vdots&\ddots&\vdots\\
l_{r1}&l_{r2}&\dots&1     \\
\vdots&\vdots& &\vdots\\
l_{m1}&l_{m2}&\dots&l_{mr}
\end{pmatrix}
\]
und $R$ die Form wie in Satz \ref{ludecomposition}.
\end{satz}

\begin{proof}[Beweis]
Man verwendet 
\[
D=\operatorname{diag}(l_{11},\dots,l_{rr}),
\]
und berechnet $L_0$ aus dem $L$ gemäss Satz \ref{ludecomposition} durch
Multiplikation mit $D^{-1}$, also $L_0=LD^{-1}$.
Es ist klar, dass
$L_0DR=LR=A$.
\end{proof}

\begin{satz}[LR-Zerlegung]
\index{LR-Zerlegung}
\index{Zerlegung!LR}
Eine $m\times n$ Matrix mit Rang $r$ kann zerlegt werden in ein Produkt $A=LR$
werden, wobei $L$ eine $m\times r$-Matrix ist mit $l_{ii}=1$ und $l_{ik}=0$
für $k>i$, und $U$ eine $r\times n$-Matrix ist mit $u_{ik}=0$ mit $k<i$.
\end{satz}

\begin{proof}[Beweis]
Aus einer Zerlegung $A=LDU$ gemäss Satz \ref{ldrdecomposition} kann man
$R=DU$ bilden, und erhält so die Zerlegung $A=LR$.
\end{proof}

\begin{beispiel}
Man finde die LU-Zerlegung und LR-Zerlegung der Matrix 
\[
A=\begin{pmatrix}
-1&0&2\\
1&3&3\\
-1&-3&1
\end{pmatrix}
\]
Der Gauss-Algorithmus liefert
\begin{align*}
\begin{tabular}{|>{$}c<{$}>{$}c<{$}>{$}c<{$}|}
\hline
-1%
\begin{picture}(0,0)
\color{red}\put(-7,4){\circle{15}}
\end{picture}%
&0&2\\
1&3&3\\
-1%
\begin{picture}(0,0)
\color{blue}\drawline(-14,-2)(-14,24)(1,24)(1,-2)
\end{picture}%
&-3&1\\
\hline
\end{tabular}
\rightarrow
\begin{tabular}{|>{$}c<{$}>{$}c<{$}>{$}c<{$}|}
\hline
1&0&-2\\
0&3%
\begin{picture}(0,0)
\color{red}\put(-3,4){\circle{12}}
\end{picture}%
&5\\
0&-3%
\begin{picture}(0,0)
\color{blue}\drawline(-14,-2)(-14,10)(1,10)(1,-2)
\end{picture}%
&-1\\
\hline
\end{tabular}
\rightarrow
\begin{tabular}{|>{$}c<{$}>{$}c<{$}>{$}c<{$}|}
\hline
1&0&-2\\
0&1&\frac53\\
0&0&4%
\begin{picture}(0,0)
\color{red}\put(-3,4){\circle{12}}
\end{picture}%
\\
\hline
\end{tabular}
\rightarrow
\begin{tabular}{|>{$}c<{$}>{$}c<{$}>{$}c<{$}|}
\hline
1&0&-2\\
0&1&\frac53\\
0&0&1\\
\hline
\end{tabular}
\end{align*}
Daraus kann man jetzt die LU-Zerlegung ablesen:
\[
L=\begin{pmatrix}
-1%
\begin{picture}(0,0)
\color{red}\put(-7,4){\circle{15}}
\end{picture}%
& 0& 0\\
 1& 3%
\begin{picture}(0,0)
\color{red}\put(-3,4){\circle{12}}
\end{picture}%
& 0\\
-1%
\begin{picture}(0,0)
\color{blue}\drawline(-14,-2)(-14,24)(1,24)(1,-2)
\end{picture}%
&-3%
\begin{picture}(0,0)
\color{blue}\drawline(-14,-2)(-14,10)(1,10)(1,-2)
\end{picture}%
& 4%
\begin{picture}(0,0)
\color{red}\put(-3,4){\circle{12}}
\end{picture}%
\end{pmatrix}
,\qquad
U=\begin{pmatrix}
1& 0&-2\\
0& 1&\frac53\\
0& 0&1
\end{pmatrix}
\]
Für die LR-Zerlegung bildet man zunächst $L_0$
\[
L
\begin{pmatrix}
-1%
\begin{picture}(0,0)
\color{red}\put(-7,4){\circle{15}}
\end{picture}%
& 0& 0\\
 0& 3%
\begin{picture}(0,0)
\color{red}\put(-3,4){\circle{12}}
\end{picture}%
& 0\\
 0& 0& 4%
\begin{picture}(0,0)
\color{red}\put(-3,4){\circle{12}}
\end{picture}%
\end{pmatrix}^{-1}
=
\begin{pmatrix}
 1& 0&0\\
-1& 1&0\\
 1&-1&1
\end{pmatrix}
=L_0
\]
und dann
\[
R=
\begin{pmatrix}
-1%
\begin{picture}(0,0)
\color{red}\put(-7,4){\circle{15}}
\end{picture}%
& 0& 0\\
 0& 3%
\begin{picture}(0,0)
\color{red}\put(-3,4){\circle{12}}
\end{picture}%
& 0\\
 0& 0& 4%
\begin{picture}(0,0)
\color{red}\put(-3,4){\circle{12}}
\end{picture}%
\end{pmatrix}U
=
\begin{pmatrix}
-1%
\begin{picture}(0,0)
\color{red}\put(-7,4){\circle{15}}
\end{picture}%
&0&2\\
0&3%
\begin{picture}(0,0)
\color{red}\put(-3,4){\circle{12}}
\end{picture}%
&5\\
0&0&4%
\begin{picture}(0,0)
\color{red}\put(-3,4){\circle{12}}
\end{picture}%
\end{pmatrix}
\]
Kontrolle
\[
L_0R=
\begin{pmatrix}
 1& 0&0\\
-1& 1&0\\
 1&-1&1
\end{pmatrix}
\begin{pmatrix}
-1&0&2\\
0&3&5\\
0&0&4
\end{pmatrix}
=\begin{pmatrix}
-1&0&2\\
1&3&3\\
-1&-3&1
\end{pmatrix}
=A.
\]
\end{beispiel}

