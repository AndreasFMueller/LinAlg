%
% abbildungen.tex
%
% (c) 2018 Prof Dr Andreas Müller, Hochschule Rapperswil
% 
\section{Orientierungserhaltende Abbildungen}
\rhead{Orientierungserhaltende Abbildungen}
Wie im Falle des Skalarproduktes bilden die Abbildungen, die den Flächeninhalt,
das Volumen und/oder die Orientierung erhalten, eine Gruppe.

\subsection{Determinante und Fläche}
Sei $A$ eine Abbildungsmatrix in einem zweidimensionalen Raum.
Die Spalten von $A$ sind die Bilder der Standardbasisvektoren.
Die Determinante von $A$ ist der orientierte Flächeninhalt des Parallelogramms 
aufgespannt von den beiden Spaltenvektoren.
Die von $A$ beschriebene Abbildung macht also aus einem Quadrat mit
orientiertem Flächeninhalt $F$
eine Parallelogramm mit orientierten Flächeninhalt $\det(A)\cdot F$.

Der Flächeninhalt eines Gebietes in der Ebene bleibt unter der Abbildung
$A$ genau dann erhalten, wenn $\det(A)=\pm 1$ gilt.
Falls $\det(A)=-1$ ändert die Orientierung.
Eine Abbildung, die nicht nur den Flächeninhalt, sondern auch die
Orientierung erhält, erfüllt $\det(A)=1$.

Eine Abbildung mit $\det(A)=1$ muss nicht orthogonal sein, wie das Beispiel
\begin{align*}
A
&=
\begin{pmatrix}2&3\\1&2\end{pmatrix}
\qquad\Rightarrow\qquad
\det(A) = 2\cdot 2-3\cdot 1 =1
\\
A^tA
&=
\begin{pmatrix}2&1\\3&2\end{pmatrix}
\begin{pmatrix}2&3\\1&2\end{pmatrix}
=
\begin{pmatrix}
5&7\\
7&13
\end{pmatrix}
\ne
\begin{pmatrix}
1&0\\
0&1
\end{pmatrix}.
\end{align*}
Ausserdem müsste eine orthogonale Matrix Spaltenvektoren der Länge $1$ haben,
während die Spalten von $A$ ganz offensichtlich länger als $1$ sind.

\subsection{Determinante und Volumen}
Aus dem gleichen Argument wie beim Flächeninhalt folgt, dass eine
Abbildungsmatrix $A$ in einem dreidimensionalen Raum einen Körper mit
Volumen $V$ auf einen Körper mit Volumen $\det(A)\cdot V$.
Eine Abbildung $A$ erhält das Volumen, wenn $\det(A)=\pm 1$,
die Orientierung bleibt ebenfalls erhalten, wenn $\det(A)=1$.

\subsection{Spezielle lineare und orthogonale Gruppe}
Die orientientierungserhaltenden Abbildungen bilden die Teilmenge
\[
\operatorname{SL}_n(\mathbb R)
=
\{
A\in M_n(\mathbb R)
\;|\;
\det(A) = 1
\}.
\]
Für $A,B\in\operatorname{SL}_n(\mathbb R)$ lässt die
Produktformel
\[
\det(AB)=\det(A)\det(B) = 1
\]
die Schlussfolgerung zu, dass auch $AB\in\operatorname{SL}_n(\mathbb R)$.
Die Menge $\operatorname{SL}_n(\mathbb R)$ ist daher eine Gruppe, sie
heisst die {\em spezielle lineare Gruppe}.

Abbildungen $A$, die das Skalarprodukt erhalten, sind in $\operatorname{O}(n)$ und erfüllen
\[
A^tA=E
\quad\Rightarrow\quad
1
=
\det(E)
=
\det(A^tA)=\det(A^t)\det(A)=\det(A)^2
\quad\Rightarrow\quad
\det(A) = \pm 1.
\]
Die Abbildungen, die sowohl das Skalarprodukt als auch die Orientierung
erhalten, zeichnen sich also dadurch aus, dass sie sowohl in
$\operatorname{O}(n)$ als auch in $\operatorname{SL}_n(\mathbb R)$ sein, wir nennen diese
Gruppe
\[
\operatorname{SO}(n) 
=
\operatorname{O}(n) \cap \operatorname{SL}_n(\mathbb R)
=
\{
A\in M_n(\mathbb R)\;|\; A^tA=E\wedge \det(A) = 1
\}
\]
die spezielle orthogonale Gruppe.
Dies sind die Bewegungen, die den Nullpunkt unverändert lassen,
Längen und Winkel erhalten und die Orientierung nicht ändern.
Dies sind genau die Drehmatrizen.
Die Determinante ist also das in Abschnitt~\ref{subsection:orthogonale gruppe}
angekündigte Kriterium, mit welchem wir Drehmatrizen von Matrizen
unterscheiden können, die eine Spiegelungskomponente enhalten.

Die Differenz $\operatorname{O}(n) \setminus \operatorname{SO}(n)$
besteht aus Matrizen, die zwar das Skalarprodukt erhalten, aber die
Orientierung umkehren.
Ist $S_x$ die Spiegelung an der Ebene senkrecht zur $x$-Achse und ist
$A\in \operatorname{O}(n) \setminus \operatorname{SO}(n)$, dann ist
\[
\det(S_xA)= \det(S_x)\det(A) = (-1)\cdot(-1)=1
\qquad\Rightarrow\qquad
S_xA\in\operatorname{SO}(n).
\]
Dies bedeutet, dass die Multiplikation mit $S_x$ eine bijektive Abbildung
\[
\operatorname{SO}(n) \to \operatorname{O}(n)\setminus\operatorname{SO}(n)
:
A\mapsto S_xA
\]
definiert.
Die Gruppe $\operatorname{O}(n)$ besteht also aus zwei gleichartig gebauten
Teilmengen, $\operatorname{SO}(n)$  und
$\operatorname{O}(n)\setminus\operatorname{SO}(n)$.



