%
% abbildungen.tex
%
% (c) 2018 Prof Dr Andreas Müller, Hochschule Rapperswil
% 
\section{Orientierungserhaltende Abbildungen}
\rhead{Orientierungserhaltende Abbildungen}
Wie im Falle des Skalarproduktes bilden die Abbildungen, die den Flächeninhalt,
das Volumen und/oder die Orientierung erhalten, eine Gruppe.

\subsection{Determinante und Flächeninhalt}
Sei $A$ eine Abbildungsmatrix in einem zweidimensionalen Raum.
Die Spalten von $A$ sind die Bilder der Standardbasisvektoren.
Die Determinante von $A$ ist der orientierte Flächeninhalt des Parallelogramms 
aufgespannt von den beiden Spaltenvektoren.
Die von $A$ beschriebene Abbildung macht also aus einem Quadrat mit
orientiertem Flächeninhalt $F$
eine Parallelogramm mit orientierten Flächeninhalt $\det(A)\cdot F$.

Der Flächeninhalt eines Gebietes in der Ebene bleibt unter der Abbildung
$A$ genau dann erhalten, wenn $\det(A)=\pm 1$ gilt.
Falls $\det(A)=-1$ ändert die Orientierung.
Eine Abbildung, die nicht nur den Flächeninhalt, sondern auch die
Orientierung erhält, erfüllt $\det(A)=1$.

Eine Abbildung mit $\det(A)=1$ muss nicht orthogonal sein, wie das Beispiel
\begin{align*}
A
&=
\begin{pmatrix}2&3\\1&2\end{pmatrix}
\qquad\Rightarrow\qquad
\det(A) = 2\cdot 2-3\cdot 1 =1
\\
A^tA
&=
\begin{pmatrix}2&1\\3&2\end{pmatrix}
\begin{pmatrix}2&3\\1&2\end{pmatrix}
=
\begin{pmatrix}
5&7\\
7&13
\end{pmatrix}
\ne
\begin{pmatrix}
1&0\\
0&1
\end{pmatrix}
\end{align*}
zeigt.
Ausserdem müsste eine orthogonale Matrix Spaltenvektoren der Länge $1$ haben,
während die Spalten von $A$ ganz offensichtlich länger als $1$ sind.

\subsection{Determinante und Volumen}
Aus dem gleichen Argument wie beim Flächeninhalt folgt, dass eine
Abbildungsmatrix $A$ in einem dreidimensionalen Raum einen Körper mit
Volumen $V$ auf einen Körper mit Volumen $\det(A)\cdot V$.
Eine Abbildung $A$ erhält das Volumen, wenn $\det(A)=\pm 1$.
Die Orientierung bleibt ebenfalls erhalten, wenn $\det(A)=1$.

