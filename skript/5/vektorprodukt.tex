%
% vektorprodukt.tex
%
% (c) 2018 Prof Dr Andreas Müller, Hochschule Rapperswil
%
\section{Vektorprodukt}
Mit dem Skalarprodukt haben wir ein Werkzeug, Längen und Winkel zu berechnen,
es fehlt jedoch noch ein Werkzeug, Volumina einfach zu berechnen.
Orthonormierte
Vektorsysteme haben wir als sehr nützlich erkannt, aber die Bestimmung eines
Vektors, der auf zwei gegebenen Vektoren senkrecht steht, ist eher kompliziert.
Uns stand bislang entweder das
aufwendige Orthogonalisierungsverfahren zur Verfügung, oder die Lösung
eines Gleichungssystems mit dem Gauss-Algorithmus.

Dabei haben wir im Kapitel \ref{chapter-determinanten} bereits alles
bereitgestellt, was wir für die Volumenberechnung benötigen.
Wir werden auf diesem Weg eine neue Vektoroperation kennenlernen,
das Vektorprodukt.

\subsection{Flächeninhalt eines Parallelogramms}
\begin{figure}
\begin{center}
%\includegraphics{images/d-1}
\includegraphics{5/images/flaeche.pdf}
\end{center}
\caption{Addition von Flächeninhalten von Parallelogrammen
\label{image-flaeche-addition}}
\end{figure}
Zwei Vektoren $\vec u$ und $\vec v$ spannen in der Ebene ein Parallelogramm
auf.
Gesucht ist der Flächeninhalt des Parallelogramms.
Statt dafür eine
Formel abzuleiten, untersuchen wir zunächst die Eigenschaften dieses
Flächeninhaltes in der Hoffnung, dass wir bereits ein Objekt mit den
gleichen Eigenschaften kennen.

Wir bezeichnen den Flächeninhalt des von $\vec u$ und $\vec v$ aufgespannten
Parallelogramms mit $A(\vec u,\vec v)$.
Der Flächeninhalt im landläufigen
Sinne ist natürlich immer positiv, es stellt sich jedoch als
zweckmässig heraus, wenn wir den Flächeninhalt hier mit einem
Vorzeichen versehen.
Und zwar soll $A(\vec u,\vec v)$ positiv sein, wenn
sich $\vec u$ mit einer Drehung von weniger als $180^\circ$ in den Vektor
$\vec v$ drehen lässt.
Wir nennen $A(\vec u, \vec v)$ den orientierten
Flächeninhalt.

Die Funktion $A(\vec u,\vec v)$ hat folgende Eigenschaften
\begin{itemize}
\item $A$ ändert das Vorzeichen bei Vertauschung der beiden Vektoren.
\item $A$ ist linear im ersten Argument:
\begin{align*}
A(\vec u'+\vec u'',\vec v)&=A(\vec u',\vec v)+A(\vec v'',\vec u)
\\
A(\lambda \vec u,\vec v)&=\lambda A(\vec u,\vec v)
\end{align*}
\item Der Flächeninhalt eines Einheitsquadrates ist $A(\vec e_1,\vec e_2)=1$.
\end{itemize}
Die beiden Eigenschaften zusammen ergeben, dass $A$ auch linear im
zweiten Argument ist.
Im Kapitel~2 haben wir gelernt, dass es nur eine
Funktion mit diesen Eigenschaften gibt, nämlich die Determinante:
\begin{satz}
Der orientierte Flächeninhalt des von den Vektoren $\vec u$ und $\vec v$
aufgespannten Parallelogramms ist
\[
A(\vec u,\vec v)=\left|\;\begin{matrix}u_1&v_1\\u_2&v_2\end{matrix}\;\right|
=u_1v_2-u_2v_1
.
\]
\end{satz}
\begin{satz}
Der Flächeninhalt eines Dreiecks mit den Ecken $(x_1,y_1)$, $(x_2,y_2)$ und
$(x_3,y_3)$ ist
\[
F=
\frac12\left|\;
\begin{matrix}
x_1-x_3&x_2-x_3\\
y_1-y_3&y_2-y_3\\
\end{matrix}
\;\right|
\]
\end{satz}

\begin{beispiel}
Man berechne den Flächeninhalt des Dreiecks mit den Ecken
$A=(1,6)$, $B=(7,5)$ und $C=(5,3)$.

\smallskip

{\parindent 0pt Die Kantenvektoren} des Dreiecks $ABC$ sind
\[
\overrightarrow{AB}=\begin{pmatrix}6\\-1\end{pmatrix}
,\qquad
\overrightarrow{AC}=\begin{pmatrix}4\\-3\end{pmatrix}
\]
und der Flächeninhalt
\[
F=\frac12\left|\;\begin{matrix}
   6&  4\\
  -1& -3
\end{matrix}\;\right|=
-7
.
\]
Der Flächeninhalt ist also $7$.
\end{beispiel}

\subsection{Volumen eines Parallelepipeds}
\begin{figure}
\begin{center}
\includegraphics{images/d-2}
\end{center}
\caption{Addition von Volumen von Parallelepipeds\label{image-volumina}}
\end{figure}
Ganz ähnlich kann man das Volumen eines Parallelepipeds in drei Dimensionen
berechnen, welches von drei Vektoren $\vec a$, $\vec b$ und $\vec c$
aufgespannt wird.
Auch hier stellt es sich als nützlich heraus,
das Volumen mit einem Vorzeichen zu versehen:
\begin{definition}
Das orientierte Volumen
$V(\vec a,\vec b,\vec c)$
eines Parallelepipeds aufgespannt von den drei
Vektoren
$\vec a$, $\vec b$ und $\vec c$ ist positiv, wenn die drei Vektoren
eine ``rechtshändiges System'' bilden, also gegeneinander orientiert
sind wie die ersten drei Finger der rechten Hand, andernfalls ist
$V(\vec a,\vec b,\vec c)$ negativ.
\end{definition}
Dieses orientierte Volumen hat die folgenden Eigenschaften:
\begin{enumerate}
\item Vertauscht man zwei der drei Vektoren, ändert das Volumen das Vorzeichen.
\item $V(\vec a,\vec b,\vec c)$ ist linear im ersten Argument, wie man
der Abbildung~\ref{image-volumina} entnehmen kann:
\begin{align*}
V(\lambda\vec a,\vec b,\vec c)
&=
\lambda V(\vec a,\vec b,\vec c)
\\
V(\vec a'+\vec a'',\vec b,\vec c)
&=
V(\vec a',\vec b,\vec c)
+
V(\vec a'',\vec b,\vec c)
\end{align*}
\item Das orientierte Volumen des Einheitswürfels ist
$V(\vec e_1,\vec e_2,\vec e_3)=1$.
\end{enumerate}
Aus den ersten beiden Eigenschaften können wir folgern, dass das orientierte
Volumen auch in allen anderen Argumenten linear ist.
Und wie im vorangegangenen Abschnitt schliessen wir,
dass $V(\vec a,\vec b,\vec c)$ die Determinante ist:
\begin{satz}
Das orientierte Volumen $V(\vec a,\vec b,\vec c)$ eines von den Vektoren
$\vec a$, $\vec b$ und $\vec c$ aufgespannten Parallepipeds  ist
\[
V(\vec a,\vec b,\vec c)=\left|\;\begin{matrix}
a_1&b_1&c_1\\
a_2&b_2&c_2\\
a_3&b_3&c_3\\
\end{matrix}\;\right|.
\]
\end{satz}

\begin{beispiel}
Man finde alle Parallelepipeds mit folgenden Eigenschaften:
\begin{compactenum}
\item Zwei Kanten sind $\vec a=\overrightarrow{OA}$ und $\vec b=\overrightarrow{OB}$
mit $A=(4,1,3)$ und $B=(5,1,2)$.
\item Die dritte Kante ist $\overrightarrow{OC}$, wobei $C$ auf der
Ebene durch $O$ mit der Normalen
\[
\vec n=\begin{pmatrix}1\\1\\1\end{pmatrix}
\]
liegt.
\item Zwei Kanten stehen senkrecht aufeinander.
\item Das Volumen ist $8$.
\end{compactenum}

\smallskip

{\parindent 0pt Wir suchen einen Vektor}
\[
\vec c=\begin{pmatrix}x\\y\\z\end{pmatrix},
\]
der die Bedingungen der Aufgabe erfüllt.
Zunächst muss $\vec c$ auf $\vec n$ senkrecht stehen,
also $\vec c\cdot\vec n=0$ oder
\[
x+y+z=0.
\]
Da $\vec a\cdot\vec b=20+1+6=27\ne 0$ stehen $\vec a$ und $\vec b$
nicht senkrecht, es ist also der gesuchte Vektor $\vec c$ der auf den bereits
bekannten Kanten senkrecht stehen muss.
Wir versuchen es zunächst
mit $\vec a$, weitere Lösungen ergeben sich, wenn man stattdessen $\vec b$
verwendet.
Dies liefert die Bedingung $\vec a\cdot\vec c=0$ oder
\[
4x+y+3z=0
\]
Das Volumen kann mit der Determinante berechnet werden:
\[
V=\left|\;
\begin{matrix}
4&5&x\\
1&1&y\\
3&2&z
\end{matrix}
\;\right|=
4z+15y+2x-3x-8y-5z=-x+7y-z=\pm8.
\]
Jetzt kann der Vektor $\vec c$ mit dem Gauss-Algorithmus bestimmt werden
\begin{align*}
\begin{tabular}{|>{$}c<{$}>{$}c<{$}>{$}c<{$}|>{$}c<{$}|}
\hline
1%
\begin{picture}(0,0)
\color{red}\put(-3,4){\circle{12}}
\end{picture}%
&1&1&0\\
-1&7&-1&\pm8\\
4%
\begin{picture}(0,0)%
\color{blue}\drawline(-12,-2)(-12,24)(4,24)(4,-2)
\end{picture}%
&1&3&0\\
\hline
\end{tabular}
&
\rightarrow
\begin{tabular}{|>{$}c<{$}>{$}c<{$}>{$}c<{$}|>{$}c<{$}|}
\hline
1&1&1&0\\
0&8%
\begin{picture}(0,0)
\color{red}\put(-3,4){\circle{12}}
\end{picture}%
&0&\pm8\\
0&-3%
\begin{picture}(0,0)%
\color{blue}\drawline(-15,-2)(-15,10)(1,10)(1,-2)
\end{picture}%
&-1&0\\
\hline
\end{tabular}
\rightarrow
\begin{tabular}{|>{$}c<{$}>{$}c<{$}>{$}c<{$}|>{$}c<{$}|}
\hline
1&1&1&0\\
0&1&0%
\begin{picture}(0,0)%
\color{blue}\drawline(-8,24)(-8,-2)(2,-2)(2,24)
\end{picture}%
&\pm1\\
0&0&-1%
\begin{picture}(0,0)
\color{red}\put(-7,4){\circle{15}}
\end{picture}%
&\pm3\\
\hline
\end{tabular}
\\
&
\rightarrow
\begin{tabular}{|>{$}c<{$}>{$}c<{$}>{$}c<{$}|>{$}c<{$}|}
\hline
1&1%
\begin{picture}(0,0)%
\color{blue}\drawline(-8,10)(-8,-2)(2,-2)(2,10)
\end{picture}%
&0&\pm3\\
0&1&0&\pm1\\
0&0&1&\mp3\\
\hline
\end{tabular}
\rightarrow
\begin{tabular}{|>{$}c<{$}>{$}c<{$}>{$}c<{$}|>{$}c<{$}|}
\hline
1&0&0&\pm 2\\
0&1&0&\pm1\\
0&0&1&\mp3\\
\hline
\end{tabular}
\end{align*}
Es folgt $C=(\pm 2,\pm1,\mp3)$.
Zwei weitere Lösungen findet
man auf die gleiche Weise, indem man statt der Bedingung $\vec a\cdot\vec c=0$
verlangt, dass $\vec b\cdot\vec c=0$, man findet dann
\begin{align*}
\begin{tabular}{|>{$}c<{$}>{$}c<{$}>{$}c<{$}|>{$}c<{$}|}
\hline
1%
\begin{picture}(0,0)
\color{red}\put(-3,4){\circle{12}}
\end{picture}%
&1&1&0\\
-1&7&-1&\pm8\\
5%
\begin{picture}(0,0)%
\color{blue}\drawline(-12,-2)(-12,24)(4,24)(4,-2)
\end{picture}%
&1&2&0\\
\hline
\end{tabular}
&
\rightarrow
\begin{tabular}{|>{$}c<{$}>{$}c<{$}>{$}c<{$}|>{$}c<{$}|}
\hline
1&1&1&0\\
0&8%
\begin{picture}(0,0)
\color{red}\put(-3,4){\circle{12}}
\end{picture}%
&0&\pm8\\
0&-4%
\begin{picture}(0,0)%
\color{blue}\drawline(-15,-2)(-15,10)(1,10)(1,-2)
\end{picture}%
&-3&0\\
\hline
\end{tabular}
\rightarrow
\begin{tabular}{|>{$}c<{$}>{$}c<{$}>{$}c<{$}|>{$}c<{$}|}
\hline
1&1&1&0\\
0&1&0%
\begin{picture}(0,0)
\color{blue}\drawline(-8,24)(-8,-2)(2,-2)(2,24)
\end{picture}%
&\pm1\\
0&0&-3%
\begin{picture}(0,0)
\color{red}\put(-7,4){\circle{15}}
\end{picture}%
&\pm4\\
\hline
\end{tabular}
\\
&
\rightarrow
\begin{tabular}{|>{$}c<{$}>{$}c<{$}>{$}c<{$}|>{$}c<{$}|}
\hline
1&1%
\begin{picture}(0,0)
\color{blue}\drawline(-8,10)(-8,-2)(2,-2)(2,10)
\end{picture}%
&0&\pm\frac43\\
0&1&0&\pm1\\
0&0&1&\mp\frac43\\
\hline
\end{tabular}
\rightarrow
\begin{tabular}{|>{$}c<{$}>{$}c<{$}>{$}c<{$}|>{$}c<{$}|}
\hline
1&0&0&\pm\frac13\\
0&1&0&\pm1\\
0&0&1&\mp\frac43\\
\hline
\end{tabular}
\end{align*}
also
$C=(\pm\frac13,\pm1,\mp\frac43)$.
\end{beispiel}

\subsection{Vektorprodukt}
Wir schreiben das Spatvolumen nach der Sarrusschen Regel aus:
\begin{align*}
V(\vec a,\vec b,\vec c)&=\det(\vec a,\vec b,\vec c)\\
&=
a_1b_2c_3+b_1c_2a_3+c_1a_2b_3
-a_3b_2c_1-b_3c_2a_1-c_3a_2b_1\\
&=
(a_2b_3-a_3b_2)c_1+(a_3b_1-a_1b_3)c_2+(a_1b_2-a_2b_1)c_3
\\
&=
\begin{pmatrix}
a_2b_3-a_3b_2\\
a_3b_1-a_1b_3\\
a_1b_2-a_2b_1
\end{pmatrix}
\cdot
\begin{pmatrix}
c_1\\c_2\\c_3
\end{pmatrix}
\end{align*}
Es gibt also einen Vektor $\vec a\times\vec b$, der die Berechnung
des Spatvolumens mit einem Skalarprodukt erlaubt:
\begin{definition}
Der Vektor
\[
\vec a\times\vec b= \begin{pmatrix}
a_2b_3-a_3b_2\\
a_3b_1-a_1b_3\\
a_1b_2-a_2b_1
\end{pmatrix}
\]
heisst das Vektorprodukt von $\vec a$ und $\vec b$.
Die Komponenten $p_i$
des Vektorproduktes $\vec p=\vec a\times \vec b$ sind
\[
p_1
=
\left|\;\begin{matrix}
a_2&b_2\\a_3&b_3
\end{matrix}\;\right|\\
,\qquad p_2=
\left|\;\begin{matrix}
a_3&b_3\\a_1&b_1
\end{matrix}\;\right|\\
,\qquad p_3=
\left|\;\begin{matrix}
a_1&b_2\\a_2&b_1
\end{matrix}\;\right|.
\]
\end{definition}
Das Vektorprodukt hat die folgenden Eigenschaften.
\begin{satz}
Sind $\vec a$ und $\vec b$ zwei dreidimensionale Vektoren, dann gilt
\begin{itemize}
\item $\vec a\times\vec b$ steht senkrecht auf $\vec a$ und $\vec b$.
\item $|\vec a\times\vec b|$ ist der Flächeninhalt des von
$\vec a$ und $\vec b$ aufgespannten Parallelogrammes im Raum
\item Ist $\alpha$ der Winkel zwischen $\vec a$ und $\vec b$, dann
ist
\[
|\vec a\times\vec b|=|\vec a|\,|\vec b|\sin \alpha.
\]
\end{itemize}
\end{satz}
\begin{proof}[Beweis]
Sei $\vec c$ ein Vektor in der von $\vec a$ und $\vec b$ aufgespannten
Ebene.
In diesem Fall degeneriert das Parallelepiped, es hat Volumen $0$,
also $V(\vec a,\vec b,\vec c)= 0$.
Da also
\[
V(\vec a,\vec b,\vec c)=(\vec a\times \vec b)\cdot \vec c=0,
\]
steht $\vec c$ senkrecht auf $\vec a\times\vec b$.
Da dies für jeden
Vektor $\vec c$ in der von $\vec a$ und $\vec b$ aufgespannten Ebene
gilt, steht $\vec a\times\vec b$ senkrecht auf dieser Ebene, und insbesondere
auf $\vec a$ und $\vec b$.
Dies beweist die Aussage a).

Sei $\vec n$ der Normalenvektor mit Länge $1$ auf der von $\vec a$ und
$\vec b$ aufgespannten Ebene.
Das Volumen $V(\vec a,\vec b,\vec n)$ ist
dasjenige eines geraden Prismas mit dem von $\vec a$ und $\vec b$
aufgespannten Prisma mit Höhe $1$, also gleich gross wie der
Flächeninhalt des von $\vec a$ und $\vec b$ aufgespannten Parallelogramms.
Andererseits ist
\[
V(\vec a,\vec b,\vec n)=(\vec a\times\vec b)\cdot \vec n
\]
die Projektion des Vektors $\vec a\times\vec b$ auf $\vec n$.
Die beiden
Vektoren haben aber die gleiche Richtung, weil sie beide senkrecht stehen
auf der von $\vec a$ und $\vec b$ aufgespannten Ebene.
Also ist die Projektion
gerade die Länge des Vektors, also ist
$|\vec a\times\vec b|$ der Flächeninhalt des Parallelogramms.

Die Höhe des von $\vec a$ und $\vec b$ aufgespannten Parallelogramms
ist $|\vec b|\sin \alpha$, der Flächeninhalt also
$|\vec a\times\vec b|=|\vec a|\,|\vec b|\sin\alpha.$
\end{proof}

\begin{beispiel}
Man bestimme das Vektorprodukt von
\[
\vec a=\begin{pmatrix}1\\2\\3\end{pmatrix}
\quad\text{und}\quad
\vec b=\begin{pmatrix}8\\5\\13\end{pmatrix}.
\]

\smallskip
{\parindent 0pt Wir verwenden direkt die Definition}
\[
\vec a\times \vec b=
\begin{pmatrix}1\\2\\3\end{pmatrix}
\times
\begin{pmatrix}8\\5\\13\end{pmatrix}
=
\begin{pmatrix}
2\cdot 13-3\cdot 5\\
3\cdot 8-1\cdot 13\\
1\cdot 5-2\cdot 8
\end{pmatrix}
=
\begin{pmatrix}
11\\
11\\
-11
\end{pmatrix}.
\]
\end{beispiel}

\subsection{Normale}
Das Vektorprodukt erlaubt uns jetzt auf einfache Weise die Normale einer
Ebene zu finden.
Sei
\[
\vec r=\vec p+t\vec u+s\vec v
\]
die Parameterdarstellung einer Ebene, dann ist
\[
\vec n=\vec u\times\vec v
\]
eine Normale, also ist
\[
(\vec r-\vec p)\cdot (\vec u\times\vec v)=0
\]
die implizite Form der Ebenengleichung.

\begin{beispiel}
Man finde die Normale der Ebene mit der Gleichung (\ref{beispielebene}).

\smallskip
{\parindent 0pt Die Normale haben wir schon einmal mit Hilfe der Gleichungen
(\ref{gleichungen-fuer-normale}) berechnet, jetzt kann dies
mit Hilfe des vereinfacht werden:}
\begin{equation}
\vec n = \vec u\times \vec v=
\begin{pmatrix}2\\2\\-2\end{pmatrix}
\times
\begin{pmatrix}3\\-3\\-1\end{pmatrix}
=
\begin{pmatrix}
2\cdot(-1)-(-2)\cdot(-3)\\
(-2)\cdot 3-2\cdot (-1)\\
2\cdot(-3)-2\cdot 3
\end{pmatrix}
=
\begin{pmatrix}
-8\\
-4\\
-12
\end{pmatrix}
=-4\begin{pmatrix}2\\1\\3\end{pmatrix}.
\label{beispielvektorprodukt}
\end{equation}
also ein Vielfaches des mit den Gleichungen (\ref{gleichungen-fuer-normale})
gefundenen Normale.
Damit wird die Ebenengleichung
\begin{align*}
0=
\begin{pmatrix} -8\\ -4\\ -12 \end{pmatrix}\cdot
\left(
\begin{pmatrix}x\\y\\z\end{pmatrix}
-
\begin{pmatrix}1\\2\\1 \end{pmatrix}
\right)
&=
-8x-4y-12z+
28
\\
\Rightarrow
2x+y+3z&=7.
\end{align*}
\end{beispiel}

\subsection{Weitere Anwendungen}

\subsubsection{Zwischenwinkel}
Für den Zwischenwinkel zweier Vektoren gilt
\[
\sin\alpha=\frac{|\vec a\times\vec b|}{|\vec a|\,|\vec b|}.
\]
\begin{beispiel}
Berechne den Zwischenwinkel der Richtungsvektoren der Ebenengleichung
\ref{beispielebene}.

\smallskip

{\parindent 0pt Das Vektorprodukt der beiden Vektoren wurde
in (\ref{beispielvektorprodukt}) schon
berechnet.} Die Länge der Vektoren ist
\[
|\vec u|=\sqrt{4+4+4}=2\sqrt{3}
,
\quad
|\vec v|=\sqrt{9+9+1}=\sqrt{19}.
\]
Die Zwischenwinkelformel liefert jetzt
\begin{align*}
\sin\alpha&=\frac{|\vec u\times \vec v|}{|\vec u|\cdot |\vec v|}
=\frac{\sqrt{64+16+144}}{\sqrt{12\cdot 19}}
=\frac{\sqrt{224}}{\sqrt{228}}=0.98246\\
\alpha&=79.252^\circ.
\end{align*}
\end{beispiel}

\subsubsection{Abstand Punkt--Gerade}
\begin{figure}
\begin{center}
\includegraphics{images/d-3}
\end{center}
\caption{Abstand Punkt--Gerade mit dem Vektorprodukt\label{punkt-gerade}}
\end{figure}
Es ist der Abstand eines Punktes mit Ortsvektor $\vec a$ von der
Geraden durch den Punkt mit Ortsvektor $\vec p$ und Richtungsvektor $\vec r$,
also mit der Parameterdarstellung
\[
\vec q=\vec p+t\vec r
\]
zu bestimmen.
Der gesuchte Abstand $d$ ist die Höhe des Parallelogramms,
welches von $\vec a-\vec p$
und $\vec r$ aufgespannt wird, wobei $\vec r$ als Grundseite zu betrachten ist.
Der Flächeninhalt $A$ des Parallelogramms kann mit dem Vektorprodukt
berechnet werden:
\begin{align*}
A&=|(\vec a-\vec p)\times\vec r|\\
d&=\frac{|(\vec a-\vec p)\times\vec r|}{|\vec r|}.
\end{align*}
\subsubsection{Abstand zweier windschiefer Geraden}
\begin{figure}
\begin{center}
\includegraphics{images/d-4}
\end{center}
\caption{Abstand windschiefer Geraden\label{windschief}}
\end{figure}
Zwei nicht parallele Geraden $g_0$ und $g_1$ im Raum,
die sich nicht schneiden, heissen
{\em windschief}.
\index{windschief}
Sie haben
einen kürzesten Abstand $d$, der auf beiden Geraden senkrecht steht.
Sind
\begin{align*}
g_0:
\vec p&=\vec p_0+t\vec r_0\\
g_1:
\vec p&=\vec p_1+t\vec r_1
\end{align*}
Parameterdarstellungen der Geraden, dann ist die Richtung des kürzesten
Abstandes die Richtung des Vektorproduktes $\vec n = \vec r_0\times\vec r_1$.
Ein Vektor zwischen zwei beliebigen Punkten auf den beiden Geraden,
zum Beispiel zwischen $P_0$ und $P_1$, also der Vektor $\vec p_1-\vec p_0$,
wird als Projektion auf die Richtung des kürzesten Abstandes immer die
Länge dieses kürzesten Abstandes haben.
Die Projektion kann mit dem
Skalarprodukt berechnet werden, der kürzeste Abstand $d$ ist
\[
d=(\vec p_0-\vec p_1)\cdot\frac{\vec r_0\times\vec r_1}{|\vec r_0\times\vec r_1|}.
\]

