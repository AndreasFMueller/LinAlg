%
% basis.tex
%
% (c) 2018 Prof Dr Andreas Müller, Hochschule Rappers2il
%
\section{Basis}
\rhead{Basis}
\index{Basis}
In den bisherigen Beispielen sind Vektorräume entweder als Nullräume oder
als Bildräume einer Matrix entstanden.
Der Test, ob ein Vektor in einem
Vektorraum drin ist, muss entsprechend auf verschiedene Art erfolgen:
\begin{enumerate}
\item Ist $V=\operatorname{ker}A$, berechnet man einfach $Av$, falls $Av=0$
kann man schliessen, dass $v\in V$.
\item Ist $V=\operatorname{im}A$, dann muss man einen Vektor $u\in\mathbb R^n$
finden mit $v=Au$, man muss also ein lineares Gleichungssystem lösen.
\end{enumerate}
Der zweite Fall ist also deutlich aufwendiger.
Schreibt man die Bedingung
$v=Au$ als Gleichung von Spaltenvektoren von $A$, bekommt man
\[
\begin{pmatrix}v_1\\\vdots\\v_m\end{pmatrix}
=
\begin{pmatrix}a_{11}\\\vdots\\a_{m1}\end{pmatrix}u_1+\dots+
\begin{pmatrix}a_{1n}\\\vdots\\a_{mn}\end{pmatrix}u_n.
\]
Man muss also entscheiden, ob sich $v$ durch die Spaltenvektoren von
$A$ ausdrücken lässt.
Selbst wenn $\operatorname{im}A$ nur ein
relativ kleiner Raum ist, zum Beispiel weil alle Spalten von $A$
Vielfache eines
einzigen Vektors sind, gibt dies ziemlich viel Arbeit.
Wir möchten
daher einen Vektorraum als Bildraum einer möglichst kleinen Matrix
schreiben können.
Gleichbedeutend damit ist, dass wir die Vektoren
des Raumes aus einer möglichst kleinen Zahl von Vektoren linear
kombinieren können möchten.

Diese Art von Darstellung beliebiger Vektoren mit Hilfe einer
ausgewählten Familie von Vektoren liegt dem Koordinaten-Systemen
zu Grunde.
Um die Koordinaten eines Punktes zu finden, zerlegt
man seinen Ortsvektor in die Vektoren entlang der Koordinatenachsen:
\[
\begin{pmatrix}2\\3\\5\end{pmatrix}
=
2\begin{pmatrix}1\\0\\0\end{pmatrix}+
3\begin{pmatrix}0\\1\\0\end{pmatrix}+
5\begin{pmatrix}0\\0\\1\end{pmatrix}
\]
Es müssen aber zwei Dinge sichergestellt werden:
\begin{enumerate}
\item Jeder Vektor muss in dieser Form darstellbar sein.
Für die
Standardbasisvektoren ist dies offensichtlich, aber wie findet man
dies bei einer beliebigen Menge von Vektoren heraus?
\item Die Darstellung muss eindeutig sein, sonst hat man offensichtlich
mehrere Beschreibungen des gleichen Punktes.
Wie kann man entscheiden,
ob ein Vektor nur auf eine Art aus den Vektoren kombiniert werden kann?
\end{enumerate}
Die nächsten zwei Abschnitte adressieren diese Fragen.

\subsection{Aufgespannter Raum}
\index{aufspannter Raum}
Wir müssen also zunächst klären, was es heisst, dass wir ``genügend''
Vektoren haben, um den Unterraum $V$ zu bilden.

\begin{definition}
\index{erzeugen}
Eine Menge $B=\{b_1,b_2,\dots,b_n\}$ von Vektoren erzeugt den
linearen Raum
\[
\langle B\rangle =
\langle b_1,\dots , b_n\rangle =
\{\lambda_1b_1+\dots+\lambda_nb_n\,|\,\lambda_i\in\mathbb R\}.
\]
Man sagt auch, $B$ spannt den Raum $\langle B\rangle$ auf.
\end{definition}

\begin{beispiel}[Beispiel: Standardbasisvektoren]
\index{Standardbasisvektoren}
Die speziellen Vektoren
\[
e_1=\begin{pmatrix}1\\0\\\vdots\\0\end{pmatrix},\quad
e_2=\begin{pmatrix}0\\1\\\vdots\\0\end{pmatrix},
\dots,
e_n=\begin{pmatrix}0\\0\\\vdots\\1\end{pmatrix}
\]
erzeugen den ganzen Raum $\mathbb R^n$, denn jeder Vektor $v\in\mathbb R^n$
lässt sich schreiben als
\[
v=\begin{pmatrix}v_1\\v_2\\\vdots\\v_n\end{pmatrix}
=
v_1\begin{pmatrix}1\\0\\\vdots\\0\end{pmatrix}+
v_2\begin{pmatrix}0\\1\\\vdots\\0\end{pmatrix}+
\dots+
v_n\begin{pmatrix}0\\0\\\vdots\\1\end{pmatrix}.
\]
Somit ist $\mathbb R^n=\langle e_1,e_2,\dots,e_n\rangle$.
die Vektoren $e_i$ heissen die Standardbasisvektoren.
\end{beispiel}

\begin{beispiel}
Gegeben seien die Vektoren
\[
a_1=\begin{pmatrix}1\\4\\7\end{pmatrix},\quad
a_2=\begin{pmatrix}2\\5\\8\end{pmatrix},\quad
a_3=\begin{pmatrix}3\\6\\9\end{pmatrix}
\]
Spannen diese drei Vektoren den ganzen Raum $\mathbb R^3$ auf,
oder anders ausgedrückt, kann man damit jeden Vektor in $\mathbb R^3$
linear kombinieren?

Dies wäre möglich, wenn die offensichtlich ganz $\mathbb R^3$
aufspannenden Vektoren $e_1,e_2,e_3$ durch $a_i$ ausgedrückt werden
können.
Also testen wir, ob $e_i\in \operatorname{im}A$, wobei
\[
A=\begin{pmatrix}
1&2&3\\
4&5&6\\
7&8&9
\end{pmatrix}
\]
ist.
Der Gauss-Algorithmus gibt folgende Tableaus:
\begin{align*}
\begin{tabular}{|>{$}c<{$}>{$}c<{$}>{$}c<{$}|>{$}c<{$}>{$}c<{$}>{$}c<{$}|}
\hline
1%
\begin{picture}(0,0)
\color{red}\put(-3,4){\circle{12}}
\end{picture}%
&2&3&1&0&0\\
4&5&6&0&1&0\\
7%
\begin{picture}(0,0)
\color{blue}\drawline(-8,-2)(-8,25)(1,25)(1,-2)
\end{picture}%
&8&9&0&0&1\\
\hline
\end{tabular}
&\rightarrow
\begin{tabular}{|>{$}c<{$}>{$}c<{$}>{$}c<{$}|>{$}c<{$}>{$}c<{$}>{$}c<{$}|}
\hline
1&2&3&1&0&0\\
0&-3%
\begin{picture}(0,0)
\color{red}\put(-7,4){\circle{15}}
\end{picture}%
&-6&-4&1&0\\
0&-6%
\begin{picture}(0,0)
\color{blue}\drawline(-15,-2)(-15,10)(1,10)(1,-2)
\end{picture}%
&-12&-7&0&1\\
\hline
\end{tabular}
\\
&\rightarrow
\begin{tabular}{|>{$}c<{$}>{$}c<{$}>{$}c<{$}|>{$}c<{$}>{$}c<{$}>{$}c<{$}|}
\hline
1&2&3&1&0&0\\
0&1&2&\frac43&-\frac13&0\\
0&0&0&1&-2&1\\
\hline
\end{tabular}
\end{align*}
Man kann also keinen einzigen der Vektoren $e_i$ mit den Vektoren $a_i$
darstellen, die Vektoren $a_i$ können also unmöglich den ganzen
$\mathbb R^3$ aufspannen.
\end{beispiel}

\subsection{Basis}
\index{Basis}
Eine Menge von Vektoren, mit denen man alle Vektoren eines Unterraums
linear kombinieren kann, muss linear unabhängig sein.
Sonst könnte
man nämlich einen der Vektoren durch die anderen ausdrücken, damit
wird er selbst unnötig.

\begin{definition}
\index{Basis}
Ein Menge $B\subset V$ von Vektoren in einem linearen Raum $V$ heisst
Basis, wenn gilt:
\begin{enumerate}
\item $B$ spannt $V$ auf: $V=\langle B\rangle$.
\item Die Vektoren in $B$ sind linear unabhängig.
\end{enumerate}
\end{definition}

\begin{beispiel}
Der Raum
\[
\left\{\left.\begin{pmatrix}x\\y\end{pmatrix}\,\right|\, x=y\right\}
\]
wird aufgespannt von 
\[
B=\left\{\begin{pmatrix}1\\1\end{pmatrix}\right\}.
\]
\end{beispiel}

Wie kann man testen, ob eine Menge von Vektoren eine Basis ist?
Schreibt man die Vektoren der Basis $B$ als Spalten in eine Matrix
$\tilde B$, müssen wir offenbar zwei Dinge testen:
\begin{enumerate}
\item $V = \operatorname{im}\tilde B$: das Gleichungssystem $\tilde Bx=b$
muss für jeden Vektor $b\in V$ lösbar sein.
\item Die Vektoren von $B$ sind linear unabhängig: die Spalten sind
linear unabhängig.
Dies kann auf zwei Arten geschehen:
\begin{enumerate}
\item Der Gauss-Algorithmus, angewendet auf $\tilde B^t$, liefert eine
Nullzeile genau dann, wenn die Zeilen von $\tilde B^t$ linear abhängig sind,
das sind aber genau die Spalten von $\tilde B$.
Wenn also keine Nullzeile
auftritt, sind die Vektoren linear unabhängig.
\item Falls der Gauss-Algorithmus für das homogene Gleichungssystem
$\tilde B x=0$ nur die Lösung $x=0$ findet, sind die Spalten linear
unabhängig.
Anders ausgedrückt: wenn das Gleichungssystem $\tilde Bx=b$
für jedes in Frage kommende $b$ genau eine Lösung hat, sind die
Spalten von $\tilde B$ linear unabhängig.
\end{enumerate}
\end{enumerate}


\begin{satz}
\index{Standardbasis}
Die Standardbasisvektoren bilden eine Basis von $\mathbb R^n$,
die Standardbasis.
\end{satz}

\begin{proof}[Beweis]
Wir wissen bereits, dass die Standardbasisvektoren $\mathbb R^n$ aufspannen,
wir müssen aber nur noch erkennen, dass sie auch linear unabhängig sind.
Schreibt man die Vektoren als Spalten in eine Matrix $\tilde B$, entsteht die
Einheitsmatrix, und ein Gleichungssystem $Ex=b$ hat immer genau die eine
Lösung $x=b$, also ist 
\end{proof}

\begin{beispiel}
Die Vektoren 
\[
b_1=\begin{pmatrix}1\\0\\0\end{pmatrix},\quad
b_2=\begin{pmatrix}1\\1\\0\end{pmatrix},\quad
b_3=\begin{pmatrix}1\\1\\1\end{pmatrix},\quad
\]
bilden eine Basis von $\mathbb R^3$.
Zwei Dinge sind zu prüfen:
spannen sie den ganzen Raum auf und sind sie linear unabhängig.
Da man alle Standardbasisvektoren durch die $b_i$ ausdrücken kann, nämlich
durch
\[
e_1=b_1,\qquad e_2=b_2-b_1,\qquad\text{und}\qquad e_3=b_3-b_2-b_1,
\]
ist jeder Vektor durch die $b_i$ ausdrückbar.
Lineare Abhängigkeit kann
man mit dem Gauss-Algorithmus testen.
Wir schreiben dazu die Vektoren
als Zeilen in ein Gauss-Tableau:
\[
\begin{tabular}{|>{$}c<{$}>{$}c<{$}>{$}c<{$}|}
\hline
1&0&0\\
1&1&0\\
1&1&1\\
\hline
\end{tabular}
\rightarrow
\begin{tabular}{|>{$}c<{$}>{$}c<{$}>{$}c<{$}|}
\hline
1&0&0\\
0&1&0\\
0&0&1\\
\hline
\end{tabular}
\]
Da keine Nullzeile entstanden ist, sind die Zeilen linear unabhängig.
Die Zeilen waren aber genau die Vektoren $b_i$.
Damit ist klar, dass
die Vektoren $b_i$ eine Basis bilden.
\end{beispiel}

\begin{definition}
\index{Dimension}
Die Dimension $\dim V$ eines Vektorraumes $V$ ist die Zahl der
Basisvektoren einer Basis von $V$.
\end{definition}

\begin{beispiel} Man finde eine Basis des Nullraumes der Matrix
\[
A=\begin{pmatrix}
1&2&3\\
4&5&6\\
7&8&9
\end{pmatrix}.
\]
Wendet man den Gauss-Algorithmus für das homogene Gleichungssystem
an, findet man:
\begin{align*}
\begin{tabular}{|>{$}c<{$}>{$}c<{$}>{$}c<{$}|>{$}c<{$}|}
\hline
 1%
\begin{picture}(0,0)
\color{red}\put(-3,4){\circle{12}}
\end{picture}%
& 2& 3&0\\
 4& 5& 6&0\\
 7%
\begin{picture}(0,0)
\color{blue}\drawline(-8,-2)(-8,24)(1,24)(1,-2)
\end{picture}%
& 8& 9&0\\
\hline
\end{tabular}
&\rightarrow
\begin{tabular}{|>{$}c<{$}>{$}c<{$}>{$}c<{$}|>{$}c<{$}|}
\hline
 1& 2&  3&0\\
 0&-3%
\begin{picture}(0,0)
\color{red}\put(-6,4){\circle{15}}
\end{picture}%
& -6&0\\
 0&-6%
\begin{picture}(0,0)
\color{blue}\drawline(-15,-2)(-15,10)(2,10)(2,-2)
\end{picture}%
&-12&0\\
\hline
\end{tabular}
\\
&\rightarrow
\begin{tabular}{|>{$}c<{$}>{$}c<{$}>{$}c<{$}|>{$}c<{$}|}
\hline
 1& 2%
\begin{picture}(0,0)
\color{blue}\drawline(-8,10)(-8,-2)(2,-2)(2,10)
\end{picture}%
&  3&0\\
 0& 1&  2&0\\
 0& 0&  0&0\\
\hline
\end{tabular}
\\
&\rightarrow
\begin{tabular}{|>{$}c<{$}>{$}c<{$}>{$}c<{$}|>{$}c<{$}|}
\hline
 1& 0& -1&0\\
 0& 1&  2&0\\
 0& 0&  0&0\\
\hline
\end{tabular}
\end{align*}
Offenbar gibt es genau eine frei wählbare Variable $z$, und
die Lösungsmenge ist 
\[
t\begin{pmatrix}
1\\-2\\1
\end{pmatrix}
\]
Der Nullraum wird also vom Vektor
\[
\begin{pmatrix}
1\\-2\\1
\end{pmatrix}
\]
aufgespannt.
\end{beispiel}

\begin{beispiel}
Man finde eine Basis des Bildraumes $\operatorname{im}A$ mit der gleichen
Matrix wie im vorangegangenen Beispiel.

Die Basis besteht aus so wenigen linear unabhängigen Vektoren wie
möglich, aber alle Spaltenvektoren von $A$ müssen damit erzeugt
werden können.
Dazu nehmen wir einfach einen Vektor um den anderen
hinzu, solange die Menge der Vektoren linear unabhängig bleibt.

Der erste Spaltenvektor ist nicht der Nullvektor, also können wir
den in die Basis hinein nehmen.
Der zweite ist nicht proportional,
also sind die ersten beiden Spalten linear unabhängig.
Als
Basis könnten wir daher
\[
\left\{
\begin{pmatrix}1\\4\\7\end{pmatrix}
,
\begin{pmatrix}2\\5\\8\end{pmatrix}
\right\}
\]
nehmen.
Den dritten Spaltenvektor dürfen wir nicht hinzunehmen.
Aus dem letzten Beispiel wissen wir ja, dass die Matrix singulär
ist.
Es kann also höchstens zwei linear unabhängige Vektoren haben.
\end{beispiel}

\subsection{Koordinaten}
\index{Koordinaten}
Gibt man eine Basis $B=\{b_1,\dots,b_k\}$ von $V$ vor,
dann kann man die Vektoren
in $V$ in der Basis $B$ ausdrücken.
Dazu muss man zu einem 
Vektor $v\in V$ Zahlen $\xi_i$ finden mit
\[
v=\xi_1 b_1+\dots +\xi_k b_k.
\]
Schreiben wir wieder die Basisvektoren von $B$ als Spaltenvektoren in 
eine Matrix $\tilde B$, dann ist 
\[
v=\tilde B\begin{pmatrix}\xi_1\\\vdots\\\xi_n\end{pmatrix}.
\]
Eine Basis eines $k$-dimensionalen Raumes $V$ ermöglicht also,
die Vektoren von $V$ mit Hilfe von $k$-Tupeln, bestehend
aus den Zahlen $\xi_i$, darzustellen.
Die $\xi_i$ heissen
Koordinaten eines Vektors $v$ in der Basis $B$, wir schreiben
oft auch einfach $\xi$ für den Vektor mit Komponenten $\xi_i$.
Das Finden der Koordinaten eines Vektors $v$ läuft immer auf die Lösung
des Gleichungssystems $\tilde B\xi=v$ hinaus.

\begin{beispiel}
Die Basis \[
B=\left\{
\begin{pmatrix}1\\2\\3\end{pmatrix},
\begin{pmatrix}3\\2\\1\end{pmatrix}
\right\}
\]
spannt einen zweidimensionalen Unterraum von $\mathbb R^3$ auf.
Man finde
die Koordinaten des Vektors 
\[
v=
\begin{pmatrix}
0\\4\\8
\end{pmatrix}
\]
in der Basis $B$.

Dazu muss man das Gleichungssystem $v=\tilde B\xi$ lösen:
\[
\begin{tabular}{|>{$}c<{$}>{$}c<{$}|>{$}c<{$}|}
\hline
1%
\begin{picture}(0,0)
\color{red}\put(-3,4){\circle{12}}
\end{picture}%
&3&0\\
2&2&4\\
3%
\begin{picture}(0,0)
\color{blue}\drawline(-8,-2)(-8,25)(2,25)(2,-2)
\end{picture}%
&1&8\\
\hline
\end{tabular}
\rightarrow
\begin{tabular}{|>{$}c<{$}>{$}c<{$}|>{$}c<{$}|}
\hline
1&3&0\\
0&-4%
\begin{picture}(0,0)
\color{red}\put(-7,4){\circle{15}}
\end{picture}%
&4\\
0&-8%
\begin{picture}(0,0)
\color{blue}\drawline(-15,-2)(-15,10)(1,10)(1,-2)
\end{picture}%
&8\\
\hline
\end{tabular}
\rightarrow
\begin{tabular}{|>{$}c<{$}>{$}c<{$}|>{$}c<{$}|}
\hline
1&3%
\begin{picture}(0,0)
\color{blue}\drawline(-8,10)(-8,-2)(1,-2)(1,10)
\end{picture}%
&0\\
0&1&-1\\
0&0&0\\
\hline
\end{tabular}
\rightarrow
\begin{tabular}{|>{$}c<{$}>{$}c<{$}|>{$}c<{$}|}
\hline
1&0&3\\
0&1&-1\\
0&0&0\\
\hline
\end{tabular}
\]
man kann also die Koordinaten $3$ und $-1$ ablesen.
Kontrolle:
\[
\tilde B\xi
=
\begin{pmatrix}
1&3\\
2&2\\
3&1\end{pmatrix}
\begin{pmatrix}3\\-1\end{pmatrix}
=\begin{pmatrix}
0\\4\\8
\end{pmatrix}
=v.
\]
\end{beispiel}

\subsection{Basiswechsel}
\index{Basiswechsel}
Jede beliebige linear unabhängige Teilmenge von $V$, welche ganz $V$
aufspannt, kann als Basis verwendet werden.
Oft ist es praktischer,
statt der Standardbasis eine andere Basis zu verwenden, zum Beispiel
um die Koordinatenachsen parallel zu den Kanten eines Werkstücks zu
bekommen, oder um eine spezielle Symmetrie des Problems einfacher
ausdrücken zu können.
Wie sind die Koordinaten zwischen zwei Basen $B$ und $B'$ umzurechnen?

Zu jeder Basis $B$ gibt es die Matrix $\tilde B$, die aus den
Koordinaten $\xi$ eines Vektors $v$ den Vektor mittels $v=\tilde B\xi$
berechnet.
Hat man zwei Basen $B$ und $B'$, hat auch jeder Vektor
in $V$ zwei verschiedene Koordinaten-Vektoren $\xi$ und $\xi'$:
\[
v=\tilde B\xi =\tilde B'\xi'.
\]
Wie kann man $\xi'$ aus $\xi$ berechnen?

\subsubsection{Spezialfall $\mathbb R^n$}
In diesem Fall sind die beiden Matrizen $\tilde B$ und $\tilde B'$
reguläre $n\times n$-Matrizen.
Die Gleichung
\[
v=\tilde B\xi =\tilde B'\xi'.
\]
Kann dann durch Multiplikation mit $\tilde B^{-1}$ oder $\tilde B^{\prime-1}$
von links aufgelöst werden:
\begin{align*}
\tilde B^{\prime-1}\tilde B\xi&= \tilde B^{\prime -1}\tilde B'\xi'=E\xi'=\xi'\\
\xi=\tilde B^{-1}B\tilde B\xi&=\tilde B^{-1}\tilde B'\xi'
\end{align*}
Daraus können wir die Matrix zur Berechnung von $\xi'$ aus $\xi$
ablesen: 
\[
T=\tilde B^{\prime-1}B \quad\Rightarrow\quad
\xi'=T\xi.
\]

\subsubsection{Allgemeiner Fall}
Im allgemeinen Fall eines beliebigen Unterraumes funktioniert diese
Methode nicht.
$\tilde B$ und $\tilde B'$ sind keine quadratischen
Matrizen, also können Sie auch nicht invertiert werden.
Vielmehr sind sie jetzt $n\times m$-Matrizen mit $m<n$.

Wenn $B$ und $B'$ Basen sind, dann lässt sich jeder Vektor $v\in
\operatorname{im}\tilde B=\operatorname{im}\tilde B'$ 
sowohl mit $\xi$-Koordinaten als auch mit $\xi'$-Koordinaten
beschreiben:
\[
v=\tilde B\xi=\tilde B'\xi'.
\]
Darin sind die $\xi$- und $\xi'$-Vektoren sind Vektoren in $\mathbb R^m$.

Speziell könnte man für die $\xi$-Vektoren die Standardbasisvektoren
in $R^m$ wählen.
Mit dem ersten Standardbasisvektor bekäme man
das Gleichungssystem
\begin{equation}
\tilde B\begin{pmatrix} 1\\0\\\vdots\\0\end{pmatrix}
=\tilde B'\xi'
\label{t-gleichung}
\end{equation}
Dieses Gleichungssystem kann man mit dem Gauss-Algorithmus auflösen,
und $\xi'$ finden.
Dasselbe kann man natürlich auch für den
zweiten Standardbasis-Vektor machen, und auch für alle folgenden.

Wir können alle diese Gleichungen zusammen in eine einzige
Matrixgleichung schreiben:
\[
\tilde B\begin{pmatrix}
1&0&\dots&0\\
0&1&\dots&0\\
\vdots&\vdots&\ddots&\vdots\\
0&0&\dots&1
\end{pmatrix}
=B'\begin{pmatrix}
\mathstrut t_{11}&t_{12}&\dots&t_{1m}\\
\mathstrut t_{21}&t_{22}&\dots&t_{2m}\\
\mathstrut \vdots&\vdots&\ddots&\vdots\\
\mathstrut t_{m1}&t_{m2}&\dots&t_{mm}\\
\end{pmatrix}
\]
Die Matrix $T$ auf der rechten Seite enthält als Spalten
die Lösungen der Gleichungen (\ref{t-gleichung}).

Da die Spalten von $T$ Lösungen eines linearen Gleichungssystems
mit Koeffizienten $\tilde B'$ und rechten Seiten $B$ ist, können
wir das Verfahren zur simultanen Lösung anwenden.
Das zugehörige Gauss-Tableau ist:
\[
\begin{tabular}{|>{$}c<{$}>{$}c<{$}>{$}c<{$}|>{$}c<{$}>{$}c<{$}>{$}c<{$}|}
\hline
\quad\mathstrut&         &\quad\mathstrut&\quad\mathstrut&        &\quad\mathstrut\\
      &         &      &      &        &      \mathstrut\\
      &         &      &      &        &      \mathstrut\\
      &\tilde B'&      &      &\tilde B&      \mathstrut\\
      &         &      &      &        &      \mathstrut\\
      &         &      &      &        &      \mathstrut\\
      &         &      &      &        &      \mathstrut\\
\hline
\end{tabular}
\rightarrow
\begin{tabular}{|>{$}c<{$}>{$}c<{$}>{$}c<{$}>{$}c<{$}|>{$}c<{$}>{$}c<{$}>{$}c<{$}|}
\hline
1     &     0&\dots   &0      & &                   & \\
0     &     1&\dots   &0      & &                   & \\
\vdots&\vdots&\ddots  &\vdots & &\raisebox{7pt}{$T$}& \\
0     &     0&\dots   &1      & &                   & \\
\hline
0     &     0&\dots   &0      &*&                   &*\\
0     &     0&\dots   &0      &*&                   &*\\
\hline
\end{tabular}
\]
Weil wir wissen, dass die Gleichungen immer eine Lösung haben, finden
wir an den Plätzen der Sterne unten rechts immer $0$, also steht im
Feld oben rechts genau die Transformationsmatrix $T$.

\begin{beispiel}
In der Ebene aufgespannt von den Vektoren 
\[
b_1=\begin{pmatrix}1\\1\\0 \end{pmatrix}
,\qquad
b_2=\begin{pmatrix}0\\1\\1 \end{pmatrix}
\]
möchte man die Basis aus den Basisvektoren
\[
b_1'=\begin{pmatrix}1\\2\\1\end{pmatrix}
,\qquad
b_2'=\begin{pmatrix}1\\0\\-1\end{pmatrix}
\]
Finden Sie die Koordinatentransformationsmatrix $T$, mit der
man von den Parameter $t$ $s$ in der Parameterdarstellung
mit Richtungsvektoren $b_1$ und $b_2$ auf die Koordinaten in
der Basis $B'$ umrechnen kann.
Man finde ausserdem die 
Koordinaten in der Basis $B'$ des Vektors, der in der Basis $B$
die Koordinaten $(2,-1)$ hat.

\begin{align*}
\begin{tabular}{|>{$}c<{$}>{$}c<{$}|>{$}c<{$}>{$}c<{$}|}
\hline
1& 1&1&0\\
2& 0&1&1\\
1&-1&0&1\\
\hline
\end{tabular}
&\rightarrow
\begin{tabular}{|>{$}c<{$}>{$}c<{$}|>{$}c<{$}>{$}c<{$}|}
\hline
1& 1& 1&0\\
0&-2&-1&1\\
0&-2&-1&1\\
\hline
\end{tabular}
\rightarrow
\begin{tabular}{|>{$}c<{$}>{$}c<{$}|>{$}c<{$}>{$}c<{$}|}
\hline
1& 1&      1&       0\\
0& 1&\frac12&-\frac12\\
\hline
0& 0&      0&       0\\
\hline
\end{tabular}
\\
&\rightarrow
\begin{tabular}{|>{$}c<{$}>{$}c<{$}|>{$}c<{$}>{$}c<{$}|}
\hline
1& 0& \frac12& \frac12\\
0& 1& \frac12&-\frac12\\
\hline
0& 0&      0&       0\\
\hline
\end{tabular}
\end{align*}
Die Transformationsmatrix ist also 
\[
T=
\frac12\begin{pmatrix} 1&1\\1&-1 \end{pmatrix}.
\]
Die Koordinaten $(2,-1)$ ergeben nach Umrechnung mit $T$ 
\[
\xi'=
T\begin{pmatrix}2\\-1\end{pmatrix}
=
\frac12\begin{pmatrix} 1&1\\1&-1 \end{pmatrix}
\begin{pmatrix}2\\-1\end{pmatrix}
=\frac12\begin{pmatrix}1\\3\end{pmatrix}.
\]
Zur Kontrolle berechnen wir den zugehörigen Vektor in $\mathbb R^3$:
\[
\tilde B\xi=
\begin{pmatrix}
1&0\\
1&1\\
0&1
\end{pmatrix}
\begin{pmatrix}2\\-1\end{pmatrix}
=\begin{pmatrix} 2\\1\\-1 \end{pmatrix}
,\qquad
\tilde B' \xi'
=
\begin{pmatrix}
1& 1\\
2& 0\\
1&-1
\end{pmatrix}\begin{pmatrix}\frac12\\\frac32\end{pmatrix}
=\begin{pmatrix}2\\1\\-1 \end{pmatrix},
\]
beide stimmen überein.
\end{beispiel}

