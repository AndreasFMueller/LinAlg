%
% vektorraum.tex
%
% (c) 2018 Prof Dr Andreas Müller
%

\section{Vektorräume \texorpdfstring{$\mathbb R^n$}{R hoch n} und Unterräume}
\index{Vektorraum@Vektorraum $\mathbb R^n$}
Die Menge $V$ der $n$-dimensionalen Spaltenvektoren bildet was man
einen Vektorraum nennt: Spaltenvektoren können addiert und mit
einem Skalar multipliziert werden.
Die beiden Operationen sind
miteinander verträglich, was durch die Distributivgesetze ausgedrückt
wird.
Intuitiv sagen diese, dass die ``übliche Algebra funktioniert'',
man kann mit diesen Vektoren genau so rechnen, wie man sich das von
der Algebra mit gewöhnlichen Zahlen gewohnt ist, wenigstens solange
man nicht versucht, Vektoren miteinander zu multiplizieren oder zu
dividieren.
Es gibt auch einen Nullvektor, der bei Addition nichts ändert,
also die Funktion der Null übernimmt.

Es gibt aber auch Teilmengen von $\mathbb R^n$, die vergleichbare
Eigenschaften haben.
Typischerweise können diese durch lineare Gleichungen definiert werden.
Die Menge
\[
V=\left\{\left.\begin{pmatrix}x\\y\end{pmatrix}\,\right|\,x=y\right\}
\subset\mathbb R^2
\]
besteht aus Vektoren der Form $\begin{pmatrix}x\\x\end{pmatrix}$,
es ist also ziemlich offensichtlich, dass die Summe und die skalaren
Vielfachen von  Vektoren aus $V$ wieder in $V$ liegen.
$V$ ist
bezüglich der Rechenoperationen abgeschlossen, und auch der Nullvektor
$0\in V$.

\index{Unterraum}
\begin{definition}
Eine Teilmenge $V\subset\mathbb R^n$ heisst Unterraum von $\mathbb R^n$,
wenn für zwei Vektoren
$u,v\in V$ und zwei reelle Zahlen $\lambda,\mu\in\mathbb R$
auch die daraus gebildete Linearkombination
$\lambda u+\mu v\in V$ ist.
\end{definition}

Der kleinste mögliche Unterraum ist $\{0\}$, also der Vektorraum, der nur aus
dem Nullvektor besteht.
Für diesen Raum schreibt man manchmal auch nur $0$.

\subsubsection{Beispiel: Nullraum}
\index{Nullraum}
\index{Kern|see{Nullraum}}
\index{ker|see{Nullraum}}
Sei $A$ eine $m\times n$-Matrix.
Die Menge
\[
U=\{v\in\mathbb R^n\,|\,Av=0\}
\]
ist ein Unterraum, der Nullraum oder Kern von $A$, geschrieben
$\operatorname{ker}A$.
Tatsächlich gilt für $u,v\in\operatorname{ker}A$
\[
A(\lambda v+\mu u)=\lambda Av+\mu Au=0,
\]
also ist $\lambda v+\mu u\in\operatorname{ker}A$.

\begin{satz} Sind $U$ und $V$ Unterräume, dann auch $U\cap V$ und
\[
U+V=\{u+v\,|\,u\in U,v\in V\}.
\]
\end{satz}

Sind $U$ und $V$ Unterräume von $\mathbb R^n$, dann ist auch
$U\cap V$ ein Unterraum.
Dazu ist zu prüfen, dass die Linearkombinationen
von zwei Vektoren $u,v\in U\cap V$ wieder in $U\cap V$ sind.
Da beide Vektoren aus $U$ sind, muss auch jede Linearkombination in $U$ sein.
Dasselbe gilt für $V$, also ist jede Linearkombination in $U\cap V$.

\begin{satz}
Sei $A$ eine $m\times n$-Matrix und $V$ ein Unterraum von $\mathbb R^n$.
Dann ist
\[
AV=\{ Av\,|v\in V\}\subset\mathbb R^m
\]
ein Unterraum von $\mathbb R^m$.
\end{satz}
\begin{proof}[Beweis]
Sind zwei Vektoren $u_1$ und $u_2$ in $AV$, dann gibt es $v_1,v_2\in V$ mit
$u_1=Av_1$ und $u_2=Av_2$.
Also ist
\[
\lambda_1u_1+\lambda_2u_2
=
\lambda_1Av_1+\lambda_2Av_2
=
A(\lambda_1v_1+\lambda_2v_2)\in AV,
\]
also ist $AV$ ein Unterraum.
\end{proof}
Falls $V=\mathbb R^n$ schreibt man auch $A\mathbb R^n=\operatorname{im}A$
und nennt $\operatorname{im}A$ das Bild von $A$.
\index{Bild}
\index{im@$\operatorname{im}A$|see{Bild}}

