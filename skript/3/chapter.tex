%
% chapter.tex -- Affine Vektorgeometrie
%
% (c) 2018 Prof Dr Andreas Müller, Hochschule Rapperswil
%
\chapter{Affine Vektorgeometrie\label{chapter:affin}}
Mit den in Kapitel~\ref{chapter-lingl} entwickelten Methoden zur Lösung
linearer Gleichungen und den algebraischen Konzepten von Matrizen und Vektoren
können auch geometrische Situationen effizient beschrieben werden.
Dazu benötigt man zunächst Beschreibung des geometrischen Raumes mit Hilfe von
Vektoren.
Dazu dient das Konzept der Basis und des zugehörigen Koordinatensystems,
welches in Abschnitt~\ref{section:basis} eingeführt wird.
Da der gleiche geometrische Raum mit verschiedenen Koordinatensystemen
beschrieben werden kann, müssen wir im Golgenden immer auch untersuchen,
wie sich die Beschreibung eines geometrischen Objektes ändert, wenn man
eine andere Basis verwendet.

In Abschnitt~\ref{section:lineare abbildungen} wird gezeigt, wie sich
geometrische Abbildungen als lineare Abbildungen verstehen lassen und
daher mit Hilfe von Matrizen beschreiben lassen.
Das Matrizenprodukt erhält damit eine anschauliche Bedeutung als die
Komposition von Abbildungen.

Geraden und Ebenen sowie höherdimensionale Unterräume von Vektorräumen
können alle mit Hilfe von linearen Abbildungen beschrieben.
Probleme der Lage wie zum Beispiel die Existenz und Eindeutigkeit von
Schnittpunkten von solchen Objekten wird damit auf algebraische Probleme
mit Matrizen und linearen Gleichungssytemen zurückgeführt.
Das Ziel von Abschnitt~\ref{section:alggeo} ist daher, eine Technik
zur Übersetzung von geometrischen Problemen in algebraische zu finden,
so dass der Gauss-Algorithmus von Kapitel~\ref{chapter-lingl} zum universellen
Lösungsalgorithmus für solche Problem wird.

In diesem Kapitel begnügen wir uns mit den in Kapitel~\ref{chapter-lingl}
entwickelten Konzepten.
Insbesondere kennen wir noch kein Konzept für Längen, Winkel,
Flächeninhalte oder Volumina.
Diese erfordern als zusätzliche Struktur das Skalarprodukt, welches in
Kapitel~\ref{chapter:orthogonalitaet} eingeführt wird, oder die Orientierung,
welche Kapitel~\ref{chapter:orientierung} zum ermöglich das Vektorprodukt
zu definieren.





