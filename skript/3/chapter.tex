%
% chapter.tex -- Affine Vektorgeometrie
%
% (c) 2018 Prof Dr Andreas Müller, Hochschule Rapperswil
%
\chapter{Affine Vektorgeometrie\label{chapter:affin}}
\rhead{Affine Vektorgeometrie}
Mit den in Kapitel~\ref{chapter-lingl} entwickelten Methoden zur Lösung
linearer Gleichungen und den algebraischen Konzepten von Matrizen und Vektoren
können auch geometrische Situationen effizient beschrieben werden.
Dazu benötigt man zunächst Beschreibung des geometrischen Raumes mit Hilfe von
Vektoren.
Dazu dient das Konzept der Basis und des zugehörigen Koordinatensystems,
welches in Abschnitt~\ref{section:basis} eingeführt wird.
Da der gleiche geometrische Raum mit verschiedenen Koordinatensystemen
beschrieben werden kann, müssen wir im Folgenden immer auch untersuchen,
wie sich die Beschreibung eines geometrischen Objektes ändert, wenn man
eine andere Basis verwendet.

In Abschnitt~\ref{section:lineare abbildungen} wird gezeigt, wie sich
geometrische Abbildungen als lineare Abbildungen verstehen lassen und
daher mit Hilfe von Matrizen beschreiben lassen.
Das Matrizenprodukt erhält damit eine anschauliche Bedeutung als die
Komposition von Abbildungen.

Geraden und Ebenen sowie höherdimensionale Unterräume von Vektorräumen
können alle mit Hilfe von linearen Abbildungen beschrieben.
Probleme der Lage wie zum Beispiel die Existenz und Eindeutigkeit von
Schnittpunkten von solchen Objekten wird damit auf algebraische Probleme
mit Matrizen und linearen Gleichungssytemen zurückgeführt.
Das Ziel von Abschnitt~\ref{section:alggeo} ist daher, eine Technik
zur Übersetzung von geometrischen Problemen in algebraische zu finden,
so dass der Gauss-Algorithmus von Kapitel~\ref{chapter-lingl} zum universellen
Lösungsalgorithmus für solche Problem wird.

In diesem Kapitel begnügen wir uns mit den in Kapitel~\ref{chapter-lingl}
entwickelten Konzepten.
Insbesondere kennen wir noch kein Konzept für Längen, Winkel,
Flächeninhalte oder Volumina.
Diese erfordern als zusätzliche Struktur das Skalarprodukt, welches in
Kapitel~\ref{chapter:orthogonalitaet} eingeführt wird, oder die Orientierung,
welche Kapitel~\ref{chapter:orientierung} zum ermöglich das Vektorprodukt
zu definieren.

\begin{verbatim}
5.1 Determinante und Orientierung
- Orientierter Flächeninhalt
- Orientiertes Volumen
5.2 Vektorprodukt
- Definition und Eigenschaften
- Normalenvektoren
- Abstandsformeln
5.3 Abbildungen
- Determinante und Volumenänderung
- Spezielle lineare Gruppe
- Spezielle orthogonale Gruppe
\end{verbatim}



%
% koordinatensysteme.tex
%
% (c) 2018 Prof Dr Andreas Müller, Hochschule Rapperswil
%
\section{Punkte und Koordinatensysteme}
\subsection{Punkte im Raum und Vektoren}
\subsubsection{Koordinatensystem}
Punkte der Ebene können mit Hilfe eines Koordinatensystems festgelegt
werden: von einem Ausgangspunkt $O$, dem Nullpunkt des Koordinatensystems
aus, werden die Koordinaten entlang zweier aufeinander senkrecht stehender
Richtungen, den Achsrichtungen abgetragen.
Die Geraden durch den Ursprung
des Koordinatensystems in Achsrichtung heissen die Koordinatenachsen.

\begin{figure}
\begin{center}
\includegraphics{images/v-1}
\end{center}
\caption{Vektor zwischen zwei Punkten $O$ und $P$, Ortsvektor des Punktes $P$
\label{image-vektor}}
\end{figure}
Ein Punkt $P$ wird durch die Länge der Projektionen der Strecke $OP$
auf die Achsrichtungen festgelegt, also durch ein Zahlenpaar $(x,y)$.
Der Punkt $(1,0)$ liegt auf der $x$-Achse, eine Einheit vom Punkt $O$
entfernt.
Der Punkt $(0,1)$ liegt gleich weit entfernt von $O$ auf der $y$-Achse.

\subsubsection{Ortsvektoren}
Die Einführung eines Koordinatensystems ermöglicht zwar, Punktmengen
in der Ebene durch algebraische Beziehungen zwischen den Koordinaten
zu beschreiben, zum Beispiel ist der Einheitskreis die Menge der
Punkte
\[
\{(x,y)\;|\;x^2+y^2=1\}.
\]
``Rechnen'' kann man mit den Punkten aber noch nicht.
Dazu bilden wir
die Spaltenvektoren, mit denen wir ja bereits zu rechnen gelernt haben,
wie folgt auf die Punkte der Ebene ab:
\[
\begin{pmatrix}x\\y\end{pmatrix}
\mapsto (x,y)
\]
Den Vektor $\begin{pmatrix}x\\y\end{pmatrix}$ können wir uns als
Vektor vom Nullpunkt des Koordinatensystems zum Punkt $P=(x,y)$ vorstellen:
\[
\begin{pmatrix}x\\y\end{pmatrix}
=
\overset{\rightarrow}{OP}.
\]
\index{Ortsvektor}
$\overrightarrow{OP}$ heisst
heisst {\em Ortsvektor} des Punktes $P$.
Analoges gilt in drei Dimensionen:
\[
\begin{pmatrix}x\\y\\z\end{pmatrix}
\mapsto
(x,y,z)
\]
Ja es gibt überhaupt keinen Grund, die Geometrie auf zwei oder drei
Dimensionen zu beschränken, wir können Vektoren beliebiger Dimension
auf Punkte
abbilden, die entsprechend viele Koordinaten haben:
\[
\begin{pmatrix}
x_1\\x_2\\\vdots\\x_n
\end{pmatrix}
\mapsto
(x_1,x_2,\dots,x_n)
\]

\subsubsection{Vektoroperationen}
Die Rechenoperationen mit Vektoren können wir ebenfalls in geometrische
Begriffe übersetzen:
\begin{figure}
\begin{center}
\begin{tabular}{cc}
\includegraphics{images/v-2}&
\includegraphics{images/v-3}
\end{tabular}
\end{center}
\caption{Algebraische Operationen mit Vektoren\label{image-vektor-operationen}}
\end{figure}
\begin{itemize}
\index{Streckung}
\item Die Multiplikation mit einer Zahl $\lambda$ ist eine Streckung mit
Zentrum $O$ und Streckfaktor $\lambda$.
\index{Aneinandersetzen}
\item Die Addition von zwei Vektoren entspricht dem ``Aneinandersetzen''
der Strecken.
\end{itemize}
\begin{figure}
\begin{center}
\includegraphics{images/v-4}
\end{center}
\caption{Vektor $\overset{\rightarrow}{AB}$
\label{image-vektorab}}
\end{figure}
Besonders im Falle der Addition spielt die ``Richtung'' der Strecken eine
Rolle, wir symbolisieren diese Richtung durch einen Pfeil.
Der Pfeil vom Punkt $A$ zum Punkt $B$ ist die Differenz der Vektoren die von
$O$ zum Punkt $B$ bzw.~zu $A$ führen:
\[
\overset{\rightarrow}{AB}=\overset{\rightarrow}{OB}-\overset{\rightarrow}{OA}
\]

\subsubsection{Vektoren als Translationen}
Vektoren können auch als Verschiebungs-Operationen  oder Translationen
des Raumes betrachtet werden.
Wenn eine Verschiebung den Punkt $A$ in den Punkt $B$ verschiebt,
verschiebt sie den Punkt $C$ in den Punkt $D$ mit Ortsvektor
\[
\overrightarrow{OD}
=
\overrightarrow{OC}+\overrightarrow{AB}
=
\overrightarrow{OC}+\overrightarrow{OB}-\overrightarrow{OA}.
\]
Die Verschiebung entspricht also der Addition eines Verschiebungsvektors
zu allen Ortsvektoren.

\subsubsection{Bemerkungen zur Notation}
In diesem geometrischen Zusammenhang werden wir oft Vektoren als
kleine Buchstaben mit einem Pfeil schreiben: $\vec v$.
Um jedoch interessante
Geometrie zu treiben, müssen wir die üblichen geometrischen Begriffe in
Vektorschreibweise übersetzen: Geraden, Ebenen, Kreise, Kugeln.
Ausserdem müssen wir lernen, wie übliche geometrische Konstruktionen in
Rechenoperationen mit Vektoren übersetzt werden können.

\subsection{Basis}
\index{Basis}
\begin{figure}
\begin{center}
\includegraphics{images/v-5}
\end{center}
\caption{Koordinatensystem und Basis\label{imagebasis}}
\end{figure}
Im Abschnitt \ref{speziellevektoren} haben wir die Vektoren $e_i$
kennengelernt.
Im aktuellen Zusammenhang schreiben wir dafür oft auch
\[
\vec e_1=\begin{pmatrix}1\\0\\0\end{pmatrix},\qquad
\vec e_2=\begin{pmatrix}0\\1\\0\end{pmatrix},\qquad
\vec e_3=\begin{pmatrix}0\\0\\1\end{pmatrix}.
\]
Diese Vektoren konnten dafür verwendet werden, einen beliebigen Vektor
mit Hilfe einer Linearkombination zusammenzusetzen:
\[
\vec v=
v_1\vec e_1+
v_2\vec e_2+
v_3\vec e_3.
\]
Die speziellen Vektoren $e_1,\dots,e_n$ sind nicht die einzig möglichen,
mit denen man die Position eines Punktes auf vektorielle Weise
beschreiben könnte.
Jeder andere Satz von $n$ Vektoren kann
dazu verwendet werden, sofern sich damit jeder beliebige Vektor
linear kombinieren lässt.
In Kapitel~1 haben wir gelernt, dass dies
gleichbedeutend damit ist, dass die Vektoren linear unabhängig sein
müssen.
\begin{definition}
$n$ linear unabhängige Vektoren $b_1,\dots,b_n$ heissen eine
Basis des $n$-dimensionalen Raumes.
\end{definition}
Um die Koordinaten eines Punktes $x$ in dieser Basis zu bestimmen,
müssen die Zahlen $\xi_1,\dots,\xi_n$ berechnet werden für die
gilt
\[
b_1\xi_1+\dots+b_n\xi_n=x.
\]
Ausgeschrieben ist dies das Gleichungssystem
\[
\begin{linsys}{3}
b_{11}\xi_1&+&\dots &+&b_{1n}\xi_n&=&x_1\\
\vdots   & &\ddots& &\vdots&&\vdots\\
b_{n1}\xi_1&+&\dots &+&b_{nn}\xi_n&=&x_n\\
\end{linsys}
\]
Sind die Vektoren linear unabhängig, dann ist die Koeffizientenmatrix
regulär, das Gleichungssystem hat also genau eine Lösung.
Die behauptete Darstellung ist also immer möglich.

Somit haben wir eine weitere Interpretation einer regulären Matrix:
die Spalten einer regulären Matrix sind Vektoren, die man dazu verwenden
kann, jeden beliebigen anderen Vektor linear zu kombinieren.

\begin{beispiel}
Man stelle den Vektor $\vec v$ in der Basis $b_1,b_2,b_3$ dar:
\[
\vec b_1=\begin{pmatrix}2\\-3\\-2\end{pmatrix},\quad
\vec b_2=\begin{pmatrix}6\\-2\\-3\end{pmatrix},\quad
\vec b_3=\begin{pmatrix}-1\\2\\1\end{pmatrix},\qquad
\vec v=\begin{pmatrix}-6\\3\\3\end{pmatrix}.
\]

\smallskip

{\parindent 0pt
Die} Koordinaten $(\xi_1,\xi_2,\xi_3)$ müssen gefunden
werden, so dass
\[
\xi_1\vec b_1+
\xi_2\vec b_2+
\xi_3\vec b_3
=
\vec v,
\]
d.~h.
\[
\xi_1\begin{pmatrix}2\\-3\\-2\end{pmatrix}+
\xi_2\begin{pmatrix}6\\-2\\-3\end{pmatrix}+
\xi_3\begin{pmatrix}-1\\2\\1\end{pmatrix}=
\begin{pmatrix}-6\\3\\3\end{pmatrix}
\quad
\Leftrightarrow
\quad
\begin{pmatrix}
2&6&-1\\
-3&-2&2\\
-2&-3&1
\end{pmatrix}
\begin{pmatrix}\xi_1\\\xi_2\\\xi_3\end{pmatrix}
=
\begin{pmatrix}-6\\3\\3\end{pmatrix}.
\]
Auflösung des Gleichungssystems mit dem Gauss-Algorithmus oder mit
dem Computer ergibt.
\[
\begin{pmatrix}\xi_1\\\xi_2\\\xi_3\end{pmatrix}
=
\begin{pmatrix}1\\-1\\2 \end{pmatrix}.
\]
\end{beispiel}


%
% basis.tex
%
% (c) 2018 Prof Dr Andreas Müller, Hochschule Rappers2il
%
\section{Basis}
\rhead{Basis}
\index{Basis}
In den bisherigen Beispielen sind Vektorräume entweder als Nullräume oder
als Bildräume einer Matrix entstanden.
Der Test, ob ein Vektor in einem
Vektorraum drin ist, muss entsprechend auf verschiedene Art erfolgen:
\begin{enumerate}
\item Ist $V=\operatorname{ker}A$, berechnet man einfach $Av$, falls $Av=0$
kann man schliessen, dass $v\in V$.
\item Ist $V=\operatorname{im}A$, dann muss man einen Vektor $u\in\mathbb R^n$
finden mit $v=Au$, man muss also ein lineares Gleichungssystem lösen.
\end{enumerate}
Der zweite Fall ist also deutlich aufwendiger.
Schreibt man die Bedingung
$v=Au$ als Gleichung von Spaltenvektoren von $A$, bekommt man
\[
\begin{pmatrix}v_1\\\vdots\\v_m\end{pmatrix}
=
\begin{pmatrix}a_{11}\\\vdots\\a_{m1}\end{pmatrix}u_1+\dots+
\begin{pmatrix}a_{1n}\\\vdots\\a_{mn}\end{pmatrix}u_n.
\]
Man muss also entscheiden, ob sich $v$ durch die Spaltenvektoren von
$A$ ausdrücken lässt.
Selbst wenn $\operatorname{im}A$ nur ein
relativ kleiner Raum ist, zum Beispiel weil alle Spalten von $A$
Vielfache eines
einzigen Vektors sind, gibt dies ziemlich viel Arbeit.
Wir möchten
daher einen Vektorraum als Bildraum einer möglichst kleinen Matrix
schreiben können.
Gleichbedeutend damit ist, dass wir die Vektoren
des Raumes aus einer möglichst kleinen Zahl von Vektoren linear
kombinieren können möchten.

Diese Art von Darstellung beliebiger Vektoren mit Hilfe einer
ausgewählten Familie von Vektoren liegt dem Koordinaten-Systemen
zu Grunde.
Um die Koordinaten eines Punktes zu finden, zerlegt
man seinen Ortsvektor in die Vektoren entlang der Koordinatenachsen:
\[
\begin{pmatrix}2\\3\\5\end{pmatrix}
=
2\begin{pmatrix}1\\0\\0\end{pmatrix}+
3\begin{pmatrix}0\\1\\0\end{pmatrix}+
5\begin{pmatrix}0\\0\\1\end{pmatrix}
\]
Es müssen aber zwei Dinge sichergestellt werden:
\begin{enumerate}
\item Jeder Vektor muss in dieser Form darstellbar sein.
Für die
Standardbasisvektoren ist dies offensichtlich, aber wie findet man
dies bei einer beliebigen Menge von Vektoren heraus?
\item Die Darstellung muss eindeutig sein, sonst hat man offensichtlich
mehrere Beschreibungen des gleichen Punktes.
Wie kann man entscheiden,
ob ein Vektor nur auf eine Art aus den Vektoren kombiniert werden kann?
\end{enumerate}
Die nächsten zwei Abschnitte adressieren diese Fragen.

\subsection{Aufgespannter Raum}
\index{aufspannter Raum}
Wir müssen also zunächst klären, was es heisst, dass wir ``genügend''
Vektoren haben, um den Unterraum $V$ zu bilden.

\begin{definition}
\index{erzeugen}
Eine Menge $B=\{b_1,b_2,\dots,b_n\}$ von Vektoren erzeugt den
linearen Raum
\[
\langle B\rangle =
\langle b_1,\dots , b_n\rangle =
\{\lambda_1b_1+\dots+\lambda_nb_n\,|\,\lambda_i\in\mathbb R\}.
\]
Man sagt auch, $B$ spannt den Raum $\langle B\rangle$ auf.
\end{definition}

\begin{beispiel}[Beispiel: Standardbasisvektoren]
\index{Standardbasisvektoren}
Die speziellen Vektoren
\[
e_1=\begin{pmatrix}1\\0\\\vdots\\0\end{pmatrix},\quad
e_2=\begin{pmatrix}0\\1\\\vdots\\0\end{pmatrix},
\dots,
e_n=\begin{pmatrix}0\\0\\\vdots\\1\end{pmatrix}
\]
erzeugen den ganzen Raum $\mathbb R^n$, denn jeder Vektor $v\in\mathbb R^n$
lässt sich schreiben als
\[
v=\begin{pmatrix}v_1\\v_2\\\vdots\\v_n\end{pmatrix}
=
v_1\begin{pmatrix}1\\0\\\vdots\\0\end{pmatrix}+
v_2\begin{pmatrix}0\\1\\\vdots\\0\end{pmatrix}+
\dots+
v_n\begin{pmatrix}0\\0\\\vdots\\1\end{pmatrix}.
\]
Somit ist $\mathbb R^n=\langle e_1,e_2,\dots,e_n\rangle$.
die Vektoren $e_i$ heissen die Standardbasisvektoren.
\end{beispiel}

\begin{beispiel}
Gegeben seien die Vektoren
\[
a_1=\begin{pmatrix}1\\4\\7\end{pmatrix},\quad
a_2=\begin{pmatrix}2\\5\\8\end{pmatrix},\quad
a_3=\begin{pmatrix}3\\6\\9\end{pmatrix}
\]
Spannen diese drei Vektoren den ganzen Raum $\mathbb R^3$ auf,
oder anders ausgedrückt, kann man damit jeden Vektor in $\mathbb R^3$
linear kombinieren?

Dies wäre möglich, wenn die offensichtlich ganz $\mathbb R^3$
aufspannenden Vektoren $e_1,e_2,e_3$ durch $a_i$ ausgedrückt werden
können.
Also testen wir, ob $e_i\in \operatorname{im}A$, wobei
\[
A=\begin{pmatrix}
1&2&3\\
4&5&6\\
7&8&9
\end{pmatrix}
\]
ist.
Der Gauss-Algorithmus gibt folgende Tableaus:
\begin{align*}
\begin{tabular}{|>{$}c<{$}>{$}c<{$}>{$}c<{$}|>{$}c<{$}>{$}c<{$}>{$}c<{$}|}
\hline
1%
\begin{picture}(0,0)
\color{red}\put(-3,4){\circle{12}}
\end{picture}%
&2&3&1&0&0\\
4&5&6&0&1&0\\
7%
\begin{picture}(0,0)
\color{blue}\drawline(-8,-2)(-8,25)(1,25)(1,-2)
\end{picture}%
&8&9&0&0&1\\
\hline
\end{tabular}
&\rightarrow
\begin{tabular}{|>{$}c<{$}>{$}c<{$}>{$}c<{$}|>{$}c<{$}>{$}c<{$}>{$}c<{$}|}
\hline
1&2&3&1&0&0\\
0&-3%
\begin{picture}(0,0)
\color{red}\put(-7,4){\circle{15}}
\end{picture}%
&-6&-4&1&0\\
0&-6%
\begin{picture}(0,0)
\color{blue}\drawline(-15,-2)(-15,10)(1,10)(1,-2)
\end{picture}%
&-12&-7&0&1\\
\hline
\end{tabular}
\\
&\rightarrow
\begin{tabular}{|>{$}c<{$}>{$}c<{$}>{$}c<{$}|>{$}c<{$}>{$}c<{$}>{$}c<{$}|}
\hline
1&2&3&1&0&0\\
0&1&2&\frac43&-\frac13&0\\
0&0&0&1&-2&1\\
\hline
\end{tabular}
\end{align*}
Man kann also keinen einzigen der Vektoren $e_i$ mit den Vektoren $a_i$
darstellen, die Vektoren $a_i$ können also unmöglich den ganzen
$\mathbb R^3$ aufspannen.
\end{beispiel}

\subsection{Basis}
\index{Basis}
Eine Menge von Vektoren, mit denen man alle Vektoren eines Unterraums
linear kombinieren kann, muss linear unabhängig sein.
Sonst könnte
man nämlich einen der Vektoren durch die anderen ausdrücken, damit
wird er selbst unnötig.

\begin{definition}
\index{Basis}
Ein Menge $B\subset V$ von Vektoren in einem linearen Raum $V$ heisst
Basis, wenn gilt:
\begin{enumerate}
\item $B$ spannt $V$ auf: $V=\langle B\rangle$.
\item Die Vektoren in $B$ sind linear unabhängig.
\end{enumerate}
\end{definition}

\begin{beispiel}
Der Raum
\[
\left\{\left.\begin{pmatrix}x\\y\end{pmatrix}\,\right|\, x=y\right\}
\]
wird aufgespannt von 
\[
B=\left\{\begin{pmatrix}1\\1\end{pmatrix}\right\}.
\]
\end{beispiel}

Wie kann man testen, ob eine Menge von Vektoren eine Basis ist?
Schreibt man die Vektoren der Basis $B$ als Spalten in eine Matrix
$\tilde B$, müssen wir offenbar zwei Dinge testen:
\begin{enumerate}
\item $V = \operatorname{im}\tilde B$: das Gleichungssystem $\tilde Bx=b$
muss für jeden Vektor $b\in V$ lösbar sein.
\item Die Vektoren von $B$ sind linear unabhängig: die Spalten sind
linear unabhängig.
Dies kann auf zwei Arten geschehen:
\begin{enumerate}
\item Der Gauss-Algorithmus, angewendet auf $\tilde B^t$, liefert eine
Nullzeile genau dann, wenn die Zeilen von $\tilde B^t$ linear abhängig sind,
das sind aber genau die Spalten von $\tilde B$.
Wenn also keine Nullzeile
auftritt, sind die Vektoren linear unabhängig.
\item Falls der Gauss-Algorithmus für das homogene Gleichungssystem
$\tilde B x=0$ nur die Lösung $x=0$ findet, sind die Spalten linear
unabhängig.
Anders ausgedrückt: wenn das Gleichungssystem $\tilde Bx=b$
für jedes in Frage kommende $b$ genau eine Lösung hat, sind die
Spalten von $\tilde B$ linear unabhängig.
\end{enumerate}
\end{enumerate}


\begin{satz}
\index{Standardbasis}
Die Standardbasisvektoren bilden eine Basis von $\mathbb R^n$,
die Standardbasis.
\end{satz}

\begin{proof}[Beweis]
Wir wissen bereits, dass die Standardbasisvektoren $\mathbb R^n$ aufspannen,
wir müssen aber nur noch erkennen, dass sie auch linear unabhängig sind.
Schreibt man die Vektoren als Spalten in eine Matrix $\tilde B$, entsteht die
Einheitsmatrix, und ein Gleichungssystem $Ex=b$ hat immer genau die eine
Lösung $x=b$, also ist 
\end{proof}

\begin{beispiel}
Die Vektoren 
\[
b_1=\begin{pmatrix}1\\0\\0\end{pmatrix},\quad
b_2=\begin{pmatrix}1\\1\\0\end{pmatrix},\quad
b_3=\begin{pmatrix}1\\1\\1\end{pmatrix},\quad
\]
bilden eine Basis von $\mathbb R^3$.
Zwei Dinge sind zu prüfen:
spannen sie den ganzen Raum auf und sind sie linear unabhängig.
Da man alle Standardbasisvektoren durch die $b_i$ ausdrücken kann, nämlich
durch
\[
e_1=b_1,\qquad e_2=b_2-b_1,\qquad\text{und}\qquad e_3=b_3-b_2-b_1,
\]
ist jeder Vektor durch die $b_i$ ausdrückbar.
Lineare Abhängigkeit kann
man mit dem Gauss-Algorithmus testen.
Wir schreiben dazu die Vektoren
als Zeilen in ein Gauss-Tableau:
\[
\begin{tabular}{|>{$}c<{$}>{$}c<{$}>{$}c<{$}|}
\hline
1&0&0\\
1&1&0\\
1&1&1\\
\hline
\end{tabular}
\rightarrow
\begin{tabular}{|>{$}c<{$}>{$}c<{$}>{$}c<{$}|}
\hline
1&0&0\\
0&1&0\\
0&0&1\\
\hline
\end{tabular}
\]
Da keine Nullzeile entstanden ist, sind die Zeilen linear unabhängig.
Die Zeilen waren aber genau die Vektoren $b_i$.
Damit ist klar, dass
die Vektoren $b_i$ eine Basis bilden.
\end{beispiel}

\begin{definition}
\index{Dimension}
Die Dimension $\dim V$ eines Vektorraumes $V$ ist die Zahl der
Basisvektoren einer Basis von $V$.
\end{definition}

\begin{beispiel} Man finde eine Basis des Nullraumes der Matrix
\[
A=\begin{pmatrix}
1&2&3\\
4&5&6\\
7&8&9
\end{pmatrix}.
\]
Wendet man den Gauss-Algorithmus für das homogene Gleichungssystem
an, findet man:
\begin{align*}
\begin{tabular}{|>{$}c<{$}>{$}c<{$}>{$}c<{$}|>{$}c<{$}|}
\hline
 1%
\begin{picture}(0,0)
\color{red}\put(-3,4){\circle{12}}
\end{picture}%
& 2& 3&0\\
 4& 5& 6&0\\
 7%
\begin{picture}(0,0)
\color{blue}\drawline(-8,-2)(-8,24)(1,24)(1,-2)
\end{picture}%
& 8& 9&0\\
\hline
\end{tabular}
&\rightarrow
\begin{tabular}{|>{$}c<{$}>{$}c<{$}>{$}c<{$}|>{$}c<{$}|}
\hline
 1& 2&  3&0\\
 0&-3%
\begin{picture}(0,0)
\color{red}\put(-6,4){\circle{15}}
\end{picture}%
& -6&0\\
 0&-6%
\begin{picture}(0,0)
\color{blue}\drawline(-15,-2)(-15,10)(2,10)(2,-2)
\end{picture}%
&-12&0\\
\hline
\end{tabular}
\\
&\rightarrow
\begin{tabular}{|>{$}c<{$}>{$}c<{$}>{$}c<{$}|>{$}c<{$}|}
\hline
 1& 2%
\begin{picture}(0,0)
\color{blue}\drawline(-8,10)(-8,-2)(2,-2)(2,10)
\end{picture}%
&  3&0\\
 0& 1&  2&0\\
 0& 0&  0&0\\
\hline
\end{tabular}
\\
&\rightarrow
\begin{tabular}{|>{$}c<{$}>{$}c<{$}>{$}c<{$}|>{$}c<{$}|}
\hline
 1& 0& -1&0\\
 0& 1&  2&0\\
 0& 0&  0&0\\
\hline
\end{tabular}
\end{align*}
Offenbar gibt es genau eine frei wählbare Variable $z$, und
die Lösungsmenge ist 
\[
t\begin{pmatrix}
1\\-2\\1
\end{pmatrix}
\]
Der Nullraum wird also vom Vektor
\[
\begin{pmatrix}
1\\-2\\1
\end{pmatrix}
\]
aufgespannt.
\end{beispiel}

\begin{beispiel}
Man finde eine Basis des Bildraumes $\operatorname{im}A$ mit der gleichen
Matrix wie im vorangegangenen Beispiel.

Die Basis besteht aus so wenigen linear unabhängigen Vektoren wie
möglich, aber alle Spaltenvektoren von $A$ müssen damit erzeugt
werden können.
Dazu nehmen wir einfach einen Vektor um den anderen
hinzu, solange die Menge der Vektoren linear unabhängig bleibt.

Der erste Spaltenvektor ist nicht der Nullvektor, also können wir
den in die Basis hinein nehmen.
Der zweite ist nicht proportional,
also sind die ersten beiden Spalten linear unabhängig.
Als
Basis könnten wir daher
\[
\left\{
\begin{pmatrix}1\\4\\7\end{pmatrix}
,
\begin{pmatrix}2\\5\\8\end{pmatrix}
\right\}
\]
nehmen.
Den dritten Spaltenvektor dürfen wir nicht hinzunehmen.
Aus dem letzten Beispiel wissen wir ja, dass die Matrix singulär
ist.
Es kann also höchstens zwei linear unabhängige Vektoren haben.
\end{beispiel}

\subsection{Koordinaten}
\index{Koordinaten}
Gibt man eine Basis $B=\{b_1,\dots,b_k\}$ von $V$ vor,
dann kann man die Vektoren
in $V$ in der Basis $B$ ausdrücken.
Dazu muss man zu einem 
Vektor $v\in V$ Zahlen $\xi_i$ finden mit
\[
v=\xi_1 b_1+\dots +\xi_k b_k.
\]
Schreiben wir wieder die Basisvektoren von $B$ als Spaltenvektoren in 
eine Matrix $\tilde B$, dann ist 
\[
v=\tilde B\begin{pmatrix}\xi_1\\\vdots\\\xi_n\end{pmatrix}.
\]
Eine Basis eines $k$-dimensionalen Raumes $V$ ermöglicht also,
die Vektoren von $V$ mit Hilfe von $k$-Tupeln, bestehend
aus den Zahlen $\xi_i$, darzustellen.
Die $\xi_i$ heissen
Koordinaten eines Vektors $v$ in der Basis $B$, wir schreiben
oft auch einfach $\xi$ für den Vektor mit Komponenten $\xi_i$.
Das Finden der Koordinaten eines Vektors $v$ läuft immer auf die Lösung
des Gleichungssystems $\tilde B\xi=v$ hinaus.

\begin{beispiel}
Die Basis \[
B=\left\{
\begin{pmatrix}1\\2\\3\end{pmatrix},
\begin{pmatrix}3\\2\\1\end{pmatrix}
\right\}
\]
spannt einen zweidimensionalen Unterraum von $\mathbb R^3$ auf.
Man finde
die Koordinaten des Vektors 
\[
v=
\begin{pmatrix}
0\\4\\8
\end{pmatrix}
\]
in der Basis $B$.

Dazu muss man das Gleichungssystem $v=\tilde B\xi$ lösen:
\[
\begin{tabular}{|>{$}c<{$}>{$}c<{$}|>{$}c<{$}|}
\hline
1%
\begin{picture}(0,0)
\color{red}\put(-3,4){\circle{12}}
\end{picture}%
&3&0\\
2&2&4\\
3%
\begin{picture}(0,0)
\color{blue}\drawline(-8,-2)(-8,25)(2,25)(2,-2)
\end{picture}%
&1&8\\
\hline
\end{tabular}
\rightarrow
\begin{tabular}{|>{$}c<{$}>{$}c<{$}|>{$}c<{$}|}
\hline
1&3&0\\
0&-4%
\begin{picture}(0,0)
\color{red}\put(-7,4){\circle{15}}
\end{picture}%
&4\\
0&-8%
\begin{picture}(0,0)
\color{blue}\drawline(-15,-2)(-15,10)(1,10)(1,-2)
\end{picture}%
&8\\
\hline
\end{tabular}
\rightarrow
\begin{tabular}{|>{$}c<{$}>{$}c<{$}|>{$}c<{$}|}
\hline
1&3%
\begin{picture}(0,0)
\color{blue}\drawline(-8,10)(-8,-2)(1,-2)(1,10)
\end{picture}%
&0\\
0&1&-1\\
0&0&0\\
\hline
\end{tabular}
\rightarrow
\begin{tabular}{|>{$}c<{$}>{$}c<{$}|>{$}c<{$}|}
\hline
1&0&3\\
0&1&-1\\
0&0&0\\
\hline
\end{tabular}
\]
man kann also die Koordinaten $3$ und $-1$ ablesen.
Kontrolle:
\[
\tilde B\xi
=
\begin{pmatrix}
1&3\\
2&2\\
3&1\end{pmatrix}
\begin{pmatrix}3\\-1\end{pmatrix}
=\begin{pmatrix}
0\\4\\8
\end{pmatrix}
=v.
\]
\end{beispiel}

\subsection{Basiswechsel}
\index{Basiswechsel}
Jede beliebige linear unabhängige Teilmenge von $V$, welche ganz $V$
aufspannt, kann als Basis verwendet werden.
Oft ist es praktischer,
statt der Standardbasis eine andere Basis zu verwenden, zum Beispiel
um die Koordinatenachsen parallel zu den Kanten eines Werkstücks zu
bekommen, oder um eine spezielle Symmetrie des Problems einfacher
ausdrücken zu können.
Wie sind die Koordinaten zwischen zwei Basen $B$ und $B'$ umzurechnen?

Zu jeder Basis $B$ gibt es die Matrix $\tilde B$, die aus den
Koordinaten $\xi$ eines Vektors $v$ den Vektor mittels $v=\tilde B\xi$
berechnet.
Hat man zwei Basen $B$ und $B'$, hat auch jeder Vektor
in $V$ zwei verschiedene Koordinaten-Vektoren $\xi$ und $\xi'$:
\[
v=\tilde B\xi =\tilde B'\xi'.
\]
Wie kann man $\xi'$ aus $\xi$ berechnen?

\subsubsection{Spezialfall $\mathbb R^n$}
In diesem Fall sind die beiden Matrizen $\tilde B$ und $\tilde B'$
reguläre $n\times n$-Matrizen.
Die Gleichung
\[
v=\tilde B\xi =\tilde B'\xi'.
\]
Kann dann durch Multiplikation mit $\tilde B^{-1}$ oder $\tilde B^{\prime-1}$
von links aufgelöst werden:
\begin{align*}
\tilde B^{\prime-1}\tilde B\xi&= \tilde B^{\prime -1}\tilde B'\xi'=E\xi'=\xi'\\
\xi=\tilde B^{-1}B\tilde B\xi&=\tilde B^{-1}\tilde B'\xi'
\end{align*}
Daraus können wir die Matrix zur Berechnung von $\xi'$ aus $\xi$
ablesen: 
\[
T=\tilde B^{\prime-1}B \quad\Rightarrow\quad
\xi'=T\xi.
\]

\subsubsection{Allgemeiner Fall}
Im allgemeinen Fall eines beliebigen Unterraumes funktioniert diese
Methode nicht.
$\tilde B$ und $\tilde B'$ sind keine quadratischen
Matrizen, also können Sie auch nicht invertiert werden.
Vielmehr sind sie jetzt $n\times m$-Matrizen mit $m<n$.

Wenn $B$ und $B'$ Basen sind, dann lässt sich jeder Vektor $v\in
\operatorname{im}\tilde B=\operatorname{im}\tilde B'$ 
sowohl mit $\xi$-Koordinaten als auch mit $\xi'$-Koordinaten
beschreiben:
\[
v=\tilde B\xi=\tilde B'\xi'.
\]
Darin sind die $\xi$- und $\xi'$-Vektoren sind Vektoren in $\mathbb R^m$.

Speziell könnte man für die $\xi$-Vektoren die Standardbasisvektoren
in $R^m$ wählen.
Mit dem ersten Standardbasisvektor bekäme man
das Gleichungssystem
\begin{equation}
\tilde B\begin{pmatrix} 1\\0\\\vdots\\0\end{pmatrix}
=\tilde B'\xi'
\label{t-gleichung}
\end{equation}
Dieses Gleichungssystem kann man mit dem Gauss-Algorithmus auflösen,
und $\xi'$ finden.
Dasselbe kann man natürlich auch für den
zweiten Standardbasis-Vektor machen, und auch für alle folgenden.

Wir können alle diese Gleichungen zusammen in eine einzige
Matrixgleichung schreiben:
\[
\tilde B\begin{pmatrix}
1&0&\dots&0\\
0&1&\dots&0\\
\vdots&\vdots&\ddots&\vdots\\
0&0&\dots&1
\end{pmatrix}
=B'\begin{pmatrix}
\mathstrut t_{11}&t_{12}&\dots&t_{1m}\\
\mathstrut t_{21}&t_{22}&\dots&t_{2m}\\
\mathstrut \vdots&\vdots&\ddots&\vdots\\
\mathstrut t_{m1}&t_{m2}&\dots&t_{mm}\\
\end{pmatrix}
\]
Die Matrix $T$ auf der rechten Seite enthält als Spalten
die Lösungen der Gleichungen (\ref{t-gleichung}).

Da die Spalten von $T$ Lösungen eines linearen Gleichungssystems
mit Koeffizienten $\tilde B'$ und rechten Seiten $B$ ist, können
wir das Verfahren zur simultanen Lösung anwenden.
Das zugehörige Gauss-Tableau ist:
\[
\begin{tabular}{|>{$}c<{$}>{$}c<{$}>{$}c<{$}|>{$}c<{$}>{$}c<{$}>{$}c<{$}|}
\hline
\quad\mathstrut&         &\quad\mathstrut&\quad\mathstrut&        &\quad\mathstrut\\
      &         &      &      &        &      \mathstrut\\
      &         &      &      &        &      \mathstrut\\
      &\tilde B'&      &      &\tilde B&      \mathstrut\\
      &         &      &      &        &      \mathstrut\\
      &         &      &      &        &      \mathstrut\\
      &         &      &      &        &      \mathstrut\\
\hline
\end{tabular}
\rightarrow
\begin{tabular}{|>{$}c<{$}>{$}c<{$}>{$}c<{$}>{$}c<{$}|>{$}c<{$}>{$}c<{$}>{$}c<{$}|}
\hline
1     &     0&\dots   &0      & &                   & \\
0     &     1&\dots   &0      & &                   & \\
\vdots&\vdots&\ddots  &\vdots & &\raisebox{7pt}{$T$}& \\
0     &     0&\dots   &1      & &                   & \\
\hline
0     &     0&\dots   &0      &*&                   &*\\
0     &     0&\dots   &0      &*&                   &*\\
\hline
\end{tabular}
\]
Weil wir wissen, dass die Gleichungen immer eine Lösung haben, finden
wir an den Plätzen der Sterne unten rechts immer $0$, also steht im
Feld oben rechts genau die Transformationsmatrix $T$.

\begin{beispiel}
In der Ebene aufgespannt von den Vektoren 
\[
b_1=\begin{pmatrix}1\\1\\0 \end{pmatrix}
,\qquad
b_2=\begin{pmatrix}0\\1\\1 \end{pmatrix}
\]
möchte man die Basis aus den Basisvektoren
\[
b_1'=\begin{pmatrix}1\\2\\1\end{pmatrix}
,\qquad
b_2'=\begin{pmatrix}1\\0\\-1\end{pmatrix}
\]
Finden Sie die Koordinatentransformationsmatrix $T$, mit der
man von den Parameter $t$ $s$ in der Parameterdarstellung
mit Richtungsvektoren $b_1$ und $b_2$ auf die Koordinaten in
der Basis $B'$ umrechnen kann.
Man finde ausserdem die 
Koordinaten in der Basis $B'$ des Vektors, der in der Basis $B$
die Koordinaten $(2,-1)$ hat.

\begin{align*}
\begin{tabular}{|>{$}c<{$}>{$}c<{$}|>{$}c<{$}>{$}c<{$}|}
\hline
1& 1&1&0\\
2& 0&1&1\\
1&-1&0&1\\
\hline
\end{tabular}
&\rightarrow
\begin{tabular}{|>{$}c<{$}>{$}c<{$}|>{$}c<{$}>{$}c<{$}|}
\hline
1& 1& 1&0\\
0&-2&-1&1\\
0&-2&-1&1\\
\hline
\end{tabular}
\rightarrow
\begin{tabular}{|>{$}c<{$}>{$}c<{$}|>{$}c<{$}>{$}c<{$}|}
\hline
1& 1&      1&       0\\
0& 1&\frac12&-\frac12\\
\hline
0& 0&      0&       0\\
\hline
\end{tabular}
\\
&\rightarrow
\begin{tabular}{|>{$}c<{$}>{$}c<{$}|>{$}c<{$}>{$}c<{$}|}
\hline
1& 0& \frac12& \frac12\\
0& 1& \frac12&-\frac12\\
\hline
0& 0&      0&       0\\
\hline
\end{tabular}
\end{align*}
Die Transformationsmatrix ist also 
\[
T=
\frac12\begin{pmatrix} 1&1\\1&-1 \end{pmatrix}.
\]
Die Koordinaten $(2,-1)$ ergeben nach Umrechnung mit $T$ 
\[
\xi'=
T\begin{pmatrix}2\\-1\end{pmatrix}
=
\frac12\begin{pmatrix} 1&1\\1&-1 \end{pmatrix}
\begin{pmatrix}2\\-1\end{pmatrix}
=\frac12\begin{pmatrix}1\\3\end{pmatrix}.
\]
Zur Kontrolle berechnen wir den zugehörigen Vektor in $\mathbb R^3$:
\[
\tilde B\xi=
\begin{pmatrix}
1&0\\
1&1\\
0&1
\end{pmatrix}
\begin{pmatrix}2\\-1\end{pmatrix}
=\begin{pmatrix} 2\\1\\-1 \end{pmatrix}
,\qquad
\tilde B' \xi'
=
\begin{pmatrix}
1& 1\\
2& 0\\
1&-1
\end{pmatrix}\begin{pmatrix}\frac12\\\frac32\end{pmatrix}
=\begin{pmatrix}2\\1\\-1 \end{pmatrix},
\]
beide stimmen überein.
\end{beispiel}


%
% geraden.tex
%
% (c) 2018 Prof Dr Andreas Müller, Hochschule Rapperswil
%
\documentclass[tikz]{standalone}
\usepackage{times}
\usepackage{amsmath}
\usepackage{txfonts}
\usepackage[utf8]{inputenc}
\usepackage{graphics}
\usetikzlibrary{arrows,intersections,math}
\usepackage{ifthen}
\begin{document}

\newboolean{showgrid}
\setboolean{showgrid}{false}
\def\breite{7}
\def\hoehe{4}

\begin{tikzpicture}[>=latex,thick]

% Povray Bild
\node at (0,0) {\includegraphics[width=14cm]{geraden.jpg}};

% Beschriftung
\node at (-3.6,1.5) [above right] {$C$};
\node at (3,-1.9) {$g$};
\node at (5,-2.9) {$\sigma$};
\node at (-2,-0.5) {$\beta$};
\node at (-6.5,2.4) {$\beta'$};

% Gitter
\ifthenelse{\boolean{showgrid}}{
\draw[step=0.1,line width=0.1pt] (-\breite,-\hoehe) grid (\breite, \hoehe);
\draw[step=0.5,line width=0.4pt] (-\breite,-\hoehe) grid (\breite, \hoehe);
\draw                            (-\breite,-\hoehe) grid (\breite, \hoehe);
\fill (0,0) circle[radius=0.05];
}{}

\end{tikzpicture}

\end{document}


%
% ebenen.tex
%
% (c) 2018 Prof Dr Andreas Müller, Hochschule Rapperswil
%
\section{Ebenen}
\index{Ebene}
\index{Parameterdarstellung!einer Ebene}
\begin{figure}
\begin{center}
%\includegraphics{images/v-8}
\includegraphics{3/images/ebene.pdf}
\end{center}
\caption{Parametrisierung einer Ebene mit Stüztvektor $\vec{p}$ und
Richtungsvektoren $\vec{u}$ und $\vec{v}$.
\label{image-parametrisierungebene}}
\end{figure}%
Ebenen zeichnen sich gegenüber Geraden dadurch aus, dass sich auf ihnen
ein Punkt nicht nur in einer Richtung, sondern unabhängig zwei Richtungen
bewegen kann.
Wir brauchen also einen zweiten Richtungsvektor und einen zweiten Parameter.
Die Punkte einer Ebene $\sigma$ werden also beschrieben durch
\[
\vec r=\vec p+t\vec v_1+s\vec v_2,
\]
(Abbildung~\ref{image-parametrisierungebene}).

\begin{beispiel}
Man finde die Parameterdarstellung der Ebene durch die Punkte
$A=(1,2,1)$,
$B=(3,4,-1)$ und
$C=(4,-1,0)$.

\smallskip

{\parindent 0pt Die} Vektoren $\vec u=\overrightarrow{AB}$ und
$\vec v=\overrightarrow{AC}$ können als Richtungsvektoren
verwendet werden, und ergeben als Parameterdarstellung:
\begin{equation}
\vec r=\begin{pmatrix}1\\2\\1 \end{pmatrix}
+
t\begin{pmatrix}2\\2\\-2\end{pmatrix}
+
s\begin{pmatrix}3\\-3\\-1\end{pmatrix}.
\label{beispielebene}
\end{equation}
\end{beispiel}

\subsection{Durchstosspunkt\label{subsection-durchstosspunkt}}
\index{Durchstosspunkt}
\begin{figure}
\begin{center}
%\includegraphics{images/v-9}
\includegraphics{3/images/durchstosspunkt.pdf}
\end{center}
\caption{Durchstosspunkt $S$ der Geraden $g$ durch die Ebene
$\sigma$.\label{image-durchstosspunkt}}
\end{figure}
Im dreidimensionalen Raum
sind eine Gerade $g$ und eine Ebene $\sigma$ je in Parameterdarstellung gegeben, also
\begin{align*}
\sigma&=
\{
\vec p_0+s\vec u+t\vec v\;
|\;s, t\in\mathbb R\}
\\
g&=
\{
\vec q_0+l\vec r\;
|\;l\in\mathbb R
\}
\end{align*}
Es soll der Durchstosspunkt $S$ der Gerade durch die Ebene gefunden
werden (Abbildung~\ref{image-durchstosspunkt}).
Dazu müssen wir die Parameterwerte für $s$, $t$ und $l$ finden,
für die gilt
\begin{align*}
\vec p_0+s\vec u+t\vec v
&=
\vec q_0+l\vec r
\\
s\vec u+t\vec v-l\vec r&=\vec q_0+\vec p_0
\end{align*}
Dies ist eine Vektorgleichung mit den drei Unbekannten $s$, $t$ und $l$,
sie entspricht drei linearen Gleichungen für diese Unbekannten.

Im Allgemeinen wird dieses Gleichungssystem genau eine Lösung haben,
nämlich den Durchstosspunkt.
Ist die Gerade parallel zur Ebene, hat das Gleichungssystem jedoch keine
Lösung, es gibt also auch keinen Durchstosspunkt.
Liegt die Gerade dagegen in der Ebene, gibt es unendlich viele Lösungen.

In den singulären Fällen ist es offenbar möglich, die Richtung $\vec r$
der Gerade durch die beiden Vektoren $\vec u$ und $\vec v$ in der Ebene
auszudrücken, die drei Vektoren $\vec u$, $\vec v$ und $\vec r$ sind
linear abhängig.

\begin{beispiel}
Finde den Durchstosspunkt der Geraden
\[
\vec r=
\begin{pmatrix} 5\\8\\3 \end{pmatrix}
+
l\begin{pmatrix} 1\\0\\1 \end{pmatrix}
\]
durch die Ebene (\ref{beispielebene}).

\smallskip

{\parindent0pt Zusammen} mit der Gleichung (\ref{beispielebene}) bekommen wir die
Vektorgleichung
\[
\begin{pmatrix}1\\2\\1 \end{pmatrix}
+
t\begin{pmatrix}2\\2\\-2\end{pmatrix}
+
s\begin{pmatrix}3\\-3\\-1\end{pmatrix}
=
\begin{pmatrix} 5\\8\\3 \end{pmatrix}
+
l\begin{pmatrix} 1\\0\\1 \end{pmatrix}
\quad\Rightarrow\quad
t\begin{pmatrix}2\\2\\-2\end{pmatrix}
+
s\begin{pmatrix}3\\-3\\-1\end{pmatrix}
+
l\begin{pmatrix} -1\\0\\-1 \end{pmatrix}
=
\begin{pmatrix} 4\\6\\2 \end{pmatrix}
\]
Das Gleichungssystem kann mit dem Gauss-Algorithmus
gelöst werden
\begin{align*}
\begin{tabular}{|>{$}c<{$}>{$}c<{$}>{$}c<{$}|>{$}c<{$}|}
\hline
 2%
\begin{picture}(0,0)
\color{red}\put(-3,4){\circle{12}}
\end{picture}%
& 3&-1&4\\
 2&-3& 0&6\\
-2%
\begin{picture}(0,0)
\color{blue}\drawline(-15,-2)(-15,24)(2,24)(2,-2)
\end{picture}
&-1&-1&2\\
\hline
\end{tabular}
&
\rightarrow
\begin{tabular}{|>{$}c<{$}>{$}c<{$}>{$}c<{$}|>{$}c<{$}|}
\hline
 1& \frac32&-\frac12&2\\
 0&-6%
\begin{picture}(0,0)
\color{red}\put(-7,3){\circle{15}}
\end{picture}%
& 1&2\\
 0& 2&-2&6\\
\hline
\end{tabular}
\rightarrow
\begin{tabular}{|>{$}c<{$}>{$}c<{$}>{$}c<{$}|>{$}c<{$}|}
\hline
 1& \frac32&-\frac12&2\\
 0& 1&-\frac16&-\frac13\\
 0& 0&-\frac53%
\begin{picture}(0,0)
\color{red}\put(-7,3){\circle{16}}
\end{picture}%
&\frac{20}3\\
\hline
\end{tabular}
\\
&
\rightarrow
\begin{tabular}{|>{$}c<{$}>{$}c<{$}>{$}c<{$}|>{$}c<{$}|}
\hline
 1& \frac32&-\frac12&2\\
 0& 1&-\frac16%
\begin{picture}(0,0)
\color{blue}\drawline(-15,24)(-15,-5)(2,-5)(2,24)
\end{picture}
&-\frac13\\
 0& 0& 1&-4\\
\hline
\end{tabular}
\rightarrow
\begin{tabular}{|>{$}c<{$}>{$}c<{$}>{$}c<{$}|>{$}c<{$}|}
\hline
 1& \frac32%
\begin{picture}(0,0)
\color{blue}\drawline(-8,10)(-8,-5)(2,-5)(2,10)
\end{picture}%
& 0&0\\
 0& 1& 0&-1\\
 0& 0& 1&-4\\
\hline
\end{tabular}
\\
&
\rightarrow
\begin{tabular}{|>{$}c<{$}>{$}c<{$}>{$}c<{$}|>{$}c<{$}|}
\hline
 1& 0& 0&\frac32\\
 0& 1& 0&-1\\
 0& 0& 1&-4\\
\hline
\end{tabular}
\end{align*}
Da $l=-4$ kann man jetzt den Durchstosspunkt berechnen, er ist $S=(1,8,-1)$.
\end{beispiel}

\subsection{Schnittgerade}
\index{Schnittgerade zweier Ebenen}
\begin{figure}
\begin{center}
%\includegraphics{images/v-10}
\includegraphics{3/images/schnittgerade.pdf}
\end{center}
\caption{Schnittgerade $g$ (rot) zweier Ebenen $\sigma_0$ (grün)
und $\sigma_1$ (blau).
\label{image-schnittgerade}}
\end{figure}
Im dreidimensionalen Raum sind zwei Ebenen
\begin{align*}
\sigma_0&=
\{\vec p_0+s_0\vec u_0+t_0\vec v_0\;|\;s_0, t_0\in\mathbb R\}
\\
\sigma_1&=
\{\vec p_1+s_1\vec u_1+t_1\vec v_1\;|\;s_1, t_1\in\mathbb R\}
\end{align*}
gegeben, gesucht ist deren Schnittgerade $g$
(Abbildung~\ref{image-schnittgerade}).
Diese besteht offenbar aus den Punkten, die sich für geeignete Parameterwerte
$s_0,t_0,s_1,t_1$ mit der Eigenschaft
\begin{align*}
p_0+s_0\vec u_0+t_0\vec v_0
&=
p_1+s_1\vec u_1+t_1\vec v_1
\\
s_0\vec u_0+t_0\vec v_0
-s_1\vec u_1-t_1\vec v_1
&=
-p_0+
p_1
\end{align*}
finden lassen.
Dies ist ein Gleichungssystem mit drei Gleichungen für
vier Unbekannte, im Allgemeinen wird sich also eine der Unbekannten
nicht bestimmen lassen, sondern die anderen Unbekannten werden sich
durch die vierte ausdrücken lassen.
Die Punkte der Schnittmenge sind also von der Form
\[
p_1+s_1(t_1)\vec u_1+t_1\vec v_1,
\]
wobei von der Form $s_1(t_1)=a+bt_1$ sein muss.
Setzt man dies ein, ergibt sich wieder eine Geradengleichung:
\[
p_1+s_1(t_1)\vec u_1+t_1\vec v_1
=
p_1+(a+bt_1)\vec u_1+t_1\vec v_1
=
(p_1+a\vec u_1)+t_1(b\vec u_1+v_1),
\]
eine Geradengleichung mit Richtungsvektor $\vec r = (b\vec u_1+v_1)$ und
Ausgangspunkt $\vec q_0=p_1+a\vec u_1$, also mit der Parametrisierung
\[
g=\{\vec q_0+t\vec r\;|\;t\in\mathbb R\}.
\]

\begin{beispiel}
Man finde die Schnittgerade der beiden Ebenen mit Parameterdarstellung
\[
\sigma_0:
\vec p_0+s_0\vec u_0+t_0\vec v_0
=\begin{pmatrix}5\\8\\6\end{pmatrix}
+s_0\begin{pmatrix}4\\6\\7\end{pmatrix}
+t_0\begin{pmatrix}2\\5\\6\end{pmatrix}
,
\qquad
\sigma_1:
\vec p_1+s_1\vec u_1+t_0\vec v_1
=\begin{pmatrix}5\\4\\4\end{pmatrix}
+s_1\begin{pmatrix}6\\6\\7\end{pmatrix}
+t_1\begin{pmatrix}5\\4\\4\end{pmatrix}
\]

\smallskip

{\parindent 0pt Wir setzen die beiden Parametrisierungen gleich und
erhalten}
\[
\begin{pmatrix}5\\8\\6\end{pmatrix}
+s_0\begin{pmatrix}4\\6\\7\end{pmatrix}
+t_0\begin{pmatrix}2\\5\\6\end{pmatrix}
=\begin{pmatrix}5\\4\\4\end{pmatrix}
+s_1\begin{pmatrix}6\\6\\7\end{pmatrix}
+t_1\begin{pmatrix}5\\4\\4\end{pmatrix}
\]
oder
\[
s_0\begin{pmatrix}4\\6\\7\end{pmatrix}
+t_0\begin{pmatrix}2\\5\\6\end{pmatrix}
-s_1\begin{pmatrix}6\\6\\7\end{pmatrix}
-t_1\begin{pmatrix}5\\4\\4\end{pmatrix}
=
\begin{pmatrix}0\\-4\\-2\end{pmatrix}
\]
Der Gauss-Algorithmus liefert für die Lösung das Tableau
\[
\begin{tabular}{|>{$}c<{$}>{$}c<{$}>{$}c<{$}>{$}c<{$}|>{$}c<{$}|}
\hline
4&2&-6&-5& 0\\
6&5&-6&-4&-4\\
7&6&-7&-4&-2\\
\hline
\end{tabular}
\rightarrow
\begin{tabular}{|>{$}c<{$}>{$}c<{$}>{$}c<{$}>{$}c<{$}|>{$}c<{$}|}
\hline
    1&   0&   0& -\frac{11}2& -26\\
    0&   1&   0&   4&  16\\
    0&   0&   1&  -\frac32& -12\\
\hline
\end{tabular}
\]
Die Variable $t_1$ ist frei wählbar, daraus lassen sich die anderen
Variablen bestimmen:
\[
\begin{linsys}{3}
s_0&=&-26&+&\frac{11}2t_1\\
t_0&=&16&-&4t_1\\
s_1&=&-12&+&\frac32t_1\\
\end{linsys}
\]
Setzt man dies in die ursprünglichen Ebenengleichungen ein, entsteht
die Parameterdarstellung der Schnittgeraden:
\begin{align*}
\begin{pmatrix}x\\y\\z\end{pmatrix}
&=
\begin{pmatrix}5\\8\\6\end{pmatrix}
+\biggl(-26+\frac{11}2t_1\biggr)\begin{pmatrix}4\\6\\7\end{pmatrix}
+(16-4t_1)\begin{pmatrix}2\\5\\6\end{pmatrix}
=
\begin{pmatrix}-67\\-68\\-80\end{pmatrix}
+t_1
\begin{pmatrix}14\\13\\\frac{29}2\end{pmatrix},\quad\text{oder}
\\
\begin{pmatrix}x\\y\\z\end{pmatrix}
&=
\begin{pmatrix}5\\4\\4\end{pmatrix}
+\biggl(-12+\frac32t_1\biggr)\begin{pmatrix}6\\6\\7\end{pmatrix}
+t_1\begin{pmatrix}5\\4\\4\end{pmatrix}
=
\begin{pmatrix}-67\\-68\\-80\end{pmatrix}
+t_1
\begin{pmatrix}14\\13\\\frac{29}2\end{pmatrix}
\end{align*}
Natürlich muss man in beiden Fällen die gleiche Gerade bekommen,
dies ist eine hübsche Kontrolle für die Richtigkeit des Resultates.
\end{beispiel}

\subsection{Vereinheitlichtes Lösungsverfahren\label{section-vereinheitlichtes-verfahren}}
Alle in den bisherigen Abschnitten vorgestellten Lösungsverfahren
laufen darauf hinaus, dass man für die Parameter der Parameterdarstellungen
Gleichungen dadurch aufstellt, dass man wo nötig Gleichungen herstellt.
Dies ist aber eigentlich ein unnötiger Schritt, den man auch dem
Gauss-Algorithmus überlassen könnte.

Traditionell wird das so gemacht, weil früher eine grosse Angst
davor bestand, auf grosse Gleichungssysteme zu stossen, deren Lösung
mühsam ist.
Diese Angst ist heute mit modernen Taschenrechnern und
Computer nicht mehr gerechtfertigt.

\subsubsection{Durchstosspunkt}
Um den Durchstosspunkt wie in \ref{subsection-durchstosspunkt} zu finden,
gehen wir wieder von den Parameterdarstellungen aus, schreiben jetzt
aber den gesuchten gemeinsamen Ortsvektor des Punktes
$({\color{red}x}, {\color{red}y}, {\color{red}z})$ explizit hin:
\[
\begin{pmatrix}
\color{red}x\\
\color{red}y\\
\color{red}z\\
\end{pmatrix}
=
\begin{pmatrix}1\\2\\1 \end{pmatrix}
+
{\color{red}t}\begin{pmatrix}2\\2\\-2\end{pmatrix}
+
{\color{red}s}\begin{pmatrix}3\\-3\\-1\end{pmatrix}
,\qquad
\begin{pmatrix}
\color{red}x\\
\color{red}y\\
\color{red}z\\
\end{pmatrix}
=
\begin{pmatrix} 5\\8\\3 \end{pmatrix}
+
{\color{red}l}\begin{pmatrix} 1\\0\\1 \end{pmatrix}
\]
Darin haben wir alle unbekannten Grössen {\color{red}rot} markiert.
Wir haben jetzt sechs Gleichungen mit sechs Unbekannten.
Der Gauss-Algorithmus liefert direkt die Lösung, sofern es eine gibt,
ohne dass man wie in Abschnitt \ref{subsection-durchstosspunkt}
die gefundenen Parameterwerte nochmals einsetzen müsste:
\[
\begin{tabular}{|>{$}c<{$}>{$}c<{$}>{$}c<{$}>{$}c<{$}>{$}c<{$}>{$}c<{$}|>{$}c<{$}|}
\hline
{\color{red}x}&
{\color{red}y}&
{\color{red}z}&
{\color{red}t}&
{\color{red}s}&
{\color{red}l}&\\
\hline
1&0&0&-2&-3& 0&1\\
0&1&0&-2& 3& 0&2\\
0&0&1& 2& 1& 0&1\\
1&0&0& 0& 0&-1&5\\
0&1&0& 0& 0& 0&8\\
0&0&1& 0& 0&-1&3\\
\hline
\end{tabular}
\rightarrow
\begin{tabular}{|>{$}c<{$}>{$}c<{$}>{$}c<{$}>{$}c<{$}>{$}c<{$}>{$}c<{$}|>{$}c<{$}|}
\hline
{\color{red}x}&
{\color{red}y}&
{\color{red}z}&
{\color{red}t}&
{\color{red}s}&
{\color{red}l}&\\
\hline
1&0&0&0&0&0&1\\
0&1&0&0&0&0&8\\
0&0&1&0&0&0&-1\\
0&0&0&1&0&0&1.5\\
0&0&0&0&1&0&-1\\
0&0&0&0&0&1&-4\\
\hline
\end{tabular}
\]
Daraus liest man einerseits den Durchstosspunkt $(1,8,-1)$ ab, andererseits
findet man die Parameterwerte, für die dieser Durchstosspunkt erreicht wird.

\subsubsection{Schnittgerade}
Die Schnittgerade der zwei Ebenen kann wie folgt gefunden werden.
Zunächst beschreiben wir mit der Parameterdarstellung, wie der
Ortsvektor eines beliebigen Punktes $(x,y,z)$ der Ebenen entsteht:
\[
\begin{pmatrix}
\color{red}x\\
\color{red}y\\
\color{red}z\\
\end{pmatrix}
=
\begin{pmatrix}5\\8\\6\end{pmatrix}
+{\color{red}s_0}\begin{pmatrix}4\\6\\7\end{pmatrix}
+{\color{red}t_0}\begin{pmatrix}2\\5\\6\end{pmatrix}
,\qquad
\begin{pmatrix}
\color{red}x\\
\color{red}y\\
\color{red}z\\
\end{pmatrix}
=
\begin{pmatrix}5\\4\\4\end{pmatrix}
+{\color{red}s_1}\begin{pmatrix}6\\6\\7\end{pmatrix}
+{\color{red}t_1}\begin{pmatrix}5\\4\\4\end{pmatrix}
\]
Hier hat man plötzlich 6 Gleichungen für 7 Unbekannte.
Wir erwarten also eine frei wählbare Variable, die dann als Parameter
für die Schnittgerade dienen kann.
Das Gausstableau ist
\[
\begin{tabular}{|>{$}c<{$}>{$}c<{$}>{$}c<{$}>{$}c<{$}>{$}c<{$}>{$}c<{$}>{$}c<{$}|>{$}c<{$}|}
\hline
{\color{red}x}&
{\color{red}y}&
{\color{red}z}&
{\color{red}s_0}&
{\color{red}t_0}&
{\color{red}s_1}&
{\color{red}t_1}&\\
\hline
1&0&0&-4&-2& 0& 0&5\\
0&1&0&-6&-5& 0& 0&8\\
0&0&1&-7&-6& 0& 0&6\\
1&0&0& 0& 0&-6&-5&5\\
0&1&0& 0& 0&-6&-4&4\\
0&0&1& 0& 0&-7&-4&4\\
\hline
\end{tabular}
\rightarrow
\begin{tabular}{|>{$}c<{$}>{$}c<{$}>{$}c<{$}>{$}c<{$}>{$}c<{$}>{$}c<{$}>{$}c<{$}|>{$}c<{$}|}
\hline
{\color{red}x}&
{\color{red}y}&
{\color{red}z}&
{\color{red}s_0}&
{\color{red}t_0}&
{\color{red}s_1}&
{\color{red}t_1}&\\
\hline
1&0&0&0&0&0&         - 14\phantom{.0}&-67\\
0&1&0&0&0&0&         - 13\phantom{.0}&-68\\
0&0&1&0&0&0&         - 14.5          &-80\\
0&0&0&1&0&0&-\phantom{0}5.5          &-26\\
0&0&0&0&1&0&\phantom{-0}4\phantom{.0}&\phantom{-}16\\
0&0&0&0&0&1&-\phantom{0}1.5          &-12\\
\hline
\end{tabular}
\]
Daraus liest man ab, dass es unendlich viele Punkte gibt, die auf beiden
Ebenen liegen, und dass $t_1$ als Parameter für die Geradengleichung
dienen kann.
Ausserdem kann man die Parameterdarstellung der Gerade ablesen:
\[
\begin{pmatrix}-67\\-68\\-80\end{pmatrix}
+t_1
\begin{pmatrix}14\\13\\14.5\end{pmatrix}.
\]





%
% vektorraum.tex
%
% (c) 2018 Prof Dr Andreas Müller
%

\section{Vektorräume \texorpdfstring{$\mathbb R^n$}{R hoch n} und Unterräume}
\rhead{Vektorräume}
\index{Vektorraum@Vektorraum $\mathbb R^n$}
Die Menge $V$ der $n$-dimensionalen Spaltenvektoren bildet was man
einen Vektorraum nennt: Spaltenvektoren können addiert und mit
einem Skalar multipliziert werden.
Die beiden Operationen sind
miteinander verträglich, was durch die Distributivgesetze ausgedrückt
wird.
Intuitiv sagen diese, dass die ``übliche Algebra funktioniert'',
man kann mit diesen Vektoren genau so rechnen, wie man sich das von
der Algebra mit gewöhnlichen Zahlen gewohnt ist, wenigstens solange
man nicht versucht, Vektoren miteinander zu multiplizieren oder zu
dividieren.
Es gibt auch einen Nullvektor, der bei Addition nichts ändert,
also die Funktion der Null übernimmt.

Es gibt aber auch Teilmengen von $\mathbb R^n$, die vergleichbare
Eigenschaften haben.
Typischerweise können diese durch lineare Gleichungen definiert werden.
Die Menge
\[
V=\left\{\left.\begin{pmatrix}x\\y\end{pmatrix}\,\right|\,x=y\right\}
\subset\mathbb R^2
\]
besteht aus Vektoren der Form $\begin{pmatrix}x\\x\end{pmatrix}$,
es ist also ziemlich offensichtlich, dass die Summe und die skalaren
Vielfachen von  Vektoren aus $V$ wieder in $V$ liegen.
$V$ ist
bezüglich der Rechenoperationen abgeschlossen, und auch der Nullvektor
$0\in V$.

\index{Unterraum}
\begin{definition}
Eine Teilmenge $V\subset\mathbb R^n$ heisst Unterraum von $\mathbb R^n$,
wenn für zwei Vektoren
$u,v\in V$ und zwei reelle Zahlen $\lambda,\mu\in\mathbb R$
auch die daraus gebildete Linearkombination
$\lambda u+\mu v\in V$ ist.
\end{definition}

Der kleinste mögliche Unterraum ist $\{0\}$, also der Vektorraum, der nur aus
dem Nullvektor besteht.
Für diesen Raum schreibt man manchmal auch nur $0$.

\subsubsection{Beispiel: Nullraum}
\index{Nullraum}
\index{Kern|see{Nullraum}}
\index{ker|see{Nullraum}}
Sei $A$ eine $m\times n$-Matrix.
Die Menge
\[
U=\{v\in\mathbb R^n\,|\,Av=0\}
\]
ist ein Unterraum, der Nullraum oder Kern von $A$, geschrieben
$\operatorname{ker}A$.
Tatsächlich gilt für $u,v\in\operatorname{ker}A$
\[
A(\lambda v+\mu u)=\lambda Av+\mu Au=0,
\]
also ist $\lambda v+\mu u\in\operatorname{ker}A$.

\begin{satz} Sind $U$ und $V$ Unterräume, dann auch $U\cap V$ und
\[
U+V=\{u+v\,|\,u\in U,v\in V\}.
\]
\end{satz}

Sind $U$ und $V$ Unterräume von $\mathbb R^n$, dann ist auch
$U\cap V$ ein Unterraum.
Dazu ist zu prüfen, dass die Linearkombinationen
von zwei Vektoren $u,v\in U\cap V$ wieder in $U\cap V$ sind.
Da beide Vektoren aus $U$ sind, muss auch jede Linearkombination in $U$ sein.
Dasselbe gilt für $V$, also ist jede Linearkombination in $U\cap V$.

\begin{satz}
Sei $A$ eine $m\times n$-Matrix und $V$ ein Unterraum von $\mathbb R^n$.
Dann ist
\[
AV=\{ Av\,|v\in V\}\subset\mathbb R^m
\]
ein Unterraum von $\mathbb R^m$.
\end{satz}
\begin{proof}[Beweis]
Sind zwei Vektoren $u_1$ und $u_2$ in $AV$, dann gibt es $v_1,v_2\in V$ mit
$u_1=Av_1$ und $u_2=Av_2$.
Also ist
\[
\lambda_1u_1+\lambda_2u_2
=
\lambda_1Av_1+\lambda_2Av_2
=
A(\lambda_1v_1+\lambda_2v_2)\in AV,
\]
also ist $AV$ ein Unterraum.
\end{proof}
Falls $V=\mathbb R^n$ schreibt man auch $A\mathbb R^n=\operatorname{im}A$
und nennt $\operatorname{im}A$ das Bild von $A$.
\index{Bild}
\index{im@$\operatorname{im}A$|see{Bild}}


%
% loesungen.tex
%
% (c) 2018 Prof Dr Andreas Müller, Hochschule Rapperswil
%
\section{Lösungen von linearen Gleichungen}
Die Sprache der linearen Räume erlaubt uns, die Erkenntnisse über
lineare Gleichungssysteme jetzt etwas geometrischer zu formulieren.

Sei $A$ eine $m\times n$-Matrix, $b$ ein $m$-dimensionaler Vektor und $x$
ein Vektor von $n$ Unbekannten.
Dann ist $Ax=b$ ein lineares Gleichungssystem
mit $m$ Gleichungen für $n$ Unbekannte.
Natürlich ist das Gleichungssystem
nur dann lösbar, wenn der Vektor $b$ im Bild der Matrix $A$ drin liegt.

Falls also $b\in\operatorname{im}A$ ist, gibt es auf jeden Fall mindestens
eine Lösung $x\in\mathbb R^n$ mit $Ax=b$.
Kann es auch mehrere geben? Nehmen
wir an, $x$ und $y$ seien beides solche Lösungen, dann ist
\[
\left.
\begin{aligned}
Ax&=b\\
Ay&=b
\end{aligned}
\right\}
\qquad\Rightarrow\qquad
Ax-Ay=A(x-y)=0
\qquad\Rightarrow\qquad
x-y\in\operatorname{ker}A.
\]
Der Nullraum von $A$ gibt also darüber Auskunft, ob ein Gleichungssystem
eine oder mehrere Lösungen hat:
\begin{satz}
Sei $A$ eine $m\times n$-Matrix und $b\in\mathbb R^m$.
Dann gilt:
\begin{enumerate}
\item Das Gleichungssystem hat keine Lösung, wenn $b\not\in \operatorname{im}A$.
\item Falls $b\in\operatorname{im}A$, und $x_p$ eine Lösung des Gleichungssystems
$Ax=b$ ist, dann ist jede andere Lösung von der Form $x=x_p+x_h$, wobei
$x_h\in\operatorname{ker}A$ ist.
\item Das Gleichungssystem hat genau dann nur eine Lösung, wenn $\operatorname{ker}A=0$
\end{enumerate}
\end{satz}
Das Gleichungssystem $Ax=0$ heisst das zu $Ax=b$ gehörige homogene System.
Der Satz sagt also, dass man eine einzige Lösung von $Ax=b$ finden muss,
dies ist das $x_p$, und die Lösungen von $Ax=0$, dass sind die Elemente
des Kerns von $A$, die im Satz mit $x_h$ bezeichnet werden.
Eine Lösung von $Ax=b$ ist dann immer von der Form $x_p+x_h$.
In der Tat bekommt
man 
\[
Ax=A(x_p+x_h)=Ax_p+Ax_h=b+0=b,
\]
also ist jedes solche $x$ tatsächlich eine Lösung.

Wie entscheidet man, ob $b\not\in\operatorname{im}A$? Natürlich mit
Hilfe des Gauss-Algorithmus, der ja klärt, ob das Gleichungssystem
$Ax=b$ eine Lösung hat.
Wie findet man die Vektoren in $ker A$?
Natürlich auch mit dem Gauss-Algorithmus, der ja auch die Lösungsmenge
von $Ax=0$ berechnen kann.



