%
% loesungen.tex
%
% (c) 2018 Prof Dr Andreas Müller, Hochschule Rapperswil
%
\section{Lösungen von linearen Gleichungen}
\rhead{Lösungen von linearen Gleichungen}
Die Sprache der linearen Räume erlaubt uns, die Erkenntnisse über
lineare Gleichungssysteme jetzt etwas geometrischer zu formulieren.

Sei $A$ eine $m\times n$-Matrix, $b$ ein $m$-dimensionaler Vektor und $x$
ein Vektor von $n$ Unbekannten.
Dann ist $Ax=b$ ein lineares Gleichungssystem
mit $m$ Gleichungen für $n$ Unbekannte.
Natürlich ist das Gleichungssystem
nur dann lösbar, wenn der Vektor $b$ im Bild der Matrix $A$ drin liegt.

Falls also $b\in\operatorname{im}A$ ist, gibt es auf jeden Fall mindestens
eine Lösung $x\in\mathbb R^n$ mit $Ax=b$.
Kann es auch mehrere geben? Nehmen
wir an, $x$ und $y$ seien beides solche Lösungen, dann ist
\[
\left.
\begin{aligned}
Ax&=b\\
Ay&=b
\end{aligned}
\right\}
\qquad\Rightarrow\qquad
Ax-Ay=A(x-y)=0
\qquad\Rightarrow\qquad
x-y\in\operatorname{ker}A.
\]
Der Nullraum von $A$ gibt also darüber Auskunft, ob ein Gleichungssystem
eine oder mehrere Lösungen hat:
\begin{satz}
Sei $A$ eine $m\times n$-Matrix und $b\in\mathbb R^m$.
Dann gilt:
\begin{enumerate}
\item Das Gleichungssystem hat keine Lösung, wenn $b\not\in \operatorname{im}A$.
\item Falls $b\in\operatorname{im}A$, und $x_p$ eine Lösung des Gleichungssystems
$Ax=b$ ist, dann ist jede andere Lösung von der Form $x=x_p+x_h$, wobei
$x_h\in\operatorname{ker}A$ ist.
\item Das Gleichungssystem hat genau dann nur eine Lösung, wenn $\operatorname{ker}A=0$
\end{enumerate}
\end{satz}
Das Gleichungssystem $Ax=0$ heisst das zu $Ax=b$ gehörige homogene System.
Der Satz sagt also, dass man eine einzige Lösung von $Ax=b$ finden muss,
dies ist das $x_p$, und die Lösungen von $Ax=0$, dass sind die Elemente
des Kerns von $A$, die im Satz mit $x_h$ bezeichnet werden.
Eine Lösung von $Ax=b$ ist dann immer von der Form $x_p+x_h$.
In der Tat bekommt
man 
\[
Ax=A(x_p+x_h)=Ax_p+Ax_h=b+0=b,
\]
also ist jedes solche $x$ tatsächlich eine Lösung.

Wie entscheidet man, ob $b\not\in\operatorname{im}A$? Natürlich mit
Hilfe des Gauss-Algorithmus, der ja klärt, ob das Gleichungssystem
$Ax=b$ eine Lösung hat.
Wie findet man die Vektoren in $ker A$?
Natürlich auch mit dem Gauss-Algorithmus, der ja auch die Lösungsmenge
von $Ax=0$ berechnen kann.

