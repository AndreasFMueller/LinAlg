%
% schnittpunkt.tex
%
% (c) 2018 Prof Dr Andreas Müller, Hochschule Rapperswil
%
\documentclass[tikz]{standalone}
\usepackage{times}
\usepackage{amsmath}
\usepackage{txfonts}
\usepackage[utf8]{inputenc}
\usepackage{graphics}
\usepackage{color}
\usepackage{pifont}
\usetikzlibrary{arrows,intersections,math,calc}
\begin{document}

\def\punkt#1{
        \fill[color=white] #1 circle[radius=0.08];
        \draw #1 circle[radius=0.08];
}

\begin{tikzpicture}[>=latex,thick]

\coordinate (O) at (0,0);
\coordinate (P) at (-1,5);
\coordinate (Q) at (7,1);
\coordinate (R) at (2,-3);
\coordinate (V) at (2,-0.2);

\coordinate (S) at (1.28571,1.57143);

\draw[->,line width=1.5pt] (O)--(P);
\draw[->,line width=1.5pt] (O)--(Q);

\begin{scope}
\clip (-3,-1) rectangle (11,7);
\draw[color=red] ($(P)-8*(R)$)--($(P)+8*(R)$);
\draw[color=blue] ($(Q)-8*(V)$)--($(Q)+8*(V)$);
\end{scope}

\draw[->,line width=1.5pt] (P)--($(P)+(R)$);
\draw[->,line width=1.5pt] (Q)--($(Q)+(V)$);
\node at ($(P)+0.5*(R)$) [above right] {$\vec{r}$};
\node at ($(Q)+0.5*(V)$) [below] {$\vec{v}$};

\node at ($0.7*(P)$) [left] {$\vec{p}$};
\node at ($0.7*(Q)$) [below] {$\vec{q}$};

\punkt{(O)} \node at (O) [below left] {$O$};
\punkt{(P)} \node at (P) [below left] {$P$};
\punkt{(Q)} \node at (Q) [above right] {$Q$};

\punkt{(S)} \node at (S) [above right] {$S$};

\end{tikzpicture}

\end{document}

