%
% drehung.tex
%
% (c) 2018 Prof Dr Andreas Müller, Hochschule Rapperswil
%
\documentclass[tikz]{standalone}
\usepackage{times}
\usepackage{amsmath}
\usepackage{txfonts}
\usepackage[utf8]{inputenc}
\usepackage{graphics}
\usepackage{color}
\usepackage{pifont}
\usetikzlibrary{arrows,intersections,math,calc}
\begin{document}

\def\punkt#1{
        \fill[color=white] #1 circle[radius=0.08];
        \draw #1 circle[radius=0.08];
}

\def\a{35}
\def\rr{1.3}

\begin{tikzpicture}[>=latex,thick]

\fill[color=gray!10] (0,0) -- (\rr,0) arc (0:\a:\rr) --cycle;
\fill[color=gray!10] (0,0) -- (0,\rr) arc (90:{90+\a}:\rr) --cycle;

\draw[->] (-4.5,0)--(4.5,0) coordinate[label={$x_1$}];
\draw[->] (0,-4.5)--(0,4.5) coordinate[label={right:$x_2$}];

\draw (0,0) circle[radius=3];

\draw[->,line width=1.5pt,color=red] (0,0)--(3,0);
\draw[->,line width=1.5pt,color=blue] (0,0)--(0,3);
\draw[->,line width=1.5pt,color=red] (0,0)--({3*cos(\a)},{3*sin(\a)});
\draw[->,line width=1.5pt,color=blue] (0,0)--({3*cos(90+\a)},{3*sin(90+\a)});

\draw (\rr,0) arc (0:\a:\rr);
\draw (0,\rr) arc (90:{90+\a}:\rr);

\node at ({\a/2}:{0.8*\rr}) {$\alpha$};
\node at ({90+\a/2}:{0.8*\rr}) {$\alpha$};

\node[color=red] at (3,0) [below right] {$\vec{e}_1$};
\node[color=blue] at (0,3) [above right] {$\vec{e}_2$};

\node[color=red] at ({\a}:3) [above right] {$R\vec{e}_1$};
\node[color=blue] at ({90+\a}:3) [above left] {$R\vec{e}_2$};

\draw[line width=0.4pt] ({3*cos(\a)},-0.1)--({3*cos(\a)},{3*sin(\a)});
\node at ({1*3*cos(\a)},0) [below] {$\cos\alpha$};
\node at ({3*cos(\a)+0.05},{0.4*3*sin(\a)}) [left] {$\sin\alpha$};

\draw[line width=0.4pt] ({0.1},{3*cos(\a)})--({-3*sin(\a)},{3*cos(\a)});
\node at (0.1,{1*3*cos(\a)}) [right] {$\cos\alpha$};
\node at ({-0.3*3*sin(\a)},{3*cos(\a)}) [above] {$-\sin\alpha$};

\punkt{(0,0)}
\node at (0,0) [below right] {$O$};

\end{tikzpicture}

\end{document}

