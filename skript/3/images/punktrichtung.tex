%
% punktrichtung.tex
%
% (c) 2018 Prof Dr Andreas Müller, Hochschule Rapperswil
%
\documentclass[tikz,12pt]{standalone}
\usepackage{times}
\usepackage{amsmath}
\usepackage{txfonts}
\usepackage[utf8]{inputenc}
\usepackage{graphics}
\usepackage{color}
\usepackage{pifont}
\usetikzlibrary{arrows,intersections,math,calc}
\begin{document}

\def\punkt#1{
        \fill[color=white] #1 circle[radius=0.08];
        \draw #1 circle[radius=0.08];
}

\begin{tikzpicture}[>=latex,thick]

\coordinate (O) at (0,0);
\coordinate (P0) at (3,5);
\coordinate (R) at (4,1);
\coordinate (P) at ($(P0)+(R)$);

\begin{scope}
\clip(-1,-0.2) rectangle (10,7.1);
\draw[color=red] ($(P0)-3*(R)$)--($(P0)+3*(R)$);
\end{scope}

\draw[->,line width=1.5pt] (O)--(P0);
\draw[->,line width=1.5pt] (P0)--(P);

\punkt{(O)} \node at (O) [above left] {$O$};
\punkt{(P0)} \node at (P0) [above left] {$P_0$};
\punkt{($(P0)+1.3*(R)$)} \node at ($(P0)+1.3*(R)$) [below right] {$P$};

\node at ($(P0)+0.5*(R)$) [above] {$\vec{r}$};
\node at ($0.5*(P0)$) [above left] {$\vec{p}_0$};

\end{tikzpicture}

\end{document}

