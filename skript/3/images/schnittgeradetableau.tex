%
% durchstosspunkttableau.tex
%
% (c) 2018 Prof Dr Andreas Müller, Hochschule Rapperswil
%
\documentclass[tikz]{standalone}
\usepackage{times}
\usepackage{amsmath}
\usepackage{txfonts}
\usepackage[utf8]{inputenc}
\usepackage{graphics}
\usepackage{color}
\usepackage{pifont}
\usepackage{ifthen}
\usepackage{array}
\usepackage{arydshln}
\usetikzlibrary{arrows,intersections,math,calc}
\begin{document}

\def\punkt#1{
        \fill[color=white] #1 circle[radius=0.08];
        \draw #1 circle[radius=0.08];
}

\newboolean{showgrid}
\setboolean{showgrid}{false}

\begin{tikzpicture}[>=latex,thick]

\node at (0,0) {
$
\displaystyle
\begin{tabular}{|>{$}c<{$}>{$}c<{$}>{$}c<{$}>{$}c<{$}>{$}c<{$}>{$}c<{$}>{$}c<{$}|>{$}c<{$}|}
\hline
&\color{red}\vec{p}&&\color{red}t_1&\color{red}s_1&\color{red}t_2&\color{red}s_2&\\
\hline
 & & &          &          &          &          &\\
 &I& &-\vec{u}_1&-\vec{v}_1&          &          &\vec{p}_1\\
 & & &          &          &          &          &\\
 & & &          &          &          &          &\\
 &I& &          &          &-\vec{u}_2&-\vec{v}_2&\vec{p}_2\\
 & & &          &          &          &          &\\
\hline
\end{tabular}
$
};

% Gitter
\ifthenelse{\boolean{showgrid}}{
\draw[step=0.1,line width=0.1pt] (-3,-3) grid (3, 3);
\draw[step=0.5,line width=0.4pt] (-3,-3) grid (3, 3);
\draw (-3,-3) grid (3, 3);
\fill (0,0) circle[radius=0.05];
}{}

\def\t{0.95}
\def\h{1.125}
\def\b{-1.4}

\draw[rounded corners] (-2.9,{\t-\h}) rectangle (-1.7,0.95); % I
\draw[rounded corners] (-1.45,{\t-\h}) rectangle (-0.75,0.95); % u_1
\draw[rounded corners] (-0.5,{\t-\h}) rectangle (0.2,0.95); % v_1
\draw[rounded corners] (2.4,{\t-\h}) rectangle (2.9,0.95); % p_1

\draw[rounded corners] (-2.9,-1.4) rectangle (-1.7,{\b+\h}); % I
\draw[rounded corners] (0.45,-1.4) rectangle (1.15,{\b+\h}); % u_2
\draw[rounded corners] (1.4,-1.4) rectangle (2.1,{\b+\h}); % v_2
\draw[rounded corners] (2.4,-1.4) rectangle (2.9,{\b+\h}); % p_2

\draw[rounded corners, color=red] (-2.9,1.1) rectangle (-1.7,1.45);

\end{tikzpicture}

\end{document}

