%
% streckenvervielfachung.tex
%
% (c) 2018 Prof Dr Andreas Müller, Hochschule Rapperswil
%
\documentclass[tikz,12pt]{standalone}
\usepackage{times}
\usepackage{amsmath}
\usepackage{txfonts}
\usepackage[utf8]{inputenc}
\usepackage{graphics}
\usepackage{color}
\usepackage{pifont}
\usetikzlibrary{arrows,intersections,math,calc}
\begin{document}

\definecolor{gruen}{rgb}{0,0.6,0}

\def\punkt#1{
        \fill[color=white] #1 circle[radius=0.08];
        \draw #1 circle[radius=0.08];
}

\begin{tikzpicture}[>=latex,thick]

\coordinate (O) at (0,0);
\coordinate (A) at (3,1);
\coordinate (B) at ($2*(A)$);
\coordinate (C) at (1,4);
\coordinate (D) at ($(A)+(C)$);

\begin{scope}
\clip (-0.3,-0.3) rectangle (6.3,5.3);

\draw[line width=1.5pt] (O)--(A);
\draw[color=gruen,line width=1.5pt] (C)--(D);
\draw[color=red,line width=1.5pt] (A)--(B);

\draw[color=gruen] ($(O)-(C)$)--($2*(C)$);
\draw[color=gruen] ($(A)-(C)$)--($(A)+2*(C)$);

\draw[color=red] ($2*(C)-(A)$)--($2*(A)-(C)$);
\draw[color=red] ($2*(C)$)--($3*(A)-(C)$);

\end{scope}

\punkt{(O)} \node at (O) [above left] {$O$};
\punkt{(A)} \node at ($(A)+(0,-0.1)$) [above left] {$A$};
\punkt{(B)} \node at (B) [right] {$B$};
\punkt{(C)} \node at (C) [left] {$C$};
\punkt{(D)} \node at (D) [right] {$D$};

\node at ($(C)+0.5*(A)$) [above] {\color{gruen}\ding{182}};
\node at ($1.5*(A)$) [below] {\color{red}\ding{183}};

\node at (10.5,2.5) {%
\begin{minipage}{6.1cm}
\parindent0pt
\raggedright
Gegeben: Strecke $OA$\\[7pt]
{\color{gruen}\ding{182}}\quad%
Wähle $C$ nicht auf der Geraden durch $O$ und $A$ und
konstruiere die Strecke $CD$ mit Hilfe der {\color{gruen}grünen}
Parallelogramm-Konstruktion\\
{\color{red}\ding{183}}\quad%
Konstruiere die Strecke $AB$ mit Hilfe der {\color{red}roten}
Parallelogramm- Konstruktion\\[7pt]
Die Strecke $AB$ ist gleich lang wie $OA$ und liegt auf der
selben Geraden.
\end{minipage}};

\end{tikzpicture}

\end{document}

