%
% checklist.tex -- Checkliste zur Pruefungsvorbereitung
%
% (c) 2009 Prof. Dr. Andreas Mueller, HSR
% $Id: skript.tex,v 1.34 2008/11/02 22:46:16 afm Exp $
%
\documentclass[a4paper,12pt,twocolumn]{article}
\usepackage{geometry}
\geometry{papersize={200mm,280mm},total={180mm,250mm}}
\usepackage{german}
\usepackage{times}
\usepackage{alltt}
\usepackage{verbatim}
\usepackage{fancyhdr}
\usepackage{amsmath}
\usepackage{amssymb}
\usepackage{amsfonts}
\usepackage{amsthm}
\usepackage{textcomp}
\usepackage{graphicx}
\usepackage{array}
\usepackage{ifthen}
\usepackage{multirow}
\usepackage{txfonts}
\usepackage{paralist}
\begin{document}
\title{Pr"ufungsvorbereitungscheckliste:\\ Lineare Algebra}
\date{}
\maketitle
\section{Begriffe}
\begin{compactenum}
\item Gauss-Algorithmus
\item Linearform
\item Spaltenvektor, Zeilenvektor
\item Vektorform eines Gleichungssystems
\item L"osungsmenge
\item Linear abh"angig
\item regul"ar, singul"ar
\item Basis
\item Standardbasisvektor
\item Matrix
\item Transponierte Matrix
\item Symmetrische Matrix
\item Kern und Bild
\item Determinante
\item Sarrus-Formel
\item Entwicklungssatz
\item Minor
\item Matrizenprodukt
\item Inverse Matrix
\item Zeilenoperation, Spaltenoperation
\item Geradengleichung
\item Orthogonale Projektion
\item Skalarprodukt
\item Einheitsvektor
\item Zwischenwinkelformel
\item Vektorprodukt
\item Normale
\item Hessesche Normalform
\item Ebenengleichung
\item Gleichungen f"ur Kreis und Kugel
\item Drehmatrix
\item Basistransformation
\item Zerlegungen: LR, QR, Cholesky
\item Eigenwert, Eigenvektor
\item Charakteristisches Polynom
\item Differenzengleichung
\item Diagonalisierung einer Matrix
\end{compactenum}
\vfill
\section{Fragen}
\begin{compactenum}
\item Wie kann man herausfinden, ob eine Matrix regul"ar oder singul"ar ist (mindestens zwei Methoden)?
\item Wie kann man herausfinden, ob eine Menge von Vektoren linear
abh"angig sind?
\item Beschreiben Sie zwei Methoden zur Berechnung der inversen Matrix.
\item Wie kann man die L"osungsmenge eines linearen Gleichungssystems
beschreiben?
\item Wie findet man den Abstand eines Punktes von einer Geraden im Raum?
\item Wie findet man Abstand eines Punktes von einer Ebene?
\item Wie findet man den Abstand zweier windschiefer Geraden?
\item Beschreiben Sie den L"osungsalgorithmus f"ur das Eigenwertproblem
einer allgemeinen $n\times n$-Matrix.
\item Wie findet man die Schnittgerade zweier Ebenen?
\item Wie kann man aus drei beliebigen Vektoren drei Einheitsvektoren
bekommen, die ausserdem senkrecht aufeinander stehen?
Unter welchen Voraussetzungen ist dies "uberhaupt m"oglich?
\item Beschreiben Sie eine L"osungsmethode f"ur eine Differenzengleichung.
\end{compactenum}
\end{document}
