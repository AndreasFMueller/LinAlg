%
% checklist.tex -- Checkliste zur Pruefungsvorbereitung
%
% (c) 2009 Prof. Dr. Andreas Mueller, HSR
% $Id: skript.tex,v 1.34 2008/11/02 22:46:16 afm Exp $
%
\documentclass[a4paper,12pt,twocolumn]{article}
\usepackage{geometry}
\geometry{papersize={200mm,280mm},total={180mm,250mm}}
\usepackage{german}
\usepackage{times}
\usepackage{alltt}
\usepackage{verbatim}
\usepackage{fancyhdr}
\usepackage{amsmath}
\usepackage{amssymb}
\usepackage{amsfonts}
\usepackage{amsthm}
\usepackage{textcomp}
\usepackage{graphicx}
\usepackage{array}
\usepackage{ifthen}
\usepackage{multirow}
\usepackage{txfonts}
\usepackage{paralist}
\begin{document}
\title{Prüfungsvorbereitungscheckliste:\\ Lineare Algebra}
\date{}
\maketitle
\section{Begriffe}
\begin{compactenum}
\item Gauss-Algorithmus
\item Linearform
\item Spaltenvektor, Zeilenvektor
\item Vektorform eines Gleichungssystems
\item Lösungsmenge
\item Linear abhängig
\item regulär, singulär
\item Basis
\item Standardbasisvektor
\item Matrix
\item Transponierte Matrix
\item Symmetrische Matrix
\item Kern und Bild
\item Determinante
\item Sarrus-Formel
\item Entwicklungssatz
\item Cramersche Regel
\item Minor
\item Matrizenprodukt
\item Inverse Matrix
\item Zeilenoperation, Spaltenoperation
\item Gleichungen für Geraden und Ebenen
\item Parameterdarstellung von Geraden und Ebenen
\item Orthogonale Projektion
\item Skalarprodukt
\item Einheitsvektor
\item Zwischenwinkelformel
\item Vektorprodukt
\item Normale
\item Normalenform von Ebene und Gerade
\item Hessesche Normalform
\item Ebenengleichung
\item Gleichungen für Kreis und Kugel
\item Basistransformation
\item Orthogonale Matrix
\item Drehmatrix
\item Spur
\item Zerlegungen: LR, QR, Cholesky
\item Eigenwert, Eigenvektor
\item Charakteristisches Polynom
\item Differenzengleichung
\item Diagonalisierung einer Matrix
\end{compactenum}
\vfill
\section{Fragen}
\begin{compactenum}
\item Wie kann man herausfinden, ob eine Matrix regulär oder singulär ist (mindestens zwei Methoden)?
\item Wie kann man herausfinden, ob eine Menge von Vektoren linear
abhängig sind?
\item Beschreiben Sie zwei Methoden zur Berechnung der inversen Matrix.
\item Wie kann man die Lösungsmenge eines linearen Gleichungssystems
beschreiben?
\item Wie findet man den Abstand eines Punktes von einer Geraden im Raum?
\item Wie fidnet man die Schnittmenge zweier Teilmengen von $\mathbb R^n$, die
durch Gleichungen beschrieben sind?
\item Wie findet man Abstand eines Punktes von einer Ebene?
\item Wie findet man den Abstand zweier windschiefer Geraden?
\item Beschreiben Sie den Lösungsalgorithmus für das Eigenwertproblem
einer allgemeinen $n\times n$-Matrix.
\item Wie findet man die Schnittgerade zweier Ebenen?
\item Welche gegenseitigen Lagen von Geraden im Raum sind möglich, und
wie berechnet man in allen diesen Fällen den Abstand?
\item Wie kann man aus drei beliebigen Vektoren drei Einheitsvektoren
bekommen, die ausserdem senkrecht aufeinander stehen?
Unter welchen Voraussetzungen ist dies überhaupt möglich?
\item Wie kann man die Transformationsmatrix zwischen zwei verschiedenen
Basen finden?
\item Wie wird eine Abbildungsmatrix in ein neues Koordinatensystem
transformiert?
\item Beschreiben Sie eine Lösungsmethode für eine Differenzengleichung.
\end{compactenum}
\end{document}
