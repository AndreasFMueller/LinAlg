%
% uebungen.tex
%
% (c) 2025 Prof Dr Andreas Müller
%
\newenvironment{beispiel}[1][Beispiel]{%
\begin{proof}[#1]%
\renewcommand{\qedsymbol}{$\bigcirc$}
}{\end{proof}}
% labels
\makeatletter
\newcommand{\customlabel}[2]{%
\protected@write \@auxout {}{\string \newlabel {#1}{{#2}{}{}{}{}}}}
\makeatother
\input{gausslabels.tex}
\def\gaussurl#1{
\edef\glnumber{\getrefnumber{#1}}
\url{https://linalg.ch/gauss/?id=\glnumber}}
\def\jacobiurl#1{
\edef\glnumber{\getrefnumber{#1}}
\url{https://linalg.ch/jacobi/?id=\glnumber}}
%
%%%%%%%%%%%%%%%%%%%%%%%
%% Copyleft
%% Walter A. Kehowski
%% Department of Mathematics
%% Glendale Community College
%% walter.kehowski@gcmail.maricopa.edu
%% \begin{linsys}{2}
%% -x & + & 4y & = & 8\\
%% -3x & - & 2y & = & 6
%% \end{linsys}
%%%%%%%%%%%%%%%%%%%%%%%
%\makeatletter
%% math-mode column types ------------------
\newcolumntype{\linsysR}{>{$}r<{$}}
\newcolumntype{\linsysL}{>{$}l<{$}}
\newcolumntype{\linsysC}{>{$}c<{$}}
\newenvironment{linsys}[1]{%
\begin{tabular}{*{#1}{\linsysR@{\;}\linsysC}@{\;}\linsysR}}%
{\end{tabular}}
%\makeatother
\endinput

\newcounter{uebungsaufgabe}
\newboolean{loesungen}
% environment fuer uebungsaufgaben
\newenvironment{uebungsaufgaben}{
\begin{list}{\arabic{uebungsaufgabe}.}
  {\usecounter{uebungsaufgabe}
  \setlength{\labelwidth}{2cm}
  \setlength{\leftmargin}{0pt}
  \setlength{\labelsep}{5mm}
  \setlength{\rightmargin}{0pt}
  \setlength{\itemindent}{0pt}
}}{\end{list}\vfill\pagebreak}
% Teilaufgaben
\newenvironment{teilaufgaben}{
\begin{enumerate}
\renewcommand{\labelenumi}{\alph{enumi})}
}{\end{enumerate}}
% Loesung
\NewEnviron{loesung}{%
\begin{proof}[Lösung]%
\renewcommand{\qedsymbol}{$\bigcirc$}
\BODY
\end{proof}}
\NewEnviron{diskussion}{
\BODY
\bigskip
}
\def\keineloesungen{%
\RenewEnviron{loesung}{\relax}
\RenewEnviron{diskussion}{\relax}
}
% Hinweis
\newenvironment{hinweis}{%
\renewcommand{\qedsymbol}{}
\begin{proof}[Hinweis]}{\end{proof}}
% Aufgabe aus der Sammlung wiedergeben
\newboolean{showthema}
\setboolean{showthema}{false}
\newboolean{themastatus}
\setboolean{themastatus}{false}
\newcounter{problemcounter}[chapter]
\def\aufgabepath{./}
\def\ainput#1{\input\aufgabepath/#1}
\def\verbatimainput#1{\expandafter\verbatiminput{\aufgabepath/#1}}
\def\aufgabetoplevel#1{%
\expandafter\def\expandafter\inputpath{#1}%
\let\aufgabepath=\inputpath
}
\def\includeagraphics[#1]#2{\expandafter\includegraphics[#1]{\aufgabepath#2}}
% \aufgabe
\newcommand{\aufgabe}[1]{%
\StrRemoveBraces{#1}[\FirstChar]%
\StrChar{\FirstChar}{1}[\FirstChar]%
  \expandafter\def\csname themalist\endcsname{}
  \setboolean{themastatus}{false}
  \refstepcounter{problemcounter}%
  \label{#1}
  \bigskip{\parindent0pt\strut}\hbox{\bf\theproblemcounter. }%
  \marginpar{\raggedright\tiny #1}%
  \expandafter\def\csname currentaufgabe\endcsname{#1}%
  \expandafter\def\csname aufgabepath\endcsname{\inputpath/\FirstChar/#1/}%
  \expandafter\input{\inputpath\FirstChar/#1.tex}
  %\medskip
  \ifthenelse{\boolean{themastatus}}{
   \ifthenelse{\boolean{showthema}}{
    \parindent 0pt
    {\sc Thema:} {\small \themalist.}}{%
    }{}
  }
  \bigskip

}
\renewcommand\theproblemcounter{\thechapter.\arabic{problemcounter}}
% Bewertung
\NewEnviron{bewertung}{\footnotesize
\renewcommand{\qedsymbol}{}
\begin{proof}[Bewertung]
\BODY
\end{proof}}
% oft benutzte Macros
\def\blank{\text{\textvisiblespace}}
%
% macros fuer den thema-Index
%
\newcommand{\openthemaindex}{%
  \newwrite\themaindex
  \immediate\openout\themaindex=thema.tix
}
%
\newcommand{\closethemaindex}{\immediate\closeout\themaindex}
%
% first argument must be ascii version of the link
\def\themalink#1#2{\hyperref[thema:#1]{#2}}
\def\themaL#1#2{%
  \ifthenelse{\boolean{themastatus}}{%
    \xappto{\themalist}{, \noexpand\themalink{#1}{#2}}
  }{%
    \xdef\themalist{\noexpand\themalink{#1}{#2}}
    \setboolean{themastatus}{true}
  }
  \immediate\write\themaindex%
  {{\thechapter}{#1}{#2}{\arabic{problemcounter}}{\thechapter.\arabic{problemcounter}}{\currentaufgabe}}%
}
\def\themaS#1{\themaL{#1}{#1}}
\def\thema#1{\themaL{#1}{#1}}
% Thema-Information anzeigen
\def\themasection#1#2{ \item[#2]\label{thema:#1} }
\newcommand{\printthemata}{
  \IfFileExists{./thema.tex}{
    \chapter*{Aufgaben nach Themen}
    \addcontentsline{toc}{chapter}{Themenindex}
    \begin{description}
    \input{thema.tex}
    \end{description}
  }{}
}
\definecolor{darkgreen}{rgb}{0,0.6,0}
\def\transpose#1{\prescript{t}{}{#1}}
%
% macros for gauss operations, they all have 4 arguments
%
% #1  width (cm)
% #2  height (cm)
% #3  offset-x (units required)
% #4  offset-y (units required)
%
\def\pivotoperationr{0.2}
% a rounded rectangle around the pivot element
\def\pivotoperation#1#2#3#4{
\smash{\rlap{\raisebox{#4}{\hspace{#3}\begin{tikzpicture}[thick]
\def\p{
        ({#1/2},0)
        -- +(0,{(#2/2)-\pivotoperationr}) arc (0:90:\pivotoperationr)
        -- +({-(#1)+2*\pivotoperationr},0) arc (90:180:\pivotoperationr)
        -- +(0,{-(#2)+2*\pivotoperationr}) arc(180:270:\pivotoperationr)
        -- +({#1-2*\pivotoperationr},0) arc(270:360:\pivotoperationr)
        -- cycle
}
\fill[color=red!20] \p;
\draw[color=red] \p;
\end{tikzpicture}}}}}
% a circle around the pivot element (this operation has only three
% arguments, width and height are the same, the diameter of the
% circle, so the first two arguments are replaced by one
\def\pivotoperationcircle#1#2#3{
\smash{\rlap{\raisebox{#3}{\hspace{#2}\begin{tikzpicture}[thick]
\def\p{ ({#1/2},0) arc (0:360:{#1/2}) -- cycle
}
\fill[color=red!20] \p;
\draw[color=red] \p;
\end{tikzpicture}}}}}
% forwardreduction
\def\forwardreduction#1#2#3#4{
\smash{\rlap{\hspace*{#3}\raisebox{#4}{\begin{tikzpicture}[thick]
\def\p{
        (#1,-0.25)
        -- +(0,{#2-\pivotoperationr}) arc(0:90:\pivotoperationr)
        -- +({-#1+2*\pivotoperationr},0) arc (90:180:\pivotoperationr)
        -- +(0,{-#2+\pivotoperationr)})
}
\fill[color=blue!20] \p -- cycle;
\draw[color=blue] \p;
\end{tikzpicture}}}}
}
% backwardsubstitution
\def\backwardsubstitution#1#2#3#4{
\smash{\rlap{\hspace*{#3}\raisebox{#4}{\begin{tikzpicture}[thick]
\def\p{
        ({#1},0.25)
        -- +(0,{-#2+\pivotoperationr}) arc(0:-90:\pivotoperationr)
        -- +({-#1+2*\pivotoperationr},0) arc(-90:-180:\pivotoperationr)
        -- +(0,{#2-\pivotoperationr})
}
\fill[color=blue!20] \p -- cycle;
\draw[color=blue] \p;
\end{tikzpicture}}}}
}

