%
% content.tex
%
% (c) 2025 Prof Dr Andreas Müller
%
\chapter{Lineare Gleichungssysteme}
\lhead{Kapitel \thechapter}
\rhead{Lineare Gleichungssysteme}
\input{1.tex}
\chapter{Determinanten}
\lhead{Kapitel \thechapter}
\rhead{Determinanten}
\input{2.tex}
\chapter{Affine Vektorgeometrie}
\lhead{Kapitel \thechapter}
\rhead{Affine Vektorgeometrie}
\input{3.tex}
\chapter{Orthogonalität}
\lhead{Kapitel \thechapter}
\rhead{Orthogonalität}
\input{4.tex}
\chapter{Orientierung}
\lhead{Kapitel \thechapter}
\rhead{Orientierung}
\input{5.tex}
\chapter{Eigenschaften linearer Abbildungen}
\lhead{Kapitel \thechapter}
\rhead{Eigenschaften linearer Abbildungen}
\input{6.tex}
\chapter{Matrixzerlegungen}
\lhead{Kapitel \thechapter}
\rhead{Matrixzerlegungen}
\input{7.tex}
\chapter{Eigenwerte und Eigenvektoren}
\lhead{Kapitel \thechapter}
\rhead{Eigenwerte und Eigenvektoren}
\input{8.tex}
\chapter{Octave}
\lhead{Kapitel \thechapter}
\rhead{Octave}
Die Aufgaben in diesem Kapitel sind als Lernaufgaben gedacht, mit denen
man sich in die Benutzung des Programms Octave einführen lassen kann.

\bigskip
\input{o.tex}
\closethemaindex
\printthemata
\input aufgabensammlung.ind
