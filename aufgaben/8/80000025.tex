Ist die Matrix
\[
A=\begin{pmatrix}
4&-24\\
8& 32
\end{pmatrix}
\]
diagonalisierbar?
Wenn ja, geben Sie eine Basis an, in der $A$ diagonal ist.

\thema{Eigenwerte}
\thema{Eigenvektoren}
\thema{charakteristisches Polynom}
\thema{diagonalisierbar}

\begin{loesung}
Wir suchen eine Basis aus Eigenvektoren, dazu berechnen wir als erstes
die 
\[
\chi_A(\lambda)
=
\det(A-\lambda I)
=
\left|\,\begin{matrix}
4-\lambda&-24\\
8&32-\lambda
\end{matrix}\,\right|
=(4-\lambda)(32-\lambda)+8\cdot 24=\lambda^2-36\lambda+320
\]
Die Nullstellen finden wir mit der Lösungs-Formel für die quadratishe
Gleichung
\[
\lambda_{\pm}
=
\frac{36}{2}\pm\sqrt{\biggl(\frac{36}{2}\biggr)^2-320}
=
18 \pm\sqrt{18^2-320}
=
18 \pm\sqrt{324-320}
=
18\pm 2
=\begin{cases}20\\16\end{cases}
\]
Für jeden Eigenwert müssen wir jetzt einen Eigenvektor finden. 
Für $\lambda=\lambda_+=20$ finden wir mit dem Gauss-Algorithmus
\begin{align*}
\begin{tabular}{|>{$}c<{$}>{$}c<{$}|}
\hline
-16&-24\\
  8& 12\\
\hline
\end{tabular}
\rightarrow
\begin{tabular}{|>{$}c<{$}>{$}c<{$}|}
\hline
  1&\frac32\\
  0& 0\\
\hline
\end{tabular}
\qquad
\Rightarrow
\qquad
v_+=\begin{pmatrix}-3\\2 \end{pmatrix}
\end{align*}
dabei haben wir aus rein kosmetischen Gründen $2$ für den Wert der
frei wählbaren Variable gewählt, um Brüche im Resultat zu vermeiden.

Für $\lambda=\lambda_-=16$ folgt analog
\begin{align*}
\begin{tabular}{|>{$}c<{$}>{$}c<{$}|}
\hline
-12&-24\\
  8& 16\\
\hline
\end{tabular}
\rightarrow
\begin{tabular}{|>{$}c<{$}>{$}c<{$}|}
\hline
  1& 2\\
  0& 0\\
\hline
\end{tabular}
\qquad
\Rightarrow
\qquad
v_-=\begin{pmatrix}-2\\1 \end{pmatrix}
\end{align*}
Hier haben wir $1$ für den Wert der frei wählbaren Variable genommen.

Zur Kontrolle berechnen wir $Av$:
\begin{align*}
Av_+
&=
\begin{pmatrix}4&-24\\8&32 \end{pmatrix}
\begin{pmatrix}-3\\2 \end{pmatrix}
=
\begin{pmatrix}
-60\\40
\end{pmatrix}
=
20
\begin{pmatrix}-3\\2 \end{pmatrix}
=
\lambda_+v_+,
\\
Av_-
&=
\begin{pmatrix}4&-24\\8&32 \end{pmatrix}
\begin{pmatrix}-2\\1 \end{pmatrix}
=
\begin{pmatrix}
-32\\16
\end{pmatrix}
=
16\begin{pmatrix}-2\\1\end{pmatrix}
=
\lambda_-v_-.
\end{align*}

In der Basis 
\[
B=\{v_+,v_-\}
=
\biggl\{
\begin{pmatrix}-3\\2 \end{pmatrix},
\begin{pmatrix}-2\\1 \end{pmatrix}
\biggr\}
\]
wird die Matrix $A$ diagonal.
\end{loesung}

\begin{bewertung}
Ansatz charakteristisches Polynom ({\bf X}) 1 Punkt,
ausmultipiziertes Polynome ({\bf P}) 1 Punkt,
Nullstellen ({\bf N}) 1 Punkt,
Berechnung der Eigenvektoren ($\textbf{E}_1$ und $\textbf{E}_2$) je 1 Punkt,
Basis ({\bf B}) 1 Punkt.
\end{bewertung}

