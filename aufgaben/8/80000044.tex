Ist die Matrix
\[
A = \begin{pmatrix*}[r]
  -44 &    10 \\
 -196 &    45 
\end{pmatrix*}
\]
diagonalisierbar? Wenn ja, finden Sie eine Basis, in
der $A$ diagonal wird.

\begin{loesung}
Wir bestimmen zunächst das charakteristische Polynom
mit Hilfe der Determinante:
\[
\det (A-\lambda I)
=
\left|\,\begin{matrix*}[r]
-44 - \lambda & 10\\
-196 & 45 - \lambda
\end{matrix*}\,\right|
=
(-44-\lambda)(45-\lambda)-(10)(-196)
=
\lambda^2 - 1 \lambda - 20
=
(\lambda + 4)(\lambda - 5).
\]
Daraus kann man die Eigenwerte $\lambda_1=-4$ und $\lambda_2=5$ ablesen.
Für jeden dieser Eigenwerte ist jetzt der Gauss-Algorithmus
durchzuführen, dazu bilden wir die Tableaux
\begin{align*}
\lambda_1&=-4:
&
\begin{tabular}{|>{$}r<{$}>{$}r<{$}|>{$}r<{$}|}
\hline
   -40 &    10 & 0 \\
  -196 &    49 & 0 \\
\hline
\end{tabular}
&\to
\begin{tabular}{|>{$}r<{$}>{$}r<{$}|>{$}r<{$}|}
\hline
 1 & \frac{10}{-40} & 0 \\
 0 & 0 & 0 \\
\hline
\end{tabular}
&&\Rightarrow
&v_1&=
\begin{pmatrix*}[r]
-10\\-40
\end{pmatrix*},
\\
\lambda_2&=5:
&
\begin{tabular}{|>{$}r<{$}>{$}r<{$}|>{$}r<{$}|}
\hline
   -49 &    10 & 0 \\
  -196 &    40 & 0 \\
\hline
\end{tabular}
&\to
\begin{tabular}{|>{$}r<{$}>{$}r<{$}|>{$}r<{$}|}
\hline
 1 & \frac{10}{-49} & 0 \\
 0 & 0 & 0 \\
\hline
\end{tabular}
&&\Rightarrow
&v_2&=
\begin{pmatrix*}[r]
-10\\-49
\end{pmatrix*}.
\end{align*}
Die Vektoren $v_1$ und $v_2$ sind linear unabhängige Eigenvektoren,
somit können sie als Basis verwendet
werden, in der $A$ diagonal wird.
\end{loesung}

\begin{bewertung}
Charakteristisches Polynom ({\bf X}) 2 Punkte,
Eigenwerte ({\bf E}) 1 Punkt,
Gauss-Algorithmus und Eigenvektoren ({\bf G}) 2 Punkte (je ein Punkt
für jeden Eigenvektor),
Diagonalisierbarkeit ({\bf D}) 1 Punkt.
\end{bewertung}



