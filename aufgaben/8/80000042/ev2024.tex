Ist die Matrix
\[
A = \begin{pmatrix*}[r]
  -12 &    28 \\
   -2 &     3 
\end{pmatrix*}
\]
diagonalisierbar? Wenn ja, finden Sie eine Basis, in
der $A$ diagonal wird.

\begin{loesung}
Wir bestimmen zunächst das charakteristische Polynom
mit Hilfe der Determinante:
\[
\det (A-\lambda I)
=
\left|\,\begin{matrix*}[r]
-12 - \lambda & 28\\
-2 & 3 - \lambda
\end{matrix*}\,\right|
=
(-12-\lambda)(3-\lambda)-(28)(-2)
=
\lambda^2 + 9 \lambda + 20
=
(\lambda + 5)(\lambda + 4).
\]
Daraus kann man die Eigenwerte $\lambda_1=-5$ und $\lambda_2=-4$ ablesen.
Für jeden dieser Eigenwerte ist jetzt der Gauss-Algorithmus
durchzuführen, dazu bilden wir die Tableaux
\begin{align*}
\lambda_1&=-5:
&
\begin{tabular}{|>{$}r<{$}>{$}r<{$}|>{$}r<{$}|}
\hline
    -7 &    28 & 0 \\
    -2 &     8 & 0 \\
\hline
\end{tabular}
&\to
\begin{tabular}{|>{$}r<{$}>{$}r<{$}|>{$}r<{$}|}
\hline
 1 & \frac{28}{-7} & 0 \\
 0 & 0 & 0 \\
\hline
\end{tabular}
&&\Rightarrow
&v_1&=
\begin{pmatrix*}[r]
-28\\-7
\end{pmatrix*},
\\
\lambda_2&=-4:
&
\begin{tabular}{|>{$}r<{$}>{$}r<{$}|>{$}r<{$}|}
\hline
    -8 &    28 & 0 \\
    -2 &     7 & 0 \\
\hline
\end{tabular}
&\to
\begin{tabular}{|>{$}r<{$}>{$}r<{$}|>{$}r<{$}|}
\hline
 1 & \frac{28}{-8} & 0 \\
 0 & 0 & 0 \\
\hline
\end{tabular}
&&\Rightarrow
&v_2&=
\begin{pmatrix*}[r]
-28\\-8
\end{pmatrix*}.
\end{align*}
Die Vektoren $v_1$ und $v_2$ sind linear unabhängige Eigenvektoren,
somit können sie als Basis verwendet
werden, in der $A$ diagonal wird.
\end{loesung}

\begin{bewertung}
Charakteristisches Polynom ({\bf X}) 2 Punkte,
Eigenwerte ({\bf E}) 1 Punkt,
Gauss-Algorithmus und Eigenvektoren ({\bf G}) 2 Punkte (je ein Punkt
für jeden Eigenvektor,
Diagonalisierbarkeit ({\bf D}) 1 Punkt.
\end{bewertung}



