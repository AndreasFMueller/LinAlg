Finden Sie eine Formel für die Zahlen $x_n$, die durch die Rekursionsformel
\[
x_{n+1}=2x_n+x_{n-1}-2x_{n-2}
\]
und die Anfangswerte
\[
\begin{aligned}
x_0&=-2,& x_1&=1,& x_2=1
\end{aligned}
\]
definiert sind.

\thema{Rekursionsgleichung}
\thema{Eigenwerte}
\thema{Eigenvektoren}
\thema{charakteristisches Polynom}

\begin{loesung}
Mit der Matrix 
\[
A
=
\begin{pmatrix}
2&1&-2\\
1&0& 0\\
0&1& 0
\end{pmatrix}
\]
kann die Rekursionsgleichung als
\[
\begin{pmatrix}
x_{n+1}\\
x_{n}\\
x_{n-1}
\end{pmatrix}
=
A
\begin{pmatrix}
x_{n}\\
x_{n-1}\\
x_{n-2}
\end{pmatrix}
\]
Wir bestimmen Eigenwerte und Eigenvektoren von $A$, dazu muss zunächst
das charakteristische Polynom ermittelt werden.
Es gilt
\begin{align*}
\chi_A(\lambda)
&=
\left|\begin{matrix}
2-\lambda&1&-2\\
1&-\lambda&0\\
0&1&-\lambda
\end{matrix}\right|
=
(2-\lambda)\left|\begin{matrix}-\lambda&0\\1&-\lambda\end{matrix}\right|
-\left|\begin{matrix}1&-2\\1&-\lambda\end{matrix}\right|
=
\lambda^2(2-\lambda)-(-\lambda+2)
\\
&=
-\lambda^3+2\lambda^2+\lambda-2
=
(\lambda^2-1)(2-\lambda)
=
(\lambda+1)(\lambda -1)(2-\lambda)
\end{align*}
Daraus liest man ab, dass die Eigenwerte $\lambda_1=-1$, $\lambda_2=1$ und
$\lambda_3=2$ sind.

Wir müssen jetzt für jeden Eigenwert einen Eigenvektor finden.
Für $\lambda=\lambda_1=-1$ gilt
\begin{align*}
\begin{tabular}{|>{$}c<{$}>{$}c<{$}>{$}c<{$}|}
\hline
2-\lambda_1&1         &-2        \\
1          &-\lambda_1& 0        \\
0          &1         &-\lambda_1\\
\hline
\end{tabular}
&=
\begin{tabular}{|>{$}c<{$}>{$}c<{$}>{$}c<{$}|}
\hline
3&1&-2\\
1&1& 0\\
0&1& 1\\
\hline
\end{tabular}
\rightarrow
\begin{tabular}{|>{$}c<{$}>{$}c<{$}>{$}c<{$}|}
\hline
1&\frac13&-\frac23\\
0&\frac23& \frac23\\
0&      1&       1\\
\hline
\end{tabular}
\rightarrow
\begin{tabular}{|>{$}c<{$}>{$}c<{$}>{$}c<{$}|}
\hline
1&\frac13&-\frac23\\
0&      1&       1\\
0&      0&       0\\
\hline
\end{tabular}
\rightarrow
\begin{tabular}{|>{$}c<{$}>{$}c<{$}>{$}c<{$}|}
\hline
1& 0&-1\\
0& 1& 1\\
0& 0& 0\\
\hline
\end{tabular}
\end{align*}
Daraus kann man die Eigenvektor $v_1$ zum Eigenwert $\lambda_1=-1$ ablesen:
\[
v_1=\begin{pmatrix}1\\-1\\1\end{pmatrix}.
\]
Für $\lambda=\lambda_2=1$ wenden wir den Gauss-Algorithmus auf das
Tableau
\begin{align*}
\begin{tabular}{|>{$}c<{$}>{$}c<{$}>{$}c<{$}|}
\hline
2-\lambda_2&1         &-2        \\
1          &-\lambda_2& 0        \\
0          &1         &-\lambda_2\\
\hline
\end{tabular}
&=
\begin{tabular}{|>{$}c<{$}>{$}c<{$}>{$}c<{$}|}
\hline
1 & 1&-2\\
1 &-1& 0\\
0 & 1&-1\\
\hline
\end{tabular}
\rightarrow
\begin{tabular}{|>{$}c<{$}>{$}c<{$}>{$}c<{$}|}
\hline
1 & 1&-2\\
0 &-2& 2\\
0 & 1&-1\\
\hline
\end{tabular}
\rightarrow
\begin{tabular}{|>{$}c<{$}>{$}c<{$}>{$}c<{$}|}
\hline
1 & 1&-2\\
0 & 1&-1\\
0 & 0& 0\\
\hline
\end{tabular}
\rightarrow
\begin{tabular}{|>{$}c<{$}>{$}c<{$}>{$}c<{$}|}
\hline
1 & 0&-1\\
0 & 1&-1\\
0 & 0& 0\\
\hline
\end{tabular}
\end{align*}
an. Der Eigenvektor $v_2$ ist daher
\[
v_2=\begin{pmatrix}
1\\1\\1
\end{pmatrix}.
\]
Schliesslich für $\lambda=\lambda_3=2$:
\begin{align*}
\begin{tabular}{|>{$}c<{$}>{$}c<{$}>{$}c<{$}|}
\hline
2-\lambda_3&1         &-2        \\
1          &-\lambda_3& 0        \\
0          &1         &-\lambda_3\\
\hline
\end{tabular}
&=
\begin{tabular}{|>{$}c<{$}>{$}c<{$}>{$}c<{$}|}
\hline
0 & 1&-2\\
1 &-2& 0\\
0 & 1&-2\\
\hline
\end{tabular}
\rightarrow
\begin{tabular}{|>{$}c<{$}>{$}c<{$}>{$}c<{$}|}
\hline
1 &-2& 0\\
0 & 1&-2\\
0 & 0& 0\\
\hline
\end{tabular}
\rightarrow
\begin{tabular}{|>{$}c<{$}>{$}c<{$}>{$}c<{$}|}
\hline
1 & 0&-4\\
0 & 1&-2\\
0 & 0& 0\\
\hline
\end{tabular}
\end{align*}
Der zugehörige Eigenvektor ist
\[
v_3=\begin{pmatrix}4\\2\\1 \end{pmatrix}.
\]

Für die Lösung der Rekursionsgleichung brauchen wir die
Transformationsmatrizen, welche auf die Eigenbasis transformieren.
Diese sind 
\[
T^{-1}
=
\begin{pmatrix}
 1&1&4\\
-1&1&2\\
 1&1&1
\end{pmatrix}
\qquad\text{und}\qquad
T
=
\begin{pmatrix}
 \frac16&-\frac12& \frac13\\
-\frac12& \frac12&1      \\
 \frac13&   0    &-\frac13
\end{pmatrix}
\]
Man kann durch Nachrechnen kontrollieren, dass
\[
A'
=
TAT^{-1}
=
\begin{pmatrix}
 \frac16&-\frac12& \frac13\\
-\frac12& \frac12&1      \\
 \frac13&   0    &-\frac13
\end{pmatrix}
\begin{pmatrix}
2&1&-2\\
1&0& 0\\
0&1& 0
\end{pmatrix}
\begin{pmatrix}
 1&1&4\\
-1&1&2\\
 1&1&1
\end{pmatrix}
=
\begin{pmatrix}
-1& 0& 0\\
 0& 1& 0\\
 0& 0& 2
\end{pmatrix}
\]
tatsächlich eine Diagonalmatrix ist.

Die Lösung der Rekursionsgleichung findet man jetzt mit Hilfe von $A^k$
aus dem Anfangsvektor
\begin{align*}
A^kv_0
=
T^{-1}A'^kTv_0
&=
\begin{pmatrix}
 1&1&4\\
-1&1&2\\
 1&1&1
\end{pmatrix}
\begin{pmatrix}
(-1)^k& 0& 0\\
     0& 1& 0\\
     0& 0& 2^k
\end{pmatrix}
\begin{pmatrix}
 \frac16&-\frac12& \frac13\\
-\frac12& \frac12&1      \\
 \frac13&   0    &-\frac13
\end{pmatrix}
\begin{pmatrix}
1\\1\\-2
\end{pmatrix}
\\
&=
\begin{pmatrix}
 1&1&4\\
-1&1&2\\
 1&1&1
\end{pmatrix}
\begin{pmatrix}
(-1)^k& 0& 0\\
     0& 1& 0\\
     0& 0& 2^k
\end{pmatrix}
\begin{pmatrix}
-1\\-2\\1
\end{pmatrix}
\\
&=
\begin{pmatrix}
 1&1&4\\
-1&1&2\\
 1&1&1
\end{pmatrix}
\begin{pmatrix}
-(-1)^k\\
-2\\
2^k
\end{pmatrix}
=
\begin{pmatrix}
\dots\\
\dots\\
-(-1)^k
-2
+
2^k
\end{pmatrix}.
\end{align*}
Daraus können wir jetzt die Lösung ablesen:
\[
x_n
=
f(n)
=
-\lambda_1^n
-2 \lambda_2^n
+\lambda_3^n
=-(-1)^n-2+2^n.
\]
Wir kontrollieren die Formel durch die Berechnung der ersten paar Werte
\begin{center}
\begin{tabular}{|>{$}r<{$}|>{$}r<{$}|>{$}r<{$}|}
\hline
 n& x_n&f(n)\\
\hline
 0&  -2&   2\\
 1&   1&   1\\
 2&   1&   1\\
 3&   7&   7\\
 4&  13&  13\\
 5&  31&  31\\
 6&  61&  61\\
 7& 127& 127\\
 8& 253& 253\\
 9& 511& 511\\
10&1021&1021\\
\hline
\end{tabular}
\end{center}
\qedhere
\end{loesung}

