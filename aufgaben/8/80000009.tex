Sei $A$ die Matrix
\[
A=\begin{pmatrix}
1&5\\
5&1
\end{pmatrix}
\]
\begin{teilaufgaben}
\item Warum ist $A$ diagonalisierbar?
\item Finden Sie eine Basis, in der $A$ diagonal ist.
\item Wie gross ist der Winkel zwischen den neuen Basisvektoren?
\item Berechnen Sie die Transformationsmatrix $T$, welche Koordinaten der Standardbasis in Koordinaten der Eigenbasis umrechnet.
\item Bestimmen Sie die diagonalisierte Matrix $A'$.
\end{teilaufgaben}

\thema{Eigenwerte}
\thema{Eigenvektoren}
\thema{charakteristisches Polynom}
\thema{diagonalisierbar}
\thema{Zwischenwinkel}

\begin{loesung}
\begin{teilaufgaben}
\item
Die Matrix $A$ ist symmetrisch, also gibt es eine Basis aus
Eigenvektoren.
\item
Zur Bestimmung der Basis müssen die Eigenwerte bestimmt werden.
Dazu berechnet man zunächst das charakteristische Polynom und
seine Nullstellen:
\begin{align*}
\chi_A(\lambda)
&=\det(A-\lambda E)=(1-\lambda)^2-25=\lambda^2-2\lambda -24
=(\lambda-6)(\lambda +4),
\end{align*}
die Nullstellen sind also $\lambda_1=6$ und $\lambda_2=-4$. Zu jeder
Nullstelle muss jetzt auch noch ein Eigenvektor gefunden werden.
Für den Eigenwert $\lambda_1=6$ wird die Matrix des dazu zu lösenden
homogenen Gleichungssystems
\[
\begin{pmatrix}-5&5\\5&-5\end{pmatrix},
\]
es ist offensichtlich dass
\[
v_1=\begin{pmatrix}1\\1\end{pmatrix}
\]
ein Lösungsvektor ist. Für $\lambda_2=-4$ ist die Matrix
\[
\begin{pmatrix}5&5\\5&5\end{pmatrix},
\]
mit dem mindestens ebenso offensichtlichen Lösungsvektor
\[
v_2=\begin{pmatrix}1\\-1\end{pmatrix}.
\]
\item
Eigenvektoren einer symmetrischen Matrix zu verschiedenen Eigenwerten
stehen immer senkrecht aufeinander.
\item Die Transformationsmatrix $T$ ist die Inverse der Matrix, deren Spalten den Eigenvektoren von $A$ entsprechen. $T$ ist folglich
\[
    T = \begin{pmatrix}v_1 & v_2 \end{pmatrix}^{-1} = \begin{pmatrix}1 & 1\\ 1 & -1 \end{pmatrix}^{-1} = -\dfrac{1}{2}\begin{pmatrix}-1 & -1\\ -1 & 1 \end{pmatrix} = \begin{pmatrix}\frac{1}{2} & \frac{1}{2}\\ \frac{1}{2} & -\frac{1}{2} \end{pmatrix}.
\]
\item Die diagonalisierte Matrix $A'$ hat als Diagonalelemente die Eigenwerte und ist folglich
\[
    A' = \begin{pmatrix}
          6 & 0\\ 0 & -4
         \end{pmatrix}.
\]
Alternativ kann sie auch berechnet werden als
\[
    A' = TAT^{-1} =  
    \begin{pmatrix}\frac{1}{2} & \frac{1}{2}\\ \frac{1}{2} & -\frac{1}{2} \end{pmatrix}
    \begin{pmatrix} 1&5\\ 5&1 \end{pmatrix}
    \begin{pmatrix}1 & 1\\ 1 & -1 \end{pmatrix}
    = \begin{pmatrix}
          6 & 0\\ 0 & -4
         \end{pmatrix}.
\]
\qedhere
\end{teilaufgaben}
\end{loesung}

