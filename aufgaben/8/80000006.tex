$A$ ist eine $n\times n$-Matrix mit verschiedenen Eigenwerten
$\lambda_1,\lambda_2,\dots,\lambda_n$.
\begin{teilaufgaben}
\item
Berechnen Sie die Determinante von $A$.
\item
Berechnen Sie die Spur von $A$.
\end{teilaufgaben}

\thema{Determinante}
\thema{Eigenwerte}
\thema{Basistransformation}
\thema{Spur}

\begin{loesung}
Da die Eigenwerte alle verschieden sind, gibt es $n$ linear unabhängige
Basisvektoren, also eine Eigenbasis.
Verwendet man die Eigenvektoren als Basis, gibt es eine Basistransformation
$T$ derart, dass
\[
A'=TAT^{-1}=\operatorname{diag}(\lambda_1,\dots,\lambda_n)
\]
\begin{teilaufgaben}
\item
Dann ist die Determinante
\[
\det(A')
=
\lambda_1\cdot \ldots \cdot\lambda_n
=
\det(TAT^{-1})
=
\det(T)\det(A)\det(T^{-1})
=
\det(A).
\]
\item
Ebenso ist die Spur
\[
\operatorname{Spur}(A')
=
\lambda_1+ \ldots +\lambda_n
=
\operatorname{Spur}(TAT^{-1})
=
\operatorname{Spur}(A).
\]
\end{teilaufgaben}
Man beachte, dass man die Eigenbasis und die Transformatioinsmatrix in die
Eigenbasis gar nicht zu kennen braucht.
Es genügt zu wissen, dass es diese gibt.
\qedhere
\end{loesung}
