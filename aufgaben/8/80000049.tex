Gegeben ist die Matrix
\[
A
= 
\begin{pmatrix*}[r]
   -9 & -4 \\
   20 &  9 
\end{pmatrix*}.
\]
\begin{teilaufgaben}
\item Bestimmen Sie das charakteristische Polynom von $A$.
\item Bestimmen Sie die Eigenwerte von $A$.
\item Einer der Eigenwerte ist positiv, finden Sie einen zugehörigen
Eigenvektor.
\end{teilaufgaben}

\begin{loesung}
\begin{teilaufgaben}
\item Das charakteristische Polynom ist
\begin{align*}
\chi_A({\color{darkred}\lambda})
&=
\det(A-{\color{darkred}\lambda} I)
=
\left|
\begin{matrix}
-9-{\color{darkred}\lambda} & -4         \\
20         &  9-{\color{darkred}\lambda}
\end{matrix}
\right|
=
(-9-{\color{darkred}\lambda})(9-{\color{darkred}\lambda}) + 80
=
{\color{darkred}\lambda}^2-81+80
=
{\color{darkred}\lambda}^2-1
\\
&=
({\color{darkred}\lambda}+1)({\color{darkred}\lambda}-1)
\end{align*}
\item
Die Nullstellen von $\chi_A({\color{darkred}\lambda})$ sind $\pm 1$.
\item 
Ein Eigenvektor für ${\color{darkred}\lambda=1}$ wird mit dem
Gauss-Algorithmus gefunden:
\[
\begin{tabular}{|>{$}c<{$}>{$}c<{$}|>{$}c<{$}|}
\hline
-9 - {\color{darkred}1} & -4                      & 0 \\
20                      &  9 - {\color{darkred}1} & 0 \\
\hline
\end{tabular}
=
\begin{tabular}{|>{$}c<{$}>{$}c<{$}|>{$}c<{$}|}
\hline
-10 & -4 & 0 \\
 20 &  8 & 0 \\
\hline
\end{tabular}
\to
\begin{tabular}{|>{$}c<{$}>{$}c<{$}|>{$}c<{$}|}
\hline
1 & \frac{2}{5} & 0 \\
0 &           0 & 0 \rlap{\hspace*{0.3cm}\smiley}\\
\hline
\end{tabular}
\]
Wie erwartet ist das Gleichungssystem singulär, die zweite Koordinate
ist frei wählbar.
Wir können zum Beispiel $y=5$ wählen und erhalten dann als Eigenvektor
\[
v_{\color{darkred}1}
=
\begin{pmatrix*}[r] -2 \\ 5 \end{pmatrix*}.
\]
\end{teilaufgaben}
Für den zweiten Eigenvektor verwendet man den Gauss-Algorithmus im
Tableau
\[
\begin{tabular}{|>{$}c<{$}>{$}c<{$}|>{$}c<{$}|}
\hline
-9 - {\color{darkred}(-1)} & -4                      & 0 \\
20                      &  9 - {\color{darkred}(-1)} & 0 \\
\hline
\end{tabular}
=
\begin{tabular}{|>{$}c<{$}>{$}c<{$}|>{$}c<{$}|}
\hline
-8 & -4 & 0 \\
20 & 10 & 0 \\
\hline
\end{tabular}
\to
\begin{tabular}{|>{$}c<{$}>{$}c<{$}|>{$}c<{$}|}
\hline
 1 & \frac12  & 0 \\
 0 &        0 & 0 \rlap{\hspace*{0.3cm}\smiley}\\
\hline
\end{tabular}
\qquad\Rightarrow\quad
v_{\color{darkred}-1}
=
\begin{pmatrix*}[r] -1\\2\end{pmatrix*}.
\]
Die Korrektheit der Lösung kann man durch Nachrechnen prüfen:
\begin{align*}
Av_{\color{darkred}1\phantom{-}}
&=
\begin{pmatrix*}[r]
-9 & -4 \\
20 &  9
\end{pmatrix*}
\begin{pmatrix*}[r]
-2\\5
\end{pmatrix*}
=
\begin{pmatrix*}[r]
-2\\5
\end{pmatrix*}
=
v_{\color{darkred}1},
\\
Av_{\color{darkred}-1}
&=
\begin{pmatrix*}[r]
-9 & -4 \\
20 &  9
\end{pmatrix*}
\begin{pmatrix*}[r]
-1\\2
\end{pmatrix*}
=
\begin{pmatrix*}[r]
1\\-2
\end{pmatrix*}
=
-v_{\color{darkred}-1}.
\end{align*}
Beide Vektoren sind Eigenvektoren.
\end{loesung}
