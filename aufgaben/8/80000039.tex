Ist die Matrix
\[
A
=
\begin{pmatrix*}[r]
    61&  168\\
   -21&  -58
\end{pmatrix*}
\]
diagonalisierbar?
Wenn ja, geben Sie eine Basis an, in der $A$ diagonal wird.

\thema{Eigenwerte}
\thema{Eigenvektoren}
\thema{charakteristisches Polynom}
\thema{diagonalisierbar}

\begin{loesung}
Das charakteristische Polynom der Matrix $A$ ist
\begin{align*}
\chi_{A}(\lambda)
=
\det(A-\lambda I)
&=
\left|
\begin{matrix}
61-\lambda&168\\
-21&-58-\lambda
\end{matrix}
\right|
\\
&=
-(61-\lambda)(58+\lambda)
+3528
\\
&=
-3538
-3\lambda +\lambda^2 +3528
\\
&=
\lambda^2 -3\lambda - 10
\\
&=
(\lambda + 2)(\lambda - 5)
\intertext{mit den Nullstellen}
\Rightarrow \qquad \lambda
&=\begin{cases}
-2\\
5.
\end{cases}
\end{align*}
Für jeden Eigenwert müssen wir jetzt einen Eigenvektor finden:
\begin{align*}
\lambda&=-2:
&
\begin{tabular}{|>{$}c<{$}>{$}c<{$}|}
\hline
61-\lambda& 168 \\
-21 & -58 - \lambda\\
\hline
\end{tabular}
&\rightarrow
\begin{tabular}{|>{$}r<{$}>{$}r<{$}|}
\hline
 63 & 168 \\
-21 & -56 \\
\hline
\end{tabular}
\rightarrow
\begin{tabular}{|>{$}r<{$}>{$}r<{$}|}
\hline
  1 & \frac{8}{3} \\
  0 &    0   \\
\hline
\end{tabular}
&&\Rightarrow&
v_{-2} &= \begin{pmatrix} 8\\-3\end{pmatrix}
\\
\lambda&=\phantom{-}5:
&
\begin{tabular}{|>{$}c<{$}>{$}c<{$}|}
\hline
61-\lambda& 168 \\
-21 & -58 - \lambda\\
\hline
\end{tabular}
&\rightarrow
\begin{tabular}{|>{$}r<{$}>{$}r<{$}|}
\hline
 56 & 168 \\
-21 & -63 \\
\hline
\end{tabular}
\rightarrow
\begin{tabular}{|>{$}r<{$}>{$}r<{$}|}
\hline
  1 & 3 \\
  0 & 0 \\
\hline
\end{tabular}
&&\Rightarrow&
v_{5} &= \begin{pmatrix} 3\\-1\end{pmatrix}
\end{align*}
Die beiden Vektoren $\mathcal{B} = \{ v_{-2}, v_{5} \}$ bilden daher eine
Basis aus Eigenvektoren.
\end{loesung}

\begin{bewertung}
Charakteristisches Polynom ({\bf X}) 1 Punkt,
Nullstellen/Eigenwerte ({\bf E}) 1 Punkt,
Gauss-Algorithmus ({\bf G}) 2 Punkt,
Eigenvektoren ({\bf v}) 1 Punkt,
Diagonalbasis ({\bf B}) 1 Punkt.
\end{bewertung}
