Ist die Matrix
\[
\begin{pmatrix}
2&6\\
1&-3
\end{pmatrix}
\]
diagonalisierbar? Wenn ja, in welcher Basis?

\thema{diagonalisierbar}
\thema{Eigenvektoren}
\thema{Eigenwerte}
\thema{charakteristisches Polynom}

\begin{loesung}
Wir müssen Eigenwerte und Eigenvektoren finden. Das charkteristische
Polynom ist
\begin{align*}
\left|\begin{matrix}
2-\lambda&6\\
1&-3-\lambda
\end{matrix}\right|
&=(2-\lambda)(-3-\lambda)-6=(\lambda-2)(\lambda+3)-6=\lambda^2+\lambda-6-6
\\
&=\lambda^2+\lambda-12=(\lambda-3)(\lambda + 4)
\end{align*}
Die Eigenwerte sind also $\lambda_1=3$ und $\lambda_2=-4$. Um die Eigenvektoren und damit
die Basisvektoren für eine Diagonalbasis zu finden, muss man
jetzt die Lösungen von $(A-\lambda_i E)x=0$.

Für $\lambda_1=3$ erhält man
\[
(2-\lambda_1)x+6y=0\quad\Rightarrow\quad -x+6y=0\Rightarrow x = 6y
\]
Der Vektor
$\begin{pmatrix}6\\1\end{pmatrix}$
ist also ein Eigenvektor. Tatsächlich ist
\[
A\begin{pmatrix}6\\1\end{pmatrix}
=
\begin{pmatrix}
2&6\\
1&-3
\end{pmatrix}
\begin{pmatrix}6\\1\end{pmatrix}
=\begin{pmatrix}
2\cdot 6+6\cdot 1\\
1\cdot 6+(-3)\cdot 1
\end{pmatrix}
=\begin{pmatrix}
18\\
3
\end{pmatrix}
=
\lambda_1 
\begin{pmatrix}6\\1\end{pmatrix}
\]

Für $\lambda_2=-4$ folgt analog
\[
(2-(-4))x+6y=0\quad\Rightarrow x=-y
\]
Somit ist der Vektor
$\begin{pmatrix}1\\-1\end{pmatrix}$
ein Eigenvektor zum Eigenwert $-4$. Tatsächlich ist
\[
A
\begin{pmatrix}1\\-1\end{pmatrix}
=
\begin{pmatrix}
2&6\\
1&-3
\end{pmatrix}
\begin{pmatrix}1\\-1\end{pmatrix}
=
\begin{pmatrix}
2\cdot 1 +6\cdot(-1)\\
1\cdot1+(-3)\cdot(-1)
\end{pmatrix}
=
\begin{pmatrix}
-4\\
4
\end{pmatrix}
=\lambda_2
\begin{pmatrix}1\\-1\end{pmatrix}.
\qedhere
\]
\end{loesung}

