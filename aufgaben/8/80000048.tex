Welche der folgenden Vektoren sind Eigenvektoren der Matrix
\[
A
= 
\begin{pmatrix*}[r]
   -9 & -4 &  0 \\
   20 &  9 &  0 \\
    0 &  0 &  0
\end{pmatrix*}
\]
und was ist der zugehörige Eigenwert:
\[
a
=
\begin{pmatrix*}[r]
0\\0\\0
\end{pmatrix*},\quad
b
=
\begin{pmatrix*}[r]
0\\0\\1
\end{pmatrix*}
,\quad
c
=
\begin{pmatrix*}[r]
2\\-5\\0
\end{pmatrix*}
,\quad
d
=
\begin{pmatrix*}[r]
-4\\10\\0
\end{pmatrix*}
,\quad
e=
\begin{pmatrix*}[r]
1\\-2\\0
\end{pmatrix*}
,\quad
f=
\begin{pmatrix*}[r]
2\\2\\0
\end{pmatrix*}
\]

\begin{loesung}
\begin{teilaufgaben}
\item Zwar ist $Aa=0a$, aber der Nullvektor $a=0$ ist kein Eigenvektor.
\item Wegen $Ab=0$ ist $b$ ein Eigenvektor zum Eigenwert $0$.
\item 
\(
\displaystyle
Ac
=
\begin{pmatrix*}[r]
   -9 & -4 &  0 \\
   20 &  9 &  0 \\
    0 &  0 &  0
\end{pmatrix*}
\begin{pmatrix*}[r]
2\\-5\\0
\end{pmatrix*}
=
\begin{pmatrix*}[r]
2\\ -5\\ 0
\end{pmatrix*}
=
c,
\)
also ist $c$ ein Eigenvektor zum Eigenwert $1$.
\item
\(
\displaystyle
Ad
=
\begin{pmatrix*}[r]
   -9 & -4 &  0 \\
   20 &  9 &  0 \\
    0 &  0 &  0
\end{pmatrix*}
\begin{pmatrix*}[r]
-4\\10\\0
\end{pmatrix*}
=
\begin{pmatrix*}[r]
-4\\10\\0
\end{pmatrix*}
\),
somit ist $d$ ebenfalls ein Eigenvektor zum Eigenwert $1$.
\item
\(
\displaystyle
Ae
=
\begin{pmatrix*}[r]
   -9 & -4 &  0 \\
   20 &  9 &  0 \\
    0 &  0 &  0
\end{pmatrix*}
\begin{pmatrix*}[r]
1\\-2\\0
\end{pmatrix*}
=
\begin{pmatrix*}[r]
-1\\2\\0
\end{pmatrix*}
=
(-1)\cdot
\begin{pmatrix*}[r]
1\\-2\\0
\end{pmatrix*}
\),
somit ist $e$ ein Eigenvektor zum Eigenwert $-1$.
\item Da
\[
Af
=
\begin{pmatrix*}[r]
   -9 & -4 &  0 \\
   20 &  9 &  0 \\
    0 &  0 &  0
\end{pmatrix*}
\begin{pmatrix*}[r]
2\\2\\0
\end{pmatrix*}
=
\begin{pmatrix*}[r]
-26\\ 58 \\ 0
\end{pmatrix*}
\]
kein Vielfaches von $f$ ist, ist $f$ kein Eigenvektor von $A$.
\end{teilaufgaben}
\end{loesung}
