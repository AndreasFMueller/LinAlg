Die Folge
\[
0,\;
1,\;
5,\;
19,\;
65,\;
211,\;
665,\;
2059,\;
6305,\;\dots
\]
ist durch die Rekursionsformel
\[
x_{n+1}=5x_{n\mathstrut}-6x_{n-1}
\]
mit den Anfangswerten
\[
x_0=0,\qquad x_1=1
\]
definiert.
Finden Sie eine Formel für $x_n$.

\thema{Rekursionsgleichung}
\thema{Eigenwerte}
\thema{Eigenvektoren}
\thema{charakteristisches Polynom}

\begin{loesung}
Die Rekursionsformel kann in vektorieller Form als
\[
\begin{pmatrix}x_{n+1}\\x_{n\mathstrut}\end{pmatrix}
=
\underbrace{
\begin{pmatrix}
5&-6\\
1& 0
\end{pmatrix}
}_{=A}
\begin{pmatrix}x_{n\mathstrut}\\x_{n-1}\end{pmatrix}
\]
geschrieben werden.
Die Matrix $A$ hat das charakteristische Polynom
\begin{align*}
\left|\begin{matrix}
5-\lambda&-6      \\
     1   &-\lambda
\end{matrix}\right|
&=
-(5-\lambda)\lambda+6
=\lambda^2-5\lambda+6=(\lambda-2)(\lambda -3)
\end{align*}
mit den Nullstellen $2$ und $3$.
Die Eigenvektoren finden wir wie folgt. Für $\lambda_1=2$:
\[
\begin{tabular}{|>{$}c<{$}>{$}c<{$}|}
\hline
5-\lambda&-6      \\
   1     &-\lambda\\
\hline
\end{tabular}
=
\begin{tabular}{|>{$}c<{$}>{$}c<{$}|}
\hline
3&-6\\
1&-2\\
\hline
\end{tabular}
\rightarrow
\begin{tabular}{|>{$}c<{$}>{$}c<{$}|}
\hline
1&-2\\
0& 0\\
\hline
\end{tabular}
\qquad\Rightarrow\qquad
v_1=\begin{pmatrix}2\\1\end{pmatrix}
\]
Für $\lambda_2=3$:
\[
\begin{tabular}{|>{$}c<{$}>{$}c<{$}|}
\hline
5-\lambda&-6      \\
   1     &-\lambda\\
\hline
\end{tabular}
=
\begin{tabular}{|>{$}c<{$}>{$}c<{$}|}
\hline
2&-6\\
1&-3\\
\hline
\end{tabular}
\rightarrow
\begin{tabular}{|>{$}c<{$}>{$}c<{$}|}
\hline
1&-3\\
0& 0\\
\hline
\end{tabular}
\qquad\Rightarrow\qquad
v_2=\begin{pmatrix}3\\1\end{pmatrix}
\]
Für die Umrechnung in die Eigenbasis brauchen wir die Matrizen $T$ und $T^{-1}$.
Die Matrix $T$ rechnet in die Eigenbasis um, sie hat also die Matrix
\[
T
=
\begin{pmatrix}
2&3\\
1&1
\end{pmatrix}^{-1}
=
\begin{pmatrix}
-1& 3\\
 1&-2
\end{pmatrix}
\qquad\Rightarrow\qquad
T^{-1}
=
\begin{pmatrix}
2&3\\
1&1
\end{pmatrix}.
\]
Durch Nachrechnen kann man auch kontrollieren, dass
\[
TAT^{-1}
=
\begin{pmatrix}
-1& 3\\
 1&-2
\end{pmatrix}
\begin{pmatrix}
5&-6\\
1& 0
\end{pmatrix}
\begin{pmatrix}
2&3\\
1&1
\end{pmatrix}
=
\begin{pmatrix}
-2& 6\\
 3&-6
\end{pmatrix}
\begin{pmatrix}
2&3\\
1&1
\end{pmatrix}
=
\begin{pmatrix}
2&0\\
0&3
\end{pmatrix}
=
\begin{pmatrix}
\lambda_1&0\\
0&\lambda_2
\end{pmatrix}
\]
tatsächlich eine Diagonalmatrix ist.
Zur Berechnung der Potenzen auf dem Startvektor verwenden wir jetzt die 
Gleichung
\begin{align*}
TAT^{-1}
=
A'
\qquad\Rightarrow\qquad
A^kv_0
=
T^{-1}A'^kTv_0
&=
\begin{pmatrix}
2&3\\
1&1
\end{pmatrix}
\begin{pmatrix}
2^k& 0\\
 0 &3^k
\end{pmatrix}
\begin{pmatrix}
-1& 3\\
 1&-2
\end{pmatrix}
\begin{pmatrix}
1\\0
\end{pmatrix}
\\
&=
\begin{pmatrix}
2&3\\
1&1
\end{pmatrix}
\begin{pmatrix}
2^k& 0\\
 0 &3^k
\end{pmatrix}
\begin{pmatrix}
-1\\1
\end{pmatrix}
\\
&=
\begin{pmatrix}
2&3\\
1&1
\end{pmatrix}
\begin{pmatrix}
-2^k\\
 3^k
\end{pmatrix}
\\
&=
\begin{pmatrix}
-2^{k+1}+3^{k+1}\\
-2^k+3^k
\end{pmatrix}
\end{align*}
Daraus kann man die Lösung als zweite Komponenten ablesen:
\[
x_n=f(n)=-2^n+3^n.
\]
Wir kontrollieren das Resultat durch Ausrechnen einiger Werte:
\[
\begin{tabular}{l|rrrrrrrrr}
$n$&0&1&2&3&4&5&6&7&8\\
\hline
$x_n$&0&1&5&19&65&211&665&2059&6305\\
$3^n-2^n$&0&1&5&19&65&211&665&2059&6305\\
\end{tabular}
\qedhere
\]
\end{loesung}

