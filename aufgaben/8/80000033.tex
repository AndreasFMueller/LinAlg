Ist die Matrix
\[
A=\begin{pmatrix}8&-\frac{3}{2}\\4 & 1\end{pmatrix}.
\]
diagonalisierbar? Wenn ja, geben Sie eine Basis an, in der $A$ diagonal
ist. Bestimmen Sie ausserdem die diagonalisierte Matrix $A'$ und 
die Basis-Transformationsmatrix, mit welcher Koordinaten der 
Standardbasis in Koordinaten der gefundenen Basis umgerechnet 
werden können.

\thema{Eigenwerte}
\thema{Eigenvektoren}
\thema{charakteristisches Polynom}
\thema{diagonalisierbar}
\thema{Basistransformation}

\begin{loesung}
Eine geeignete Basis besteht aus den Eigenvektoren von $A$, 
weshalb wir die Eigenwerte und Eigenvektor der Matrix finden müssen.

Das charakteristische Polynom von $A$ ist
\begin{align*}
\det(A-\lambda E)
=\left|\begin{matrix}8-\lambda&-\frac{3}{2}\\4 & 1-\lambda\end{matrix}\right|
&=(8-\lambda)(1-\lambda)+6\\
&=8-9\lambda+\lambda^2+6\\
&=\lambda^2-9\lambda+14\\
&=(\lambda-7)(\lambda-2).
\end{align*}
Die Nullstellen können direkt abgelesen werden. Sie sind
\begin{align*}
\lambda_{1} = 7\qquad \text{und}\qquad \lambda_2 = 2.
\end{align*}
Für jeden Eigenwert muss jetzt mit Hilfe des Gaussalgorithmus ein
Eigenvektor gefunden werden, indem die Gleichung $(A-\lambda E )v = 0$
gelöst wird.

Für $\lambda_1=7$ findet man
\[
\begin{tabular}{|>{$}c<{$}>{$}c<{$}|>{$}c<{$}|}
\hline
8-7&-\frac{3}{2} & 0\\
4&1-7 & 0\\
\hline
\end{tabular}
=
\begin{tabular}{|>{$}c<{$}>{$}c<{$}|>{$}c<{$}|}
\hline
1&-\frac{3}{2} & 0\\
4&-6 & 0\\
\hline
\end{tabular}
\rightarrow
\begin{tabular}{|>{$}c<{$}>{$}c<{$}|>{$}c<{$}|}
\hline
1&-\frac{3}{2} &0\\
0&0&0\\
\hline
\end{tabular}
\]
Die zweite Koordinate ist wie erwartet frei wählbar, indem wir sie auf
$2$ setzen erhalten wir einen Eigenvektor ohne Brüche
\[
v_1=\begin{pmatrix}3\\2\end{pmatrix}.
\]

Für $\lambda_2=2$ erhalten wir
\[
\begin{tabular}{|>{$}c<{$}>{$}c<{$}|>{$}c<{$}|}
\hline
8-2&-\frac{3}{2} & 0\\
4&1-2 & 0\\
\hline
\end{tabular}
=
\begin{tabular}{|>{$}c<{$}>{$}c<{$}|>{$}c<{$}|}
\hline
6&-\frac{3}{2} & 0\\
4&-1 & 0\\
\hline
\end{tabular}
\rightarrow
\begin{tabular}{|>{$}c<{$}>{$}c<{$}|>{$}c<{$}|}
\hline
1&-\frac{1}{4} &0\\
0&0&0\\
\hline
\end{tabular}.
\]
Wieder ist die zweite Variable frei wählbar, die Wahl $4$ liefert 
den Eigenvektor
\[
v_2=\begin{pmatrix}1\\4\end{pmatrix}.
\]

Kontrolle: Wir kontrollieren die Rechnung durch Multiplikation von $A$
mit den gefundenen Eigenvektoren:
\begin{align*}
Av_1&=\begin{pmatrix}8&-\frac{3}{2}\\4 & 1\end{pmatrix}\begin{pmatrix}3\\2\end{pmatrix}
=\begin{pmatrix}24-3\\ 12 + 2\end{pmatrix}
=\begin{pmatrix}21\\ 14\end{pmatrix}=7\begin{pmatrix}3\\2\end{pmatrix}
=\lambda_1v_1,\\
Av_2&=\begin{pmatrix}8&-\frac{3}{2}\\4 & 1\end{pmatrix}\begin{pmatrix}1\\4\end{pmatrix}
=\begin{pmatrix}8-6\\4+4 \end{pmatrix}
=\begin{pmatrix}2\\8 \end{pmatrix}=2\begin{pmatrix}1\\4\end{pmatrix}
=\lambda_2v_2.
\end{align*}
Da wir zwei linear unabhängige Eigenvektoren bzw. Basisvektoren gefunden haben,
ist die Matrix $A$ folglich diagonalisierbar.
Die diagonalisierte Matrix $A'$ enthält auf der Diagonalen die
Eigenwerte. Es ist folglich die Matrix
\[
A'=\begin{pmatrix}7&0 \\0 & 2\end{pmatrix}.
\]
Mit den gefundenen Basisvektoren $v_1$ und $v_2$ kann nun auch noch die 
Basis-Transformationsmatrix bestimmt werden als:
\[
  T = \begin{pmatrix}3&1\\2&4\end{pmatrix}^{-1}
  = \dfrac{1}{3\cdot 4 - 2\cdot 1}\begin{pmatrix}4&-1\\-2&3\end{pmatrix}
  = \dfrac{1}{10}\begin{pmatrix}4&-1\\-2&3\end{pmatrix}.
  \qedhere
\]
\end{loesung}

\begin{bewertung}
Charakteristisches Polynom ({\bf X}) 1 Punkt,
Eigenwerte ({\bf E}) 1 Punkt,
Eigenvektoren ({\bf V}) 2 Punkte,
Transformationsmatrix ({\bf T}) 1 Punkt,
Diagonalisierte Matrix ({\bf D}) 1 Punkt.
\end{bewertung}


