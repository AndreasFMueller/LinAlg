Ist die Matrix
\[
A
=
\begin{pmatrix}
12& 5&-1\\
 0& 6& 0\\
36&30& 0
\end{pmatrix}
\]
diagonalisierbar?

\begin{loesung}
Wir müssen eine Basis aus Eigenvektoren suchen.
Dazu berechnen wir zunächst das charakteristische Polynom
\begin{align*}
\chi_{A}(\lambda)
&=
\left|\;\begin{matrix}
12-\lambda &  5         & -1      \\
 0         &  6-\lambda &  0      \\
36         & 30         & -\lambda
\end{matrix}\;\right|
=
-(12-\lambda)(6-\lambda)\lambda
+0+0
-(-1)\cdot36(6-\lambda)-0-0
\\
&=
-\lambda(72-18\lambda+\lambda^2) + 216 -36\lambda
\\
&=
-\lambda^3+18\lambda^2-108\lambda +216
=
-(\lambda -6)^3.
\end{align*}
Das charakteristische Polynom hat die dreifache Nullstelle $\lambda=6$.
Wir suchen nach Eigenvektoren mit Hilfe des Gaussalgorithmus:
\begin{align*}
\begin{tabular}{|>{$}c<{$} >{$}c<{$} >{$}c<{$}|}
\hline
12-\lambda &  5         & -1       \\
 0         &  6-\lambda &  0       \\
36         & 30         & -\lambda \\
\hline
\end{tabular}
&=
\begin{tabular}{|>{$}c<{$} >{$}c<{$} >{$}c<{$}|}
\hline
 6         &  5         & -1       \\
 0         &  0         &  0       \\
36         & 30         & -6       \\
\hline
\end{tabular}
\rightarrow
\begin{tabular}{|>{$}c<{$} >{$}c<{$} >{$}c<{$}|}
\hline
 6         &  5         & -1       \\
 0         &  0         &  0       \\
 0         &  0         &  0       \\
\hline
\end{tabular}
\end{align*}
Damit keine Brüche auftreten, haben wir hier auf die Pivot-Divison
verzichtet und direkt das sechsfache der ersten Zeile von der letzten
subtrahiert.
Wie man sehen kann, ist der Rang der Matrix $1$, es gibt also nur
zwei linear unabhängige Eigenvektoren.
Insbesondere ist die Matrix $A$ nicht diagonalisierbar.

Wir finden zum Beispiel die beiden Vektoren
\[
v_1
=
\begin{pmatrix}1\\0\\6\end{pmatrix}
\quad\text{und}\quad
v_2
=
\begin{pmatrix}0\\1\\5\end{pmatrix}
\]
als Eigenvektoren.

Kontrolle:
\begin{align*}
Av_1
&=
\begin{pmatrix}
12& 5&-1\\
 0& 6& 0\\
36&30& 0
\end{pmatrix}
\begin{pmatrix}1\\0\\6\end{pmatrix}
=
\begin{pmatrix}
12-6\\
0\\
36
\end{pmatrix}
=
6\cdot
\begin{pmatrix}1\\0\\6\end{pmatrix}
=
\lambda v_1,
\\
Av_2
&=
\begin{pmatrix}
12& 5&-1\\
 0& 6& 0\\
36&30& 0
\end{pmatrix}
\begin{pmatrix}0\\1\\5\end{pmatrix}
=
\begin{pmatrix}
5-5\\
6\\
30
\end{pmatrix}
=
6\cdot
\begin{pmatrix}0\\1\\5\end{pmatrix}
=
\lambda v_2.
\qedhere
\end{align*}
\end{loesung}
