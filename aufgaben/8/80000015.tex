\begin{teilaufgaben}
\item
Kann man die Matrix
\[
A=\begin{pmatrix}
5&4\\
-3&-2
\end{pmatrix}
\]
diagonalisieren?
Wenn ja, welche Basis müsste man dazu verwenden?
\item
Kann man $A$ in einer Basis aus orthonormierten Vektoren
diagonalisieren?
\end{teilaufgaben}

\thema{Eigenwerte}
\thema{Eigenvektoren}
\thema{charakteristisches Polynom}
\thema{diagonalisierbar}

\begin{loesung}
\begin{teilaufgaben}
\item
Wenn $A$ diagonalisierbar ist, dann muss dazu eine Basis aus
Eigenvektoren verwendet werden, wir bestimmen also als erstes
die Eigenwerte und Eigenvektoren von $A$.

Das charakeristische Polynom von $A$ ist
\begin{align*}
\det(A-\lambda E)
&=
\left|\,
\begin{matrix}
5-\lambda&4\\-3&-2-\lambda
\end{matrix}
\,\right|
=(5-\lambda)(-2-\lambda)+12
\\
&
=
\lambda^2-3\lambda-10+12
=
\lambda^2-3\lambda+2
=0
\\
\Rightarrow
\lambda_\pm&=\frac32\pm\sqrt{\frac94-2}=\frac32\pm\sqrt{\frac14}=\begin{cases}
2\\
1
\end{cases}
\end{align*}
Jetzt müssen wir für jeden Eigenwert den zugehörigen Eigenvektor
finden, indem wir den Gauss-Algorithmus auf die $A-\lambda E$ anwenden.

Für $\lambda_+=2$ findet man
\begin{align*}
\begin{tabular}{|>{$}c<{$}>{$}c<{$}|>{$}c<{$}|}
\hline
5-2&4&0\\
-3&-2-2&0\\
\hline
\end{tabular}
=
\begin{tabular}{|>{$}c<{$}>{$}c<{$}|>{$}c<{$}|}
\hline
3&4&0\\
-3&-4&0\\
\hline
\end{tabular}
&
\rightarrow
\begin{tabular}{|>{$}c<{$}>{$}c<{$}|>{$}c<{$}|}
\hline
1&\frac43&0\\
0&0&0\\
\hline
\end{tabular}.
\end{align*}
Daraus kann man ablesen, dass
\[
v_+
=
\begin{pmatrix}1\\-\frac34 \end{pmatrix}
\]
ein Eigenvektor zum Eigenwert $\lambda_+=2$ ist.

Für $\lambda_-=1$  gilt entsprechend
\begin{align*}
\begin{tabular}{|>{$}c<{$}>{$}c<{$}|>{$}c<{$}|}
\hline
5-1&4&0\\
-3&-2-1&0\\
\hline
\end{tabular}
=
\begin{tabular}{|>{$}c<{$}>{$}c<{$}|>{$}c<{$}|}
\hline
4&4&0\\
-3&-3&0\\
\hline
\end{tabular}
&
\rightarrow
\begin{tabular}{|>{$}c<{$}>{$}c<{$}|>{$}c<{$}|}
\hline
1&1&0\\
0&0&0\\
\hline
\end{tabular}.
\end{align*}
Daraus kann man ablesen, dass
\[
v_-
=
\begin{pmatrix}1\\-1 \end{pmatrix}
\]
ein Eigenvektor zum Eigenwert $\lambda_-=1$ ist.

Man muss also die Basis
\[
\left\{
\begin{pmatrix}1\\-\frac34 \end{pmatrix}
,
\begin{pmatrix}1\\-1 \end{pmatrix}
\right\}
\]
verwenden, um die Matrix zu diagonalisieren.
\item
Die Basisvektoren, in denen $A$ diagonal ist, müssen auf jeden Fall
Eigenvektoren von $A$ sein.
Wir haben die Eigenvektoren bestimmt, wenigstens bis auf Vielfache.
Wenn die Diagonalisierung mit einer orthonormierten Basis möglich wäre,
dann müsste man dafür Vielfache der bereits gefundenen Vektoren verwenden
können, und diese Vielfachen müssten senkrecht aufeinander stehen.
Orthogonalität ändert sich durch Skalieren der Vektoren nicht.
Das Skalarprodukt der bereits gefundenen Eigenvektoren $v_{\pm}$
ist
\[
v_+\cdot v_-=
\begin{pmatrix}1\\-\frac34 \end{pmatrix}
\cdot
\begin{pmatrix}1\\-1 \end{pmatrix}
=1+\frac34=\frac74 >0,
\]
die beiden Vektoren sind also niemals senkrecht.
Also kann man die Matrix $A$ nicht mit einer Orthonormalbasis diagonalisieren.
\qedhere
\end{teilaufgaben}
\end{loesung}

