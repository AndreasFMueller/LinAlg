Ist die Matrix
\[
A
=
\begin{pmatrix}
3&0&0\\
0&22&-6\\
0&-6&13
\end{pmatrix}
\]
diagonalisierbar?
Wenn ja finden Sie eine Basis, in der $A$ diagonal ist.

\thema{Eigenwerte}
\thema{Eigenvektoren}
\thema{charakteristisches Polynom}
\thema{diagonalisierbar}

\begin{loesung}
Der Standardbasisvektor $e_1$ ist offensichtlich ein Eigenvektor zum Eigenwert
$\lambda_1=3$, wir nennen ihn auch $v_1$.
Zur Bestimmung der verbleibenden Eigenwerte und Eigenvektoren verwenden wir
das charakteristische Polynom:
\begin{align*}
\chi_A(\lambda)
&=
\left|\begin{matrix}
3-\lambda& 0 & \\
0&22-\lambda&-6\\
0&-6&13-\lambda
\end{matrix}\right|
=
(3-\lambda)
\bigl((22-\lambda)(13-\lambda)-36)\bigr)
=
(3-\lambda)(\lambda^2 - 35\lambda + 250)
=
0
\end{align*}
Der quadratische Faktor lässt sich faktorisieren in
\[
\lambda^2-35\lambda+250
=
(\lambda-10)(\lambda-25),
\]
woraus man die Eigenwerte $\lambda_2=10$ und $\lambda_3=25$ ablesen kann.
Diese kann man natürlich auch mit Hilfe der Lösungsformel für quadratische
Gleichungen finden:
\[
\lambda_{2,3}
=
\frac{35\pm\sqrt{35^2-4\cdot 250}}{2}
=
\frac{35\pm\sqrt{225}}2
=
\frac{35\pm15}2
=
\begin{cases}
25&\\
10.&
\end{cases}
\]
Die zugehörigen Eigenvektoren können wir mit dem Gauss-Algorithmus
bestimmen:
\begin{align*}
\begin{tabular}{|>{$}c<{$}>{$}c<{$}|>{$}c<{$}|}
\hline
22-\lambda_2&          -6&0\\
          -6&13-\lambda_2&0\\
\hline
\end{tabular}
&=
\begin{tabular}{|>{$}c<{$}>{$}c<{$}|>{$}c<{$}|}
\hline
-3&-6&0\\
-6&-12&0\\
\hline
\end{tabular}
\rightarrow
\begin{tabular}{|>{$}c<{$}>{$}c<{$}|>{$}c<{$}|}
\hline
1&2&0\\
0&0&0\\
\hline
\end{tabular}
&&\Rightarrow&
v_2
&=
\begin{pmatrix}0\\-2\\1\end{pmatrix}
\\
\begin{tabular}{|>{$}c<{$}>{$}c<{$}|>{$}c<{$}|}
\hline
22-\lambda_3&          -6&0\\
          -6&13-\lambda_3&0\\
\hline
\end{tabular}
&=
\begin{tabular}{|>{$}c<{$}>{$}c<{$}|>{$}c<{$}|}
\hline
12&-6&0\\
-6& 3&0\\
\hline
\end{tabular}
\rightarrow
\begin{tabular}{|>{$}c<{$}>{$}c<{$}|>{$}c<{$}|}
\hline
1&-\frac12&0\\
0&0&0\\
\hline
\end{tabular}
&&\Rightarrow&
v_3
&=
\begin{pmatrix}0\\1\\2\end{pmatrix}
\end{align*}
Zur Kontrolle rechnen wir nach:
\begin{align*}
Av_2
&=
\begin{pmatrix}
3&0&0\\
0&22&-6\\
0&-6&13
\end{pmatrix}
\begin{pmatrix}0\\-2\\1\end{pmatrix}
=
\begin{pmatrix}
0\\
-50\\
 25
\end{pmatrix}
=
25\,v_2
=
\lambda_2v_2,
&
Av_3
&=
\begin{pmatrix}
3&0&0\\
0&22&-6\\
0&-6&13
\end{pmatrix}
\begin{pmatrix}0\\1\\2\end{pmatrix}
=
\begin{pmatrix}
0\\
10\\
20
\end{pmatrix}
=
10\,v_3
=
\lambda_3v_3.
\end{align*}
Also sind die Vektoren $v_2$ und $v_3$ Eigenvektoren.
In der Basis $\{v_1,v_2,v_3\}$ hat die Matrix $A$ Diagonalform.
\end{loesung}

\begin{bewertung}
Charakteristisches Polynom ({\bf X}) 1 Punkt,
Nullstellen ({\bf N}) 1 Punkt,
Gleichungssystem für jeden Eigenwert ({\bf G}) 1 Punkt,
Eigenvektor zu $\lambda = 3$ ($\text{\bf E}_3$) 1 Punkt,
Eigenvektor zu $\lambda = 10$ ($\text{\bf E}_{10}$) 1 Punkt,
Eigenvektor zu $\lambda = 25$ ($\text{\bf E}_{25}$) 1 Punkt.
\end{bewertung}
