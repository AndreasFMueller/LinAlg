Gegeben ist die Matrix
\[
A=\begin{pmatrix}
0&1&0\\
0&0&1\\
0&0&0\\
\end{pmatrix}
\]
\begin{teilaufgaben}
\item Berechnen Sie die Matrix $B=(A-\lambda I)^3$
\item Bestimmen Sie die Determinante von $B$.
\item Für welche Werte von $\lambda$ ist $B$ singulär?
\end{teilaufgaben}

\thema{Matrixalgebra}
\themaL{singular}{singulär}
\thema{Determinante}

\begin{loesung}
\begin{teilaufgaben}
\item Es gilt die binomische Formel:
\begin{align*}
(A-\lambda I)^3&=A^3-3\lambda A^2+3\lambda^2 A-\lambda^3 I
\end{align*}
Die Potenzen von $A$ sind aber besonders leicht auszurechnen:
\begin{align*}
A^2&=
\begin{pmatrix}
0&1&0\\
0&0&1\\
0&0&0
\end{pmatrix}
\begin{pmatrix}
0&1&0\\
0&0&1\\
0&0&0
\end{pmatrix}
=
\begin{pmatrix}
0&0&1\\
0&0&0\\
0&0&0
\end{pmatrix}
\\
A^3&=A^2A=
\begin{pmatrix}
0&0&1\\
0&0&0\\
0&0&0
\end{pmatrix}
\begin{pmatrix}
0&1&0\\
0&0&1\\
0&0&0
\end{pmatrix}
=\begin{pmatrix}
0&0&0\\
0&0&0\\
0&0&0
\end{pmatrix}
\\
\end{align*}
Die gesuchte Matrix $B$ ist also
\[
B=0 -3\lambda \begin{pmatrix}
0&0&1\\
0&0&0\\
0&0&0
\end{pmatrix}
+3\lambda^2
\begin{pmatrix}
0&1&0\\
0&0&1\\
0&0&0
\end{pmatrix}
-\lambda^3\begin{pmatrix}1&0&0\\0&1&0\\0&0&1\end{pmatrix}
=\begin{pmatrix}
-\lambda^3& 3\lambda^2&-3\lambda\\
         0&- \lambda^3&3\lambda^3\\
         0&          0&- \lambda^3
\end{pmatrix}
\]
\item
Da $B$ eine Dreiecksmatrix ist, ist die Determinante einfach das Produkt
der Diagonalelemente, also
\[
\det(B)=-\lambda^9.
\]
Oder man verwendet
\[
\det((A-\lambda I)^3)
=
\det(A-\lambda I)^3=(-\lambda^3)^3=-\lambda^9.
\]
\item Die Determinante verschwindet genau dann, wenn $\lambda=0$.
\qedhere
\end{teilaufgaben}
\end{loesung}

