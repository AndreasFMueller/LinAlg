Ist die Matrix
\[
A=\begin{pmatrix}
 5&4&0\\
-1&1&0\\
 0&0&2
\end{pmatrix}
\]
diagonalisierbar?

\begin{loesung}
Wir müssen untersuchen, ob es eine Basis aus Eigenvektoren gibt.
Zunächst ist klar, dass $\lambda=2$ ein Eigenwert ist und $e_3$ der
zugehörige Eigenvektor, wir müssen also nur noch die $2\times 2$-Matrix
\[
A_0=\begin{pmatrix}
 5&4\\
-1&1
\end{pmatrix}
\]
in der linken oberen Ecke von $A$ studieren.
Dazu berechnen wir das charakteristische Polynom
\begin{align*}
\chi_{A_0}(\lambda)
&=
\left|\;\begin{matrix}
5-\lambda&4\\
-1&1-\lambda
\end{matrix}\;\right|
=
(5-\lambda)(1-\lambda)+4
\\
&=
5-6\lambda+\lambda^2+4=\lambda^2-6\lambda+9=(\lambda-3)^2.
\end{align*}
Das charakteristische Polynom hat die doppelte Nullstelle $\lambda=3$.

An dieser Stelle könnten wir eine Abkürzung nehmen: wäre $A_0$
diagonalisierbar, dann müssten $A_0 = \lambda E$ sein.
Da dies nicht der Fall ist, können wir schliessen, dass $A_0$ und
damit $A$ nicht diagonalisierbar ist.

In etwas mehr Detail bestimmen wir Eigenvektoren mit Hilfe des
Gauss-Algorithmus.
Dazu bearbeiten wir das Tableau
\begin{align*}
\begin{tabular}{|>{$}c<{$}>{$}c<{$}|}
\hline
5-\lambda&4\\
-1&1-\lambda\\
\hline
\end{tabular}
&=
\begin{tabular}{|>{$}c<{$}>{$}c<{$}|}
\hline
2&4\\
-1&-2\\
\hline
\end{tabular}
\rightarrow
\begin{tabular}{|>{$}c<{$}>{$}c<{$}|}
\hline
1&2\\
0&0\\
\hline
\end{tabular}.
\end{align*}
Daraus lesen wir ab, dass alle Eigenvektoren von $A_0$ Vielfache von
\[
v_\lambda = v_3 = \begin{pmatrix}-2\\1\end{pmatrix}
\]
sein müssen.
Insbesondere gibt es keine Basis aus Eigenvektoren des zweidimensionalen
Raumes $\mathbb R^2$, also ist $A_0$ und damit auch $A$ nicht diagonalisierbar.
\end{loesung}

\begin{bewertung}
Charakteristisches Polynom ({\bf X}) 1 Punkt,
Nullstellen/Eigenwerte ({\bf W}) 1 Punkt,
Eigenvektor für $\lambda=2$ ({\bf Z}) 1 Punkt,
Eigenvektor für $\lambda=3$ ({\bf D}) 1 Punkt,
Nur ein Eigenvektor für $\lambda=3$ ({\bf E}) 1 Punkt,
Antwort Diagonalisierbarkeit ({\bf A}) 1 Punkt.
\end{bewertung}
