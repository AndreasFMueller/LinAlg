Ist die Matrix
\[
A
=
\begin{pmatrix*}[r]
-78&-30\\
200& 77
\end{pmatrix*}
\]
diagonalisierbar?
Wenn ja, geben Sie eine Basis an, in der $A$ diagonal wird.

\thema{Eigenwerte}
\thema{Eigenvektoren}
\thema{charakteristisches Polynom}
\thema{diagonalisierbar}

\begin{loesung}
Das charakteristische Polynom der Matrix $A$ ist
\begin{align*}
\chi_{A}(\lambda)
=
\det(A-\lambda E)
&=
\left|
\begin{matrix}
-78-\lambda&-30\\
200&77-\lambda
\end{matrix}
\right|
\\
&=
-(78+\lambda)(77-\lambda)
+6000
\\
&=
-6006 +\lambda +\lambda^2 +6000
\\
&=
\lambda^2 + \lambda - 6
\\
&=
(\lambda - 2)(\lambda + 3)
\intertext{mit den Nullstellen}
\Rightarrow \qquad \lambda
&=\begin{cases}
2\\
-3.
\end{cases}
\end{align*}
Für jeden Eigenwert müssen wir jetzt einen Eigenvektor finden:
\begin{align*}
\lambda&=\phantom{-}2:
&
\begin{tabular}{|>{$}c<{$}>{$}c<{$}|}
\hline
-78-\lambda& -30 \\
200 & 77 - \lambda\\
\hline
\end{tabular}
&\rightarrow
\begin{tabular}{|>{$}r<{$}>{$}r<{$}|}
\hline
-80 & -30 \\
200 &  75 \\
\hline
\end{tabular}
\rightarrow
\begin{tabular}{|>{$}r<{$}>{$}r<{$}|}
\hline
  1 & \frac{3}{8} \\
  0 &    0   \\
\hline
\end{tabular}
&&\Rightarrow&
v_2 &= \begin{pmatrix} 3\\-8\end{pmatrix}
\\
\lambda&=-3:
&
\begin{tabular}{|>{$}c<{$}>{$}c<{$}|}
\hline
-78-\lambda& -30 \\
200 & 77 - \lambda\\
\hline
\end{tabular}
&\rightarrow
\begin{tabular}{|>{$}r<{$}>{$}r<{$}|}
\hline
-75 & -30 \\
200 &  80 \\
\hline
\end{tabular}
\rightarrow
\begin{tabular}{|>{$}r<{$}>{$}r<{$}|}
\hline
  1 & \frac{2}{5} \\
  0 &    0   \\
\hline
\end{tabular}
&&\Rightarrow&
v_{-3} &= \begin{pmatrix} 2\\-5\end{pmatrix}
\end{align*}
Die beiden Vektoren $\mathcal{B} = \{ v_2, v_{-3} \}$ bilden daher eine
Basis aus Eigenvektoren.
\end{loesung}

\begin{bewertung}
Charakteristisches Polynom ({\bf X}) 1 Punkt,
Nullstellen/Eigenwerte ({\bf E}) 1 Punkt,
Gauss-Algorithmus ({\bf G}) 1 Punkt,
Eigenvektoren ({\bf v}) 2 Punkt,
Diagonalbasis ({\bf B}) 1 Punkt.
\end{bewertung}
