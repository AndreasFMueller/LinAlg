Ist die Matrix
\[
\begin{pmatrix}
4&-7\\
2&-5
\end{pmatrix}
\]
diagonalisierbar? Wenn ja, in welcher Basis?

\thema{Eigenwerte}
\thema{Eigenvektoren}
\thema{charakteristisches Polynom}
\thema{diagonalisierbar}

\begin{loesung}
Zunächst müssen die Eigenwerte der Matrix
\[
A=
\begin{pmatrix}
4&-7\\
2&-5
\end{pmatrix}
\]
bestimmt werden. Das
charakteristische Polynom ist
\begin{align*}
\chi_A(\lambda)&=
\det(A-\lambda E)
=
\left|\,\begin{matrix}4-\lambda&-7\\2&-5-\lambda\end{matrix}\,\right|
=(4-\lambda)(-5-\lambda)+14\\
&=-20+5\lambda-4\lambda+\lambda^2+14
=\lambda^2+\lambda-6=(\lambda -2)(\lambda +3).
\end{align*}
Seine Nullstellen sind $\lambda_1=2$ und $\lambda_2=-3$. Da zwei
verschiedene Eigenwerte vorliegen, und es zu jedem Eigenwert
mindestens einen Eigenvektor geben muss, kann man die Frage, ob
die Matrix diagonalisierbar ist, bereits positiv beantworten. Um
eine geeignete Basis zu finden, müssen jetzt noch die Eigenvektoren
gefunden werden. Dazu muss das Gleichungssystem $(A-\lambda_i)v_i=0$
gelöst werden. Für $\lambda_1$ ergibt sich
\[
A-\lambda_1 E=\begin{pmatrix}
2&-7\\
2&-7
\end{pmatrix}
\qquad \Rightarrow\qquad v_1=\begin{pmatrix}7\\2\end{pmatrix}
\]
und für $\lambda_2$ erhält man
\[
A-\lambda_2 E=\begin{pmatrix}
7&-7\\
2&-2
\end{pmatrix}
\qquad \Rightarrow\qquad v_2=\begin{pmatrix}1\\1\end{pmatrix}
\]
Die Vektoren $v_1$ und $v_2$ bilden eine Basis, in der die Matrix $A$
diagonal wird.
\end{loesung}

\begin{bewertung}
Charakteristisches Polyom ({\bf X}) 1 Punkt,
Nullstellen ({\bf N}) 1 Punkt,
Ansatz für Eigenvektoren ({\bf A}) 1 Punkt,
zwei Eigenvektoren ({\bf E}) 2 Punkte,
Basis ({\bf B}) 1 Punkt.
\end{bewertung}
