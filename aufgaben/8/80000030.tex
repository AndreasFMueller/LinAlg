Ist die Matrix
\[
A=
\begin{pmatrix}
2&-2&0\\
3& 7&0\\
0& 0&2
\end{pmatrix}
\]
diagonalisierbar? Wenn ja finden Sie eine Basis, in der $A$ diagonal ist.

\thema{Eigenwerte}
\thema{Eigenvektoren}
\thema{charakteristisches Polynom}
\thema{diagonalisierbar}

\begin{loesung}
Ein Eigenvektor ist $e_3$ mit dem Eigenwert $2$. Es muss also nur noch
der $2\times 2$-Block in der linken oberen Ecke untersucht werden.
Das charakteristische Polynom desselben ist
\begin{align*}
\det(A-\lambda I)
&=
(\lambda -2)\left|\,\begin{matrix}2-\lambda&-2\\3&7-\lambda\end{matrix}\,\right|
\\
&=(2-\lambda)\bigl( (2-\lambda)(7-\lambda)+6 \bigr)
\\
&=(2-\lambda)(\lambda^2-9\lambda+20)
\\
&=(2-\lambda)(\lambda-4)(\lambda -5)
\end{align*}
Die Matrix hat also noch die zwei Eigenwert $\lambda=4$ und $\lambda=5$.
Die Eigenvektoren für diese beiden Eigenwerte finden wir mit dem
Gauss-Algorithmus.

Für $\lambda=4$ finden wir
\begin{align*}
\begin{tabular}{|>{$}c<{$}>{$}c<{$}|>{$}c<{$}|}
\hline
2-4&-2 &0\\
3  &7-4&0\\
\hline
\end{tabular}
&
\rightarrow
\begin{tabular}{|>{$}c<{$}>{$}c<{$}|>{$}c<{$}|}
\hline
-2&-2 &0\\
3  &3&0\\
\hline
\end{tabular}
\rightarrow
\begin{tabular}{|>{$}c<{$}>{$}c<{$}|>{$}c<{$}|}
\hline
1&1&0\\
0&0&0\\
\hline
\end{tabular}
&&\Rightarrow&
v&=\begin{pmatrix}1\\-1\end{pmatrix}
%\end{align*}
\intertext{%
Für $\lambda=5$ finden wir
}
%\begin{align*}
\begin{tabular}{|>{$}c<{$}>{$}c<{$}|>{$}c<{$}|}
\hline
2-5&-2 &0\\
3  &7-5&0\\
\hline
\end{tabular}
&
\rightarrow
\begin{tabular}{|>{$}c<{$}>{$}c<{$}|>{$}c<{$}|}
\hline
-3&-2 &0\\
3  &2&0\\
\hline
\end{tabular}
\rightarrow
\begin{tabular}{|>{$}c<{$}>{$}c<{$}|>{$}c<{$}|}
\hline
1&\frac23&0\\
0&      0&0\\
\hline
\end{tabular}
&&\Rightarrow&
v&=\begin{pmatrix}2\\-3\end{pmatrix}
\end{align*}
Die Matrix $A$ wird daher in der Basis aus den Vektoren
\[
\begin{aligned}
v_1&=\begin{pmatrix}1\\-1\\0\end{pmatrix},\qquad
&
v_2&=\begin{pmatrix}2\\-3\\0\end{pmatrix}\qquad\text{und}\qquad
&
v_3&=\begin{pmatrix}0\\0\\1\end{pmatrix}
\end{aligned}
\]
diagonal sein.
\end{loesung}

\begin{bewertung}
Charakteristisches Polynom ({\bf X}) 1 Punkt,
Nullstellen ({\bf N}) 1 Punkt,
Gleichungssystem für jeden Eigenwert ({\bf G}) 1 Punkt,
Eigenvektor zu $\lambda=4$ ($\text{\bf E}_4$) 1 Punkt,
Eigenvektor zu $\lambda=5$ ($\text{\bf E}_5$) 1 Punkt,
EIgenvektor zu $\lambda=2$ ($\text{\bf E}_2$) 1 Punkt.
\end{bewertung}



