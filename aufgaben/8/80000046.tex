Die pellschen Zahlen $P_n$ sind definiert durch die Rekursionsformel
\begin{equation}
P_{n+1} = 2P_n + P_{n-1}
\label{80000046:rekursion}
\end{equation}
mit Startwerten $P_0=0$ und $P_1=1$.
\begin{teilaufgaben}
\item
Schreiben Sie die Rekursion mit Hilfe einer Matrix.
\item
Bestimmen Sie Eigenwerte und Eigenvektoren der Matrix
\item
Finden Sie eine Formel für $P_n$ im Stile der Binet-Formel für die
Fibonacci-Zahlen.
\item
Berechnen Sie den Grenzwert
\[
\lim_{n\to \infty} \frac{P_{n+1}}{P_n}.
\]
Er ist bekannt als das Verhältnis des {\em slibernen Schnittes}, der
in der japanischen Ästhetik als Verhältnis besonderer Harmonie gilt
(\begin{CJK}{UTF8}{min}大和比\end{CJK}, Yamato-Hi).
Man findet Rechtecke mit Seitenverhältnissen des silbernen Schnittes
zum Beispiel in der Architektur japanischer Tempel.
\end{teilaufgaben}


\begin{loesung}
\definecolor{darkred}{rgb}{0.8,0,0}
Für spätere Beispielrechnungen führen wir die Rekursion
\eqref{80000046:rekursion}
für einige Schritte aus.
Die Folge der pellschen Zahlen ist also
\[
0,\;
1, \;
5,\;
12,\;
29,\;
70,\;
169,\;
408,\;
985,\;
2378,\;
5741,\;
13860,\;
33461,\dots
\]
\begin{teilaufgaben}
\item
Indem wir wie bei den Fibonacci-Zahlen die pellschen Zahlen in Vektoren
der Form.
\[
\begin{pmatrix}
P_{n+1}\\
P_n
\end{pmatrix}
=
P
\begin{pmatrix}
P_n\\
P_{n-1}
\end{pmatrix}
\]
Die zugehörige Matrix $P$ ist
\[
P
=
\begin{pmatrix}
2&1\\
1&0
\end{pmatrix}.
\]
\item
Die Eigenwerte von $P$ können mit der charakteristischen Gleichung
\begin{align*}
0
&=
\det(P-\lambda I)
\\
&=
\left|
\begin{matrix}
2-\lambda &     1    \\
    1     & -\lambda
\end{matrix}
\right|
\\
&=
-(2-\lambda)\lambda-1
\\
&=
\lambda^2-2\lambda -1.
\end{align*}
Sie hat die Nullstellen
\[
\lambda_{\pm}
=
1\pm\!\sqrt{(-1)^2+1}
=
1\pm\!\sqrt{2}
=
\left\{
\renewcommand{\arraycolsep}{2pt}
\begin{array}{ccl}
1+\!\sqrt{2} &=& \phantom{-}2.4142135\dots \\
1-\!\sqrt{2} &=&           -0.4142135\dots
\end{array}
\right.
\]
Für später beachten wir noch, dass 
\begin{equation}
\renewcommand{\arraycolsep}{2pt}
\begin{array}{cccccccl}
    & 2-\lambda_+ &=& 2-(1+\!\sqrt{2}) &=& 1-\!\sqrt{2} &=& \lambda_- \\
\text{und}\qquad
    & 2-\lambda_- &=& 2-(1-\!\sqrt{2}) &=& 1+\!\sqrt{2} &=& \lambda_+.
\end{array}
\label{80000046:lambda}
\end{equation}
Für den Gauss-Algorithmus brauchen wir ausserdem die Kehrwerte
\begin{align*}
\frac{1}{\lambda_+}
&=
\frac{1}{1+\!\sqrt{2}}
=
\frac{1-\!\sqrt{2}}{(1+\!\sqrt{2})(1-\!\sqrt{2})}
=
\frac{\lambda_-}{1-2}=-\lambda_-
\\
\frac{1}{\lambda_-}
&=
\frac{1}{1-\!\sqrt{2}}
=
\frac{1+\!\sqrt{2}}{(1-\!\sqrt{2})(1+\!\sqrt{2})}
=
\frac{\lambda_+}{1-2}=-\lambda_+.
\end{align*}

Für die beiden Eigenwerte müssen jetzt die zugehörigen Eigenvektoren
berechnet werden.
Für $\lambda_+=1+\!\sqrt{2}$ ergibt sich wegen 
\eqref{80000046:lambda}
\[
\begin{tabular}{|>{$}c<{$}>{$}c<{$}|>{$}c<{$}|}
\hline
v_1&v_2&1\\
\hline
2-\lambda_+ &      1     & 0 \\
    1       & -\lambda_+ & 0 \\
\hline
\end{tabular}
=
\begin{tabular}{|>{$}c<{$}>{$}c<{$}|>{$}c<{$}|}
\hline
v_1&v_2&1\\
\hline
\lambda_- &     1     & 0\\
    1     &-\lambda_+ & 0\\
\hline
\end{tabular}
\to
\begin{tabular}{|>{$}c<{$}>{$}c<{$}|>{$}c<{$}|}
\hline
v_1&v_2&1\\
\hline
    1     &-\lambda_+ & 0\\
    0     &     0     & 0\\
\hline
\end{tabular}
\]
Eine mögliche Lösung ist
\begin{align*}
v_+ = \begin{pmatrix} \lambda_+\\1\end{pmatrix}
\qquad\text{mit}\qquad
Pv_+
&=
\begin{pmatrix}2&1\\1&0\end{pmatrix}
\begin{pmatrix}\lambda_+\\1\end{pmatrix}
=
\begin{pmatrix}2\lambda_++1\\\lambda_+\end{pmatrix}
=
\lambda_+
\begin{pmatrix} 2 +\frac{1}{\lambda_+} \\ 1\end{pmatrix}
=
\lambda_+
\begin{pmatrix} 2 -\lambda_- \\ 1\end{pmatrix}
\\
&=
\lambda_+
\begin{pmatrix} \lambda_+ \\ 1\end{pmatrix}
=
\lambda_+v_+
\end{align*}
Für den Eigenwert $\lambda_-$ ergibt die analoge Rechnung den Eigenvektor
$v_-$ aus dem Gauss-Tableau
\[
\begin{tabular}{|>{$}c<{$}>{$}c<{$}|>{$}c<{$}|}
\hline
v_1&v_2&1\\
\hline
2-\lambda_- &      1     & 0 \\
    1       & -\lambda_- & 0 \\
\hline
\end{tabular}
=
\begin{tabular}{|>{$}c<{$}>{$}c<{$}|>{$}c<{$}|}
\hline
v_1&v_2&1\\
\hline
\lambda_+ &     1     & 0\\
    1     &-\lambda_- & 0\\
\hline
\end{tabular}
\to
\begin{tabular}{|>{$}c<{$}>{$}c<{$}|>{$}c<{$}|}
\hline
v_1&v_2&1\\
\hline
    1     &-\lambda_- & 0\\
    0     &     0     & 0\\
\hline
\end{tabular}
\]
Eine mögliche Lösung ist
\begin{align*}
v_- = \begin{pmatrix} \lambda_-\\1\end{pmatrix}
\qquad\text{mit}\qquad
Pv_-
&=
\begin{pmatrix}2&1\\1&0\end{pmatrix}
\begin{pmatrix}\lambda_-\\1\end{pmatrix}
=
\begin{pmatrix}2\lambda_-+1\\\lambda_-\end{pmatrix}
=
\lambda_-
\begin{pmatrix} 2 +\frac{1}{\lambda_-} \\ 1\end{pmatrix}
=
\lambda_-
\begin{pmatrix} 2 -\lambda_+ \\ 1\end{pmatrix}
\\
&=
\lambda_-
\begin{pmatrix} \lambda_- \\ 1\end{pmatrix}
=
\lambda_-v_-.
\end{align*}
Damit sind die Eigenvektoren $v_{\pm}$ zu den Eigenwerten $\lambda_{\pm}$
gefunden.
\item
Um eine Formel für die pellschen Zahlen zu finden schreiben wir die Folge
mit Hilfe der Matrixpotenz
\[
\begin{pmatrix}
P_{n}\\
P_{n-1}
\end{pmatrix}
=
P^n
\begin{pmatrix}1\\0\end{pmatrix}.
\]
Um die Potenz zu berechnen, müssen wir den Vektor auf der rechten Seite
in der Eigenbasis schreiben.
Wir müssen also das Vektorgleichungssystem
\[
a_+v_+ + a_-v_- = \begin{pmatrix}1\\0\end{pmatrix}
\qquad\Leftrightarrow\qquad
\begin{array}{rcrcl}
\lambda_+ {\color{darkred}a_+} &+& \lambda_- {\color{darkred}a_-} &=& 1 \\
          {\color{darkred}a_+} &+&           {\color{darkred}a_-} &=& 0
\end{array}
\]
lösen.
Mit der Cramerschen Formel finden wir die Lösungen
\begin{align*}
a_+
&=
\frac{
\left|\begin{matrix}1&\lambda_-\\0&1\end{matrix}\right|
}{
\left|\begin{matrix}\lambda_+&\lambda_-\\1&1\end{matrix}\right|
}
=
\frac{1}{\lambda_+-\lambda_-}
=
\frac{1}{2\!\sqrt{2}}
\\
a_-
&=
\frac{
\left|\begin{matrix}\lambda_+&1\\1&0\end{matrix}\right|
}{
\left|\begin{matrix}\lambda_+&\lambda_-\\1&1\end{matrix}\right|
}
=
\frac{-1}{\lambda_+-\lambda_-}
=
-\frac{1}{2\!\sqrt{2}}.
\end{align*}
Einsetzen ergibt jetzt die Binet-ähnliche Formel
\begin{equation}
P_n
=
\frac{\lambda_+^n - \lambda_-^n}{2\!\sqrt{2}}
=
\frac{(1+\!\sqrt{2})^n}{2\!\sqrt{2}}
-
\frac{(1-\!\sqrt{2})^n}{2\!\sqrt{2}}
\label{80000046:binet}
\end{equation}
für die pellschen Zahlen.
\item
Der Quotient aufeinanderfolgender pellscher Zahlen wird mit der Formel
\eqref{80000046:binet}
\begin{align}
\frac{P_{n+1}}{P_n}
&=
\frac{\lambda_+^{n+1}-\lambda_-^{n+1}}{\lambda_+^n-\lambda_-^n}
=
\lambda_+
\frac{1 - \lambda_-^{n+1}/\lambda_+^{n+1}}{1+\lambda_-^n/\lambda_+^n}
=
\lambda_+
\frac{1-q^{n+1}}{1-q^{n}}
\label{80000046:grenzwert}
\end{align}
mit $q=\lambda_-/\lambda_+$.
Da $|q|<0$ ist, konvergiert der zweite Term in Zähler und Nenner von
\eqref{80000046:grenzwert} jeweils gegen $0$ und es folgt
\[
\lim_{n\to\infty}
\frac{P_{n+1}}{P_n}
=
\lambda_+
\lim_{n\to\infty}
\frac{1-q^{n+1}}{1-q^{n}}
=
\lambda_+.
\]
Der Quotient aufeinanderfolgender pellscher Zahlen konvergiert also
gegen das Verhältnis des silbernen Schnittes.
\qedhere
\end{teilaufgaben}
\end{loesung}

