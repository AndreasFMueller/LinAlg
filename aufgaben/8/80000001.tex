Bestimmen Sie alle Eigenwerte und die zugehörigen Eigenvektoren der Matrix
\[
A=\begin{pmatrix}
1&1&0\\
1&0&1\\
0&1&1
\end{pmatrix}.
\]

\thema{Eigenwerte}
\thema{Eigenvektoren}
\thema{charakteristisches Polynom}

\begin{loesung}
Das charakteristische Polynom ist
\begin{align*}
\left|\;\begin{matrix}
1-\lambda&1&0\\
1&-\lambda&1\\
0&1&1-\lambda
\end{matrix}\;\right|
&=
-(1-\lambda)^2\lambda-(1-\lambda)-(1-\lambda)\\
&=
-(1-\lambda)^2\lambda-2(1-\lambda)\\
&=
(1-\lambda)(-\lambda(1-\lambda)-2)\\
&=
(1-\lambda)(\lambda^2-\lambda-2)\\
&=
(1-\lambda)(\lambda -2 )(\lambda+1)
\end{align*}
Dieses Polynom hat die Nullstellen $\lambda_1=-1$, $\lambda_2=1$
und $\lambda_3=2$. Für jeden dieser Werte müssen jetzt auch noch
Eigenvektoren gefunden werden.

Für $\lambda=-1$ muss das homogene Gleichungssystem mit der Matrix
$A-\lambda E=A+E$ gelöst werden:
\begin{align*}
\begin{tabular}{|>{$}c<{$}>{$}c<{$}>{$}c<{$}|}
\hline
2&1&0\\
1&1&1\\
0&1&2\\
\hline
\end{tabular}
\rightarrow
\begin{tabular}{|>{$}c<{$}>{$}c<{$}>{$}c<{$}|}
\hline
1&\frac12&0\\
0&\frac12&1\\
0&      1&2\\
\hline
\end{tabular}
\rightarrow
\begin{tabular}{|>{$}c<{$}>{$}c<{$}>{$}c<{$}|}
\hline
1&\frac12&0\\
0&      1&2\\
0&      1&2\\
\hline
\end{tabular}
\rightarrow
\begin{tabular}{|>{$}c<{$}>{$}c<{$}>{$}c<{$}|}
\hline
1&\frac12&0\\
0&      1&2\\
0&      0&0\\
\hline
\end{tabular}
\rightarrow
\begin{tabular}{|>{$}c<{$}>{$}c<{$}>{$}c<{$}|}
\hline
1&0&-1\\
0&1&2\\
0&0&0\\
\hline
\end{tabular}
\end{align*}
Die dritte Variable ist frei wählbar, man kann sie zum Beispiel
willkürlich auf $1$ festlegen, und erhält als Eigenvektor
\[
v_1=\begin{pmatrix}1\\-2\\1\end{pmatrix}.
\]
Tatsächlich ergibt
\[
Av_1
=
\begin{pmatrix}1&1&0\\1&0&1\\0&1&1\end{pmatrix}
\begin{pmatrix}1\\-2\\1\end{pmatrix}
=
\begin{pmatrix}
-1\\2\\-1
\end{pmatrix}=-v_1,
\]
$v_1$ ist also wirklich ein Eigenvektor zum Eigenwert $\lambda_1=-1$.

Analog kann man Eigenvektoren für die anderen Eigenwerte finden.
Für $\lambda_2=1$ gilt
\begin{align*}
\begin{tabular}{|>{$}c<{$}>{$}c<{$}>{$}c<{$}|}
\hline
0&1&0\\
1&-1&1\\
0&1&0\\
\hline
\end{tabular}
\rightarrow
\begin{tabular}{|>{$}c<{$}>{$}c<{$}>{$}c<{$}|}
\hline
1&-1&1\\
0&1&0\\
0&0&0\\
\hline
\end{tabular}
\rightarrow
\begin{tabular}{|>{$}c<{$}>{$}c<{$}>{$}c<{$}|}
\hline
1&0&1\\
0&1&0\\
0&0&0\\
\hline
\end{tabular},
\end{align*}
was auf den Eigenvektor
\[
v_2=\begin{pmatrix}-1\\0\\1\end{pmatrix}
\]
führt.

Oder für $\lambda_3=2$ findet man das Gleichungssystem
\begin{align*}
\begin{tabular}{|>{$}c<{$}>{$}c<{$}>{$}c<{$}|}
\hline
-1&1&0\\
1&-2&1\\
0&1&-1\\
\hline
\end{tabular}
\rightarrow
\begin{tabular}{|>{$}c<{$}>{$}c<{$}>{$}c<{$}|}
\hline
1&-1&0\\
0&-1&1\\
0&1&-1\\
\hline
\end{tabular}
\rightarrow
\begin{tabular}{|>{$}c<{$}>{$}c<{$}>{$}c<{$}|}
\hline
1&-1&0\\
0&1&-1\\
0&0&0\\
\hline
\end{tabular}
\rightarrow
\begin{tabular}{|>{$}c<{$}>{$}c<{$}>{$}c<{$}|}
\hline
1&0&-1\\
0&1&-1\\
0&0&0\\
\hline
\end{tabular}
\end{align*}
mit dem Eigenvektor
\[
v_3=\begin{pmatrix}1\\1\\1\end{pmatrix}.
\qedhere
\]
\end{loesung}

