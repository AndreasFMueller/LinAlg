Betrachten Sie die Matrix
\[
A=\begin{pmatrix}11&t & 0\\2 & 4 & 0 \\ 0 & 0 & 2\end{pmatrix}.
\]

\begin{teilaufgaben}
\item
Berechnen Sie das charakteristische Polynom von $A$.
\item
Wie muss man $t$ wählen, damit $A$ einen Eigenwert $\lambda = 10$ hat? 
Bestimmen Sie auch die anderen Eigenwerte für diesen Fall.
\item
Finden Sie einen Eigenvektor zum Eigenwert $\lambda = 10$.
\item
Ist $A$ in diesem Fall diagonalisierbar? 
Falls ja, bestimmen Sie die diagonalisierte Matrix $A'$
\end{teilaufgaben}

\thema{Eigenwerte}
\thema{Eigenvektoren}
\thema{charakteristisches Polynom}
\thema{diagonalisierbar}

\begin{loesung}
\begin{teilaufgaben}
\item
Das charakteristische Polynom ist
\begin{align*}
\det(A-\lambda E) 
=
\left|\begin{matrix}
11-\lambda &t & 0\\
2 & 4-\lambda & 0 \\ 
0 & 0 & 2-\lambda
\end{matrix}\right|
&=
(2-\lambda)((11-\lambda)\cdot(4-\lambda)-2t)\\
&=
(2-\lambda)(\lambda^2 - 15\lambda + 44 -2t )
\end{align*}
\item
Die Eigenwerte sind die Nullstellen des charakteristischen Polynoms.
Wenn wir also den Eigenwert $\lambda = 10$ einsetzen muss es Null geben:
\begin{align*}
(2-10)(10^2 - 15\cdot 10 + 44 -2t )
&= -8\cdot(100 - 150 + 44 -2t ) \\
&= 48+16t = 0 &\Rightarrow\quad & t = -\dfrac{48}{16} = -3.
\end{align*}
Durch einsetzen des gefundenen Wertes für $t$ können wir nun auch die 
anderen Eigenwerte bestimmen.
\begin{align*}
(2-\lambda)(\lambda^2 - 15\lambda + 44 -2\cdot (-3) ) 
&= (2-\lambda)(\lambda^2 - 15\lambda + 50 )\\
&= (2-\lambda)(\lambda-10)(\lambda-5).
\end{align*}
Die Nullstellen bzw. die Eigenwerte sind folglich
\[
\lambda_1 = 2,\qquad \lambda_2 = 10\qquad \text{und}\qquad \lambda_3 = 5.
\]
\item
Einen Eigenvektor zum Eigenwert $\lambda = 10$ finden wir durch lösen des
Gleichungssystems $(A-\lambda E)\vec v = 0$, wozu wir den Gaussalgorithmus
verwenden:
\begin{align*}
\begin{tabular}{|>{$}c<{$}>{$}c<{$}>{$}c<{$}|>{$}c<{$}|}
\hline
11-\lambda&t        &0&0\\
    2   &4-\lambda&0&0\\
0 & 0 & 2-\lambda & 0\\
\hline
\end{tabular}
&=
\begin{tabular}{|>{$}c<{$}>{$}c<{$}>{$}c<{$}|>{$}c<{$}|}
\hline
1 &-3        &0&0\\
    2   &-6&0&0\\
0 & 0 & -8 & 0\\
\hline
\end{tabular}
\rightarrow
\begin{tabular}{|>{$}c<{$}>{$}c<{$}>{$}c<{$}|>{$}c<{$}|}
\hline
1 &-3        &0&0\\
0 &0&0&0\\
0 & 0 & 1 & 0\\
\hline
\end{tabular}
\end{align*}
Daraus liest man ab, dass 
\[
\vec v=\begin{pmatrix}3\\1\\0\end{pmatrix}
\]
ein Eigenvektor ist.
Zur Kontrolle berechnen wir
\[
A\vec v
=
\begin{pmatrix}
11 &-3        &0\\
    2   &4&0\\
0 & 0 & 2 \\
\end{pmatrix}
\begin{pmatrix}3\\1\\0\end{pmatrix}
=
\begin{pmatrix}30\\10\\0\end{pmatrix} = 10\vec v.
\]
Der Vektor $\vec v$ ist also tatsächlich ein Eigenvektor zum Eigenwert $10$.
\item
Da wir drei verschiedene Eigenwerte gefunden haben, wissen wir, dass es zu jedem
Eigenwert einen linear unabhängigen Eigenvektor gibt. Die Matrix $A$ ist folglich
diagonalisierbar und die diagonalisierte Matrix $A'$ enthält auf der Diagonalen die
Eigenwerte. Es ist folglich die Matrix
\[
A'=\begin{pmatrix}2&0 & 0\\0 & 10 & 0 \\ 0 & 0 & 5\end{pmatrix}.
\]
\qedhere
\end{teilaufgaben}
\end{loesung}

\begin{bewertung}
Charakteristisches Polynom ({\bf X}) 1 Punkt,
Bedingung für $t$ ({\bf B}) 1 Punkt,
Wert von $t$ ({\bf T}) 1 Punkt,
Eigenwerte ({\bf E}) 1 Punkt,
Eigenvektor ({\bf V}) 1 Punkt,
Begründung Diagonalisierbarkeit und diagonalisierte Matrix ({\bf D}) 1 Punkt.
\end{bewertung}

