Ist die Matrix
\[
A=\begin{pmatrix}
 1&16& 0\\
-1& 9& 0\\
 0& 0&-2
\end{pmatrix}
\]
diagonalisierbar?

\thema{Eigenwerte}
\thema{Eigenvektoren}
\thema{charakteristisches Polynom}
\thema{diagonalisierbar}

\begin{loesung}
Wir müssen untersuchen, ob es eine Basis aus Eigenvektoren gibt.
Zunächst ist klar, dass $\lambda=-2$ ein Eigenwert ist und $e_3$ der
zugehörige Eigenvektor, wir müssen also nur noch die $2\times 2$-Matrix
\[
A_0=\begin{pmatrix}
 1&16\\
-1& 9
\end{pmatrix}
\]
in der linken oberen Ecke von $A$ studieren.
Dazu berechnen wir das charakteristische Polynom
\begin{align*}
\chi_{A_0}(\lambda)
&=
\left|\;\begin{matrix}
1-\lambda&16\\
-1&9-\lambda
\end{matrix}\;\right|
=
(1-\lambda)(9-\lambda)+16
\\
&=
9-10\lambda+\lambda^2+16
=
\lambda^2 -10\lambda + 25 
=
(\lambda-5)^2.
\end{align*}
Das charakteristische Polynom hat die doppelte Nullstelle $\lambda=5$.

An dieser Stelle könnten wir eine Abkürzung nehmen: wäre $A_0$
diagonalisierbar, dann müssten $A_0 = \lambda E$ sein.
Da dies nicht der Fall ist, können wir schliessen, dass $A_0$ und
damit $A$ nicht diagonalisierbar ist.

In etwas mehr Detail bestimmen wir Eigenvektoren mit Hilfe des
Gauss-Algorithmus.
Dazu bearbeiten wir das Tableau
\begin{align*}
\begin{tabular}{|>{$}c<{$}>{$}c<{$}|}
\hline
1-\lambda&16\\
-1&9-\lambda\\
\hline
\end{tabular}
&=
\begin{tabular}{|>{$}c<{$}>{$}c<{$}|}
\hline
-4&16\\
-1&4\\
\hline
\end{tabular}
\rightarrow
\begin{tabular}{|>{$}c<{$}>{$}c<{$}|}
\hline
1&-4\\
0&0\\
\hline
\end{tabular}.
\end{align*}
Daraus lesen wir ab, dass alle Eigenvektoren von $A_0$ Vielfache von
\[
v_\lambda = v_5 = \begin{pmatrix}4\\1\end{pmatrix}
\]
sein müssen.
Insbesondere gibt es keine Basis aus Eigenvektoren des zweidimensionalen
Raumes $\mathbb R^2$, also ist $A_0$ und damit auch $A$ nicht diagonalisierbar.
\end{loesung}

\begin{bewertung}
Erster Eigenwert -2 ({\bf E}) 1 Punkt,
charakteristisches Polynome ({\bf X}) 1 Punkt,
doppelter Eigenwert 5 ({\bf D}) 1 Punkt,
Eigenvektor zu $\lambda=-2$ ({\bf V$\mathstrut_{-2}$}) 1 Punkt,
Eigenvektor zu $\lambda=5$ ({\bf V$\mathstrut_5$}) 1 Punkt,
Nicht diagonalisierbarkeit ({\bf N}) 1 Punkt.
\end{bewertung}
