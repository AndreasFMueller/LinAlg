Betrachten Sie die Matrix
\[
A
=
\begin{pmatrix}
3-s &  1 \\
-s^2&s+3
\end{pmatrix}
\]
mit $s\ne 0$.
\begin{teilaufgaben}
\item 
Bestimmen Sie die Eigenwerte von $A$.
\item
Finden Sie so viele linear unabhängige Eigenvektoren wie möglich.
\item
Ist die Matrix diagonalisierbar?
\end{teilaufgaben}

\begin{loesung}
\begin{teilaufgaben}
\item 
Wir berechnen das charakteristische Polynom
\[
\chi_A(\lambda)
=
\left|
\begin{matrix}
3-s-\lambda & 1 \\
-s^2 & s+3-\lambda
\end{matrix}
\right|
=
(3-\lambda+s)(3-\lambda-s)+s^2
=
(3-\lambda)^2-s^2+s^2
=
(\lambda-3)^2.
\]
Die Eigenwerte sind Nullstellen des charakteristischen Polynoms, also
Lösungen der Gleichung $\chi_A(\lambda)=(\lambda-3)^2 = 0$.
$\lambda=3$ ist eine doppelte Nullstelle, dies ist also der einzige Eigenwert.
\item 
Für die Eigenvektoren müssen wir das homogene Gleichungssystem mit 
Koeffizientenmatrix $A-3E$ mit dem Gauss-Algorithmus lösen:
\begin{align*}
\begin{tabular}{|>{$}c<{$}>{$}c<{$}|>{$}c<{$}|}
\hline
3-s-\lambda & 1 & 0\\
 -s^2 & s+3-\lambda& 0\\
\hline
\end{tabular}
&=
\begin{tabular}{|>{$}c<{$}>{$}c<{$}|>{$}c<{$}|}
\hline
-s & 1 & 0\\
-s^2 & s & 0\\
\hline
\end{tabular}
\\
& \rightarrow
\begin{tabular}{|>{$}c<{$}>{$}c<{$}|>{$}c<{$}|}
\hline
1 & -\frac{1}s & 0\\
0 & 0 & 0\\
\hline
\end{tabular}
\end{align*}
Die Division durch das Pivot-Element $-s$ im ersten Schritt ist möglich,
weil $s\ne 0$ in der Aufgabenstellung vorausgesetzt war.
Die zweite Variable ist frei wählbar, zum Beispiel könnten wir den Wert $s$
wählen.
Dann wird ein möglicher Eigenvektor:
\[
v
=
\begin{pmatrix}1\\s\end{pmatrix}.
\]
Alle anderen Eigenvektoren sind davon linear abhängig.

Kontrolle:
\[
Av
=
\begin{pmatrix}
3-s &  1 \\
-s^2&s+3
\end{pmatrix}
\begin{pmatrix}1\\s\end{pmatrix}
=
\begin{pmatrix}
(3-s)+s\\
-s^2+(s+3)s
\end{pmatrix}
=
\begin{pmatrix}
3\\
3s
\end{pmatrix}
=
3v.
\]
\item
Da der Eigenraum von $\lambda=3$ nur eindimensional ist, gibt es keine
Basis aus Eigenvektoren und die Matrix $A$ ist daher nicht diagonalisierbar.
\qedhere
\end{teilaufgaben}
\end{loesung}

\begin{bewertung}
Charakteristisches Polynome ({\bf X}) 2 Punkte,
Eigenwerte ({\bf L}) 1 Punkt,
Gauss-Algorithmus ({\bf G}) 1 Punkt,
Eigenvektor ({\bf V}) 1 Punkt,
Diagonalisierbarkeit ({\bf D}) 1 Punkt.
\end{bewertung}
