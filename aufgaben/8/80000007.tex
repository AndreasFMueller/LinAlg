Warum ist eine $n\times n$-Matrix $A$ mit einem Eigenwert 0
immer singulär?

\thema{Eigenwerte}
\themaL{singular}{singulär}

\begin{loesung}
Wenn $\lambda=0$ ein Eigenwert ist, dann ist $0$ eine Nullstelle
von $\det(A-\lambda I)=0$ für $\lambda=0$, also $\det(A)=0$,
das sind aber genau die singulären Matrizen.
\end{loesung}

