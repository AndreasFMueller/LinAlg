Ist die Matrix
\[
A
=
\begin{pmatrix}
-1&  0& 0\\
 9&-15& 7\\
18&-28&13
\end{pmatrix}
\]
diagonalisierbar?

\thema{Eigenwerte}
\thema{Eigenvektoren}
\thema{charakteristisches Polynom}
\thema{diagonalisierbar}

\begin{loesung}
Wir müssen eine Basis aus Eigenvektoren suchen.
Dazu berechnen wir zunächst das charakteristische Polynom
\begin{align*}
\chi_{A}(\lambda)
&=
\left|\;\begin{matrix}
-1-\lambda &  0         &  0        \\
 9         &-15-\lambda &  7        \\
18         &-28         & 13-\lambda
\end{matrix}\;\right|
=
(1+\lambda)(15+\lambda)(13-\lambda)
-(-28)\cdot 7\cdot(-1-\lambda)
\\
&=
-\lambda^3-3\lambda^2+193\lambda+195
-196-196\lambda
=
-\lambda^3-\lambda^2-3\lambda-1
\\
&=
-(\lambda +1)^3.
\end{align*}
Das charakteristische Polynom hat die dreifache Nullstelle $\lambda=-1$.
Wir suchen nach Eigenvektoren mit Hilfe des Gaussalgorithmus:
\begin{align*}
\begin{tabular}{|>{$}c<{$} >{$}c<{$} >{$}c<{$}|}
\hline
-1-\lambda &  0         &  0         \\
 9         &-15-\lambda &  7         \\
18         &-28         & 13-\lambda \\
\hline
\end{tabular}
&=
\begin{tabular}{|>{$}c<{$} >{$}c<{$} >{$}c<{$}|}
\hline
 0         &  0         &  0       \\
 9         &-14         &  7       \\
18         &-28         & 14       \\
\hline
\end{tabular}
\rightarrow
\begin{tabular}{|>{$}c<{$} >{$}c<{$} >{$}c<{$}|}
\hline
 9         &-14         &  7       \\
 0         &  0         &  0       \\
 0         &  0         &  0       \\
\hline
\end{tabular}
\end{align*}
Wir haben die Nullzeile im zweiten Tableau nach unten permutiert.
Damit keine Brüche auftreten, haben wir hier auf die Pivot-Divison
verzichtet und direkt das zweifache der ersten Zeile von der letzten
subtrahiert.
Wie man sehen kann, ist der Rang der Matrix $1$, es gibt also nur
zwei linear unabhängige Eigenvektoren.
Insbesondere ist die Matrix $A$ nicht diagonalisierbar.

Wir finden zum Beispiel die beiden Vektoren
\[
v_1
=
\begin{pmatrix}14\\9\\0\end{pmatrix}
\quad\text{und}\quad
v_2
=
\begin{pmatrix}-7\\0\\9\end{pmatrix}
\]
als Eigenvektoren.

Kontrolle:
\begin{align*}
Av_1
&=
\begin{pmatrix}
-1&  0& 0\\
 9&-15& 7\\
18&-28&13
\end{pmatrix}
\begin{pmatrix}14\\9\\0\end{pmatrix}
=
\begin{pmatrix}
-14\\
9\cdot 14-9\cdot 15\\
18\cdot14-28\cdot 9
\end{pmatrix}
=
\begin{pmatrix}
-14\\
9\cdot 14-9\cdot 15\\
-9\\
0
\end{pmatrix}
=
-1\cdot
\begin{pmatrix}14\\9\\0\end{pmatrix}
=
\lambda v_1,
\\
Av_2
&=
\begin{pmatrix}
-1&  0& 0\\
 9&-15& 7\\
18&-28&13
\end{pmatrix}
\begin{pmatrix}-7\\0\\9\end{pmatrix}
=
\begin{pmatrix}
7\\
-63+63\\
18\cdot (-7) + 13\cdot 9
\end{pmatrix}
=
\begin{pmatrix}
7\\0\\-9
\end{pmatrix}
=
-1\cdot
\begin{pmatrix}-7\\0\\9\end{pmatrix}
=
\lambda v_2.
\qedhere
\end{align*}
\end{loesung}
