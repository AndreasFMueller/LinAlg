Betrachten Sie die Matrix
\[
A
=
\begin{pmatrix}
1&\frac1t\\
-2t&4
\end{pmatrix}
\]
mit $t\ne 0$.
\begin{teilaufgaben}
\item
Bestimmen Sie die Eigenwerte von $A$.
\item
Bestimmen Sie so viele linear unabhängige Eigenvektoren von $A$ wie
möglich.
\item
Ist die Matrix $A$ diagonalisierbar?
\end{teilaufgaben}

\begin{loesung}
\begin{teilaufgaben}
\item
Das charakteristische Polynom von $A$ ist
\[
\chi_A(\lambda)
=
\det(A-\lambda I)
=
\left|\begin{matrix}
1-\lambda&\frac1t\\
-2t&4-\lambda
\end{matrix}\right|
=
(1-\lambda)(4-\lambda)+2
=
\lambda^2-5\lambda+6
=
(\lambda-2)(\lambda-3).
\]
Die Nullstellen von $\chi_A(\lambda)$ sind die Eigenwerte
$\lambda=2$ und $\lambda=3$.
\item
Für die Eigenvektoren müssen jetzt die homogenen linearen Gleichungssysteme
mit Koeffizientenmatrix $A-\lambda I$ gelöst werden:
\begin{align*}
\lambda&=2:&
\begin{tabular}{|>{$}c<{$}>{$}c<{$}|>{$}c<{$}|}
\hline
1-\lambda&\frac1t&0\\
-2t&4-\lambda&0\\
\hline
\end{tabular}
&=
\begin{tabular}{|>{$}c<{$}>{$}c<{$}|>{$}c<{$}|}
\hline
-1&\frac1t&0\\
-2t&2&0\\
\hline
\end{tabular}
&\lambda&=3:&
\begin{tabular}{|>{$}c<{$}>{$}c<{$}|>{$}c<{$}|}
\hline
1-\lambda&\frac1t&0\\
-2t&4-\lambda&0\\
\hline
\end{tabular}
&=
\begin{tabular}{|>{$}c<{$}>{$}c<{$}|>{$}c<{$}|}
\hline
-2&\frac1t&0\\
-2t&1&0\\
\hline
\end{tabular}
\\
&&&\rightarrow
\begin{tabular}{|>{$}c<{$}>{$}c<{$}|>{$}c<{$}|}
\hline
1&-\frac1t&0\\
0&0&0\\
\hline
\end{tabular}
&&&&\rightarrow
\begin{tabular}{|>{$}c<{$}>{$}c<{$}|>{$}c<{$}|}
\hline
1&-\frac1{2t}&0\\
0&0&0\\
\hline
\end{tabular}
\\
&&\Rightarrow\quad v_2&=
\begin{pmatrix}1\\t\end{pmatrix},
&&&\Rightarrow\quad v_3&=
\begin{pmatrix}1\\2t\end{pmatrix}.
\end{align*}
Kontrolle:
\begin{align*}
l&=2:&
Av_2
&=
\begin{pmatrix}
1&\frac1t\\
-2t&4
\end{pmatrix}
\begin{pmatrix}1\\t\end{pmatrix}
=
\begin{pmatrix}
1+t\frac1t\\
-2t+4t
\end{pmatrix}
=
\begin{pmatrix}2\\2t\end{pmatrix}
=
2 \begin{pmatrix}1\\t\end{pmatrix}
=
\lambda v_2
\\
l&=3:&
Av_3
&=
\begin{pmatrix}
1&\frac1t\\
-2t&4
\end{pmatrix}
\begin{pmatrix}1\\2t\end{pmatrix}
=
\begin{pmatrix}
1+\frac1t\cdot 2t\\
-2t+8t
\end{pmatrix}
=
\begin{pmatrix}3\\6t\end{pmatrix}
=
3\begin{pmatrix}1\\2t\end{pmatrix}
=
\lambda v_3.
\end{align*}
\item
Da es zwei linear unabhängige Eigenvektoren und damit eine Basis aus
Eigenvektoren gibt, ist die Matrix $A$ diagonalisierbar.
\qedhere
\end{teilaufgaben}
\end{loesung}

\begin{diskussion}
Für die Diagonalisierbarkeit ist es nicht nötig, die Transformationsmatrix
zu berechnen.
\end{diskussion}

\begin{bewertung}
Charakteristisches Polynom ({\bf X}) 2 Punkte,
Eigenwerte ({\bf E}) 1 Punkt,
Gauss-Algorithmus und Eigenvektoren ({\bf G}) 2 Punkte (je ein Punkt
für jeden Eigenvektor,
Diagonalisierbarkeit ({\bf D}) 1 Punkt.
\end{bewertung}
