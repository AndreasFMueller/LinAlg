Zwei Eigenvektoren der Matrix
\[
A
= 
\begin{pmatrix*}[r]
   -9 & -4 \\
   20 &  9 
\end{pmatrix*}
\]
zu den Eigenwerten $1$ und $-1$ sind
\[
v_1
=
\begin{pmatrix*}[r]
2 \\ -5
\end{pmatrix*}
\quad\text{und}\quad
v_2
=
\begin{pmatrix*}[r]
1 \\ -2
\end{pmatrix*}
\]
\begin{teilaufgaben}
\item
Welche Matrix $C$ bildet die Standardbasisvektoren $e_1$ und $e_2$
auf $v_1$ und $v_2$ ab.
\item
Bestimmen Sie $ACe_1$ und $ACe_2$ direkt aus der Definition
\item
Bestimmen Sie $C^{-1}ACe_1$ und $C^{-1}ACe_2$ direkt aus der Definition
\item
Bestimmen Sie die Matrix $D=C^{-1}AC$ direkt aus der Definition
\end{teilaufgaben}

\begin{loesung}
\begin{teilaufgaben}
\item
Die Matrix $C$ enthält die Vektoren $v_1$ und $v_2$ als Spalten:
\[
C
=
\begin{pmatrix*}[r]
 2& 1\\
-5&-2
\end{pmatrix*}.
\]
\item
$ACe_1=Av_1=v_1$
und
$ACe_2=Av_2=v_2$
\item
$C^{-1}ACe_1=C^{-1}v_1 = e_1$
und
$C^{-1}ACe_2=C^{-1}v_2 = -e_2$
\item
Wegen $De_1=e_1$ und $De_2=-e_2$ bedeutet, dass 
\[
D
=
\begin{pmatrix*}[r]
1&0\\0&-1
\end{pmatrix*}.
\qedhere
\]
\end{teilaufgaben}
Die Matrix $C^{-1}$ wird nicht gebraucht, kann aber leicht mit Hilfe
der Minoren gefunden werden.
Es ist
\[
C^{-1}
=
\begin{pmatrix*}[r]
-2&-1\\
 5& 2
\end{pmatrix*}
.
\qedhere
\]
\end{loesung}
