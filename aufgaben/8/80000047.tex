Eine Folge von grösser werdenden Rechtecken wird wie folgt konstruiert:
Einem Rechteck wird an jeder langen Seite ein Quadrat mit der langen Seite
als Seitenlänge angesetzt.
Der Prozess wird mit dem so entstandenen neuen Rechteck wiederholt.
Es entsteht eine Folge $(R_n)_{n\in\mathbb{N}}$ von Rechtecken
mit Seitenlängen $u_n$ und $v_n$, wobei $u_n>v_n$.
Berechnen Sie den Grenzwert $\lim_{n\to\infty} \frac{u_n}{v_n}$.

\begin{loesung}
$u_n$ ist die lange Seite des Rechtecks $R_n$.
Die Konstruktion fügt zwei Quadrate mit Seitenlänge $u_n$ an, wodurch
die kurze Seite $v_n$ auf $v_n+2u_n$ anwächst. 
Das neue Rechteck hat daher die Seitenlängen $u_{n+1}=2u_n + v_n$ und
$v_{n+1}=u_n$.
Nach Wiederholung der Konstruktion kann man $v_n = u_{n-1}$ schreiben.
Durch Einsetzen in der ersten Formel folgt die Rekursionsformel
\begin{equation}
u_{n+1} = 2u_n + u_{n-1},
\label{80000047:rekursion}
\end{equation}
die auch in der Aufgabe~\ref{80000046} für die pellschen Zahlen
aufgetreten ist.

Die Lösung von Aufgabe~\ref{80000046} hat gezeigt, dass Lösungen
der Rekursionsgleichung \eqref{80000047:rekursion} in der Form
\begin{equation}
\begin{pmatrix} u_n\\v_n\end{pmatrix}
=
P^n
\begin{pmatrix} u_0\\v_0\end{pmatrix}
\label{800000467:loesung}
\end{equation}
geschrieben werden können.
Indem man den Vektor auf der rechten Seite linear aus den Eigenvektoren
$v_+$ und $v_-$ in der Form
\[
\begin{pmatrix}u_0\\v_0\end{pmatrix}
=
a_+v_+ + a_-v_-
\]
kombiniert, kann man auch \eqref{800000467:loesung} als
\begin{align*}
\begin{pmatrix}u_n\\v_n\end{pmatrix}
&=
P^n(a_+v_++a_-v_-)
=
a_+P^nv_++a_-P^nv_-
=
a_+
\lambda_+^n
\begin{pmatrix} \lambda_+\\1\end{pmatrix}
+
a_-
\lambda_-^n
\begin{pmatrix} \lambda_-\\1\end{pmatrix},
\intertext{dessen Komponenten}
%\Rightarrow\qquad
u_n &= a_+\lambda_+^{n+1} + a_-\lambda_-^{n+1} \\
v_n &= a_-\lambda_-^{n\phantom{\mathstrut+1}}   + a_-\lambda_-^{n\phantom{\mathstrut+1}}
\end{align*}
sind.
Für das Verhältnis der beiden Seiten folgt daher
\begin{align*}
\frac{u_n}{v_n}
&=
\frac{
a_+\lambda_+^{n+1}+a_-\lambda_-^{n+1}
}{
a_+\lambda_+^{n}+a_-\lambda_-^{n}
}
=
\lambda_+
\frac{
a_+ + a_-\lambda_-^{n+1}/\lambda_+^{n+1}
}{
a_+ + a_-\lambda_-^{n}/\lambda_+^{n}
}
=
\lambda_+
\frac{a_++a_-q^{n+1}}{a_++a_-q^{n\phantom{\mathstrut+1}}}
\end{align*}
mit $q=\lambda_-/\lambda_+$ wie in Aufgaben~\ref{80000046}.
Im Gegensatz zur Aufgabe~\ref{80000046} kann man jetzt aber nicht mehr 
sofort schliessen, dass der Bruch auf der rechten Seite gegen 1 konvergiert.
Falls $a_+\ne 0$ ist folgt dies tatsächlich immer noch.
Wenn aber $a_+=0$ ist, dann folgt
\[
\lim_{n\to\infty} \frac{u_n}{v_n}
=
\lim_{n\to\infty} 
\lambda_+
\frac{a_-q^{n+1}}{a_-q^n}
=
\lambda_+
\lim_{n\to\infty} q
=
\lambda_+q
=
\lambda_-.
\]
Für $n=0$ gilt dann aber
\begin{align*}
u_0 &= a_-\lambda_- \\
v_0 &= a_-.
\end{align*}
Da $\lambda_-<0$ ist, müssten die Seitenlängen des Ausgangsrechtecks
verschiedene Vorzeichen haben.
Dies zeigt, dass der Fall $a_+=0$ für Rechtecke mit positiven Seitenlängen
nicht auftreten kann.
Für solche Rechtecke konvergiert das Verhältnis der Recheckseiten
immer gegen $\lambda_+=1+\!\sqrt{2}$.
\end{loesung}
