Berechnen Sie den Flächeninhalt des Quadrates mit den Ecken
$A=(1,-2)$, $B=(5,1)$, $C=(2,5)$ und $D=(-2,2)$ auf zwei Arten:

\hbox to\hsize{%
\begin{minipage}{0.5\textwidth}
\begin{teilaufgaben}
\item Mit der Schuhbändelformel
\item Berechnen Sie den Flächeninhalt aus der Länge der Strecke $AB$
\end{teilaufgaben}
\end{minipage}%
\begin{minipage}{0.5\textwidth}
\begin{center}
\def\punkt#1{
	\fill[color=white] #1 circle[radius=0.08];
	\draw[color=darkred] #1 circle[radius=0.08];
}
\begin{tikzpicture}[>=latex,thick,scale=0.6]
\coordinate (A) at (1,-2);
\coordinate (B) at (5,1);
\coordinate (C) at (2,5);
\coordinate (D) at (-2,2);
\draw[->] (-2.8,0) -- (6.2,0) coordinate[label={$x$}];
\draw[->] (0,-2.1) -- (0,5.5) coordinate[label={right:$y$}];
\fill[color=darkred!20,opacity=0.5] (A) -- (B) -- (C) -- (D) -- cycle;
\draw[color=darkred] (A) -- (B) -- (C) -- (D) -- cycle;
\punkt{(A)}
\punkt{(B)}
\punkt{(C)}
\punkt{(D)}
\node at (A) [left] {$A$};
\node at (B) [right] {$B$};
\node at (C) [right] {$C$};
\node at (D) [left] {$D$};
\end{tikzpicture}
\end{center}
\end{minipage}}

\begin{loesung}
\begin{teilaufgaben}
\item Mit der Schuhbändelformel bekommt man
\begin{align*}
F
&=
\frac12
\left|
\begin{matrix*}[r]
 1&-2\\
 5& 1\\
 2& 5\\
-2& 2
\end{matrix*}
\right|
\\
&=
\frac12
\left|
\begin{matrix*}[r]
 1& -2\\
 5&  1
\end{matrix*}
\right|
+
\frac12
\left|
\begin{matrix*}[r]
 5&  1\\
 2&  5
\end{matrix*}
\right|
+
\frac12
\left|
\begin{matrix*}[r]
 2&  5\\
-2&  2
\end{matrix*}
\right|
+
\frac12
\left|
\begin{matrix*}[r]
-2&  2\\
 1& -2
\end{matrix*}
\right|
\\
&=
\frac12(1+10)
+
\frac12(25-2)
+
\frac12(4+10)
+
\frac12(4-2)
=
\frac12(11+23+14+2)
=
\frac12\cdot 50
=
25.
\end{align*}
\item
Die Länge der Strecke $AB$ ist
\begin{align*}
\overline{AB}
&=
\sqrt{(5-1)^2+(1-(-2))^2}
=
\sqrt{4^2+3^2}
=
\sqrt{16+9}
=
\sqrt{25}
=
5.
\end{align*}
Es folgt, dass der Flächeninhalt $F=5^2=25$ sein muss, in Übereinstimmung
mit dem Resultat von a).
\qedhere
\end{teilaufgaben}
\end{loesung}

