Gegeben ist eine Ebene $\sigma_1$, welche durch die Punkte
\[
A=(5,0,0), \qquad
B=(0,8,0) \qquad \text{und}\qquad  
C=(0,0,2)
\]
geht.

\begin{teilaufgaben}
\item
Finden Sie eine Ebene $\sigma_2$ in Normalenform, welche parallel zur 
Ebene $\sigma_1$ ist und durch den Punkt $S=(-1,-1,-1)$ geht.
\item 
Finden Sie einen Punkt $Q$, der auf der Geraden
\begin{equation}
\vec{p}
=
\vec{p}_0 + t\vec{r}
=
\begin{pmatrix}5\\-2\\3\end{pmatrix}
+
t
\begin{pmatrix}-3\\2\\1\end{pmatrix},
\label{50000029:gerade}
\end{equation}
liegt und von beiden Ebenen $\sigma_1$ und $\sigma_2$ gleich weit entfernt ist.
\end{teilaufgaben}


\thema{Normalenform}
\thema{Vektorprodukt}
\thema{Durchstosspunkt}

\begin{loesung}
\begin{teilaufgaben}
\item
Für die Normalenform benötigen wir einen Normalenvektor $\vec n$. Da die 
beiden Ebenen jedoch parallel sind, haben sie den selben Normalenvektor.
Wir können $\vec n$ also als Kreuzprodukt der beiden Richtungsvektoren
\[
\vec r_1 = \vec b-\vec a = \begin{pmatrix}-5\\ 8\\ 0\end{pmatrix}
\qquad \text{und}\qquad 
\vec r_2 = \vec c-\vec a = \begin{pmatrix}-5\\ 0\\ 2\end{pmatrix}
\]
berechnen:
\[
\vec n = \vec r_1 \times \vec r_2 
= \begin{pmatrix}-5\\ 8\\ 0\end{pmatrix} \times \begin{pmatrix}-5\\ 0\\ 2\end{pmatrix} 
= \begin{pmatrix}8\cdot2 - 0\cdot 0\\ 0\cdot(-5) - 2\cdot (-5)\\ -5\cdot 0 - (-5)\cdot 8\end{pmatrix} 
= \begin{pmatrix}16\\ 10\\ 40\end{pmatrix}.
\]
Mit dem Punkt S, der in der Ebene $\sigma_2$ enthalten ist, können wir 
die Ebenengleichung in Normalenform aufstellen:
\[
  \vec n \cdot (\vec p -\vec s ) = 
  \begin{pmatrix}16\\ 10\\ 40\end{pmatrix} \cdot \begin{pmatrix}x-(-1)\\ y-(-1)\\ z-(-1)\end{pmatrix}
  = 16x + 10 y + 40 z +66 = 0.
\]
\item 
Der Punkt Q ist der Schnittpunkt der Geraden (\ref{50000029:gerade}) und der Ebene $\sigma_3$, welche 
zwischen den Ebenen $\sigma_1$ und $\sigma_2$ liegt. Die Ebene $\sigma_3$ hat wiederum den selben
Normalenvektor $\vec n$ und ein Punkt, der auf der Ebene liegt ist
\[
  \vec e = \dfrac{1}{2}(\vec a + \vec s) = \dfrac{1}{2}\begin{pmatrix}5-1\\0-1\\0-1\end{pmatrix}
  = \begin{pmatrix}2\\-0.5\\-0.5\end{pmatrix}
\]
Damit könne wir auch für $\sigma_3$ die Ebenengleichung in Normalenform aufstellen:
\[
  \vec n \cdot (\vec p -\vec e ) = 
  \begin{pmatrix}16\\ 10\\ 40\end{pmatrix} \cdot \begin{pmatrix}x-2\\ y-(-0.5)\\ z-(-0.5)\end{pmatrix}
  = 16x + 10 y + 40 z -7 = 0.
\]
Den Schnittpunkt $Q$ finden wir und mit dem Gauss-Tableau:
\begin{align*}
\begin{tabular}{|>{$}c<{$}>{$}c<{$}>{$}c<{$}>{$}c<{$}|>{$}c<{$}|}
\hline
\color{red}x&\color{red}y&\color{red}z&\color{red}t&\\
\hline
 1& 0& 0& 3& 5\\
 0& 1& 0& -2& -2\\
 0& 0& 1& -1&  3\\
16& 10& 40& 0&7\\
\hline
\end{tabular}
&\rightarrow...\rightarrow
\begin{tabular}{|>{$}c<{$}>{$}c<{$}>{$}c<{$}>{$}c<{$}|>{$}c<{$}|}
\hline
\color{red}x&\color{red}y&\color{red}z&\color{red}t&\\
\hline
 1& 0& 0& 0& 48.25\\
 0& 1& 0& 0& -30.8333\\
 0& 0& 1& 0& -11.4167\\
 0& 0& 0& 1& -14.4167\\
\hline
\end{tabular}
\end{align*}
Daraus lesen wir ab, dass $Q=(48.25,-30.83,-11.42)$ ist.
\end{teilaufgaben}
\end{loesung}

\begin{bewertung}
\begin{teilaufgaben}
\item
Richtungsvektoren ({\bf R}) 1 Punkt,
Normale ({\bf N}) 1 Punkt,
Ebenengleichung $\sigma_2$ ({\bf E}) 1 Punkt,
\item
Abstandsbedingung ({\bf A}) 1 Punkt,
Gleichung für Abstandsbedingung (z.B. $\sigma_3$) ({\bf G}) 1 Punkt,
Bestimmung des Punktes ({\bf Q}) 1 Punkt,
\end{teilaufgaben}
\end{bewertung}





