Gegeben sind die beiden Vektoren
\[
\vec{a}_0
=
\begin{pmatrix}
  -12\\
    1\\
  -12
\end{pmatrix}
\qquad\text{und}\qquad
\vec{a}_1
=
\begin{pmatrix}
   -3\\
   12\\
    4
\end{pmatrix}.
\]
\begin{teilaufgaben}
\item
Berechnen Sie das Vektorprodukt $\vec{a}_2 = \vec{a}_0\times\vec{a}_1$.
\item
Berechnen Sie Länge und Zwischenwinkel der Vektoren $\vec{a}_0$ und $\vec{a}_1$.
\item
Wir definieren rekursiv
$\vec{a}_{n+1} = \vec{a}_0\times \vec{a}_n$.
Berechnen Sie $\vec{a}_{13}$.
\end{teilaufgaben}

\thema{Zwischenwinkel}
\thema{Vektorprodukt}

\begin{loesung}
\begin{teilaufgaben}
\item
Das Vektorprodukt ist
\[
\vec{a}_0\times\vec{a}_1
=
\begin{pmatrix}
  -12\\
    1\\
  -12
\end{pmatrix}
\times
\begin{pmatrix}
   -3\\
   12\\
    4
\end{pmatrix}
=
\begin{pmatrix}
1\cdot 4-12\cdot(-12)\\
(-12)\cdot(-3)-4\cdot(-12)\\
(-12)\cdot12-1\cdot(-3)
\end{pmatrix}
=
\begin{pmatrix}
148\\84\\-141
\end{pmatrix}.
\]
\item Die Längen der beiden Vektoren sind
\begin{align*}
\vec{a}_0
&=
\sqrt{2\cdot 144+1}
=
17,
\\
\vec{a}_1
&=
\sqrt{9+144+16} = 13.
\end{align*}
Da wir das Vektorprodukt bereits berechnet haben, können wir die
Zwischenwinkelformel
\[
\sin\alpha = \frac{|\vec{a}_0\times\vec{a}_1|}{|\vec{a}_0|\cdot|\vec{a}_1|}
\]
für das Vektorprodukt verwenden, um den Zwischenwinkel zu berechnen.
Dazu brauchen wir die Länge des Vektorproduktes, sie ist
\[
|\vec{a}_0\times\vec{a}_1|
=
\sqrt{148^2 + 84^2 + 141^2}
=
\sqrt{48841} = 221.
\]
Damit erhalten wir jetzt für den Zwischenwinkel
\[
\sin\alpha = \frac{221}{17\cdot 13} = 1
\quad\Rightarrow\quad
\sin\alpha=90^\circ,
\]
die Vektoren stehen also senkrecht aufeinander.
\item
Die Rekursionsformel multipliziert immer wieder von links mit $\vec{a}_0$,
alle Vektoren $\vec{a}_n$ sind daher orthogonal auf $\vec{a}_0$.
Die Richtung des Vektors $\vec{a}_{n+1}$ entsteht aus der Richtung
des Vektors $\vec{a}_{n\mathstrut}$, indem man ihn um $90^\circ$ um die
Achse $\vec{a}_0$ dreht.
Somit hat der Vektor $\vec{a}_n$ immer die gleiche Richtung wie
$\vec{a}_{n-4}$.
Im Speziellen hat $\vec{a}_{13}$ die gleiche Richtung wie $\vec{a}_1$.

Wir müssen jetzt nur noch die Länge von $\vec{a}_{13}$ bestimmen.
Da alle Vektoren orthogonal sind, ist die Länge des Vektorprodukts
einfach das Produkt der Länge der Faktoren, also
$|\vec{a}_n|=|\vec{a}_0|\cdot|\vec{a}_{n-1}|$.
Daraus folgt
\[
\vec{a}_n
=
|\vec{a}_0|^{n-1} |\vec{a}_1|
=
13^{n-1}\cdot 17.
\]
Damit können wir jetzt $\vec{a}_{13}$ bestimmen:
\[
\vec{a}_{13}
=
17^{13-1}\cdot \vec{a}_1
=
17^{12}\cdot
\begin{pmatrix}
   -3\\
   12\\
    4
\end{pmatrix}
=
\begin{pmatrix}
-1747866711689283 \\
6991466846757132 \\
2330488948919044
\end{pmatrix}.
\qedhere
\]
\end{teilaufgaben}
\end{loesung}

%ans =  17
%ans =  13
%a2 =
%
%   148
%    84
%  -141

\begin{bewertung}
\begin{teilaufgaben}
\item 1 Punkt
\item Längen ({\bf L}) 2 Punkte, Zwischenwinkel ({\bf W}) 1 Punkt
\item Lösungsverfahren mit Rekursion ({\bf R}) 1 Punkt,
Vektor $\vec{a}_{13}$ ({\bf V}) 1 Punkt.
\end{teilaufgaben}
\end{bewertung}
