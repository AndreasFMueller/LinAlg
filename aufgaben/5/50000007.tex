K"onnen Sie die symmetrische Matrix
\[
A=\begin{pmatrix}
    4&   2&   6\\
    2&   2&   4\\
    6&   4&  14
\end{pmatrix}
\]
als Produkt $A=BB^t$ schreiben? Geben Sie eine m"ogliche Matrix $B$ an.

\begin{loesung}
Man kann sogar eine untere Dreiecksmatrix $L$ mit dieser Eigesnchaft finden,
dies ist die Cholesky-Zerlegung von $A$.

Im ersten Schritt sucht man die erste Spalte von $L$ zu bestimmen.
Es muss gelten
\[
LL^t=
\begin{pmatrix}
l_{11}&  0&  0\\
l_{21}&  ?&  0\\
l_{31}&  ?&  ?
\end{pmatrix}
\begin{pmatrix}
l_{11}&l_{21}&l_{31}\\
     0&     ?&     ?\\
     0&     0&     ?
\end{pmatrix}
=
\begin{pmatrix}
    l_{11}^2&l_{11}l_{21}&l_{11}l_{31}\\
l_{21}l_{11}&           *&           *\\
l_{31}l_{11}&           *&           *
\end{pmatrix}
=
\begin{pmatrix}
    4&   2&   6\\
    2&   2&   4\\
    6&   4&  14
\end{pmatrix}
\]
Daraus kann man ablesen, dass $l_{11}=2$ sein muss, und weiter,
dass
$l_{21}=1$ und $l_{31}=3$. Damit ist die erste Spalte bestimmt.

Im zweiten Schritt versucht man, die zweite Spalte zu bestimmen.
Dazu schreibt man wieder
\[
LL^t
=
\begin{pmatrix}
2&     0&0\\
1&l_{22}&0\\
3&l_{32}&?
\end{pmatrix}
\begin{pmatrix}
2&     1&     3\\
0&l_{22}&l_{32}\\
0&     0&?
\end{pmatrix}
=
\begin{pmatrix}
4&2           &           6\\
2&1+l_{22}^2  &3+l_{22}l_{32}\\
6&3+l_{32}l_{22}&         *
\end{pmatrix}
=
\begin{pmatrix}
    4&   2&   6\\
    2&   2&   4\\
    6&   4&  14
\end{pmatrix}
\]
Daraus liest man ab $1+l_{22}^2=2$ und damit $l_{22}=1$, und weiter
$l_{32}=1$.

Im dritten Schritt ist jetzt nur noch das Element unten rechts zu bestimmen:
\[
LL^t=
\begin{pmatrix}
2&0&     0\\
1&1&     0\\
3&1&l_{33}
\end{pmatrix}
\begin{pmatrix}
2&1&     3\\
0&1&     1\\
0&0&l_{33}
\end{pmatrix}
=
\begin{pmatrix}
    4&   2&   6\\
    2&   2&   4\\
    6&   4& 10+l_{33}^2
\end{pmatrix}
=
\begin{pmatrix}
    4&   2&   6\\
    2&   2&   4\\
    6&   4&  14
\end{pmatrix}
\]
woraus man ablesen kann: $10+l_{33}^2=14$ oder $l_{33}=2$. Die gesuchte
Matrix $L$ ist also
\[
L=
\begin{pmatrix}
2&0&0\\
1&1&0\\
3&1&2
\end{pmatrix}.
\]
Kontrolle:
\[
\begin{pmatrix}
2&0&0\\
1&1&0\\
3&1&2
\end{pmatrix}
\begin{pmatrix}
2&1&3\\
0&1&1\\
0&0&2
\end{pmatrix}
=
\begin{pmatrix}
    4&   2&   6\\
    2&   2&   4\\
    6&   4&  14
\end{pmatrix}.
\]
\end{loesung}

\begin{bewertung}
Cholesky-Zerlegung ({\bf C}) 1 Punkt,
Pro richtigem nicht verschwindendem Matrix-Element in $B$ ein Punkt,
maximal 6 Punkte.
\end{bewertung}
