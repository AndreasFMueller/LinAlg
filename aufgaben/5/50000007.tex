Gegeben sind
\[
A=\begin{pmatrix}1&2\\2&1\\2&1\end{pmatrix},\qquad
b=\begin{pmatrix}2\\2\\2\end{pmatrix}.
\]
Das Gleichungssystem $Ax=b$ ist "uberbestimmt, finden Sie die Least-Squares
L"osung mit Hilfe der QR-Zerlegung.

\begin{loesung}
F"ur die Matrix $A$ muss zun"achst eine QR-Zerlegung gefunden werden.
Mit Octave findet man zum Beispiel
\verbatimainput{qroctave}
Die rechte Seite des Gleichungssystems nach dem L"osungsverfahren 
unter Verwendung der QR-Zerlegung ist $Q^tb$:
\verbatimainput{qrb}
Das Gleichungssystem wird damit zu
\[
\begin{linsys}{3}
-3.00000x&-&2.00000y&=&-3.33333\\
         & &1.41421y&=&\phantom{-}0.94281,
\end{linsys}
\]
mit der L"osung
\[
\begin{pmatrix}
0.66666\\
0.66666
\end{pmatrix}.
\]
Man kann die QR-Zerlegung nat"urlich auch mit dem Verfahren von Gram-Schmidt
von Hand bestimmen. Dazu orthonormalisiert man die Vektoren
\[
b_1=\begin{pmatrix}1\\2\\2\end{pmatrix}
\quad\text{und}\quad
b_2=\begin{pmatrix}2\\1\\1\end{pmatrix}.
\]
Man findet:
\begin{align*}
b_1'&=\frac{b_1}{|b_1|}=\frac13\begin{pmatrix}1\\2\\2\end{pmatrix}\\
b_2'&
=\frac{b_2 - (b_2\cdot b_1') b_1'}{\dots}
=\frac{b_2 - 2 b_1'}{\dots}
=\frac{\displaystyle\begin{pmatrix}2\\1\\1\end{pmatrix}-\frac23\begin{pmatrix}1\\2\\2\end{pmatrix}
}{\dots}
=\frac{\displaystyle\frac13\begin{pmatrix}4\\-1\\-1\end{pmatrix}}{\dots}
=\frac{\displaystyle\frac13\begin{pmatrix}4\\-1\\-1\end{pmatrix}}{\sqrt{2}}
=\frac1{3\sqrt{2}}\begin{pmatrix}4\\-1\\-1 \end{pmatrix}
\end{align*}
Aus dieser Rechnung kann man auch ablesen, dass 
\[
R=\begin{pmatrix}
3&2\\
0&\sqrt{2}\\
0&0
\end{pmatrix}.
\]
Dies deckt sich bis auf die Wahl des Vorzeichens von $b_1'$ mit der
maschinell gefundenen L"osung.
\end{loesung}

