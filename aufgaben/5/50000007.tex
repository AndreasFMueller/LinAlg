Gegeben sind
\[
A=\begin{pmatrix}1&2\\2&3\\2&4\end{pmatrix},\qquad
b=\begin{pmatrix}2\\2\\2\end{pmatrix}.
\]
Das Gleichungssystem $Ax=b$ ist überbestimmt, finden Sie die Least-Squares
Lösung mit Hilfe der QR-Zerlegung.

\thema{QR-Zerlegung}

\begin{loesung}
Für die Matrix $A$ muss zunächst eine QR-Zerlegung gefunden werden.
Mit Octave findet man zum Beispiel
\verbatimainput{qroctave}
Die rechte Seite des Gleichungssystems nach dem Lösungsverfahren 
unter Verwendung der QR-Zerlegung ist $Q^tb$:
\verbatimainput{qrb}
Das Gleichungssystem wird damit zu
\[
\begin{linsys}{3}
-3.00000x&-&5.33333y&=&-3.33333\\
         & &0.74536y&=&\phantom{-}0.29814
\end{linsys}
\]
mit der Lösung
\[
\begin{pmatrix}
0.4\\
0.4
\end{pmatrix}.
\]
Man kann die QR-Zerlegung natürlich auch mit dem Verfahren von Gram-Schmidt
von Hand bestimmen. Dazu orthonormalisiert man die Vektoren
\[
b_1=\begin{pmatrix}1\\2\\2\end{pmatrix}
\quad\text{und}\quad
b_2=\begin{pmatrix}2\\3\\4\end{pmatrix}.
\]
Man findet:
\begin{align*}
b_1'&=\frac{b_1}{|b_1|}=\frac13\begin{pmatrix}1\\2\\2\end{pmatrix}\\
b_2'&
=\frac{b_2 - (b_2\cdot b_1') b_1'}{\dots}
=\frac{b_2 - \frac{16}{3} b_1'}{\dots}
=\frac{\displaystyle\begin{pmatrix}2\\3\\4\end{pmatrix}-\frac{16}9\begin{pmatrix}1\\2\\2\end{pmatrix}
}{\dots}
=\frac{\displaystyle\frac19\begin{pmatrix}18-16\\27-32\\36-32\end{pmatrix}}{\dots}
=\frac{\displaystyle\frac19\begin{pmatrix}2\\-5\\4\end{pmatrix}}{\sqrt{\frac{5}{9}}}
=\frac1{\sqrt{45}}\begin{pmatrix}2\\-5\\4\end{pmatrix}
\end{align*}
Daraus findet man zunächst die Matrix 
\[
Q
=
\begin{pmatrix}
\frac13&\frac{ 2}{\sqrt{45}}\\
\frac23&\frac{-4}{\sqrt{45}}\\
\frac23&\frac{ 5}{\sqrt{45}}
\end{pmatrix}
\]
Man kann aus dem Ausdruck auch ablesen, dass
\[
\frac{16}{3}b_1'+\sqrt{\frac{5}{9}}b_2'=b_2,
\]
also ist die $R$-Matrix
\[
R
=
\begin{pmatrix}
3&\frac{16}{3}\\
0&\sqrt{\frac{5}{9}}
\end{pmatrix}.
\]
Dies deckt sich bis auf die Wahl des Vorzeichens von $b_1'$ mit der
maschinell gefundenen Lösung.
Wir berechnen zur Kontrolle das Produkt $QR$:
\[
QR=
\begin{pmatrix}
\frac13&\frac{ 2}{\sqrt{45}}\\
\frac23&\frac{-4}{\sqrt{45}}\\
\frac23&\frac{ 5}{\sqrt{45}}
\end{pmatrix}
\begin{pmatrix}
3&\frac{16}{3}\\
0&\sqrt{\frac{5}{9}}
\end{pmatrix}
=
\begin{pmatrix}
1&\frac13\cdot \frac{16}{3}+\frac{2}{\sqrt{45}}\cdot\sqrt{\frac{5}{9}}\\
2&\frac23\cdot \frac{16}{3}-\frac{5}{\sqrt{45}}\cdot\sqrt{\frac{5}{9}}\\
2&\frac23\cdot \frac{16}{3}+\frac{4}{\sqrt{45}}\cdot\sqrt{\frac{5}{9}}
\end{pmatrix}
=
\begin{pmatrix}
1&\frac{16}{9}+\frac{2}{9}\\
2&\frac{32}{9}-\frac{5}{9}\\
2&\frac{32}{9}+\frac{3}{9}
\end{pmatrix}
=
\begin{pmatrix}
1&\frac{18}{9}\\
2&\frac{27}{9}\\
2&\frac{36}{9}
\end{pmatrix}
=
\begin{pmatrix}
1&2\\
2&3\\
2&4
\end{pmatrix}
=A.
\]
Damit ist bestätigt, dass die Matrizen $Q$ und $R$ die QR-Zerlegung der
Matrix $A$ bilden.
\end{loesung}

