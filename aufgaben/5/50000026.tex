Die drei Ebenen $\sigma_1$, $\sigma_2$ und $\sigma_3$ durch den Nullpunkt
haben die Normalen
\[
\vec n_1=
\begin{pmatrix}1\\2\\1\end{pmatrix},
\qquad
\vec n_2=
\begin{pmatrix}3\\0\\4\end{pmatrix}
\qquad\text{und}\qquad
\vec n_3=
\begin{pmatrix}5\\1\\0\end{pmatrix}.
\]
Sei $g_{ij}$ die Schnittgerade der Ebenen $\sigma_i$ und $\sigma_j$.
Finden Sie die Gleichung der Ebene, die die Schnittgeraden $g_{12}$
und $g_{23}$ enthält.

\thema{Schnittgerade}
\thema{Schnittpunkt}

\begin{loesung}
Da die Ebene durch den Nullpunkt geht, genügt es, deren Normale $\vec n$
zu bestimmen.
Sind
$\vec n_{12}$
und
$\vec n_{23}$
die Richtungsvektoren der beiden Geraden, dann ist die gesuchte Normale
$\vec n=\vec n_{12}\times\vec n_{23}$.
Die Richtungsvektoren stehen aber auch auf den beiden Normalen senkrecht,
daher gilt auch
\begin{align*}
\vec n_{12}
&=
\vec n_1\times\vec n_2
&
&\wedge&
\vec n_{23}
&=
\vec n_2\times\vec n_3
&
&\Rightarrow&
\vec n
&=
(\vec n_1\times\vec n_2)
\times
(\vec n_2\times\vec n_3).
\\
\vec n_{12}&=
\begin{pmatrix}1\\2\\1\end{pmatrix}
\times
\begin{pmatrix}3\\0\\4\end{pmatrix}
=
\begin{pmatrix}8\\-1\\-6\end{pmatrix}
&&\wedge&
\vec n_{23}
&=
\begin{pmatrix}3\\0\\4\end{pmatrix}
\times
\begin{pmatrix}5\\1\\0\end{pmatrix}
=
\begin{pmatrix}-4\\20\\3\end{pmatrix}
&
&\Rightarrow&
\vec n_{12}\times\vec n_{23}
&=
\begin{pmatrix}8\\-1\\-6\end{pmatrix}
\times
\begin{pmatrix}-4\\20\\3\end{pmatrix}
=
\begin{pmatrix}117\\0\\156\end{pmatrix}=39\begin{pmatrix}3\\0\\4\end{pmatrix}
\end{align*}
Die Ebenengleichung ist daher
\[
3x+4z=0.
\]
Man kann natürlich die Vektoren ${\vec n}_{12}$ und ${\vec n}_{23}$
auch für eine Parameterdarstellung der Ebene verwenden:
\[
\vec p
=
s {\vec n}_{12} + t {\vec n}_{23}
=
s\begin{pmatrix} 8\\-1\\-6 \end{pmatrix}
+
t\begin{pmatrix} -4\\20\\3 \end{pmatrix}.
\qedhere
\]
\end{loesung}

\begin{bewertung}
6 Punkte.
\end{bewertung}

