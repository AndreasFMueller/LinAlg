Die beiden Ebenen durch den Nullpunkt mit Parameterdarstellungen
\[
\vec{p}
=
s\vec{r}_1
+
t\vec{r}_2
=
s
\begin{pmatrix*}[r]
   2\\
   2\\
  -1
\end{pmatrix*}
+
t
\begin{pmatrix*}[r]
 -2\\
  4\\
 -5
\end{pmatrix*}
\qquad
\text{und}
\qquad
\vec{p}
=
s\vec{r}_3
+
t\vec{r}_4
=
s
\begin{pmatrix*}[r]
  -3\\
  -5\\
   4
\end{pmatrix*}
+
t
\begin{pmatrix*}[r]
 -6\\
 -1\\
  2
\end{pmatrix*}
\]
schneiden sich mit dem Zwischenwinkel $\alpha$.
Bestimmen Sie den Sinus
des Zwischenwinkels $\sin\alpha$.

\begin{hinweis}
Der Zwischenwinkel $\alpha$ muss nicht bestimmt werden.
\end{hinweis}

\begin{loesung}
Der Zwischenwinkel zwischen den Ebenen ist auch der Zwischenwinkel zwischen
den Normalenvektoren, es sind also zuerst die Vektorprodukt
\begin{align*}
\vec{n}_1
&=
\begin{pmatrix*}[r]
   2\\
   2\\
  -1
\end{pmatrix*}
\times
\begin{pmatrix*}[r]
 -2\\
  4\\
 -5
\end{pmatrix*}
=
\begin{pmatrix*}[r]
-6\\
12\\
\phantom{-}12
\end{pmatrix*}
=
6\cdot
\begin{pmatrix*}[r]
-1\\
 2\\
 2
\end{pmatrix*}
,
&
|\vec{n}_1|
&=
6\cdot 3,
\\
\vec{n}_2
&=
\begin{pmatrix*}[r]
  -3\\
  -5\\
   4
\end{pmatrix*}
\times
\begin{pmatrix*}[r]
 -6\\
 -1\\
  2
\end{pmatrix*}
=
\begin{pmatrix*}[r]
 -6\\
-18\\
-27
\end{pmatrix*}
=
3\cdot
\begin{pmatrix*}[r]
 -2\\
 -6\\
 -9
\end{pmatrix*}
,&
|\vec{n}_2|
&=
3\cdot 11
\end{align*}
zu berechnen.
Für den Sinus des Zwischenwinkels können wir die Zwischenwinkelformel
mit dem Vektorprodukt verwenden:
\[
\sin\alpha
=
\frac{|\vec{n}_1\times\vec{n}_2|}{|\vec{n}_1|\cdot|\vec{n}_2|}.
\]
Das Vektorprodukt ist
\begin{align*}
\vec{n}_1\times\vec{n}_2
&=
6\cdot
3\cdot
\begin{pmatrix*}[r]
-1\\
 2\\
 2
\end{pmatrix*}
\times
\begin{pmatrix*}[r]
-2\\
-6\\
-9
\end{pmatrix*}
=
18\cdot
\begin{pmatrix*}[r]
 -6\\
-13\\
 10
\end{pmatrix*}
\\
|\vec{n}_1\times\vec{n}_2|
&=
6\cdot 3\cdot \sqrt{36+169+100}
=
6\cdot 3\cdot \sqrt{305},
\intertext{woraus wir jetzt den Sinus des Zwischenwinkels}
\sin\alpha
=
\frac{|\vec{n}_1\times\vec{n}_2|}{|\vec{n}_1|\cdot|\vec{n}_2|}
&=
\frac{6\cdot 3\cdot \sqrt{305}}{6\cdot 3\cdot 3 \cdot 11}
=
\frac{\sqrt{305}}{33}
=
0.52921967
\end{align*}
bekommen.
\end{loesung}

\begin{bewertung}
Berechung der beiden Normalen ({\bf N}) je 1 Punkt,
Vektorprodukt der Normalen ({\bf V}) 1 Punkt,
Länge der Normalen ({\bf L}) 1 Punkt,
Zwischenwinkelformel mit dem Vektorprodukt ({\bf Z}) 1 Punkt,
Berechnung des Sinus ({\bf S}) 1 Punkt.
\end{bewertung}

