Sei $A$ die Matrix
\[
A=\frac13\begin{pmatrix}
2&-2& 1\\
1& 2& 2\\
2& 1&-2
\end{pmatrix}.
\]
\begin{teilaufgaben}
\item Ist $A$ orthogonal?
\item Zeigen Sie: $A$ ist keine Drehmatrix.
\item Finden Sie die Matrix $S$ einer Spiegelung an der $x$-$y$-Ebene
\item Berechnen Sie $AS$
\item Zeigen Sie: $AS$ ist eine Drehmatrix
\item Bestimmen Sie den Drehwinkel von $AS$
\end{teilaufgaben}

\thema{orthogonale Matrix}
\thema{Spiegelung}
\thema{Drehmatrix}
\thema{Drehwinkel}

\begin{loesung}
\begin{teilaufgaben}
\item $A^tA=E$ durch nachrechnen, also ist $A$ orthogonal.
\item $\det(A)=-1$, eine Drehmatrix hätte $\det(A)=1$.
\item Die Matrix $S$ ist
\[
S=\begin{pmatrix}1&0&0\\0&1&0\\0&0&-1\end{pmatrix},
\]
sie ist orthogonal und $\det(S)=-1$.
\item
\[
AS=\frac13\begin{pmatrix}
2&-2& 1\\
1& 2& 2\\
2& 1&-2
\end{pmatrix}
\begin{pmatrix}1&0&0\\0&1&0\\0&0&-1\end{pmatrix}
=
\frac13\begin{pmatrix}
2&-2&-1\\
1& 2&-2\\
2& 1& 2
\end{pmatrix}
\]
\item $AS$ ist als Produkt orthogonaler Matrizen auch orthogonal,
und $\det(AS)=\det(A)\det(S)=(-1)\cdot(-1)=1$.
\item Den Drehwinkel kann man mit der Spurformel ermitteln:
\begin{align*}
\cos\alpha&= \frac{\operatorname{Spur} A -1}2=\frac{\frac136-1}2=\frac12
\\
\alpha&=60^\circ
\qedhere
\end{align*}
\end{teilaufgaben}
\end{loesung}

