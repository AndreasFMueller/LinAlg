Gesucht ist ein orthonormiertes Koordinaten-System, in dem die
Drehung um die Achse mit Richtung
\[
\vec{r}
=
\begin{pmatrix} 2\\3\\6 \end{pmatrix}
\]
als Drehung um die dritte Achse beschrieben werden kann.
Der erste Basisvektor soll in der $x$-$y$-Ebene liegen und positive
$x$-Komponente haben.
Die Basis soll ein Rechtssystem sein.


\begin{loesung}
Der Vektor $\vec{b}_3$ muss ein Einheitsvektor mit Richung $\vec{r}$ sein,
also
\[
\vec{b}_3
=
\frac{\vec{r}}{|\vec{r}|}
=
\frac{1}{7}
\begin{pmatrix} 2\\3\\6 \end{pmatrix}.
\]
Der Vektor $\vec{b}_1$ soll sowohl auf $\vec{r}$ als auch auf der
Vertikalen $\vec{e}_3$ senkrecht stehen, dazu kann man das Vektorprodukt
\[
a_1
=
\vec{b}_3\times \vec{e}_3
=
\frac17
\begin{pmatrix} 2\\3\\6 \end{pmatrix}
\times
\begin{pmatrix} 0\\0\\1\end{pmatrix}
=
\frac17
\begin{pmatrix} 3\\-2\\0 \end{pmatrix}
\quad\Rightarrow\quad
|a_1|=\frac17 \sqrt{13}
\quad\Rightarrow\quad
\vec{b}_1
=
\frac{1}{\sqrt{13}}
\begin{pmatrix} 3\\-2\\0 \end{pmatrix}
\]
verwenden.
Der dritte Basisvektor $\vec{b}_2$ muss auf den beiden anderen senkrecht
stehen, dazu kann man wieder das Vektorprodukt verwenden.
Für jedes Paar $\vec{a}$ und $\vec{b}$ von Vektoren bilden die drei
Vektoren $\vec{a}$, $\vec{b}$ und $\vec{a}\times\vec{b}$ ein Rechtssystem,
daher bilden auch zyklische Vertauschungen davon ein Rechtssystem.
Da $\vec{b}_3$, $\vec{b}_1$ und $\vec{b}_3\times\vec{b}_1$ ein Rechtssystem
bilden, dann auch
$\vec{b}_1$, $\vec{b}_3\times\vec{b}_1$ und $\vec{b}_3$.
Der gesuchte Vektor $\vec{b}_2$ ist also
\[
\vec{b}_2
=
\vec{b}_3\times \vec{b}_1
=
\frac{1}{7}
\begin{pmatrix} 2\\3\\6 \end{pmatrix}
\times
\frac{1}{\sqrt{13}}
\begin{pmatrix} 3\\-2\\0 \end{pmatrix}
=
\frac{1}{7\sqrt{13}}
\begin{pmatrix}
12\\18\\-13
\end{pmatrix}.
\qedhere
\]
\end{loesung}

\begin{bewertung}
Normierung von $\vec{b}_3$ ({\bf B$\mathstrut_3$}) 1 Punkt,
Vektorprodukt und Normierung für $\vec{b}_1$ ({\bf B$\mathstrut_1$}) 2 Punkte,
Vektorprdoukt und Normierung für $\vec{b}_2$ ({\bf B$\mathstrut_2$}) 2 Punkte,
Orientierung der Basis ({\bf O}) 1 Punkt.
\end{bewertung}
