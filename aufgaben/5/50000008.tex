Finden Sie die LU- und LR-Zerlegung der Matrix
\[
A=\begin{pmatrix}
2&1\\
3&4
\end{pmatrix}.
\]

\begin{loesung}
F"ur die LU-Zerlegung muss der Gauss-Algorithmus durchgef"uhrt werden:
\[
\begin{tabular}{|>{$}c<{$}>{$}c<{$}|}
\hline
2\begin{picture}(0,0)
\color{red}\put(-3,4){\circle{12}}
\end{picture}%
&1\\
3%
\begin{picture}(0,0)
\color{blue}\drawline(-8,-2)(-8,10)(1,10)(1,-2)
\end{picture}
&4\\
\hline
\end{tabular}
\rightarrow
\begin{tabular}{|>{$}c<{$}>{$}c<{$}|}
\hline
1&\frac12\\
0&\frac52
\begin{picture}(0,0)
\color{red}\put(-3,3){\circle{15}}
\end{picture}\\
\hline
\end{tabular}
\]
Daraus liest man 
\[
L=\begin{pmatrix}
2\begin{picture}(0,0)
\color{red}\put(-3,4){\circle{12}}
\end{picture}%
&0\\3%
\begin{picture}(0,0)
\color{blue}\drawline(-8,-2)(-8,10)(1,10)(1,-2)
\end{picture}
&\frac52
\begin{picture}(0,0)
\color{red}\put(-3,3){\circle{15}}
\end{picture}
\end{pmatrix}\qquad\text{und}\qquad
U=\begin{pmatrix}
1&\frac12\\0&1
\end{pmatrix}
\]
ab.
Man kann durch Nachrechnen pr"ufen, dass $A=LU$.

In der LR-Zerlegung m"ochte man die Diagonal-Elemente, die jetzt in
der Matrix $L$ stehen, in die Matrix $U$ verschieben. Die Diagonalelemente
sind
\[
D=\begin{pmatrix}2&0\\0&\frac52\end{pmatrix}.
\]
Die neue Zerlegung ist damit:
\begin{align*}
L_0&=LD^{-1}=
\begin{pmatrix}
1&0\\
\frac32&1
\end{pmatrix}
\\
R&=DU=
\begin{pmatrix}
2&1\\
0&\frac52
\end{pmatrix}
\\
L_0R&=
\begin{pmatrix}
1&0\\
\frac32&1
\end{pmatrix}
\begin{pmatrix}
2&1\\
0&\frac52
\end{pmatrix}
=\begin{pmatrix}
1\cdot 2+0\cdot 0 & 1 \cdot 1+0\cdot \frac52\\
\frac32\cdot 2+1\cdot 0&\frac32\cdot 1+1\cdot\frac52
\end{pmatrix}
=
\begin{pmatrix}
2&1\\3&4
\end{pmatrix}.
\qedhere
\end{align*}
\end{loesung}


