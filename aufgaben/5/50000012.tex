In der Wikipedia findet man die Aussage, dass die zwanzig Punkte mit
den Koordinaten gemäss untenstehender Tabelle die Ecken eines
Dodekaeders bilden. Dabei ist
$\varphi=\frac12(1+\sqrt{5})$ und die Vorzeichen symbolisiert durch
das Zeichen $\pm$ können unabhängig voneinander verändert werden:
\begin{center}
\begin{tabular}{|c|c|}
\hline
Punkt&Anzahl\\
\hline
$(\pm 1,\pm 1,\pm 1)$&$8$\\
$(0,\pm\frac1{\varphi},\pm\varphi)$&$4$\\
$(\pm\frac1{\varphi},\pm\varphi,0)$&$4$\\
$(\pm\varphi,0,\pm\frac1{\varphi})$&$4$\\
\hline
&$20$\\
\hline
\end{tabular}
\end{center}
Die meisten dieser Punkte interessieren uns nicht, wir betrachten
nur die beiden Kanten
\[
\text{von $(1,1,1)$ nach $(0,\frac1{\varphi},\varphi)$}
\qquad\text{und}\qquad
\text{von $(-1,-1,-1)$ nach $(-\frac1{\varphi},-\varphi,0).$}
\]
Sie sind nicht parallel. Wie gross ist der minimale
Abstand der Geraden, die durch diese Kanten verlaufen?

\thema{Abstand}

{\it Hinweis:} Sie dürfen die folgenden Formeln für $\varphi$ verwenden,
um die algebraischen Rechnungen zu vereinfachen:
\[
\frac1{\varphi}=\varphi-1,\qquad
\frac1{\varphi}-1=\varphi-2,\qquad
\varphi^2=\varphi+1
\]
Damit können Sie beliebige Polynome in $\varphi$ zu linearen Ausdrücken
vereinfachen, zum Beispiel $(\varphi-1)^2=\varphi^2-2\varphi+1=\varphi+1-2\varphi+1=2-2\varphi$.

\begin{loesung}
Es geht darum, den windschiefen Abstand zweier Geraden mit den
Richtungsvektoren
\[
\vec r_1=\begin{pmatrix}
-1\\
\frac1{\varphi}-1\\
\varphi-1
\end{pmatrix}
\qquad\text{und}\qquad
\vec r_2
=\begin{pmatrix}
-\frac1{\varphi}+1\\
-\varphi+1\\
1
\end{pmatrix}
=-\begin{pmatrix}
\frac1{\varphi}-1\\
\varphi-1\\
-1
\end{pmatrix}
\]
durch die Punkte $(1,1,1)$ und $(-1,-1,-1)$ zu bestimmen. Dazu
muss die Länge der Projektion des Vektors
\[
\vec d= \begin{pmatrix} 2\\2\\2\end{pmatrix}
\]
auf den Vektor $\vec n = \vec r_1\times\vec r_2$ bestimmt werden.
Diese kann mit der in der Vorlesung abgeleiteten Formel
\[
\frac{\vec  n\cdot\vec d}{|\vec n|}
\]
bestimmt werden.

Die Anwendung der im Hinweis genannten Formeln erlaubt,
die Vektoren $\vec r_1$ und $\vec r_2$ so zu schreiben,
dass $\varphi$ nicht mehr im Nenner vorkommt:
\[
\vec r_1=\begin{pmatrix}
-1\\
\varphi-2\\
\varphi-1
\end{pmatrix}
\qquad\text{und}\qquad
\vec r_2
=-\begin{pmatrix}
\varphi-2\\
\varphi-1\\
-1
\end{pmatrix}
\]
Damit berechnen wir jetzt das Vektorprodukt von $\vec r_1$ und $\vec r_2$:
\begin{align*}
\vec n=
\vec r_1\times\vec r_2
&=
-\begin{pmatrix}
-1\\
\varphi-2\\
\varphi-1
\end{pmatrix}
\times
\begin{pmatrix}
\varphi-2\\
\varphi-1\\
-1
\end{pmatrix}
=
-
\begin{pmatrix}
-\varphi+2-(\varphi-1)^2\\
(\varphi-1)(\varphi-2)-1\\
-\varphi+1-(\varphi-2)^2
\end{pmatrix}
\\
&=
-\begin{pmatrix}
-\varphi+2-\varphi^2+2\varphi-1\\
\varphi^2-3\varphi+2-1\\
-\varphi+1-\varphi^2+4\varphi-4
\end{pmatrix}
\\
&=
-\begin{pmatrix}
-\varphi+2-\varphi-1+2\varphi-1\\
\varphi+1-3\varphi+2-1\\
-\varphi+1-\varphi-1+4\varphi-4
\end{pmatrix}
\\
&=
-\begin{pmatrix}
0\\
2-2\varphi\\
2\varphi-4
\end{pmatrix}
=
\begin{pmatrix}
0\\
2\varphi-2\\
-2\varphi+4
\end{pmatrix}
\end{align*}
Der gesuchte Abstand ist das Skalarprodukt dieses Vektors $\vec n$
mit $\vec d$, geteilt durch die Länge von $\vec n$. Das Skalarprodukt
ist
\[
\vec n\cdot\vec d=2\cdot 0+2(2\varphi-2)+2(-2\varphi+4)
=4\varphi-4-4\varphi+8=4.
\]
Die Länge von $\vec n$ ist
\begin{align*}
|\vec n|^2&=(2\varphi-2)^2+(-2\varphi+4)^2\\
&=4 \varphi^2-8\varphi+4+4\varphi^2-16\varphi+16\\
&=4\varphi+4-8\varphi+4+4\varphi+4-16\varphi+16\\
&=28-16\varphi\\
|\vec n|&=\sqrt{28-16\varphi}
\end{align*}
Der gesucht Abstand ist also
\[
\frac{\vec n\cdot \vec d}{|\vec n|}=\frac{4}{\sqrt{28-16\varphi}}\simeq
2.7528
\qedhere
\]
\end{loesung}

