% 2 6 9 11
Eine Kamera soll im Punkt $C=(5,1,3)$ montiert werden und den Punkt
$A=(3,-5,-6)$ beobachten.
Sie soll so orientiert sein, dass die horizontale Chipkante auch im 
Raum horizontal liegt.
Finden Sie zu diesem Zweck orthonormierte Vektoren $\vec{b}_i$ derart,
dass $\vec{b}_1$ von $C$ aus auf den Punkt $A$ zeigt,
$\vec{b}_2$ horizontal ist, also senkrecht auf die Vertikale,
und $\vec{b}_3$ zusammen mit $\vec{b}_1$ und $\vec{b}_2$ ein
Rechtssystem bildet.

\begin{loesung}
Der Vektor $\vec{b}_1$  ist der normierte Vektor
\[
\vec{a}_1
=
\vec{a}-\vec{c}
=
\begin{pmatrix} 3\\-5\\-6 \end{pmatrix}
-
\begin{pmatrix} 5\\1\\3\end{pmatrix}
=
\begin{pmatrix} -2\\-6\\-9\end{pmatrix}
\quad\Rightarrow\quad
|\vec{a}_1|
=
\sqrt{4+36+81}=11
\quad\Rightarrow\quad
\vec{b}_1
=
\frac{1}{11}
\begin{pmatrix} -2\\-6\\-9\end{pmatrix}.
\]
Der Vektor $b_2$ soll senkrecht stehen auf $\vec{e}_3$ und $\vec{b}_1$.
Ein solcher wird durch das Vektorprodukt
\[
\vec{a}_2
=
\vec{b}_1\times\vec{e}_3
=
\frac{1}{11}
\begin{pmatrix} -2\\-6\\-9\end{pmatrix}
\times
\begin{pmatrix} 0\\0\\1\end{pmatrix}
=
\frac1{11}
\begin{pmatrix} -6\\2\\ 0 \end{pmatrix}
\quad\Rightarrow\quad
|\vec{a}_2|
= 
\frac{2\sqrt{10}}{11}
\quad\Rightarrow\quad
\vec{b}_2
=
\frac{\vec{a}_2}{|\vec{a}_2|}
=
\frac{1}{\sqrt{10}}
\begin{pmatrix} -3\\1\\ 0 \end{pmatrix}.
\]
Schliesslich müssen $\vec{b}_1$, $\vec{b}_2$ und $\vec{b}_3$ ein
Rechtssystem bilden, also können wir
\[
\vec{b}_3
=
\vec{b}_1\times\vec{b}_2
=
\frac1{11}
\begin{pmatrix} -2\\-6\\-9\end{pmatrix}
\times
\frac{1}{\sqrt{10}}
\begin{pmatrix} -3\\1\\ 0 \end{pmatrix}
=
\frac{1}{11\sqrt{10}}
\begin{pmatrix}
9\\
27\\
-20
\end{pmatrix}
\]
verwenden.

Diese Aufgabe kann auch mit Hilfe des Gram-Schmidtschen
Orthonormalisierungsverfahrens gelöst werden.
Man muss dabei allerdings beachten, dass die Orthonormalisierung eines
horizontalen Vektors, z.~B.~$\vec{e}_1$ und $\vec{b}_1$ einen Vektor 
in der von den beiden Vektoren aufgespannten Ebene ergibt, der insbesondere
nicht mehr horizontal zu sein braucht, weil ja $\vec{b}_1$ nach unten zeigt.
Das spielt allerdings keine Rolle beim dritten Vektor, der ja nur in der
Ebene von $\vec{b}_1$ und der Vertikalen $\vec{e}_3$ liegen muss.
Die gesuchten Vektoren kann man also bekommen, indem man
\[
\vec{a}_1
=
\vec{a}-\vec{c}
=
\begin{pmatrix} -2\\-6\\-9\end{pmatrix},
\vec{a}_2 = \vec{e}_3,
\qquad\text{und}\qquad
\vec{a}_3 = \vec{e}_1
\]
in dieser Reihenfolge orthonormiert.
Der Vektor $\vec{b}_1$ wird natürlich derselbe sein.
Die anderen Vektoren entstehen in der Reihenfolge $\vec{b}_3$ und $\vec{b}_2$.
Es ist aber noch nicht sichergestellt, dass die Vektoren auch ein Rechtssystem
bilden, das muss man also mit der Determinanten ebenfalls noch nachprüfen.
Sollten die Vektoren nicht orthonormiert sein, kann man von $\vec{b}_2$
oder $\vec{b}_3$ das Vorzeichen kehren, am besten so, dass $\vec{b}_3$ nach
oben zeigt.
Die Details dieser Rechnung sind dem Leser überlassen.
\end{loesung}

\begin{bewertung}
Richtung für $\vec{b}_1$ ({\bf R}) 1 Punkt,
Normierung von $\vec{b}_1$ ({\bf B1}) 1 Punkt,
Vektorprodukt und Normierung für $\vec{b}_2$ ({\bf B2}) 2 Punkte,
Vektorprdoukt und Normierung für $\vec{b}_3$ ({\bf B3}) 2 Punkte.
\end{bewertung}
