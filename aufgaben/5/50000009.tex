Die sechs Punkte $(\pm 1,0,0)$, $(0,\pm1,0)$ und $(0,0,\pm1)$ bilden
die Ecken eines Oktaeders.
Wie gross ist der kleinste Abstand zweier Kanten, die sich
nicht schneiden und nicht parallel sind.

\thema{Abstand windschiefer Geraden}

\begin{loesung}
Wegen der Symmetrie des Oktaeders gibt es nur einen solchen Abstand.
Wir wählen als erste Kante jene von $(1,0,0)$ nach $(0,1,0)$. Eine
Kante, die diese nicht schneidet und nicht parallel ist, muss einen
Punkt ausserhalb der $x$-$y$-Ebene haben, denn die Kanten in der
$x$-$y$-Ebene sind entweder parallel oder schneiden. Wir wählen die
Kante von $(0,0,1)$ zu $(-1,0,0)$.

Für den windschiefen Abstand der Kanten benötigen wir die Richtungen
der Kanten: dies sind die Vektoren
\[
v_1=\begin{pmatrix}
-1\\1\\0
\end{pmatrix}
,\qquad
v_2=\begin{pmatrix}
-1\\0\\-1
\end{pmatrix}
\]
und deren Vektorprodukt:
\[
n=v_1\times v_2
=
\begin{pmatrix}
-1\\1\\0
\end{pmatrix}
\times
\begin{pmatrix}
-1\\0\\-1
\end{pmatrix}
=
\begin{pmatrix}
-1\\
-1\\
1\\
\end{pmatrix}
\]
Die Länge  dieses Vektorproduktes ist
\[
|v_1\times v_2|=\sqrt{3}.
\]
Der windschiefe Abstand ist jetzt die Projektion des Vektors
\[
q=\begin{pmatrix}
-1\\0\\1
\end{pmatrix},
\]
der die ``Anfangspunkte''  der beiden Vektoren $v_1$ und $v_2$
verbindet, auf $n$. Diese kann man mit Hilfe des Skalarproduktes
berechnen:
\[
d=\frac{n\cdot q}{|n|}=
\frac1{\sqrt{3}} \begin{pmatrix}-1\\-1\\1\end{pmatrix}\cdot\begin{pmatrix}
-1\\0\\1
\end{pmatrix}
=\frac2{\sqrt{3}}
=1.15470053837925152902
\qedhere
\]
\end{loesung}

