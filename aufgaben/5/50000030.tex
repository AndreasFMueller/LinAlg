%a0 =
%
%   14
%   -2
%    5
%
%a1 =
%
%    2
%  -11
%  -10
%
Betrachten Sie die Vektoren
\[
\vec{a}_0
=
\begin{pmatrix} 14\\-2\\5 \end{pmatrix}
\qquad\text{und}\qquad
\vec{a}_1
=
\begin{pmatrix} 2\\-11\\-10 \end{pmatrix}.
\]
\begin{teilaufgaben}
\item
Berechnen Sie $\vec{a}_2 = \vec{a}_0\times\vec{a}_1$.
\item
Bestimmen Sie Länge und Zwischenwinkel der Vektoren $\vec{a}_0$ und
$\vec{a}_1$.
\item
Betrachten Sie die rekursiv definierte Folge
$\vec{a}_{n+1} = \vec{a}_{n-1} \times\vec{a}_n$.
Berechnen Sie $\vec{a}_{6}$.
\end{teilaufgaben}

\thema{Zwischenwinkel}
\thema{Vektorprodukt}

\begin{loesung}
\begin{teilaufgaben}
\item
Das Vektorprodukt ist
\begin{align*}
\vec{a}_0\times\vec{a}_1
&=
\begin{pmatrix} 14\\-2\\5 \end{pmatrix}
\times
\begin{pmatrix} 2\\-11\\-10 \end{pmatrix}
=
\begin{pmatrix}
(-2)\cdot(-10)-(-11)\cdot 5 \\
5\cdot 2 - (-10)\cdot 14 \\
14\cdot(-11) -2\cdot(-2)
\end{pmatrix}
=
\begin{pmatrix}
75\\150\\-150
\end{pmatrix}.
\end{align*}
\item
Die Länge der Vektoren ist
\begin{align*}
|\vec{a}_0|
&=
\sqrt{196+4+25} = 15,
\\
|\vec{a}_1|
&=
\sqrt{4+121+100}=15.
\end{align*}
Da wir das Vektorprodukt bereits berechnet haben, können wir die 
Zwischenwinkelformel 
\[
\sin\alpha = \frac{|\vec{a}_0\times \vec{a}_1|}{|\vec{a}_0|\cdot |\vec{a}_1|}
\]
für das Vektorprodukt verwenden.
Dazu brauchen wir die Länge des Vektorproduktes:
\[
|\vec{a}_0\times\vec{a}_1|
=
\sqrt{75^2 + 2 \cdot 150^2}
=
25\sqrt{3^2 + 2\cdot 6^2}
=
25\sqrt{9+2\cdot 36}
=
25\sqrt{81}
=
225
\]
und finden für den Zwischenwinkel
\[
\sin\alpha = \frac{225}{15\cdot 15} = 1
\quad\Rightarrow\quad
\alpha = 90^\circ,
\]
die beiden Vektoren stehen also senkrecht aufeinander.
\item
Die Vektoren $\vec{a}_0$ und $\vec{a}_1$ stehen senkrecht und haben
beide Länge $15$.
Der Vektor $\vec{a}_2$ ist auf beiden senkrecht, die drei Vektoren
$\vec{a}_0$, $\vec{a}_1$ und $\vec{a}_2$ bilden ein Rechtssystem.
Alle Vektorprodukte $\vec{a}_n$ sind daher immer Vielfache dieser Vektoren,
$\vec{a}_3$ ist ein Vielfaches von $\vec{a}_0$, $\vec{a}_4$ ist ein
Vielfaches von $\vec{a}_1$ und allgemein ist 
$\vec{a}_n$ ein Vielfaches von $\vec{a}_{n-3}$.
Wir kennen also von allen Vektoren die Richtung, wir müssen nur noch die
Länge bestimmen.
Da die Vektoren senkrecht stehen, hat das Vektorprodukt jeweils als
Länge das Produkt der Längen der Faktoren, also
$|\vec{a}_{n+1}| = |\vec{a}_{n-1}|\cdot|\vec{a}_n|.$
Die Längen sind
\begin{center}
\renewcommand{\arraystretch}{1.15}
\begin{tabular}{l|
>{$}c<{$}
>{$}c<{$}
>{$}c<{$}
>{$}c<{$}
>{$}c<{$}
>{$}c<{$}
>{$}c<{$}
>{$}c<{$}
}
Vektor&\vec{a}_0
        &\vec{a}_1
           &\vec{a}_2
                &\vec{a}_3
                     &\vec{a}_4
                          &\vec{a}_5
                               &\vec{a}_6
                                       &\vec{a}_7
\\
\hline
Länge&15&15&15^2&15^3&15^5&15^8&15^{13}&15^{21}
\\
\end{tabular}
\end{center}
Daraus können wir die Länge des Vektors $\vec{a}_n$  ablesen.
Es ist $|\vec{a}_n| = 15^{F_n}$, wobei $F_n$ die $n$-te Fibonacci-Zahl ist.
Insbesondere wissen wir jetzt genau, wie lang der Vektor $\vec{a}_6$ ist, 
nämlich $|\vec{a}_6|=15^{13}$.

Die Richtung von $\vec{a}_6$ ist die Richtung von $\vec{a}_0$, der selbst
die Länge $15$ hat.
Es folgt
\[
\vec{a}_6
=
15^{13}\cdot \frac{\vec{a}_0}{|\vec{a}_0|}
=
15^{12}\vec{a}_0
=
129746337890625
\begin{pmatrix}
14\\-2\\5
\end{pmatrix}
=
\begin{pmatrix}
 1816448730468750\\
 -259492675781250\\
  648731689453125 
\end{pmatrix}.
\qedhere
\]
\end{teilaufgaben}
\end{loesung}

\begin{bewertung}
\begin{teilaufgaben}
\item Vektorprodukt ({\bf V}) 1 Punkt
\item Länge der Vektoren ({\bf L}) 2 Punkte,
Zwischenwinkel ({\bf Z}) 1 Punkt
\item
Orthogonalität aufeinanderfolgender Vektoren der Folge $a_n$ oder
andere Begründung für das Resultat ({\bf O}) 1 Punkt,
Resultat ({\bf R}) 1 Punkt.
\end{teilaufgaben}
\end{bewertung}
