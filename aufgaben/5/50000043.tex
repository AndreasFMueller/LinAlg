Berechnen Sie den Abstand des Punktes $Q=(12,9,1)$ von der Geraden
mit der Parameterdarstellung
\[
\vec{p}
=
\begin{pmatrix*}[r]
12 \\ 8 \\ 0
\end{pmatrix*}
+
t
\begin{pmatrix*}[r]
1\\1\\0
\end{pmatrix*}.
\]

\begin{loesung}
Mit der Abstandsformel für den Abstand eines Punktes von einer Geraden
finden wir
\begin{align*}
h
&=
\frac{
|\vec{r}\times(\vec{q}-\vec{p_0})|
}{
|\vec{r}|
}
=
\frac{\left|
\begin{pmatrix}1\\1\\0\end{pmatrix}
\times
\left(
\begin{pmatrix*}[r] 12\\9\\ 1 \end{pmatrix*}
-
\begin{pmatrix*}[r] 12\\8\\ 0 \end{pmatrix*}
\right)
\right|}{
\left|
\begin{pmatrix} 1 \\ 1 \\ 0 \end{pmatrix}
\right|
}
=
\frac{\left|
\begin{pmatrix}1\\1\\0\end{pmatrix}
\times
\begin{pmatrix}0\\1\\1\end{pmatrix}
\right|}{
\sqrt{2}
}
=
\frac{\left|
\begin{pmatrix*}[r]1\\-1\\1\end{pmatrix*}
\right|}{\sqrt{2}}
=
\frac{\!\sqrt{3}}{\!\sqrt{2}}
\approx
1.224745.
\qedhere
\end{align*}
\end{loesung}
