Rechnen Sie mithilfe der Definition des Vektorproduktes nach,
dass $\vec{a}\cdot(\vec{a}\times\vec{b}) = 0$ ist.

\begin{loesung}
\definecolor{farbea}{rgb}{1,0.6,0.6}
\definecolor{farbeb}{rgb}{0.6,0.6,1}
\definecolor{farbec}{rgb}{0.6,0.8,0.6}
Die Definition des Vektorproduktes ist
\[
\vec{a}
\times
\vec{b}
=
\begin{pmatrix} a_1 \\ a_2 \\ a_3 \end{pmatrix}
\times
\begin{pmatrix} b_1 \\ b_2 \\ b_3 \end{pmatrix}
=
\begin{pmatrix}
a_2b_3-a_3b_2\\
a_3b_1-a_1b_3\\
a_1b_2-a_2b_1
\end{pmatrix}.
\]
Das Skalarprodukt mit $\vec{a}$ ist daher
\begin{align*}
\vec{a}\cdot(\vec{a}\times\vec{b})
&=
\begin{pmatrix} a_1 \\ a_2 \\ a_3 \end{pmatrix}
\cdot
\begin{pmatrix}
a_2b_3-a_3b_2\\
a_3b_1-a_1b_3\\
a_1b_2-a_2b_1
\end{pmatrix}
\\
&=
{\color{farbea}a_1a_2b_3}-{\color{farbeb}a_1a_3b_2}
+
{\color{farbec}a_2a_3b_1}-{\color{farbea}a_2a_1b_3}
+
{\color{farbeb}a_3a_1b_2}-{\color{farbec}a_3a_2b_1}.
\intertext{Die Terme gleicher Farbe heben sich weg, so dass}
\vec{a}\cdot(\vec{a}\times\vec{b})
&=
0
\end{align*}
folgt.
\end{loesung}
