% Lehraufgabe zur QR-Zerlegung
%Q =
%
%  -0.727607  -0.505790   0.463428
%   0.485071  -0.857032  -0.173785
%   0.485071   0.098348   0.868927
%
%R =
%
%  -4.12311  -2.91043   2.66789
%   0.00000   4.18681  -0.29504
%   0.00000   0.00000  -2.60678
%
%octave:10> Q*R
%ans =
%
%   3.0000e+00   8.8818e-16  -3.0000e+00
%  -2.0000e+00  -5.0000e+00   2.0000e+00
%  -2.0000e+00  -1.0000e+00  -1.0000e+00
%
Das Ziel dieser Aufgabe ist zu zeigen, dass jede beliebige reguläre 
$n\times n$ Matrix $A$ geschrieben werden kann in der Form $A=QR$,
wobei $Q$ eine orthogonale Matrix ist und $R$ eine obere Dreicksmatrix.
\begin{teilaufgaben}
\item
Gegeben ist die Matrix 
\[
A=\begin{pmatrix}
 3& 0&-3\\
-2&-5& 2\\
-2&-1&-1
\end{pmatrix}.
\]
Wenden Sie den Gram-Schmidtschen Orthonormalisierungsprozess auf
die Spalten von $A$ an, die neuen Vektoren bilden die Matrix $Q$.
\item
Ist die Matrix $Q$ orthogonal?
Ist das immer so, also nicht nur für diese spezielle Matrix $A$?
\item
Finden Sie eine Matrix $R$ derart, dass $A=QR$.
Was fällt ihnen an der Matrix $R$ auf?
\item 
Im Gram-Schmidt Prozess wurden die Spalten $q_k$ so konstruiert, dass
$a_i$ eine Linearkombination von $q_k$ mit $1\le k\le i$ ist.
Schreiben Sie diese Bedingung für jedes $i$ mit Hilfe von Koeffizienten
$r_{ki}$
\item
Zeigen Sie, dass die Linearkombinationen von Teilaufgabe d) auch
als Matrixprodukt $A=QR$
geschrieben werden kann.
\end{teilaufgaben}

\thema{QR-Zerlegung}

\begin{loesung}
\begin{teilaufgaben}
\item
\item
\item
\item
\item
\end{teilaufgaben}
\end{loesung}

