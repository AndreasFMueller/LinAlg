Gegeben ist eine Ebene $\sigma$, welche durch die Punkte
\[
A=(12,0,0), \qquad
B=(0,0,5) \qquad \text{und}\qquad  
C=(12,5,0)
\]
geht, sowie ein gerades Rohr mit Durchmesser 1,
dass zwischen den Punkten
\[
D=(4,-5,6)   \qquad \text{und}\qquad  
E=(8,5,10) 
\]
installiert ist.
Nun wird eine Kugel mit Radius 2 im Punkt $B$ auf die Ebene gelegt
und losgelassen. Die Kugel rollt auf den Punkt $A$ zu. 
Wird die Kugel unter dem Rohr hindurch passen und damit den Punkt $A$ erreichen?
\begin{center}
\begin{tikzpicture}[thick,>=latex]
\node at (0,0) {\includeagraphics[width=6.8cm]{kugel.jpg}};
\node at (-3.1,0.5) {$x$};
\node at (2.025,-3.25) {$y$};
\node at (2.75,2.45) {$z$};
\node at (-2.2,0.6) {$A$};
\node at (2.7,1.1) {$B$};
\node at (-2.8,-1.1) {$C$};
\node at (1.4,3.0) {$D$};
\node at (-1.8,1.5) {$E$};
\end{tikzpicture}
\end{center}

\thema{Normalenform}
\thema{Vektorprodukt}
\thema{Kugel}
\thema{Abstand windschiefer Geraden}

\begin{loesung}
Die Kugel wird genau dann den Punkt $A$ erreichen, wenn der Abstand zwischen
der Geraden $\vec g_1$, die das Rohr beschreibt, und der Geraden $\vec g_2$, 
auf der sich der Mittelpunkt der Kugel bewegt, grösser ist als die Summe der 
beiden Radien (Radius der Kugel und des Rohres):
\[
  d > 2+0.5 = 2.5.
\]
Wir benötigen also Stütz- und Richtungsvektoren der beiden Geraden.

Der Richtungsvektor der Geraden $\vec g_1$ ist
\[
  \vec r_1 = \vec e - \vec d =
  \begin{pmatrix}
   8-4\\ 
   5-(-5)\\
   10-6
  \end{pmatrix}
  =  \begin{pmatrix}
   4\\ 
   10\\
   4
  \end{pmatrix}
\]
und als Stützvektor verwenden wir $\vec d$. Die Gerade $\vec g_1$ 
ist somit beschrieben:
\[
  \vec g_1 = 
  \begin{pmatrix}
   4\\ 
   -5\\
   6
  \end{pmatrix}
  + t\cdot 
  \begin{pmatrix}
   4\\ 
   10\\
   4
  \end{pmatrix}.
\]

Die Gerade $\vec g_2$ beschreibt die Bewegung des Kugel-Mittelpunktes.
Sie verläuft parallel zum Vektor $\overrightarrow{ AB}$ und hat zur Ebene $\sigma$ 
einen konstanten Abstand von 2. Als Richtungsvektor der Geraden 
$\vec g_2$ verwenden wir daher
\[
  \vec r_2 = \overrightarrow{AB}= \vec b - \vec a =
  \begin{pmatrix}
   0-12\\ 
   0-0\\
   5-0
  \end{pmatrix}
  =  \begin{pmatrix}
   -12\\ 
   0\\
   5
  \end{pmatrix}
\]
und als Stützvektor
\[
\vec p = \vec a + 2\cdot \vec n_0,
\]
wobei $\vec n_0$ der auf Länge 1 normierte Normalenvektor der Ebene ist.
Um den Normalenvektor zu berechnen verwenden wird das Kreuzprodukt:
\[
 \vec n = \overrightarrow{AB}\times \overrightarrow{AC}
 = \begin{pmatrix} -12\\ 0\\ 5 \end{pmatrix}\times\begin{pmatrix} 0\\ 5\\ 0 \end{pmatrix}
 = \begin{pmatrix} 0\cdot 0 - 5\cdot 5\\ 5\cdot 0 - 0\cdot -(12)\\ -12\cdot 5 - 0\cdot 0 \end{pmatrix}
 = \begin{pmatrix} -25\\ 0 \\ -60 \end{pmatrix}
\]
und normieren ihn anschliessend auf die Länge 1:
\[
 \vec n_0 = (-1)\cdot \dfrac{1}{|\vec n|}\vec n
 = (-1)\cdot \dfrac{1}{\sqrt{(-25)^2+0^2+(-60)^2}}\begin{pmatrix} -25\\ 0 \\ -60 \end{pmatrix}
 = \dfrac{1}{13}\begin{pmatrix} 5\\ 0 \\ 12 \end{pmatrix}.
\]
Zusätzlich wurde noch mit dem Faktor $(-1)$ multipliziert, um sicherzustellen, dass
der Normalenvektor eine positive $z$-Komponente hat und damit nach oben schaut.
Die Richtung des Normalenvektors ist hier von Bedeutung, da die Kugel auf der 
Ebene rollt und nicht unter der Ebene.
Die Gerade $\vec g_2$  ist somit ebenfalls beschrieben:
\begin{align*}
  \vec g_2 &= \vec a + 2\cdot \vec n_0 + s\cdot \vec r_2
 =  \begin{pmatrix} 12\\ 0\\ 0 \end{pmatrix} + 2\cdot \dfrac{1}{13}\begin{pmatrix} 5\\ 0 \\ 12 \end{pmatrix}
 +  s\cdot \begin{pmatrix}-12\\ 0\\ 5  \end{pmatrix}\\
 &=
   \dfrac{1}{13}\begin{pmatrix}
   166\\ 
   0\\
   24
  \end{pmatrix}
  + s\cdot 
  \begin{pmatrix}
   -12\\ 
   0\\
   5
  \end{pmatrix}.
\end{align*}
Um zum Schluss nun noch den Abstand zwischen den beiden Geraden zu berechnen,
verwenden wir die Abstandsformel für den windschiefen Abstand:
\begin{align*}
 d & = \dfrac{(\vec d -\vec p)\cdot (\vec r_1\times \vec r_2)}{|\vec r_1\times \vec r_2|}\\
 &= \dfrac{\left(
 \begin{pmatrix}
   4\\ 
   -5\\
   6
  \end{pmatrix}-
 \dfrac{1}{13}\begin{pmatrix}
   166\\ 
   0\\
   24
  \end{pmatrix}\right)\cdot \left(
  \begin{pmatrix}
   4\\ 
   10\\
   4
  \end{pmatrix}\times 
  \begin{pmatrix}
   -12\\ 
   0\\
   5
  \end{pmatrix}\right)}{|\vec r_1\times \vec r_2|}
  =\dfrac{
 \dfrac{1}{13}\begin{pmatrix}
   -114\\ 
   -65\\
   54
  \end{pmatrix}\cdot
  \begin{pmatrix}
   50\\ 
   -68\\
   120
  \end{pmatrix}}{\sqrt{50^2+(-68)^2+120^2}}\\
  &= \dfrac{400}{146.711} = 2.726.
\end{align*}
Da der Abstand $d$ grösser ist als $2.5$ wird die Kugel unter dem Rohr
hindurch passen und den Punkt $A$ erreichen.

\end{loesung}

\begin{bewertung}
Stütz- und Richtungsvektor der Rohr-Geraden ({\bf G}) 1 Punkt,
Richtungsvektor der Mittelpunkt-Geraden ({\bf R}) 1 Punkt,
Normale ({\bf N}) 1 Punkt,
Stützvektor der Mittelpunkt-Geraden ({\bf S}) 1 Punkt,
Berechnung des Windschiefen Abstandes ({\bf W}) 1 Punkt,
Schlussfolgerung und Begründung ({\bf B}) 1 Punkt.
\end{bewertung}





