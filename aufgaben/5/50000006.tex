Die Ebene $\sigma_1$ sei die Ebene durch die Punkte $(1,0,0)$,
$(0,1,0)$ und $(0,0,1)$.
Die Ebene $\sigma_2$ hat die Gleichung
\[
x+2y+3z=4.
\]
Die Ebene $\sigma_3$ steht auf beiden Ebenen senkrecht und geht
durch den Nullpunkt des Koordinatensystems.
\begin{teilaufgaben}
\item
Finden Sie den Fusspunkt $F$ des Lotes vom Punkt $P=(10,-1,6)$ auf die
Ebene $\sigma_3$.
\item
Bestimmen Sie ausserdem die Länge des Lotes von $P$ auf die Ebene $\sigma_3$.
\end{teilaufgaben}

\thema{Vektorprodukt}
\thema{Abstand}

\begin{loesung}
% \begin{teilaufgaben}
% \item
Um den Fusspunkt $F$ zu berechnen, braucht man die Normale 
$\vec n_3$ von $\sigma_3$, welche auf den Normalen
$\vec n_1$ von $\sigma_1$ und $\vec n_2$ von $\sigma_2$ senkrecht steht.
Die Normale $\vec n_1$ kann mit dem Vektorprodukt gefunden werden
\[
\vec n_1
=
\begin{pmatrix}-1\\1\\0\end{pmatrix}
\times
\begin{pmatrix}-1\\0\\1\end{pmatrix}
=
\begin{pmatrix}
1\cdot 1 - 0 \cdot 0 \\
0\cdot (-1) - 1 \cdot (-1)\\ 
(-1)\cdot 0 - (-1) \cdot 1
\end{pmatrix}
=
\begin{pmatrix}
1\\1\\1
\end{pmatrix}.
\]
und $\vec n_2$ kann aus der Gleichung abgelesen werden
\[
\vec n_2
=
\begin{pmatrix}
1\\2\\3
\end{pmatrix}.
\]
Damit kann $\vec n_3$ nun ebenfalls mit dem Vektorprodukt gefunden werden:
\[
\vec n_3
=
\vec n_1 \times \vec n_2
=
\begin{pmatrix}1\\1\\1\end{pmatrix}
\times
\begin{pmatrix}1\\2\\3\end{pmatrix}
=
\begin{pmatrix}
1\cdot 3 - 2 \cdot 1 \\
1\cdot 1 - 3 \cdot 1\\ 
1\cdot 2 - 1 \cdot 1
\end{pmatrix}
=
\begin{pmatrix}
1\\-2\\1
\end{pmatrix}.
\]
Da die Ebene $\sigma_3$ durch den Nullpunkt geht, ist ihre Gleichung in Normalenform
\[
x-2y+z = 0.
\]
Teilt man dies Gleichung noch durch die Länge der Normalen
\[
|\vec n_3|
=
\sqrt{1^2+2^2 + 1^2}=\sqrt{6} 
\]
erhält man die Hessesche Normalform
\[
\dfrac{1}{\sqrt{6}}x-\dfrac{2}{\sqrt{6}}y+\dfrac{1}{\sqrt{6}}z = 0,
\]
mit welcher die Teilaufgabe b) bereits gelöst werden kann.
Die Länge des Lotes von $P$ auf die Ebene $\sigma_3$ ist der Abstand
des Punktes $P$ von der Ebene $\sigma_3$. setzt man $P$ in die Gleichung
ein, erhält man
\[
d = \overline{PF}
=
\dfrac{1}{\sqrt{6}}\cdot 10-\dfrac{2}{\sqrt{6}}\cdot (-1)+\dfrac{1}{\sqrt{6}} \cdot 6
= 
\dfrac{18}{\sqrt{6}}
=
3\sqrt{6}
=
7.348.
\]
Da $\overrightarrow{PF}$ parallel zu $-\vec n_3$ sein muss
und die Länge $d$ hat, kann
der Fusspunkt $F$ nun ermittelt werden als
\[
\vec f = \vec p + \overrightarrow{PF}
= \vec p - d \cdot \dfrac{\vec n_3}{|\vec n_3|}
=
\begin{pmatrix}
10\\ -1\\ 6
\end{pmatrix}
- 3\sqrt{6} \cdot \dfrac{1}{\sqrt{6}}
\begin{pmatrix}
1\\-2\\1
\end{pmatrix}
=
\begin{pmatrix}
7\\5\\3
\end{pmatrix}.
\]
Zur Kontrolle kann der gefundene Punkt noch in die Gleichung der
Ebene $\sigma_3$ eingesetzt werden, da der Fusspunkt $F$ des Lotes 
auf der Ebene $\sigma_3$ liegen muss:
\[
x-2y+z = 7 -2\cdot 5 + 3 = 0.
\qedhere
\]
% \end{teilaufgaben}
\end{loesung}

\begin{bewertung}
Normale $\vec n_1$ ({\bf N}) 1 Punkt,
Normale $\vec n_3$ (Vektorprodukt) ({\bf V}) 1 Punkt,
Ebenengleichung für $\sigma_3$ ($\mathbf{\Sigma}$) 1 Punkt,
Länge des Lotes ({\bf L}) 1 Punkt,
Gleichung für Fusspunkt ({\bf G}) 1 Punkt,
Fusspunkt ({\bf F}) 1 Punkt,
\end{bewertung}


