Die Gerade mit Parameterdarstellung
\[
\vec{p}_0
+t
\vec{r}
=
\begin{pmatrix*}[r]
  -7 \\
  -6 \\
  -5
\end{pmatrix*}
+
t
\begin{pmatrix*}[r]
  -7 \\
   3 \\
  -3
\end{pmatrix*}
\]
wird mit der Matrix
\[
A
=
\begin{pmatrix*}[r]
  -4 & -7 & 6 \\
  -3 & -5 & 2 \\
  -4 & -7 & 5
\end{pmatrix*}
\]
abgebildet.
Wie gross ist der Abstand zwischen der Bildgeraden und der $x$-Achse?

\begin{loesung}
Die Parameterdarstellung der Bildgerade $g$ ist
\[
A\vec{p}_0 + tA\vec{r}
=
\vec{p}_0' + t\vec{r}'
=
\begin{pmatrix*}[r]
   40 \\
   41 \\
   45
\end{pmatrix*}
+t
\begin{pmatrix*}[r]
  -11 \\
    0 \\
   -8
\end{pmatrix*}.
\]
Zur Berechnnung des Abstands kann die Abstandsformel für windschiefe
Geraden verwendet werden:
\begin{align*}
d
&=
\frac{(\vec{e}_1\times \vec{r}')\cdot \vec{p}_0'}{|\vec{e}_1\times\vec{r}'|}
.
\end{align*}
Sie wird vereinfacht durch die Tatsache, dass der Stützvektor der
$x$-Achse der Nullvektor ist.
Das Vektorprodukt ist
\begin{align*}
\vec{n}
=
\vec{e}_1\times\vec{r}'
=
\begin{pmatrix*}[r]
1\\0\\0
\end{pmatrix*}
\times
\begin{pmatrix*}[r]
  -11 \\
    0 \\
   -8
\end{pmatrix*}
=
\begin{pmatrix*}[r]
0\\8\\0
\end{pmatrix*}.
\end{align*}
Damit kann jetzt der Abstand berechnet werden:
\[
d
=
\frac{\vec{n}\cdot \vec{p}_0'}{|\vec{n}|}
=
-41.
\]
Der Abstand der $x$-Achse zur Geraden $g$ ist also 41 Einheiten.
\end{loesung}

\begin{bewertung}
Abbildung des Stützvektors, ({\bf S}) 1 Punkt,
und des Richtungsvektors ({\bf R}) 1 Punkt,
Abstandsformel ({\bf A}) 1 Punkt,
Vektorprodukt ({\bf K}) 1 Punkt,
Richtungsvektor und Stützvektor der $x$-Achse ({\bf X}) 1 Punkt,
Abstand $d$ ({\bf D}) 1 Punkt.
\end{bewertung}


