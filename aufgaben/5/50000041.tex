Berechnen Sie für die Vektoren
\begin{align*}
\vec{a}
&=
\begin{pmatrix} 1 \\ 1 \\ 0 \end{pmatrix}
&&\text{und}&
\vec{b}
&=
\begin{pmatrix} 0 \\ 1 \\ 1 \end{pmatrix}
\end{align*}
von Aufgabe \ref{50000038}
\begin{teilaufgaben}
\item den Flächeninhalt des aufgespannten Parallelogramms und
\item den Zwischenwinkel.
\end{teilaufgaben}

\begin{loesung}
Das Vektorprodukt ist
\[
\vec{a}\times\vec{b}
=
\begin{pmatrix*}[r] 1 \\ -1 \\ 1 \end{pmatrix*},
\]
damit können jetzt die Teilaufgaben gelöst werden.
\begin{teilaufgaben}
\item Der Flächeninhalt ist
\[
F
=
|\vec{a}\times\vec{b}| 
=
\sqrt{1^2+(-1)^2+1^2}
=
\sqrt{3}.
\]
\item
Die Länge der Vektoren ist
$|\vec{a}|=\sqrt{2}$
und
$|\vec{b}|=\sqrt{2}$.
Daraus folgt für den Zwischenwinkel
\begin{align*}
|\vec{a}|
\cdot
|\vec{b}|
\cdot
\sin\alpha
&=
|\vec{a}\times\vec{b}|
\\
\sqrt{2}
\cdot
\sqrt{2}
\cdot
\sin\alpha
&=
\vec{3}
&&\Rightarrow&
\sin\alpha
&=
\frac{\sqrt{3}}{2}
&&\Rightarrow&
\alpha
&=
\begin{cases}
\frac{\pi}3\\
\frac{2\pi}3.
\end{cases}
\qedhere
\end{align*}
\end{teilaufgaben}
\end{loesung}
