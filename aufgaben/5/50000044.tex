Finden Sie den Abstand der beiden Geraden mit Parameterdarstellungen
\[
\begin{pmatrix*}[r] 12 \\ 9 \\ 1 \end{pmatrix*}
+
t
\begin{pmatrix*}[r] 1 \\ 1 \\ 0 \end{pmatrix*}
\qquad\text{und}\qquad
\begin{pmatrix*}[r] 18 \\ 4 \\ 8 \end{pmatrix*}
+
t
\begin{pmatrix*}[r] 0 \\ 1 \\ 1 \end{pmatrix*}
\]

\begin{loesung}
In der Formel
\begin{align*}
d 
&=
(\vec{p}_2-\vec{p}_1)
\cdot
\frac{
\vec{r}_2\times \vec{r}_1
}{
|\vec{r}_2\times \vec{r}_1|
}
\end{align*}
für den windschiefen Abstand ist das Vektorprodukt der Richtungsvektoren
\begin{align*}
\vec{r}_1\times\vec{r}_2
&=
\begin{pmatrix*}[r] 1 \\ 1 \\ 0 \end{pmatrix*}
\times
\begin{pmatrix*}[r] 0 \\ 1 \\ 1 \end{pmatrix*}
=
\begin{pmatrix*}[r] 1 \\ -1 \\ 1 \end{pmatrix*}
&&\text{mit}&
|\vec{r}_1\times\vec{r}_2|
&=
\!\sqrt{3},
\end{align*}
wie schon in Aufgabe~\ref{50000038} gefunden wurde.
Damit kann man jetzt den Abstand berechnen:
\begin{align*}
d
&=
\left(
\begin{pmatrix*}[r] 18 \\ 4 \\ 8 \end{pmatrix*}
-
\begin{pmatrix*}[r] 12 \\ 9 \\ 1 \end{pmatrix*}
\right)
\cdot
\frac{1}{\!\sqrt{3}}
\begin{pmatrix*}[r] 1 \\ -1 \\ 1 \end{pmatrix*}
=
\frac{1}{\!\sqrt{3}}
\begin{pmatrix*}[r] 6 \\ -5 \\ 7 \end{pmatrix*}
\cdot
\begin{pmatrix*}[r] 1 \\ -1 \\ 1 \end{pmatrix*}
\\
&=
\frac{1}{\!\sqrt{3}}
(6\cdot 1 + (-5)\cdot(-1)+7\cdot 1)
=
\frac{1}{\!\sqrt{3}}
(6+5+7)
=
\frac{15}{\!\sqrt{3}}
=
5\!\sqrt{3}.
\qedhere
\end{align*}
\end{loesung}

