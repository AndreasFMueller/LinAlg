Sei $\sigma$ die zur $x$-$y$-Ebene parallele Ebene, die die $z$-Achse
im Punkt $(0,0,1)$ schneidet.
Sie wird von der Matrix
\[
A=
\begin{pmatrix*}[r]
  -10 & -6 & 11 \\
    4 &  7 &  9 \\
    9 & 10 & -4
\end{pmatrix*}
\]
auf eine Ebene abgebildet, die mit $\sigma'$ bezeichnet wird.
Wie weit ist der Punkt mit Ortsvektor
\[
\vec{b}
=
\begin{pmatrix*}[r]
  -7 \\
  -8 \\
   9
\end{pmatrix*}
\]
von der Ebene $\sigma'$ entfernt?

\begin{loesung}
Da die Ebene $\sigma$ die Parameterdarstellung
\[
\begin{pmatrix}0\\0\\1\end{pmatrix}
+
t
\begin{pmatrix}1\\0\\0\end{pmatrix}
+
s
\begin{pmatrix}0\\1\\0\end{pmatrix}
\]
hat, hat die Ebene $\sigma'$ hat die Parameterdarstellung
\[
\vec{p}
+
t\vec{r}_1
+
s\vec{r}_2
=
\begin{pmatrix*}[r]
 11 \\
  9 \\
 -4
\end{pmatrix*}
+
t
\begin{pmatrix*}[r]
 -10 \\
   4 \\
   9
\end{pmatrix*}
+
s
\begin{pmatrix*}[r]
  -6 \\
   7 \\
  10
\end{pmatrix*}.
\]
Für den Abstand kann die Abstandsformel
\[
d
=
\frac{
(\vec{r}_1\times\vec{r}_2)\cdot (\vec{b}-\vec{p})
}{
|\vec{r}_1\times\vec{r}_2|
}
\]
verwendet werden.
Das Vektorprodukt ist
\begin{align*}
\vec{n}
=
\vec{r}_1\times\vec{r}_2
&=
\begin{pmatrix*}[r]
 -10 \\
   4 \\
   9
\end{pmatrix*}
\times
\begin{pmatrix*}[r]
  -6 \\
   7 \\
  10
\end{pmatrix*}
=
\begin{pmatrix*}
(\phantom{-0}4)\cdot(\phantom{-}10)-(\phantom{-1}9)\cdot(\phantom{-}7)\\
(\phantom{-0}9)\cdot(\phantom{0}{-6})-(-10)\cdot(\phantom{-}\llap{1}0) \\
(-10)\cdot(\phantom{-0}7)-(\phantom{-1}4)\cdot(-6)
\end{pmatrix*}
=
\begin{pmatrix*}[r]
-23\\
 46\\
-46
\end{pmatrix*}
=
23
\begin{pmatrix*}[r]
-1\\
 2\\
-2\\
\end{pmatrix*},
\\
|\vec{r}_1\times\vec{r}_2|
&=
23\cdot 3.
\end{align*}
Die Differenz zwischen $\vec{b}$ und Stützvektor ist
\[
\vec{b}-\vec{p}
=
\begin{pmatrix*}[r]
-7\\-8\\9
\end{pmatrix*}
-
\begin{pmatrix*}[r]
11\\9\\-4
\end{pmatrix*}
=
\begin{pmatrix*}[r]
-18\\
-17\\
 13
\end{pmatrix*}.
\]
Mit der Abstandsformel findet man jetzt
\begin{align*}
\vec{n}\cdot(\vec{b}-\vec{p})
&=
23\cdot
\begin{pmatrix*}[r]
-1\\2\\-2
\end{pmatrix*}
\cdot
\begin{pmatrix*}[r]
-18\\
-17\\
 13
\end{pmatrix*}
=
23\cdot(18-34-26)
=
23\cdot(-42)
=
23\cdot3\cdot(-14)
\\
\Rightarrow\qquad
d
=
\frac{\vec{n}\cdot(\vec{b}-\vec{p})}{|\vec{n}|}
&=
\frac{23\cdot 3 \cdot(-14)}{23\cdot 3}
=
-14.
\end{align*}
Der Punkt $B$ ist also $14$ Einheiten von der Ebene $\sigma$ entfernt.
\end{loesung}

\begin{bewertung}
Parameterdarstellung der Ebene $\sigma$ ({\bf E}) 1 Punkt,
Abstandsformel ({\bf F}) 1 Punkt,
Vektorprodukt ({\bf V}) 1 Punkt,
Länge des Vektorprodukts ({\bf L}) 1 Punkt,
Skalarprodukt ({\bf S}) 1 Punkt,
Abstand ({\bf D}) 1 Punkt.
\end{bewertung}


