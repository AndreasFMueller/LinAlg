Auf den Koordinatenachsen werden jeweils in positiver und
negativer Richtung Strecken gleicher Länge abgetragen. Für die
$x$-Achse ist die Länge $2$, für die $y$-Achse $3$ und für die
$z$-Achse $5$. Durch Verbinden der Endpunkte entsteht ein
unregelmässiges Oktaeder. Sei $g$ die
Gerade durch die beiden Punkte auf der positiven $x$- und $y$-Achse,
und $h$ die Gerade durch den Punkte auf der negativen $x$-Achse
und auf der positiven $z$-Achse. Bestimmen Sie den Abstand von $g$ und $h$.

\thema{Abstand windschiefer Geraden}
\thema{Vektorprodukt}

\begin{loesung}
Die beiden Kanten sind windschiefe Geraden, deren Abstand mit
der Abstandsformel gefunden werden kann. Dazu benötigt man zunächst die
Richtungsvektoren
\begin{align*}
\vec r_1&=\begin{pmatrix}2\\0\\0\end{pmatrix}-\begin{pmatrix}0\\3\\0\end{pmatrix}
=
\begin{pmatrix}2\\-3\\0\end{pmatrix}
,\\
\vec r_2&=\begin{pmatrix}-2\\0\\0\end{pmatrix}-\begin{pmatrix}0\\0\\5\end{pmatrix}
=
\begin{pmatrix}-2\\0\\-5\end{pmatrix}
.
\end{align*}
Daraus berechnet man jetzt das Vektorprodukt
\[
\vec r_1\times \vec r_2
=
\begin{pmatrix}2\\-3\\0\end{pmatrix}
\times
\begin{pmatrix}-2\\0\\-5\end{pmatrix}
=
\begin{pmatrix}
(-3)\cdot (-5) - 0 \cdot 0 \\
0\cdot (-2) - (-5) \cdot 2\\ 
2\cdot 0 - (-2) \cdot (-3)
\end{pmatrix}
=
\begin{pmatrix}
15\\10\\ -6
\end{pmatrix},
\]
welches wir auch noch normieren müssen:
\begin{align*}
|\vec r_1\times\vec r_2|^2
&=15^2+10^2+6^2=225+100+36=19^2
\\
\vec n&=
\frac{\vec r_1\times\vec r_2}{|\vec r_1\times\vec r_2|}=
\frac1{19}
\begin{pmatrix}
15\\10\\ -6
\end{pmatrix}.
\end{align*}
Ausserdem braucht man die Differenz zweier Punkte,
wir nehmen die Punkte auf der $x$-Achse
\[
\vec q_1-\vec q_2
=
\begin{pmatrix}2\\0\\0\end{pmatrix}
-
\begin{pmatrix}-2\\0\\0\end{pmatrix}
=
\begin{pmatrix}4\\0\\0\end{pmatrix}
\]
Die Abstandsformel liefert dann
\[
d=\vec n\cdot (\vec q_1-\vec q_2)=
\frac1{19}
\begin{pmatrix}
15\\10\\ -6
\end{pmatrix}
\cdot
\begin{pmatrix}4\\0\\0\end{pmatrix}
=\frac{60}{19}
=
3.158.
\qedhere
\]
\end{loesung}

