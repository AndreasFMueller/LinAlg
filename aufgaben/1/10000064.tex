Gegeben ist das Gleichungssystem mit der Koeffizientenmatrix $A$ und der
rechten Seite $b$:
\[
A
=
\begin{pmatrix}
4&20&4\\
1&7&7\\
1&9&a
\end{pmatrix},
\qquad
b=\begin{pmatrix}
-8\\8\\2a-c
\end{pmatrix}.
\]
\begin{teilaufgaben}
\item
Für welchen Wert von $a$ ist der $\operatorname{Rang}A=2$?
Verwenden Sie in den nachfolgenden Teilaufgaben diesen Wert von $a$.
\item 
Für welchen Wert von $c$ hat das Gleichungssystem mindestens eine Lösung?
\item
Finden Sie die Lösungsmenge des Gleichungssystems mit den Werten der
Variablen $a$ und $c$, die Sie in den ersten beiden Teilaufgaben gefunden
haben.
\end{teilaufgaben}

\thema{Gauss-Algorithmus}
\themaL{Losungsmenge}{Lösungsmenge}

\begin{loesung}
Wir führen zunächst den Gauss-Algorithmus für die Matrix $A$ und die rechte
Seite $b$ durch:
\begin{align}
\begin{tabular}{|>{$}c<{$}>{$}c<{$}>{$}c<{$}|>{$}c<{$}|}
\hline
 4& 20&  4&    -8\\
 1&  7&  7&     8\\
 1&  9&  a&  2a-c\\
\hline
\end{tabular}
&\rightarrow
\begin{tabular}{|>{$}c<{$}>{$}c<{$}>{$}c<{$}|>{$}c<{$}|}
\hline
 1&  5&    1&      -2\\
 0&  2&    6&      10\\
 0&  4&  a-1&  2a-c+2\\
\hline
\end{tabular}
\notag
\\
&\rightarrow
\begin{tabular}{|>{$}c<{$}>{$}c<{$}>{$}c<{$}|>{$}c<{$}|}
\hline
 1&  5&     1&     -2\\
 0&  1&     3&      5\\
 0&  0&  a-13&  2a-18-c\\
\hline
\end{tabular}
\label{10000064:last}
\\
&\rightarrow
\begin{tabular}{|>{$}c<{$}>{$}c<{$}>{$}c<{$}|>{$}c<{$}|}
\hline
 1&  0&   -14&    -27                \\
 0&  1&     3&      5                \\
 0&  0&  a-13&  {\color{red}2a-18-c} \\
\hline
\end{tabular}
\notag
\end{align}
Wir haben im letzten Schritt bereits mit dem Rückwärtseinsetzen begonnen,
weil in der Aufgabe verlangt wird, dass wir nur den Fall behandeln, in
dem der Rang 2 ist.
Damit können wir jetzt die Teilaufgaben lösen.
\begin{teilaufgaben}
\item
Rang 2 tritt auf, wenn die letzte Zeile von~\eqref{10000064:last}
eine Nullzeile ist.
Dazu muss $a=13$ sein.
\item
Damit das Gleichungssystem eine Lösung hat, muss der {\color{red}rote}
Ausdruck verschwinden, also
\[
0 = 2a-18-c =26-18-c = 8-c
\qquad\Rightarrow\qquad
c=8.
\]
\item
Die dritte Variable ist frei wählbar, so dass die Lösungsmenge
\[
\mathbb{L}
=
\left\{
\left.
\begin{pmatrix}x\\y\\z\end{pmatrix}
=
\begin{pmatrix*}[r]-27\\5\\0\end{pmatrix*}
+z
\begin{pmatrix*}[r]14\\-3\\1\end{pmatrix*}
\;\right|\;
z\in\mathbb{R}
\right\}
\qedhere
\]
\end{teilaufgaben}
\end{loesung}

\begin{bewertung}
Gauss-Algorithmus ({\bf G}) 3 Punkte, pro Schritt ein Punkt,
Bestimmung des Wertes für $a$ ({\bf A}) 1 Punkt,
Bestimmung des Wertes für $c$ ({\bf C}) 1 Punkt,
Lösungsmenge ({\bf L}) 1 Punkt.
\end{bewertung}

