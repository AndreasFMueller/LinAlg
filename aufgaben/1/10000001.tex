L"osen Sie die folgenden Gleichungssysteme und geben Sie an,
ob das System genau eine, keine oder unendlich viele L"osungen
hat.
\begin{teilaufgaben}
\item
\[
\begin{linsys}{2}
 x & + & 3y & = &  1\\
-x & + & 5y & = & -1
\end{linsys}
\]
\item
\[
\begin{linsys}{3}
x & + & 2y & + &  z & = & 1\\
  &   &  y & + & 2z & = & 3\\
x & + & 2y & + & 2z & = & 5 
\end{linsys}
\]
\item
\[
\begin{linsys}{2}
  x & - & 3y & = & 4 \\
-2x & + & 6y & = & 5
\end{linsys}
\]
\item
\[
\begin{linsys}{2}
  x & - & 3y & =  4 \\
-2x & + & 6y & = -8
\end{linsys}
\]
\end{teilaufgaben}

\begin{loesung}
\begin{teilaufgaben}
\item
Fr"uher h"atten Sie diese Aufgabe vielleicht so gel"ost:
Addition ergibt
\[
\begin{linsys}{2}
x&+&3y&=&1\\
&&8y&=&0
\end{linsys}
\]
so dass $y=0$ sein muss. Einsetzen in die erste Gleichung ergibt
$x=1$.
 
Mit dem Gauss-Algorithmus entstehen die folgenden Tableaux:
\begin{align*}
\begin{tabular}{|>{$}c<{$}>{$}c<{$}|>{$}c<{$}|}
\hline
1&3&1\\
-1&5&-1\\
\hline
\end{tabular}
&
\rightarrow
\begin{tabular}{|>{$}c<{$}>{$}c<{$}|>{$}c<{$}|}
\hline
1&3&1\\
0&8&0\\
\hline
\end{tabular}
\rightarrow
\begin{tabular}{|>{$}c<{$}>{$}c<{$}|>{$}c<{$}|}
\hline
1&3&1\\
0&1&0\\
\hline
\end{tabular}
\rightarrow
\begin{tabular}{|>{$}c<{$}>{$}c<{$}|>{$}c<{$}|}
\hline
1&0&1\\
0&1&0\\
\hline
\end{tabular}
\end{align*}
\item
Fr"uher h"atten Sie diese Aufgabe vielleicht so gel"ost:
Subtrahiert man die erste Gleichung von der letzten erh"alt man
$z=4$. Setzt man dies in die Gleichungen ein, erh"alt man
das Gleichugnssystem
$$
\begin{linsys}{2}
x&+&2y&=&-3\\
 & & y&=&-5
\end{linsys}
$$
d.~h.~man kann aus der zweiten Gleichung $y=-5$ ablesen, und zusammen
mit der ersten Gleichung folgt $x=7$.

Mit dem Gauss-Algorithmus wird daraus
\begin{align*}
\begin{tabular}{|>{$}c<{$}>{$}c<{$}>{$}c<{$}|>{$}c<{$}|}
\hline
1&2&1&-3\\
0&1&2&3\\
1&2&2&5\\
\hline
\end{tabular}
&
\rightarrow
\begin{tabular}{|>{$}c<{$}>{$}c<{$}>{$}c<{$}|>{$}c<{$}|}
\hline
1&2&1&1\\
0&1&2&3\\
0&0&1&4\\
\hline
\end{tabular}
\rightarrow
\begin{tabular}{|>{$}c<{$}>{$}c<{$}>{$}c<{$}|>{$}c<{$}|}
\hline
1&2&0&-3\\
0&1&0&-5\\
0&0&1&4\\
\hline
\end{tabular}
\rightarrow
\begin{tabular}{|>{$}c<{$}>{$}c<{$}>{$}c<{$}|>{$}c<{$}|}
\hline
1&0&0& 7\\
0&1&0&-5\\
0&0&1&4\\
\hline
\end{tabular}
\end{align*}
\item
Fr"uher h"atten Sie die Aufgaben vielleicht so gel"ost:
Multipliziert man die erste Gleichung mit $2$ und die zweite mit $-1$,
erh"alt man
\[
\begin{linsys}{2}
 2x & - & 6y & = & 8 \\
 2x & - & 6y & = &-5
\end{linsys}
\]
Diese beiden Gleichungen widersprechen sich, es ist unm"oglich beide
Gleichungen zu erf"ullen, das System hat also keine L"osung.

Mit dem Gauss-Algorithmus finden sie folgende Tableaux:
\begin{align*}
\begin{tabular}{|>{$}c<{$}>{$}c<{$}|>{$}c<{$}|}
\hline
1&-3&4\\
-2&6&5\\
\hline
\end{tabular}
&
\rightarrow
\begin{tabular}{|>{$}c<{$}>{$}c<{$}|>{$}c<{$}|}
\hline
1&-3&4\\
0& 0&13\\
\hline
\end{tabular}
\end{align*}
An dieser Stelle kommt der Gauss-Algorithmus nicht mehr weiter, weil
er eine Nullzeile gefunden hat. Auf der rechten Seite steht $-13$,
wir haben hier also eine Gleichung $0=-13$, welche niemals erf"ullt
werden kann. Also gibt es keine L"osung.
\item
Fr"uher h"atten Sie diese Aufgabe vielleicht wie folgt gel"ost:
Die gleichen Operationen wie in Teilaufgabe c) f"uhren hier auf
\[
\begin{linsys}{2}
 2x & - & 6y & =  8 \\
 2x & - & 6y & =  8
\end{linsys}
\]
Die beiden Gleichungen sind also im wesentlichen identisch, die L"osungsmenge
ist 
\[
\mathbb L=\{ (x,y)| x=4+3y\}
\]
also unendlich.

Mit dem Gauss-Algorithmus bekommen Sie folgende Tableaux:
\begin{align*}
\begin{tabular}{|>{$}c<{$}>{$}c<{$}|>{$}c<{$}|}
\hline
1&-3&4\\
-2&6&-8\\
\hline
\end{tabular}
&
\rightarrow
\begin{tabular}{|>{$}c<{$}>{$}c<{$}|>{$}c<{$}|}
\hline
1&-3&4\\
0& 0&0\\
\hline
\end{tabular}
\end{align*}
Die letzte Zeile entspricht der Gleichung $0=0$ die immer
zutrifft. Die Gleichungen gen"ugen offenbar nicht, um $y$ 
festzulegen, $y$ ist also frei w"ahlbar und $x$ kann mit
der ersten Gleichung daraus berechnet werden:
\[
x-3y=4\quad\Rightarrow\quad x=4+3y.
\]
\end{teilaufgaben}
\end{loesung}
