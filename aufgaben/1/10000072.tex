Es gibt zwei Werte des Parameters $k$, für die 
das Gleichungssystem mit der Koeffizientenmatrix
\[
A=\begin{pmatrix}
 k& 5k&    k  \\
 3& 17&   -7  \\
 1&  7& 2k -12
\end{pmatrix}
\qquad\text{und rechter Seite}\qquad
b=\begin{pmatrix}
k(l+1)\\0\\l
\end{pmatrix}
\]
singulär ist.
\begin{teilaufgaben}
\item Finden Sie die beiden Werte von $k$.
\item Für beide Werte von $k$ bestimmen Sie jeweils die Werte von $l$,
für die das Gleichungssystem unendlich viele Lösungen hat.
\end{teilaufgaben}

\begin{loesung}
Wir lösen das Gleichungssystem mit dem Gauss-Algorithmus.
Dabei stellen wir fest, dass die erste Zeile zu einer Nullzeile wird,
wenn $k=0$ ist.
Das Gauss-Tableau wird in diesem Fall
\[
\begin{tabular}{|>{$}c<{$}>{$}c<{$}>{$}c<{$}|>{$}c<{$}|}
\hline
 0&  0&  0& 0\\
 3& 17& -7& 0\\
 1&  7&-12& l \\
\hline
\end{tabular}
\to
\begin{tabular}{|>{$}c<{$}>{$}c<{$}>{$}c<{$}|>{$}c<{$}|}
\hline
 1&  7&-12& l \\
 3& 17& -7& 0\\
 0&  0&  0& 0\\
\hline
\end{tabular}
\to
\begin{tabular}{|>{$}c<{$}>{$}c<{$}>{$}c<{$}|>{$}c<{$}|}
\hline
 1&  7&-12& l \\
 0& -4& 29& 0\\
 0&  0&  0& 0\\
\hline
\end{tabular}.
\]
Insbesondere sieht man, dass der Wert von $l$ keinen Einfluss darauf
hat, ob das Gleichungssystem im Fall $k=0$ unendlich viele oder gar
keine Lösung hat.
In diesem Fall ist also $l\in\mathbb{R}$.

Für $k\ne 0$ dürfen wir den Gauss-Algorithmus wie gewohnt durchführen
und erhalten
\begin{align*}
\begin{tabular}{|>{$}c<{$}>{$}c<{$}>{$}c<{$}|>{$}c<{$}|}
\hline
 k& 5k&    k  &k(l+1)\\
 3& 17&   -7  & 0\\
 1&  7& 2k -12& l \\
\hline
\end{tabular}
&\rightarrow
\begin{tabular}{|>{$}c<{$}>{$}c<{$}>{$}c<{$}|>{$}c<{$}|}
\hline
 1& 5&    1  &l+1\\
 0& 2&   -10  &-3l-3\\
 0& 2& 2k -13& -1 \\
\hline
\end{tabular}
\\
&\rightarrow
\begin{tabular}{|>{$}c<{$}>{$}c<{$}>{$}c<{$}|>{$}c<{$}|}
\hline
 1& 5&    1  &l+1\\
 0& 1&   -5  &-\frac32(l+1)\\
 0& 0& 2k -3& 3l+2 \\
\hline
\end{tabular}.
\end{align*}
Damit das Gleichungssystem unendlich viele Lösungen hat, muss eine Nullzeile
entstehen, also $2k-3=0$ und $3l+2=0$, was auf $k=\frac32$ und $l=-\frac23$ 
führt.

Zusammengefasst:
\begin{teilaufgaben}
\item $k\in \{0,\frac32\}$.
\item Falls $k=0$ kann $l\in\mathbb{R}$ beliebig sein. 
Falls $k=\frac32$ muss $l=-\frac23$ sein.
\qedhere
\end{teilaufgaben}
\end{loesung}

\begin{bewertung}
Gauss-Algorithmus ({\bf G}) 2 Punkte,
Spezialfall $k=0$ ({\bf K}) 1 Punkt,
Kein Einfluss von $l$ im Fall $k=0$ ({\bf L}$\mathstrut_0$), 1 Punkt,
Zweiter Wert von $k$ ({\bf K}$\mathstrut_1$) 1 Punkt,
und zugehöriger $l$-Wert ({\bf L}$\mathstrut_1$) 1 Punkt.
\end{bewertung}

\begin{diskussion}
Man kann die beiden Werte von $k$, für die die Koeffizientenmatrix
singulär wird, natürlich auch mit der Determinanten finden.
Sie ist
\[
\det(A)
=
\left|
\begin{matrix}
 k& 5k&    k  \\
 3& 17&   -7  \\
 1&  7& 2k -12
\end{matrix}
\right|
=
4k^2-6k = 2k(2k-3).
\]
Die Determinante verschwindet also genau für $k=0$ und $k=\frac32$.
Es bleibt, wie vorhin die Werte von $l$ mit dem Gauss-Algorithmus zu
bestimmen.
\end{diskussion}
