Ist $\frac{23}{67}=\frac{33}{97}$? Diese Frage stellt David Pengelley
in der Dezember-Ausgabe der Zeitschrift {\it The American Mathematical
Monthly}. Natürlich erwartet er eine andere Lösung als einfach nur
in den Taschenrechner eintippen. Zum Beispiel sind die beiden Brüche
gleich, wenn die Vektoren
\[
\begin{pmatrix}
23\\67
\end{pmatrix}
\qquad\text{und}\qquad
\begin{pmatrix}
33\\97
\end{pmatrix}
\]
linear abhängig sind, oder wenn die Matrix
\[
A=
\begin{pmatrix}
23&33\\
67&97
\end{pmatrix}
\]
singulär ist. Diese Idee kann man auch auf drei Brüche erweitern: Die
Brüche $\frac{13}{37}$, $\frac{23}{67}$ und $\frac{33}{97}$ sind genau
dann gleich, wenn die Matrix
\[
B=\begin{pmatrix}
13&23&33\\
37&67&97
\end{pmatrix}
\]
den Rang 1 hat.

\thema{lineare Abhängigkeit}
\thema{Rang}

\begin{teilaufgaben}
\item Wie können Sie prüfen, ob $B$ den Rang 1 hat? Führen Sie die
Rechnung durch.
\item Einige Mikrokontroller haben keine Multiplikationsoperation, und
die meisten haben keine Divisionsoperation.
Können Sie Ihre Lösung von a) so verallgemeinern, dass sie nur mit
Addition und Subtraktion auskommt?
\end{teilaufgaben}

\begin{loesung}
\begin{teilaufgaben}
\item
Den Rang bestimmt man, indem man den Gauss-Algorithmus durchführt. Für
die Matrix $B$ führt er auf die folgenden Tableaux:
\[
\begin{tabular}{|>{$}c<{$}>{$}c<{$}>{$}c<{$}|}
\hline
13&23&33\\
37&67&97\\
\hline
\end{tabular}
\rightarrow
\begin{tabular}{|>{$}c<{$}>{$}c<{$}>{$}c<{$}|}
\hline
1&\frac{23}{13}&\frac{33}{13}\\
0&\frac{20}{13}&\frac{40}{13}\\
\hline
\end{tabular}
\rightarrow
\begin{tabular}{|>{$}c<{$}>{$}c<{$}>{$}c<{$}|}
\hline
1&0&-1\\
0&1& 2\\
\hline
\end{tabular}.
\]
\item Der Gaussalgorithmus funktioniert offenbar nicht direkt, da
er immer wieder Divisionen durch Pivot-Elemente verlangt. Die ``roten''
Operationen sind also nicht zulässig, man muss mit den ``blauen''
Operationen, den Zeilenoperationen auskommen.

Ohne Multiplikationen muss man halt mit wiederholten Additionen oder
Subtraktionen das gleiche Ziel zu erreichen versuchen. Also subtrahiert
man die erste Zeile so lange von der zweiten, bis dort etwas ``kleines''
steht:
\[
\begin{tabular}{|>{$}c<{$}>{$}c<{$}>{$}c<{$}|}
\hline
13&23&33\\
37&67&97\\
\hline
\end{tabular}
\rightarrow
\begin{tabular}{|>{$}c<{$}>{$}c<{$}>{$}c<{$}|}
\hline
13&23&33\\
24&44&64\\
\hline
\end{tabular}
\rightarrow
\begin{tabular}{|>{$}c<{$}>{$}c<{$}>{$}c<{$}|}
\hline
13&23&33\\
11&21&31\\
\hline
\end{tabular}
\rightarrow
\begin{tabular}{|>{$}c<{$}>{$}c<{$}>{$}c<{$}|}
\hline
13&23&33\\
-2&-2&-2\\
\hline
\end{tabular}
\]
Jetzt addiert man die zweite Zeile so oft zur ersten, bis in der
linken oberen Ecke eine 1 steht:
\[
\begin{tabular}{|>{$}c<{$}>{$}c<{$}>{$}c<{$}|}
\hline
13&23&33\\
-2&-2&-2\\
\hline
\end{tabular}
\rightarrow
\begin{tabular}{|>{$}c<{$}>{$}c<{$}>{$}c<{$}|}
\hline
11&21&31\\
-2&-2&-2\\
\hline
\end{tabular}
\rightarrow
\dots
\rightarrow
\begin{tabular}{|>{$}c<{$}>{$}c<{$}>{$}c<{$}|}
\hline
 1&11&21\\
-2&-2&-2\\
\hline
\end{tabular}
\]
Wenn man jetzt noch die erste Zeile zweimal zur zweiten hinzuaddiert,
erhält man 
\[
\begin{tabular}{|>{$}c<{$}>{$}c<{$}>{$}c<{$}|}
\hline
 1&11&21\\
-2&-2&-2\\
\hline
\end{tabular}
\rightarrow
\begin{tabular}{|>{$}c<{$}>{$}c<{$}>{$}c<{$}|}
\hline
 1&11&21\\
 0&20&40\\
\hline
\end{tabular}
\]
Jetzt hat man das gleiche erreicht wie im Gauss-Algorithmus: in der
linken oberen Ecke steht eine Eins, darunter nur Nullen. Das reicht
aber um ablesen zu können, dass die zweite Zeile keine Nullzeile
wird, dass der Rang also 2 ist.
\qedhere
\end{teilaufgaben}
\end{loesung}

\begin{bewertung}
Definition des Ranges ({\bf R}) 1 Punkt,
Durchführung des Gaussalgorithmus ({\bf G}) 2 Punkte,
Wert des Ranges ({\bf W}) 1 Punkt.
Ersetzen der Multiplikationen durch wiederholte Additionen/Subtraktionen
({\bf A}) 1 Punkt, Durchführung der Berechnung bis eine
Schlussfolgerung möglich ist ({\bf B}) 1 Punkt.
\end{bewertung}

