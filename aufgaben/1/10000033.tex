F"ur die Standardbasisvektoren $e_1$, $e_2$ und $e_3$ und das Vektorprodukt
gilt die Algebra
\begin{align*}
e_1\times e_2&=0,&
e_1\times e_2&=e_3,\\
e_2\times e_2&=0,&
e_2\times e_3&=e_1,\\
e_3\times e_3&=0,&
e_3\times e_1&=e_2.
\end{align*}
Rechnen Sie nach, dass das dieselbe Algebra f"ur die Matrizen
\begin{align*}
E_1&=\frac12\begin{pmatrix}
 0& 0& 0& 1\\
 0& 0&-1& 0\\
 0& 1& 0& 0\\
-1& 0& 0& 0
\end{pmatrix},
&
E_2&=\frac12\begin{pmatrix}
 0& 0&-1& 0\\
 0& 0& 0&-1\\
 1& 0& 0& 0\\
 0& 1& 0& 0
\end{pmatrix},
&
E_3&=\frac12\begin{pmatrix}
 0& 1& 0& 0\\
-1& 0& 0& 0\\
 0& 0& 0&-1\\
 0& 0& 1& 0
\end{pmatrix}
&
\end{align*}
und den {\em Kommutator} $[A,B]=AB-BA$ also,
\begin{align*}
[E_1, E_2]&=0,&
[E_1, E_2]&=E_3,\\
[E_2, E_2]&=0,&
[E_2, E_3]&=E_1,\\
[E_3, E_3]&=0,&
[E_3, E_1]&=E_2.
\end{align*}

\begin{loesung}
Der Kommuator einer Matrix mit sich selbst ist $[A,A]=AA-AA=0$, die
Gleichungen in der ersten Spalte sind also automatisch erf"ullt.
Wir m"ussen jetzt die gemischten Kommutatoren nachrechnen:
\begin{align*}
[E_1,E_2]&=E_1E_2-E_2E_1=\frac14\left[\;
\begin{pmatrix}
 0& 1& 0& 0\\
-1& 0& 0& 0\\
 0& 0& 0&-1\\
 0& 0& 1& 0
\end{pmatrix}
-
\begin{pmatrix}
 0&-1& 0& 0\\
 1& 0& 0& 0\\
 0& 0& 0& 1\\
 0& 0&-1& 0
\end{pmatrix}\;
\right]
=E_3,
\\
[E_2,E_3]&=E_2E_3-E_3E_2=\frac14\left[\;
\begin{pmatrix}
 0& 0& 0& 1\\
 0& 0&-1& 0\\
 0& 1& 0& 0\\
-1& 0& 0& 0
\end{pmatrix}
-
\begin{pmatrix}
 0& 0& 0&-1\\
 0& 0& 1& 0\\
 0&-1& 0& 0\\
 1& 0& 0& 0
\end{pmatrix}\;
\right]
=E_1,
\\
[E_3,E_1]&=E_3E_1-E_1E_3=\frac14\left[\;
\begin{pmatrix}
 0& 0&-1& 0\\
 0& 0& 0&-1\\
 1& 0& 0& 0\\
 0& 1& 0& 0
\end{pmatrix}
-
\begin{pmatrix}
 0& 0& 1& 0\\
 0& 0& 0& 1\\
-1& 0& 0& 0\\
 0&-1& 0& 0
\end{pmatrix}\;
\right]
=E_2.
\end{align*}
Damit ist gezeigt, dass die Kommutatorgleichungen eingehalten werden.
\end{loesung}

\begin{diskussion}
Dies ist eine reelle Variante der Algebra der sogenannten Pauli-Matrizen,
die in der Quantenmechanik eine bedeutende Rolle spielen.
\end{diskussion}
