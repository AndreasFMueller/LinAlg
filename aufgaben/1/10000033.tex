Für die Standardbasisvektoren $e_1$, $e_2$ und $e_3$ und das Vektorprodukt
gilt die Algebra
\begin{align*}
e_1\times e_1&=0,&
e_1\times e_2&=e_3,\\
e_2\times e_2&=0,&
e_2\times e_3&=e_1,\\
e_3\times e_3&=0,&
e_3\times e_1&=e_2.
\end{align*}
\begin{teilaufgaben}
\item
Rechnen Sie nach, dass dieselbe Algebra für die Matrizen
\begin{align*}
E_1&=\frac12\begin{pmatrix}
 0& 0& 0& 1\\
 0& 0&-1& 0\\
 0& 1& 0& 0\\
-1& 0& 0& 0
\end{pmatrix},
&
E_2&=\frac12\begin{pmatrix}
 0& 0&-1& 0\\
 0& 0& 0&-1\\
 1& 0& 0& 0\\
 0& 1& 0& 0
\end{pmatrix},
&
E_3&=\frac12\begin{pmatrix}
 0& 1& 0& 0\\
-1& 0& 0& 0\\
 0& 0& 0&-1\\
 0& 0& 1& 0
\end{pmatrix}
&
\end{align*}
und den {\em Kommutator} $[A,B]=AB-BA$ gilt, also
\begin{align*}
[E_1, E_1]&=0,&
[E_1, E_2]&=E_3,\\
[E_2, E_2]&=0,&
[E_2, E_3]&=E_1,\\
[E_3, E_3]&=0,&
[E_3, E_1]&=E_2.
\end{align*}
\item
Rechnen Sie ausserdem nach:
Definiert man das Skalarprodukt von zwei Matrizen $A$ und $B$ als
$\operatorname{Spur}(A^tB)$
(für Vektoren ist dies bis auf die Spur die Definition des Skalarproduktes),
dann sind $E_1$, $E_2$
und $E_3$ bezüglich dieses Skalarproduktes orthonormiert.
\end{teilaufgaben}

\thema{Matrizenprodukt}

\begin{loesung}
\begin{teilaufgaben}
\item
Der Kommutator einer Matrix mit sich selbst ist $[A,A]=AA-AA=0$, die
Gleichungen in der ersten Spalte sind also automatisch erfüllt.
Wir müssen jetzt die gemischten Kommutatoren nachrechnen:
\begin{align*}
[E_1,E_2]&=E_1E_2-E_2E_1=\frac14\left[\;
\begin{pmatrix}
 0& 1& 0& 0\\
-1& 0& 0& 0\\
 0& 0& 0&-1\\
 0& 0& 1& 0
\end{pmatrix}
-
\begin{pmatrix}
 0&-1& 0& 0\\
 1& 0& 0& 0\\
 0& 0& 0& 1\\
 0& 0&-1& 0
\end{pmatrix}\;
\right]
=E_3,
\\
[E_2,E_3]&=E_2E_3-E_3E_2=\frac14\left[\;
\begin{pmatrix}
 0& 0& 0& 1\\
 0& 0&-1& 0\\
 0& 1& 0& 0\\
-1& 0& 0& 0
\end{pmatrix}
-
\begin{pmatrix}
 0& 0& 0&-1\\
 0& 0& 1& 0\\
 0&-1& 0& 0\\
 1& 0& 0& 0
\end{pmatrix}\;
\right]
=E_1,
\\
[E_3,E_1]&=E_3E_1-E_1E_3=\frac14\left[\;
\begin{pmatrix}
 0& 0&-1& 0\\
 0& 0& 0&-1\\
 1& 0& 0& 0\\
 0& 1& 0& 0
\end{pmatrix}
-
\begin{pmatrix}
 0& 0& 1& 0\\
 0& 0& 0& 1\\
-1& 0& 0& 0\\
 0&-1& 0& 0
\end{pmatrix}\;
\right]
=E_2.
\end{align*}
Damit ist gezeigt, dass die Kommutatorgleichungen eingehalten werden.
\item
Wir müssen nachrechnen, dass $E_i^tE_j=\delta_{ij}$ ist.
Wir können die Rechnung etwas vereinfachen, wenn wir berücksichtigen,
dass die Matrizen alle antisymmetrisch sind, dass also $E_i^t=-E_i$ ist.
Die gemischten Produkte $E_iE_j$ haben wir oben alle ausgerechnet, und
man kann ablesen, dass auf der Diagonalen nur Nullen stehen. Also gilt
\[
\operatorname{Spur}E_i^tE_j
=
-\operatorname{Spur}E_iE_j=0
\qquad
\text{für $i\ne j$}.
\]
Für $i=j$ rechnen wir
\begin{align*}
\operatorname{Spur}E_1^tE_1
&=-\operatorname{Spur}E_1^2
=-\operatorname{Spur}\frac14\begin{pmatrix}
-1& 0& 0& 0\\
 0&-1& 0& 0\\
 0& 0&-1& 0\\
 0& 0& 0&-1
\end{pmatrix}
=1,
\\
\operatorname{Spur}E_2^tE_2
&=-\operatorname{Spur}E_2^2
=-\operatorname{Spur}\frac14\begin{pmatrix}
-1& 0& 0& 0\\
 0&-1& 0& 0\\
 0& 0&-1& 0\\
 0& 0& 0&-1
\end{pmatrix}
=1,
\\
\operatorname{Spur}E_3^tE_3
&=-\operatorname{Spur}E_3^2
=-\operatorname{Spur}\frac14\begin{pmatrix}
-1& 0& 0& 0\\
 0&-1& 0& 0\\
 0& 0&-1& 0\\
 0& 0& 0&-1
\end{pmatrix}
=1.
\end{align*}
Damit ist gezeigt, dass die Matrizen bezüglich dieses Skalarproduktes
orthonormiert sind.
\qedhere
\end{teilaufgaben}
\end{loesung}

\begin{diskussion}
Statt in der Standardbasisvektoren könnte man die ganze Vektorgeometrie
also auch in einer Basis bestehend aus $E_1$, $E_2$ und $E_3$ machen.
Diese Matrizen sind eine reelle Variante der sogenannten
Pauli-Matrizen,
die in der Quantenmechanik eine bedeutende Rolle spielen.
Das Skalarprodukt der Matrizen $\operatorname{Spur}A^tB$ ist übrigens
genau das Skalarprodukt der Vektoren. Nimmt man für $A$ und $B$
$n$-dimensionale Spaltenvektoren, dann ist $A^tB$ eine $1\times 1$-Matrix,
und die Spur davon ist genau das übliche Skalarprodukt von Spaltenvektoren.
\end{diskussion}

\begin{bewertung}
\begin{teilaufgaben}
\item
Berechnung der Kommutatoren in $[E_i,E_j]$ mit $i\ne j$ (die Fälle
$i=j$ sind trivial) ({\bf K}) je 1 Punkt.
\item
Berechnung $\operatorname{Spur}E_i^tE_j=0$ für ein Paar $i\ne j$ ({\bf O})
1 Punkt,
Berechnung der Norm 
$\operatorname{Spur}E_i^tE_i=1$ für ein $i$ ({\bf N}) 2 Punkte.
\end{teilaufgaben}
\end{bewertung}
