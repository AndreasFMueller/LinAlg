Finden Sie die reduzierte Zeilenstufenform des Tableaus
\begin{center}
\begin{tabular}{| >{$}r<{$}  >{$}r<{$}  >{$}r<{$}  >{$}r<{$}  >{$}r<{$}  >{$}r<{$} | >{$}r<{$} |}
\hline
    3 & -3 & -3 &-12 & -9 &  0 &-42\\
   -5 &  7 &  3 & 20 & 17 &  4 & 68\\
   -5 &  3 &  7 & 25 & 13 & -9 & 92\\
   -5 &  3 &  7 & 20 & 13 & -7 & 75\\
\hline
\end{tabular}
\end{center}

\begin{loesung}
Wir verwenden den Gauss-Algorithmus ohne Spaltenvertauschungen:
\begin{align*}
\begin{tabular}{| >{$}r<{$}  >{$}r<{$}  >{$}r<{$}  >{$}r<{$}  >{$}r<{$}  >{$}r<{$} | >{$}r<{$} |}
\hline
    3
\begin{picture}(0,0)
\color{red}\put(-3.5,3){\circle{12}}
\end{picture}
      & -3 & -3 &-12 & -9 &  0 &-42\\
   -5
\begin{picture}(0,0)
\color{blue}\drawline(-16,-30)(-16,10)(3,10)(3,-30)
\end{picture}
      &  7 &  3 & 20 & 17 &  4 & 68\\
   -5 &  3 &  7 & 25 & 13 & -9 & 92\\
   -5 &  3 &  7 & 20 & 13 & -7 & 75\\
\hline
\end{tabular}
&\rightarrow
\begin{tabular}{| >{$}r<{$}  >{$}r<{$}  >{$}r<{$}  >{$}r<{$}  >{$}r<{$}  >{$}r<{$} | >{$}r<{$} |}
\hline
\color{red}
    1 & -1 & -1 & -4 & -3 &  0 &-14\\
    0 &  2
\begin{picture}(0,0)
\color{red}\put(-3.5,3){\circle{12}}
\end{picture}
           & -2 &  0 &  2 &  4 & -2\\
    0 & -2
\begin{picture}(0,0)
\color{blue}\drawline(-16,-16)(-16,10)(3,10)(3,-16)
\end{picture}
           &  2 &  5 & -2 & -9 & 22\\
    0 & -2 &  2 &  0 & -2 & -7 &  5\\
\hline
\end{tabular}
\\
\rightarrow
\begin{tabular}{| >{$}r<{$}  >{$}r<{$}  >{$}r<{$}  >{$}r<{$}  >{$}r<{$}  >{$}r<{$} | >{$}r<{$} |}
\hline
\color{red}
    1 & -1 & -1 & -4 & -3 &  0 &-14\\
    0 &\color{red}  1 & -1 &  0 &  1 &  2 & -1\\
    0 &  0 &  0 &  5
\begin{picture}(0,0)
\color{red}\put(-3.5,3){\circle{12}}
\end{picture}
                     &  0 & -5 & 20\\
    0 &  0 &  0 &  0
\begin{picture}(0,0)
\color{blue}\drawline(-16,-2)(-16,10)(3,10)(3,-2)
\end{picture}
                     &  0 & -3 &  3\\
\hline
\end{tabular}
&\rightarrow
\begin{tabular}{| >{$}r<{$}  >{$}r<{$}  >{$}r<{$}  >{$}r<{$}  >{$}r<{$}  >{$}r<{$} | >{$}r<{$} |}
\hline
\color{red}
    1 & -1 & -1 & -4 & -3 &  0 &-14\\
    0 & \color{red} 1 & -1 &  0 &  1 &  2 & -1\\
    0 &  0 &  0 & \color{red} 1 &  0 & -1 &  4\\
    0 &  0 &  0 &  0 &  0 & -3
\begin{picture}(0,0)
\color{red}\put(-6,3.5){\circle{15}}
\end{picture}
                               &  3\\
\hline
\end{tabular}
\\
\rightarrow
\begin{tabular}{| >{$}r<{$}  >{$}r<{$}  >{$}r<{$}  >{$}r<{$}  >{$}r<{$}  >{$}r<{$} | >{$}r<{$} |}
\hline
\color{red}
    1 & -1 & -1 & -4 & -3 &  0 &-14\\
    0 & \color{red} 1 & -1 &  0 &  1 &  2 & -1\\
    0 &  0 &  0 & \color{red} 1 &  0 & -1 
\begin{picture}(0,0)
\color{blue}\drawline(-16,38)(-16,-2)(3,-2)(3,38)
\end{picture}
                               &  4\\
    0 &  0 &  0 &  0 &  0 & \color{red} 1 & -1\\
\hline
\end{tabular}
&\rightarrow
\begin{tabular}{| >{$}r<{$}  >{$}r<{$}  >{$}r<{$}  >{$}r<{$}  >{$}r<{$}  >{$}r<{$} | >{$}r<{$} |}
\hline
\color{red}
    1 & -1 & -1 & -4 & -3 &  0 &-14\\
    0 & \color{red} 1 & -1 &  0
\begin{picture}(0,0)
\color{blue}\drawline(-16,23)(-16,-2)(3,-2)(3,23)
\end{picture}
                     &  1 &  0 &  1\\
    0 &  0 &  0 & \color{red} 1 &  0 &  0 &  3\\
    0 &  0 &  0 &  0 &  0 & \color{red} 1 & -1\\
\hline
\end{tabular}
\\
\rightarrow
\begin{tabular}{| >{$}r<{$}  >{$}r<{$}  >{$}r<{$}  >{$}r<{$}  >{$}r<{$}  >{$}r<{$} | >{$}r<{$} |}
\hline
\color{red}
    1 & -1
\begin{picture}(0,0)
\color{blue}\drawline(-16,9)(-16,-2)(3,-2)(3,9)
\end{picture}
           & -1 &  0 & -3 &  0 & -2\\
    0 & \color{red} 1 & -1 &  0 &  1 &  0 &  1\\
    0 &  0 &  0 & \color{red} 1 &  0 &  0 &  3\\
    0 &  0 &  0 &  0 &  0 & \color{red} 1 & -1\\
\hline
\end{tabular}
&\rightarrow
\begin{tabular}{| >{$}r<{$}  >{$}r<{$}  >{$}r<{$}  >{$}r<{$}  >{$}r<{$}  >{$}r<{$} | >{$}r<{$} |}
\hline
\color{red}
    1 &  0 & -2 &  0 & -2 &  0 & -1\\
    0 &\color{red}  1 & -1 &  0 &  1 &  0 &  1\\
    0 &  0 &  0 & \color{red} 1 &  0 &  0 &  3\\
    0 &  0 &  0 &  0 &  0 & \color{red} 1 & -1\\
\hline
\end{tabular}
\end{align*}
Zur Verdeutlichung sind die Einsen an Pivot-Positionen {\color{red}rot} hervorgehoben.
\end{loesung}

\begin{diskussion}
Weitere Aufgaben dieser Art, bei denen w"ahrend der Berechnung der reduzierten
Zeilenstufenform keine Br"uche auftreten, k"onnen mit dem Octave-/Matlab-Programm
\texttt{10000038.m} erzeugt werden.
\end{diskussion}

