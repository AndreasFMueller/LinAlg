Gegeben sind die Matrizen
\[
A=\begin{pmatrix}
-11&6\\
-20&11
\end{pmatrix}
\qquad
\text{und}
\qquad
B=
\begin{pmatrix}
1&3\\
2&5
\end{pmatrix}.
\]
\begin{teilaufgaben}
\item Berechnen Sie $B^{-1}AB$.
\item Berechnen Sie $A^{2012}$.
\end{teilaufgaben}

\begin{loesung}
\begin{teilaufgaben}
\item
Zun"achst berechnen wir die inverse Matrix von $B$ mit dem Gauss-Algorithmus:
\begin{align*}
\begin{tabular}{|>{$}c<{$}>{$}c<{$}|>{$}c<{$}>{$}c<{$}|}
\hline
1&3&1&0\\
2&5&0&1\\
\hline
\end{tabular}
&
\rightarrow
\begin{tabular}{|>{$}c<{$}>{$}c<{$}|>{$}c<{$}>{$}c<{$}|}
\hline
1&3&1&0\\
0&-1&-2&1\\
\hline
\end{tabular}
\rightarrow
\begin{tabular}{|>{$}c<{$}>{$}c<{$}|>{$}c<{$}>{$}c<{$}|}
\hline
1&3&1&0\\
0&1&2&-1\\
\hline
\end{tabular}
\\
&\rightarrow
\begin{tabular}{|>{$}c<{$}>{$}c<{$}|>{$}c<{$}>{$}c<{$}|}
\hline
1&3&-5&3\\
0&1&2&-1\\
\hline
\end{tabular}
\end{align*}
\[
B^{-1}
=
\begin{pmatrix}-5&3\\2&-1\end{pmatrix}.
\]

Damit kann man jetzt $B^{-1}AB$ bestimmen:
\begin{align*}
B^{-1}A
&=
\begin{pmatrix}-5&3\\2&-1\end{pmatrix}
\begin{pmatrix} -11&6\\ -20&11 \end{pmatrix}
=
\begin{pmatrix}
55-60&-30+33\\
-22+20&12-11
\end{pmatrix}
=
\begin{pmatrix}
-5&3\\
-2&1
\end{pmatrix},
\\
B^{-1}AB
&=
\begin{pmatrix}
-5&3\\
-2&1
\end{pmatrix}
\begin{pmatrix}
1&3\\
2&5
\end{pmatrix}
=
\begin{pmatrix}
-5+6&-15+15\\
-2+2&-6+5
\end{pmatrix}
=\begin{pmatrix}
1&0\\
0&-1
\end{pmatrix}.
\end{align*}

\item
Berechnung der Potenzen von $A$ liefert:
\begin{align*}
A^2
&=
\begin{pmatrix} -11&6\\-20&11\end{pmatrix}
\begin{pmatrix} -11&6\\-20&11\end{pmatrix}
=\begin{pmatrix}
(-11)\cdot(-11)+6\cdot(-20)&(-11)\cdot 6+6\cdot 11\\
(-20)\cdot(-11)+11\cdot(-20)&(-20)\cdot 6+11\cdot 11
\end{pmatrix}
\\
&=\begin{pmatrix}1&0\\0&1\end{pmatrix}=E\\
A^3&=A^2\cdot A=EA=A\\
A^4&=A^3\cdot A=A\cdot A=A^2=E\\
\end{align*}
Die geraden Potenzen sind also Einheitsmatrizen, ungerade 
Potenzen sind identisch mit $A$:
\begin{align*}
A^{2k}&=E&A^{2k+1}&=A.
\end{align*}
f"ur $k\in \mathbb N$.
\end{teilaufgaben}
\end{loesung}
