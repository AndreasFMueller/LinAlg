Wieviele L"osungen hat das Gleichungssystem
\[
%   1   1   0   1   0   0
%   3   4   1   0   1   0
%   3   6   3   0   0   1
%   1.00000   0.00000  -1.00000   0.00000   1.00000  -0.66667
%   0.00000   1.00000   1.00000   0.00000  -0.50000   0.50000
%   0.00000   0.00000   0.00000   1.00000  -0.50000   0.16667
\begin{linsys}{3}
 x&+& y& &  &=& 3\\
3x&+&4y&+& z&=&17\\
3x&+&6y&+&3z&=&33\\
\end{linsys}
\]
Geben Sie die L"osungsmenge an.

\begin{loesung}
Mit dem Gauss-Algorithmus findet man:
\begin{align*}
\begin{tabular}{|ccc|c|}
\hline
1&1&0& 3\\
3&4&1&17\\
3&6&3&33\\
\hline
\end{tabular}
&\rightarrow
\begin{tabular}{|ccc|c|}
\hline
1&1&0& 3\\
0&1&1& 8\\
0&3&3&24\\
\hline
\end{tabular}
\rightarrow
\begin{tabular}{|ccc|c|}
\hline
1&1&0&1\\
0&1&1&8\\
0&0&0&0\\
\hline
\end{tabular}
\rightarrow
\begin{tabular}{|ccc|c|}
\hline
1&0&$-1$&$-5$\\
0&1&1&8\\
0&0&0&0\\
\hline
\end{tabular}
\end{align*}
Offenbar hat die Gleichung unendlich viele L"osungen,
die man schreiben kann als
L"osungsmenge
\[
\mathbb L=
\left\{
\left.
\begin{pmatrix}-5\\8\\0\end{pmatrix}+z\begin{pmatrix}-1\\1\\0\end{pmatrix}\;\right|\;z\in\mathbb R
\right\}.
\qedhere
\]
\end{loesung}

