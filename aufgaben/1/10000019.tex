Bestimmen Sie die Werte f"ur $a$ und $b$, f"ur die das Gleichungssystem
\[
\begin{linsys}{3}
x&+&2y&+&3z&=&1\\
x& &  &+& z&=&0\\
 & &2y&+&(a+2)z&=&b+1\\
\end{linsys}
\]
in den drei Unbekannten $x$, $y$ und $z$ keine, genau eine bzw.~unendlich
viele L"osungen hat. Bestimmen Sie jeweils die L"osungsmenge.

\begin{loesung}
Wir f"uhren den Gauss-Algorithmus so weit wie m"oglich durch:
\begin{align*}
\begin{tabular}{|ccc|c|}
\hline
1&2&3&1\\
1&0&1&0\\
0&2&$a+2$&$b+1$\\
\hline
\end{tabular}
&\rightarrow
\begin{tabular}{|ccc|c|}
\hline
1&2&3&1\\
0&$-2$&$-2$&$-1$\\
0&2&$a+2$&$b+1$\\
\hline
\end{tabular}
\\
&\rightarrow
\begin{tabular}{|ccc|c|}
\hline
1&2&3&1\\
0&1&1&$\frac12$\\
0&0&$a$&$b$\\
\hline
\end{tabular}
\end{align*}
An dieser Stelle k"onnen wir entscheiden, wieviele L"osungen das Gleichungssystem
hat.
\begin{enumerate}
\item[Fall 1:]
Das Gleichungssystem ist regul"ar, wenn $a\ne 0$. In diesem Fall l"asst sich der
Gauss-Algorithmus zu Ende f"uhren:
\[
\rightarrow
\begin{tabular}{|ccc|c|}
\hline
1&2&0&$1-\frac{3b}{a}$\\
0&1&0&$\frac12-\frac{b}{a}$\\
0&0&1&$\frac{b}{a}$\\
\hline
\end{tabular}
\rightarrow
\begin{tabular}{|ccc|c|}
\hline
1&0&0&$-\frac{b}{a}$\\
0&1&0&$\frac12-\frac{b}{a}$\\
0&0&1&$\frac{b}{a}$\\
\hline
\end{tabular}
\]
Die einzige L"osung ist also $(-\frac{b}a,\frac12-\frac{b}a,\frac{b}a)$, wie man auch durch
Einsetzen kontrollieren kann:
\[
\begin{linsys}{3}
-\frac{b}a&+&2(\frac12-\frac{b}a)&+&3\frac{b}a&=&1\\
-\frac{b}a& &  &+& \frac{b}a&=&0\\
 & &2(\frac12-\frac{b}a)&+&(a+2)\frac{b}a&=&b+1\\
\end{linsys}
\]
\item[Fall 2:] Falls $a=0$ ist das Gleichungssystem singul"ar.
Ist ausserdem $b\ne 0$
ist die letzte Gleichung gleichbedeuten mit $0=b$, ein Widerspruch.
In diesem Fall gibt es also keine L"osung.
\item[Fall 3:] Falls $a=0$ und $b=0$ entsteht im Gauss-Tableau eine Nullzeile,
die letzte
Gleichung ist also redundant, es gibt unendlich viele L"osungen.
Der Gauss-Algorithmus kann wie folgt zu Ende gef"uhrt werden:
\[
\rightarrow
\begin{tabular}{|ccc|c|}
\hline
1&2&3&1\\
0&1&1&$\frac12$\\
0&0&0&0\\
\hline
\end{tabular}
\rightarrow
\begin{tabular}{|ccc|c|}
\hline
1&0&1&0\\
0&1&1&$\frac12$\\
0&0&0&0\\
\hline
\end{tabular}
\]
Daraus kann man die L"osungsmenge
\[
\mathbb L=\left\{
\left.\begin{pmatrix}-z\\-z+\frac12\\z\end{pmatrix}\;\right|\;z\in\mathbb R
\right\}
\]
ablesen.
\qedhere
\end{enumerate}
\end{loesung}


