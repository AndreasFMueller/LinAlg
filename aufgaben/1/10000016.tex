Wieviele Lösungen hat das Gleichungssystem
\[
\begin{linsys}{3}
4a&+&3b&+&2c&=&6\\
2a&+&3b&+&4c&=&0\\
a&+&b&+&c&=&1
\end{linsys}
\qquad
?
\]
Geben Sie die Lösungsmenge in Vektorschreibweise an.

\begin{hinweis}
Gauss-Calculator: \gaussurl{gausscalc:10000016}
\end{hinweis}

\themaL{Losungsmenge}{Lösungsmenge}
\thema{Gauss-Algorithmus}

\begin{loesung}
Die dritte Gleichung ist ein Sechstel der Summe der ersten und der
zweiten Gleichung, somit ist das System singulär, und hat unendlich
viele Lösungen. Um sie zu finden, kann man zum Beispiel den Gauss-Algorithmus
anwenden. Dazu ist es vorteilhaft, die letzte Zeile an die erste
Position zu nehmen:
\begin{align*}
\begin{tabular}{|>{$}c<{$}>{$}c<{$}>{$}c<{$}|>{$}c<{$}|}
\hline
a&b&c&1\\
\hline
1&1&1&1\\
4&3&2&6\\
2&3&4&0\\
\hline
\end{tabular}
&
\rightarrow
\begin{tabular}{|>{$}c<{$}>{$}c<{$}>{$}c<{$}|>{$}c<{$}|}
\hline
a&b&c&1\\
\hline
1&1&1&1\\
0&$-1$&$-2$&2\\
0&1&2&$-2$\\
\hline
\end{tabular}
\\
&
\rightarrow
\begin{tabular}{|>{$}c<{$}>{$}c<{$}>{$}c<{$}|>{$}c<{$}|}
\hline
a&b&c&1\\
\hline
1&1&1&1\\
0&1&2&$-2$\\
0&0&0&0\\
\hline
\end{tabular}
\\
&
\rightarrow
\begin{tabular}{|>{$}c<{$}>{$}c<{$}>{$}c<{$}|>{$}c<{$}|}
\hline
a&b&c&1\\
\hline
1&0&$-1$&3\\
0&1&2&$-2$\\
0&0&0&0\\
\hline
\end{tabular}
\end{align*}
Man kann also die Unbekannte $c$ frei wählen, und damit die Unbekannten
$a$ und $b$ ausdrücken:
\[
a=3+c,\qquad b=-2-2c
\]
oder als Parameterdarstellung einer Geraden:
\[
\mathbb L =\left\{\left.
\begin{pmatrix}
3\\-2\\0
\end{pmatrix}
+c\begin{pmatrix}
1\\-2\\1
\end{pmatrix}\right | c\in \mathbb R
\right\}
\qedhere
\]
\end{loesung}

