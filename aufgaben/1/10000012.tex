Das Gleichungssystems $Ax=b$ mit
\[
A=\begin{pmatrix}
-1& 5&-3& 3& 2\\
-1& 5&-3& 4& 3\\
-2&-2& 2& 1& 3\\
-3& 3&-1&-4&-3
\end{pmatrix}
%\qquad\text{und}\qquad
%b=\begin{pmatrix}
%6\\
%8\\
%2\\
%-8
%\end{pmatrix}
\]
hat den Vektor
\[
\begin{pmatrix}1\\1\\1\\1\\1\end{pmatrix}
\]
als L"osung.
Finden Sie die L"osungsmenge.

\begin{loesung}
Der gegebene Vektor ist partikul"are L"osung des inhomogenen Systems
$Ax=b$, es ist also nur noch die L"osungsmenge des homogenen Systems
$Ax=0$ zu bestimmen.
Dazu brauchen wir die rechte Seite gar nicht zu betrachten,
und finden mit der {\tt rref}-Funktion von Octave folgendes Gauss-Tableau:
\begin{center}
\begin{tabular}{|ccccc|}
\hline
1&0&$-\frac13$&0&$-\frac23$\\
0&1&$-\frac23$&0&$-\frac13$\\
0&0&         0&1&1\\
0&0&         0&0&0\\
\hline
\end{tabular}
\end{center}
Die frei w"ahlbaren Variablen sind also die dritte und die letzte,
wir wollen sie $\lambda$ und $\mu$ nennen. Dann ist die L"osungsmenge
des homogenen Systems
\[
{\mathbb L}_h
=
\left\{
\left.
\begin{pmatrix}
\frac13\lambda+\frac23\mu\\
\frac23\lambda+\frac13\mu\\
\lambda\\
-\mu\\
\mu
\end{pmatrix}
\;
\right|
\;
\lambda,\mu\in\mathbb R
\right\}
\]
und die L"osungsmenge des urspr"unglichen Systems
\[
{\mathbb L}
=
\left\{
\left.
\begin{pmatrix}
1+\frac13\lambda+\frac23\mu\\
1+\frac23\lambda+\frac13\mu\\
1+\lambda\\
1-\mu\\
1+\mu
\end{pmatrix}
\;
\right|
\;
\lambda,\mu\in\mathbb R
\right\}.
\qedhere
\]
\end{loesung}
