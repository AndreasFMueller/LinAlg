Das folgende Gleichungssystem ist unterbestimmt, beschreiben
Sie die Lösungsmenge:
\[
\begin{linsys}{3}
 x & + & 2y & - & 4z & = & 7 \\
3x & - & 5y & + & 6z & = & 8
\end{linsys}
\]

\thema{Lösungsmenge}
\thema{rref}

\begin{loesung}
Mit dem Gauss-Algorithmus bekommt man
\begin{align*}
\begin{tabular}{|>{$}c<{$}>{$}c<{$}>{$}c<{$}|>{$}c<{$}|}
\hline
1&2&-4&7\\
3&-5&6&8\\
\hline
\end{tabular}
&\rightarrow
\begin{tabular}{|>{$}c<{$}>{$}c<{$}>{$}c<{$}|>{$}c<{$}|}
\hline
1&2&-4&7\\
0&-11&18&-13\\
\hline
\end{tabular}
\\
&\rightarrow
\begin{tabular}{|>{$}c<{$}>{$}c<{$}>{$}c<{$}|>{$}c<{$}|}
\hline
1&2&-4&7\\
0&1&-\frac{18}{11}&\frac{13}{11}\\
\hline
\end{tabular}
\\
&\rightarrow
\begin{tabular}{|>{$}c<{$}>{$}c<{$}>{$}c<{$}|>{$}c<{$}|}
\hline
1&0&-\frac{8}{11}&\frac{51}{11}\\
0&1&-\frac{18}{11}&\frac{13}{11}\\
\hline
\end{tabular}
\end{align*}
Die Tableaux reichen nicht aus, um $z$ zu bestimmen, diese Variable
ist also frei wählbar, die anderen Variablen müssen daraus
bestimmt werden.
Lösungen sind also die Vektoren
\[
{\mathbb L}
=
\left\{
\left.
\begin{pmatrix}
\frac{51}{11}\\
\frac{13}{11}\\
0
\end{pmatrix}
+\begin{pmatrix}
\frac{8}{11}\\
\frac{18}{11}\\
1
\end{pmatrix}
z\;
\right|
\,z\in\mathbb R
\right\}.
\]
Man kann diese Lösungsmenge auch als
\[
\mathbb L=\left\{(x,y,z)\left| x=\frac{51}{11}+\frac{8}{11}z, y=\frac{13}{11}+\frac{18}{11}z,\,z\in\mathbb R\right.\right\}
\]
schreiben.
\end{loesung}
