Betrachten Sie die Matrix
\[
A=\begin{pmatrix}
2&  4&   2\\
1&c+1&3c-2\\
1&  4& c+5
\end{pmatrix}.
\]
\begin{teilaufgaben}
\item F"ur welche Werte von $c$ ist die Matrix $A$ regul"ar?
\item L"osen Sie f"ur jeden der in Teilaufgabe a) gefundene Werte von $c$ 
das Gleichungssystem $Ax=b$ mit
\[
b=\begin{pmatrix}2\\2\\3\end{pmatrix}.
\]
\end{teilaufgaben}

\begin{loesung}
\begin{teilaufgaben}
\item
Der Gaussalgorithmus kann dazu verwendet werden, die Regularit"at von $A$
zu ermitteln:
\begin{align*}
\begin{tabular}{|>{$}c<{$}>{$}c<{$}>{$}c<{$}|}
\hline
2&  4&   2\\
1&c+1&3c-2\\
1&  4& c+5\\
\hline
\end{tabular}
&\rightarrow
\begin{tabular}{|>{$}c<{$}>{$}c<{$}>{$}c<{$}|}
\hline
1&  2&   1\\
0&c-1&3c-3\\
0&  2& c+4\\
\hline
\end{tabular}
\rightarrow
\begin{tabular}{|>{$}c<{$}>{$}c<{$}>{$}c<{$}|}
\hline
1&  2&   1\\
0&  1&   3\\
0&  0& c-2\\
\hline
\end{tabular}.
\end{align*}
Im zweiten Schritt ist das Pivot-Element $c-1$, der zweite Schritt
kann also nur durchgef"uhrt werden, wenn $c\ne 1$ ist.
Analog erfordert der n"achst Gauss-Schritt, dass $c\ne 2$ ist.
Die Werte $c\in\{1,2\}$ sind also die gesuchten Werte, f"ur die die Matrix
nicht regul"ar ist.
\item
Da wir wiederholt ein Gleichungssystem mit der nicht
regul"aren Matrix $A$ l"osen m"ussen, ist die LR-Zerlegung
n"utzlich, damit wir die Arbeit des Gauss-Algorithmus nicht immer wieder
neu durchf"uhren m"ussen.
Wir k"onnen die LR-Zerlegung der Matrix $A$ aus der Rechnung von Teilaufgabe a) 
ablesen:
\[
A=
\underbrace{
\begin{pmatrix}
2&  0&  0\\
1&c-1&  0\\
1&  2&c-2
\end{pmatrix}}_{\displaystyle =L}
\underbrace{
\begin{pmatrix}
1&2&1\\
0&1&3\\
0&0&1
\end{pmatrix}}_{\displaystyle =R}
.
\]
Die Matrix $R$ ist regul"ar.
Damit das Gleichungssystem $Ax=b$ gel"ost werden kann, muss zun"achst mal
das Gleichungssystem $Ly=b$ gel"ost werden, danach kann man $x$ in einem
zweiten Schritt durch L"osung des Gleichungssystems $Rx=y$ bekommen.

F"ur $c\in\{1,2\}$ ist die Matrix $L$ nicht regul"ar, das Gleichungssystem
muss mit dem Gauss-Algorithmus gel"ost werden.
Wir betrachten jeden Fall einzeln:

F"ur $c=1$ erhalten wir die Rechnung
\[
\begin{tabular}{|>{$}c<{$}>{$}c<{$}>{$}c<{$}|>{$}c<{$}|}
\hline
2&  0&  0&2\\
1&c-1&  0&2\\
1&  2&c-2&3\\
\hline
\end{tabular}
=
\begin{tabular}{|>{$}c<{$}>{$}c<{$}>{$}c<{$}|>{$}c<{$}|}
\hline
2&  0&  0&2\\
1&  0&  0&2\\
1&  2& -1&3\\
\hline
\end{tabular}.
\]
Da die ersten beiden Gleichungen widersprechen sich, das Gleichungssystem
ist daher gar nicht l"osbar.

F"ur $c=2$ erhalten wir die Rechnung
\[
\begin{tabular}{|>{$}c<{$}>{$}c<{$}>{$}c<{$}|>{$}c<{$}|}
\hline
2&  0&  0&2\\
1&c-1&  0&2\\
1&  2&c-2&3\\
\hline
\end{tabular}
=
\begin{tabular}{|>{$}c<{$}>{$}c<{$}>{$}c<{$}|>{$}c<{$}|}
\hline
2&  0&  0&2\\
1&  1&  0&2\\
1&  2&  0&3\\
\hline
\end{tabular}
\rightarrow
\begin{tabular}{|>{$}c<{$}>{$}c<{$}>{$}c<{$}|>{$}c<{$}|}
\hline
1&  0&  0&1\\
0&  1&  0&1\\
0&  2&  0&2\\
\hline
\end{tabular}
\rightarrow
\begin{tabular}{|>{$}c<{$}>{$}c<{$}>{$}c<{$}|>{$}c<{$}|}
\hline
1&  0&  0&1\\
0&  1&  0&1\\
0&  0&  0&0\\
\hline
\end{tabular}.
\]
Daraus kann man die allgemeine L"osung
\[
y=\begin{pmatrix} 1\\1\\z \end{pmatrix}
\]
ablesen.
Jetzt muss man noch $Rx=y$ l"osen, das kann man zum Beispiel mit der
Inversen von $R$ machen.
Man kann sie ebenfalls mit dem Gauss-Algorithmus finden, der hier
besonders einfach ist, weil die Vorw"artsreduktion ja schon gemacht
ist, man muss nur noch R"uckw"artseinsetzen:
\[
\begin{tabular}{|>{$}c<{$}>{$}c<{$}>{$}c<{$}|>{$}c<{$}>{$}c<{$}>{$}c<{$}|}
\hline
1&2&1&1&0&0\\
0&1&3&0&1&0\\
0&0&1&0&0&1\\
\hline
\end{tabular}
\rightarrow
\begin{tabular}{|>{$}c<{$}>{$}c<{$}>{$}c<{$}|>{$}c<{$}>{$}c<{$}>{$}c<{$}|}
\hline
1&2&0&1&0&-1\\
0&1&0&0&1&-3\\
0&0&1&0&0& 1\\
\hline
\end{tabular}
\rightarrow
\begin{tabular}{|>{$}c<{$}>{$}c<{$}>{$}c<{$}|>{$}c<{$}>{$}c<{$}>{$}c<{$}|}
\hline
1&0&0&1&-2& 5\\
0&1&0&0& 1&-3\\
0&0&1&0& 0& 1\\
\hline
\end{tabular}.
\]
Die allgemeine L"osung $x$ finden wir jetzt durch Matrixmultiplikation
\[
R^{-1}y
=
\begin{pmatrix}
1&-2& 5\\
0& 1&-3\\
0& 0& 1\\
\end{pmatrix}
\begin{pmatrix} 1\\1\\z \end{pmatrix}
=
\begin{pmatrix} -1+5z\\1-3z\\z \end{pmatrix}.
\]
Zur Kontrolle rechnen wir $Ax$ nach f"ur $c=2$:
\[
Ax
=
\begin{pmatrix}
2&  4&   2\\
1&c+1&3c-2\\
1&  4& c+5
\end{pmatrix}
\begin{pmatrix} -1+5z\\1-3z\\z \end{pmatrix}
=
\begin{pmatrix}
2&4&2\\
1&3&4\\
1&4&7
\end{pmatrix}
\begin{pmatrix} -1+5z\\1-3z\\z \end{pmatrix}.
\]

Statt die LR-Zerlegung zu bem"uhen kann man in den beiden F"allen das
Gleichungssystem nat"urlich auch direkt l"osen.
\begin{align*}
c&=1:&
\begin{tabular}{|>{$}c<{$}>{$}c<{$}>{$}c<{$}|>{$}c<{$}|}
\hline
2&  4&   2&2\\
1&c+1&3c-2&2\\
1&  4& c+5&3\\
\hline
\end{tabular}
&=
\begin{tabular}{|>{$}c<{$}>{$}c<{$}>{$}c<{$}|>{$}c<{$}|}
\hline
2&  4&   2&2\\
1&  2&   1&2\\
1&  4&   6&3\\
\hline
\end{tabular}
\rightarrow
\begin{tabular}{|>{$}c<{$}>{$}c<{$}>{$}c<{$}|>{$}c<{$}|}
\hline
1&  2&   1&1\\
0&  0&   0&\color{red}1\\
0&  2&   5&2\\
\hline
\end{tabular},
\\
c&=2:&
\begin{tabular}{|>{$}c<{$}>{$}c<{$}>{$}c<{$}|>{$}c<{$}|}
\hline
2&  4&   2&2\\
1&c+1&3c-2&2\\
1&  4& c+5&3\\
\hline
\end{tabular}
&=
\begin{tabular}{|>{$}c<{$}>{$}c<{$}>{$}c<{$}|>{$}c<{$}|}
\hline
2&  4&   2&2\\
1&  3&   4&2\\
1&  4&   7&3\\
\hline
\end{tabular}
\rightarrow
\begin{tabular}{|>{$}c<{$}>{$}c<{$}>{$}c<{$}|>{$}c<{$}|}
\hline
1&  2&   1&1\\
0&  1&   3&1\\
0&  2&   6&2\\
\hline
\end{tabular}
\rightarrow
\begin{tabular}{|>{$}c<{$}>{$}c<{$}>{$}c<{$}|>{$}c<{$}|}
\hline
1&  0&  -5&-1\\
0&  1&   3& 1\\
0&  0&   0& 0\\
\hline
\end{tabular}.
\end{align*}
Im Fall $c=1$ zeigt die {\color{red}rote} 1, dass das Gleichungssystem
keine L"osung hat, im Fall $c=2$ kann man f"ur die allgemeine L"osung
\[
x=
\begin{pmatrix}
-1+5z\\
 1-3z\\
    z
\end{pmatrix}
\]
ablesen.
\qedhere
\end{teilaufgaben}
\end{loesung}


