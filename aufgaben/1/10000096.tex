Berechnen Sie das Matrizenprodukt
\[
\begin{pmatrix} 1 & 0 \\ 3 & 1 \end{pmatrix}
\begin{pmatrix} 1 & 0 \\ 2 & 1 \end{pmatrix}
\]
ohne erneut die Multiplikation $\text{Zeile} \times \text{Spalte}$
auszuführen.

\begin{hinweis}
Verwenden Sie die Aufgabe \ref{10000094}.
\end{hinweis}

\begin{loesung}
Die Matrizen sind die Matrizen, die in Aufgabe \ref{10000094} a) vorgekommen
sind.
Schreiben wir 
\[
A = \begin{pmatrix} 1 & 2 \\ 0 & 1 \end{pmatrix}
\qquad\text{und}\qquad
B = \begin{pmatrix} 1 & B \\ 0 & 1 \end{pmatrix},
\]
dann wird verlangt, das Produkt $\transpose{B}\transpose{A}$ zu berechnen.
Es ist aber
\(
\transpose{B}\transpose{A}
=
\transpose{(AB)}
\),
weil bei der Transposition eines Produktes die Faktoren die Plätze tauschen.
Das Produkt $AB$ wurde in der Aufgabe \ref{10000094} berechnet.
Somit ist
\[
\begin{pmatrix} 1 & 0 \\ 3 & 1 \end{pmatrix}
\begin{pmatrix} 1 & 0 \\ 2 & 1 \end{pmatrix}
=
\transpose{B}\transpose{A}
=
\transpose{(AB)}
=
\transpose{
\begin{pmatrix} 1 & 6 \\ 0 & 1 \end{pmatrix}
}
=
\begin{pmatrix} 1 & 0 \\ 6 & 1 \end{pmatrix}.
\qedhere
\]
\end{loesung}
