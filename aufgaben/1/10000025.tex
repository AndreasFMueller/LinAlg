Der Rang der Matrix
\[
\begin{pmatrix}
1&2&3\\
4&5&6\\
7&8&9
\end{pmatrix}
\]
ist 2, es muss also eine lineare
Abhängigkeit zwischen den drei Zeilen der Matrix geben. Finden
Sie eine nichttriviale Linearkombination der drei Zeilen, die verschwindet.

\thema{Rang}
\thema{lineare Abhängigkeit}

\begin{loesung}
Die Koeffizienten einer solchen Linearkombination seien $\lambda_1$,
$\lambda_2$ und $\lambda_3$. Dann müssen die Gleichungen
\[
\begin{linsys}{3}
 \lambda_1&+&4\lambda_2&+&7\lambda_3&=&0\\
2\lambda_1&+&5\lambda_2&+&8\lambda_3&=&0\\
3\lambda_1&+&6\lambda_2&+&9\lambda_3&=&0
\end{linsys}
\]
erfüllt sein. Dieses Gleichungssystem kann mit dem Gaussalgorithmus
gelöst werden:
\begin{align*}
\begin{tabular}{|>{$}c<{$}>{$}c<{$}>{$}c<{$}|}
\hline
1&4&7\\
2&5&8\\
3&6&9\\
\hline
\end{tabular}
&
\rightarrow
\begin{tabular}{|>{$}c<{$}>{$}c<{$}>{$}c<{$}|}
\hline
1&4&7\\
0&-3&-6\\
0&-6&-12\\
\hline
\end{tabular}
\rightarrow
\begin{tabular}{|>{$}c<{$}>{$}c<{$}>{$}c<{$}|}
\hline
1&4&7\\
0&1&2\\
0&0&0\\
\hline
\end{tabular}
\rightarrow
\begin{tabular}{|>{$}c<{$}>{$}c<{$}>{$}c<{$}|}
\hline
1&0&-1\\
0&1&2\\
0&0&0\\
\hline
\end{tabular}
\end{align*}
Die Variable $\lambda_3$ ist frei wählbar, wir setzen $\lambda_3=1$ und
finden die anderen Variablen
\[
\lambda_3=1\qquad \lambda_1=\lambda_3=1,\qquad \lambda_2=-2\lambda_3=-2.
\]
Tatsächlich ist die Summe der ersten und letzten Zeile das Doppelte
der mittleren Zeile.
\end{loesung}
