Berechnen Sie die Determinante und die inverse Matrix von
\[
A=
\begin{pmatrix}
1&1& a  \\
2&3&2a+1\\
3&5&3a+3
\end{pmatrix}.
\]

\thema{inverse Matrix}
\thema{Gauss-Algorithmus}

\begin{loesung}
Wir wenden den Gaussalgorithmus an
\begin{align*}
\begin{tabular}{|>{$}c<{$}>{$}c<{$}>{$}c<{$}|>{$}c<{$}>{$}c<{$}>{$}c<{$}|}
\hline
1&1& a  &1&0&0\\
2&3&2a+1&0&1&0\\
3&5&3a+3&0&0&1\\
\hline
\end{tabular}
&
\rightarrow
\begin{tabular}{|>{$}c<{$}>{$}c<{$}>{$}c<{$}|>{$}c<{$}>{$}c<{$}>{$}c<{$}|}
\hline
1&1& a  & 1& 0& 0\\
0&1&   1&-2& 1& 0\\
0&2&   3&-3& 0& 1\\
\hline
\end{tabular}
\rightarrow
\begin{tabular}{|>{$}c<{$}>{$}c<{$}>{$}c<{$}|>{$}c<{$}>{$}c<{$}>{$}c<{$}|}
\hline
1&1& a  & 1& 0& 0\\
0&1&   1&-2& 1& 0\\
0&0&   1& 1&-2& 1\\
\hline
\end{tabular}
\\
&
\rightarrow
\begin{tabular}{|>{$}c<{$}>{$}c<{$}>{$}c<{$}|>{$}c<{$}>{$}c<{$}>{$}c<{$}|}
\hline
1&1&   0& 1-a& 2a&-a\\
0&1&   0&-3  & 3 &-1\\
0&0&   1& 1  &-2 & 1\\
\hline
\end{tabular}
\\
&
\rightarrow
\begin{tabular}{|>{$}c<{$}>{$}c<{$}>{$}c<{$}|>{$}c<{$}>{$}c<{$}>{$}c<{$}|}
\hline
1&0&   0& 4-a& 2a-3&-a+1\\
0&1&   0&-3  & 3 &-1\\
0&0&   1& 1  &-2 & 1\\
\hline
\end{tabular}
\\
&
\Rightarrow
\quad
A^{-1}
=
\begin{pmatrix}
 4-a&2a-3& 1-a\\
-3  &   3&-1  \\
 1  &  -2& 1
\end{pmatrix}.
\end{align*}
Die Pivot-Elemente waren alle $1$, daher ist $\det(A)=1$.

Wir kontrollieren das Resultat durch Ausmultiplizieren des Produktes
\begin{align*}
AA^{-1}
&=
\begin{pmatrix}
1&1& a  \\
2&3&2a+1\\
3&5&3a+3
\end{pmatrix}
\begin{pmatrix}
 4-a&2a-3& 1-a\\
-3  &   3&-1  \\
 1  &  -2& 1
\end{pmatrix}
\\
&=
\begin{pmatrix}
4-a-3+a       & 2a-3+3-2a    & 1-a-1+a     \\
8-2a-9+2a+1   & 4a-6+9-4a-2  & 2-2a-3+2a+1 \\
12-3a-15+3a+3 & 6a-9+15-6a-6 & 3-3a-5+3a+3
\end{pmatrix}
\\
&=
\begin{pmatrix}
1&0&0\\
0&1&0\\
0&0&1
\end{pmatrix}.
\end{align*}
Die Determinante hätte man natürlich auch mit der Sarrusformel berechnen
können:
\begin{align*}
\det(A)
&=
1\cdot 3\cdot (3a+3)
+
1\cdot (2a+1)\cdot 3
+
a\cdot2\cdot 5
-
3\cdot 3\cdot a
-
5\cdot (2a+1)\cdot 1
-
(3a+3)\cdot 2\cdot 1
\\
&=
9a+9 + 6a+3 + 10a -9a - 10a-5 -6a-6=1.
\qedhere
\end{align*}
\end{loesung}

\begin{bewertung}
Determinante als Produkt der Pivots ({\bf P}) 1 Punkt,
alternativ: Sarrusformel ({\bf S}) 1 Punkt,
Wert der Determinante ({\bf D}) 1 Punkt,
inverse Matrix mit Hilfe des $3\times 6$-Tableau und Gaussalgorithmus
({\bf I}),
Gauss-Algorithmus: Vorwartsreduktion mit roten ({\bf R}) und blauen ({\bf B})
Operationen je ein Punkt,
Rückwärseinsetzen ({\bf E}) 1 Punkt.
\end{bewertung}


