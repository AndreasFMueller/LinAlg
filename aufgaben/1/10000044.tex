Für welche ganzzahligen Werte der beiden Parameter $c$ und $u$ hat
das Gleichungssystem
\[
\begin{linsys}{3}
2x&+& 8y&+&2cz&=&2u\\
  & & 3y&+&12z&=&6\\
cx&+&2cy&-&15z&=&5
\end{linsys}
\]
%\[
%A=\begin{pmatrix}
%2& 8 &  2c\\
%0& 3 & 12 \\
%c& 2c&-15
%\end{pmatrix}
%\]
unendlich viele Lösungen?
Bestimmen Sie die Lösungsmenge für all diese Fälle.

\thema{Matrix mit Parameter}
\thema{Gauss-Algorithmus}

\begin{loesung}
Man könnte versuchen herauszufinden, ob die Matrix
\[
A=\begin{pmatrix}
2& 8 &  2c\\
0& 3 & 12 \\
c& 2c&-15
\end{pmatrix}
\]
regulär ist, und dazu die Determinante verwenden.
Da man das Gleichungssytem aber auch lösen muss, ist effizienter,
gleich von Anfang an den Gauss-Algorithmus anzuwenden.
\begin{align*}
\begin{tabular}{|>{$}c<{$}>{$}c<{$}>{$}c<{$}|>{$}c<{$}|}
\hline
 2&8 &  2c&2u\\
 0&3 & 12 & 6\\
 c&2c&-15 & 5\\
\hline
\end{tabular}
&
\rightarrow
\begin{tabular}{|>{$}c<{$}>{$}c<{$}>{$}c<{$}|>{$}c<{$}|}
\hline
 1&4 &  c     & u\\
 0&3 & 12     & 6\\
 0&-2c&-15-c^2&5-uc\\
\hline
\end{tabular}
\rightarrow
\begin{tabular}{|>{$}c<{$}>{$}c<{$}>{$}c<{$}|>{$}c<{$}|}
\hline
 1&4 &  c           & u\\
 0&1 &  4           & 2\\
 0&0 &-15-c^2-4(-2c)& 5-uc+4c\\
\hline
\end{tabular}
\end{align*}
Unendlich viele Lösungen sind möglich, wenn einerseits im
Gauss-Algorithmus Nullzeilen auftreten, und andereseits die zugehörigen
Einträge der rechten Seite verschwinden.
Eine Nullzeile kann genau dann auftreten, wenn
\begin{align*}
-15-c^2+8c&=0\\
c^2-8c+15&=0\\
(c-3)(c-5)&=0
\end{align*}
Für $c$ kommen also nur die Werte $3$ und $5$ in Frage.

Damit auch die rechte Seite verschwindet, muss
\[
5-uc+4c=5-(u-4)c=0
\]
sein.
Löst man nach $u$ auf, findet man
\[
u=4+\frac{5}{c}
\qquad\Rightarrow\qquad
u=\begin{cases}
5&\qquad\text{für $c=5$}\\
\frac{17}{3}&\qquad\text{für $c=3$}
\end{cases}
\]
Da ganzzahlige Parameterwerte verlangt wurden, kommt nur $c=5$ und $u=5$
in Frage.

Um nun auch noch die Lösungsmenge zu bestimmen, werden die gefundenen 
Parameter $c=5$ und $u=5$ im Gauss-Tableau eingesetzt und der Gauss-Algorithmus
zu Ende geführt.
\begin{align*}
\begin{tabular}{|>{$}c<{$}>{$}c<{$}>{$}c<{$}|>{$}c<{$}|}
\hline
 1&4 & 5 & 5\\
 0&1 & 4 & 2\\
 0&0 & 0 & 0\\
\hline
\end{tabular}
\rightarrow
\begin{tabular}{|>{$}c<{$}>{$}c<{$}>{$}c<{$}|>{$}c<{$}|}
\hline
 1&0 & -11 & -3\\
 0&1 & 4 & 2\\
 0&0 & 0 & 0\\
\hline
\end{tabular}
\end{align*}
Daraus kann jetzt die Lösungsmenge abgelesen werden:
\[
\mathbb L =\left\{\left.
\begin{pmatrix}
-3\\
2\\
0
\end{pmatrix}
+z
\begin{pmatrix}
11\\
 -4\\
 1
\end{pmatrix}
\;
\right|
z\in \mathbb R
\right\}.
\]
\end{loesung}

\begin{bewertung}
Durchführung des Gauss-Algorithmus ({\bf G}) 2 Punkte,
Bedingung für unendlich viele Lösungen ({\bf B}) 1 Punkt,
Bestimmung der zwei möglichen Werte für $c$ ({\bf C}) 1 Punkt,
Bestimmung von $u$ ({\bf U}) 1 Punkt.
Bestimmung der Lösungsmenge ({\bf L}) 1 Punkt,
\end{bewertung}


