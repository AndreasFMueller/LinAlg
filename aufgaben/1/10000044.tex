F"ur welche ganzzahligen Werte der beiden Parameter von $c$ und $u$ hat
das Gleichungssytem
\[
\begin{linsys}{3}
2x&+& 8y&+&2cz&=&2u\\
  & & 3y&+&12z&=&6\\
cx&+&2cy&-&15z&=&5
\end{linsys}
\]
%\[
%A=\begin{pmatrix}
%2& 8 &  2c\\
%0& 3 & 12 \\
%c& 2c&-15
%\end{pmatrix}
%\]
unendlich viele L"osungen?

\begin{loesung}
Man k"onnte versuchen herauszufinden, ob die Matrix
\[
A=\begin{pmatrix}
2& 8 &  2c\\
0& 3 & 12 \\
c& 2c&-15
\end{pmatrix}
\]
regul"ar ist, und dazu die Determinante verwenden.
Da man das Gleichungssytem aber auch l"osen muss, ist effizienter,
gleich von Anfang an den Gauss-Algorithmus anzuwenden.
\begin{align*}
\begin{tabular}{|>{$}c<{$}>{$}c<{$}>{$}c<{$}|>{$}c<{$}|}
\hline
 2&8 &  2c&2u\\
 0&3 & 12 & 6\\
 c&2c&-15 & 5\\
\hline
\end{tabular}
&
\rightarrow
\begin{tabular}{|>{$}c<{$}>{$}c<{$}>{$}c<{$}|>{$}c<{$}|}
\hline
 1&4 &  c     & u\\
 0&3 & 12     & 6\\
 0&-2c&-15-c^2&5-uc\\
\hline
\end{tabular}
\rightarrow
\begin{tabular}{|>{$}c<{$}>{$}c<{$}>{$}c<{$}|>{$}c<{$}|}
\hline
 1&4 &  c           & u\\
 0&1 &  4           & 2\\
 0&0 &-15-c^2-4(-2c)& 5-uc+4c\\
\hline
\end{tabular}
\end{align*}
Unendlich viele L"osungen sind m"oglich, wenn einerseits im
Gauss-Algorithmus Nullzeilen auftreten, und andereseits die zugeh"origen
Eintr"age der rechten Seite verschwinden.
Eine Nullzeile kann genau dann auftreten, wenn
\begin{align*}
-15-c^2+8c&=0\\
c^2-8c+15&=0\\
(c-3)(c-5)&=0
\end{align*}
F"ur $c$ kommen also nur die Werte $3$ und $5$ in Frage.

Damit auch die rechte Seite verschwindet, muss
\[
5-uc+4c=5-(u-4)c=0
\]
sein.
L"ost man nach $u$ auf, findet man
\[
u=4+\frac{5}{c}
\qquad\Rightarrow\qquad
u=\begin{cases}
5&\qquad\text{f"ur $c=5$}\\
\frac{17}{3}&\qquad\text{f"ur $c=3$}
\end{cases}
\]
Da ganzzahlige Parameterwerte verlangt wurden, kommt nur $c=5$ und $u=5$
in Frage.
\end{loesung}

\begin{bewertung}
\end{bewertung}


