Verwenden Sie den \texttt{rref}-Algorithmus oder -Befehl um folgendes
Gleichungssystem zu lösen.
\[
\begin{linsys}{3}
 x &+& 14y & &    &=&  6\\
6x &+& 42y &+& 4z &=& -2\\
5x &+& 35y &+&  z &=& 10
\end{linsys}
\qquad
\]
\begin{teilaufgaben}
\item
Wie viele Lösungen hat dieses Gleichungssystem?
\item
Geben Sie die Lösungsmenge in Vektorschreibweise an.
\end{teilaufgaben}

\begin{loesung}
\begin{teilaufgaben}
\item
Übergeben wir dem \texttt{rref}-Befehl des Taschenrechners das Gleichungssystem
als Gauss-Tableau oder rechnen wir von Hand, dann erhalten wir folgendes
Tableau als Resultat:
\begin{align*}
\begin{tabular}{|>{$}r<{$}>{$}r<{$}>{$}r<{$}|>{$}r<{$}|}
\hline
2&14&0& 6\\
6&42&4&-2\\
5&35&1&10\\
\hline
\end{tabular}
&
\rightarrow
\begin{tabular}{|>{$}r<{$}>{$}r<{$}>{$}r<{$}|>{$}r<{$}|}
\hline
1& 7& 0&  3\\
0& 0& 4&-20\\
0& 0& 1& -5\\
\hline
\end{tabular}
\rightarrow
\begin{tabular}{|>{$}r<{$}>{$}r<{$}>{$}r<{$}|>{$}r<{$}|}
\hline
1& 7& 0& 3\\
0& 0& 1&-5\\
0& 0& 0& 0\\
\hline
\end{tabular}
\end{align*}
Das resultierende Gauss-Tableau enthält eine Nullzeile, womit wir im
singulären Fall sind.
Da auf der rechten Seite $0$ steht, haben wir die Gleichung $0=0$,
welche immer erfüllt ist.
Folglich gibt es unendlich viele Lösungen. 
Die zweite Variable hat sich als frei wählbar herausgestellt.
\item
Die Lösungsmenge kann aus den ersten beiden
Zeilen abgelesen werden.
Die entsprechenden Gleichungen dazu sind
\[
x+7y=3
\qquad\text{und}\qquad
z=-5.
\]
Durch Auflösen nach der Variablen $x$ erhalten wir
\[
x = 3-7y
\qquad\text{und}\qquad
z=-5
\]
oder in Vektorschreibweise:
\[
\mathbb L =\left\{\left.
\begin{pmatrix}
3\\
0\\
-5
\end{pmatrix}
+y
\begin{pmatrix}
-7\\
 1\\
 0
\end{pmatrix}
\;
\right|
y\in \mathbb R
\right\}.
\qedhere
\]
\end{teilaufgaben}
\end{loesung}

