Verwenden sie den \texttt{rref}-Befehl um folgendes Gleichungssystem zu lösen.
\[
\begin{linsys}{3}
7x&+&3y&+&2z&=&2\\
x&+&y&+&z&=&7\\
13x&+&5y&+&3z&=&-3
\end{linsys}
\qquad
\]
\begin{teilaufgaben}
\item
Wie viele Lösungen hat dieses Gleichungssystem?
\item
Geben Sie die Lösungsmenge in Vektorschreibweise an.
\end{teilaufgaben}
\begin{loesung}
\begin{teilaufgaben}
\item
Übergeben wir dem \texttt{rref}-Befehl des Taschenrechners das Gleichungssystem
als Gauss-Tableau, erhalten wir folgendes Tableau als Resultat.
\begin{align*}
\begin{tabular}{|ccc|c|}
\hline
7&3&2&2\\
1&1&1&7\\
13&5&3&-3\\
\hline
\end{tabular}
&
\rightarrow
\begin{tabular}{|ccc|c|}
\hline
1&0&-$\frac{1}{4}$&-$\frac{19}{4}$\\
0&1&$\frac{5}{4}$&$\frac{47}{4}$\\
0&0&0&0\\
\hline
\end{tabular}
\end{align*}
Das resultierende Gauss-Tableau enthält eine Nullzeile, womit wir im singulären Fall sind.
Da auf der rechten Seite 0 steht haben wir die Gleichung $0=0$, welche immer erfüllt ist.
Folglich gibt es unendlich viele Lösungen. 
\item
Die Lösungsmenge kann aus den ersten beiden
Zeilen abgelesen werden. Die entsprechenden Gleichungen dazu sind
\[
x-\frac{1}{4}z = -\frac{19}{4}\qquad\text{und}\qquad y+\frac{5}{4}z = \frac{47}{4}.
\]
Durch Isolieren der Variablen x bzw. y erhalten wir
\[
x = -\frac{19}{4} +\frac{1}{4}z\qquad\text{und}\qquad y = \frac{47}{4} -\frac{5}{4}z
\]
oder in Vektorschreibweise:
\[
\mathbb L =\left\{\left.
 \begin{pmatrix}
-\frac{19}{4}\\\frac{47}{4}\\0
\end{pmatrix}
+z\begin{pmatrix}
\frac{1}{4}\\-\frac{5}{4}\\1
\end{pmatrix}\right | z\in \mathbb R
\right\}
\qedhere
\]
\end{teilaufgaben}
\end{loesung}

