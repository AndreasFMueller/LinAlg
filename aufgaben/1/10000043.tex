Gegeben sind die Matrizen
\[
A
=
\begin{pmatrix}
 4& 9& 8\\
-1&-2&-1\\
-2&-5&-5
\end{pmatrix}
\qquad\text{und}\qquad
P
=
\begin{pmatrix}
-7&-16&-8\\
 1&  3& 1\\
 5& 10& 6
\end{pmatrix}
\]
\begin{teilaufgaben}
\item Berechnen Sie $P^2$.
\item Berechnen Sie die Inverse $A^{-1}$ von $A$.
\item Berechnen Sie $P_0=A^{-1}PA$.
\item Warum erklärt c) das Resultat von a)
\end{teilaufgaben}

\thema{inverse Matrix}
\thema{Matrizenprodukt}

\begin{loesung}
\begin{teilaufgaben}
\item Das Matrizenprodukt ist
\begin{align*}
P^2
&=
\begin{pmatrix}
-7&-16&-8\\
 1&  3& 1\\
 5& 10& 6
\end{pmatrix}
\begin{pmatrix}
-7&-16&-8\\
 1&  3& 1\\
 5& 10& 6
\end{pmatrix}
\\
&=
{
\small
\begin{pmatrix}
(-7)\cdot(-7)+(-16)\cdot1+(-8)\cdot 5
	&(-7)\cdot(-16) + (-16)\cdot 3 + (-8)\cdot 10
		&(-7)\cdot(-8) + (-16)\cdot 1 + (-8)\cdot 6\\
1\cdot(-7) + 3\cdot 1 + 1\cdot 5
	&1\cdot(-16) + 3\cdot 3 + 1\cdot10
		&1\cdot(-8) + 3\cdot 1 + 1\cdot 8\\
5\cdot(-7) + 10\cdot 1+ 6\cdot 5
	&5\cdot(-16) + 10\cdot 3 + 6\cdot 10
		&5\cdot(-8) + 10\cdot + + 6\cdot 6\\
\end{pmatrix}
}
\\
&=
\begin{pmatrix}
-7&-16&-8\\
 1&  3& 1\\
 5& 10& 6
\end{pmatrix}
=P.
\end{align*}
\item Die Inverse von $A$ kann mit dem Gauss-Algorithmus berechnet werden:
\begin{align*}
\begin{tabular}{|>{$}c<{$}>{$}c<{$}>{$}c<{$}|>{$}c<{$}>{$}c<{$}>{$}c<{$}|}
\hline
 4& 9& 8& 1& 0& 0\\
-1&-2&-1& 0& 1& 0\\
-2&-5&-5& 0& 0& 1\\
\hline
\end{tabular}
&
\rightarrow
\begin{tabular}{|>{$}c<{$}>{$}c<{$}>{$}c<{$}|>{$}c<{$}>{$}c<{$}>{$}c<{$}|}
\hline
 1& \frac94& 2& \frac14& 0& 0\\
 0& \frac14& 1& \frac14& 1& 0\\
 0&-\frac12&-1& \frac12& 0& 1\\
\hline
\end{tabular}
\rightarrow
\begin{tabular}{|>{$}c<{$}>{$}c<{$}>{$}c<{$}|>{$}c<{$}>{$}c<{$}>{$}c<{$}|}
\hline
 1& \frac94& 2& \frac14& 0& 0\\
 0&      1 & 4&      1 & 4& 0\\
 0&      0 & 1&      1 & 2& 1\\
\hline
\end{tabular}
\\
&
\rightarrow
\begin{tabular}{|>{$}c<{$}>{$}c<{$}>{$}c<{$}|>{$}c<{$}>{$}c<{$}>{$}c<{$}|}
\hline
 1& \frac94& 0&-\frac74&-4&-2\\
 0&      1 & 0&     -3 &-4&-4\\
 0&      0 & 1&      1 & 2& 1\\
\hline
\end{tabular}
\rightarrow
\begin{tabular}{|>{$}c<{$}>{$}c<{$}>{$}c<{$}|>{$}c<{$}>{$}c<{$}>{$}c<{$}|}
\hline
 1&      0 & 0&      5 & 5& 7\\
 0&      1 & 0&     -3 &-4&-4\\
 0&      0 & 1&      1 & 2& 1\\
\hline
\end{tabular}
\end{align*}
Daraus liest man ab
\[
A^{-1}
=
\begin{pmatrix}
 5& 5& 7\\
-3&-4&-4\\
 1& 2& 1
\end{pmatrix}
\]
\item
Das Produkt ist
\begin{align*}
A^{-1}PA
&=
\begin{pmatrix}
 5& 5& 7\\
-3&-4&-4\\
 1& 2& 1
\end{pmatrix}
\begin{pmatrix}
-7&-16&-8\\
 1&  3& 1\\
 5& 10& 6
\end{pmatrix}
\begin{pmatrix}
 4& 9& 8\\
-1&-2&-1\\
-2&-5&-5
\end{pmatrix}
\\
&=
\begin{pmatrix}
 5& 5& 7\\
-3&-4&-4\\
 0& 0& 0
\end{pmatrix}
\begin{pmatrix}
 4& 9& 8\\
-1&-2&-1\\
-2&-5&-5
\end{pmatrix}
=
\begin{pmatrix}
1&0&0\\
0&1&0\\
0&0&0
\end{pmatrix}
=P_0,
\end{align*}
\item
$P_0$ ist ganz wegen
$P_0^2=P_0$
offensichtlich eine Projektionsmatrix.
Es folgt
\[
P=AP_0A^{-1}
\quad\Rightarrow\quad
P^2
=
AP_0A^{-1} AP_0A^{-1}
=
AP_0\underbrace{(A^{-1} A)}_{\displaystyle =E}P_0A^{-1}
=
AP_0^2A^{-1}
=
AP_0A^{-1}
=
P,
\]
was das Resultat von a) erklärt.
\qedhere
\end{teilaufgaben}
\end{loesung}

