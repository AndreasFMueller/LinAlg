Betrachten Sie das folgende Gleichungssystem für $x$, $y$ und $z$, 
welches von einem Parameter $t$ abhängt.

\[
\begin{linsys}{3}
(2t-1)x&-&(t-1)y&+& z&=& 1\\
    2tx&-&(t-1)y&+&2z&=& 1\\
(2t+4)x&-&(t+2)y&+&2z&=&-1
\end{linsys}
\]
\begin{teilaufgaben}
\item
Für welche Werte des Parameters $t$ hat das Gleichungssystem
keine, genau eine oder unendlich viele Lösungen?
\item
Lässt man $t$ grösser werden, verändert sich auch die Lösung
des Gleichungssystems. Gegen welchen Vektor strebt die Lösung
des Gleichungssystem im Grenzfall $t\to\infty$?
\end{teilaufgaben}

\begin{loesung}
\begin{teilaufgaben}
\item
Das Gauss-Tableau dieses Gleichungssystems ist
\begin{align*}
\begin{tabular}{|>{$}c<{$}>{$}c<{$}>{$}c<{$}|>{$}c<{$}|}
\hline
2t-1&-t+1&1& 1\\
  2t&-t+1&2& 1\\
2t+4&-t-2&2&-1\\
\hline
\end{tabular}
&\rightarrow
\begin{tabular}{|>{$}c<{$}>{$}c<{$}>{$}c<{$}|>{$}c<{$}|}
\hline
  -1&   0&-1& 0\\
  2t&-t+1& 2& 1\\
   4&  -3& 0&-2\\
\hline
\end{tabular}
\\
&\rightarrow
\begin{tabular}{|>{$}c<{$}>{$}c<{$}>{$}c<{$}|>{$}c<{$}|}
\hline
   1&   0&    1&   0\\
   0&-t+1&-2t+2&   1\\
   0&  -3&   -4&  -2\\
\hline
\end{tabular}
\\
&\rightarrow
\begin{tabular}{|>{$}c<{$}>{$}c<{$}>{$}c<{$}|>{$}c<{$}|}
\hline
   1&     0&      1&   0\\
   0&    -3&     -4&  -2\\
   0&-(t-1)&-2(t-1)&   1\\
\hline
\end{tabular}
\end{align*}
Im letzten Tableau kann man ablesen, dass das Gleichungssystem singulär
ist, wenn $t=1$, dann verschwinden nämlich die Klammerausdrücke.
Setzt man $t=1$ ein, wird das Tableau reduziert auf
\[
\begin{tabular}{|>{$}c<{$}>{$}c<{$}>{$}c<{$}|>{$}c<{$}|}
\hline
   1&   0&    1&   0\\
   0&  -3&   -4&  -2\\
   0&   0&    0&   1\\
\hline
\end{tabular}
\]
Da in der letzten Seite die rechten Seite nicht Null ist, hat das
Gleichungssystem keine Lösung.

\item
Die allgemeine Lösung des Gleichungssystems kann man für $t\ne 1$
ebenfalls ausrechnen:
\begin{align*}
\begin{tabular}{|>{$}c<{$}>{$}c<{$}>{$}c<{$}|>{$}c<{$}|}
\hline
   1&   0&    1&   0\\
   0&-t+1&-2t+2&   1\\
   0&  -3&   -4&  -2\\
\hline
\end{tabular}
&\rightarrow
\begin{tabular}{|>{$}c<{$}>{$}c<{$}>{$}c<{$}|>{$}c<{$}|}
\hline
   1&   0&    1&            0\\
   0&   1&    2&\frac{1}{1-t}\\
   0&  -3&   -4&           -2\\
\hline
\end{tabular}
\end{align*}
Die rationale Funktion auf der rechten Seite strebt für $t\to\infty$
gegen $0$, damit können wir weiterrechnen:
\begin{align*}
\begin{tabular}{|>{$}c<{$}>{$}c<{$}>{$}c<{$}|>{$}c<{$}|}
\hline
   1&   0&    1&            0\\
   0&   1&    2&            0\\
   0&  -3&   -4&           -2\\
\hline
\end{tabular}
&\rightarrow
\begin{tabular}{|>{$}c<{$}>{$}c<{$}>{$}c<{$}|>{$}c<{$}|}
\hline
   1&   0&    1&  0\\
   0&   1&    2&  0\\
   0&   0&    2& -2\\
\hline
\end{tabular}
\rightarrow
\begin{tabular}{|>{$}c<{$}>{$}c<{$}>{$}c<{$}|>{$}c<{$}|}
\hline
   1&   0&    0&     1\\
   0&   1&    0&     2\\
   0&   0&    1&    -1\\
\hline
\end{tabular}
\end{align*}
Für $t\to\infty$ strebt also $x\to 1$, $y\to 2$ und $z\to -1$.
\qedhere
\end{teilaufgaben}
\end{loesung}
