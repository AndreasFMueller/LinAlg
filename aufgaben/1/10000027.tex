Ein lineares Gleichungssystem hat die Koeffizientenmatrix
\[
A=\begin{pmatrix}
1&1&0\\
3&4&1\\
3&6&3
\end{pmatrix}
\]
\begin{teilaufgaben}
\item Ist $A$ regulär?
\item Welchen Rang hat $A$?
\item Bestimmen Sie die Lösungsmenge des homogenen Gleichungssystems mit
Koeffizientenmatrix $A$.
\item Der Vektor $x_p$ ist eine Lösung des inhomogenen Gleichungssystems mit
rechter Seite $b$:
\[
x_p=
\begin{pmatrix}
-5\\8\\0
\end{pmatrix}
,\qquad
b=\begin{pmatrix}
3\\17\\33
\end{pmatrix}.
\]
Bestimmen Sie die Lösungsmenge von $Ax=b$.
\end{teilaufgaben}

\thema{Rang}
\thema{homogenes Gleichungssystem}

\begin{loesung}
\begin{teilaufgaben}
\item Wir wenden den Gaussalgorithmus auf $A$ an:
\begin{align*}
\begin{tabular}{|>{$}c<{$}>{$}c<{$}>{$}c<{$}|}
\hline
1&1&0\\
3&4&1\\
3&6&3\\
\hline
\end{tabular}
&\rightarrow
\begin{tabular}{|>{$}c<{$}>{$}c<{$}>{$}c<{$}|}
\hline
1&1&0\\
0&1&1\\
0&3&3\\
\hline
\end{tabular}
\\
&\rightarrow
\begin{tabular}{|>{$}c<{$}>{$}c<{$}>{$}c<{$}|}
\hline
1&1&0\\
0&1&1\\
0&0&0\\
\hline
\end{tabular}
\\
&\rightarrow
\begin{tabular}{|>{$}c<{$}>{$}c<{$}>{$}c<{$}|}
\hline
1&0&-1\\
0&1& 1\\
0&0& 0\\
\hline
\end{tabular}
\end{align*}
Aus dem Schlusstableau kann man ableiten, dass $A$ singulär ist, die
Lösung eines Gleichungssystems mit $A$ als Koeffizientenmatrix ist
nicht regulär.
\item Das Schlusstableau zeigt auch, dass $\operatorname{Rang}(A)=2$.
\item Aus dem Schlusstableau kann man auch die Lösungsmenge ableiten
\[
\mathbb L_h=\left\{
\left.
x_3
\begin{pmatrix}
1\\-1\\1
\end{pmatrix}
\right|
x_3\in\mathbb R
\right\}.
\]
\item Die Lösungsmenge von $Ax=b$ ist
\[
\mathbb L= \{x_p+x_h\,|\,x_h\in\mathbb L_h\}
=
\left\{
\left.
\begin{pmatrix}
-5+x_3\\
8-x_3\\
x_3
\end{pmatrix}
\right|
x_3\in\mathbb R
\right\}
\qedhere
\]
\end{teilaufgaben}
\end{loesung}
