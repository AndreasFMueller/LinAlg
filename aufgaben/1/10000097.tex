Können die leeren Felder in
\[
\def\feld#1#2{\raisebox{-0.2cm}{\begin{tikzpicture}[>=latex,thick]
\draw[color=#2] (0,0) rectangle (0.8,0.7);
\node at (0.8,0.35) [left] {$#1$};
\end{tikzpicture}}}
\begin{pmatrix}
\feld{3}{white} & \feld{\phantom{-}7}{white} \\[2pt]
%\feld{\phantom{-}2}
\feld{}{gray}
& \feld{5}{white}
\end{pmatrix}
\begin{pmatrix*}[r]
\feld{ 5}{white} & \feld{-7}{white}
\\[2pt]
\feld{%-2
}{gray} & \feld{ 3}{white}
\end{pmatrix*}
\]
so mit Zahlen ausgefüllt werden, dass die beiden
Matrizen invers zueinander sind?

\begin{loesung}
Wir bezeichnen die beiden Matrizen mit $A$ und $B$ mit den Matrixelemente
$a_{ik}$ und $b_{ik}$.

Damit in der linken oberen Ecke des Produktes eine $1$ zu stehen kommt,
muss das Matrixelement $b_{21}$ die Gleichung
\[
3\cdot 5 + 7\cdot b_{11} = 1
\qquad
\Rightarrow
\qquad
7\cdot b_{11} = -14
\qquad
\Rightarrow
\qquad
b_{11} = -2
\]
erfüllen.

Um das Matrixelement $a_{11}$ zu bestimmen, berechnen wird das Produkt
der zweiten Zeile von $A$ mit der ersten Spalte von $B$, dieses Produkt
muss $0$ ergeben.
Ausmultiplizieren ergibt
\[
a_{11}\cdot 5 + 5\cdot b_{11}
=
a_{11}\cdot 5 + 5\cdot (-2)
=
5a_{11}-10=0
\qquad
\Rightarrow
\qquad
5a_{11}=10
\qquad
\Rightarrow
\qquad
a_{11}=2.
\]
Die beiden Matrizen sind daher
\[
\begin{pmatrix*}[r] 3 &  7 \\  2 & 5 \end{pmatrix*}
\begin{pmatrix*}[r] 5 & -7 \\ -2 & 3 \end{pmatrix*}
=
\begin{pmatrix}
1&0\\0&1
\end{pmatrix}.
\qedhere
\]
\end{loesung}

