Betrachten Sie das Gleichungssystem
\[
\begin{linsys}{3}
2x&&&-&z&=&6\\
&&2y&-&4z&=&-2\\
&&ay&+&z&=&0\\
\end{linsys}
\]
mit den Unbekannten $x$, $y$ und $z$. Für welche Werte von $a$
hat dieses Gleichungssystem keine, genau eine oder unendlich viele
Lösungen?

Geben Sie die Lösung in dem Falle an, wo sie eindeutig ist.

\thema{Matrix mit Parameter}
\thema{Gauss-Algorithmus}
\themaL{Losungsmenge}{Lösungsmenge}

\begin{loesung}
Wir führen den Gauss-Algorithmus durch
\begin{align*}
\begin{tabular}{|>{$}c<{$}>{$}c<{$}>{$}c<{$}|>{$}c<{$}|}
\hline
2&0&-1&6\\
0&2&-4&-2\\
0&a&1&0\\
\hline
\end{tabular}
&
\rightarrow
\begin{tabular}{|>{$}c<{$}>{$}c<{$}>{$}c<{$}|>{$}c<{$}|}
\hline
1&0&-\frac12&3\\
0&2&-4&-2\\
0&a&1&0\\
\hline
\end{tabular}
\rightarrow
\begin{tabular}{|>{$}c<{$}>{$}c<{$}>{$}c<{$}|>{$}c<{$}|}
\hline
1&0&-\frac12&3\\
0&1&-2&-1\\
0&0&1+2a&a\\
\hline
\end{tabular}
\end{align*}
Damit das Gleichungssystem eine Lösung hat, muss an dieser Stelle
der Gauss-Algorithmus fortgesetzt werden können. Regulär ist es
also nur, wenn $1+2a\ne 0$.
Singulär ist das System genau dann, wenn $a=-\frac12$.

In allen anderen
Fällen, also bei $a\ne -\frac12$  ist es regulär, und hat damit
genau eine
Lösung. Diese findet man, indem man den Gauss-Algorithmus zu Ende führt:
\begin{align*}
\begin{tabular}{|>{$}c<{$}>{$}c<{$}>{$}c<{$}|>{$}c<{$}|}
\hline
1&0&-\frac12&3\\
0&1&-2&-1\\
0&0&1+2a&a\\
\hline
\end{tabular}
&
\rightarrow
\begin{tabular}{|>{$}c<{$}>{$}c<{$}>{$}c<{$}|>{$}c<{$}|}
\hline
1&0&0&3+\frac{a}{2+4a}\\
0&1&0&-1+\frac{2a}{1+2a}\\
0&0&1&\frac{a}{1+2a}\\
\hline
\end{tabular}
=
\begin{tabular}{|>{$}c<{$}>{$}c<{$}>{$}c<{$}|>{$}c<{$}|}
\hline
1&0&0&\frac{6+13a}{2(1+2a)}\\
0&1&0&\frac{-1}{1+2a}\\
0&0&1&\frac{a}{1+2a}\\
\hline
\end{tabular}
\end{align*}
Die Lösung ist also
\[
\frac1{1+2a}
\begin{pmatrix}
\frac{6+13a}2\\-1\\a
\end{pmatrix}.
\]

Setzt man $a=-\frac12$ in das Gauss-Tableau ein, bekommt man
\[
\begin{tabular}{|>{$}c<{$}>{$}c<{$}>{$}c<{$}|>{$}c<{$}|}
\hline
1&0&-\frac12&3\\
0&1&-2&-1\\
0&0&0&-\frac12\\
\hline
\end{tabular}
\]
Die letzte Zeile entspricht der Gleichung $0=-\frac12$, die niemals
erfüllbar ist.  Im singulären Fall gibt es also gar keine Lösung,
der Fall unendlich vieler Lösungen kommt nicht vor.
\end{loesung}
