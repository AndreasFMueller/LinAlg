Bestimmen Sie den Rang der folgenden Matrizen:
\begin{teilaufgaben}
\item $\begin{pmatrix}1&2\\3&4\end{pmatrix}$
\item $\begin{pmatrix}1&2&3\\4&5&6\\7&8&9\end{pmatrix}$
\item $\begin{pmatrix}1&2&3&4\\5&6&7&8\\9&10&11&12\\13&14&15&16\end{pmatrix}$
\end{teilaufgaben}

\themaS{Rang}

\begin{loesung}
\begin{teilaufgaben}
\item Mit dem Gaussalgorithmus findet man
\[
\begin{tabular}{|>{$}c<{$}>{$}c<{$}|}
\hline
1&2\\
3&4\\
\hline
\end{tabular}
\rightarrow
\begin{tabular}{|>{$}c<{$}>{$}c<{$}|}
\hline
1&2\\
0&-2\\
\hline
\end{tabular}
\rightarrow
\begin{tabular}{|>{$}c<{$}>{$}c<{$}|}
\hline
1&0\\
0&1\\
\hline
\end{tabular}
\]
Der Rang ist also 2.
\item Mit dem Gaussalgorithmus findet man
\[
\begin{tabular}{|>{$}c<{$}>{$}c<{$}>{$}c<{$}|}
\hline
1&2&3\\
4&5&6\\
7&8&9\\
\hline
\end{tabular}
\rightarrow
\begin{tabular}{|>{$}c<{$}>{$}c<{$}>{$}c<{$}|}
\hline
1&2&3\\
0&-3&-6\\
0&-6&-12\\
\hline
\end{tabular}
\rightarrow
\begin{tabular}{|>{$}c<{$}>{$}c<{$}>{$}c<{$}|}
\hline
1&2&3\\
0&1&2\\
0&0&0\\
\hline
\end{tabular}
\rightarrow
\begin{tabular}{|>{$}c<{$}>{$}c<{$}>{$}c<{$}|}
\hline
1&0&-1\\
0&1&2\\
0&0&0\\
\hline
\end{tabular}
\]
Der Rang ist also 2.
\item Mit dem Gaussalgorithmus findet man
\begin{align*}
\begin{tabular}{|>{$}c<{$}>{$}c<{$}>{$}c<{$}>{$}c<{$}|}
\hline
1&2&3&4\\
5&6&7&8\\
9&10&11&12\\
13&14&15&16\\
\hline
\end{tabular}
&\rightarrow
\begin{tabular}{|>{$}c<{$}>{$}c<{$}>{$}c<{$}>{$}c<{$}|}
\hline
1&2&3&4\\
0&-4&-8&-12\\
0&-8&-16&-24\\
0&-12&-24&-36\\
\hline
\end{tabular}
\rightarrow
\begin{tabular}{|>{$}c<{$}>{$}c<{$}>{$}c<{$}>{$}c<{$}|}
\hline
1&2&3&4\\
0&1&2&3\\
0&0&0&0\\
0&0&0&0\\
\hline
\end{tabular}
\\
&\rightarrow
\begin{tabular}{|>{$}c<{$}>{$}c<{$}>{$}c<{$}>{$}c<{$}|}
\hline
1&0&-1&-2\\
0&1&2&3\\
0&0&0&0\\
0&0&0&0\\
\hline
\end{tabular}
\end{align*}
Der Rang ist also 2.
\qedhere
\end{teilaufgaben}
\end{loesung}
