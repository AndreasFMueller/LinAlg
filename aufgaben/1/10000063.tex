Gegeben ist das von den Parametern $w$ und $t$ abhängige lineare
Gleichungssystem
\[
\begin{linsys}{3}
 2x &+&      10y &+&  6z      &=&       24,\\
  x &+&  (4w+9)y &+&(12w+15)z &=& 28w+t+38,\\
-3x &-&      11y &-&   z      &=&       -8.
\end{linsys}
\]
\begin{teilaufgaben}
\item
Finden Sie die Lösung des Gleichungssystems im Falle $w=0$ und $t=6$.
\item
Für welche Werte des Parameters $w$ ist das Gleichungsystem nicht eindeutig
lösbar?
\item
Wie muss $t$ gewählt werden, damit das Gleichungssystem immer mindestens
eine Lösung hat?
\item
Bestimmen Sie die Lösungsmenge in dem Fall, wo das Gleichungssystem unendlich
viele Lösungen hat.
\end{teilaufgaben}

\begin{loesung}
Wir wenden den Gauss-Algorithmus auf das zugehörige Tableau an:
\begin{align}
\begin{tabular}{|>{$}c<{$} >{$}c<{$} >{$}c<{$}|>{$}c<{$}|}
\hline
 2&  10&      6&      24\\
 1&4w+9& 12w+15&28w+t+38\\
-3& -11&     -1&      -8\\
\hline
\end{tabular}
&\rightarrow
\begin{tabular}{|>{$}c<{$} >{$}c<{$} >{$}c<{$}|>{$}c<{$}|}
\hline
 1&   5&      3&      12\\
 0&4w+4& 12w+12&28w+t+26\\
 0&   4&      8&      28\\
\hline
\end{tabular}
\notag
\intertext{An dieser Stelle könnte $w+1=0$ ein nicht geeignetes
Pivot sein, wir verschieben die Behandlung dieses Falles auf 
später, indem wir die zweite und dritte Zeile vertauschen}
\begin{tabular}{|>{$}c<{$} >{$}c<{$} >{$}c<{$}|>{$}c<{$}|}
\hline
 1&   5&       3&      12\\
 0&   4&       8&      28\\
 0&4w+4& 12(w+1)&28w+t+26\\
\hline
\end{tabular}
&\rightarrow
\begin{tabular}{|>{$}c<{$} >{$}c<{$} >{$}c<{$}|>{$}c<{$}|}
\hline
 1&    5 &       3& 12\\
 0&    1 &       2&  7\\
 0&    0 &  4(w+1)&t-2\\
\hline
\end{tabular}
\label{10000063:singulaer}
\intertext{Wenn $w+1=0$ erhalten wir eine Nullzeile.
Das Gleichungssystem wird nur eine Lösung haben, wenn die rechte Seite
ebenfalls verschwindet.
Wir nehmen für die weitere Rechnung an, dass $w\ne-1$.}
\begin{tabular}{|>{$}c<{$} >{$}c<{$} >{$}c<{$}|>{$}c<{$}|}
\hline
 1&     5 & 3     &   12\\
 0&     1 & 2     &    7\\
 0&     0 & 4(w+1)&  t-2\\
\hline
\end{tabular}
&\rightarrow
\begin{tabular}{|>{$}c<{$} >{$}c<{$} >{$}c<{$}|>{$}c<{$}|}
\hline
1&   5& 0& 12-3q\\
0&   1& 0&  7-2q\\
0&   0& 1&     q\\
\hline
\end{tabular}
\notag
\\
&\rightarrow
\begin{tabular}{|>{$}c<{$} >{$}c<{$} >{$}c<{$}|>{$}c<{$}|}
\hline
1&   0& 0&-23+7q\\
0&   1& 0&  7-2q\\
0&   0& 1&     q\\
\hline
\end{tabular}
\notag
\end{align}
Dabei haben wir den etwas unschönen Bruch $q=\frac{t-2}{4(w+1)}$ abgekürzt.
Nach dieser Rechnung können wir jetzt die einzelnen Fragen beantworten.
\begin{teilaufgaben}
\item
Im Falle $w=0$ und $t=6$ wird
\[
q = \frac{t-2}{4(w+1)} = \frac{6-2}{4(0+1)} = 1.
\]
Daraus lassen sich die Unbekannten $x$, $y$ und $z$ leicht bestimmen
\begin{align*}
x &=-16+9q=-16+9=-7,\\
y&=7-3q=4\\
\text{und}\quad
z&=1.
\end{align*}
\item
Das Gleichungssystem ist nicht eindeutig lösbar, wenn $w+1=0$ ist, also für
$w=-1$.
\item
Für $w\ne-1$ spielt der Wert von $t$ keine Rolle, nur für $w=-1$ muss
genauer untersucht werden, was auf der rechten Seite der Nullzeile steht.
Dies kann im Schritt \eqref{10000063:singulaer} abgelesen werden.
Dort steht, dass die rechte Seite für $w=-1$ zu $t-2$ wird, d.~h.~es gibt
unendlich viele Lösungen genau dann, wenn $t=2$, andernfalls gibt es keine
Lösung.
Da gefordert wurde, dass das Gleichungssystem mindestens eine Lösung haben
muss, muss man im Falle $w=-1$ zusätzlich $t=2$ verlangen.
\item
Um die Lösungsmenge abzulesen, müssen wir ausgehend von
\eqref{10000063:singulaer} zu einem Schlusstableau gelangen:
\[
\begin{tabular}{|>{$}c<{$} >{$}c<{$} >{$}c<{$}|>{$}c<{$}|}
\hline
 1&     5 & 3     &   12\\
 0&     1 & 2     &    7\\
 0&     0 & 0     &    0\\
\hline
\end{tabular}
\rightarrow
\begin{tabular}{|>{$}c<{$} >{$}c<{$} >{$}c<{$}|>{$}c<{$}|}
\hline
 1&     0 & -12 & -23\\
 0&     1 &   2 &   7\\
 0&     0 &   0 &   0\\
\hline
\end{tabular}
\]
woraus wir die Lösungsmenge
\[
\mathbb L
=
\left\{
\left.
\begin{pmatrix}-23\\7\\0\end{pmatrix}
+
z\begin{pmatrix}12\\-3\\1\end{pmatrix}
\; \right| \;
z\in\mathbb R
\right\}.
\qedhere
\]
\end{teilaufgaben}
\end{loesung}

\begin{bewertung}
Gauss-Algorithmus Vorwärtsreduktion ({\bf V}) 1 Punkt und
Rückwärtseinsetzen ({\bf R}) 1 Punkt,
\begin{teilaufgaben}
\item ({\bf A}) 1 Punkt,
\item ({\bf B}) 1 Punkt,
\item ({\bf C}) 1 Punkt,
\item ({\bf D}) 1 Punkt.
\end{teilaufgaben}
\end{bewertung}
