%
% -1  -1   5
% -3   1  -3
% -0  -1   2
% inverse:
% -0.10000  -0.30000  -0.20000
%  0.60000  -0.20000  -1.80000
%  0.30000  -0.10000  -0.40000


Betrachten Sie das Gleichungssystem
\[
\begin{linsys}{3}
 -x&-& y&+&5z&=&-1\\
-3x&+& y&-&3z&=& 3\\
   &-& y&+&2z&=& 1
\end{linsys}
\]
\begin{teilaufgaben}
\item Berechnen Sie die Inverse der Koeffizientenmatrix.
\item Finden Sie die Lösung des Gleichungssystems mit Hilfe der Inversen.
\end{teilaufgaben}

\thema{inverse Matrix}
\thema{Gauss-Algorithmus}

\begin{loesung}
Wir schreiben $A$ für die Koeffizienten-Matrix:
\[
\begin{pmatrix}
  -1& -1&  5\\
  -3&  1& -3\\
   0& -1&  2
\end{pmatrix}
\]
\begin{teilaufgaben}
\item
Die Inverse kann man mit Hilfe des Gauss-Algorithmus berechnen:
\begin{align*}
\begin{tabular}{|>{$}c<{$}|>{$}c<{$}|}
\hline
A&E\\
\hline
\end{tabular}
&=
\begin{tabular}{|>{$}c<{$}>{$}c<{$}>{$}c<{$}|>{$}c<{$}>{$}c<{$}>{$}c<{$}|}
\hline
  -1& -1&  5&  1&  0&  0\\
  -3&  1& -3&  0&  1&  0\\
   0& -1&  2&  0&  0&  1\\
\hline
\end{tabular}
\rightarrow
\begin{tabular}{|>{$}c<{$}>{$}c<{$}>{$}c<{$}|>{$}c<{$}>{$}c<{$}>{$}c<{$}|}
\hline
   1&  1& -5& -1&  0&  0\\
   0&  4&-18& -3&  1&  0\\
   0& -1&  2&  0&  0&  1\\
\hline
\end{tabular}
\\
&\rightarrow
\begin{tabular}{|>{$}c<{$}>{$}c<{$}>{$}c<{$}|>{$}c<{$}>{$}c<{$}>{$}c<{$}|}
\hline
   1&  1& -5     & -1      &  0      &  0\\
   0&  1&-\frac92& -\frac34&  \frac14&  0\\
   0&  0&-\frac52& -\frac34&  \frac14&  1\\
\hline
\end{tabular}
\rightarrow
\begin{tabular}{|>{$}c<{$}>{$}c<{$}>{$}c<{$}|>{$}c<{$}>{$}c<{$}>{$}c<{$}|}
\hline
   1&  1& -5     & -1        &  0         &  0\\
   0&  1&-\frac92& -\frac34  &  \frac14   &  0\\
   0&  0&  1     & \frac3{10}& -\frac1{10}&-\frac25\\
\hline
\end{tabular}
\\
&\rightarrow
\begin{tabular}{|>{$}c<{$}>{$}c<{$}>{$}c<{$}|>{$}c<{$}>{$}c<{$}>{$}c<{$}|}
\hline
   1&  1&  0&\frac12      &-\frac12     &-2\\
   0&  1&  0&\frac{12}{20}&-\frac{4}{20}&-\frac{9}{5}\\
   0&  0&  1&\frac{3}{10} &-\frac1{10}  &-\frac25\\
\hline
\end{tabular}
\rightarrow
\begin{tabular}{|>{$}c<{$}>{$}c<{$}>{$}c<{$}|>{$}c<{$}>{$}c<{$}>{$}c<{$}|}
\hline
   1&  0&  0&-\frac1{10}  &-\frac3{10}  &-\frac15\\
   0&  1&  0&\frac35      &-\frac15     &-\frac95\\
   0&  0&  1&\frac{3}{10} &-\frac1{10}  &-\frac25\\
\hline
\end{tabular}
\\
&=\begin{tabular}{|>{$}c<{$}|>{$}c<{$}|}
\hline
E&A^{-1}\\
\hline
\end{tabular}
\end{align*}
Daraus kann jetzt die Inverse Matrix abgelesen werden:
\[
A^{-1}
=
\frac1{10}
\begin{pmatrix}
-1&-3& -2\\
 6&-2&-18\\
 3&-1& -4
\end{pmatrix}
\]
Wir kontrollieren das Resultat durch Ausmultiplizieren:
\begin{align*}
A\cdot A^{-1}
&=
\frac1{10}
\begin{pmatrix}
  -1& -1&  5\\
  -3&  1& -3\\
   0& -1&  2
\end{pmatrix}
\begin{pmatrix}
-1&-3& -2\\
 6&-2&-18\\
 3&-1& -4
\end{pmatrix}
\\
&=
\frac1{10}
{
\tiny
\begin{pmatrix}
(-1)\cdot(-1) + (-1)\cdot6 + 5\cdot3
	&(-1)\cdot(-3) + (-1)\cdot(-2) + 5\cdot(-1)
		&(-1)\cdot(-2) + (-1)\cdot(-18) + 5\cdot(-4)\\
(-3)\cdot(-1) + 1\cdot6 + (-3)\cdot3
	&(-3)\cdot(-3) + 1\cdot(-2) + (-3)\cdot(-1)
		&(-3)\cdot(-2) + 1\cdot(-18) + (-3)\cdot(-4)\\
0\cdot(-1) + (-1)\cdot6 + 2\cdot3
	&0\cdot(-3) + (-1)\cdot(-2) + 2\cdot(-1)
		&0\cdot(-2) + (-1)\cdot(-18) + 2\cdot(-4)
\end{pmatrix}
}
\\
&=
\frac1{10}
\begin{pmatrix}
1-6+15 & 3+2-5 & 2 +18 - 20\\
3+6-9  & 9 -2+3& 6-18+12\\
0-6+6  & 0+2-2&0+18-8
\end{pmatrix}
=\begin{pmatrix}
1&0&0\\
0&1&0\\
0&0&1
\end{pmatrix}=E.
\end{align*}
\item Die Lösung des Gleichungssystems kann jetzt mit $x=A^{-1}b$ gefunden
werden:
\begin{align*}
b&=
\begin{pmatrix}-1\\3\\1\end{pmatrix}
&
x&=A^{-1}b=
\frac1{10}
\begin{pmatrix}
-1&-3& -2\\
 6&-2&-18\\
 3&-1& -4
\end{pmatrix}
\begin{pmatrix}-1\\ 3\\ 1\end{pmatrix}
=
\frac1{10}
\begin{pmatrix}
 1-9-2\\
-6-6-18\\
-3-3-4
\end{pmatrix}
=
\begin{pmatrix}
-1\\
-3\\
-1
\end{pmatrix}.
\end{align*}
Auch dieses Resultat können wir durch Ausrechnen von $Ax$ kontrollieren:
\begin{align*}
Ax
&=
\begin{pmatrix}
  -1& -1&  5\\
  -3&  1& -3\\
   0& -1&  2
\end{pmatrix}
\begin{pmatrix}
-1\\-3\\-1
\end{pmatrix}
=
\begin{pmatrix}
1+3-5\\
3-3+3\\
0+3-2
\end{pmatrix}
=
\begin{pmatrix}
-1\\3\\1
\end{pmatrix}
=b.
\end{align*}
\end{teilaufgaben}
Die inverse Matrix in Teil a) kann natürlich auch mit Hilfe der
Minoren berechnet werden.
Dazu brauchen wir zunächst die Determinante.
Diese könnte man zwar mit
Hilfe der Sarrusschen Formel erhalten, da wir aber oben den Gauss-Algorithmus
durchgeführt haben, können wir auch die Pivot-Elemente zusammentragen:
\[
\det(A)
=
(-1)\cdot 4\cdot \biggl(-\frac52\biggr)
=
10.
\]
Die Inverse kann man jetzt direkt hinschreiben:
\begin{align*}
A^{-1}
&=
\frac1{10}
\begin{pmatrix}
\left|\begin{matrix} 1&-3\\-1& 2\end{matrix}\right|&
-\left|\begin{matrix}-1& 5\\-1& 2\end{matrix}\right|&
\left|\begin{matrix}-1& 5\\ 1&-3\end{matrix}\right|\\
-\left|\begin{matrix}-3&-3\\ 0& 2\end{matrix}\right|&
\left|\begin{matrix}-1& 5\\ 0& 2\end{matrix}\right|&
-\left|\begin{matrix}-1& 5\\-3&-3\end{matrix}\right|\\
\left|\begin{matrix}-3& 1\\ 0&-1\end{matrix}\right|&
-\left|\begin{matrix}-1&-1\\ 0&-1\end{matrix}\right|&
\left|\begin{matrix}-1&-1\\-3& 1\end{matrix}\right|
\end{pmatrix}
=
\frac1{10}
\begin{pmatrix}
-1&-3& -2\\
 6&-2&-18\\
 3&-1& -4
\end{pmatrix},
\end{align*}
was mit der Mittels Gaussalgorithmus gefundenen Inversen übereinstimmt.
\end{loesung}

\begin{bewertung}
\begin{teilaufgaben}
\item
Ansatz (\textbf{A}) 1 Punkt,
Durchführung des Gauss Algorithmus (\textbf{G}) 2 Punkte,
Inverse Matrix (\textbf{I}) 1 Punkt,
\item
Lösungsansatz $x=A^{-1}b$ (\textbf{B}) 1 Punkt,
korrekte Lösung (\textbf{L}) 1 Punkt.
\end{teilaufgaben}
\end{bewertung}

