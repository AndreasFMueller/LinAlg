Die Formeln \begin{align*}
y_0&=\frac1n(x_1+\dots + x_n)\\
y_1&=x_2-x_1\\
y_2&=x_3-x_2\\
&\vdots\\
y_{n-1}&=x_n-x_{n-1}
\end{align*}
beschreiben eine Transformation der Variablen $x_1,\dots,x_n$
in die Variablen $y_0,\dots,y_{n-1}$.
\begin{teilaufgaben}
\item Schreiben Sie die Transformation als Produkt von Matrizen und Vektoren.
\item Finden Sie die Inverse der Matrix im Falle $n=5$.
\item Können Sie daraus eine Vermutung für die Form der inversen Matrix im allgemeinen Fall
ableiten, und eventuell beweisen?
\end{teilaufgaben}

\begin{loesung}
\begin{teilaufgaben}
\item Wir schreiben die Variablen $x_i$ und $y_j$ als Spaltenvektoren $x$
bzw.~$y$. Dann kann man die Transformation in Matrixschreibweise als
\begin{equation}
\begin{pmatrix}
y_0\\
y_1\\
y_2\\
\vdots\\
y_{n-1}
\end{pmatrix}
=
\underbrace{
\begin{pmatrix}
\frac1n&\frac1n&\frac1n&\dots &\frac1n&\frac1n\\
     -1&      1&      0&\dots &      0&      0\\
      0&     -1&      1&\dots &      0&      0\\
\vdots &\vdots &\vdots &\ddots&\vdots &\vdots \\
      0&      0&      0&      &     -1&      1\\
\end{pmatrix}}_{=A}
\begin{pmatrix}
x_1\\x_2\\x_3\\\vdots\\x_{n-1}\\x_n
\end{pmatrix}
\label{10000035:trafo}
\end{equation}
schreiben.
Wir nennen die Matrix in (\ref{10000035:trafo}) $A$.
\item
Die Inverse der Matrix $A$ kann mit dem Gauss-Algorithmus ermittelt werden.
\allowdisplaybreaks
\begin{align*}
&\begin{tabular}{|>{$}c<{$}>{$}c<{$}>{$}c<{$}>{$}c<{$}>{$}c<{$}|>{$}c<{$}>{$}c<{$}>{$}c<{$}>{$}c<{$}>{$}c<{$}|}
\hline
\frac15&\frac15&\frac15&\frac15&\frac15&     1&     0&     0&     0&     0\\
     -1&      1&      0&      0&      0&     0&     1&     0&     0&     0\\
      0&     -1&      1&      0&      0&     0&     0&     1&     0&     0\\
      0&      0&     -1&      1&      0&     0&     0&     0&     1&     0\\
      0&      0&      0&     -1&      1&     0&     0&     0&     0&     1\\
\hline
\end{tabular}
\\
&\rightarrow
\begin{tabular}{|>{$}c<{$}>{$}c<{$}>{$}c<{$}>{$}c<{$}>{$}c<{$}|>{$}c<{$}>{$}c<{$}>{$}c<{$}>{$}c<{$}>{$}c<{$}|}
\hline
      1&      1&      1&      1&      1&     5&     0&     0&     0&     0\\
     -1&      1&      0&      0&      0&     0&     1&     0&     0&     0\\
      0&     -1&      1&      0&      0&     0&     0&     1&     0&     0\\
      0&      0&     -1&      1&      0&     0&     0&     0&     1&     0\\
      0&      0&      0&     -1&      1&     0&     0&     0&     0&     1\\
\hline
\end{tabular}
\\
&\rightarrow
\begin{tabular}{|>{$}c<{$}>{$}c<{$}>{$}c<{$}>{$}c<{$}>{$}c<{$}|>{$}c<{$}>{$}c<{$}>{$}c<{$}>{$}c<{$}>{$}c<{$}|}
\hline
      1&      1&      1&      1&      1&     5&     0&     0&     0&     0\\
      0&      2&      1&      1&      1&     5&     1&     0&     0&     0\\
      0&     -1&      1&      0&      0&     0&     0&     1&     0&     0\\
      0&      0&     -1&      1&      0&     0&     0&     0&     1&     0\\
      0&      0&      0&     -1&      1&     0&     0&     0&     0&     1\\
\hline
\end{tabular}
\\
&\rightarrow
\begin{tabular}{|>{$}c<{$}>{$}c<{$}>{$}c<{$}>{$}c<{$}>{$}c<{$}|>{$}c<{$}>{$}c<{$}>{$}c<{$}>{$}c<{$}>{$}c<{$}|}
\hline
      1&      1&      1&      1&      1&      5&      0&      0&      0&      0\\
      0&      1&\frac12&\frac12&\frac12&\frac52&\frac12&      0&      0&      0\\
      0&      0&\frac32&\frac12&\frac12&\frac52&\frac12&      1&      0&      0\\
      0&      0&     -1&      1&      0&      0&      0&      0&      1&      0\\
      0&      0&      0&     -1&      1&      0&      0&      0&      0&      1\\
\hline
\end{tabular}
\\
&\rightarrow
\begin{tabular}{|>{$}c<{$}>{$}c<{$}>{$}c<{$}>{$}c<{$}>{$}c<{$}|>{$}c<{$}>{$}c<{$}>{$}c<{$}>{$}c<{$}>{$}c<{$}|}
\hline
      1&      1&      1&      1&      1&      5&      0&      0&      0&      0\\
      0&      1&\frac12&\frac12&\frac12&\frac52&\frac12&      0&      0&      0\\
      0&      0&      1&\frac13&\frac13&\frac53&\frac13&\frac23&      0&      0\\
      0&      0&      0&\frac43&\frac13&\frac53&\frac13&\frac23&      1&      0\\
      0&      0&      0&     -1&      1&      0&      0&      0&      0&      1\\
\hline
\end{tabular}
\\
&\rightarrow
\begin{tabular}{|>{$}c<{$}>{$}c<{$}>{$}c<{$}>{$}c<{$}>{$}c<{$}|>{$}c<{$}>{$}c<{$}>{$}c<{$}>{$}c<{$}>{$}c<{$}|}
\hline
      1&      1&      1&      1&      1&      5&      0&      0&      0&      0\\
      0&      1&\frac12&\frac12&\frac12&\frac52&\frac12&      0&      0&      0\\
      0&      0&      1&\frac13&\frac13&\frac53&\frac13&\frac23&      0&      0\\
      0&      0&      0&      1&\frac14&\frac54&\frac14&\frac24&\frac34&      0\\
      0&      0&      0&      0&\frac54&\frac54&\frac14&\frac24&\frac34&      1\\
\hline
\end{tabular}
\\
&\rightarrow
\begin{tabular}{|>{$}c<{$}>{$}c<{$}>{$}c<{$}>{$}c<{$}>{$}c<{$}|>{$}c<{$}>{$}c<{$}>{$}c<{$}>{$}c<{$}>{$}c<{$}|}
\hline
      1&      1&      1&      1&      1&      5&      0&      0&      0&      0\\
      0&      1&\frac12&\frac12&\frac12&\frac52&\frac12&      0&      0&      0\\
      0&      0&      1&\frac13&\frac13&\frac53&\frac13&\frac23&      0&      0\\
      0&      0&      0&      1&\frac14&\frac54&\frac14&\frac24&\frac34&      0\\
      0&      0&      0&      0&      1&\frac55&\frac15&\frac25&\frac35&\frac45\\
\hline
\end{tabular}
\\
&\rightarrow
\begin{tabular}{|>{$}c<{$}>{$}c<{$}>{$}c<{$}>{$}c<{$}>{$}c<{$}|>{$}c<{$}>{$}c<{$}>{$}c<{$}>{$}c<{$}>{$}c<{$}|}
\hline
      1&      1&      1&      1&      0&      4&-\frac15   &-\frac25  &-\frac35   &-\frac45\\
      0&      1&\frac12&\frac12&      0&      2& \frac25   &-\frac15  &-\frac3{10}&-\frac25\\
      0&      0&      1&\frac13&      0&\frac43& \frac4{15}&\frac8{15}&-\frac25   &-\frac4{15}\\
      0&      0&      0&      1&      0&      1& \frac15   &\frac25   & \frac35   &-\frac15\\
      0&      0&      0&      0&      1&      1& \frac15   &\frac25   & \frac35   & \frac45\\
\hline
\end{tabular}
\\
&\rightarrow
\begin{tabular}{|>{$}c<{$}>{$}c<{$}>{$}c<{$}>{$}c<{$}>{$}c<{$}|>{$}c<{$}>{$}c<{$}>{$}c<{$}>{$}c<{$}>{$}c<{$}|}
\hline
      1&      1&      1&      0&      0&      3&-\frac25   &-\frac45  &-\frac65&-\frac35\\
      0&      1&\frac12&      0&      0&\frac32& \frac3{10}&-\frac25  &-\frac35&-\frac3{10}\\
      0&      0&      1&      0&      0&      1& \frac15   & \frac25  &-\frac25&-\frac15\\
      0&      0&      0&      1&      0&      1& \frac15   & \frac25  & \frac35&-\frac15\\
      0&      0&      0&      0&      1&      1& \frac15   & \frac25  & \frac35& \frac45\\
\hline
\end{tabular}
\\
&\rightarrow
\begin{tabular}{|>{$}c<{$}>{$}c<{$}>{$}c<{$}>{$}c<{$}>{$}c<{$}|>{$}c<{$}>{$}c<{$}>{$}c<{$}>{$}c<{$}>{$}c<{$}|}
\hline
      1&      1&      0&      0&      0&      2&-\frac35   &-\frac65  &-\frac45&-\frac25\\
      0&      1&      0&      0&      0&      1& \frac15   &-\frac35  &-\frac25&-\frac15\\
      0&      0&      1&      0&      0&      1& \frac15   & \frac25  &-\frac25&-\frac15\\
      0&      0&      0&      1&      0&      1& \frac15   & \frac25  & \frac35&-\frac15\\
      0&      0&      0&      0&      1&      1& \frac15   & \frac25  & \frac35& \frac45\\
\hline
\end{tabular}
\\
&\rightarrow
\begin{tabular}{|>{$}c<{$}>{$}c<{$}>{$}c<{$}>{$}c<{$}>{$}c<{$}|>{$}c<{$}>{$}c<{$}>{$}c<{$}>{$}c<{$}>{$}c<{$}|}
\hline
      1&      0&      0&      0&      0&      1&-\frac45   &-\frac35  &-\frac25&-\frac15\\
      0&      1&      0&      0&      0&      1& \frac15   &-\frac35  &-\frac25&-\frac15\\
      0&      0&      1&      0&      0&      1& \frac15   & \frac25  &-\frac25&-\frac15\\
      0&      0&      0&      1&      0&      1& \frac15   & \frac25  & \frac35&-\frac15\\
      0&      0&      0&      0&      1&      1& \frac15   & \frac25  & \frac35& \frac45\\
\hline
\end{tabular}
\end{align*}
Daraus kann man die Inverse ablesen:
\[
A^{-1}=\frac15\begin{pmatrix}
5&-4&-3&-2&-1\\
5& 1&-3&-2&-1\\
5& 1& 2&-2&-1\\
5& 1& 2& 3&-1\\
5& 1& 2& 3& 4\\
\end{pmatrix}.
\]
\item Die allgemeine Form könnte wie folgt aussehen:
\[
B=\begin{pmatrix}
1     &-\frac{n-1}n&-\frac{n-2}n&-\frac{n-3}n&\dots   &-\frac1n    \\
1     & \frac1n    &-\frac{n-2}n&-\frac{n-3}n&\dots   &-\frac1n    \\
1     & \frac1n    & \frac2n    &-\frac{n-3}n&\dots   &-\frac1n    \\
1     & \frac1n    & \frac2n    & \frac3n    &\dots   &-\frac1n    \\
\vdots&\vdots      &\vdots      &\vdots      &\ddots  &\vdots      \\
1     & \frac1n    & \frac2n    & \frac3n    &  \dots & \frac{n-1}n
\end{pmatrix}
\]
Die Einträge oberhalb der Diagonalen sind alle negativ, und die Differenz zu den
Einträgen unter der Diagonalen ist 1.
In der Spalte $k\ge 2$ stehen die Elemente
\[
a_{lk}=\begin{cases}
-\frac{n-k+1}n&\qquad\text{$l< k$, d.~h.~oberhalb der Diagonalen,}\\
\frac{k-1}n&\qquad\text{$l\ge k$, d.~h.~auf und unterhalb der Diagonalen.}
\end{cases}
\]

Um zu beweisen, dass dies tatsächlich die Inverse im allgemeinen Fall ist,
dass also $A^{-1}=B$ ist, überlegen wir uns das Produkt $AB$,
wir erwarten, dass $AB=E$. 

Das Produkt der ersten Zeile von $A$ mit einer Spalte von $B$ ist der
Mittelwert der Einträge in dieser Spalte.
Für die erste Spalte ergibt das wie gewünscht $0$.
Für die Spalte $k$ mit $k\ge 2$ müssen wir den Mittelwert der $a_{lk}$ berechnen.
Dieser ist
\[
\frac1n\biggl(-(k-1)\frac{n-k+1}n+(n-k+1)\frac{k-1}n\biggr)
%=\frac1n\biggl(-\frac{(k-1)(n-k+1)}n+\frac{(n-k+1)(k-1)}n\biggr)
=0,
\]
also ist die erste Zeile der Produktmatrix korrekt, d.~h.~die erste Zeile
einer Einheitsmatrix.

Die Folgezeilen in $A$ haben jeweils nur zwei Einträge $-1$ und $1$,
das Produkt mit einer Spalte von $B$ ergibt die Differenz der entsprechenden
Einträge in der Spalte. Die Differenz ergibt genau dann $1$, wenn $-1$ 
auf einen negativen Eintrag und $1$ auf eine positiven Eintrag der Spalte
fällt, und die Differenz ist in diesem Fall $1$.
In allen anderen Fällen sind die Werte einer Spalte konstant, es ergibt sich
keine Differenz.
Damit ist nachgewiesen, dass das Matrizenprodukt $AB=E$ ist, also ist
$A^{-1}=B$.
\qedhere
\end{teilaufgaben}
\end{loesung}

\begin{diskussion}
Aus einem Anwendungsproblem, vorgeschlagen von Tabea M\'endez.
\end{diskussion}

