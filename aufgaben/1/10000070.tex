Auf dem Youtube-Kanal MindYourDecisions wurde die folgende Aufgabe
gestellt\footnote{\url{https://youtu.be/UpS9VdY7Nxg}}:
Ein Sack enthält Kugeln in vier verschiedenen Farben.
Es gilt:
\begin{center}
\begin{tabular}{rl}
1)&14 Kugeln sind nicht blau\\
2)&16 Kugeln sind nicht gelb\\
3)&24 Kugeln sind nicht rot\\
4)&12 Kugeln sind nicht grün
\end{tabular}
\end{center}
\begin{teilaufgaben}
\item
Finden Sie die Anzahl der Kugeln jeder Farbe mit Hilfe eines linearen 
Gleichungssystems.
\item
Warum hat die Matrix des Gleichungsssystems den Eigenwert $-1$?
\end{teilaufgaben}

\begin{loesung}
\begin{teilaufgaben}
\item
Die Anzahlen werden mit $x_i$, $i=1,\dots,4$ bezeichnet.
Die vier Bedingungen besagen, dass die Summe von jeweils drei der Variablen
bekannt ist.
Es entsteht so das Gleichungssystem
\[
\begin{linsys}{4}
   & &x_2&+&x_3&+&x_4&=&14\\
x_1& &   &+&x_3&+&x_4&=&16\\
x_1&+&x_2& &   &+&x_4&=&24\\
x_1&+&x_2&+&x_3& &   &=&12\\
\end{linsys}
\]
mit Koeffizientenmatrix
\[
A=\begin{pmatrix}
0&1&1&1\\
1&0&1&1\\
1&1&0&1\\
1&1&1&0
\end{pmatrix}
\qquad\text{und rechter Seite}\qquad
b=\begin{pmatrix}
14\\16\\24\\12
\end{pmatrix}.
\]
Um das Gleichungsystem mit dem Gauss-Algorithmus zu lösen, muss zuerst
eine andere Zeile nach oben geschoben werden, da sonst das erste
Pivot-Element 0 wäre.
Wir wählen die letzte Zeile:
\begin{align*}
\begin{tabular}{|>{$}r<{$}>{$}r<{$}>{$}r<{$}>{$}r<{$}|>{$}r<{$}|}
\hline
 1& 1& 1& 0& 12\\
 0& 1& 1& 1& 14\\
 1& 0& 1& 1& 16\\
 1& 1& 0& 1& 24\\
\hline
\end{tabular}
&\rightarrow
\begin{tabular}{|>{$}r<{$}>{$}r<{$}>{$}r<{$}>{$}r<{$}|>{$}r<{$}|}
\hline
 1& 1& 1& 0& 12\\
 0& 1& 1& 1& 14\\
 0&-1& 0& 1&  4\\
 0& 0&-1& 1& 12\\
\hline
\end{tabular}
\rightarrow
\begin{tabular}{|>{$}r<{$}>{$}r<{$}>{$}r<{$}>{$}r<{$}|>{$}r<{$}|}
\hline
 1& 1& 1& 0& 12\\
 0& 1& 1& 1& 14\\
 0& 0& 1& 2& 18\\
 0& 0&-1& 1& 12\\
\hline
\end{tabular}
\\
&\rightarrow
\begin{tabular}{|>{$}r<{$}>{$}r<{$}>{$}r<{$}>{$}r<{$}|>{$}r<{$}|}
\hline
 1& 1& 1& 0& 12\\
 0& 1& 1& 1& 14\\
 0& 0& 1& 2& 18\\
 0& 0& 0& 3& 30\\
\hline
\end{tabular}
\rightarrow
\begin{tabular}{|>{$}r<{$}>{$}r<{$}>{$}r<{$}>{$}r<{$}|>{$}r<{$}|}
\hline
 1& 1& 1& 0& 12\\
 0& 1& 1& 0&  4\\
 0& 0& 1& 0& -2\\
 0& 0& 0& 1& 10\\
\hline
\end{tabular}
\rightarrow
\begin{tabular}{|>{$}r<{$}>{$}r<{$}>{$}r<{$}>{$}r<{$}|>{$}r<{$}|}
\hline
 1& 1& 0& 0& 14\\
 0& 1& 0& 0&  6\\
 0& 0& 1& 0& -2\\
 0& 0& 0& 1& 10\\
\hline
\end{tabular}
\\
&\rightarrow
\begin{tabular}{|>{$}r<{$}>{$}r<{$}>{$}r<{$}>{$}r<{$}|>{$}r<{$}|}
\hline
 1& 0& 0& 0&  8\\
 0& 1& 0& 0&  6\\
 0& 0& 1& 0& -2\\
 0& 0& 0& 1& 10\\
\hline
\end{tabular}
\end{align*}
Das Gleichungssytem kann zwar gelöst werden, aber die Anzahl $-2$ der roten
Kugeln lässt sich natürlich nicht realisieren.
Die gestellte Aufgabe hat keine Lösung.
\item
Die Matrix $A$ hat $\lambda$ als Eigenwert, wenn $A-\lambda I$ nicht regulär
ist.
Setzt man den Wert $\lambda=-1$ ein, erhält man
\[
A-(-1)I
=
\begin{pmatrix}
0&1&1&1\\
1&0&1&1\\
1&1&0&1\\
1&1&1&0
\end{pmatrix}
+
\begin{pmatrix}
1&0&0&0\\
0&1&0&0\\
0&0&1&0\\
0&0&0&1
\end{pmatrix}
=
\begin{pmatrix}
1&1&1&1\\
1&1&1&1\\
1&1&1&1\\
1&1&1&1
\end{pmatrix},
\]
die Rang $1$ hat.
\end{teilaufgaben}
\end{loesung}

\begin{diskussion}
Aus der Tatsache, dass $\operatorname{Rang}(A-(-1)I)=3$ ist, kann man
weiter folgern, dass $\lambda=-1$ ein dreifacher Eigenwert ist.
In der Tat ist das charakteristische Polynom
\[
\chi_A(\lambda)
=
\lambda^4-6\lambda^2-8\lambda - 3.
\]
Dies muss durch $(\lambda+1)^3=\lambda^3+3\lambda^2+3\lambda+1$ teilbar
sein.
Der Quotient muss ein Faktor der Form $\lambda - a$ sein.
Dies ist nur möglich wenn $1\cdot (-a) = -3$ ist, also muss $a=3$ sein.
Tatsächlich kann man nachrechnen, dass
\[
\chi_A(\lambda)=
(\lambda+1)^3(\lambda-3).
\]
\end{diskussion}

