Wie viele Lösungen hat das Gleichungssystem
\[
\begin{linsys}{3}
3u&+&5v&+&7w&=&4\\
5u&+&7v&+&11w&=&6\\
4u&+&6v&+&9w&=&5\\
\end{linsys}
\qquad
?
\]
Geben Sie die Lösungsmenge an.

\begin{loesung}
Die dritte Gleichung ist das arithmetische Mittel der ersten zwei,
die gleichungen sind also linear abhängig, das System ist singular
und hat unendlich viele Lösungen. Diese findet man mit dem
Gauss-Algorithmus, doch um den Gang der Rechnung etwas zu vereinfachen,
ersetzen wir im ersten Schritt die erste Zeile durch die Differenz von
zweiter und erster Zeile:
\begin{align*}
\begin{tabular}{|ccc|c|}
\hline
3&5&7&4\\
5&7&11&6\\
4&6&9&5\\
\hline
\end{tabular}
&\rightarrow
\begin{tabular}{|ccc|c|}
\hline
2&2&4&2\\
5&7&11&6\\
4&6&9&5\\
\hline
\end{tabular}
\\
&\rightarrow
\begin{tabular}{|ccc|c|}
\hline
1&1&2&1\\
0&2&1&1\\
0&2&1&1\\
\hline
\end{tabular}
\\
&\rightarrow
\begin{tabular}{|ccc|c|}
\hline
1&1&2&1\\
0&1&$\frac12$&$\frac12$\\
0&0&0&0\\
\hline
\end{tabular}
\\
&\rightarrow
\begin{tabular}{|ccc|c|}
\hline
1&0&$\frac32$&$\frac12$\\
0&1&$\frac12$&$\frac12$\\
0&0&0&0\\
\hline
\end{tabular}
\end{align*}
Aus der letzten Form kann man ablesen, dass die Matrix den
Rang 1 hat und damit der Lösungsraum eindimensional wird.
Die Variable $w$ kann frei gewählt werden, als Lösungsraum
ergibt sich
\[
{\mathbb L}=\left\{\left.\begin{pmatrix}\frac12-\frac32w\\\frac12-\frac12w\\w\end{pmatrix}\right|\;w\in\mathbb R\right\}.
\qedhere
\]
\end{loesung}

