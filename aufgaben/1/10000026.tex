Finden Sie für das Netzwerk
\begin{center}
\begin{tikzpicture}[>=latex,thick]
\def\r{0.3}
\def\l{1.5}
\def\w{1.5pt}
\coordinate (A1) at (-\l,\l);
\coordinate (A2) at (\l,\l);
\coordinate (A3) at (-\l,-\l);
\coordinate (A4) at (\l,-\l);
\node at ($0.5*(A1)+0.5*(A2)$) [above] {$\scriptstyle 1\mathstrut$};
\node at ($0.5*(A3)+0.5*(A4)$) [below] {$\scriptstyle 6\mathstrut$};
\node at ($0.5*(A1)+0.5*(A3)$) [left] {$\scriptstyle 2\mathstrut$};
\node at ($0.5*(A2)+0.5*(A4)$) [right] {$\scriptstyle 5\mathstrut$};
\node at ($0.65*(A1)+0.35*(A4)+(-0.2,0)$)
	[above right] {$\scriptstyle 3\mathstrut$};
\node at ($0.65*(A2)+0.35*(A3)+(0.2,0)$)
	[above left] {$\scriptstyle 4\mathstrut$};
\draw[line width=\w] (A1) -- (A4);
\draw[color=white,line width=8pt] (A2) -- (A3);
\draw[line width=\w] (A1) -- (A2);
\draw[line width=\w] (A1) -- (A3);
\draw[line width=\w] (A2) -- (A3);
\draw[line width=\w] (A2) -- (A4);
\draw[line width=\w] (A3) -- (A4);
\def\knoten#1#2{
	\fill[color=white] #1 circle[radius=\r];
	\draw[line width=\w] #1 circle[radius=\r];
	\node at #1 {$#2\mathstrut$};
}
\knoten{(A1)}{1}
\knoten{(A2)}{2}
\knoten{(A3)}{3}
\knoten{(A4)}{4}
\end{tikzpicture}
\end{center}
eine minimale Menge von Zyklen.

\begin{hinweis}
Gauss-Calculator: \gaussurl{gausscalc:10000026}
\end{hinweis}

\thema{Zyklen}

\begin{loesung}
\definecolor{greenT}{rgb}{0,0.666,0}
Die Matrix $\partial$ dieses Netzwerkes ist
\[
\partial
=
\begin{pmatrix*}[r]
-1&-1&-1& 0& 0& 0\\
 1& 0& 0&-1&-1& 0\\
 0& 1& 0& 1& 0&-1\\
 0& 0& 1& 0& 1& 1
\end{pmatrix*}.
\]
Wendet man auf $\partial$ den Gaussalgorithmus an, findet man
\begin{align*}
\begin{tabular}{|>{$}r<{$}>{$}r<{$}>{$}r<{$}>{$}r<{$}>{$}r<{$}>{$}r<{$}|}
\hline
x_1&x_2&x_3&x_4&x_5&x_6\\
\hline
-1&-1&-1& 0& 0& 0\\
 1& 0& 0&-1&-1& 0\\
 0& 1& 0& 1& 0&-1\\
 0& 0& 1& 0& 1& 1\\
\hline
\end{tabular}
&\rightarrow
\begin{tabular}{|>{$}r<{$}>{$}r<{$}>{$}r<{$}>{$}r<{$}>{$}r<{$}>{$}r<{$}|}
\hline
x_1&x_2&x_3&x_4&x_5&x_6\\
\hline
\phantom{-}1& 1& 1& 0& 0& 0\\
 0&-1&-1&-1&-1& 0\\
 0& 1& 0& 1& 0&-1\\
 0& 0& 1& 0& 1& 1\\
\hline
\end{tabular}
\\
&\rightarrow
\begin{tabular}{|>{$}r<{$}>{$}r<{$}>{$}r<{$}>{$}r<{$}>{$}r<{$}>{$}r<{$}|}
\hline
x_1&x_2&x_3&x_4&x_5&x_6\\
\hline
\phantom{-}1&\phantom{-}1& 1&\phantom{-}0& 0& 0\\
 0& 1& 1& 1& 1& 0\\
 0& 0&-1& 0&-1&-1\\
 0& 0& 1& 0& 1& 1\\
\hline
\end{tabular}
\\
&\rightarrow
\begin{tabular}{|>{$}r<{$}>{$}r<{$}>{$}r<{$}>{$}r<{$}>{$}r<{$}>{$}r<{$}|}
\hline
x_1&x_2&x_3&x_4&x_5&x_6\\
\hline
\phantom{-}1&\phantom{-} 1&\phantom{-} 1&\phantom{-} 0&\phantom{-} 0&\phantom{-} 0\\
 0& 1& 1& 1& 1& 0\\
 0& 0& 1& 0& 1& 1\\
 0& 0& 0& 0& 0& 0\\
\hline
\end{tabular}
\\
&\rightarrow
\begin{tabular}{|>{$}r<{$}>{$}r<{$}>{$}r<{$}>{$}r<{$}>{$}r<{$}>{$}r<{$}|}
\hline
x_1&x_2&x_3&x_4&x_5&x_6\\
\hline
\phantom{-} 1&\phantom{-} 1&\phantom{-} 0&\phantom{-} 0&-1&-1\\
 0& 1& 0& 1& 0&-1\\
 0& 0& 1& 0& 1& 1\\
 0& 0& 0& 0& 0& 0\\
\hline
\end{tabular}
\\
&\rightarrow
\begin{tabular}{|>{$}r<{$}>{$}r<{$}>{$}r<{$}>{$}r<{$}>{$}r<{$}>{$}r<{$}|}
\hline
x_1&x_2&x_3&x_4&x_5&x_6\\
\hline
\phantom{-} 1&\phantom{-} 0&\phantom{-} 0&-1&-1& 0\\
 0& 1& 0& 1& 0&-1\\
 0& 0& 1& 0& 1& 1\\
 0& 0& 0& 0& 0& 0\\
\hline
  &  &  &\color{greenT}*&\color{greenT}*&\color{greenT}*\\
\hline
\end{tabular}
\end{align*}
Die frei wählbaren Variablen entsprechen den Kanten 4, 5 und 6, man kann
sie auf $1$ oder $-1$ setzen, je nachdem in welcher Richtung die Kante
im entsprechenden Zyklus durchlaufen werden soll.
Daraus liest man ab, dass drei Zyklen gebildet werden, die dadurch
bestimmt sind, ob man die Kanten 4, 5 oder 6 im Zyklus drin haben will
oder nicht. Die drei Zyklen haben die Kantenvektoren
\[
z_1=\begin{pmatrix*}[r]
 1\\
-1\\
 0\\
\color{greenT} 1\\
\color{greenT} 0\\
\color{greenT} 0\\
\end{pmatrix*},\quad
z_2=\begin{pmatrix*}[r]
 1\\
 0\\
-1\\
\color{greenT} 0\\
\color{greenT} 1\\
\color{greenT} 0
\end{pmatrix*},\quad
z_3= \begin{pmatrix*}[r]
 0\\
 1\\
-1\\
\color{greenT} 0\\
\color{greenT} 0\\
\color{greenT} 1
\end{pmatrix*}
\]
Die folgende Abbildung zeigt die drei Zyklen:
\begin{center}
\begin{tikzpicture}[>=latex,thick]
\def\r{0.3}
\def\l{1.5}
\def\knoten#1#2{
	\fill[color=white] #1 circle[radius=\r];
	\draw[line width=1pt] #1 circle[radius=\r];
	\node at #1 {$#2\mathstrut$};
}
\def\alleknoten{
	\knoten{(A1)}{1}
	\knoten{(A2)}{2}
	\knoten{(A3)}{3}
	\knoten{(A4)}{4}
}
\def\kantenlabels{
	\node at ($0.5*(A1)+0.5*(A2)$) [above] {$\scriptstyle 1\mathstrut$};
	\node at ($0.5*(A3)+0.5*(A4)$) [below] {$\scriptstyle 6\mathstrut$};
	\node at ($0.5*(A1)+0.5*(A3)$) [left] {$\scriptstyle 2\mathstrut$};
	\node at ($0.5*(A2)+0.5*(A4)$) [right] {$\scriptstyle 5\mathstrut$};
	\node at ($0.65*(A1)+0.35*(A4)+(-0.2,0)$)
		[above right] {$\scriptstyle 3\mathstrut$};
	\node at ($0.65*(A2)+0.35*(A3)+(0.2,0)$)
		[above left] {$\scriptstyle 4\mathstrut$};
}
\def\w{1.5pt}

\begin{scope}[xshift=-5cm]
\coordinate (A1) at (-\l,\l);
\coordinate (A2) at (\l,\l);
\coordinate (A3) at (-\l,-\l);
\coordinate (A4) at (\l,-\l);
\draw[line width=\w] (A1) -- (A4);
\draw[color=white,line width=8pt] (A2) -- (A3);
\draw[color=red,line width=\w] (A1) -- (A2);
\draw[color=red,line width=\w] (A1) -- (A3);
\draw[color=red,line width=\w] (A2) -- (A3);
\draw[line width=\w] (A2) -- (A4);
\draw[line width=\w] (A3) -- (A4);
\kantenlabels
\alleknoten
\end{scope}

\begin{scope}[xshift=0cm]
\coordinate (A1) at (-\l,\l);
\coordinate (A2) at (\l,\l);
\coordinate (A3) at (-\l,-\l);
\coordinate (A4) at (\l,-\l);
\draw[color=red,line width=\w] (A1) -- (A4);
\draw[color=white,line width=8pt] (A2) -- (A3);
\draw[color=red,line width=\w] (A1) -- (A2);
\draw[line width=\w] (A1) -- (A3);
\draw[line width=\w] (A2) -- (A3);
\draw[color=red,line width=\w] (A2) -- (A4);
\draw[line width=\w] (A3) -- (A4);
\kantenlabels
\alleknoten
\end{scope}

\begin{scope}[xshift=5cm]
\coordinate (A1) at (-\l,\l);
\coordinate (A2) at (\l,\l);
\coordinate (A3) at (-\l,-\l);
\coordinate (A4) at (\l,-\l);
\draw[color=red,line width=\w] (A1) -- (A4);
\draw[color=white,line width=8pt] (A2) -- (A3);
\draw[line width=\w] (A1) -- (A2);
\draw[color=red,line width=\w] (A1) -- (A3);
\draw[line width=\w] (A2) -- (A3);
\draw[line width=\w] (A2) -- (A4);
\draw[color=red,line width=\w] (A3) -- (A4);
\kantenlabels
\alleknoten
\end{scope}

\end{tikzpicture}
\end{center}
\end{loesung}
