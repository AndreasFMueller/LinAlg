Finden Sie für das Netzwerk
\begin{center}
\includeagraphics[width=0.3\hsize]{10000026-1.pdf}
\end{center}
eine minimale Menge von Zyklen.

\begin{hinweis}
Gauss-Calculator: \gaussurl{gausscalc:10000026}
\end{hinweis}

\thema{Zyklen}

\begin{loesung}
\definecolor{greenT}{rgb}{0,0.666,0}
Die Matrix $\partial$ dieses Netzwerkes ist
\[
\partial
=
\begin{pmatrix}
-1&-1&-1& 0& 0& 0\\
 1& 0& 0&-1&-1& 0\\
 0& 1& 0& 1& 0&-1\\
 0& 0& 1& 0& 1& 1
\end{pmatrix}.
\]
Wendet man auf $\partial$ den Gaussalgorithmus an, findet man
\begin{align*}
\begin{tabular}{|>{$}c<{$}>{$}c<{$}>{$}c<{$}>{$}c<{$}>{$}c<{$}>{$}c<{$}|}
\hline
-1&-1&-1& 0& 0& 0\\
 1& 0& 0&-1&-1& 0\\
 0& 1& 0& 1& 0&-1\\
 0& 0& 1& 0& 1& 1\\
\hline
\end{tabular}
&\rightarrow
\begin{tabular}{|>{$}c<{$}>{$}c<{$}>{$}c<{$}>{$}c<{$}>{$}c<{$}>{$}c<{$}|}
\hline
 1& 1& 1& 0& 0& 0\\
 0&-1&-1&-1&-1& 0\\
 0& 1& 0& 1& 0&-1\\
 0& 0& 1& 0& 1& 1\\
\hline
\end{tabular}
\\
&\rightarrow
\begin{tabular}{|>{$}c<{$}>{$}c<{$}>{$}c<{$}>{$}c<{$}>{$}c<{$}>{$}c<{$}|}
\hline
 1& 1& 1& 0& 0& 0\\
 0& 1& 1& 1& 1& 0\\
 0& 0&-1& 0&-1&-1\\
 0& 0& 1& 0& 1& 1\\
\hline
\end{tabular}
\\
&\rightarrow
\begin{tabular}{|>{$}c<{$}>{$}c<{$}>{$}c<{$}>{$}c<{$}>{$}c<{$}>{$}c<{$}|}
\hline
 1& 1& 1& 0& 0& 0\\
 0& 1& 1& 1& 1& 0\\
 0& 0& 1& 0& 1& 1\\
 0& 0& 0& 0& 0& 0\\
\hline
\end{tabular}
\\
&\rightarrow
\begin{tabular}{|>{$}c<{$}>{$}c<{$}>{$}c<{$}>{$}c<{$}>{$}c<{$}>{$}c<{$}|}
\hline
 1& 1& 0& 0&-1&-1\\
 0& 1& 0& 1& 0&-1\\
 0& 0& 1& 0& 1& 1\\
 0& 0& 0& 0& 0& 0\\
\hline
\end{tabular}
\\
&\rightarrow
\begin{tabular}{|>{$}c<{$}>{$}c<{$}>{$}c<{$}>{$}c<{$}>{$}c<{$}>{$}c<{$}|}
\hline
x_1&x_2&x_3&x_4&x_5&x_6\\
\hline
 1& 0& 0&-1&-1& 0\\
 0& 1& 0& 1& 0&-1\\
 0& 0& 1& 0& 1& 1\\
 0& 0& 0& 0& 0& 0\\
\hline
  &  &  &\color{greenT}*&\color{greenT}*&\color{greenT}*\\
\hline
\end{tabular}
\end{align*}
Die frei wählbaren Variablen entsprechen den Kanten 4, 5 und 6, man kann
sie auf $1$ oder $-1$ setzen, je nachdem in welcher Richtung die Kante
im entsprechenden Zyklus durchlaufen werden soll.
Daraus liest man ab, dass drei Zyklen gebildet werden, die dadurch
bestimmt sind, ob man die Kanten 4, 5 oder 6 im Zyklus drin haben will
oder nicht. Die drei Zyklen haben die Kantenvektoren
\[
z_1=\begin{pmatrix}
 1\\
-1\\
 0\\
\color{greenT} 1\\
\color{greenT} 0\\
\color{greenT} 0\\
\end{pmatrix},\quad
z_2=\begin{pmatrix}
 1\\
 0\\
-1\\
\color{greenT} 0\\
\color{greenT} 1\\
\color{greenT} 0
\end{pmatrix},\quad
z_3= \begin{pmatrix}
 0\\
 1\\
-1\\
\color{greenT} 0\\
\color{greenT} 0\\
\color{greenT} 1
\end{pmatrix}
\]
Die folgende Abbildung zeigt die drei Zyklen:
\begin{center}
\begin{tabular}{ccc}
\includeagraphics[width=0.2\hsize]{10000026-2.pdf}&
\includeagraphics[width=0.2\hsize]{10000026-3.pdf}&
\includeagraphics[width=0.2\hsize]{10000026-4.pdf}
\end{tabular}
\end{center}
\end{loesung}
