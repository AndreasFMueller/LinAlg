Betrachten Sie das Gleichungssystem
\[
\begin{linsys}{3}
x&+&2y&+&4z&=&11\\
x&+&4y&+&3z&=&14\\
x&+&3y&+&4z&=&13
\end{linsys}
\]
\begin{teilaufgaben}
\item Berechnen Sie die Inverse der Koeffizientenmatrix.
\item Finden Sie die Lösung des Gleichungssystems mit Hilfe der Inversen.
\end{teilaufgaben}

\thema{inverse Matrix}
\thema{Gauss-Algorithmus}

\begin{loesung}
Wir schreiben $A$ für die Koeffizienten-Matrix:
\[
\begin{pmatrix}
   1&  2&  4\\
   1&  4&  3\\
   1&  3&  4
\end{pmatrix}
\]
\begin{teilaufgaben}
\item
Die Inverse kann man mit Hilfe des Gauss-Algorithmus berechnen:
\begin{align*}
\begin{tabular}{|>{$}c<{$}|>{$}c<{$}|}
\hline
A&E\\
\hline
\end{tabular}
&=
\begin{tabular}{|>{$}c<{$}>{$}c<{$}>{$}c<{$}|>{$}c<{$}>{$}c<{$}>{$}c<{$}|}
\hline
   1&  2&  4&  1&  0&  0\\
   1&  4&  3&  0&  1&  0\\
   1&  3&  4&  0&  0&  1\\
\hline
\end{tabular}
\rightarrow
\begin{tabular}{|>{$}c<{$}>{$}c<{$}>{$}c<{$}|>{$}c<{$}>{$}c<{$}>{$}c<{$}|}
\hline
   1&  2&  4&  1&  0&  0\\
   0&  2& -1& -1&  1&  0\\
   0&  1&  0& -1&  0&  1\\
\hline
\end{tabular}
\\
&\rightarrow
\begin{tabular}{|>{$}c<{$}>{$}c<{$}>{$}c<{$}|>{$}c<{$}>{$}c<{$}>{$}c<{$}|}
\hline
   1&  2&  4&  1&  0&  0\\
   0&  1& -\frac12& -\frac12&  \frac12&  0\\
   0&  0& \frac12& -\frac12& -\frac12&  1\\
\hline
\end{tabular}
\rightarrow
\begin{tabular}{|>{$}c<{$}>{$}c<{$}>{$}c<{$}|>{$}c<{$}>{$}c<{$}>{$}c<{$}|}
\hline
   1&  2&  0&  5&  4& -8\\
   0&  1&  0& -1&  0&  1\\
   0&  0&  1& -1& -1&  2\\
\hline
\end{tabular}
\\
&\rightarrow
\begin{tabular}{|>{$}c<{$}>{$}c<{$}>{$}c<{$}|>{$}c<{$}>{$}c<{$}>{$}c<{$}|}
\hline
   1&  2&  0&  7&  4&-10\\
   0&  1&  0& -1&  0&  1\\
   0&  0&  1& -1& -1&  2\\
\hline
\end{tabular}
=\begin{tabular}{|>{$}c<{$}|>{$}c<{$}|}
\hline
E&A^{-1}\\
\hline
\end{tabular}
\end{align*}
Daraus kann jetzt die Inverse Matrix abgelesen werden:
\[
A^{-1}
=
\begin{pmatrix}
 7& 4&-10\\
-1& 0&  1\\
-1&-1&  2
\end{pmatrix}
\]
Wir kontrollieren das Resultat durch Ausmultiplizieren:
\begin{align*}
A\cdot A^{-1}
&=
\begin{pmatrix}
   1&  2&  4\\
   1&  4&  3\\
   1&  3&  4
\end{pmatrix}
\begin{pmatrix}
 7& 4&-10\\
-1& 0&  1\\
-1&-1&  2
\end{pmatrix}
\\
&=
\begin{pmatrix}
1\cdot 7 + 2\cdot (-1)+4\cdot (-1)&1\cdot4 +2\cdot0 +4\cdot(-1)&1\cdot(-10) +2\cdot1 +4\cdot1\\
1\cdot7 +4\cdot(-1) +3\cdot(-1)&1\cdot4 +4\cdot0 +3\cdot(-1)&1\cdot(-10) +4\cdot1 +3\cdot2\\
1\cdot7 +3\cdot(-1) +4\cdot(-1)&1\cdot4 +3\cdot0 +4\cdot(-1)&1\cdot(-10) +3\cdot1 +4\cdot2
\end{pmatrix}
\\
&=
\begin{pmatrix}
7-2-4&4+0-4&-10+2+8\\
7-4-3&4+0-3&-10+4+6\\
7-3-4&4+0-4&-10+3+8
\end{pmatrix}
=\begin{pmatrix}
1&0&0\\
0&1&0\\
0&0&1
\end{pmatrix}=E.
\end{align*}
\item Die Lösung des Gleichungssystems kann jetzt mit $x=A^{-1}b$ gefunden
werden:
\begin{align*}
b&=
\begin{pmatrix}11\\14\\13\end{pmatrix}
&
x&=A^{-1}b=
\begin{pmatrix}
 7& 4&-10\\
-1& 0&  1\\
-1&-1&  2
\end{pmatrix}
\begin{pmatrix}11\\14\\13\end{pmatrix}
=
\begin{pmatrix}
77+56-130\\
-11+13\\
-11-14+26
\end{pmatrix}
=
\begin{pmatrix}
3\\
2\\
1
\end{pmatrix}.
\qedhere
\end{align*}
\end{teilaufgaben}
\end{loesung}

\begin{bewertung}
\begin{teilaufgaben}
\item
Ansatz (\textbf{A}) 1 Punkt,
Durchführung des Gauss Algorithmus (\textbf{G}) 2 Punkte,
Inverse Matrix (\textbf{I}) 1 Punkt,
\item
Lösungsansatz $x=A^{-1}b$ (\textbf{B}) 1 Punkt,
korrekte Lösung (\textbf{L}) 1 Punkt.
\end{teilaufgaben}
\end{bewertung}

