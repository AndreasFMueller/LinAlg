Betrachten Sie das Gleichungssystem 
mit Koeffizientenmatrix
\[
A
=
\begin{pmatrix*}[r]
2& -14&  -2\\
1& a-6&-a-2\\
3& -20&   1
\end{pmatrix*}
\qquad
\text{und rechter Seite}
\qquad
b
=
\begin{pmatrix*}[c]
2c+12\\
c-2a+4\\
3c+21
\end{pmatrix*}.
\]
Es hängt von den zwei Parametern $a$ und $c$ ab.
\begin{teilaufgaben}
\item
Für welche Werte von $a$ ist die Lösung des Gleichungssystems $Ax=b$
nicht eindeutig bestimmt?
\item
Welchen Rang hat die die Matrix $A$ im Falle der in a) gefundenen
Parameterwerte?
Bestimmen Sie auch die Lösungsmenge.
\item 
Für andere Werte von $a$ ist die Lösung eindeutig bestimmt,
finden Sie die Lösung.
\end{teilaufgaben}

\begin{loesung}
\begin{teilaufgaben}
\item
Wir wenden den Gauss-Algorithmus an:
\begin{align}
\begin{tabular}{|>{$}c<{$}>{$}c<{$}>{$}c<{$}|>{$}c<{$}|}
\hline
\pivotoperationcircle{0.42}{-1.3mm}{-1.0mm}
2& -14&  -2&  2c+12\\
\forwardreduction{0.42}{0.98}{-1.3mm}{-6.2mm}
1& a-6&-a-2& c-2a+4\\
3& -20&   1&  3b+21\\
\hline
\end{tabular}
&\to
\begin{tabular}{|>{$}c<{$}>{$}c<{$}>{$}c<{$}|>{$}c<{$}|}
\hline
1& -7&  -1&  c+6\\
0&
\pivotoperation{1.1}{0.42}{-1.3mm}{-1.0mm}
   a+1&-a-1& -2a-2\\
0&
\forwardreduction{0.42}{0.48}{-1.3mm}{-1.3mm}
     1&   4&      3\\
\hline
\end{tabular}
\label{10000075:regtableau}
\intertext{Im Fall $a+1=0$ entsteht auf der zweiten Zeile eine Nullzeile,
das Gleichungssystem ist in diesem Fall singulär, das Tableau wird
nach Vertauschung der zweiten und dritten Zeile zu}
\to
\begin{tabular}{|>{$}c<{$}>{$}c<{$}>{$}c<{$}|>{$}c<{$}|}
\hline
1&
\backwardsubstitution{0.68}{0.44}{-1.2mm}{-1.2mm}
   {-7}&  -1&  c+6\\
0&   1 &   4&    3\\
0&   0 &   0&    0\\
\hline
\end{tabular}
&\to
\begin{tabular}{|>{$}c<{$}>{$}c<{$}>{$}c<{$}|>{$}c<{$}|}
\hline
1&   0&  27& c+27\\
0&   1&   4&    3\\
0&   0&   0&    0\\
\hline
\end{tabular}.
\label{10000075:schlusstableau}
\end{align}
Die Lösung ist also für $a= -1$  nicht eindeutig bestimmt.
\item
Im Fall $a=-1$ gibt es eine frei wählbare Variable, somit ist 
der Rang $\rank A = 2$.
Aus dem Schlusstableau~\eqref{10000075:schlusstableau}
kann man die Lösungsmenge
\[
\mathbb{L}
=
\left\{
\left.
\begin{pmatrix*}[r]
c+27\\
   3\\
   0
\end{pmatrix*}
+z
\begin{pmatrix*}[r]
-27\\
 -4\\
  1
\end{pmatrix*}
\right|
z\in\mathbb{R}
\right\}
\]
bestimmen.
\item
Für $a\ne -1$ ist das Gleichungssystem regulär und das Element
$a+1$ in der zweiten Zeile des rechten Tableaus in
\eqref{10000075:regtableau} kann als Pivot-Element
verwendet werden.
Der Gauss-Algorithmus lässt sich damit zu Ende führen:
\begin{align*}
\begin{tabular}{|>{$}c<{$}>{$}c<{$}>{$}c<{$}|>{$}c<{$}|}
\hline
1&  -7&
\backwardsubstitution{0.68}{0.88}{-1.2mm}{-5.8mm}
       {-1}& c+6\\
0&   1&  -1&  -2\\
0&   0&
\pivotoperationcircle{0.42}{-1.3mm}{-1.0mm}
          5&   5\\
\hline
\end{tabular}
&\to
\begin{tabular}{|>{$}c<{$}>{$}c<{$}>{$}c<{$}|>{$}c<{$}|}
\hline
1&
\backwardsubstitution{0.68}{0.44}{-1.2mm}{-1.2mm}
  {-7}&   0& c+7\\
0&   1&   0&  -1\\
0&   0&   1&   1\\
\hline
\end{tabular}
\to
\begin{tabular}{|>{$}c<{$}>{$}c<{$}>{$}c<{$}|>{$}c<{$}|}
\hline
1&   0&   0&   c\\
0&   1&   0&  -1\\
0&   0&   1&   1\\
\hline
\end{tabular}.
\end{align*}
Daraus kann man die Lösung
\[
x
=
\begin{pmatrix*}[r]
c\\-1\\1
\end{pmatrix*}
\]
ablesen.
\qedhere
\end{teilaufgaben}
\end{loesung}

\begin{bewertung}
Gauss-Algorithmus ({\bf G}) 2 Punkte,
Regularitätsbedingung $a+1\ne 0$ ({\bf A}) 1 Punkt.
Rang ({\bf R}) 1 Punkt.
Lösungsmenge ({\bf L}) 1 Punkt,
Lösung im regulären Fall ({\bf X}) 1 Punkt.
\end{bewertung}

