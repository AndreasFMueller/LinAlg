Gegeben ist die Matrix
\[
A
=
\begin{pmatrix}1&2\\3&4\end{pmatrix}.
\]
\begin{teilaufgaben}
\item Berechnen Sie $B=\transpose{A}A$.
\item In a) hat sich herausgestellt, dass $\transpose{B}=B$ gilt.
Warum gilt das für jede beliebige Ausgangsmatrix $A$?
\item
Berechnen Sie $AB$.
\item
Finden Sie $B(\transpose{A})$, ohne erneut ein Matrizenprodukt auszurechnen.
\end{teilaufgaben}

\thema{Matrizenprodukt}

\begin{loesung}
\begin{teilaufgaben}
\item $B$ ist Die Matrix
\[
B=\transpose{A}A=
\begin{pmatrix}1&3\\2&4\end{pmatrix}
\begin{pmatrix}1&2\\3&4\end{pmatrix}
=
\begin{pmatrix}
10&14\\
14&20
\end{pmatrix}.
\]
\item Die Formel für die Transposition eines Produktes ergibt
\[
\transpose{B}
=
\transpose{(\transpose{A}A)}
=
\transpose{A}\transpose{(\transpose{A})}
=
\transpose{A}A
=
B.
\]
\item
Das Produkt ist
\[
AB
=
\begin{pmatrix}1&2\\3&4\end{pmatrix}
\begin{pmatrix}
10&14\\
14&20
\end{pmatrix}
=
\begin{pmatrix}
38& 54\\
86&122
\end{pmatrix}.
\]
\item
Die Formel für die Transposition eines Produktes liefert
\[
\transpose{(AB)} = \transpose{B}\transpose{A}=B\transpose{A},
\]
gefragt ist also nach der Transponierten der Matrix, die in c) berechnet
worden war.
Also ist
\[
B(\transpose{A})
= 
\transpose{
\begin{pmatrix}
38& 54\\
86&122
\end{pmatrix}
}
=
\begin{pmatrix}
38& 86\\
54&122
\end{pmatrix}.
\qedhere
\]
\end{teilaufgaben}
\end{loesung}


