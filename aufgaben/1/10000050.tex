Gegeben ist die Matrix
\[
A
=
\begin{pmatrix}1&2\\3&4\end{pmatrix}.
\]
\begin{teilaufgaben}
\item Berechnen Sie $B=A^tA$.
\item In a) hat sich herausgestellt, dass $B^t=B$ gilt.
Warum gilt das für jede beliebige Ausgangsmatrix $A$?
\item
Berechnen Sie $AB$.
\item
Finden Sie $BA^t$, ohne erneut ein Matrizenprodukt auszurechnen.
\end{teilaufgaben}

\thema{Matrizenprodukt}

\begin{loesung}
\begin{teilaufgaben}
\item $B$ ist Die Matrix
\[
B=A^tA=
\begin{pmatrix}1&3\\2&4\end{pmatrix}
\begin{pmatrix}1&2\\3&4\end{pmatrix}
=
\begin{pmatrix}
10&14\\
14&20
\end{pmatrix}.
\]
\item Die Formel für die Transposition eines Produktes ergibt
\[
B^t
=
(A^tA)^t
=
A^t(A^t)^t
=
A^tA
=
B.
\]
\item
Das Produkt ist
\[
AB
=
\begin{pmatrix}1&2\\3&4\end{pmatrix}
\begin{pmatrix}
10&14\\
14&20
\end{pmatrix}
=
\begin{pmatrix}
38& 54\\
86&122
\end{pmatrix}.
\]
\item
Die Formel für die Transposition eines Produktes liefert
\[
(AB)^t = B^tA^t=BA^t,
\]
gefragt ist also nach der Transponierten der Matrix, die in c) berechnet
worden war.
Also ist
\[
BA^t
= 
\begin{pmatrix}
38& 54\\
86&122
\end{pmatrix}^t
=
\begin{pmatrix}
38& 86\\
54&122
\end{pmatrix}.
\qedhere
\]
\end{teilaufgaben}
\end{loesung}


