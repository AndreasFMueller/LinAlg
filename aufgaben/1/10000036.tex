In seinem Video {\em Morley's Miracle}\footnote{Youtube: \url{https://www.youtube.com/watch?v=gjhmh3yWiTI}}
zeigt der mathematische Video-Blogger
{\em Mathologer} den von John Conway erdachten Beweis des Satzes von Morley:
Teilt man in einem beliebigen Dreieck die Winkel mit Geraden durch drei, dann
schneiden sich diese Geraden in den Ecken eines gleichseitigen Dreiecks
(in der Zeichnung rot):
\begin{center}
\includeagraphics[]{morley-1.pdf}
\end{center}
Beim Timestamp 3:30 erklärt Mathologer, dass man für die Winkel
$\delta_i,1\le i\le 6$, er nennt sie $a$, $b$, $c$, $d$, $e$, $f$, ein
lineares Gleichungssystem aufstellen kann. 
Dazu verwendet er einerseits die Winkelsumme in den Dreiecken
$\triangle_A=\triangle AGF$,
$\triangle_B=\triangle BEG$ und $\triangle_C=\triangle CFE$
und andererseits die Tatsache, dass sich die
Winkel an den Punkten $E$, $F$ und $G$ zu $180^\circ$ summieren müssen.
die Winkel $\angle AGB$, $\angle BEC$ und $\angle CFA$ kann man aus
der Winkelsumme der zugehörigen Dreiecke bestimmen.
\begin{teilaufgaben}
\item
Ist das so gefundene Gleichungssystem regulär?
\item
Bestimmen Sie den Rang der Koeffizientenmatrix
\end{teilaufgaben}

\thema{Gauss-Algorithmus}
\thema{Rang}

\begin{loesung}
Der Winkel $\angle AGB$ ist $180^\circ-\frac{\alpha}3-\frac{\beta}3$, und
entsprechend für die Winkel bei $E$ und $F$.
Das Gleichungssystem wird damit
\[
\begin{linsys}{9}
\Delta_A:& &\delta_1&+&\delta_2& &        & &        & &        & &        &=&180^\circ-\frac{\alpha}3\\
\Delta_B:& &        & &        & &\delta_3&+&\delta_4& &        & &        &=&180^\circ-\frac{\beta}3\\
\Delta_C:& &        & &        & &        & &        & &\delta_5&+&\delta_6&=&180^\circ - \frac{\gamma}3\\
E:& &\delta_1& &        & &        & &        & &        &+&\delta_6&=&180^\circ \phantom{-\frac{\alpha}3}-\frac{\beta}3-\frac{\gamma}3\\
F:& &        & &\delta_2&+&\delta_3& &        & &        & &        &=&180^\circ -\frac{\alpha}3\phantom{-\frac{\beta}3}-\frac{\gamma}3\\
G:& &        & &        & &        & &\delta_4&+&\delta_5& &        &=&180^\circ -\frac{\alpha}3-\frac{\beta}3\phantom{-\frac{\gamma}3}\\
\end{linsys}
\]
Die rechten Seiten sind nicht weiter wichtig, da wir ja nur die Lösbarkeit
des Gleichungssystems untersuchen müssen.
Die Koeffizientenmatrix des Gleichungssystems ist
\[
\begin{pmatrix}
1&1&0&0&0&0\\
0&0&1&1&0&0\\
0&0&0&0&1&1\\
1&0&0&0&0&1\\
0&1&1&0&0&0\\
0&0&0&1&1&0
\end{pmatrix}
\]
Wir verwenden daher den Gauss-Algorithmus, um den Rang zu bestimmen:
\begin{align*}
\begin{tabular}{|>{$}r<{$}>{$}r<{$}>{$}r<{$}>{$}r<{$}>{$}r<{$}>{$}r<{$}|}
\hline
1&1&0&0&0&0\\
0&0&1&1&0&0\\
0&0&0&0&1&1\\
1&0&0&0&0&1\\
0&1&1&0&0&0\\
0&0&0&1&1&0\\
\hline
\end{tabular}
&
\rightarrow
\begin{tabular}{|>{$}r<{$}>{$}r<{$}>{$}r<{$}>{$}r<{$}>{$}r<{$}>{$}r<{$}|}
\hline
 1& 1& 0& 0& 0& 0\\
 0& 0& 1& 1& 0& 0\\
 0& 0& 0& 0& 1& 1\\
 0&-1& 0& 0& 0& 1\\
 0& 1& 1& 0& 0& 0\\
 0& 0& 0& 1& 1& 0\\
\hline
\end{tabular}
\rightarrow
\begin{tabular}{|>{$}r<{$}>{$}r<{$}>{$}r<{$}>{$}r<{$}>{$}r<{$}>{$}r<{$}|}
\hline
 1& 1& 0& 0& 0& 0\\
 0&-1& 0& 0& 0& 1\\
 0& 0& 1& 1& 0& 0\\
 0& 0& 0& 0& 1& 1\\
 0& 1& 1& 0& 0& 0\\
 0& 0& 0& 1& 1& 0\\
\hline
\end{tabular}
\\
&
\rightarrow
\begin{tabular}{|>{$}r<{$}>{$}r<{$}>{$}r<{$}>{$}r<{$}>{$}r<{$}>{$}r<{$}|}
\hline
 1& 1& 0& 0& 0& 0\\
 0& 1& 0& 0& 0&-1\\
 0& 0& 1& 1& 0& 0\\
 0& 0& 0& 0& 1& 1\\
 0& 0& 1& 0& 0& 1\\
 0& 0& 0& 1& 1& 0\\
\hline
\end{tabular}
\rightarrow
\begin{tabular}{|>{$}r<{$}>{$}r<{$}>{$}r<{$}>{$}r<{$}>{$}r<{$}>{$}r<{$}|}
\hline
 1& 1& 0& 0& 0& 0\\
 0& 1& 0& 0& 0&-1\\
 0& 0& 1& 1& 0& 0\\
 0& 0& 0& 0& 1& 1\\
 0& 0& 0&-1& 0& 1\\
 0& 0& 0& 1& 1& 0\\
\hline
\end{tabular}
\\
&
\rightarrow
\begin{tabular}{|>{$}r<{$}>{$}r<{$}>{$}r<{$}>{$}r<{$}>{$}r<{$}>{$}r<{$}|}
\hline
 1& 1& 0& 0& 0& 0\\
 0& 1& 0& 0& 0&-1\\
 0& 0& 1& 1& 0& 0\\
 0& 0& 0&-1& 0& 1\\
 0& 0& 0& 0& 1& 1\\
 0& 0& 0& 1& 1& 0\\
\hline
\end{tabular}
\rightarrow
\begin{tabular}{|>{$}r<{$}>{$}r<{$}>{$}r<{$}>{$}r<{$}>{$}r<{$}>{$}r<{$}|}
\hline
 1& 1& 0& 0& 0& 0\\
 0& 1& 0& 0& 0&-1\\
 0& 0& 1& 1& 0& 0\\
 0& 0& 0& 1& 0&-1\\
 0& 0& 0& 0& 1& 1\\
 0& 0& 0& 0& 1& 1\\
\hline
\end{tabular}
\\
&
\rightarrow
\begin{tabular}{|>{$}r<{$}>{$}r<{$}>{$}r<{$}>{$}r<{$}>{$}r<{$}>{$}r<{$}|}
\hline
 1& 1& 0& 0& 0& 0\\
 0& 1& 0& 0& 0&-1\\
 0& 0& 1& 1& 0& 0\\
 0& 0& 0& 1& 0&-1\\
 0& 0& 0& 0& 1& 1\\
 0& 0& 0& 0& 0& 0\\
\hline
\end{tabular}
\rightarrow
\begin{tabular}{|>{$}r<{$}>{$}r<{$}>{$}r<{$}>{$}r<{$}>{$}r<{$}>{$}r<{$}|}
\hline
 1& 1& 0& 0& 0& 0\\
 0& 1& 0& 0& 0&-1\\
 0& 0& 1& 0& 0& 1\\
 0& 0& 0& 1& 0&-1\\
 0& 0& 0& 0& 1& 1\\
 0& 0& 0& 0& 0& 0\\
\hline
\end{tabular}
\\
&
\rightarrow
\begin{tabular}{|>{$}r<{$}>{$}r<{$}>{$}r<{$}>{$}r<{$}>{$}r<{$}>{$}r<{$}|}
\hline
 1& 0& 0& 0& 0& 1\\
 0& 1& 0& 0& 0&-1\\
 0& 0& 1& 0& 0& 1\\
 0& 0& 0& 1& 0&-1\\
 0& 0& 0& 0& 1& 1\\
 0& 0& 0& 0& 0& 0\\
\hline
\end{tabular}
\end{align*}
Daraus kann man ablesen, dass der Rang der Matrix $5$ ist, es gibt genau
einen frei wählbaren Winkel.
Insbesonder ist das Gleichungssystem nicht regulär.
\end{loesung}
