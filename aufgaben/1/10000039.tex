Finden Sie die inverse Matrix von
\[
A=\begin{pmatrix}
    1 &  2 &  3 &  0 \\
    4 &  9 & 13 &  2 \\
    2 &  5 &  8 &  4 \\
   -2 & -3 & -3 &  7
\end{pmatrix}
\]

\begin{loesung}
Wir verwenden den Gauss-Algorithmus
\begin{align*}
\begin{tabular}{| >{$}r<{$} >{$}r<{$} >{$}r<{$} >{$}r<{$} | >{$}r<{$} >{$}r<{$} >{$}r<{$} >{$}r<{$} |}
\hline
    1
\begin{picture}(0,0)
\color{red}\put(-3.5,3){\circle{12}}
\end{picture}
      &  2 &  3 &  0 &  1 &  0 &  0 &  0 \\
    4
\begin{picture}(0,0)
\color{blue}\drawline(-16,-30)(-16,10)(3,10)(3,-30)
\end{picture}
      &  9 & 13 &  2 &  0 &  1 &  0 &  0 \\
    2 &  5 &  8 &  4 &  0 &  0 &  1 &  0 \\
   -2 & -3 & -3 &  7 &  0 &  0 &  0 &  1 \\
\hline
\end{tabular}
&\rightarrow
\begin{tabular}{| >{$}r<{$} >{$}r<{$} >{$}r<{$} >{$}r<{$} | >{$}r<{$} >{$}r<{$} >{$}r<{$} >{$}r<{$} |}
\hline
    1 &  2 &  3 &  0 &  1 &  0 &  0 &  0 \\
    0 &  1
\begin{picture}(0,0)
\color{red}\put(-3.5,3){\circle{12}}
\end{picture}
           &  1 &  2 & -4 &  1 &  0 &  0 \\
    0 &  1
\begin{picture}(0,0)
\color{blue}\drawline(-10,-16)(-10,10)(3,10)(3,-16)
\end{picture}
           &  2 &  4 & -2 &  0 &  1 &  0 \\
    0 &  1 &  3 &  7 &  2 &  0 &  0 &  1 \\
\hline
\end{tabular}
\\
\rightarrow
\begin{tabular}{| >{$}r<{$} >{$}r<{$} >{$}r<{$} >{$}r<{$} | >{$}r<{$} >{$}r<{$} >{$}r<{$} >{$}r<{$} |}
\hline
    1 &  2 &  3 &  0 &  1 &  0 &  0 &  0 \\
    0 &  1 &  1 &  2 & -4 &  1 &  0 &  0 \\
    0 &  0 &  1
\begin{picture}(0,0)
\color{red}\put(-3.5,3){\circle{12}}
\end{picture}
                &  2 &  2 & -1 &  1 &  0 \\
    0 &  0 &  2
\begin{picture}(0,0)
\color{blue}\drawline(-10,-2)(-10,10)(3,10)(3,-2)
\end{picture}
                &  5 &  6 & -1 &  0 &  1 \\
\hline
\end{tabular}
&
\rightarrow
\begin{tabular}{| >{$}r<{$} >{$}r<{$} >{$}r<{$} >{$}r<{$} | >{$}r<{$} >{$}r<{$} >{$}r<{$} >{$}r<{$} |}
\hline
    1 &  2 &  3 &  0 &  1 &  0 &  0 &  0 \\
    0 &  1 &  1 &  2 & -4 &  1 &  0 &  0 \\
    0 &  0 &  1 &  2
\begin{picture}(0,0)
\color{blue}\drawline(-10,38)(-10,-2)(3,-2)(3,38)
\end{picture}
                     &  2 & -1 &  1 &  0 \\
    0 &  0 &  0 &  1
\begin{picture}(0,0)
\color{red}\put(-3.5,3){\circle{12}}
\end{picture}
                     &  2 &  1 & -2 &  1 \\
\hline
\end{tabular}
\\
\rightarrow
\begin{tabular}{| >{$}r<{$} >{$}r<{$} >{$}r<{$} >{$}r<{$} | >{$}r<{$} >{$}r<{$} >{$}r<{$} >{$}r<{$} |}
\hline
    1 &  2 &  3 &  0 &  1 &  0 &  0 &  0 \\
    0 &  1 &  1
\begin{picture}(0,0)
\color{blue}\drawline(-10,23)(-10,-2)(3,-2)(3,23)
\end{picture}
                &  0 & -8 & -1 &  4 & -2 \\
    0 &  0 &  1 &  0 & -2 & -3 &  5 & -2 \\
    0 &  0 &  0 &  1 &  2 &  1 & -2 &  1 \\
\hline
\end{tabular}
&
\rightarrow
\begin{tabular}{| >{$}r<{$} >{$}r<{$} >{$}r<{$} >{$}r<{$} | >{$}r<{$} >{$}r<{$} >{$}r<{$} >{$}r<{$} |}
\hline
    1 &  2
\begin{picture}(0,0)
\color{blue}\drawline(-10,9)(-10,-2)(3,-2)(3,9)
\end{picture}
           &  0 &  0 &  7 &  9 &-15 &  6 \\
    0 &  1 &  0 &  0 & -6 &  2 & -1 &  0 \\
    0 &  0 &  1 &  0 & -2 & -3 &  5 & -2 \\
    0 &  0 &  0 &  1 &  2 &  1 & -2 &  1 \\
\hline
\end{tabular}
\\
\rightarrow
\begin{tabular}{| >{$}r<{$} >{$}r<{$} >{$}r<{$} >{$}r<{$} | >{$}r<{$} >{$}r<{$} >{$}r<{$} >{$}r<{$} |}
\hline
    1 &  0 &  0 &  0 & 19 &  5 &-13 &  6 \\
    0 &  1 &  0 &  0 & -6 &  2 & -1 &  0 \\
    0 &  0 &  1 &  0 & -2 & -3 &  5 & -2 \\
    0 &  0 &  0 &  1 &  2 &  1 & -2 &  1 \\
\hline
\end{tabular}
\end{align*}
Kontrolle:
\[
\begin{pmatrix}
    1 &  2 &  3 &  0 \\
    4 &  9 & 13 &  2 \\
    2 &  5 &  8 &  4 \\
   -2 & -3 & -3 &  7 
\end{pmatrix}
\begin{pmatrix}
 19 &  5 &-13 &  6 \\
 -6 &  2 & -1 &  0 \\
 -2 & -3 &  5 & -2 \\
  2 &  1 & -2 &  1 
\end{pmatrix}
=
\begin{pmatrix}
1&0&0&0\\
0&1&0&0\\
0&0&1&0\\
0&0&0&1
\end{pmatrix}.
\]
\end{loesung}

\begin{diskussion}
Mit dem Matlab-/Octave-Programm \texttt{10000039.m} k"onnen beliebig viele
weitere Aufgaben dieser Art generiert werden, die w"ahrend der Berechnung der
Inversen nicht zu Br"uchen f"uhren werden.
\end{diskussion}

