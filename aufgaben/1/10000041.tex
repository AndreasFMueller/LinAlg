Der Youtuber MindYourDecisions stellt in seinem Video
\url{https://www.youtube.com/watch?v=xVL9qbnHYrc} die folgende Aufgabe,
im folgenden Diagramm die Fragezeichen durch Zahlen so zu ersetzen, dass
alle Gleichungen stimmen:
\begin{center}
\begin{tabular}{>{$}c<{$}>{$}c<{$}>{$}c<{$}>{$}c<{$}>{$}c<{$}}
 ?&-&?&=& 9\\
 +& &+& & \\
 ?&-&?&=&14\\
 =& &=& &  \\
12& &2& &
\end{tabular}
\end{center}
\begin{teilaufgaben}
\item
Ist dies möglich?
\item
Kann man eine Bedingung angeben, die die angegebenen Zahlen erfüllen
müssen, damit das Rätsel lösbar wird?
\item
Finden Sie die allgemeine Lösung des Problems für alle Quadrupel
$b_1,\dots,b_4$ von Zahlen, die die in b) gefundene Bedingung erfüllen.
\end{teilaufgaben}

\themaS{Gauss-Algorithmus}
\thema{lineare Abhangigkeit}{lineare Abhängigkeit}

\begin{loesung}
\begin{teilaufgaben}
\item
Wir schreiben $x$, $y$, $z$ und $t$ für die Fragezeichen, und erhalten
\begin{center}
\begin{tabular}{>{$}c<{$}>{$}c<{$}>{$}c<{$}>{$}c<{$}>{$}c<{$}}
 x&-&y&=& 9\\
 +& &+& & \\
 z&-&t&=&14\\
 =& &=& &  \\
12& &2& &
\end{tabular}
\end{center}
Als Gleichungssystem geschrieben
\[
\begin{linsys}{5}
x&-&y& & & & &=& 9\\
 & & & &z&-&t&=&14\\
x& & &+&z& & &=&12\\
 & &y& & &+&t&=& 2
\end{linsys}
\]
lässt sich die Aufgabe mit dem Gauss-Algorithmus lösen:
\begin{align*}
\begin{tabular}{|>{$}c<{$}>{$}c<{$}>{$}c<{$}>{$}c<{$}|>{$}c<{$}|}
\hline
1&-1& 0& 0& 9\\
0& 0& 1&-1&14\\
1& 0& 1& 0&12\\
0& 1& 0& 1& 2\\
\hline
\end{tabular}
&
\rightarrow
\begin{tabular}{|>{$}c<{$}>{$}c<{$}>{$}c<{$}>{$}c<{$}|>{$}c<{$}|}
\hline
1&-1& 0& 0& 9\\
0& 0& 1&-1&14\\
0& 1& 1& 0& 3\\
0& 1& 0& 1& 2\\
\hline
\end{tabular}
\rightarrow
\begin{tabular}{|>{$}c<{$}>{$}c<{$}>{$}c<{$}>{$}c<{$}|>{$}c<{$}|}
\hline
1&-1& 0& 0& 9\\
0& 1& 1& 0& 3\\
0& 0& 1&-1&14\\
0& 0&-1& 1&-1\\
\hline
\end{tabular}
\\
&
\rightarrow
\begin{tabular}{|>{$}c<{$}>{$}c<{$}>{$}c<{$}>{$}c<{$}|>{$}c<{$}|}
\hline
1&-1& 0& 0& 9\\
0& 1& 1& 0& 3\\
0& 0& 1&-1&14\\
0& 0& 0& 0&13\\
\hline
\end{tabular}
\end{align*}
Da in der letzten Zeile auf der rechten Seite eine von $0$ verschiedene
Zahl steht, hat das Gleichungssystem keine Lösung.
\item
Das Gleichungssystem hat genau dann eine Lösung, wenn beim ``Gaussen''
auf der rechten Seite in der letzten Zeile ebenfalls eine 0 entsteht.
Da die Zeilen der Matrix linear abhängig sind, erfüllen die Zeilen
$z_1,\dots,z_4$ eine Gleichung der Form
\[
\lambda_1z_1+ \lambda_2z_2+ \lambda_3z_3+ \lambda_4z_4=0.
\]
Das Gleichungssytem kann nur dann lösbar sein, wenn auch die rechten
Seiten diese Bedingung erfüllen.
Wir müssen also die Koeffizienten $\lambda_i$ finden, auch dies
kann man mit dem Gauss-Algorithmus
\begin{align*}
\begin{tabular}{|>{$}c<{$}>{$}c<{$}>{$}c<{$}>{$}c<{$}|>{$}c<{$}|}
\hline
 1& 0& 1& 0\\
-1& 0& 0& 1\\
 0& 1& 1& 0\\
 0&-1& 0& 1\\
\hline
\end{tabular}
&
\rightarrow
\begin{tabular}{|>{$}c<{$}>{$}c<{$}>{$}c<{$}>{$}c<{$}|>{$}c<{$}|}
\hline
 1& 0& 1& 0\\
 0& 0& 1& 1\\
 0& 1& 1& 0\\
 0&-1& 0& 1\\
\hline
\end{tabular}
\rightarrow
\begin{tabular}{|>{$}c<{$}>{$}c<{$}>{$}c<{$}>{$}c<{$}|>{$}c<{$}|}
\hline
 1& 0& 1& 0\\
 0& 1& 1& 0\\
 0& 0& 1& 1\\
 0& 0& 1& 1\\
\hline
\end{tabular}
\rightarrow
\begin{tabular}{|>{$}c<{$}>{$}c<{$}>{$}c<{$}>{$}c<{$}|>{$}c<{$}|}
\hline
 1& 0& 1& 0\\
 0& 1& 1& 0\\
 0& 0& 1& 1\\
 0& 0& 0& 0\\
\hline
\end{tabular}
\\
&
\rightarrow
\begin{tabular}{|>{$}c<{$}>{$}c<{$}>{$}c<{$}>{$}c<{$}|>{$}c<{$}|}
\hline
 1& 0& 0&-1\\
 0& 1& 0&-1\\
 0& 0& 1& 1\\
 0& 0& 0& 0\\
\hline
\end{tabular}
\end{align*}
Daraus können wir ablesen
\[
\begin{aligned}
\lambda_1&= 1,&
\lambda_2&= 1,&
\lambda_3&=-1,&
\lambda_4&= 1
\end{aligned}
\]
Die rechten Seiten $b_1,\dots,b_4$ müssen also die Gleichung
\[
b_1+b_2-b_3+b_4=0
\]
erfüllen, damit die Gleichung lösbar ist.
Die Zahlen aus der Aufgabe haben dagegen
\[
9+14-12+2=13\ne 0,
\]
was erneut zeigt, dass die ursprüngliche Aufgabe nicht lösbar ist.
Dagegen müsste das Gleichungssystem für die rechten Seiten
\[
\begin{aligned}
b_1&=1
&
b_2&=3
&
b_3&=9
&
b_4&=5
\end{aligned}
\]
lösbar sein.
Tatsächlich liefert der Gauss-Algorithmus die unendlich vielen
Lösungen des Gleichungssystems mit der frei wählbaren Variable $t$:
\begin{align*}
x&=6-t\\
y&=5-t\\
z&=3+t
\end{align*}
\item
Wir lösen das Gleichungssystem mit den rechten Seiten $b_i$:
\begin{align*}
\begin{tabular}{|>{$}c<{$}>{$}c<{$}>{$}c<{$}>{$}c<{$}|>{$}c<{$}|}
\hline
1&-1& 0& 0&b_1\\
0& 0& 1&-1&b_2\\
1& 0& 1& 0&b_3\\
0& 1& 0& 1&b_4\\
\hline
\end{tabular}
&
\rightarrow
\begin{tabular}{|>{$}c<{$}>{$}c<{$}>{$}c<{$}>{$}c<{$}|>{$}c<{$}|}
\hline
1&-1& 0& 0&b_1\\
0& 0& 1&-1&b_2\\
0& 1& 1& 0&b_3-b_1\\
0& 1& 0& 1&b_4\\
\hline
\end{tabular}
\rightarrow
\begin{tabular}{|>{$}c<{$}>{$}c<{$}>{$}c<{$}>{$}c<{$}|>{$}c<{$}|}
\hline
1&-1& 0& 0&b_1\\
0& 1& 1& 0&b_3-b_1\\
0& 0& 1&-1&b_2\\
0& 0&-1& 1&b_4-b_3+b_1\\
\hline
\end{tabular}
\\
&
\rightarrow
\begin{tabular}{|>{$}c<{$}>{$}c<{$}>{$}c<{$}>{$}c<{$}|>{$}c<{$}|}
\hline
1&-1& 0& 0&b_1\\
0& 1& 1& 0&b_3-b_1\\
0& 0& 1&-1&b_2\\
0& 0& 0& 0&b_1+b_2-b_3+b_4=0\\
\hline
\end{tabular}
\\
&\rightarrow
\begin{tabular}{|>{$}c<{$}>{$}c<{$}>{$}c<{$}>{$}c<{$}|>{$}c<{$}|}
\hline
1&-1& 0& 0&b_1\\
0& 1& 0& 1&b_3-b_1-b_2=b_4\\
0& 0& 1&-1&b_2\\
\hline
\end{tabular}
\\
&\rightarrow
\begin{tabular}{|>{$}c<{$}>{$}c<{$}>{$}c<{$}>{$}c<{$}|>{$}c<{$}|}
\hline
1& 0& 0& 1&b_1+b_4\\
0& 1& 0& 1&b_3-b_1-b_2=b_4\\
0& 0& 1&-1&b_2\\
\hline
\end{tabular}
\end{align*}
Daraus liest man ab:
\begin{align*}
x&=b_1+b_4-t\\
y&=b_4-t\\
z&=b_2+t,
\end{align*}
wobei $t$ frei wählbar ist.
Das Beispiel aus Teilaufgabe b) ergibt sich daraus als Spezialfall.
\qedhere
\end{teilaufgaben}
\end{loesung}

