Betrachten Sie das Gleichungssystem mit der Koeffizientenmatrix
\[
A
=
\begin{pmatrix*}
 u &   u v  &   u v   \\
 1 &  v + u & u v + v \\
 1 &  v + 1 & 2 v + u 
\end{pmatrix*}
\quad\text{und der rechten Seite}\quad
b
=
\begin{pmatrix*}
-uv-u\\
-2uv-v+u-1\\
-3v-2u
\end{pmatrix*}.
\]
\begin{teilaufgaben}
\item
Für welche Werte der Parameter $u$ und $v$ ist die Lösung des
Gleichungssystems nicht eindeutig bestimmt?
Bestimmen Sie den Rang der Matrix $A$ in diesen Fällen.
\item
Finden Sie die Lösung des Gleichungssystmes $Ax=b$ in dem Fall, wo
die Lösung eindeutig bestimmt ist.
\item
Bestimmen Sie die LU-Zerlegung von $A$.
\end{teilaufgaben}

\begin{hinweis}
In a) ist die Lösungsmenge nicht verlangt.
\end{hinweis}

\begin{loesung}
\begin{teilaufgaben}
\item
Wir wenden den Gauss-Algorithmus an.
Das erste Pivot-Element ist $u$, für $u=0$ wird die erste Zeile
zu einer Nullzeile.
In diesem Fall wird das Gleichungssystem zu
\begin{align}
\begin{tabular}{|>{$}c<{$}>{$}c<{$}>{$}c<{$}|>{$}c<{$}|}
\hline
\pivotoperationcircle{0.42}{-1.3mm}{-1.0mm}
u& uv  & uv   & -uv-u       \\
\forwardreduction{0.42}{0.98}{-1.3mm}{-6.2mm}
1& u+v & uv+v & -2uv+u-v+1  \\
1& v+1 & u+2v & -2u-3v      \\
\hline
\end{tabular}
&\to
\begin{tabular}{|>{$}c<{$}>{$}c<{$}>{$}c<{$}|>{$}c<{$}|}
\hline
0&  0  &  0 &     0 \\
1&  v  &  v &  -v-1 \\
1& v+1 & 2v & -3v   \\
\hline
\end{tabular}
\notag
\\
\to
\begin{tabular}{|>{$}c<{$}>{$}c<{$}>{$}c<{$}|>{$}c<{$}|}
\hline
\pivotoperationcircle{0.42}{-1.3mm}{-1.0mm}
1&  v  &  v &  -v-1 \\
\forwardreduction{0.42}{0.98}{-1.3mm}{-6.2mm}
1& v+1 & 2v & -3v   \\
0&  0  &  0 &     0 \\
\hline
\end{tabular}
&\to
\begin{tabular}{|>{$}c<{$}>{$}c<{$}>{$}c<{$}|>{$}c<{$}|}
\hline
1&  v &  v &  -v-1 \\
0&  1 &  v & -2v-1 \\
0&  0 &  0 &   0   \\
\hline
\end{tabular}
\label{10000075:singtableau}
\end{align}
Da die Lösungsmenge nicht verlangt war, ist es auch nicht nötig, den
Gauss-Algorithmus ganz zu Ende zu führen (siehe auch die Diskussion 
weiter unten).
Aus dem Tableau \eqref{10000075:singtableau} kann man ablesen,
dass es keine weiteren Nullzeilen mehr geben kann, daher ist
der Rang $\rank A=2$.
\item
Im Fall $u\ne 0$ kann der Gauss-Algorithmus durchgeführt werden:
\begin{align*}
\begin{tabular}{|>{$}c<{$}>{$}c<{$}>{$}c<{$}|>{$}c<{$}|}
\hline
\pivotoperationcircle{0.42}{-1.3mm}{-1.0mm}
u& uv  & uv   & -uv-u       \\
\forwardreduction{0.42}{0.98}{-1.3mm}{-6.2mm}
1& u+v & uv+v & -2uv+u-v-1  \\
1& v+1 & u+2v & -2u-3v      \\
\hline
\end{tabular}
&\to
\begin{tabular}{|>{$}c<{$}>{$}c<{$}>{$}c<{$}|>{$}c<{$}|}
\hline
1&  v  &  v  & - v-1       \\
0&
\pivotoperationcircle{0.42}{-1.3mm}{-1.0mm}
    u  & uv  & -2uv+u    \\
0&
\forwardreduction{0.42}{0.48}{-1.3mm}{-1.3mm}
    1  & u+v & -2u-2v+1      \\
\hline
\end{tabular}
\\
\to
\begin{tabular}{|>{$}c<{$}>{$}c<{$}>{$}c<{$}|>{$}c<{$}|}
\hline
1&  v  &  v  & - v-1     \\
0&  1  &  v  & -2v+1    \\
0&  0  &
\pivotoperationcircle{0.42}{-1.3mm}{-1.0mm}
          u  & -2u        \\
\hline
\end{tabular}
&\to
\begin{tabular}{|>{$}c<{$}>{$}c<{$}>{$}c<{$}|>{$}c<{$}|}
\hline
1&  v  &
\backwardsubstitution{0.42}{0.88}{-1.2mm}{-5.8mm}
          v  &  -v-1 \\
0&  1  &  v  & -2v+1 \\
0&  0  &  1  &    -2 \\
\hline
\end{tabular}
\\
\to
\begin{tabular}{|>{$}c<{$}>{$}c<{$}>{$}c<{$}|>{$}r<{$}|}
\hline
1&
\backwardsubstitution{0.42}{0.44}{-1.2mm}{-1.2mm}
    v  &  0  &  v-1 \\
0&  1  &  0  &  1   \\
0&  0  &  1  & -2   \\
\hline
\end{tabular}
&\to
\begin{tabular}{|>{$}c<{$}>{$}c<{$}>{$}c<{$}|>{$}r<{$}|}
\hline
1&  0  &  0  & -1 \\
0&  1  &  0  &  1 \\
0&  0  &  1  & -2 \\
\hline
\end{tabular}
\qquad\Rightarrow\qquad
x
=
\begin{pmatrix*}[r]
-1\\1\\-2
\end{pmatrix*}.
\end{align*}
\item
Aus der Durchführung des Gauss-Algorithmus im Fall $u\ne 0$ kann man
die LU-Zerlegung ablesen:
\[
L
=
\begin{pmatrix}
u&0&0\\
1&u&0\\
1&1&u
\end{pmatrix}
,
\;
U=
\begin{pmatrix}
1&v&v\\
0&1&v\\
0&0&1
\end{pmatrix}
\qquad\Rightarrow\qquad
LU
=
A.
\qedhere
\]
\end{teilaufgaben}
\end{loesung}

\begin{bewertung}
Gauss-Algorithmus ({\bf G}) 2 Punkte,
Regularitätsbedingung $u\ne 0$ ({\bf U}) 1 Punkt,
Rang ({\bf R}) 1 Punkt,
Lösung im regulären Fall ({\bf X}) 1 Punkt,
LU-Zerlegung ({\bf L}) 1 Punkt.
\end{bewertung}

\begin{diskussion}
Im Falle $u=0$ war es nicht nötig, die Lösungsmenge zu bestimmen
und daher musste auch der Gauss-Algorithmus nicht vollständig durchgeführt
werden.
Hier soll dies nachgeholt werden.
Im Anschluss an \eqref{10000075:singtableau} findet man
\begin{align*}
\begin{tabular}{|>{$}c<{$}>{$}c<{$}>{$}c<{$}|>{$}c<{$}|}
\hline
1&
\backwardsubstitution{0.42}{0.44}{-1.2mm}{-1.3mm}
    v &  v &  -v-1 \\
0&  1 &  v & -2v-1 \\
0&  0 &  0 &   0   \\
\hline
\end{tabular}
\to
\begin{tabular}{|>{$}c<{$}>{$}c<{$}>{$}c<{$}|>{$}c<{$}|}
\hline
1&  0 &  v-v^2 & -1+2v^2 \\
0&  1 &  v     & -2v-1 \\
0&  0 &  0     &   0   \\
\hline
\end{tabular}
\end{align*}
für das Schlusstableau.
Daraus ergibt sich die Lösungsmenge
\[
\mathbb{L}
=
\left\{\left.
\begin{pmatrix}
2v^2-1\\
-2v-1 \\
0
\end{pmatrix}
+
z
\begin{pmatrix}
v^2-v\\-v\\1
\end{pmatrix}
\;
\right|
\;
z\in\mathbb{R}
\right\}.
\]
\end{diskussion}
