Berechnen Sie die Determinante und die inverse Matrix von
\[
A
=
\begin{pmatrix}
1&1&1\\
1&2&2\\
1&2&a+1
\end{pmatrix}.
\]
Für welche Werte von $a$ existiert die inverse Matrix?

\thema{inverse Matrix}
\thema{Gauss-Algorithmus}

\begin{loesung}
Wir verwenden den Gauss-Algorithmus:
\begin{align*}
\begin{tabular}{|>{$}c<{$}>{$}c<{$}>{$}c<{$}|>{$}c<{$}>{$}c<{$}>{$}c<{$}|}
\hline
1& 1&  1&1&0&0\\
1& 2&  2&0&1&0\\
1& 2&a+1&0&0&1\\
\hline
\end{tabular}
&\rightarrow
\begin{tabular}{|>{$}c<{$}>{$}c<{$}>{$}c<{$}|>{$}c<{$}>{$}c<{$}>{$}c<{$}|}
\hline
1& 1&  1&1&0&0\\
0& 1&  1&-1&1&0\\
0& 1&a  &-1&0&1\\
\hline
\end{tabular}
\rightarrow
\begin{tabular}{|>{$}c<{$}>{$}c<{$}>{$}c<{$}|>{$}c<{$}>{$}c<{$}>{$}c<{$}|}
\hline
1& 1&  1& 1& 0&0\\
0& 1&  1&-1& 1&0\\
0& 0&a-1& 0&-1&1\\
\hline
\end{tabular}
\\
\rightarrow
{
\renewcommand{\arraystretch}{1.15}
\begin{tabular}{|>{$}c<{$}>{$}c<{$}>{$}c<{$}|>{$}c<{$}>{$}c<{$}>{$}c<{$}|}
\hline
1& 1& 0& 1&  \frac1{a-1}& -\frac1{a-1}\\
0& 1& 0&-1&\frac{a}{a-1}& -\frac1{a-1}\\
0& 0& 1& 0& -\frac1{a-1}&  \frac1{a-1}\\[2pt]
\hline
\end{tabular}
}
&
\rightarrow
{
\renewcommand{\arraystretch}{1.15}
\begin{tabular}{|>{$}c<{$}>{$}c<{$}>{$}c<{$}|>{$}c<{$}>{$}c<{$}>{$}c<{$}|}
\hline
1& 1& 0& 2&      -1     &            0\\
0& 1& 0&-1&\frac{a}{a-1}& -\frac1{a-1}\\
0& 0& 1& 0& -\frac1{a-1}&  \frac1{a-1}\\[2pt]
\hline
\end{tabular}
}
\end{align*}
Der dritte Schritt ist allerdings nur möglich, wenn 
$a-1\ne 0$, oder $a\ne 1$.
Das bestätigt auch ein Blick auf die ursprüngliche Matrix $A$: wenn $a=1$ ist,
sind die letzten zwei Zeilen gleich, die Matrix ist dann nicht mehr regulär.

Die Determinente ergibt sich jetzt als Produkt der Pivot-Elemente.
Die Pivot-Elemente sind $1$, $1$ und $a-1$, so dass wir schliessen
können, dass $\det A=a-1$.

Die inverse Matrix existert also genau dann, wenn $a\ne 1$ ist und in diesem
Fall gilt
\[
A^{-1}
=
{
\renewcommand{\arraystretch}{1.15}
\begin{pmatrix}
 2&      -1     &            0\\
-1&\frac{a}{a-1}& -\frac1{a-1}\\
 0& -\frac1{a-1}&  \frac1{a-1}\\
\end{pmatrix}
}
=
\frac1{a-1}
\begin{pmatrix}
2(a-1)& -a+1 &  0\\
 -a+1 &  a   & -1\\
    0 &   -1 &  1
\end{pmatrix}.
\]
Kontrolle:
{
\renewcommand{\arraystretch}{1.15}
\begin{align*}
AA^{-1}
&=
\begin{pmatrix}
1&1&1\\
1&2&2\\
1&2&a+1
\end{pmatrix}
\begin{pmatrix}
 2&      -1     &            0\\
-1&\frac{a}{a-1}& -\frac1{a-1}\\
 0& -\frac1{a-1}&  \frac1{a-1}\\
\end{pmatrix}
\\
&=
\begin{pmatrix}
2-1 & -1+\frac{a}{a-1}-\frac{1}{a-1}   & -\frac1{a-1}+\frac1{a-1}     \\
2-2 & -1+2\frac{a}{a-1}-\frac{2}{a-1}  & -\frac{2}{a-1}+\frac{2}{a-1} \\
2-2 & -1+\frac{2a}{a-1}-\frac{a+1}{a-1}& -\frac2{a-1}+\frac{a+1}{a-1}
\end{pmatrix}
=
\begin{pmatrix}
1&0&0\\
0&1&0\\
0&0&1
\end{pmatrix}.
\end{align*}%
}%
Alternativ kann die inverse Matrix auch mit Hilfe von Minoren
berechnet werden.
Dazu braucht man zunächst die Determinante, die man zum Beispiel
durch Zeilenoperationen wie folgt berechnen kann:
\begin{align*}
\det A
&=
\left|\begin{matrix}
1&1&1\\
1&2&2\\
1&2&a+1
\end{matrix}\right|
=
\left|\begin{matrix}
1&1&1\\
0&1&1\\
0&1&a
\end{matrix}\right|
=
\left|\begin{matrix}
1&1&1\\
0&1&1\\
0&0&a-1
\end{matrix}\right|
=
a-1.
\end{align*}
Die inverse Matrix kann also nur dann bestimmt werden, wenn $a\ne 1$.
Sie ist 
\begin{align*}
A^{-1}
&=
\frac1{a-1}
{
\renewcommand{\arraystretch}{1.15}
\begin{pmatrix}
\left|\begin{matrix}2&2\\2&a\end{matrix}\right|
&-\left|\begin{matrix}1&1\\2&a+1\end{matrix}\right|
&\left|\begin{matrix}1&1\\2&2\end{matrix}\right|
\\
-\left|\begin{matrix}1&2\\1&a+1\end{matrix}\right|
&\left|\begin{matrix}1&1\\1&a+1\end{matrix}\right|
&-\left|\begin{matrix}1&1\\1&2\end{matrix}\right|
\\
\left|\begin{matrix}1&2\\1&2\end{matrix}\right|
&-\left|\begin{matrix}1&1\\1&2\end{matrix}\right|
&\left|\begin{matrix}1&1\\1&2\end{matrix}\right|
\end{pmatrix}
}
=
\frac1{a-1}
\begin{pmatrix}
2a-2&-a+1&0\\
-a+1&a  &-1\\
   0& -1&1
\end{pmatrix},
\end{align*}
in Übereinstimmung mit dem Resultat, das mit dem Gauss-Algorithmus
gefundenen wurde.
\end{loesung}

\begin{bewertung}
Berechnungsmethode für die inverse Matrix ({\bf M}) 1 Punkt,
Durchführung des Gauss-Algorithmus ({\bf G}) 2 Punkt,
Bedingung $a\ne 1$ ({\bf B}) 1 Punkt,
Determinante ({\bf D}) 1 Punkt,
inverse Matrix ({\bf I}) 1 Punkt.
\end{bewertung}


