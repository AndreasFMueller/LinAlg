Gegeben ist das folgende Gleichungssystem mit dem Parameter $q$.
\[
\begin{linsys}{3}
-3a&+&12b&+&3c&=&21\\
a&-&7b&-&c&=&-13\\
-a&+&5b&+&(q-1)\,c&=& 9
\end{linsys}
\]
\begin{teilaufgaben}
\item Für welche Werte von $q$ ist das Gleichungssystem singulär? 
Bestimmen Sie zudem, welcher singuläre Fall vorliegt.
\item Bestimmen Sie den Rang der Koeffizientenmatrix für den singulären Fall.
\item Bestimmen Sie die Lösungsmenge des Gleichungssystems für den regulären Fall.
\end{teilaufgaben}

\thema{Rang}
\themaL{singular}{singulär}
\themaL{regular}{regulär}
\thema{Gauss-Algorithmus}
\thema{Matrix mit Parameter} 

\begin{loesung}
\begin{teilaufgaben}
\item 
Man könnte versuchen herauszufinden, für welche Werte von $q$ die Matrix
\[
A=\begin{pmatrix}
-3& 12 &  3\\
1& -7 & -1 \\
-1& 5& q-1
\end{pmatrix}
\]
singulär ist, und dazu die Determinante verwenden.
Da man das Gleichungssystem aber auch lösen muss, ist es effizienter,
gleich von Anfang an den Gauss-Algorithmus anzuwenden.
\begin{align*}
\begin{tabular}{|>{$}c<{$}>{$}c<{$}>{$}c<{$}|>{$}c<{$}|}
\hline
-3 &  12 &  3  & 21 \\
 1 & -7  & -1  & -13\\
-1 &  5  & q-1 & 9 \\
\hline
\end{tabular}
&
\rightarrow
\begin{tabular}{|>{$}c<{$}>{$}c<{$}>{$}c<{$}|>{$}c<{$}|}
\hline
 1 &  -4 &  -1  & -7 \\
 0 & -3  & 0  & -6\\
 0 &  1  & q-2 & 2 \\
\hline
\end{tabular}
\rightarrow
\begin{tabular}{|>{$}c<{$}>{$}c<{$}>{$}c<{$}|>{$}c<{$}|}
\hline
 1 &  -4 &  -1  & -7 \\
 0 &   1 & 0   & 2\\
 0 &   0 & q-2 & 0 \\
\hline
\end{tabular}
\end{align*}
Das Gleichungssystem ist singulär, wenn im Gauss-Algorithmus Nullzeilen auftreten, 
also wenn 
\[
q-2 = 0 \quad \text{bzw.} \quad q=2
\]
 ist. Da auch der zugehörige Eintrag der rechten Seite
verschwindet liegt hier folglich der Fall \textit{unendlich viele Lösungen} vor.
\item
Im singulären Fall ($q=2$) hat das Gleichungssystem zwei linear unabhängige Zielen, womit der Rang also 2 ist.
\item 
Um die Lösungsmenge des Gleichungssystem für den regulären Fall zu bestimmen, nehmen 
wir an, dass $q\neq 2$ ist und führen den Gauss-Algorithmus zu Ende.
\begin{align*}
\begin{tabular}{|>{$}c<{$}>{$}c<{$}>{$}c<{$}|>{$}c<{$}|}
\hline
 1 &  -4 &  -1  & -7 \\
 0 &   1 & 0   & 2\\
 0 &   0 & q-2 & 0 \\
\hline
\end{tabular}
\rightarrow
\begin{tabular}{|>{$}c<{$}>{$}c<{$}>{$}c<{$}|>{$}c<{$}|}
\hline
 1 &  -4 &  0  & -7 \\
 0 &   1 & 0   & 2\\
 0 &   0 & 1 & 0 \\
\hline
\end{tabular}
\rightarrow
\begin{tabular}{|>{$}c<{$}>{$}c<{$}>{$}c<{$}|>{$}c<{$}|}
\hline
 1 &  0 &  0  & 1 \\
 0 &   1 & 0   & 2\\
 0 &   0 & 1 & 0 \\
\hline
\end{tabular}
\end{align*}
Daraus kann jetzt die Lösungsmenge abgelesen werden:
\[
\mathbb L =\left\{
\begin{pmatrix}
a\\
b\\
c 
\end{pmatrix}
=
\begin{pmatrix}
1\\
2\\
0 
\end{pmatrix}
\right\}.
\]
\end{teilaufgaben}
\end{loesung}

\begin{bewertung}
Durchführung der Vorwärtsreduktion des Gauss-Algorithmus ({\bf V}) 2 Punkte,
Bedingung für Singularität inkl. Art der Singularität und Bestimmung des Wertes für $q$ ({\bf Q}) 1 Punkt,
Bestimmung des Ranges ({\bf R}) 1 Punkt,
Durchführung des Rückwärtseinsetzen des Gauss-Algorithmus ({\bf G}) 1 Punkte,
Bestimmung der Lösung ({\bf L}) 1 Punkt,
\end{bewertung}


