Betrachten Sie das Gleichungssystem
\[
\begin{linsys}{3}
2x&+&4y&+&6y&=&16\\
x&+&2ay&+&5z&=&4a+8\\
x&+&(a+1)y&+&5z&=&13
\end{linsys}.
\]
\begin{teilaufgaben}
\item
Für welche Werte von $a$ hat das Gleichungssystem
keine, unendlich viele oder genau eine Lösung?
\item
Geben Sie die Lösung des Gleichungssystems in Abhängigkeit von $a$
in den Fällen, in denen es genau eine Lösung gibt.
\item
Geben Sie eine numerische Lösung für den Fall $a=0$.
\end{teilaufgaben}

\thema{Matrix mit Parameter}
\thema{Gauss-Algorithmus}
\themaL{Losungsmenge}{Lösungsmenge}

\begin{loesung}
\begin{teilaufgaben}
\item
Wir versuchen, das Gleichungssystem soweit möglich mit dem
Gaussalgorithmus zu lösen:
\begin{align*}
\begin{tabular}{|>{$}c<{$}>{$}c<{$}>{$}c<{$}|>{$}c<{$}|}
\hline
2&10&0&16\\
1&2a&5&4a+8\\
1&a+1&5&13\\
\hline
\end{tabular}
&
\rightarrow
\begin{tabular}{|>{$}c<{$}>{$}c<{$}>{$}c<{$}|>{$}c<{$}|}
\hline
1&5&0&8\\
0&2a-5&5&4a\\
0&a-4&5&5\\
\hline
\end{tabular}
\end{align*}
Statt wie im Standardalgorithmus vorgeschrieben weiter zu machen,
subtrahieren wir an dieser Stelle das Doppelte der dritten
Zeile von der zweiten:
\begin{align*}
\begin{tabular}{|>{$}c<{$}>{$}c<{$}>{$}c<{$}|>{$}c<{$}|}
\hline
1&5&0&8\\
0&2a-5&5&4a\\
0&a-4&5&5\\
\hline
\end{tabular}
&
\rightarrow
\begin{tabular}{|>{$}c<{$}>{$}c<{$}>{$}c<{$}|>{$}c<{$}|}
\hline
1&5&0&8\\
0&3&-5&4a-10\\
0&a-4&5&5\\
\hline
\end{tabular}
\\
&
\rightarrow
\begin{tabular}{|>{$}c<{$}>{$}c<{$}>{$}c<{$}|>{$}c<{$}|}
\hline
1&5&0&8\\
0&1&-\frac53&\frac{4a-10}3\\
0&0&5+\frac53(a-4)&5-(a-4)\frac{4a-10}{3}\\
\hline
\end{tabular}
\\
&
=
\begin{tabular}{|>{$}c<{$}>{$}c<{$}>{$}c<{$}|>{$}c<{$}|}
\hline
1&5&0&8\\
0&1&-\frac53&\frac{4a-10}3\\
0&0&\frac{5a-5}3&5-(a-4)\frac{4a-10}{3}\\
\hline
\end{tabular}
\end{align*}
Offenbar kann man den Gauss-Algorithmus also genau dann weiterführen,
wenn das Element $\frac{5(a-1)}3$ nicht verschwindet, also wenn $a\ne 1$.
Falls $a=1$, wird die rechte Seite dieser Gleichung zu $5-(-3)\frac{-6}{3}=
11$, das Gleichungssystem kann dann also gar keine Lösung haben.

Man kann natürlich $a=1$ auch in das ursprüngliche Gleichungssystem
einsetzen, und erhält dann das Tableau
\[
\begin{tabular}{|>{$}c<{$}>{$}c<{$}>{$}c<{$}|>{$}c<{$}|}
\hline
2&10&0&16\\
1&2&5&12\\
1&2&5&13\\
\hline
\end{tabular}
\]
Subtrahiert man die zweite Zeile von der Dritten erhält man
\[
\begin{tabular}{|>{$}c<{$}>{$}c<{$}>{$}c<{$}|>{$}c<{$}|}
\hline
2&10&0&16\\
1&2&5&12\\
0&0&0&1\\
\hline
\end{tabular}
\]
Die letzte Zeile entspricht der Gleichung $0=1$, das Gleichungssystem
kann also keine Lösung haben.

Man kann die Regularität natürlich auch mit der Determinanten
entscheiden, die man zum Beispiel mit Hilfe der Sarrus-Formel
berechnen kann:
\begin{align*}
\left|\;\begin{matrix}
2&10&0\\
1&2a&5\\
1&a+1&5
\end{matrix}\;\right|
&=
2\cdot 2a\cdot 5 + 10\cdot 5\cdot 1+0\cdot 1\cdot(a+1)
\\
&\qquad -1\cdot 2a\cdot 0-(a+1)\cdot5\cdot 2-5\cdot 1\cdot 10
\\
&=
20a+50-10a-10-50
\\
&=
10a-10=10(a-1)
\end{align*}
Die Determinante verschwindet also genau dann, wenn $a=1$ ist.
\item
Für $a=0$ kann man das Gleichungssystem direkt lösen:
\begin{align*}
\begin{tabular}{|>{$}c<{$}>{$}c<{$}>{$}c<{$}|>{$}c<{$}|}
\hline
2&10&0&16\\
1&0&5&8\\
1&1&5&13\\
\hline
\end{tabular}
&\rightarrow
\begin{tabular}{|>{$}c<{$}>{$}c<{$}>{$}c<{$}|>{$}c<{$}|}
\hline
2&10&0&16\\
1&1&5&13\\
1&0&5&8\\
\hline
\end{tabular}
\\
&\rightarrow
\begin{tabular}{|>{$}c<{$}>{$}c<{$}>{$}c<{$}|>{$}c<{$}|}
\hline
1&5&0&8\\
0&1&0&5\\
1&0&5&8\\
\hline
\end{tabular}
\\
&\rightarrow
\begin{tabular}{|>{$}c<{$}>{$}c<{$}>{$}c<{$}|>{$}c<{$}|}
\hline
1&5&0&8\\
0&1&0&5\\
0&-5&5&0\\
\hline
\end{tabular}
\\
&\rightarrow
\begin{tabular}{|>{$}c<{$}>{$}c<{$}>{$}c<{$}|>{$}c<{$}|}
\hline
1&5&0&8\\
0&1&0&5\\
0&0&5&25\\
\hline
\end{tabular}
\\
&\rightarrow
\begin{tabular}{|>{$}c<{$}>{$}c<{$}>{$}c<{$}|>{$}c<{$}|}
\hline
1&5&0&8\\
0&1&0&5\\
0&0&1&5\\
\hline
\end{tabular}
\\
&\rightarrow
\begin{tabular}{|>{$}c<{$}>{$}c<{$}>{$}c<{$}|>{$}c<{$}|}
\hline
1&0&0&-17\\
0&1&0&5\\
0&0&1&5\\
\hline
\end{tabular}
\end{align*}
\end{teilaufgaben}
\end{loesung}

\begin{diskussion}
Eigentlich steckte ein Druckfehler in der Aufgabe, gemeint war
eigentlich das Gleichungssystem
\[
\begin{linsys}{3}
2x&+&4y&+&6z&=&16\\
x&+&2ay&+&5z&=&4a+8\\
x&+&(a+1)y&+&5z&=&13
\end{linsys}.
\]
Für dieses Gleichungssystem kann man die Lösung wie folgt durchführen.
\begin{teilaufgaben}
\item
Wir lösen das Gleichungssystem mit Hilfe des Gauss-Algorithmus:
\begin{align*}
\begin{tabular}{|>{$}c<{$}>{$}c<{$}>{$}c<{$}|>{$}c<{$}|}
\hline
2& 4&6&16\\
1&2a&5&4a+8\\
1&a+1&5&13\\
\hline
\end{tabular}
&
\rightarrow
\begin{tabular}{|>{$}c<{$}>{$}c<{$}>{$}c<{$}|>{$}c<{$}|}
\hline
1& 2&3&8\\
0&2a-2&2&4a\\
0&a-1&2&5\\
\hline
\end{tabular}
\end{align*}
Bereits jetzt kann man ablesen, dass das System singulär ist
genau für $a=1$. Wir diskutieren also zunächst diesen Fall,
bevor wir das System fertig lösen. Setzt man $a=1$, wird das
Gleichungssystem zu
\begin{align*}
\rightarrow
\begin{tabular}{|>{$}c<{$}>{$}c<{$}>{$}c<{$}|>{$}c<{$}|}
\hline
1& 2&3&8\\
0&0&2&4\\
0&0&2&5\\
\hline
\end{tabular}
&
\rightarrow
\begin{tabular}{|>{$}c<{$}>{$}c<{$}>{$}c<{$}|>{$}c<{$}|}
\hline
1& 2&3&8\\
0&0&1&2\\
0&0&0&1\\
\hline
\end{tabular}
\end{align*}
In der letzten Zeile ist eine Nullzeile entstanden, aber die
rechte Seite ist nicht 0. In diesem Fall hat das Gleichungssystem
also keine Lösung.

In allen anderen Fällen, also für $a\ne 1$ ist das Gleichungssystem
regulär, hat also genau eine Lösung.
\item
Wir brauchen nur noch den Fall $a\ne 1$ zu betrachten. In diesem
Fall darf man durch $a-1$ dividieren, und kann so den Gauss-Algorithmus
zu Ende führen.
\begin{align*}
\rightarrow
\begin{tabular}{|>{$}c<{$}>{$}c<{$}>{$}c<{$}|>{$}c<{$}|}
\hline
1& 2&3&8\\
0&1&\frac{1}{a-1}&\frac{2a}{a-1}\\
0&0&1&5-2a\\
\hline
\end{tabular}
&
\rightarrow
\begin{tabular}{|>{$}c<{$}>{$}c<{$}>{$}c<{$}|>{$}c<{$}|}
\hline
1&2&0&6a-7\\
0&1&0&\frac{2a}{a-1}-\frac{5-2a}{a-1}\\
0&0&1&5-2a\\
\hline
\end{tabular}
=
\begin{tabular}{|>{$}c<{$}>{$}c<{$}>{$}c<{$}|>{$}c<{$}|}
\hline
1&2&0&6a-7\\
0&1&0&\frac{4a-5}{a-1}\\
0&0&1&5-2a\\
\hline
\end{tabular}
\\
&
\rightarrow
\begin{tabular}{|>{$}c<{$}>{$}c<{$}>{$}c<{$}|>{$}c<{$}|}
\hline
1&2&0&6a-7-2\frac{4a-5}{a-1}\\
0&1&0&\frac{4a-5}{a-1}\\
0&0&1&5-2a\\
\hline
\end{tabular}
=
\begin{tabular}{|>{$}c<{$}>{$}c<{$}>{$}c<{$}|>{$}c<{$}|}
\hline
1&2&0&\frac{6a^2-21a+17}{a-1}\\
0&1&0&\frac{4a-5}{a-1}\\
0&0&1&5-2a\\
\hline
\end{tabular}
\end{align*}
wegen
\begin{align*}
6a-7-2\frac{4a-5}{a-1}
&=
\frac{
(6a-7)(a-1)-2(4a-5)
}{a-1}
\\
&
=
\frac{
6a^2-13a+7-8a+10
}{a-1}
=
\frac{6a^2-21a+17}{a-1}
\end{align*}
Die Lösung ist also
\[
x=\frac{6a^2-21a+17}{a-1}
,\qquad
y=\frac{4a-5}{a-1}
,\qquad
z=5-2a.
\]
\item Setzen wir in obigen Formeln $a=0$, finden wir
\[
x=-17
,\qquad
y=5
,\qquad
z=5
\]
\item Im Spezialfall $a=0$ wird die Folge der Gauss-Tableau zu
\begin{align*}
\begin{tabular}{|>{$}c<{$}>{$}c<{$}>{$}c<{$}|>{$}c<{$}|}
\hline
2&4&6&16\\
1&0&5& 8\\
1&1&5&13\\
\hline
\end{tabular}
&\rightarrow
\begin{tabular}{|>{$}c<{$}>{$}c<{$}>{$}c<{$}|>{$}c<{$}|}
\hline
1& 2& 3& 8\\
0&-2& 2& 0\\
0&-1& 2& 5\\
\hline
\end{tabular}
%\\
%&
\rightarrow
\begin{tabular}{|>{$}c<{$}>{$}c<{$}>{$}c<{$}|>{$}c<{$}|}
\hline
1& 2& 3& 8\\
0& 1&-1& 0\\
0& 0& 1& 5\\
\hline
\end{tabular}
\\
&\rightarrow
\begin{tabular}{|>{$}c<{$}>{$}c<{$}>{$}c<{$}|>{$}c<{$}|}
\hline
1& 2& 0&-7\\
0& 1& 0& 5\\
0& 0& 1& 5\\
\hline
\end{tabular}
%\\
%&
\rightarrow
\begin{tabular}{|>{$}c<{$}>{$}c<{$}>{$}c<{$}|>{$}c<{$}|}
\hline
1& 0& 0&-17\\
0& 1& 0&  5\\
0& 0& 1&  5\\
\hline
\end{tabular}
\end{align*}
Die eindeutige Lösung ist also $x=-17$, $y=5$ und $z=5$,
wie man durch Einsetzen nachprüfen kann.
\qedhere
\end{teilaufgaben}
\end{diskussion}
