Berechnen Sie die Determinante und die inverse Matrix von
\[
A
=
\begin{pmatrix}
2&4&6\\
1&a&3a-3\\
1&4&12
\end{pmatrix}.
\]
Für welche Werte von $a$ existiert die inverse Matrix?

\themaS{inverse Matrix}
\themaS{Gauss-Algorithmus}

\begin{loesung}
Wir verwenden den Gauss-Algorithmus:
\begin{align*}
\begin{tabular}{|
>{$}c<{$}
>{$}c<{$}
>{$}c<{$}|
>{$}c<{$}
>{$}c<{$}
>{$}c<{$}|}
\hline
2&4&6   &1&0&0\\
1&a&3a-3&0&1&0\\
1&4&12  &0&0&1\\
\hline
\end{tabular}
&\rightarrow
\begin{tabular}{|>{$}c<{$}>{$}c<{$}>{$}c<{$}|>{$}c<{$}>{$}c<{$}>{$}c<{$}|}
\hline
1&  2&3   & \frac12& 0& 0\\
0&a-2&3a-6&-\frac12& 1& 0\\
0&  2&9   &-\frac12& 0& 1\\
\hline
\end{tabular}
\\
\intertext{
In der zweiten Zeile entsteht eine Nullzeile, wenn $a=2$ ist, dann existiert
die inverse Matrix nicht.
Wir nehmen daher für die weitere Rechnung an, dass $a\ne 0$.
}
\rightarrow
\begin{tabular}{|>{$}c<{$}>{$}c<{$}>{$}c<{$}|>{$}c<{$}>{$}c<{$}>{$}c<{$}|}
\hline
1&  2&3   & \frac12         &              0& 0\\
0&  1&3   &-\frac1{2(a-2)}  &    \frac1{a-2}& 0\\
0&  0&3   &-\frac{a-4}{2a-4}& -\frac{2}{a-2}& 1\\
\hline
\end{tabular}
&\rightarrow
\begin{tabular}{|>{$}c<{$}>{$}c<{$}>{$}c<{$}|>{$}c<{$}>{$}c<{$}>{$}c<{$}|}
\hline
1&  2&0   & \frac{a-3}{a-2}    & \frac{2}{a-2}       & -1     \\
0&  1&0   & \frac{a-5}{2(a-2)} &    \frac3{a-2}      & -1     \\
0&  0&1   &-\frac{a-4}{2a-4}   & -\frac{a-4}{6(a-2)} & \frac13\\
\hline
\end{tabular}
\\
\rightarrow
\begin{tabular}{|>{$}c<{$}>{$}c<{$}>{$}c<{$}|>{$}c<{$}>{$}c<{$}>{$}c<{$}|}
\hline
1&  0&0   & \frac{2}{a-2}      & -\frac{4}{a-2}      &  1     \\
0&  1&0   & \frac{a-5}{2(a-2)} &    \frac3{a-2}      & -1     \\
0&  0&1   &-\frac{a-4}{6(a-2)} & -\frac{2}{3(a-2)}   & \frac13\\
\hline
\end{tabular}
\end{align*}
Die Pivot-Elemente waren $2$, $(a-2)$ und $3$, also ist die Determinante
\[
\det A = 6(a-2).
\]
Die inverse Matrix ist
\[
A^{-1}
=
\begin{pmatrix}
\frac{2}{a-2}      & -\frac{4}{a-2}  &  1       \\
\frac{a-5}{2a-4}   &  \frac{3}{a-2}  & -1       \\
-\frac{a-4}{6a-12} & -\frac{2}{3a-6} & \frac13
\end{pmatrix}
\]
Alternativ könnten wir die Inverse Matrix auch mit Hilfe der Adjungierten
berechnen.
\begin{align*}
A^{-1}
&=
\frac1{6(a-2)}
\begin{pmatrix}
   12   &  -24  &   6(a-2) \\
 3(a-5) &   18  &  -6(a-2) \\
-(a-4)  &   -4  &  2(a-2)
\end{pmatrix}
\end{align*}
\end{loesung}

\begin{bewertung}
\end{bewertung}



