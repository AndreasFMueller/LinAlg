Finden sie die Koeffzienten $\lambda_1,\dots,\lambda_3$ einer Linearkombination
der Zeilen der Matrix
\[
A=\begin{pmatrix}
1&2&3\\
2&3&c\\
3&4&5
\end{pmatrix},
\]
die verschwindet.

\begin{loesung}
Die Koeffizienten sind Lösung eines Gleichungssystems mit der
transponierten Matrix $A^t$, also mit dem Tableau
\begin{align*}
\begin{tabular}{|>{$}c<{$}>{$}c<{$}>{$}c<{$}|}
\hline
1&2&3\\
2&3&4\\
3&c&5\\
\hline
\end{tabular}
&\rightarrow
\begin{tabular}{|>{$}c<{$}>{$}c<{$}>{$}c<{$}|}
\hline
1&  2& 3\\
0& -1&-2\\
0&c-6&-4\\
\hline
\end{tabular}
\rightarrow
\begin{tabular}{|>{$}c<{$}>{$}c<{$}>{$}c<{$}|}
\hline
1&  2& 3\\
0&  1& 2\\
0&  0&-4-2(c-6)\\
\hline
\end{tabular}
=
\begin{tabular}{|>{$}c<{$}>{$}c<{$}>{$}c<{$}|}
\hline
1&  2& 3\\
0&  1& 2\\
0&  0&8-2c\\
\hline
\end{tabular}
\rightarrow
\begin{tabular}{|>{$}c<{$}>{$}c<{$}>{$}c<{$}|}
\hline
1&  0&-1\\
0&  1& 2\\
0&  0&8-2c\\
\hline
\end{tabular}
\end{align*}
Damit das Gleichungssystem überhaupt eine Lösung $\ne 0$ hat, muss eine
Nullzeile entstehen, das ist nur möglichm wenn der Eintrag rechts unten
verschwindet, also
\[
8-2c = 0 \qquad\Rightarrow\qquad c=4.
\]
Jetzt müssen aber noch die Koeffizienten $\lambda_i$ gefunden werden.
$\lambda_3$ ist offenbar frei wählbar, wir wählen $\lambda_3=1$.
Dann lesen wird ab
\begin{align*}
\lambda_1 &= \lambda_3=1\\
\lambda_2 &= -2\lambda_3=2.
\end{align*}
Eine mögliche Linearkombination hat also die Koeffizienten
$\lambda_1=1$,
$\lambda_2=-2$
und
$\lambda_1=1$.
\end{loesung}

