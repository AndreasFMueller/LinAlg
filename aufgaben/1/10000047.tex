Für welchen Wert des Parameters $c$ in der Matrix
\[
A=\begin{pmatrix}
1&2&3\\
2&3&c\\
3&4&5
\end{pmatrix},
\]
kann man Koeffzienten $\lambda_1,\dots,\lambda_3$ einer Linearkombination
von Zeilen der Matrix $A$ finden, die verschwindet?
Bestimmen Sie auch die $\lambda_i$.

\thema{lineare Abhängigkeit}
\thema{Matrix mit Parameter}

\begin{loesung}
Die Koeffizienten sind Lösung eines Gleichungssystems mit der
transponierten Matrix $A^t$, also mit dem Tableau
\begin{align*}
\begin{tabular}{|>{$}c<{$}>{$}c<{$}>{$}c<{$}|>{$}c<{$}|}
\hline
1&2&3&0\\
2&3&4&0\\
3&c&5&0\\
\hline
\end{tabular}
&\rightarrow
\begin{tabular}{|>{$}c<{$}>{$}c<{$}>{$}c<{$}|>{$}c<{$}|}
\hline
1&  2& 3&0\\
0& -1&-2&0\\
0&c-6&-4&0\\
\hline
\end{tabular}
\\
&\rightarrow
\begin{tabular}{|>{$}c<{$}>{$}c<{$}>{$}c<{$}|>{$}c<{$}|}
\hline
1&  2& 3&0\\
0&  1& 2&0\\
0&  0&-4-2(c-6)&0\\
\hline
\end{tabular}
=
\begin{tabular}{|>{$}c<{$}>{$}c<{$}>{$}c<{$}|>{$}c<{$}|}
\hline
1&  2& 3&0\\
0&  1& 2&0\\
0&  0&8-2c&0\\
\hline
\end{tabular}
\\
&\rightarrow
\begin{tabular}{|>{$}c<{$}>{$}c<{$}>{$}c<{$}|>{$}c<{$}|}
\hline
1&  0&-1&0\\
0&  1& 2&0\\
0&  0&8-2c&0\\
\hline
\end{tabular}
\end{align*}
Damit das Gleichungssystem überhaupt eine Lösung $\ne 0$ hat, muss eine
Nullzeile entstehen. Das ist nur möglich, wenn der Eintrag rechts unten
verschwindet, also
\[
8-2c = 0 \qquad\Rightarrow\qquad c=4.
\]
Damit ist der Wert des Parameters $c$ bestimmt.

Jetzt müssen aber noch die Koeffizienten $\lambda_i$ gefunden werden.
$\lambda_3$ ist offenbar frei wählbar, wir wählen $\lambda_3=1$.
Dann lesen wird ab
\begin{align*}
\lambda_1 &= \lambda_3=1\qquad\text{und}\\
\lambda_2 &= -2\lambda_3=-2.
\end{align*}
Eine mögliche Linearkombination für $c=4$ hat also die Koeffizienten
$\lambda_1=1$,
$\lambda_2=-2$
und
$\lambda_3=1$.
Kontrolle:
\[
1\cdot\begin{pmatrix}1&2&3\end{pmatrix}
+
(-2)\cdot \begin{pmatrix}2&3&4\end{pmatrix}
+
1\cdot\begin{pmatrix}3&4&5\end{pmatrix}
=
\begin{pmatrix}
1-4+3& 2-6+4&3-8+5
\end{pmatrix}
=
\begin{pmatrix}0&0&0\end{pmatrix}.
\qedhere
\]
\end{loesung}

