Lösen Sie das Gleichungssystem
$Ax=b$ mit
\[
A=
\begin{pmatrix}
   1&  0&  1\\
   1& -5&  2\\
  -1&  1& -1
\end{pmatrix}
,\qquad
b=\begin{pmatrix}1\\2\\3\end{pmatrix}
\]
mit Hilfe der inversen Matrix.

\themaS{inverse Matrix}
\themaS{Gauss-Algorithmus}

\begin{loesung}
Es ist also zunächst die Inverse von $A$ zu ermitteln, was mit dem
Gauss-Algorithmus geschehen kann:
\begin{align*}
\begin{tabular}{|>{$}c<{$}>{$}c<{$}>{$}c<{$}|>{$}c<{$}>{$}c<{$}>{$}c<{$}|}
\hline
   1 &  0 &  1 &1&0&0\\
   1 & -5 &  2 &0&1&0\\
  -1 &  1 & -1 &0&0&1\\
\hline
\end{tabular}
&\rightarrow
\begin{tabular}{|>{$}c<{$}>{$}c<{$}>{$}c<{$}|>{$}c<{$}>{$}c<{$}>{$}c<{$}|}
\hline
   1 &  0 &  1 &  1 &0&0\\
   0 & -5 &  1 & -1 &1&0\\
   0 &  1 &  0 &  1 &0&1\\
\hline
\end{tabular}
\\
&\rightarrow
\begin{tabular}{|>{$}c<{$}>{$}c<{$}>{$}c<{$}|>{$}c<{$}>{$}c<{$}>{$}c<{$}|}
\hline
   1 &  0 &  1       &  1      &0       &0\\
   0 &  1 & -\frac15 & \frac15 &-\frac15&0\\
   0 &  1 &  0       &  1      &0       &1\\
\hline
\end{tabular}
\\
&\rightarrow
\begin{tabular}{|>{$}c<{$}>{$}c<{$}>{$}c<{$}|>{$}c<{$}>{$}c<{$}>{$}c<{$}|}
\hline
   1 &  0 &  1       &  1      &0       &0\\
   0 &  1 & -\frac15 & \frac15 &-\frac15&0\\
   0 &  0 &  \frac15 & \frac45 & \frac15&1\\
\hline
\end{tabular}
\\
&\rightarrow
\begin{tabular}{|>{$}c<{$}>{$}c<{$}>{$}c<{$}|>{$}c<{$}>{$}c<{$}>{$}c<{$}|}
\hline
   1 &  0 &  0 & -3 & -1 & -5 \\
   0 &  1 &  0 &  1 &  0 &  1\\
   0 &  0 &  1 &  4 &  1 &  5\\
\hline
\end{tabular}
\end{align*}
Die Inverse von $A$ ist also
\[
A^{-1}
=
\begin{pmatrix}
  -3& -1& -5\\
   1&  0&  1\\
   4&  1&  5
\end{pmatrix}.
\]
Die Korrektheit dieses Zwischenresultates kann man durch Ausmultiplizieren
von $AA^{-1}$ überprüfen:
\[
\begin{pmatrix}
   1&  0&  1\\
   1& -5&  2\\
  -1&  1& -1
\end{pmatrix}
\begin{pmatrix}
  -3& -1& -5\\
   1&  0&  1\\
   4&  1&  5
\end{pmatrix}
=
\begin{pmatrix}
-3+0+4&-1+0+1&-5+0+5\\
-3-5+8&-1+0+2&-5-5+10\\
3+1-4&1+0-1&5+1-5
\end{pmatrix}
=E.
%\begin{pmatrix}
%1&0&0\\
%0&1&0\\
%0&0&1
%\end{pmatrix}
\]
Daraus kann man jetzt die Lösung des Gleichungssystems finden:
\[
x=A^{-1}b=
\begin{pmatrix}
  -3& -1& -5\\
   1&  0&  1\\
   4&  1&  5
\end{pmatrix}
\begin{pmatrix}1\\2\\3\end{pmatrix}
=
\begin{pmatrix}
(-3)\cdot 1+(-1)\cdot 2+(-5)\cdot 3\\
1\cdot 1+0\cdot 2+1\cdot 3\\
4\cdot 1+1\cdot 2+5\cdot 3
\end{pmatrix}
=
\begin{pmatrix}
-20\\
4\\
21
\end{pmatrix}.
\qedhere
\]
\end{loesung}
