Ein Oktaeder hat 8 Ecken und 12 Kanten. Finden Sie eine minimale
Menge von Zyklen.

\begin{loesung}
\setcounter{MaxMatrixCols}{20}
Die Randmatrix hat die Form
\[
\partial=
\begin{pmatrix}
-1&-1&-1&-1& 0& 0& 0& 0& 0& 0& 0& 0\\
 1& 0& 0& 0&-1&-1&-1& 0& 0& 0& 0& 0\\
 0& 1& 0& 0& 1& 0& 0&-1&-1& 0& 0& 0\\
 0& 0& 1& 0& 0& 1& 0& 1& 0&-1&-1& 0\\
 0& 0& 0& 1& 0& 0& 0& 0& 0& 1& 0&-1\\
 0& 0& 0& 0& 0& 0& 1& 0& 1& 0& 1& 1
\end{pmatrix}
\]
Der Gaussalgorithmus findet daraus
\begin{center}
\begin{tabular}{|>{$}c<{$}>{$}c<{$}>{$}c<{$}>{$}c<{$}>{$}c<{$}>{$}c<{$}>{$}c<{$}>{$}c<{$}>{$}c<{$}>{$}c<{$}>{$}c<{$}>{$}c<{$}|}
\hline
x_1&x_2&x_3&x_4&x_5&x_6&x_7&x_8&x_9&x_{10}&x_{11}&x_{12}\\
\hline
   1&  0&  0&  0& -1& -1&  0&  0&  1&  0&  1&  1\\
   0&  1&  0&  0&  1&  0&  0& -1& -1&  0&  0&  0\\
   0&  0&  1&  0&  0&  1&  0&  1&  0& -1& -1&  0\\
   0&  0&  0&  1&  0&  0&  0&  0&  0&  1&  0& -1\\
   0&  0&  0&  0&  0&  0&  1&  0&  1&  0&  1&  1\\
   0&  0&  0&  0&  0&  0&  0&  0&  0&  0&  0&  0\\
\hline
    &   &   &   &  \color{green}*&  \color{green}*&   &  \color{green}*&  \color{green}*&  \color{green}*&  \color{green}*&  \color{green}*\\
\hline
\end{tabular}
\end{center}
Man kann dies schneller dadurch finden, dass man die erste Zeile in die
f"unfte Position verscheibt, und dann den Gauss-Algorithmus startet.
Die Kanten mit den Nummern 5, 6 und 8--12 sind frei w"ahlbar.
Es gibt also 7 linear unabh"angige Zyklen, mit folgenden Vektoren
\[
z_1=
\begin{pmatrix}
              1\\
             -1\\
              0\\
              0\\
\color{green} 1\\
\color{green} 0\\
              0\\
\color{green} 0\\
\color{green} 0\\
\color{green} 0\\
\color{green} 0\\
\color{green} 0
\end{pmatrix}
,\quad
z_1=\begin{pmatrix}
              1\\
              0\\
             -1\\
              0\\
\color{green} 0\\
\color{green} 1\\
              0\\
\color{green} 0\\
\color{green} 0\\
\color{green} 0\\
\color{green} 0\\
\color{green} 0
\end{pmatrix}
,\quad
z_3=\begin{pmatrix}
              0\\
              1\\
             -1\\
              0\\
\color{green} 0\\
\color{green} 0\\
              0\\
\color{green} 1\\
\color{green} 0\\
\color{green} 0\\
\color{green} 0\\
\color{green} 0
\end{pmatrix}
,\quad
z_4=\begin{pmatrix}
             -1\\
              1\\
              0\\
              0\\
\color{green} 0\\
\color{green} 0\\
             -1\\
\color{green} 0\\
\color{green} 1\\
\color{green} 0\\
\color{green} 0\\
\color{green} 0
\end{pmatrix}
,\quad
z_5=
\begin{pmatrix}
              0\\
              0\\
              1\\
             -1\\
\color{green} 0\\
\color{green} 0\\
              0\\
\color{green} 0\\
\color{green} 0\\
\color{green} 1\\
\color{green} 0\\
\color{green} 0
\end{pmatrix}
,\quad
z_6=\begin{pmatrix}
             -1\\
              0\\
              1\\
              0\\
\color{green} 0\\
\color{green} 0\\
             -1\\
\color{green} 0\\
\color{green} 0\\
\color{green} 0\\
\color{green} 1\\
\color{green} 0
\end{pmatrix}
,\quad
z_7=\begin{pmatrix}
             -1\\
              0\\
              0\\
              1\\
\color{green} 0\\
\color{green} 0\\
             -1\\
\color{green} 0\\
\color{green} 0\\
\color{green} 0\\
\color{green} 0\\
\color{green} 1
\end{pmatrix}
\]

\end{loesung}
