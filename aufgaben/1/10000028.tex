Ein Oktaeder hat sechs Ecken und zwölf Kanten (Abbildung~\ref{10000028:oktaeder}).
Finden Sie eine minimale Menge von Zyklen.
\begin{figure}
\centering
\includeagraphics[]{octahedron-1.pdf}
\caption{Numerierung der Ecken und Kanten eines Oktaeders für die
Aufgabe~\ref{10000028}
\label{10000028:oktaeder}}
\end{figure}

\thema{Zyklen}

\begin{loesung}
\definecolor{greenT}{rgb}{0,0.666,0}
\setcounter{MaxMatrixCols}{20}
Die Randmatrix hat die Form
\[
\partial=
\begin{pmatrix}
%1  2  3  4  5  6  7  8  9 10 11 12
-1&-1&-1&-1& 0& 0& 0& 0& 0& 0& 0& 0\\ % 1
 1& 0& 0& 0&-1& 0& 0& 1&-1& 0& 0& 0\\ % 2
 0& 1& 0& 0& 1&-1& 0& 0& 0&-1& 0& 0\\ % 3
 0& 0& 1& 0& 0& 1&-1& 0& 0& 0&-1& 0\\ % 4
 0& 0& 0& 1& 0& 0& 1&-1& 0& 0& 0&-1\\ % 5
 0& 0& 0& 0& 0& 0& 0& 0& 1& 1& 1& 1   % 6
\end{pmatrix}
\]
Der Gaussalgorithmus findet daraus
\begin{center}
\begin{tabular}{|>{$}r<{$}>{$}r<{$}>{$}r<{$}>{$}r<{$}>{$}r<{$}>{$}r<{$}>{$}r<{$}>{$}r<{$}>{$}r<{$}>{$}r<{$}>{$}r<{$}>{$}r<{$}|}
\hline
z_1&z_2&z_3&z_4&z_5&z_6&z_7&z_8&z_9&z_{10}&z_{11}&z_{12}\\
\hline
   1&  0&  0&  0& -1&  0&  0&  1&  0&  1&  1&  1\\
   0&  1&  0&  0&  1& -1&  0&  0&  0& -1&  0&  0\\
   0&  0&  1&  0&  0&  1& -1&  0&  0&  0& -1&  0\\
   0&  0&  0&  1&  0&  0&  1& -1&  0&  0&  0& -1\\
   0&  0&  0&  0&  0&  0&  0&  0&  1&  1&  1&  1\\
\hline
    &   &   &   &  \color{greenT}*&  \color{greenT}*&   \color{greenT}*&  \color{greenT}*&  &\color{greenT}*&  \color{greenT}*&  \color{greenT}*\\
\hline
\end{tabular}
\end{center}
was am einfachsten mit dem Computer geht.
Die Kanten mit den Nummern 5--8 und 10--12 sind frei wählbar.
Wir bezeichnen die Zyklen mit $z^{(k)}$, mit $k=1,\dots,7$.
Es gibt also 7 linear unabhängige Zyklen, mit folgenden Vektoren
\[
z^{(1)}=
\begin{pmatrix}
              1\\
             -1\\
              0\\
              0\\
\color{greenT} 1\\
\color{greenT} 0\\
\color{greenT} 0\\
\color{greenT} 0\\
              0\\
\color{greenT} 0\\
\color{greenT} 0\\
\color{greenT} 0
\end{pmatrix}
,\quad
z^{(2)}=\begin{pmatrix}
              0\\
              1\\
             -1\\
              0\\
\color{greenT} 0\\
\color{greenT} 1\\
\color{greenT} 0\\
\color{greenT} 0\\
              0\\
\color{greenT} 0\\
\color{greenT} 0\\
\color{greenT} 0
\end{pmatrix}
,\quad
z^{(3)}=\begin{pmatrix}
              0\\
              0\\
              1\\
             -1\\
\color{greenT} 0\\
\color{greenT} 0\\
\color{greenT} 1\\
\color{greenT} 0\\
              0\\
\color{greenT} 0\\
\color{greenT} 0\\
\color{greenT} 0
\end{pmatrix}
,\quad
z^{(4)}=\begin{pmatrix}
             -1\\
              0\\
              0\\
              1\\
\color{greenT} 0\\
\color{greenT} 0\\
\color{greenT} 0\\
\color{greenT} 1\\
              0\\
\color{greenT} 0\\
\color{greenT} 0\\
\color{greenT} 0
\end{pmatrix}
,\quad
z^{(5)}=
\begin{pmatrix}
             -1\\
              1\\
              0\\
              0\\
\color{greenT} 0\\
\color{greenT} 0\\
\color{greenT} 0\\
\color{greenT} 0\\
             -1\\
\color{greenT} 1\\
\color{greenT} 0\\
\color{greenT} 0
\end{pmatrix}
,\quad
z^{(6)}=\begin{pmatrix}
             -1\\
              0\\
              1\\
              0\\
\color{greenT} 0\\
\color{greenT} 0\\
\color{greenT} 0\\
\color{greenT} 0\\
             -1\\
\color{greenT} 0\\
\color{greenT} 1\\
\color{greenT} 0
\end{pmatrix}
,\quad
z^{(7)}=\begin{pmatrix}
             -1\\
              0\\
              0\\
              1\\
\color{greenT} 0\\
\color{greenT} 0\\
\color{greenT} 0\\
\color{greenT} 0\\
             -1\\
\color{greenT} 0\\
\color{greenT} 0\\
\color{greenT} 1
\end{pmatrix}.
\]
Die Zyklen $z^{(1)}$ bis $z^{(4)}$ sind in Abbildung~\ref{10000028:z1bisz4}
dargestellt, die Zyklen $z^{(5)}$ bis $z^{(7)}$ in Abbildung~\ref{10000028:z5bisz7}.
\begin{figure}
\centering
\includeagraphics[width=0.22\hsize]{octahedron-2.pdf}
\includeagraphics[width=0.22\hsize]{octahedron-3.pdf}
\includeagraphics[width=0.22\hsize]{octahedron-4.pdf}
\includeagraphics[width=0.22\hsize]{octahedron-5.pdf}
\caption{Zyklen $z^{(1)}$ bis $z^{(4)}$, zu frei wählbaren Variablen
gehörende Kanten in grün
\label{10000028:z1bisz4}}
\end{figure}
\begin{figure}
\centering
\includeagraphics[width=0.22\hsize]{octahedron-6.pdf}
\includeagraphics[width=0.22\hsize]{octahedron-7.pdf}
\includeagraphics[width=0.22\hsize]{octahedron-8.pdf}
\caption{Zyklen $z^{(5)}$ bis $z^{(7)}$, zu frei wählbaren Variablen
gehörende Kanten in grün
\label{10000028:z5bisz7}}
\end{figure}

\end{loesung}
