Gegeben ist die Matrix $A$ und der Vektor $b$:
\[
A
=
\begin{pmatrix}
1&2&4\\
1&4&c\\
1&3&4
\end{pmatrix}
\qquad
\text{und}
\qquad
b=\begin{pmatrix}1\\3\\2\end{pmatrix}.
\]
\begin{teilaufgaben}
\item
Berechnen Sie die Inverse der Matrix $A$.
\item
Für welche Werte von $c$ existiert die inverse Matrix?
\item
Finden Sie die Lösungsmenge des Gleichungssystems $Ax=b$ für diejenigen
Werte, wo $A$ regulär ist. Verwenden Sie dazu die Inverse.
%\item
%Finden Sie die Lösungsmenge des Gleichungssystems $Ax=b$ für diejenigen
%Werte von $c$, wo die Inverse nicht existiert.
\end{teilaufgaben}

\thema{Gauss-Algorithmus}
\thema{Matrix mit Parameter}
\thema{inverse Matrix}

\begin{loesung}
\begin{teilaufgaben}
\item
Die inverse Matrix kann mit dem Gauss-Algorithmus berechnet werden.
\begin{align*}
\begin{tabular}{|>{$}c<{$}>{$}c<{$}>{$}c<{$}|>{$}c<{$}>{$}c<{$}>{$}c<{$}|}
\hline
1&2&4&1&0&0\\
1&4&c&0&1&0\\
1&3&4&0&0&1\\
\hline
\end{tabular}
&
\rightarrow
\begin{tabular}{|>{$}c<{$}>{$}c<{$}>{$}c<{$}|>{$}c<{$}>{$}c<{$}>{$}c<{$}|}
\hline
1&2&  4& 1&0&0\\
0&2&c-4&-1&1&0\\
0&1&  0&-1&0&1\\
\hline
\end{tabular}
\\
&
\rightarrow
\begin{tabular}{|>{$}c<{$}>{$}c<{$}>{$}c<{$}|>{$}c<{$}>{$}c<{$}>{$}c<{$}|}
\hline
1&2&  4& 1&0&0\\
0&1&  0&-1&0&1\\
0&2&c-4&-1&1&0\\
\hline
\end{tabular}
\\
&
\rightarrow
\begin{tabular}{|>{$}c<{$}>{$}c<{$}>{$}c<{$}|>{$}c<{$}>{$}c<{$}>{$}c<{$}|}
\hline
1&2&  4& 1&0& 0\\
0&1&  0&-1&0& 1\\
0&0&c-4& 1&1&-2\\
\hline
\end{tabular}
\\
&
\rightarrow
\begin{tabular}{|>{$}c<{$}>{$}c<{$}>{$}c<{$}|>{$}c<{$}>{$}c<{$}>{$}c<{$}|}
\hline
1&2&  0&\frac{c-8}{c-4}&-\frac{4}{c-4}&\frac{ 8}{c-4}\\
0&1&  0&             -1&             0&             1\\
0&0&  1&  \frac{1}{c-4}& \frac{1}{c-4}&\frac{-2}{c-4}\\
\hline
\end{tabular}
\\
&
\rightarrow
\begin{tabular}{|>{$}c<{$}>{$}c<{$}>{$}c<{$}|>{$}c<{$}>{$}c<{$}>{$}c<{$}|}
\hline
1&0&  0&\frac{3c-16}{c-4}&-\frac{4}{c-4}&-\frac{2c-16}{c-4}\\
0&1&  0&               -1&             0&                 1\\
0&0&  1&    \frac{1}{c-4}& \frac{1}{c-4}&    \frac{-2}{c-4}\\
\hline
\end{tabular}
\end{align*}
Im dritten Schritt wurde zur Vermeidung einer Division durch 2 zwei
Zeilen vertauscht.
Daraus liest man ab, dass die inverse Matrix
\[
A^{-1}
=
\frac{1}{c-4}
\begin{pmatrix}
3c-16&-4& -2c+16\\
-c+ 4& 0&   c- 4\\
    1& 1&    - 2\\
\end{pmatrix}
\]
ist. Zur Kontrolle rechnen wir nach:
\begin{align*}
AA^{-1}
&=
\frac{1}{c-4}
\begin{pmatrix}
1&2&4\\
1&4&c\\
1&3&4
\end{pmatrix}
\begin{pmatrix}
3c-16&-4& -2c+16\\
-c+ 4& 0&   c- 4\\
    1& 1&    - 2\\
\end{pmatrix}
\\
&=
\frac1{c-4}
\begin{pmatrix}
3c-16-2c+8+4 &-4+0+4&-2c+16+2c-8-8\\
3c-16-4c+16+c&-4+0+c&-2c+16+4c-16-2c\\
3c-16-3c+12+4&-4+0+4&-2c+16+3c-12-8\\
\end{pmatrix}
\\
&=
\frac1{c-4}
\begin{pmatrix}
c-4&  0&  0\\
  0&c-4&  0\\
  0&  0&c-4\\
\end{pmatrix}
=
\begin{pmatrix}
1&0&0\\
0&1&0\\
0&0&1\\
\end{pmatrix}
=E.
\end{align*}

Natürlich kann man die Inverse auch mit Hilfe der Minoren ausrechnen,
dazu braucht man zunächst die Determinante der Matrix:
\begin{align*}
\left|\begin{matrix}
1&2&4\\
1&4&c\\
1&3&4
\end{matrix}\right|
&=1\cdot 4\cdot 4+2\cdot c\cdot 1+4\cdot 1\cdot 3
-1\cdot 4\cdot 4-3\cdot c\cdot 1-4\cdot 1\cdot 2
\\
&=
16+2c+12-16-3c-8
=
-c+4
\end{align*}
Die Einträge der inversen Matrix kann man jetzt mit den Minoren bilden:
\begin{align*}
A^{-1}
&=
\def\arraystretch{2.3}
\frac1{-c+4}\begin{pmatrix}
\def\arraystretch{1}
\phantom{-}\left|\begin{matrix}4&c\\3&4\end{matrix}\right|
	&\def\arraystretch{1}
	-\left|\begin{matrix}2&4\\3&4\end{matrix}\right|
		&\def\arraystretch{1}
		\phantom{-}\left|\begin{matrix}2&4\\4&c\end{matrix}\right|
\\
\def\arraystretch{1}
-\left|\begin{matrix}1&c\\1&4\end{matrix}\right|
	&\def\arraystretch{1}
	\phantom{-}\left|\begin{matrix}1&4\\1&4\end{matrix}\right|
		&\def\arraystretch{1}
		-\left|\begin{matrix}1&4\\1&c\end{matrix}\right|
\\
\def\arraystretch{1}
\phantom{-}\left|\begin{matrix}1&4\\1&3\end{matrix}\right|
	&\def\arraystretch{1}
	-\left|\begin{matrix}1&2\\1&3\end{matrix}\right|
		&\def\arraystretch{1}
		\phantom{-}\left|\begin{matrix}1&2\\1&4\end{matrix}\right|
\end{pmatrix}
\\
&=
\frac1{-c+4}
\begin{pmatrix}
   16-3c&      4&  2c-16\\
    -4+c&      0&   -c+4\\
      -1&     -1&      2
\end{pmatrix}.
\end{align*}
Dies stimmt mit der mit dem Gaussalgorithmus gefundenen Inversen überein.
\item
Die Inverse existiert nicht für $c=4$, in allen anderen Fällen
liefert der Gauss-Algorithmus die Inverse.
\item
Die Lösung kann mit Hilfe von $x=A^{-1}b$ bestimmt werden:
\begin{align*}
x&=
A^{-1}b=
\frac{1}{c-4}
\begin{pmatrix}
3c-16&-4& -2c+16\\
-c+ 4& 0&   c- 4\\
    1& 1&    - 2\\
\end{pmatrix}
\begin{pmatrix}1\\3\\2\end{pmatrix}
=
\frac{1}{c-4}
\begin{pmatrix}
3c-16-12-4c+32\\
-c+4+0+2c-8\\
1+3-4
\end{pmatrix}
\\
&=
\frac{1}{c-4}
\begin{pmatrix}
-c+4\\
c-4\\
0
\end{pmatrix}
=\begin{pmatrix}
-1\\1\\0
\end{pmatrix}.
\end{align*}
Wir kontrollieren dieses Resultat durch Einsetzen:
\[
Ax=
\begin{pmatrix}
1&2&4\\
1&4&c\\
1&3&4
\end{pmatrix}
\begin{pmatrix}
-1\\1\\0
\end{pmatrix}
=
\begin{pmatrix}
1\\3\\2
\end{pmatrix}=b.
\qedhere
\]
%\item
%Wir müssen für den Fall $c=4$ das Gleichungssystem lösen.
%Auch dies geht mit dem Gauss-Algorithmus,
%wobei wir schon zu Beginn $c=4$ setzen:
%\begin{align*}
%\begin{tabular}{|>{$}c<{$}>{$}c<{$}>{$}c<{$}|>{$}c<{$}|}
%\hline
%1&2&4&1\\
%1&4&4&3\\
%1&3&4&2\\
%\hline
%\end{tabular}
%&
%\rightarrow
%\begin{tabular}{|>{$}c<{$}>{$}c<{$}>{$}c<{$}|>{$}c<{$}|}
%\hline
%1&2&4&1\\
%0&2&0&2\\
%0&1&0&1\\
%\hline
%\end{tabular}
%\\
%&\rightarrow
%\begin{tabular}{|>{$}c<{$}>{$}c<{$}>{$}c<{$}|>{$}c<{$}|}
%\hline
%1&2&4&1\\
%0&1&0&1\\
%0&0&0&0\\
%\hline
%\end{tabular}
%\\
%&\rightarrow
%\begin{tabular}{|>{$}c<{$}>{$}c<{$}>{$}c<{$}|>{$}c<{$}|}
%\hline
%1&0&4&-1\\
%0&1&0& 1\\
%0&0&0& 0\\
%\hline
%\end{tabular}
%\end{align*}
%Daraus können wir die Lösungsmenge
%\[
%{\mathbb L}
%=
%\left\{\left.\begin{pmatrix}-1\\1\\0\end{pmatrix}+\lambda\begin{pmatrix}-4\\0\\0\end{pmatrix}\;\right|\;\lambda\in\mathbb R\right\}
%\]
%ablesen.
\end{teilaufgaben}
\end{loesung}

\begin{bewertung}
\begin{teilaufgaben}
\item
Berechnung der Inversion mit Gauss ({\bf G}) 3 Punkte,
\item
Existenz der Inversen ({\bf E}) 1 Punkt,
\item
Lösungsformel $x=A^{-1}b$ ({\bf F}) 1 Punkt,
Lösung ({\bf L}) 1 Punkt.
\end{teilaufgaben}
\end{bewertung}
