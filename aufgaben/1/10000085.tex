Der Gauss-Algorithmus liefert für 
\begin{equation*}
A
=
\bgroup
\renewcommand{\arraycolsep}{1pt}
\begin{pmatrix*}[r]
   5 & -20 &  15 &  10 \\
  -2 &   3 & -11 & -19 \\
   1 &  -5 &   2 &  -1 \\
  -4 &  18 & -10 &  -2 
\end{pmatrix*}\!,
\egroup
\;
b
=
\begin{pmatrix*}[r]
 -170 \\
   88 \\
  -30 \\
  128 
\end{pmatrix*}
\;\text{das Schlusstableau}\quad
%\begin{tabular}{|>{$}r<{$}>{$}r<{$}>{$}r<{$}>{$}r<{$}|>{$}r<{$}|}
%\hline
% x_1 & x_2 & x_3 & x_4 &    1 \\
%\hline
%   5 & -20 &  15 &  10 & -170 \\
%  -2 &   3 & -11 & -19 &   88 \\
%   1 &  -5 &   2 &  -1 &  -30 \\
%  -4 &  18 & -10 &  -2 &  128 \\
%\hline
%\end{tabular}
%\to
\bgroup
\renewcommand{\tabcolsep}{4pt}
\begin{tabular}{|>{$}r<{$}>{$}r<{$}>{$}r<{$}>{$}r<{$}|>{$}r<{$}|}
\hline
 x_1 & x_2 & x_3 & x_4 &    1 \\
\hline
   1 &   0 &   7 &  14 &  -50 \\
   0 &   1 &   1 &   3 &   -4 \\
   0 &   0 &   0 &   0 &    0 \\
   0 &   0 &   0 &   0 &    0 \\
\hline
\end{tabular}.
\egroup
\end{equation*}
\begin{teilaufgaben}
\item Wie gross ist der Rang der Matrix $A$?
\item Wieviele Lösungen hat das Gleichungssystem $Ax=b$?
\end{teilaufgaben}

\begin{loesung}
\begin{teilaufgaben}
\item $\operatorname{Rang}A=2$
\item Da es zwei frei wählbare Variablen gibt, gibt es unendlich
viele Lösungen.
\qedhere
\end{teilaufgaben}
\end{loesung}


