Betrachten Sie die Koeffizientenmatrix
\[
A
=
\begin{pmatrix}
a-1&2a-2&3a-3\\
 1 & a  &2a-1\\
 1 & 3  & a+2
\end{pmatrix}.
\]
\begin{teilaufgaben}
\item
Es gibt drei Werte des Parameters $a$, für die die Matrix 
nicht regulär ist.
Bestimmen Sie diese Werte.
\item
Lösen Sie das Gleichungssystem $Ax=b$ mit der rechten Seite
\[
b
=
\begin{pmatrix}
2a^2+2a-4\\
a^2+2a\\
4a+3
\end{pmatrix}
\]
für Werte $a$, für die das Gleichungssystem regulär ist.
\end{teilaufgaben}

\begin{loesung}
Wir wenden den Gauss-Algorithmus an und achten darauf, für welche
Werte von $a$ das aktuelle Pivotelement verschwinden könnte.
Für das erste Pivot-Element ist dies der Werte $a=1$, der
die erste Zeile zu einer Nullzeile machen würde.
Für $a\ne 1$ kann man den ersten Gauss-Schritt durchführen:
\begin{align}
\begin{tabular}{|>{$}c<{$}>{$}c<{$}>{$}c<{$}|>{$}c<{$}|}
\hline
\pivotoperation{1.1}{0.42}{-1.3mm}{-1.0mm}
a-1 & 2a-2 & 3a-3 & 2a^2+2a-4 \\
\forwardreduction{0.42}{0.98}{-1.3mm}{-6.2mm}
 1  & a    & 2a-1 &  a^2+2a   \\
 1  &    3 &  a+2 &       a+3 \\
\hline
\end{tabular}
&\to
\begin{tabular}{|>{$}c<{$}>{$}c<{$}>{$}c<{$}|>{$}c<{$}|}
\hline
 1  &   2 &    3 &     2a+4 \\
 0  &
\pivotoperation{1.1}{0.42}{-1.3mm}{-1.0mm}
      a-2 & 2a-4 & a^2   -4 \\
 0  &
\forwardreduction{0.42}{0.48}{-1.3mm}{-1.3mm}
        1 &  a-1 &      a-1 \\
\hline
\end{tabular}.
\notag
\intertext{Das zweite Pivotelement ist $a-2$.
Es verschwindet, wenn $a=2$ ist.
In diesem Fall wird die zweite Zeile zu einer Nullzeile.
Für $a\ne 2$ kann man den zweiten Gauss-Schritt durchführen
und erhält}
&\to
\begin{tabular}{|>{$}c<{$}>{$}c<{$}>{$}c<{$}|>{$}c<{$}|}
\hline
 1  &   2 &
\backwardsubstitution{0.42}{0.88}{-1.2mm}{-5.8mm}
              3 & 2a+4 \\
 0  &   1 &   2 &  a+2 \\
 0  &   0 &
\pivotoperation{1.1}{0.42}{-1.3mm}{-1.0mm}
            a-3 &  a-3 \\
\hline
\end{tabular}.
\notag
\intertext{Das dritte Pivot-Element ist $a-3$, somit ist $a=3$
der dritte Wert, für den die Matrix singulär wird.
Für $a\ne 3$ kann man den Gauss-Algorithmus zu Ende führen:}
\to
\begin{tabular}{|>{$}c<{$}>{$}c<{$}>{$}c<{$}|>{$}c<{$}|}
\hline
 1  &
\backwardsubstitution{0.42}{0.44}{-1.2mm}{-1.3mm}
        2 &   0 & 2a+1 \\
 0  &   1 &   0 &    a \\
 0  &   0 &   1 &    1 \\
\hline
\end{tabular}
&\to
\begin{tabular}{|>{$}c<{$}>{$}c<{$}>{$}c<{$}|>{$}c<{$}|}
\hline
 1  &   0 &   0 &   1 \\
 0  &   1 &   0 &   a \\
 0  &   0 &   1 &   1 \\
\hline
\end{tabular}.
\label{10000077:schlusstableau}
\end{align}
Aus der Durchführung des Gauss-Algorithmus kann man jetzt die
Antworten auf die gestellten Fragen ableiten.
\begin{teilaufgaben}
\item
Die Werte, die ausgeschlossen werden müssen, weil sie die Matrix
$A$ singulär machen, sind $a\in \{1,2,3\}$.
\item
Für Werte $a$, für das Gleichungssystem regulär ist, kann man aus dem
Schlusstableau \eqref{10000077:schlusstableau} die Lösung
\[
x=\begin{pmatrix}1\\a\\1\end{pmatrix}
\]
ablesen.
\qedhere
\end{teilaufgaben}
\end{loesung}

\begin{bewertung}
Gauss-Algorithmus 3 Punkte: Vorwärtsreduktion ({\bf V}),
Rückwärtseinsetzen ({\bf R}), Ablauf ({\bf A}),
Regularitätsbedingung Pivotelemente $\ne 0$ ({\bf U}) 1 Punkt,
Werte für $a$ ({\bf W}) 1 Punkt,
Lösung in b) ({\bf L}) 1 Punkt.
\end{bewertung}

\begin{diskussion}
Man kann die Werte, für die $A$ singulär wird, natürlich auch mit der
Determinanten bestimmen.
Zur Vereinachung zeihen wir vor der Anwendung der Sarrus-Formel
den gemeinsamen Faktor $a-1$ aus der ersten Zeile:
\begin{align*}
\det A
&=
(a-1)
\left|\begin{matrix}
1&2&3\\
1&a&2a-1\\
1&3&a+2
\end{matrix}\right|
\\
&=
(a-1)\bigl(
a(a+2)+2(2a-1)+9-3a-3(2a-1)-2(a+2)
\bigr)
\\
&=
(a-1)(a^2+2a+4a-2+9-3a-6a+3-2a-4)
\\
&=
(a-1)(a^2-5a+6)
=
(a-1)(a-2)(a-3).
\end{align*}
Die Nullstellen der Determinante sind $a\in\{1,2,3\}$, für diese
Werte ist die Matrix $A$ nicht regulär.
\end{diskussion}
