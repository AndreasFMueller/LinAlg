Betrachten Sie das Gleichungssystem
\[
\begin{linsys}{2}
x&+&ay&=&4\\
ax&+&9y&=&b
\end{linsys}
\]
\begin{teilaufgaben}
\item Für welche Werte von $a$ und $b$ hat das Gleichungssystem eine
eindeutig bestimmte Lösung?
\item Finden Sie die Werte von $a$ und $b$, für die das Gleichungssystem
mehr als eine Lösung hat.
\end{teilaufgaben}

\thema{Gauss-Algorithmus}
\thema{Matrix mit Parameter}

\begin{loesung}
Aus dem Gauss-Verfahren bekommen wir
\begin{align*}
\begin{tabular}{|>{$}c<{$}>{$}c<{$}|>{$}c<{$}|}
\hline
1&a&4\\
a&9&b\\
\hline
\end{tabular}
&\rightarrow
\begin{tabular}{|>{$}c<{$}>{$}c<{$}|>{$}c<{$}|}
\hline
1&a&4\\
0&9-a^2&b-4a\\
\hline
\end{tabular}
\\
&\rightarrow
\begin{tabular}{|>{$}c<{$}>{$}c<{$}|>{$}c<{$}|}
\hline
1&a&4\\
0&1&\frac{b-4a}{9-a^2}\\
\hline
\end{tabular}
\\
&\rightarrow
\begin{tabular}{|>{$}c<{$}>{$}c<{$}|>{$}c<{$}|}
\hline
1&0&4-a \frac{b-4a}{9-a^2}\\
0&1&\frac{b-4a}{9-a^2}\\
\hline
\end{tabular}
\\
\end{align*}
\begin{teilaufgaben}
\item Die Durchführung des Gaussverfahrens ist nur möglich, wenn $9-a^2\ne 0$, also
$|a|\ne 3$. Wie gross $b$ ist spielt in diesem Fall keine Rolle.
\item Dies ist der singuläre Fall, er tritt ein, wenn $9-a^2=0$ ist. Damit das Gleichungssystem
überhaupt eine Lösungen hat, muss die letzte Zeile im Gaussalgorithmus verschwinden,
also $b=4a$. Ist dies der Fall, dann sind alle Punkte der Geraden 
\[
\left\{(x,y)\left|y=-\frac1ax+\frac4a\right.\right\}
\]
Lösungen des Gleichungssystems.
\qedhere
\end{teilaufgaben}
\end{loesung}
