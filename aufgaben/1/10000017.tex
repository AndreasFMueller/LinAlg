Wieviele Lösungen hat das Gleichungssystem
\[
\begin{linsys}{3}
7x&+&5y&+&3z&=&4\\
x&+&2y&+&3z&=&1\\
3x&+&3y&+&3z&=&2\\
\end{linsys}
\]
Geben Sie die Lösungsmenge in Vektorschreibweise an.

\begin{loesung}
Addiert man des Doppelte der dritten Zeile zur ersten und teilt
durch drei, erhält man die dritte Zeile. Die Zeilen sind also
linear abhängig, man erwartet
unendlich viele Lösungen. Dies kann man auch mit Hilfe des
Gauss-Algorithmus herausfinden. Da die $7$ zu Beginn der ersten
Zeile zu ``schmerzhaften'' Brüchen führen würde, vertauschen
wir die ersten zwei Zeilen zu Beginn der Rechnung
\begin{align*}
\begin{tabular}{|ccc|c|}
\hline
7&5&3&4\\
1&2&3&1\\
3&3&3&2\\
\hline
\end{tabular}
&\rightarrow
\begin{tabular}{|ccc|c|}
\hline
1&2&3&1\\
7&5&3&4\\
3&3&3&2\\
\hline
\end{tabular}
\rightarrow
\begin{tabular}{|ccc|c|}
\hline
1&2&3&1\\
0&$-9$&$-18$&$-3$\\
0&$-3$&$-6$&$-1$\\
\hline
\end{tabular}
\rightarrow
\begin{tabular}{|ccc|c|}
\hline
1&2&3&1\\
0&1&2&$\frac13$\\
0&0&0&0\\
\hline
\end{tabular}
\\
&\rightarrow
\begin{tabular}{|ccc|c|}
\hline
1&0&$-1$&$\frac13$\\
0&1&2&$\frac13$\\
0&0&0&0\\
\hline
\end{tabular}
\end{align*}
Damit bestätigt sich, dass das Gleichungssystem unendlich viele Lösungen
hat.
Aus dem letzten Tableau liest man ab, dass die Variable $z$ frei gewählt
werden kann, und dass $x$ und $y$ daraus mit Hilfe der Gleichungen
\begin{align*}
x&=\frac13+z\\
y&=\frac13-2z
\end{align*}
bestimmt werden können.
Die Lösungsmenge ist also
\[
\mathbb L =\left\{\left.
\begin{pmatrix}\frac13\\\frac13\\0\end{pmatrix}
+
t\begin{pmatrix}1\\-2\\1\end{pmatrix}
\right| t\in\mathbb R
\right\}
\qedhere
\]
\end{loesung}

