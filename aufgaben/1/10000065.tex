Betrachten Sie das Gleichungssystem
\[
\begin{linsys}{3}
3x &+&  9y &+&      6z &=& 6   \\
4x &+& 16y &+& (4t+4)z &=& 8t-4\\
2x &+&  7y &+& (2t+2)z &=& 4t-1
\end{linsys}
\]
\begin{teilaufgaben}
\item
Für welche Werte von $t$ ist das Gleichungssystem singulär?
\item
Finden Sie die Lösung für den Fall, dass die Koeffizientenmatrix regulär ist.
\item
Bestimmen Sie die Lösungsmenge in dem Fall, wo das Gleichungssystem singulär
ist.
\end{teilaufgaben}

\thema{Gauss-Algorithmus}
\themaL{Losungsmenge}{Lösungsmenge}


\begin{loesung}
Wir führen den Gauss-Algorithmus für das Gleichungssystem in Tableau-Form
durch:
\begin{align*}
\begin{tabular}{|>{$}c<{$}>{$}c<{$}>{$}c<{$}|>{$}c<{$}|}
\hline
 3 &  9 & 6    & 6    \\
 4 & 16 & 4t+4 & 8t-4 \\
 2 &  7 & 2t+2 & 4t-1 \\
\hline
\end{tabular}
&\rightarrow
\begin{tabular}{|>{$}c<{$}>{$}c<{$}>{$}c<{$}|>{$}c<{$}|}
\hline
 1 &  3 & 2    & 2     \\
 0 &  4 & 4t-4 & 8t-12 \\
 0 &  1 & 2t-2 & 4t-5  \\
\hline
\end{tabular}
\\
&\rightarrow
\begin{tabular}{|>{$}c<{$}>{$}c<{$}>{$}c<{$}|>{$}c<{$}|}
\hline
 1 &  3 &    2 &    2 \\
 0 &  1 &  t-1 & 2t-3 \\
 0 &  0 &  t-1 & 2t-2 \\
\hline
\end{tabular}
\intertext{An dieser Stelle können wir nicht weiterfahren, falls $t=1$ ist,
für diesen Wert ist das Gleichungssystem singulär.
Dies beantwortet Teilaufgabe b).
Wir führen den Algorithmus unter der Annahme $t\ne 1$ weiter.
}
&\rightarrow
\begin{tabular}{|>{$}c<{$}>{$}c<{$}>{$}c<{$}|>{$}c<{$}|}
\hline
 1 &  3 &    0 & -2 \\
 0 &  1 &    0 & -1 \\
 0 &  0 &    1 &  2 \\
\hline
\end{tabular}
\\
&\rightarrow
\begin{tabular}{|>{$}c<{$}>{$}c<{$}>{$}c<{$}|>{$}c<{$}|}
\hline
 1 &  0 &    0 &  1 \\
 0 &  1 &    0 & -1 \\
 0 &  0 &    1 &  2 \\
\hline
\end{tabular}
\end{align*}
\begin{teilaufgaben}
\item
Nach ober Rechung ist die Matrix singulär genau für $t=1$.
\item
Für $t\ne 1$ kann man den Gauss-Algorithmus wie oben durchführen und
erhält die Lösung
\[
x = \begin{pmatrix*}[r]1\\-1\\2\end{pmatrix*}.
\]
\item
Um die Lösungsmenge für die Teilaufgabe c) zu bestimmen, setzen wir den
Wert $t=1$ ein und führen den Gauss-Algorithmus unter dieser Voraussetzung
weiter:
\begin{align*}
\begin{tabular}{|>{$}c<{$}>{$}c<{$}>{$}c<{$}|>{$}c<{$}|}
\hline
 1 &  3 &    2 &    2 \\
 0 &  1 &    0 &   -1 \\
 0 &  0 &    0 &    0 \\
\hline
\end{tabular}
&\rightarrow
\begin{tabular}{|>{$}c<{$}>{$}c<{$}>{$}c<{$}|>{$}c<{$}|}
\hline
 1 &  0 &    2 &    5 \\
 0 &  1 &    0 &   -1 \\
 0 &  0 &    0 &    0 \\
\hline
\end{tabular}
\end{align*}
Daraus liest man die Lösungsmenge
\[
\mathbb{L}
=
\left\{\left.
\begin{pmatrix}x\\y\\z\end{pmatrix}
=
\begin{pmatrix*}[r]5\\-1\\0\end{pmatrix*}
+z
\begin{pmatrix*}[r]-2\\0\\1\end{pmatrix*}
\;\right|
z\in\mathbb{R}
\right\}
\]
ab.
\qedhere
\end{teilaufgaben}
\end{loesung}

\begin{bewertung}
Gauss-Algorithmus ({\bf G}) 2 Punkt,
Bestimmung des Wertes $t=1$ ({\bf T}) 1 Punkt,
Lösung $x$ ({\bf X}) 1 Punkt,
Abschluss des Gauss-Algorithmus im singulären Fall ({\bf S}) 1 Punkt,
Lösungsmenge ({\bf L}) 1 Punkt.
\end{bewertung}

