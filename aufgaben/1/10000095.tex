Multiplizieren Sie aus:
\begin{teilaufgaben}
\item $(A+B)(A+B)$
\item $(A+B)(A-B)$
\item $(I+A)^3$
\end{teilaufgaben}

\begin{loesung}
\begin{teilaufgaben}
\item $(A+B)(A+B) = A^2 + AB + BA + B^2$
\item $(A+B)(A-B) = A^2-BA+BA-B^2$
\item Die 3. Potenz kann durch wiederholtes Ausmultiplizieren berechnet werden:
\begin{align*}
(I+A)^3 
&=
(I+A)(I+A)(I+A)
\\
&=
(I^2+IA+AI+A^2)(I+A)
=
(I+2A+A^2)(I+A)
\\
&=
I^2+2AI+A^2I+IA+2A^2+A^3
=
I+2A+A^2+A+2A^2+A^3
\\
&=
I+3A+3A^2+A^3
\qedhere
\end{align*}
\end{teilaufgaben}
\end{loesung}
