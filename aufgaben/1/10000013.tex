Betrachten Sie das folgende Netzwerk
\[
\entrymodifiers={++[o][F]}
\xymatrix{
1\ar[r]^1 \ar[d]^2 \ar[dr]^3
	&2\ar[d]^5 \ar[dl]^4
\\
3 \ar[r]^6
	&4
}
\]
\begin{teilaufgaben}
\item Berechnen Sie die $\partial$-Matrix dieses Netzwerkes.
\item Bestimmen Sie die Zyklen.
\item Welche Ströme sind frei wählbar? Bestimmen Sie die Gleichungen dafür.
\item Berechnen Sie den Laplace-Operator $\partial\partial^t$ 
durch Ausmultiplizieren des Matrizenproduktes.
\item Berechnen Sie den Laplace-Operator $\partial S\partial^t$, wenn
die Leitwerte proportional zur Länge der Drähte in der graphischen
Darstellung des Netzwerkes sind.
\end{teilaufgaben}

\thema{Zyklen}
\thema{Matrizenprodukt}
\thema{rref}

\begin{loesung}
\begin{teilaufgaben}
\item Die Matrix $\partial$ ist eine $4\times 6$-Matrix,
sie codiert die Verbindungen im Graphen:
\[
\partial=
\begin{pmatrix}
-1&-1&-1& 0& 0& 0\\
 1& 0& 0&-1&-1& 0\\
 0& 1& 0& 1& 0&-1\\
 0& 0& 1& 0& 1& 1
\end{pmatrix}
\]
\item Die Zyklen findet man als Lösungen von $\partial z=0$.
Der Gauss-Algorithmus ergibt das folgende Tableau
\[
\begin{tabular}{|>{$}c<{$}>{$}c<{$}>{$}c<{$}>{$}c<{$}>{$}c<{$}>{$}c<{$}|}
\hline
   1&  0&  0& -1& -1&  0\\
   0&  1&  0&  1&  0& -1\\
   0&  0&  1&  0&  1&  1\\
   0&  0&  0&  0&  0&  0\\
\hline
\end{tabular}
\]
Offenbar sind die drei Variablen $z_4$, $z_5$ und $z_6$ frei wählbar.
Wählt man jeweils eine diese Variablen als $1$, die anderen als $0$, bekommt
man die folgenden drei Zyklen als Lösung:
\[
\begin{pmatrix}
1\\-1\\0\\1\\0\\0
\end{pmatrix},
\qquad
\begin{pmatrix}
1\\0\\-1\\0\\1\\0
\end{pmatrix},
\qquad
\text{und}
\qquad
\begin{pmatrix}
0\\1\\-1\\0\\0\\1
\end{pmatrix}.
\]
Diesen Vektoren entsprechen die Zyklen $1\to2\to3\to 1$, $1\to 2\to 4\to 1$
und $1\to 3\to 4\to 1$.
\item
Dazu braucht man zunächst die Matrix 
\[
Z=
\begin{pmatrix}
 1& 1& 0\\
-1& 0& 1\\
 0&-1&-1\\
 1& 0& 0\\
 0& 1& 0\\
 0& 0& 1
\end{pmatrix}
\]
Die frei wählbaren Ströme ergeben sich als Lösung der
Gleichung
\[
Z^tRZj=Ze
\]
Wir berechnen $Z^tRZ$:
\begin{align*}
Z^tRZ&=
\begin{pmatrix}
   1& -1&  0&  1&  0&  0\\
   1&  0& -1&  0&  1&  0\\
   0&  1& -1&  0&  0&  1
\end{pmatrix}
\begin{pmatrix}
R_1&   &   &   &   &   \\
   &R_2&   &   &   &   \\
   &   &R_3&   &   &   \\
   &   &   &R_4&   &   \\
   &   &   &   &R_5&   \\
   &   &   &   &   &R_6
\end{pmatrix}
\begin{pmatrix}
 1& 1& 0\\
-1& 0& 1\\
 0&-1&-1\\
 1& 0& 0\\
 0& 1& 0\\
 0& 0& 1
\end{pmatrix}
\\
&=
\begin{pmatrix}
 R_1&-R_2&   0& R_4&   0&   0\\
 R_1&   0&-R_3&   0& R_5&   0\\
   0& R_2&-R_3&   0&   0& R_6
\end{pmatrix}
\begin{pmatrix}
 1& 1& 0\\
-1& 0& 1\\
 0&-1&-1\\
 1& 0& 0\\
 0& 1& 0\\
 0& 0& 1
\end{pmatrix}
\\
&=
\begin{pmatrix}
R_1+R_2+R_4&R_1&-R_2\\
R_1&R_1+R_3+R_5&R_3\\
-R_2&R_3&R_1+R_2+R_3
\end{pmatrix}
\end{align*}
Damit sind alle Matrizen des Gleichungssystems bekannt, es kann mit
dem Computer gelöst werden.
\item Ausmultiplizieren ergibt
\[
\begin{pmatrix}
-1&-1&-1& 0& 0& 0\\
 1& 0& 0&-1&-1& 0\\
 0& 1& 0& 1& 0&-1\\
 0& 0& 1& 0& 1& 1
\end{pmatrix}
\begin{pmatrix}
  -1&  1&  0&  0\\
  -1&  0&  1&  0\\
  -1&  0&  0&  1\\
   0& -1&  1&  0\\
   0& -1&  0&  1\\
   0&  0& -1&  1
\end{pmatrix}
=\begin{pmatrix}
   3& -1& -1& -1\\
  -1&  3& -1& -1\\
  -1& -1&  3& -1\\
  -1& -1& -1&  3
\end{pmatrix}
\]
Die letzte Matrix sagt, was man auch von blossem Auge sehen kann,
nämlich dass jeder Knoten mit drei Nachbarn verbunden ist.
\item In diesem Fall sind die diagonalen Drähte länger, haben
als Widerstand $\sqrt{2}$ statt $1$. Der Laplace-Operator wird 
damit
\begin{align*}
\partial S\partial^t
&=
\begin{pmatrix}
-1&-1&-1& 0& 0& 0\\
 1& 0& 0&-1&-1& 0\\
 0& 1& 0& 1& 0&-1\\
 0& 0& 1& 0& 1& 1
\end{pmatrix}
\begin{pmatrix}
1& &                &                & & \\
 &1&                &                & & \\
 & &\frac{\sqrt{2}}2&                & & \\
 & &                &\frac{\sqrt{2}}2& & \\
 & &                &                &1& \\
 & &                &                & &1
\end{pmatrix}
\begin{pmatrix}
  -1&  1&  0&  0\\
  -1&  0&  1&  0\\
  -1&  0&  0&  1\\
   0& -1&  1&  0\\
   0& -1&  0&  1\\
   0&  0& -1&  1
\end{pmatrix}
\\
&=
\begin{pmatrix}
-1&-1&-\frac{\sqrt{2}}2&                0& 0& 0\\
 1& 0&                0&-\frac{\sqrt{2}}2&-1& 0\\
 0& 1&                0& \frac{\sqrt{2}}2& 0&-1\\
 0& 0& \frac{\sqrt{2}}2&                0& 1& 1
\end{pmatrix}
\begin{pmatrix}
  -1&  1&  0&  0\\
  -1&  0&  1&  0\\
  -1&  0&  0&  1\\
   0& -1&  1&  0\\
   0& -1&  0&  1\\
   0&  0& -1&  1
\end{pmatrix}
\\
&=
\begin{pmatrix}
 1+\frac{\sqrt{2}}2&                -1&                -1& -\frac{\sqrt{2}}2\\
-1                 &2+\frac{\sqrt{2}}2& -\frac{\sqrt{2}}2&                -1\\
-1                 & -\frac{\sqrt{2}}2&2+\frac{\sqrt{2}}2&                -1\\
  -\frac{\sqrt{2}}2&                -1&                -1&2+\frac{\sqrt{2}}2
\end{pmatrix}.
\qedhere
\end{align*}
\end{teilaufgaben}
\end{loesung}
