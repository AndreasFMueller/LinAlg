Für welche Werte der Parameter $c$ und $d$ hat das Gleichungssystem
mit der Matrix
\[
A=\begin{pmatrix}
1&    5&    1\\
c& 5c+2& c-10\\
9&   47& c- 1\\
\end{pmatrix}
\qquad\text{und rechter Seite}\qquad
b=\begin{pmatrix}
     42\\
42c- 48\\
 6c+d\\
\end{pmatrix}
\]
unendlich viele Lösungen?
Geben Sie auch die Lösungsmenge an.

\begin{loesung}
Die Durchführung des Gauss-Algorithmus ergibt
\begin{align*}
\begin{tabular}{|>{$}c<{$}>{$}c<{$}>{$}c<{$}|>{$}c<{$}|}
\hline
1&    5&    1&     42\\
c& 5c+2& c-10&42c -48\\
9&   47& c- 1& 6c+d\\
\hline
\end{tabular}
&\rightarrow
\begin{tabular}{|>{$}c<{$}>{$}c<{$}>{$}c<{$}|>{$}c<{$}|}
\hline
1&    5&    1&     42\\
0&    2&  -10&    -48\\
0&    2& c-10& 6c -378+d\\
\hline
\end{tabular}
\\
&\rightarrow
\begin{tabular}{|>{$}c<{$}>{$}c<{$}>{$}c<{$}|>{$}c<{$}|}
\hline
1&    5&    1&     42\\
0&    1&   -5&    -24\\
0&    0&    c& 6c -330 +d\\
\hline
\end{tabular}
\end{align*}
Damit das Gleichungssystem unendlich viele Lösungen hat, muss eine
Nullzeile entstehen, was nur noch möglich ist, wenn $c=0$ ist.
Damit das Gleichungssystem mit $c=0$ überhaupt eine Lösung hat, muss
das Element in der rechten unteren Ecke des Gauss-Tableau verschwinden,
also $-330+d=0$ oder $d=330$.
Damit sind die Werte der Parameter $c$ und $d$ bestimmt.

Um die Lösungsmenge zu bestimmen, muss man jetzt den Gauss-Algorithmus
noch zu Ende führen:
\begin{align*}
\begin{tabular}{|>{$}c<{$}>{$}c<{$}>{$}c<{$}|>{$}c<{$}|}
\hline
1&    5&    1& 42\\
0&    1&   -5&-24\\
0&    0&    0&  0\\
\hline
\end{tabular}
&\rightarrow
\begin{tabular}{|>{$}c<{$}>{$}c<{$}>{$}c<{$}|>{$}c<{$}|}
\hline
1&    0&   26&162\\
0&    1&   -5&-24\\
0&    0&    0&  0\\
\hline
\end{tabular}
\end{align*}
Daraus liest man die Lösungsmenge
\[
\mathbb{L}
=
\left\{
\left.
x
=
\begin{pmatrix}162\\-24\\0\end{pmatrix}
+
z\begin{pmatrix}-26\\5\\1\end{pmatrix}
\;
\right|
\;z\in\mathbb{R}
\right\}
\qedhere
\]
\end{loesung}

\begin{bewertung}
Gauss-Algorithmus ({\bf G}) 2 Punkte,
Kriterium und Wert für $c$ ({\bf C}) 1 Punkt,
Kriterium und Wert für $d$ ({\bf D}) 1 Punkt,
Gauss-Algorithmus zuende führen ({\bf Z}) 1 Punkt,
Lösungsmenge ablesen ({\bf L}) 1 Punkt.
\end{bewertung}
