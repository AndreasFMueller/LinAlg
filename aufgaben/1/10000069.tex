Berechnen Sie die ersten sieben Näherungsbrüche des Kettenbruchs
\[
x
=
1+\cfrac{1}{2+\cfrac{1}{3+\cfrac{1}{4+\cfrac{1}{5+\cfrac{1}{6+\cfrac{1}{\dots}}}}}}
\]
mit Hilfe der Matrizennotation.

\begin{loesung}
Die ersten Näherungsbrüche sind
\[
\frac{p_0}{q_0} = 1,\qquad
\frac{p_1}{q_1} = 1+\cfrac{1}{2} = \frac{3}{2},\qquad
\frac{p_2}{q_2} = 1 + \cfrac{1}{2+\cfrac{1}{3}} = 1+\cfrac{3}{7}=\cfrac{10}{7},\dots
\]
Diese Brüche können auch mit Hilfe der Matrizenmultiplikation gewonnen werden.
Dazu definieren wir die Matrix 
\[
B_k
=
\begin{pmatrix}
0&1\\
1&k+1
\end{pmatrix}
.
\]
Als Startwert der Folge benötigen wir
\begin{align*}
\begin{pmatrix}
p_0&p_1\\
q_0&q_1
\end{pmatrix}
&=
\begin{pmatrix}
1&3\\
1&2
\end{pmatrix}
\end{align*}
\begin{align*}
\begin{pmatrix}
p_1&p_2\\
q_1&q_2
\end{pmatrix}
&=
\begin{pmatrix}
p_0&p_1\\
q_0&q_1
\end{pmatrix}
B_2
=
\begin{pmatrix}
1&3\\
1&2
\end{pmatrix}
\begin{pmatrix}
0&1\\
1&3
\end{pmatrix}
=
\begin{pmatrix*}[r]
3&10\\
2& 7
\end{pmatrix*}
&
\frac{p_2}{q_2}
&=
1.\underline{4}28571428571428603
\\
\begin{pmatrix}
p_2&p_3\\
q_2&q_3
\end{pmatrix}
&=
\begin{pmatrix}
p_1&p_2\\
q_1&q_2
\end{pmatrix}
B_3
=
\begin{pmatrix*}[r]
3&10\\
2& 7
\end{pmatrix*}
\begin{pmatrix}
0&1\\
1&4
\end{pmatrix}
=
\begin{pmatrix*}[r]
10&43\\
 7&30
\end{pmatrix*}
&
\frac{p_3}{q_3}
&=
1.\underline{433}333333333333348
\\
\begin{pmatrix}
p_3&p_4\\
q_3&q_4
\end{pmatrix}
&=
\begin{pmatrix}
p_2&p_3\\
q_2&q_3
\end{pmatrix}
B_4
=
\begin{pmatrix*}[r]
10&43\\
 7&30
\end{pmatrix*}
\begin{pmatrix}
0&1\\
1&5
\end{pmatrix}
=
\begin{pmatrix}
43&225\\
30&157
\end{pmatrix}
&
\frac{p_4}{q_4}
&=
1.\underline{43312}1019108280159
\\
\begin{pmatrix}
p_4&p_5\\
q_4&q_5
\end{pmatrix}
&=
\begin{pmatrix}
p_3&p_4\\
q_3&q_4
\end{pmatrix}
B_5
=
\begin{pmatrix}
43&225\\
30&157
\end{pmatrix}
\begin{pmatrix}
0&1\\
1&6
\end{pmatrix}
=
\begin{pmatrix*}[r]
225&1393\\
157& 972
\end{pmatrix*}
&
\frac{p_5}{q_5}
&=
1.\underline{433127}572016460904
\\
\begin{pmatrix}
p_5&p_6\\
q_5&q_6
\end{pmatrix}
&=
\begin{pmatrix}
p_4&p_5\\
q_4&q_5
\end{pmatrix}
B_6
=
\begin{pmatrix*}[r]
225&1393\\
157& 972
\end{pmatrix*}
\begin{pmatrix}
0&1\\
1&7
\end{pmatrix}
=
\begin{pmatrix*}[r]
1393&9976\\
 972&6961
\end{pmatrix*}
&
\frac{p_6}{q_6}
&=
1.\underline{43312742}4220657975
\\
\begin{pmatrix}
p_6&p_7\\
q_6&q_7
\end{pmatrix}
&=
\begin{pmatrix}
p_5&p_6\\
q_5&q_6
\end{pmatrix}
B_7
=
\begin{pmatrix*}[r]
1393&9976\\
 972&6961
\end{pmatrix*}
\begin{pmatrix}
0&1\\
1&8
\end{pmatrix}
=
\begin{pmatrix}
9976&81201\\
6961&56660
\end{pmatrix}
&
\frac{p_7}{q_7}
&=
1.\underline{4331274267}56088951
\\
&&
\frac{p_8}{q_8}
&=
1.\underline{43312742672}1944930
\\
&&
\frac{p_9}{q_9}
&=
1.\underline{43312742672231}5079
\\
&&
\frac{p_{10}}{q_{10}}
&=
1.\underline{43312742672231}1748
\end{align*}
Die Konvergenz der Näherungsbrüche ist recht schnell und die Genauigkeit
bereits bei kleinen Nennern sehr gut.
\end{loesung}

