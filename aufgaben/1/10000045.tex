Betrachten Sie die ganzzahlige Matrix
\[
A=\begin{pmatrix}
1&3&1\\
1&4&c\\
2&8&5
\end{pmatrix}.
\]
Für welche Werte des Parameters $c$ sind alle Einträge der inversen
Matrix ganzzahlig?
Bestimmen Sie die inverse Matrix in all diesen Fällen.

\thema{inverse Matrix}
\thema{Matrix mit Parameter}

%A =
%
%   1   3   1
%   1   4   2
%   2   8   5
%
%ans =
%
%   4  -7   2
%  -1   3  -1
%   0  -2   1

\begin{loesung}
Wir berechnen die inverse Matrix mit dem Gaussalgorithmus.
\begin{align*}
\begin{tabular}{|>{$}c<{$}>{$}c<{$}>{$}c<{$}|>{$}c<{$}>{$}c<{$}>{$}c<{$}|}
\hline
1&3&1&1&0&0\\
1&4&c&0&1&0\\
2&8&5&0&0&1\\
\hline
\end{tabular}
&
\rightarrow
\begin{tabular}{|>{$}c<{$}>{$}c<{$}>{$}c<{$}|>{$}c<{$}>{$}c<{$}>{$}c<{$}|}
\hline
1&3&  1& 1&0&0\\
0&1&c-1&-1&1&0\\
0&2&  3&-2&0&1\\
\hline
\end{tabular}
\rightarrow
\begin{tabular}{|>{$}c<{$}>{$}c<{$}>{$}c<{$}|>{$}c<{$}>{$}c<{$}>{$}c<{$}|}
\hline
1&3&  1 & 1& 0&0\\
0&1&c-1 &-1& 1&0\\
0&0&5-2c& 0&-2&1\\
\hline
\end{tabular}
\end{align*}
Im nächsten Schritt muss durch das Pivot-Element $5-2c$ geteilt werden.
Das Element in der rechten unteren Ecke wird danach nicht mehr ändern,
$c$ muss also so gewählt werden, dass $1/(5-2c)$ ganzzahlig ist.
Dies ist nur mögliche, wenn
\[
5-2c=\pm 1
\qquad\Rightarrow\qquad
c= \frac{5 \mp 1}{2}=\begin{cases}2\\3\end{cases}
\]
ist.
Wir berechnen daher die weiteren Schritt des Gauss-Algorithmus für diese
beiden konkreten Werte.

Für $c=2$ erhalten wir
\begin{align*}
\begin{tabular}{|>{$}c<{$}>{$}c<{$}>{$}c<{$}|>{$}c<{$}>{$}c<{$}>{$}c<{$}|}
\hline
1&3&  1 & 1& 0&0\\
0&1&  1 &-1& 1&0\\
0&0&  1 & 0&-2&1\\
\hline
\end{tabular}
&\rightarrow
\begin{tabular}{|>{$}c<{$}>{$}c<{$}>{$}c<{$}|>{$}c<{$}>{$}c<{$}>{$}c<{$}|}
\hline
1&3&  0 & 1& 2&-1\\
0&1&  0 &-1& 3&-1\\
0&0&  1 & 0&-2& 1\\
\hline
\end{tabular}
\rightarrow
\begin{tabular}{|>{$}c<{$}>{$}c<{$}>{$}c<{$}|>{$}c<{$}>{$}c<{$}>{$}c<{$}|}
\hline
1&0&  0 & 4&-7& 2\\
0&1&  0 &-1& 3&-1\\
0&0&  1 & 0&-2& 1\\
\hline
\end{tabular}
\end{align*}

Für $c=3$ bekommt man
\begin{align*}
\begin{tabular}{|>{$}c<{$}>{$}c<{$}>{$}c<{$}|>{$}c<{$}>{$}c<{$}>{$}c<{$}|}
\hline
1&3&  1 & 1& 0&0\\
0&1&  2 &-1& 1&0\\
0&0& -1 & 0&-2&1\\
\hline
\end{tabular}
&\rightarrow
\begin{tabular}{|>{$}c<{$}>{$}c<{$}>{$}c<{$}|>{$}c<{$}>{$}c<{$}>{$}c<{$}|}
\hline
1&3&  0 & 1&-2& 1\\
0&1&  0 &-1&-3& 2\\
0&0&  1 & 0& 2&-1\\
\hline
\end{tabular}
\rightarrow
\begin{tabular}{|>{$}c<{$}>{$}c<{$}>{$}c<{$}|>{$}c<{$}>{$}c<{$}>{$}c<{$}|}
\hline
1&0&  0 & 4& 7&-5\\
0&1&  0 &-1&-3& 2\\
0&0&  1 & 0& 2&-1\\
\hline
\end{tabular}
\end{align*}
Wir haben also zwei Werte $c$ gefunden, für die die inverse Matrix
ganzzahlig ist
\[
\begin{aligned}
c&=2
&&\Rightarrow
&
A^{-1}
&=\begin{pmatrix}
 4&-7& 2\\
-1& 3&-1\\
 0&-2& 1
\end{pmatrix}
\\
c&=3
&&\Rightarrow
&
A^{-1}
&=\begin{pmatrix}
 4& 7&-5\\
-1&-3& 2\\
 0& 2&-1
\end{pmatrix}
\end{aligned}
\]
\end{loesung}

\begin{bewertung}
Gaussalgorithmus ({\bf G}) 2 Punkte,
zwei Werte für $c$ ({\bf C}) 2 Punkte,
zwei inverse Matrizen $A^{-1}$ ({\bf I}) 2 Punkte.
\end{bewertung}




