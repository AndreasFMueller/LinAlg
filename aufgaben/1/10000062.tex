Gegeben ist das von den Parametern $u$ und $v$ abhängige lineare
Gleichungssystem
\[
\begin{linsys}{3}
3x&+&   12 y&+& 3    z&=&     3,\\
2x&+&(u+ 7)y&+&(5-3u)z&=& v-u+1,\\
5x&+&   19 y&+& 10    z&=&    6.
\end{linsys}
\]
\begin{teilaufgaben}
\item
Finden Sie die Lösung im Falle $u=-1$ und $v=-4$.
\item
Für welche Werte des Parameters $u$ ist das Gleichungsystem nicht eindeutig
lösbar?
\item
Wie muss $v$ gewählt werden, damit das Gleichungssystem immer mindestens
eine Lösung hat?
\item
Bestimmen Sie die Lösungsmenge in dem Falle, in dem es unendlich viele
Lösungen gibt.
\end{teilaufgaben}

\thema{Gauss-Algorithmus}
\themaL{Losungsmenge}{Lösungsmenge}

\begin{loesung}
Wir wenden den Gauss-Algorithmus auf das zugehörige Tableau an:
\begin{align}
\begin{tabular}{|>{$}c<{$} >{$}c<{$} >{$}c<{$}|>{$}c<{$}|}
\hline
3&  12& 3   &     3\\
2&u+ 7& 5-3u& v-u+1\\
5&  19&10   &     6\\
\hline
\end{tabular}
&\rightarrow
\begin{tabular}{|>{$}c<{$} >{$}c<{$} >{$}c<{$}|>{$}c<{$}|}
\hline
1&   4& 1   &      1\\
0&u- 1& 3-3u&  v-u-1\\
0&  -1& 5   &      1\\
\hline
\end{tabular}
\notag
\intertext{An dieser Stelle könnte $u-1=0$ ein nicht geeignetes
Pivot sein, wir verschieben die Behandlung dieses Falles auf 
später, indem wir die zweite und dritte Zeile vertauschen}
\begin{tabular}{|>{$}c<{$} >{$}c<{$} >{$}c<{$}|>{$}c<{$}|}
\hline
1&   4&    1   &     1 \\
0&  -1&    5   &     1 \\
0&u- 1& -3(u-1)&v-2-(u-1)\\
\hline
\end{tabular}
&\rightarrow
\begin{tabular}{|>{$}c<{$} >{$}c<{$} >{$}c<{$}|>{$}c<{$}|}
\hline
1&   4& 1     &    1\\
0&   1&-5     &   -1\\
0&   0& 2(u-1)&  v-2\\
\hline
\end{tabular}
\label{10000062:singulaer}
\intertext{Wenn $u-1=0$ erhalten wir eine Nullzeile.
Das Gleichungssystem wird nur eine Lösung haben, wenn die rechte Seite
ebenfalls verschwindet.
Wir nehmen für die weitere Rechnung an, dass $u\ne 1$.}
\begin{tabular}{|>{$}c<{$} >{$}c<{$} >{$}c<{$}|>{$}c<{$}|}
\hline
1&   4& 1& 1\\
0&   1&-5&-1\\
0&   0& 1&\frac{v-2}{2(u-1)}\\
\hline
\end{tabular}
&\rightarrow
\begin{tabular}{|>{$}c<{$} >{$}c<{$} >{$}c<{$}|>{$}c<{$}|}
\hline
1&   4& 0& 1- q\\
0&   1& 0&-1+5q\\
0&   0& 1&    q\\
\hline
\end{tabular}
\notag
\\
&\rightarrow
\begin{tabular}{|>{$}c<{$} >{$}c<{$} >{$}c<{$}|>{$}c<{$}|}
\hline
1&   0& 0& 5-21q\\
0&   1& 0&-1+ 5q\\
0&   0& 1&     q\\
\hline
\end{tabular}
\notag
\end{align}
Dabei haben wir den etwas unschönen Bruch $q=\frac{v-2}{2(u-1)}$ abgekürzt.
Nach dieser Rechnung können wir jetzt die einzelnen Fragen beantworten.
\begin{teilaufgaben}
\item
Im Falle $u=-1$ und $v=-4$ wird
\[
q = \frac{v-2}{-4} = \frac32.
\]
Daraus lassen sich die Unbekannten $x$, $y$ und $z$ leicht bestimmen
\begin{align*}
x &=5-21=-\frac{53}{2},\\
y&=-1+5\cdot \frac32=\frac{13}{2}\\
\text{und}\quad
z&=\frac32.
\end{align*}
\item
Das Gleichungssystem ist nicht eindeutig lösbar, wenn $u-1=0$ ist, also für
$u=1$.
\item
Für $u\ne 1$ spielt der Wert von $v$ keine Rolle, nur für $u=1$ muss
genauer untersucht werden, was auf der rechten Seite der Nullzeile steht.
Dies kann im Schritt \eqref{10000062:singulaer} abgelesen werden.
Dort steht, dass die rechte Seite für $u=1$ zu $v-2$ wird, d.~h.~es gibt
unendlich viele Lösungen genau dann, wenn $v=2$.
\item
Um die Lösungsmenge abzulesen, müssen wir ausgehend von
\eqref{10000062:singulaer} zu einem Schlusstableau gelangen:
\[
\begin{tabular}{|>{$}c<{$} >{$}c<{$} >{$}c<{$}|>{$}c<{$}|}
\hline
1&   4& 1     &    1\\
0&   1&-5     &   -1\\
0&   0& 0     &    0\\
\hline
\end{tabular}
\rightarrow
\begin{tabular}{|>{$}c<{$} >{$}c<{$} >{$}c<{$}|>{$}c<{$}|}
\hline
1&   0&21     &    5\\
0&   1&-5     &   -1\\
0&   0& 0     &    0\\
\hline
\end{tabular}
\]
woraus wir die Lösungsmenge
\[
\mathbb L
=
\left\{
\left.
\begin{pmatrix}5\\-1\\0\end{pmatrix}
+
z\begin{pmatrix}-21\\5\\1\end{pmatrix}
\; \right| \;
z\in\mathbb R
\right\}.
\qedhere
\]
\end{teilaufgaben}
In Teilaufgabe a) kann man natürlich auch die Werte für $u$ und $v$ 
einsetzen, und den Gauss-Algorithmus danach durchführen:
\begin{align*}
\begin{tabular}{|>{$}c<{$} >{$}c<{$} >{$}c<{$}|>{$}c<{$}|}
\hline
3&  12& 3&  3\\
2&   6& 8& -2\\
5&  19&10&  6\\
\hline
\end{tabular}
&\rightarrow
\begin{tabular}{|>{$}c<{$} >{$}c<{$} >{$}c<{$}|>{$}c<{$}|}
\hline
1&   4& 1&  1\\
0&  -2& 6& -4\\
0&  -1& 5&  1\\
\hline
\end{tabular}
\rightarrow
\begin{tabular}{|>{$}c<{$} >{$}c<{$} >{$}c<{$}|>{$}c<{$}|}
\hline
1&   4& 1&  1\\
0&   1&-3&  2\\
0&   0& 2&  3\\
\hline
\end{tabular}
\\
\rightarrow
\begin{tabular}{|>{$}c<{$} >{$}c<{$} >{$}c<{$}|>{$}c<{$}|}
\hline
1&   4& 0& -\frac12\\
0&   1& 0&  \frac{13}2\\
0&   0& 1&  \frac32\\
\hline
\end{tabular}
&\rightarrow
\begin{tabular}{|>{$}c<{$} >{$}c<{$} >{$}c<{$}|>{$}c<{$}|}
\hline
1&   0& 0& -\frac{53}2\\
0&   1& 0&  \frac{13}2\\
0&   0& 1&  \frac32\\
\hline
\end{tabular}
\end{align*}
\end{loesung}

\begin{bewertung}
Gaussalgorithmus Vorwärtsreduktion ({\bf V}) 1 Punkt,
Rückwärtseinsetzen ({\bf R}) 1 Punkt,
Lösung in Teilaufgabe a) ({\bf A}) 1 Punkt,
Singularitätskriterium $u=1$ ({\bf S}) 1 Punkt,
Lösbarkeitsbedingung $v\in\mathbb R$ ({\bf V}) 1 Punkt,
Lösungsmenge ({\bf L}) 1 Punkt.
\end{bewertung}



