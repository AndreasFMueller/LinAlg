Gegeben ist das folgende Gleichungssystem mit dem Parameter $s$
\[
\begin{linsys}{3}
-x&-&y&+&2z&=&2\\
sx&+&y&+&3z&=&-1\\
-x&-&y&+&z&=& 3
\end{linsys}
\]
\begin{teilaufgaben}
\item Für welche Werte von $s$ ist das Gleichungssystem regulär?
\item Bestimmen Sie die Inverse der Koeffizientenmatrix für $s=2$.
\item Finden Sie die Lösung des Gleichungssystems für $s=2$ mit Hilfe der Inversen.
\end{teilaufgaben}

\thema{regular}{regulär}
\themaS{Gauss-Algorithmus}
\themaS{inverse Matrix} 

\begin{loesung}
\begin{teilaufgaben}
\item 
Man könnte versuchen herauszufinden, für welche Werte von $s$ die Matrix
\[
A=\begin{pmatrix}
-1& -1 &  2\\
s& 1 & 3 \\
-1& -1& 1
\end{pmatrix}
\]
singulär ist, und dazu die Determinante verwenden.
Da man das Gleichungssystem aber auch lösen muss, ist es effizienter,
gleich von Anfang an den Gauss-Algorithmus anzuwenden.
\begin{align*}
\begin{tabular}{|>{$}c<{$}>{$}c<{$}>{$}c<{$}|>{$}c<{$}>{$}c<{$}>{$}c<{$}|}
\hline
-1& -1 & 2 & 1 & 0 & 0 \\
 s&  1 & 3 & 0 & 1 & 0\\
-1& -1 & 1 & 0 & 0 & 1 \\
\hline
\end{tabular}
&
\rightarrow
\begin{tabular}{|>{$}c<{$}>{$}c<{$}>{$}c<{$}|>{$}c<{$}>{$}c<{$}>{$}c<{$}|}
\hline
 1&  1   & -2    & -1 & 0 & 0 \\
 0&  1-s &  3+2s &  s & 1 & 0\\
 0&  0   & -1    &  -1 & 0 & 1 \\
\hline
\end{tabular}
\end{align*}
Beim nächsten Gauss-Schritt muss nun durch $(1-s)$ geteilt werden. 
Dies ist nur möglich, wenn $s\neq 1$ ist.
\begin{align*}
\begin{tabular}{|>{$}c<{$}>{$}c<{$}>{$}c<{$}|>{$}c<{$}>{$}c<{$}>{$}c<{$}|}
\hline
 1&  1   & -2    & -1 & 0 & 0 \\
 0&  1 &  \frac{3+2s}{1-s} &  \frac{s}{1-s} & \frac{1}{1-s} & 0\\
 0&  0   & -1    &  -1 & 0 & 1 \\
\hline
\end{tabular}
&
\rightarrow
\begin{tabular}{|>{$}c<{$}>{$}c<{$}>{$}c<{$}|>{$}c<{$}>{$}c<{$}>{$}c<{$}|}
\hline
 1&  1   & -2    & -1 & 0 & 0 \\
 0&  1 &  \frac{3+2s}{1-s} &  \frac{s}{1-s} & \frac{1}{1-s} & 0\\
 0&  0   & 1    &  1 & 0 & -1 \\
\hline
\end{tabular}
\end{align*}
Beim Gauss-Algorithmus entstehen folglich keine Nullzeilen, wenn $s\neq 1$ ist.
Somit ist die Koeffizientenmatrix für alle $s\neq 1$ regulär.
\item
Um die Inverse der Koeffizientenmatrix zu berechnen, setzen wir im Gauss-Tableau 
für $s=2$ ein und führen den Gauss-Algorithmus zu Ende.
\begin{align*}
\begin{tabular}{|>{$}c<{$}>{$}c<{$}>{$}c<{$}|>{$}c<{$}>{$}c<{$}>{$}c<{$}|}
\hline
 1&  1 & -2 & -1 & 0 & 0 \\
 0&  1 & -7 & -2 & -1 & 0\\
 0&  0 & 1  &  1 & 0 & -1 \\
\hline
\end{tabular}
&
\rightarrow
\begin{tabular}{|>{$}c<{$}>{$}c<{$}>{$}c<{$}|>{$}c<{$}>{$}c<{$}>{$}c<{$}|}
\hline
 1&  1 & 0 & 1 & 0 & -2 \\
 0&  1 & 0 & 5 & -1 & -7\\
 0&  0 & 1  &  1 & 0 & -1 \\
\hline
\end{tabular}
\rightarrow
\begin{tabular}{|>{$}c<{$}>{$}c<{$}>{$}c<{$}|>{$}c<{$}>{$}c<{$}>{$}c<{$}|}
\hline
 1&  0 & 0 & -4 & 1 & 5 \\
 0&  1 & 0 & 5 & -1 & -7\\
 0&  0 & 1  &  1 & 0 & -1 \\
\hline
\end{tabular}
\end{align*}
Daraus liest man ab, dass
\[
A^{-1} = 
\begin{pmatrix}
-4 & 1 & 5 \\
5 & -1 & -7\\
1 & 0 & -1 \\
\end{pmatrix}.
\]
Zur Kontrolle berechnen wir
\[
AA^{-1} = 
\begin{pmatrix}
-1& -1 &  2\\
2& 1 & 3 \\
-1& -1& 1
\end{pmatrix}
\begin{pmatrix}
-4 & 1 & 5 \\
5 & -1 & -7\\
1 & 0 & -1 \\
\end{pmatrix}
=
\begin{pmatrix}
1& 0 &0\\
0& 1 & 0 \\
0&0& 1
\end{pmatrix}
\]
\item Die Lösung des Gleichungssystems kann jetzt durch Multiplikation $x=A^{-1}b$
gefunden werden:
\[
 x = A^{-1}b = 
 \begin{pmatrix}
-4 & 1 & 5 \\
5 & -1 & -7\\
1 & 0 & -1 \\
\end{pmatrix}
\begin{pmatrix}
2\\ -1\\ 3
\end{pmatrix}
= 
\begin{pmatrix}
(-4)\cdot2 + 1\cdot (-1) + 5\cdot 3\\ 
5\cdot2 + (-1)\cdot (-1) + (-7)\cdot 3\\ 
1\cdot2 + 0\cdot (-1) + (-1)\cdot 3\\ 
\end{pmatrix}
=
\begin{pmatrix}
6\\ -10\\ -1
\end{pmatrix}
\]
\end{teilaufgaben}
\end{loesung}

\begin{bewertung}
Durchführung der Vorwärtsreduktion des Gauss-Algorithmus ({\bf V}) 2 Punkte,
Bedingung für Regularität und Bestimmung des Wertes für $s$ ({\bf S}) 1 Punkt,
Durchführung des Rückwärtseinsetzen des Gauss-Algorithmus ({\bf R}) 1 Punkt,
Bestimmung der Inversen ({\bf I}) 1 Punkt,
Lösungsformel $x=A^{-1}b$ und Lösung ({\bf L}) 1 Punkt.
\end{bewertung}


