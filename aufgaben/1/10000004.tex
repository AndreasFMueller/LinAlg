Verwenden Sie die Erweiterung des Gauss-Algorithmus, die mehrere 
Gleichungssysteme gleichzeitig löst, um die Gleichungssysteme
\begin{equation}
\begin{linsys}{3}
 x&+&2y&+& 3z&=&22\\
2x&+&5y&+& 8z&=&57\\
3x&+&8y&+&14z&=&97
\end{linsys}
\end{equation}
und
\begin{equation}
\begin{linsys}{3}
 x&+&2y&+& 3z&=& 28\\
2x&+&5y&+& 8z&=& 72\\
3x&+&8y&+&14z&=&122
\end{linsys}
\end{equation}
zu lösen.


\begin{loesung}
Man arbeitet mit dem Gauss-Algorithmus in fünf Spalten:
\begin{align*}
\begin{tabular}{|>{$}c<{$}>{$}c<{$}>{$}c<{$}|>{$}c<{$}>{$}c<{$}|}
\hline
1&2& 3&22& 28\\
2&5& 8&57& 72\\
3&8&14&97&122\\
\hline
\end{tabular}
&
\rightarrow
\begin{tabular}{|>{$}c<{$}>{$}c<{$}>{$}c<{$}|>{$}c<{$}>{$}c<{$}|}
\hline
1&2& 3&22& 28\\
0&1& 2&13& 16\\
0&2& 5&31& 38\\
\hline
\end{tabular}
\rightarrow
\begin{tabular}{|>{$}c<{$}>{$}c<{$}>{$}c<{$}|>{$}c<{$}>{$}c<{$}|}
\hline
1&2& 3&22& 28\\
0&1& 2&13& 16\\
0&0& 1& 5&  6\\
\hline
\end{tabular}
\\
&
\rightarrow
\begin{tabular}{|>{$}c<{$}>{$}c<{$}>{$}c<{$}|>{$}c<{$}>{$}c<{$}|}
\hline
1&2& 0& 7& 10\\
0&1& 0& 3&  4\\
0&0& 1& 5&  6\\
\hline
\end{tabular}
\rightarrow
\begin{tabular}{|>{$}c<{$}>{$}c<{$}>{$}c<{$}|>{$}c<{$}>{$}c<{$}|}
\hline
1&0& 0& 1&  2\\
0&1& 0& 3&  4\\
0&0& 1& 5&  6\\
\hline
\end{tabular}
\end{align*}
Daraus liest man ab, dass die Gleichungssysteme die Lösungsvektoren
\[
\begin{pmatrix} 1\\3\\5 \end{pmatrix}
\qquad
\text{bzw.}
\qquad
\begin{pmatrix} 2\\4\\6 \end{pmatrix}
\]
haben.
\end{loesung}
