Die Zahlen $a$, $b$, $c$ und $d$ erfüllen
\[
\frac{a-b}{c-d}=2
\qquad\text{und}\qquad
\frac{a-c}{b-d}=3.
\]
Welche Werte kann
\[
u = \frac{a-d}{b-c}
\]
annehmen?


\begin{hinweis}
Verwenden Sie die Unbekannten $x=a-d$, $y=b-d$ und $z=c-d$.
\end{hinweis}

\begin{loesung}
Die gegebenen Bedingungen werden mit
\begin{align*}
a-b &= x-y \\
a-c &= x-z \\
b-c &= y-z
\end{align*}
zu
\[
\frac{x-y}{z} = 2 ,
\qquad
\frac{x-z}{y} = 3
\qquad\text{und}\qquad
\frac{x}{y-z}=u.
\]
Durch Multiplizieren mit den Nennern kann man diese Bedinungen in die
Form der linearen Gleichungen
\[
\renewcommand{\arraycolsep}{3pt}
\begin{array}{rcrcrcrc}
x&-& y&-&2z&=&0\\
x&-&3y&-& z&=&0\\
x&-&uy&+&uz&=&0
\end{array}.
\]
Der Gauss-Algorithmus ergibt
\bgroup
\renewcommand\arraystretch{1.3}
\begin{align*}
\begin{tabular}{| >{$}r<{$} >{$}r<{$} >{$}r<{$}| >{$}r<{$}|}
\hline
x&y&z&1\\
\hline
1&-1&-2&0\\
1&-3&-1&0\\
1&-u& u&0\\
\hline
\end{tabular}
&
\to
\begin{tabular}{| >{$}r<{$} >{$}r<{$} >{$}r<{$}| >{$}r<{$}|}
\hline
x&y&z&1\\
\hline
1& -1& -2&0\\
0& -2&  1&0\\
0&1-u&2+u&0\\
\hline
\end{tabular}
\to
\begin{tabular}{| >{$}r<{$} >{$}r<{$} >{$}r<{$}| >{$}r<{$}|}
\hline
x&y&z&1\\
\hline
1& -1& -2&0\\
0&  1& -\frac12&0\\
0&  0&\frac{5}{2}+\frac12u&0\\
\hline
\end{tabular}
\to
\begin{tabular}{| >{$}r<{$} >{$}c<{$} >{$}c<{$}| >{$}c<{$}|}
\hline
x&y&z&1\\
\hline
1&  0& -\frac52&0\\
0&  1& -\frac12&0\\
0&  0&\frac{5}{2}+\frac12u&0\\
\hline
\end{tabular}
\end{align*}
\egroup
Die letzte Gleichung bedeutet
\[
\frac12(5+u)z=0.
\]
Da $z\ne 0$ sein muss, damit die gegebenen Bedingungen überhaupt
sinnvoll sind, folgt, dass $u=-5$ sein muss.
In diesem Fall ist die dritte Gleichung linear unabhängig,
Zu gegebenem $z$ können $x$ und $y$ als
\begin{equation}
x = \frac52z,\quad y = \frac12z
\label{10000074:eqn}
\end{equation}
bestimmt werden.
Der Wert von $z$ ergibt sich, indem man irgendwelche Werte für
$d$ und $c$ wählt.
Zum Beispiel kann man $d=0$ und $c=2$ wählen, dann wird zunächst
$z=c-d=2$.
Diese Wahl ist vorteilhaft, weil damit die Nenner verschwinden.
Mit den Gleichungen~\eqref{10000074:eqn} folgen dann
$x=5$ und $y=1$.
Die Werte von $a$ und $b$ sind für die Beantwortung der Frage 
nicht nötig, dazu sind nur die Differenzen nötig:
\[
\left.
\begin{aligned}
a-d&=x=5 \\
b-c&=y-z=1-2=-1\\
\end{aligned}
\right\}
\quad\Rightarrow\quad
u
=
\frac{a-d}{b-c}
=
\frac{5}{-1}
=
-5,
\]
womit das gefundene Resultat bestätigt ist.
\end{loesung}
