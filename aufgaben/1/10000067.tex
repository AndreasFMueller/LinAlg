Bestimmen Sie einen regulären Kettenbruch für die Zahl $\sqrt{3}$.

\begin{loesung}
Wir führen den in der Vorlesung vorgestellten Prozess durch:
\begin{align*}
\sqrt{3} &= a_0 + \cfrac{1}{x_0}
&&\Rightarrow &
a_0 &=1&&\text{und}&x_0&=\frac{1}{\sqrt{3}-a_0}
\\
x_0=1.366025403784438646&=a_1+\cfrac{1}{x_1}
&&\Rightarrow &
a_1 &=1&&\text{und}&x_1&=\frac{1}{x_0-a_1}
\\
x_1=2.732050807568877293&=a_2+\cfrac{1}{x_2}
&&\Rightarrow &
a_2 &=2&&\text{und}&x_2&=\frac{1}{x_1-a_2}
\\
x_2=1.366025403784438646&=a_3+\cfrac{1}{x_3}
&&\Rightarrow &
a_3 &=1&&\text{und}&x_3&=\frac{1}{x_2-a_3}
\\
x_3=2.732050807568877293&=a_4+\cfrac{1}{x_4}
&&\Rightarrow &
a_4 &=2&&\text{und}&x_4&=\frac{1}{x_3-a_4}
\\
&\vdots
\end{align*}
Es folgt die Vermutung $\sqrt{3} = [1;1,2,1,2,1,2,\dots]$.
Um das zu überprüfen, stellt man die Gleichung für den Wert des Kettenbruches
auf.
Dabei ist es einfacher, den Kettenbruch $y=[2;1,2,1,2,\dots]$ zu untersuchen,
der genau um $1$ grösser ist.
Wir möchten also verifizieren, dass $-1+y = \sqrt{3}$ ist.
Wir erhalten für $y$ die Gleichung 
\begin{align*}
y
&=
2 + \smash{\cfrac{1}{1+\cfrac{1}{{\color{red}2+\cfrac{1}{1+\cfrac{1}{\dots}}}}}}
=
2 + \cfrac{1}{1+\cfrac{1}{{\color{red}y}}}
&&\Rightarrow& y &=2+\cfrac{1}{\cfrac{y+1}{y}}
\\
&&&&
y&=2+\cfrac{y}{1+y}
\\
&&&&
y&=\cfrac{2+3y}{1+y}
\\
&&&&
y+y^2&=2+3y
\\
&&&&
y^2-2y-2&=0
\\
&&&&
y&=1\pm\sqrt{1+2}=1\pm\sqrt{3}
\\
&&&&
-1+y&=\sqrt{3}
\qedhere
\end{align*}
\end{loesung}
