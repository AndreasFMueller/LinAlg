Gegeben sind die beiden Matrizen
\[
A
=
\begin{pmatrix}
\cos\gamma&         - \sin\gamma&0\\
\sin\gamma&\phantom{-}\cos\gamma&0\\
    0     &\phantom{-}    0     &1
\end{pmatrix}
\qquad\text{und}\qquad
B
=
\begin{pmatrix}
\phantom{-}\cos\beta&0&\sin\beta\\
\phantom{-}    0    &1&    0    \\
         - \sin\beta&0&\cos\beta
\end{pmatrix}.
\]
\begin{teilaufgaben}
\item Berechnen Sie $AB$.
\item Berechnen Sie $BA$.
\item Für welche Werte von $\beta,\gamma\in[0,2\pi)$ gilt
$AB=BA$?
\end{teilaufgaben}

\begin{hinweis}
$[0,2\pi)$ ist das rechts offene Interval der reellen Zahlen zwischen 
$0$ und $2\pi$.
Eine Zahl $x$ ist in $[0,2\pi)$ genau dann,
wenn $0\le x<2\pi$,
\end{hinweis}

\begin{loesung}
\begin{teilaufgaben}
\item Das Produkt $AB$ ist
\begin{align*}
AB
&=
\begin{pmatrix}
\cos\gamma&         - \sin\gamma&0\\
\sin\gamma&\phantom{-}\cos\gamma&0\\
    0     &\phantom{-}    0     &1
\end{pmatrix}
\begin{pmatrix}
\phantom{-}\cos\beta&0&\sin\beta\\
\phantom{-}    0    &1&    0    \\
         - \sin\beta&0&\cos\beta
\end{pmatrix}
\\
&=
\begin{pmatrix}
\cos\gamma\cos\beta&-\sin\gamma&\cos\gamma\sin\beta\\
\sin\gamma\cos\beta& \cos\gamma&\sin\gamma\sin\beta\\
         -\sin\beta&     0     &          \cos\beta
\end{pmatrix}.
\end{align*}
\item Das Produkt $BA$ ist
\begin{align*}
BA
&=
\begin{pmatrix}
\phantom{-}\cos\beta&0&\sin\beta\\
\phantom{-}    0    &1&    0    \\
         - \sin\beta&0&\cos\beta
\end{pmatrix}
\begin{pmatrix}
\cos\gamma&         - \sin\gamma&0\\
\sin\gamma&\phantom{-}\cos\gamma&0\\
    0     &\phantom{-}    0     &1
\end{pmatrix}
\\
&=
\begin{pmatrix}
 \cos\beta\cos\gamma&-\cos\beta\sin\gamma&\sin\beta\\
          \sin\gamma&          \cos\gamma&     0   \\
-\sin\beta\cos\gamma& \sin\beta\sin\gamma&\cos\beta
\end{pmatrix}.
\end{align*}
\item
Wir müssen herausfinden, für welche Werte von $\beta$ und $\gamma$ die
beiden Produkte $AB$ und $BA$ übereinstimmen.
Die Elemente auf den Diagonalen beider Matrizen stimmen bereits
überein, wir müssen also nur noch die Elemente ausserhalb der
Diagonale vergleichen:
\begin{align}
&\text{Zeile $1$, Spalte $2$:}
&
-\sin\gamma&=-\cos\beta\sin\gamma
\label{10000049:e1}
\\
&\text{Zeile $1$, Spalte $3$:}
&
\cos\gamma\sin\beta&=\sin\beta
\label{10000049:e2}
\\
&\text{Zeile $2$, Spalte $1$:}
&
\sin\gamma\cos\beta&=\sin\gamma
\label{10000049:e3}
\\
&\text{Zeile $2$, Spalte $3$:}
&
\sin\gamma\sin\beta&=0
\label{10000049:e4}
\\
&\text{Zeile $3$, Spalte $2$:}
&
-\sin\beta&=-\sin\beta\cos\gamma
\label{10000049:e5}
\\
&\text{Zeile $3$, Spalte $3$:}
&
0&=\sin\beta\sin\gamma
\label{10000049:e6}
\end{align}
Gleichungen
\eqref{10000049:e4}
und
\eqref{10000049:e6}
stimmen überein und implizieren, dass $\sin\gamma=0$ oder $\sin\beta=0$.
Gleichungen
\eqref{10000049:e2}
und
\eqref{10000049:e5}
stimmen überein, ebenso
\eqref{10000049:e1}
und
\eqref{10000049:e3}.
Damit bleiben die Gleichungen
\begin{align}
\sin\beta   \sin\gamma &=0, \label{10000049:r1}\\
\sin\beta(1-\cos\gamma)&=0\qquad\text{und} \label{10000049:r2}\\
\sin\gamma(1-\cos\beta)&=0.\label{10000049:r3}
\end{align}
Alle drei Gleichungen besagen, dass jeweils einer der Faktoren verschwinden
muss.
Sollte $\sin\gamma\ne0$ sein, folgt aus \eqref{10000049:r3}, dass $\cos\beta=1$
ist, was im verlangten Interval für $\beta$ nur für $\beta=0$  möglich ist.
Aber dann ist $\sin\beta=0$.
Analog schliesst man für den Fall $\sin\beta\ne 0$, auch in diesem Fall folgt
$\sin\gamma=0$.
Somit gilt in allen Fällen, dass $\beta$ und $\gamma$ entweder $0$ oder
$\pi$ sein müssen: $\beta,\gamma\in\{0,\pi\}$.
Die möglichen Matrizen sind daher
\[
A=\begin{pmatrix}
\pm1&   0&0\\
   0&\pm1&0\\
   0&   0&1
\end{pmatrix},
\qquad
B=\begin{pmatrix}
\pm1&0&   0\\
   0&1&   0\\
   0&0&\pm1
\end{pmatrix}.
\qedhere
\]
\end{teilaufgaben}
\end{loesung}


