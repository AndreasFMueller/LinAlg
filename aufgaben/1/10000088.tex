Was muss an den leeren Stellen im Tableau stehen, damit die Zeilen linear
abhängig sind?
\begin{teilaufgaben}
\item
\(
\begin{tabular}{|>{$}r<{$}>{$}r<{$}|}
\hline
1&-1\\
2&  \\
\hline
\end{tabular}
\)
\item
\(
\begin{tabular}{|>{$}r<{$}>{$}r<{$}>{$}r<{$}|}
\hline
1&2&3\\
0&0&1\\
2& &0\\
\hline
\end{tabular}
\)
\end{teilaufgaben}

\begin{hinweis}
Schreiben Sie eine Unbekannte in des leere Feld und führen Sie den
Gauss-Algorithmus durch.
\end{hinweis}

\begin{loesung}
\begin{teilaufgaben}
\item
Damit in der zweiten Zeile eine Nullzeile entsteht, muss im Feld rechts
unten der Wert $-2$ stehen.
Schreibt man $d$ für das unbekannte Element, ergibt die Vorwärtsreduktion
\[
\begin{tabular}{|>{$}r<{$}>{$}r<{$}|}
\hline
1&-1\\
2& d\\
\hline
\end{tabular}
\to
\begin{tabular}{|>{$}r<{$}>{$}r<{$}|}
\hline
1&-1\\
0& d+2\\
\hline
\end{tabular}.
\]
Damit eine Nullzeile entsteht, muss $d+2=0$ oder $d=-2$ sein.
\item
Damit bei Anwendung des RREF-Algorithmus in der letzten Zeile eine
Nullzeile steht, muss das zweite Feld in der letzten Zeile zu $0$
werden.
Die Zeilenoperation, die die $2$ im ersten Feld der letzten Zeile
zu $0$ macht, subtrahiert $4$ vom zweiten Feld der letzten Zeile.
Es dort muss also eine $4$ stehen.


Im dritten Feld der letzten Zeile steht dann aber die Zahl $-6$.
Dieses verschwindet aber, wenn wir die Vorwärtsreduktion mit der 
$1$ im dritten Feld der zweiten Zeile als Pivot-Element durchführen.

Schreiben wir $b$ für das unbekannte Feld, ergibt die Vorwärtsreduktion
\[
\begin{tabular}{|>{$}r<{$}>{$}r<{$}>{$}r<{$}|}
\hline
1&2&3\\
0&0&1\\
2&b&0\\
\hline
\end{tabular}
\to
\begin{tabular}{|>{$}r<{$}>{$}r<{$}>{$}r<{$}|}
\hline
1&2&3\\
0&0&1\\
0&b-4&-6\\
\hline
\end{tabular}
\to
\begin{tabular}{|>{$}r<{$}>{$}r<{$}>{$}r<{$}|}
\hline
1&2&3\\
0&0&1\\
0&b-4&0\\
\hline
\end{tabular}.
\]
Damit eine Nullzeile entsteht, muss $b-4=0$ oder $b=4$ sein.
\qedhere
\end{teilaufgaben}
\end{loesung}
