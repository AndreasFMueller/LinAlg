Betrachten Sie das Gleichungssystem
\[
\begin{linsys}{3}
x&+& y&+&3z&=&45\\
x& &  &+&2z&=&27\\
x&-& y& &  &=&-2
\end{linsys}
\]
\begin{teilaufgaben}
\item Berechnen Sie die Inverse der Koeffizientenmatrix.
\item Finden Sie die Lösung des Gleichungssystems mit Hilfe der Inversen.
\end{teilaufgaben}

\thema{inverse Matrix}
\thema{Gauss-Algorithmus}

\begin{loesung}
Wir schreiben $A$ für die Koeffizientenmatrix:
\[
A=
\begin{pmatrix}
1& 1&3\\
1& 0&2\\
1&-1&0
\end{pmatrix}
\]
\begin{teilaufgaben}
\item Die Inverse kann man mit Hilfe des Gauss-Algorithmus berechnen:
\begin{align*}
\begin{tabular}{|c|c|}
\hline
$A$&$E$\\
\hline
\end{tabular}
&=
\begin{tabular}{|>{$}c<{$}>{$}c<{$}>{$}c<{$}|>{$}c<{$}>{$}c<{$}>{$}c<{$}|}
\hline
1& 1&3&1&0&0\\
1& 0&2&0&1&0\\
1&-1&0&0&0&1\\
\hline
\end{tabular}
\rightarrow
\begin{tabular}{|>{$}c<{$}>{$}c<{$}>{$}c<{$}|>{$}c<{$}>{$}c<{$}>{$}c<{$}|}
\hline
1& 1& 3& 1&0&0\\
0&-1&-1&-1&1&0\\
0&-2&-3&-1&0&1\\
\hline
\end{tabular}
\\
&
\rightarrow
\begin{tabular}{|>{$}c<{$}>{$}c<{$}>{$}c<{$}|>{$}c<{$}>{$}c<{$}>{$}c<{$}|}
\hline
1& 1& 3& 1& 0&0\\
0& 1& 1& 1&-1&0\\
0& 0&-1& 1&-2&1\\
\hline
\end{tabular}
\rightarrow
\begin{tabular}{|>{$}c<{$}>{$}c<{$}>{$}c<{$}|>{$}c<{$}>{$}c<{$}>{$}c<{$}|}
\hline
1& 1& 0& 4&-6& 3\\
0& 1& 0& 2&-3& 1\\
0& 0& 1&-1& 2&-1\\
\hline
\end{tabular}
\\
&
\rightarrow
\begin{tabular}{|>{$}c<{$}>{$}c<{$}>{$}c<{$}|>{$}c<{$}>{$}c<{$}>{$}c<{$}|}
\hline
1& 0& 0& 2&-3& 2\\
0& 1& 0& 2&-3& 1\\
0& 0& 1&-1& 2&-1\\
\hline
\end{tabular}
=\begin{tabular}{|c|c|}\hline
$E$&$A^{-1}$\\
\hline\end{tabular}
\end{align*}
Daraus lesen wir
\[
A^{-1}
=
\begin{pmatrix}
 2&-3& 2\\
 2&-3& 1\\
-1& 2&-1
\end{pmatrix}
\]
ab. Das Resultat können wir durch Ausmultiplizieren kontrollieren:
\[
A^{-1}A=
\begin{pmatrix}
 2&-3& 2\\
 2&-3& 1\\
-1& 2&-1
\end{pmatrix}
\begin{pmatrix}
1& 1&3\\
1& 0&2\\
1&-1&0
\end{pmatrix}
=
\begin{pmatrix}
1&0&0\\
0&1&0\\
0&0&1
\end{pmatrix}=E.
\]
\item
Die Lösung des Gleichungssystems kann jetzt durch Multiplikation $x=A^{-1}b$
gefunden werden:
\[
x=
A^{-1}b=
\begin{pmatrix}
 2&-3& 2\\
 2&-3& 1\\
-1& 2&-1
\end{pmatrix}
\begin{pmatrix} 45\\27\\-2 \end{pmatrix}
=\begin{pmatrix}
2\cdot 45-3\cdot 27-2\cdot 2\\
2\cdot 45-3\cdot 27-1\cdot 2\\
-1\cdot 45+2\cdot 27+1\cdot 2
\end{pmatrix}
=
\begin{pmatrix}
5\\
7\\
11
\end{pmatrix}.
\qedhere
\]
\end{teilaufgaben}
\end{loesung}
