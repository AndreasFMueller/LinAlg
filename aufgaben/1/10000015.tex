Betrachten Sie das Gleichungssytem
\[
\begin{linsys}{3}
-x&-&3y&-&3z&=&-21\\
 x&+&4y&+& z&=& 16\\
 x&-&5y&-&3z&=&  3
\end{linsys}
\]
\begin{teilaufgaben}
\item Berechnen Sie die Inverse der Koeffizientenmatrix.
\item Finden Sie die L"osung des Gleichungssystems mit Hilfe der Inversen.
\end{teilaufgaben}

\begin{loesung}
\begin{teilaufgaben}
\item
Wir berechnen die Inverse der Koeffizientenmatrix
\[
A=\begin{pmatrix}
-1&-3&-3\\
 1& 4& 1\\
 1&-5&-3
\end{pmatrix},
\]
da damit dann auch die L"osung berechnet werden kann.
Wir verwenden den Gauss-Algorithmus daf"ur:
\begin{align*}
\begin{tabular}{|>{$}c<{$}>{$}c<{$}>{$}c<{$}|>{$}c<{$}>{$}c<{$}>{$}c<{$}|}
\hline
-1&-3&-3&1&0&0\\
 1& 4& 1&0&1&0\\
 1&-5&-3&0&0&1\\
\hline
\end{tabular}
&
\rightarrow
\begin{tabular}{|>{$}c<{$}>{$}c<{$}>{$}c<{$}|>{$}c<{$}>{$}c<{$}>{$}c<{$}|}
\hline
 1& 3& 3&-1&0&0\\
 0& 1&-2& 1&1&0\\
 0&-8&-6& 1&0&1\\
\hline
\end{tabular}
\\
&
\rightarrow
\begin{tabular}{|>{$}c<{$}>{$}c<{$}>{$}c<{$}|>{$}c<{$}>{$}c<{$}>{$}c<{$}|}
\hline
 1& 3&  3&-1& 0&0\\
 0& 1& -2& 1& 1&0\\
 0& 0&-22& 9& 8&1\\
\hline
\end{tabular}
\\
&
\rightarrow
\begin{tabular}{|>{$}c<{$}>{$}c<{$}>{$}c<{$}|>{$}c<{$}>{$}c<{$}>{$}c<{$}|}
\hline
 1& 3&  3&      -1     & 0           &       0\\
 0& 1& -2&       1     & 1           &       0\\
 0& 0&  1&-\frac{9}{22}&-\frac{8}{22}&-\frac1{22}\\
\hline
\end{tabular}
\\
&
\rightarrow
\begin{tabular}{|>{$}c<{$}>{$}c<{$}>{$}c<{$}|>{$}c<{$}>{$}c<{$}>{$}c<{$}|}
\hline
 1& 3&  0& \frac{5}{22}& \frac{24}{22}& \frac{3}{22}\\
 0& 1&  0& \frac{4}{22}& \frac{ 6}{22}&-\frac{2}{22}\\
 0& 0&  1&-\frac{9}{22}&-\frac{ 8}{22}&-\frac{1}{22}\\
\hline
\end{tabular}
\\
&
\rightarrow
\begin{tabular}{|>{$}c<{$}>{$}c<{$}>{$}c<{$}|>{$}c<{$}>{$}c<{$}>{$}c<{$}|}
\hline
 1& 3&  0&-\frac{7}{22}& \frac{ 6}{22}& \frac{9}{22}\\
 0& 1&  0& \frac{4}{22}& \frac{ 6}{22}&-\frac{2}{22}\\
 0& 0&  1&-\frac{9}{22}&-\frac{ 8}{22}&-\frac{1}{22}\\
\hline
\end{tabular}
\end{align*}
Die inverse Matrix ist also
\[
A^{-1}=
\begin{pmatrix}
-\frac{7}{22}& \frac{ 6}{22}& \frac{9}{22}\\
 \frac{4}{22}& \frac{ 6}{22}&-\frac{2}{22}\\
-\frac{9}{22}&-\frac{ 8}{22}&-\frac{1}{22}
\end{pmatrix}
=
\frac1{22}
\begin{pmatrix}
-7&  6& 9\\
 4&  6&-2\\
-9&- 8&-1
\end{pmatrix}
\simeq
\begin{pmatrix}
  -0.318&  0.272&  0.409\\
   0.181&  0.272& -0.090\\
  -0.409& -0.363& -0.045
\end{pmatrix}
\]
\item
Die L"osung $x$ findet man jetzt durch Multiplikation der Inversen $A^{-1}$
mit der rechten Seite
\[
b=
\begin{pmatrix}
-21\\16\\3
\end{pmatrix}.
\]
Man findet
\begin{align*}
x&
=
A^{-1}b=
\frac1{22}
\begin{pmatrix}
-7&  6& 9\\
 4&  6&-2\\
-9&- 8&-1
\end{pmatrix}
\begin{pmatrix}
-21\\16\\3
\end{pmatrix}
\\
&=
\frac1{22}
\begin{pmatrix}
(-7)\cdot(-21)+6\cdot 16+9\cdot 3\\
4\cdot (-21)+ 6\cdot 16+ (-2)\cdot 3\\
(-9)\cdot(-21)+ (-8)\cdot 16+ (-1)\cdot 3
\end{pmatrix}
\\
&
=
\frac1{22}
\begin{pmatrix}
270\\
6\\
58
\end{pmatrix}
=
\frac1{11}
\begin{pmatrix}
135\\
3\\
29
\end{pmatrix}
\simeq
\begin{pmatrix}
   12.272\\
    0.272\\
    2.636
\end{pmatrix}
\end{align*}
\end{teilaufgaben}
\end{loesung}
