Berechnen Sie die Determinante der Matrix
\[
A=
\begin{pmatrix}
1&1&0&0&0\\
1&0&1&0&0\\
0&1&0&1&0\\
0&0&1&0&1\\
0&0&0&1&1
\end{pmatrix}
\]
mit Hilfe des Entwicklungssatzes.

\thema{Determinante}
\thema{Entwicklungssatz}

\begin{loesung}
Mit dem Entwicklungssatz findet man durch Entwicklung nach der
ersten Spalte:
\begin{align*}
\det(A)
&=
\left|\;
\begin{matrix}
0&1&0&0\\
1&0&1&0\\
0&1&0&1\\
0&0&1&1
\end{matrix}
\;\right|
-
\left|\;
\begin{matrix}
1&0&0&0\\
1&0&1&0\\
0&1&0&1\\
0&0&1&1
\end{matrix}
\;\right|
\end{align*}
Die einzelnen Determinanten kann man erneut nach der ersten Zeile
entwicklen und erhält
\begin{align*}
\det(A)
&=
-\left|\;
\begin{matrix}
1&1&0\\
0&0&1\\
0&1&1
\end{matrix}
\;\right|
-
\left|\;
\begin{matrix}
0&1&0\\
1&0&1\\
0&1&1
\end{matrix}
\;\right|
\\
&=
\left|\;
\begin{matrix}
1&1&0\\
0&1&1\\
0&0&1
\end{matrix}
\;\right|
+
\left|\;
\begin{matrix}
1&0&1\\
0&1&0\\
0&1&1
\end{matrix}
\;\right|
\\
&=
\left|\;
\begin{matrix}
1&1&0\\
0&1&1\\
0&0&1
\end{matrix}
\;\right|
+
\left|\;
\begin{matrix}
1&0&1\\
0&1&0\\
0&0&1
\end{matrix}
\;\right|
=2
\end{align*}
Dabei hat man in der zweiten Zeile versucht, mit Zeilenvertauschungen
soweit möglich Dreiecksform zu erreichen.
In der dritten Zeile wurde in der zweiten Determinante die
zweite Zeile von der dritten Zeile subtrahiert, was den Wert
dieser Determinante nicht ändert.
Beide Determinanten sind jetzt Determinanten von Dreiecksmatrizen,
die einfach auszuwerten sind: sie sind das Produkt der Diagonalelemente.

Alternativ kann man die Determinante auch mit dem Gauss-Algorithmus berechnen.
Dazu müssen die folgenden Tableaux ausgerechnet werden:
\begin{align*}
\begin{tabular}{|>{$}c<{$}>{$}c<{$}>{$}c<{$}>{$}c<{$}>{$}c<{$}|}
\hline
1&1&0&0&0\\
1&0&1&0&0\\
0&1&0&1&0\\
0&0&1&0&1\\
0&0&0&1&1\\
\hline
\end{tabular}
&\rightarrow
\begin{tabular}{|>{$}c<{$}>{$}c<{$}>{$}c<{$}>{$}c<{$}>{$}c<{$}|}
\hline
1& 1&0&0&0\\
0&-1&1&0&0\\
0& 1&0&1&0\\
0& 0&1&0&1\\
0& 0&0&1&1\\
\hline
\end{tabular}
\\
&\rightarrow
\begin{tabular}{|>{$}c<{$}>{$}c<{$}>{$}c<{$}>{$}c<{$}>{$}c<{$}|}
\hline
1& 1& 0&0&0\\
0& 1&-1&0&0\\
0& 0& 1&1&0\\
0& 0& 1&0&1\\
0& 0& 0&1&1\\
\hline
\end{tabular}
\\
&\rightarrow
\begin{tabular}{|>{$}c<{$}>{$}c<{$}>{$}c<{$}>{$}c<{$}>{$}c<{$}|}
\hline
1& 1& 0& 0&0\\
0& 1&-1& 0&0\\
0& 0& 1& 1&0\\
0& 0& 0&-1&1\\
0& 0& 0& 1&1\\
\hline
\end{tabular}
\\
&\rightarrow
\begin{tabular}{|>{$}c<{$}>{$}c<{$}>{$}c<{$}>{$}c<{$}>{$}c<{$}|}
\hline
1& 1& 0& 0& 0\\
0& 1&-1& 0& 0\\
0& 0& 1& 1& 0\\
0& 0& 0& 1&-1\\
0& 0& 0& 0& 2\\
\hline
\end{tabular}
\end{align*}
Im Laufe des Verfahrens sind die Pivots $1$, $-1$, $1$, $-1$, $2$
aufgetreten, die Determinaten ist ihr Produkt, also $\det(A)=2$.
\end{loesung}
