Betrachten Sie das Gleichungssystem \[
\begin{linsys}{3}
 x_1&+&ux_2&+&vx_3&=&1\\
vx_1&+& x_2&+&ux_3&=&2\\
ux_1&+&vx_2&+& x_3&=&1
\end{linsys}
\]
für die Unbekannten $x_1$, $x_2$ und $x_3$ mit den Parametern $u$ und $v$.
\begin{teilaufgaben}
\item
Finden Sie eine Gleichung für die Parameter $u$ und $v$, für die das
Gleichungssystem nicht eindeutig lösbar ist.
\item
Der Punkt $(1,1)$ der $u$-$v$-Ebene erfüllt die in a) gefundene Bedingung. 
Welchen Rang hat die Koeffizientenmatrix, wenn man diese Werte einsetzt?
\item
Die Punkte $(-1,0)$ und $(0,-1)$ der $u$-$v$-Ebene erfüllen die Bedingung
ebenfalls.
Welchen Rang hat die Koeffizientenmatrix für diese Werte?
\item
Lösen Sie das Gleichungssystem für diejenigen Werte von $u$ und $v$,
für die eine eindeutige Lösung existiert.
\end{teilaufgaben}

\themaS{Matrix mit Parameter}
\themaS{Determinante}
\themaS{Rang}

\begin{loesung}
\begin{teilaufgaben}
\item
Wir berechnen die Determinante der Koeffizientenmatrix
\[
A=\begin{pmatrix}
1&u&v\\
v&1&u\\
u&v&1
\end{pmatrix}
\]
des Gleichungssystems mit Hilfe der Sarrus-Formel:
\[
\det(A)
=
1+u^3+v^3-3uv.
\]
Wenn die Determinante verschwindet, ist die Matrix singulär und es kann
keine eindeutige Lösung existieren.
Die gesuchte Gleichung ist also
\begin{equation}
u^3+v^3-3uv+1=0.
\label{20000031:bedingung}
\end{equation}
Wir kürzen im folgenden die linke Seite von (\ref{20000031:bedingung})
als $p(u,v)$ ab.
\item
Für $u=1$ und $v=1$ wird die Koeffizientenmatrix zu
\[
\begin{pmatrix}
1&1&1\\
1&1&1\\
1&1&1
\end{pmatrix},
\]
die den Rang 1 hat.
\item
Setzt man die Werte ein, erhält man die Matrizen
\begin{align*}
&\begin{pmatrix}
1&-1&0\\
0&1&-1\\
-1&0&1
\end{pmatrix}
&
&\text{und}&
&\begin{pmatrix}
1&0&-1\\
-1&1&0\\
0&-1&1
\end{pmatrix},
\end{align*}
die Transponierte voneinander sind.
Es reicht also, den Rang einer der beiden Matrizen zu bestimmen.

Die Summe der drei Zeilen ist 0, der Rang muss also kleiner als 3 sein,
was wir ja auch erwarten, $u$ und $v$ waren so gewählt, dass eine singuläre
Matrix entsteht.
Da aber auch die ersten zwei Zeilen linear unabhängig sind, muss der
Rang 2 sein.
\item
Das Gleichungsystem kann jetzt mit der Kramerschen Regel gelöst werden:
\begin{align*}
x_1&=\frac1{p(u,v)}\left|\,\begin{matrix}
\color{green}1&u&v\\
\color{green}2&1&u\\
\color{green}1&v&1
\end{matrix}\,\right|
=\frac1{p(u,v)}
(1+u^2+2v^2-v-uv-2u)
=\frac{1+u^2+2v^2-v-uv-2u}{u^3+v^3-3uv+1},
%\label{20000031:x1}
\\
x_2&=\frac1{p(u,v)}\left|\,\begin{matrix}
1&\color{green}1&v\\
v&\color{green}2&u\\
u&\color{green}1&1
\end{matrix}\,\right|
=\frac1{p(u,v)}
(2+u^2+v^2-2uv-u-v)
=\frac{2+u^2+v^2-2uv-u-v}{u^3+v^3-3uv+1},
%\label{20000031:x2}
\\
x_3&=\frac1{p(u,v)}\left|\,\begin{matrix}
1&u&\color{green}1\\
v&1&\color{green}2\\
u&v&\color{green}1
\end{matrix}\,\right|
=\frac1{p(u,v)}
(1+2u^2+v^2-u-2v-uv)
=\frac{1+2u^2+v^2-u-2v-uv}{u^3+v^3-3uv+1}.
%\label{20000031:x3}
\end{align*}
Wir kontrollieren die Lösung durch Einsetzen in das Gleichungssystem.
Setzt man den Klammerausdruck in der Formel für $x_1$ in die erste Gleichung
ein, sollte $p(u,v)$ entstehen:
\begin{align*}
&(1+u^2+2v^2-v-uv-2u)
+
u(2+u^2+v^2-2uv-u-v)
+
v(1+2u^2+v^2-u-2v-uv)
\\
&=
1+u^2+2v^2-v-uv-2u
+
2u+u^3+uv^2-2u^2v-u^2-uv
+
v+2u^2v+v^3-uv-2v^2-uv^2
\\
&=u^3+v^3-3uv+1=p(u,v).
\end{align*}
Analog folgt durch Einsetzen in die zweite und dritte Gleichung:
\begin{align*}
&
v(1+u^2+2v^2-v-uv-2u)
+
(2+u^2+v^2-2uv-u-v)
+
u(1+2u^2+v^2-u-2v-uv)
\\
&=
v+u^2v+2v^3-v^2-uv^2-2uv
+
2+u^2+v^2-2uv-u-v
+
u+2u^3+uv^2-u^2-2uv-u^2v
\\
&=
2u^3+2v^3-6uv+2=2p(u,v)
\\
&
u(1+u^2+2v^2-v-uv-2u)
+
v(2+u^2+v^2-2uv-u-v)
+
(1+2u^2+v^2-u-2v-uv)
\\
&=
u+u^3+2uv^2-uv-u^2v-2u^2
+
2v+u^2v+v^3-2uv^2-uv-v^2
+
1+2u^2+v^2-u-2v-uv
\\
&=
u^3+v^3-3uv+1=p(u,v).
\end{align*}

\end{teilaufgaben}
\end{loesung}

\begin{diskussion}
Die Gleichung (\ref{20000031:bedingung}) definiert eine Menge von Punkten
in der $u$-$v$-Ebene.
Diese Menge zu finden ist nicht trivial, und auch nicht unbedingt Gegenstand
der linearen Algebra, sondern eher der algebraischen Geometrie, die sich
mit den Eigenschaften von Lösungsmengen von solchen Polynomgleichungen befasst.
Trotzdem sei hier ein Argument angegeben, wie man diese Menge finden kann.

Man kann durch Probieren die Punkte $(-1,0)$, $(0,-1)$ und $(1,1)$ als Lösungen
der Gleichung (\ref{20000031:bedingung}) finden.
Da die Gleichung symmetrisch in $u$ und $v$ ist, muss die Lösung
spiegelungssymmetrisch bezüglich der Geraden $x=y$ sein, und man kann
vermuten, dass die Lösung ein Kurvenstück enthalten muss, welches
die ersten beiden Punkte verbindet.
Dieses Kurvenstück muss eine Gerade, eine Parabel (quadratische Kurve) oder
eine kubische Kurve sein. 
Wir probieren eine Gerade durch die beiden Punkte, sie hat die Gleichung $u=-1-v$.
Setzt man dies in (\ref{20000031:bedingung}) ein, findet man
\begin{align*}
(-1-v)^3+v^3-3(-1-v)v+1
&=
-1-3v-3v^2-v^3+v^3+3v+3v^2+1=0.
\end{align*}
Die Punkte auf der genannten Geraden mit der Gleichung
$u+v+1=0$
erfüllen die Gleichung also automatisch.

Da das Polynom $p(u,v)$ immer dann verschwindet, wenn das Polynom $q(u,v)=u+v+1$
verschwindet, muss das Polynom $q(u,v)$ ein Teiler von $p(u,v)$ sein.
Nach einer etwas mühsamen Rechnung findet man 
\[
\frac{p(u,v)}{q(u,v)}= u^2 - uv + v^2-u-v+1=r(u,v).
\]
Ausser aus der Geraden $u+v+1=0$ muss die gesuchte Lösungsmenge also auch noch aus
der Lösungsmenge von $r(u,v)=0$ bestehen.

$r(u,v)$ ist ein quadratisches Polynom, und die Koeffizienten der quadratischen
Terme sind positiv, also muss die Lösungsmenge eine Ellipse sein.
Wegen der Symmetrieeigenschaften der Lösungsmenge müssen die Halbachsen der 
Ellipse gegenüber dem $u$-$v$-Koordinatensystem um $45^\circ$ verdreht sein,
wir ersetzen daher $u$ und $v$ durch
\begin{equation}
%\left.
\begin{aligned}
u&=x+y
\\
v&=x-y.
\end{aligned}
%\quad
%\right\}
%\qquad
%\Rightarrow
%\qquad
%\left\{
%\quad
%\begin{aligned}
%x&=\frac12(u+v)
%\\
%y&=\frac12(u-v)
%\end{aligned}
%\right.
\label{20000031:transformation}
\end{equation}
Setzt man diese Ausdrücke für $u$ und $v$ in $r(u,v)$ ein erhält man
\begin{align*}
r(u,v)
&=(x+y)^2-(x+y)(x-y)+(x-y)^2-(x+y)-(x-y)+1\\
&=x^2+2xy+y^2-x^2+y^2+x^2-2xy+y^2-x-y-x+y+1\\
&=x^2-2x+1+3y^2\\
&=(x-1)^2+3y^2.
\end{align*}
Jeder Term im letzten Ausdruck ist positiv, die Gleichung $r(u,v)=0$ kann
also nur dann erfüllt sein, wenn beide Terme verschwinden, oder $x=1$ und $y=0$.
Setzt man diese Werte wieder in (\ref{20000031:transformation}) ein,
erhält man $u=1$ und $v=1$, also genau den dritten der geratenen Punkte.
Die erwartete Ellipse ist also zu einem Punkt entartet.

Die Lösungsmenge der Gleichung $p(u,v)=0$ besteht also aus der Geraden
mit der Gleichung $u+v+1=0$ und dem Punkt $(1,1)$.
Sie ist in der Abbildung~\ref{20000031:loesungsmenge} dargestellt.
\begin{figure}
\centering
\includeagraphics[]{kubisch-1.pdf}
\caption{Lösungsmenge der Gleichung $p(u,v) = u^3+v^3-3uv+1=0$
in Aufgabe~\ref{20000031}. Sie besteht aus der Geraden durch die
Punkte $(-1,0)$ und $(0,-1)$ sowie dem isolierten Punkt $(1,1)$.
\label{20000031:loesungsmenge}}
\end{figure}

Alternativ kann man die Lösungsmenge auch dadurch finden, dass man die
Funktion $p(u,v)$ als Fläche über der $u$-$v$-Ebene darstellt, und die
Schnittmenge mit der $u$-$v$-Ebene sucht.
Dies  ist in Abbildung~\ref{20000031:flaeche} dargestellt.
\begin{figure}
\centering
\includeagraphics[width=0.7\hsize]{cubic.jpg}
\caption{Graph der Funktion $p(u,v)=u^3+v^3-3uv+1$ für $-2\le u,v\le 2$ (rot),
die Fläche schneidet die $u$-$v$-Ebene in der Geraden durch $(-1,0)$ und
$(0,-1)$, und berührt die $u$-$v$-Ebene im Punkt $(1,1)$ (beide gelb).
\label{20000031:flaeche}}
\end{figure}
\end{diskussion}

