Wir betrachten Matrizen der Form
\[
A=\begin{pmatrix}a&b\\-b&a\end{pmatrix}
=aE+bJ=a\begin{pmatrix}1&0\\0&1\end{pmatrix}+b\begin{pmatrix}0&1\\-1&0\end{pmatrix}.
\]
\begin{teilaufgaben}
\item
Berechnen Sie $A^{-1}$.
\item
Drücken Sie $A^{-1}$ in der Form $cE+dJ$ aus.
\item
Berechnen Sie $AA^{-1}=(aE+bJ)(cE+dJ)$ unter Verwendung der Identität
$J^2=-E$.
\end{teilaufgaben}

\thema{Matrix mit Parameter}
\thema{Matrizenprodukt}
\thema{inverse Matrix}

\begin{loesung}
\begin{teilaufgaben}
\item
Die Inverse kann mit Minoren berechnet werden.
Dazu wird zunächst die Determinante von $A$ benötigt:
\[
\det(A)=\left|\,\begin{matrix}a&b\\-b&a\end{matrix}\,\right|=a^2+b^2.
\]
Die Inverse wird damit
\[
A^{-1}=\frac1{a^2+b^2}\begin{pmatrix}a&-b\\b&a\end{pmatrix}.
\]
Kontrolle:
\[
AA^{-1}
=
\begin{pmatrix}a&b\\-b&a\end{pmatrix}
\frac{1}{a^2+b^2}
\begin{pmatrix}a&-b\\b&a\end{pmatrix}
=
\frac{1}{a^2+b^2}
\begin{pmatrix}a^2+b^2&0\\0&a^2+b^2 \end{pmatrix}=E
\]
\item
Die Inverse $A^{-1}$ kann geschrieben werden als
\[
A^{-1}
=
\frac{a}{a^2+b^2}E
+
\frac{-b}{a^2+b^2}J.
\]
\item Unter Verwendung von $J^2=-E$ findet man
\begin{align*}
AA^{-1}&=(aE+bJ)\frac1{a^2+b^2}(aE-bJ)
=\frac1{a^2+b^2}(a^2EE-abEJ+baJE-b^2J^2)
\\
&=\frac1{a^2+b^2}(a^2E+b^2E)=E.
\qedhere
\end{align*}
\end{teilaufgaben}
\end{loesung}

\begin{diskussion}
Die Algebra der Matrizen $E$ und $J$ ist die Algebra der komplexen Zahlen.
Die Aufgabe zeigt, dass die zu $aE+bJ$ reziproke komplexe Zahl
$(aE-bJ)/(a^2+b^2)$ ist.
$aE-bJ$ heisst auch die zu $aE+bJ$ konjugiert komplexe Zahl.
\end{diskussion}

