% Determinante einer nicht regulären Matrix
Welche der folgenden Matrizen haben Determinante 0?
\begin{teilaufgaben}
\item  $A = \left|\,\begin{matrix}1&2&3\\4&5&6\\0&0&0\end{matrix}\,\right|$
\item  $B = \left|\,\begin{matrix}1&2&0\\3&4&0\\5&6&0\end{matrix}\,\right|$
\item  $C = \left|\,\begin{matrix}1&2\\2&4\end{matrix}\,\right|$
\item  $D = \left|\,\begin{matrix}1&2&3\\0&4&5\\0&0&6\end{matrix}\,\right|$
\item  $E = \left|\,\begin{matrix}1&0&0\\2&3&0\\4&5&6\end{matrix}\,\right|$
\end{teilaufgaben}


\begin{loesung}
\begin{teilaufgaben}
\item
Wegen der Nullzeile ist $A$ singulär, also ist $\det A = 0$.
\item
Wegen der Nullspalte wird es im Gauss-Algorithmus ein unvermeidbares 
Null-Pivot geben, daher ist $B$ singulär und damit $\det B = 0$.
\item
Die zweite Zeile von $C$ ist ein Vielfaches der ersten,
im Gaussalgorithmus wird sie schon im ersten Schritt zu einer Nullzeile.
Daher ist $C$ singulär und damit$\det C = 0$.
\item
$D$ ist eine Dreiecksmatrix, in der Gauss-Algorithmus ohne blaue Operationen
durchgeführt werden kann.
Die Pivot-Elemente sind alle von 0 verschieden, daher ist $D$ regulär
und damit $\det D \ne 0$.
\item
Die Matrix $E$ wird durch Umkehrung der Reihenfolge aller Zeilen und
Spalten zu einer Matrix der Form von $D$, daher ist auch $D$ regulär
und damit $\det E \ne 0$.
\qedhere
\end{teilaufgaben}
\end{loesung}
