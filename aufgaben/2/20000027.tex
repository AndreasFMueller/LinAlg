Berechnen Sie die Determinante der folgenden ``Schnecken''-Matrix
\[
A=
\begin{pmatrix}
1&1&1&2\\
4&5&5&2\\
4&6&5&2\\
4&3&3&3
\end{pmatrix}.
\]

\begin{loesung}
Die Berechnung der Determinanten wird einfacher, wenn sich darin viele
Nullen befinden.
Daher subtrahieren wir zunächst die erste Zeile in jeder weiteren Zeile
so viele Male, wie die Zahlen in der zweiten Spalte angeben.
Die Determinanten ändert dabei ihren Wert nicht:
\begin{align*}
\det(A)
&=
\left|
\begin{matrix}
1&1&1&2\\
-1&0&0&-8\\
-2&0&-1&-10\\
1&0&0&-3
\end{matrix}\right|
=
-
\left|
\begin{matrix}
1& 1& 1&  2\\
0&-1&-2&-10\\
0& 0&-1& -8\\
0& 0& 1& -3
\end{matrix}\right|
=
\left|
\begin{matrix}
1& 1& 1&  2\\
0& 1& 2& 10\\
0& 0& 1&  8\\
0& 0&-1&  3
\end{matrix}\right|
\\
&=
\left|
\begin{matrix}
1&8\\-1&3
\end{matrix}\right|
=1\cdot 3-8\cdot(-1)=3+8=11.
\end{align*}
Beim zweiten Gleichheitszeichen haben wir Zeilen und spalten so umgeordnet,
dass die Nullen nach links unten kommen, dies ist mit drei Vertauschungen
möglich, so dass das Vorzeichen dreimal ändert.
Beim dritten Gleichheitszeichen haben wir aus allen Zeilen ausser der ersten
eine Minuszeichen ausgeklammert, das gibt erneut drei Vorzeichenwechsel.
Nach dem Entwicklungssatz ist die Determinante dann durch die
$2\times 2$-Determinante rechts unten gegeben.

Natürlich kann man auch den Gauss-Algorithmus verwenden:
\begin{align*}
\begin{tabular}{|>{$}c<{$}>{$}c<{$}>{$}c<{$}>{$}c<{$}|}
\hline
1&1&1&2\\
4&5&5&2\\
4&6&5&2\\
4&3&3&3\\
\hline
\end{tabular}
&\rightarrow
\begin{tabular}{|>{$}c<{$}>{$}c<{$}>{$}c<{$}>{$}c<{$}|}
\hline
1&1&1&2\\
0&1&1&-6\\
0&2&1&-6\\
0&-1&-1&-5\\
\hline
\end{tabular}
\rightarrow
\begin{tabular}{|>{$}c<{$}>{$}c<{$}>{$}c<{$}>{$}c<{$}|}
\hline
1&1&1&2\\
0&1&1&-6\\
0&0&-1&6\\
0&0&0&-11\\
\hline
\end{tabular}
\end{align*}
Die Determinante ist dann das Produkt der Pivot-Elemente, die in diesem Fall
$1$, $1$, $-1$ und $-11$ sind, also
\[
\det(A)=1\cdot 1\cdot (-1)\cdot (-11)=11.
\]

Natürlich kann man die Determinante auch mit dem Entwicklungssatz ausrechnen:
\begin{align*}
\det(A)
&=
1\cdot\left|\begin{matrix}
5&5&2\\
6&5&2\\
3&3&3
\end{matrix}\right|
-
1\cdot\left|\begin{matrix}
4&5&2\\
4&5&2\\
4&3&3
\end{matrix}\right|
+
1\cdot\left|\begin{matrix}
4&5&2\\
4&6&2\\
4&3&3
\end{matrix}\right|
-
2\cdot\left|\begin{matrix}
4&5&5\\
4&6&5\\
4&3&3
\end{matrix}\right|
\end{align*}
Die $3\times 3$-Determinanten können zum Beispiel mit der Sarrus-Formel
ausgerechnet werden:
\begin{align*}
\left|\begin{matrix}
5&5&2\\
6&5&2\\
3&3&3
\end{matrix}\right|
&=
5\cdot 5\cdot 3 + 5\cdot 2 \cdot3 + 2\cdot 6\cdot 3
-3\cdot 5\cdot 2-3\cdot 2\cdot 5-3\cdot 6\cdot5
\\
&=
75+30+36-30-30-90=-9
\\
\left|\begin{matrix}
4&5&2\\
4&5&2\\
4&3&3
\end{matrix}\right|
&=0
\\
\left|\begin{matrix}
4&5&2\\
4&6&2\\
4&3&3
\end{matrix}\right|
&=
4\cdot 6\cdot 3 + 5\cdot 2\cdot4+2\cdot 4\cdot 3
-4\cdot 6\cdot 2-3\cdot 2\cdot 4-3\cdot 4\cdot 5
\\
&=72+40+24-48-24-60=4
\\
\left|\begin{matrix}
4&5&5\\
4&6&5\\
4&3&3
\end{matrix}\right|
&=
4\cdot 6\cdot 3+5\cdot 5\cdot 4 +5\cdot 4\cdot 3
-4\cdot 6\cdot 5-3\cdot 5\cdot 4-3\cdot 4\cdot 5
\\
&= 72+100+60-120-60-60
=-8
\end{align*}
Damit kann man jetzt die  Determinante ausrechnung:
\begin{align*}
\det(A)
&=
1\cdot (-9)
-1\cdot 0
+1\cdot 4
-2\cdot (-8)
\\
&=-9+4+16=11.
\qedhere
\end{align*}
\end{loesung}

\begin{bewertung}
Wahl einer geeigneten Berechnungsmethode ({\bf M}) 1 Punkt,
Teilresultate ({\bf T}) 4 Punkte,
Korrektes Resultate ({\bf D}) 1 Punkt.
\end{bewertung}
