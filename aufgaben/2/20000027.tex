Berechnen Sie die Determinante der folgenden ``Schnecken''-Matrix
\[
A=
\begin{pmatrix}
1&1&1&2\\
4&5&5&2\\
4&6&5&2\\
4&3&3&3
\end{pmatrix}.
\]

\begin{loesung}
Die Berechnung der Determinanten wird einfacher, wenn sich darin viele
Nullen befinden.
Daher subtrahieren wir zun"achst die erste Zeile in jeder weiteren Zeile
so viele Male, wie die Zahlen in der zweiten Spalte angeben.
Die Determinanten "andert dabei ihren Wert nicht:
\begin{align*}
\det(A)
&=
\left|
\begin{matrix}
1&1&1&2\\
-1&0&0&-8\\
-2&0&-1&-10\\
1&0&0&-3
\end{matrix}\right|
=
-
\left|
\begin{matrix}
1& 1& 1&  2\\
0&-1&-2&-10\\
0& 0&-1& -8\\
0& 0& 1& -3
\end{matrix}\right|
=
\left|
\begin{matrix}
1& 1& 1&  2\\
0& 1& 2& 10\\
0& 0& 1&  8\\
0& 0&-1&  3
\end{matrix}\right|
\\
&=
\left|
\begin{matrix}
1&8\\-1&3
\end{matrix}\right|
=1\cdot 3-8\cdot(-1)=3+8=11.
\end{align*}
Beim zweiten Gleichheitszeichen haben wir Zeilen und spalten so umgeordnet,
dass die Nullen nach links unten kommen, dies ist mit drei Vertauschungen
m"oglich, so dass das Vorzeichen dreimal "andert.
Beim dritten Gleichheitszeichen haben wir aus lle Zeilen ausser der ersten
eine Minuszeichen ausgeklammert.
Nach dem Entwicklungssatz ist die Determinante dann durch die
$2\times 2$-Determinante rechts unten gegeben.
\end{loesung}

