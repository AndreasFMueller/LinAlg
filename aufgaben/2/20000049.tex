Die HSR wurde 1972 gegründet. Zudem hat sie 1600 Studierende, 16 Institute und 400
aktuelle Forschungsprojekte. Schreibt man diese Kennzahlen in eine Matrix erhält man
die Matrix
\[
A=
\begin{pmatrix}
 1 & 9 & 7 & 2\\
 1 & 6 & 0 & 0\\
 0 & 0 & 1 & 6\\
 0 & 4 & 0 & 0
\end{pmatrix}.
\]
\begin{teilaufgaben}
\item 
Berechnen Sie die Determinante der Matrix $A$. Ist die Matrix regulär?
\item Wie ändert sich der Wert der Determinante, wenn die HSR doppelt so viele 
Forschungsprojekte hätte? Berechnen Sie die Determinante nicht nochmals neu.
\end{teilaufgaben}

\thema{Determinante}
\thema{Entwicklungssatz}

\begin{loesung}
\begin{teilaufgaben}
\item
Wir berechnen die Determinante mit Hilfe des Entwicklungssatzes 
und entwickeln zuerst nach der vierten Zeile und anschliessend nach 
der zweiten Zeile:
\begin{align*}
\det(A)
&=
4\cdot\left|
\begin{matrix}
 1 & 7 & 2\\
 1 & 0 & 0\\
 0 & 1 & 6\\
\end{matrix}
\right|
= 
4\cdot(-1)\cdot\left|
\begin{matrix}
 7 & 2\\
 1 & 6\\
\end{matrix}
\right|
= 
-4\cdot (7\cdot 6 - 2 \cdot 1)
= -160
\end{align*}
Da $\det(A)\neq 0$ ist, ist die Matrix regulär.
\item 
Wenn die HSR doppelt so viele Forschungsprojekte hätte, würde 
in der Matrix die vierte Zeile mit 2 multipliziert werden.
Eine Multiplikation einer Zeile mit einem Faktor führt dazu,
dass sie die Determinante um genau diesen Faktor ändert.
Folglich würde sich der Wert der Determinante verdoppeln: 
$\det(A') = -320$.
\end{teilaufgaben}
\end{loesung}

\begin{bewertung}
Entwicklungssatz: Zeile oder Spalte ({\bf Z}) 1 Punkt,
Vorzeichen ({\bf V}) 1 Punkt,
Unterdeterminanten ({\bf U}) 1 Punkt,
Determinantenwert ({\bf D}) 1 Punkt,
Schlussfolgerung regulär ({\bf R}) 1 Punkt,
Determinante von $A'$ ({\bf A}) 1 Punkt.
\end{bewertung}

