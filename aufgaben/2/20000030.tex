Berechnen Sie die Inverse der Matrix
\[
A
=
\begin{pmatrix}
t&0&1\\
1&t&0\\
0&1&t
\end{pmatrix}
\]
Kontrollieren Sie Ihr Resultat durch Ausrechnen des Produktes $AA^{-1}$.

\begin{loesung}
Wir bestimmen die Inverse mit Hilfe von Determinanten.
Dazu brauchen wir zun"achst die Determinaten von $A$, die wir durch Entwicklung
nach der ersten Spalte berechnen:
\begin{align*}
\det(A)
&=
\left|\,\begin{matrix}
t&0&1\\
1&t&0\\
0&1&t
\end{matrix}\,\right|
=
t\cdot\left|\,\begin{matrix}t&0\\1&t\end{matrix}\,\right|
-1\cdot\left|\,\begin{matrix}0&1\\1&t\end{matrix}\,\right|
=
t\cdot t^2-1\cdot(-1)=t^3+1.
\end{align*}
Jetzt k"onnen wir die inverse Matrix mit Hilfe der Minoren berechnen:
\begin{align}
A^{-1}
&=
\frac1{\det{A}}\begin{pmatrix}
\phantom{-}\left|\,\begin{matrix}t&0\\1&t\end{matrix}\,\right|
	&-\left|\,\begin{matrix}0&1\\1&t\end{matrix}\,\right|
		&\phantom{-}\left|\,\begin{matrix}0&1\\t&0\end{matrix}\,\right|
			\\[11pt]
-\left|\,\begin{matrix}1&0\\0&t\end{matrix}\,\right|
	&\phantom{-}\left|\,\begin{matrix}t&1\\0&t\end{matrix}\,\right|
		&-\left|\,\begin{matrix}t&1\\1&0\end{matrix}\,\right|
			\\[11pt]
\phantom{-}\left|\,\begin{matrix}1&t\\0&1\end{matrix}\,\right|
	&-\left|\,\begin{matrix}t&0\\0&1\end{matrix}\,\right|
		&\phantom{-}\left|\,\begin{matrix}t&0\\1&t\end{matrix}\,\right|
\end{pmatrix}
=\frac1{t^3+1}
\renewcommand{\arraystretch}{1}
\begin{pmatrix}
t^2&  1& -t\\
 -t&t^2&  1\\
  1& -t&t^2
\end{pmatrix}
\label{20000030:inverse}
\end{align}
Kontrolle:
\begin{align*}
AA^{-1}
&=
\begin{pmatrix}
t&0&1\\
1&t&0\\
0&1&t
\end{pmatrix}
\frac{1}{t^3+1}
\begin{pmatrix}
t^2&  1& -t\\
 -t&t^2&  1\\
  1& -t&t^2
\end{pmatrix}
\\
&=
\frac{1}{t^3+1}
\begin{pmatrix}
t\cdot t^2+1\cdot 1
	&t\cdot 1+1\cdot(-t)
		&t\cdot(-t)+1\cdot t^2\\
1\cdot t^2+t\cdot (-t)
	&1\cdot 1+t\cdot t^2
		&1\cdot(-t)+t\cdot 1\\
t\cdot (-t)+t\cdot 1§
	&t^2\cdot 1+t\cdot (-t)
		&1\cdot 1+t\cdot t^2
\end{pmatrix}
\\
&=
\frac1{t^3+1}
\begin{pmatrix}
t^3+1&    0&    0\\
    0&t^3+1&    0\\
    0&    0&t^3+1
\end{pmatrix}=E.
\end{align*}
Damit ist gezeigt, dass (\ref{20000030:inverse}) tats"achlich die inverse Matrix
von $A$ ist.
\end{loesung}
