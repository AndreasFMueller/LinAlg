Berechnen Sie die Inverse der Matrix
\[
A=
\begin{pmatrix}
a&b&0\\
c&a&b\\
0&c&a
\end{pmatrix}.
\]
Kontrollieren Sie Ihr Resultat, indem Sie $AA^{-1}$ ausmultiplizieren.

\thema{inverse Matrix}
\thema{Matrix mit Parameter}

\begin{loesung}
Die Inverse kann mit Hilfe der Determinanten und der Minoren gefunden werden.
Wir berechnen dazu zunächst die Determinate von $A$ durch Entwicklung
nach der ersten Zeile:
\begin{align*}
\det(A)=\left|\,\begin{matrix}
a&b&0\\
c&a&b\\
0&c&a
\end{matrix}\,\right|
=
a
\cdot
\left|\,\begin{matrix}a&b\\c&a\end{matrix}\,\right|
-b
\cdot
\left|\,\begin{matrix}c&b\\0&a\end{matrix}\,\right|
=
a(a^2-bc)-b\cdot ac=a^3-2abc.
\end{align*}
Jetzt kann die inverse Matrix mit Hilfe der Minoren hingeschrieben werden
\begin{align}
A^{-1}
&=
\frac1{a^3-2abc}\begin{pmatrix}
\phantom{-}\left|\,\begin{matrix}a&b\\c&a\end{matrix}\,\right|
	&-\left|\,\begin{matrix}b&0\\c&a\end{matrix}\,\right|
		&\phantom{-}\left|\,\begin{matrix}b&0\\a&b\end{matrix}\,\right|
			\\[13pt]
-\left|\,\begin{matrix}c&b\\0&a\end{matrix}\,\right|
	&\phantom{-}\left|\,\begin{matrix}a&0\\0&a\end{matrix}\,\right|
		&-\left|\,\begin{matrix}a&0\\c&b\end{matrix}\,\right|
			\\[13pt]
\phantom{-}\left|\,\begin{matrix}c&a\\0&c\end{matrix}\,\right|
	&-\left|\,\begin{matrix}a&b\\0&c\end{matrix}\,\right|
		&\phantom{-}\left|\,\begin{matrix}a&b\\c&a\end{matrix}\,\right|
\end{pmatrix}
=
\frac1{a^3-2abc}\begin{pmatrix}
a^2-bc&-ab&b^2\\
-ac&a^2&-ab\\
c^2&-ac&a^2-bc
\end{pmatrix}.
\label{20000032:inverse}
\end{align}
Kontrolle:
\begin{align*}
AA^{-1}
&=
\frac1{a^3-2abc}
\begin{pmatrix}
a&b&0\\
c&a&b\\
0&c&a
\end{pmatrix}
\begin{pmatrix}
a^2-bc&-ab&b^2\\
-ac&a^2&-ab\\
c^2&-ac&a^2-bc
\end{pmatrix}
\\
&=
\frac1{a^3-2abc}
\begin{pmatrix}
a(a^2-bc)-abc      &-a^2b+a^2b  &ab^2-ab^2          \\
c(a^2-bc)-a^2b+bc^2&-abc+a^3-abc&b^2c-a^2b+b(a^2-bc)\\
-ac^2+ac^2         &a^2c-a^2c   &-abc+a(a^2-bc)
\end{pmatrix}
\\
&=
\frac1{a^3-2abc}
\begin{pmatrix}
a^3-2abc&0&0\\
0&a^3-2abc&0\\
0&0&a^3-2abc
\end{pmatrix}
=E.
\end{align*}
Damit ist gezeigt, dass die Matrix (\ref{20000032:inverse}) tatsächlich
die Inverse von $A$ ist.
\end{loesung}

