Betrachten Sie die Matrizen
\[
A=\begin{pmatrix}
13& 9&-3&15\\
 0&11& 7& 5\\
 0& 0&17&13\\
 0& 0& 0&19
\end{pmatrix}
\qquad\text{und}\qquad
B
=
\begin{pmatrix}
0&1&0&0\\
0&0&1&0\\
0&0&0&1\\
1&0&0&0
\end{pmatrix}.
\]
\begin{teilaufgaben}
\item Berechnen Sie $\det(A)$.
\item Berechnen Sie $C=BAB^t$.
\item Berechnen Sie $\det(C)$.
\end{teilaufgaben}

\thema{Produktsatz}
\thema{Matrizenprodukt}

\begin{loesung}
\begin{teilaufgaben}
\item $A$ ist eine Dreiecksmatrix, die Determinante ist das Produkt der
Diagonalelemente:
\[
\det(A) = 13\cdot 11 \cdot 17\cdot 19=46189.
\]
\item Die Matrix $C$ ist
\[
C
=
BAB^t
=
\begin{pmatrix}
    0&  11&   7&   5\\
    0&   0&  17&  13\\
    0&   0&   0&  19\\
   13&   9&  -3&  15
\end{pmatrix}
\begin{pmatrix}
0&0&0&1\\
1&0&0&0\\
0&1&0&0\\
0&0&1&0
\end{pmatrix}
=
\begin{pmatrix}
  11&   7&   5&   0\\
   0&  17&  13&   0\\
   0&   0&  19&   0\\
   9&  -3&  15&  13
\end{pmatrix}.
\]
\item Aus der Produktformel für Determinanten folgt
\[
\det(C)
=
\det(BAB^t)
=
\det(B)\det(A)\det(B^t)=\det(B)\det(A)\det(B).
\]
Im letzten Schritt wurde $\det(B)=\det(B^t)$ verwendet.
Die Determinante $\det(A)$ auf der rechten Seite wurde in Teilaufgabe a)
bereits berechnet.
Es muss also nur noch $\det(B)$ berechnet werden.
Da in jeder Zeile von $B$ genau eine Eins vorkommt und sonst nur Nullen,
kann man durch Zeilenvertauschungen die Determinante immer in Diagonalform
bringen.
\[
\det(B)
=
\left|
\begin{matrix}
   0&  1&  0&  0\\
   0&  0&  1&  0\\
   0&  0&  0&  1\\
   1&  0&  0&  0
\end{matrix}\right|
=
-
\left|\begin{matrix}
1&0&0&0\\
0&1&0&0\\
0&0&1&0\\
0&0&0&1
\end{matrix}\right|
=-1.
\]
Damit folgt
\[
\det(BAB^t)
=
\det(A)\det(B)^2
=
46189\cdot(-1)^2=46189.\qedhere
\]
\end{teilaufgaben}
\end{loesung}

\begin{diskussion}
Man könnte die Determinante von $C$ auch direkt mit dem Entwicklungssatz
berechnen.
Hier ist diese Lösung:
\begin{align*}
\det(C)
&=
\left|\begin{matrix}
  11&   7&   5&   0\\
   0&  17&  13&   0\\
   0&   0&  19&   0\\
   9&  -3&  15&  13
\end{matrix}\right|
%\\
%&
=
13\cdot
\left|\begin{matrix}
  11&   7&   5\\
   0&  17&  13\\
   0&   0&  19
\end{matrix}\right|
=
13\cdot 11\cdot
\left|\begin{matrix}
17&  13\\
 0&  19
\end{matrix}\right|
=
13\cdot 11\cdot 17\cdot 19
=
46189.
\end{align*}
Im ersten Schritt wird nach der letzten Spalte entwickelt, im zweiten
nach der erste Spalte.
\end{diskussion}

