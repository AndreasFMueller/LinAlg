Berechnen Sie die Determinante der Matrix
\[
A
=
\begin{pmatrix}
-2& 5&-2&-1\\
 1& 1& 1& 4\\
 5&-2& 3& 5\\
 6& 1& 1&-1
\end{pmatrix}.
\]

\begin{loesung}
Damit der Rechenaufwand bei der Anwendung des Entwicklungssatzes gering bleibt, subtrahieren wir die zweite Zeile von der letzten und entwickeln dann
nach der letzten Zeile:
\begin{align*}
\det(A)
&=
\left|\begin{matrix}
-2& 5&-2&-1\\
 1& 1& 1& 4\\
 5&-2& 3& 5\\
 5& 0& 0&-5
\end{matrix}\right|
=
5\cdot
\left|\begin{matrix}
 5&-2&-1\\
 1& 1& 4\\
-2& 3& 5
\end{matrix}\right|
-(-5)\cdot
\left|\begin{matrix}
-2& 5&-2\\
 1& 1& 1\\
 5&-2& 3
\end{matrix}\right|
\end{align*}
Die einzelnen Determinanten k"onnen mit der Sarrus-Regel ausgwertet werden:
\begin{align*}
\left|\begin{matrix}
 5&-2&-1\\
 1& 1& 4\\
-2& 3& 5
\end{matrix}\right|
&=
25+16-3 -2-60+10
=
-14,
\\
\left|\begin{matrix}
-2& 5&-2\\
 1& 1& 1\\
 5&-2& 3
\end{matrix}\right|
&=
-6+25+4+10-4-15
=
14.
\end{align*}
Damit kann man jetzt die Determinante berechnen:
\begin{align*}
\det(A)
&=
5\cdot (-14)-(-5)\cdot 14
0
-5\cdot 14 + 5\cdot 14 =0.
\end{align*}
Die Matrix $A$ ist also singul"ar.
\end{loesung}

\begin{bewertung}
Entwicklungssatz ({\bf E}) 2 Punkte,
Berechnung der $3\times 3$-Matrizen mit Sarrus ({\bf S}) 3 Punkte,
korrekter Wert f"ur die Determinante ({\bf D}) 1 Punkt.
\end{bewertung}

