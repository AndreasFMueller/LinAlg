Sei $D_n$ die $n\times n$-Matrix
\[
D_n=\begin{pmatrix}
-2& 1&     0&      &  \\
 1&-2&     1&      &  \\
 0& 1&    -2&\ddots&  \\
  &  &\ddots&\ddots&  \\
  &  &      &      &-2
\end{pmatrix}
\]
und setzen Sie $d_n=\det(D_n)$.
\begin{teilaufgaben}
\item Berechnen Sie $d_n$ f"ur $n\le 3$.
\item Stellen Sie eine Rekursionsformel f"ur $d_n$ auf, also eine Formel,
die $d_{n}$ aus  $d_{n-1}$ und $d_{n-2}$ bestimmt.
\item Berechnen Sie die Diagonalelemente von $D_n^{-1}$
\end{teilaufgaben}

\begin{loesung}
\begin{teilaufgaben}
\item
Die ersten Werte der Folge $(d_n)_{n\in\mathbb N}$ kann man direkt ausrechnen:
\begin{align*}
d_1&=\det(-2)=-2\\
d_2&=\left|\,
\begin{matrix}-2&1\\1&-2\end{matrix}
\,\right|=4-1=3
\\
d_3&=\left|\,
\begin{matrix}
-2& 1& 0\\
 1&-2& 1\\
 0& 1&-2
\end{matrix}
\,\right|=-8+0+0-0-(-2)-(-2)=-4
\end{align*}
\item
Man kann die Determinante mit dem Entwicklungssatz berechnen
\begin{align*}
\det(D_n)
&=
-2\cdot
\left|\,
\begin{matrix}
-2&     1&     0&  \\
 1&    -2&\ddots&  \\
 0&\ddots&\ddots&  \\
  &      &      &-2
\end{matrix}
\,\right|
-1\cdot
\left|\,
\begin{matrix}
 1&     0&\dots & 0\\
 1&    -2&\ddots&  \\
 0&\ddots&\ddots&  \\
  &      &      &-2
\end{matrix}
\,\right|
\\
&=
-2\det(D_{n-1})
-\underbrace{\left|\,
\begin{matrix}
    -2&\ddots&  \\
\ddots&\ddots&  \\
      &      &-2
\end{matrix}
\,\right|}_{\det(D_{n-2})}
+\underbrace{\left|\,
\begin{matrix}
     0&\dots & 0\\
\ddots&\ddots&  \\
      &      &-2
\end{matrix}
\,\right|}_{=0}
\\
&=-2d_{n-1}-d_{n-2}
\end{align*}
Die Werte der Determinanten erf"ullen also die Rekursionsformel
\[
d_n=-2d_{n-1}-d_{n-2}
\]


Mit der Rekursionsformel kann man daraus jetzt die weiteren Werte berechnen
\begin{align*}
d_4&=-2\cdot(-4) - 3 = 5\\
d_5&=-2\cdot 5 -(-4) = -6\\
d_6&=-2\cdot(-6) - 5 = 7\\
d_7&=-2\cdot 7 -(-6) = -8\\
d_8&=-2\cdot(-8) - 7 = 9\\
d_9&=-2\cdot 9 -(-8) = -10
\end{align*}
Daraus kann man ablesen, dass
\[
d_n=(-1)^n(n+1).
\]
\item
Die Diagonalelemente $c_{ii}$ von $D_n^{-1}$ k"onnen mit Hilfe der Minoren berechnet werden,
sie sind
\begin{align*}
c_{44}
&=
(-1)^{2\cdot 4}\frac{1}{\det(D_n)}
\left|\,
\begin{matrix}
-2& 1& 0& 0& 0& 0&\dots\\
 1&-2& 1& 0& 0& 0&\dots\\
 0& 1&-2& 0& 0& 0&\dots\\
 0& 0& 0&-2& 1& 0&\dots\\
 0& 0& 0& 1&-2& 1&\dots\\
 0& 0& 0& 0& 1&-2&\dots\\
\vdots&\vdots&\vdots&\vdots&\vdots&\vdots&\ddots
\end{matrix}
\,\right|
\\
&=
(-1)^{2\cdot 4}\frac{1}{\det(D_n)}
\left|\,
\begin{matrix}
-2& 1& 0\\
 1&-2& 1\\
 0& 1&-2\\
\end{matrix}
\,\right|
\cdot
\left|\,
\begin{matrix}
-2& 1& 0&\dots\\
 1&-2& 1&\dots\\
 0& 1&-2&\dots\\
 0& 0& 1&\dots\\
\vdots&\vdots&\vdots&\ddots
\end{matrix}
\,\right|
\\
&=
(-1)^{2\cdot 4}\frac{1}{\det(D_n)}\det(D_3)\det(D_{n-4})
\end{align*}
oder allgemein
\[
c_{ii}
=\frac{\det(D_{i-1})\det(D_{n-i})}{\det(D_n)}
=\frac{(-1)^{i-1}i(-1)^{n-i}(n-i+1)}{(-1)^n(n+1)}
=-\frac{i(n-i+1)}{n+1}
\qedhere
\]
\end{teilaufgaben}
\end{loesung}

