Berechnen Sie die Inverse der Matrix
\[
A=
\begin{pmatrix}
 0& 1& 2\\
 0& 2& 3\\
-t& 0& 1
\end{pmatrix}
\]
mit Hilfe der Minoren. Kontrollieren Sie Ihr Resultat
mit Hilfe der Matrixmultiplikation.

\themaS{Determinante}
\themaS{Minoren}
\themaS{inverse Matrix}
\themaS{Matrixmultiplikation}

\begin{loesung}
Die Inverse kann mit Hilfe der Minoren bestimm werden. Dazu wird zunächst die
Determinante von $A$ benötigt:
\begin{align*}
\det(A)&=(-t)\left|\,
\begin{matrix}
1&2\\2&3
\end{matrix}
\,\right|
=
-t(1\cdot 3-2\cdot 2)=t
\end{align*}
Daraus kann man jetzt die inverse Matrix berechnen:
\[
A^{-1}
=
\frac1t
\begin{pmatrix}
 +\left|\,\begin{matrix} 2&3\\0&1\end{matrix}\,\right|
&-\left|\,\begin{matrix} 1&2\\0&1\end{matrix}\,\right|
&+\left|\,\begin{matrix} 1&2\\2&3\end{matrix}\,\right|\\
 -\left|\,\begin{matrix} 0&3\\-t&1\end{matrix}\,\right|
&+\left|\,\begin{matrix} 0&2\\-t&1\end{matrix}\,\right|
&-\left|\,\begin{matrix} 0&2\\0&3\end{matrix}\,\right|\\
 +\left|\,\begin{matrix} 0&2\\-t&0\end{matrix}\,\right|
&-\left|\,\begin{matrix} 0&1\\-t&0\end{matrix}\,\right|
&+\left|\,\begin{matrix} 0&1\\0&2\end{matrix}\,\right|\\
\end{pmatrix}
=
\frac1t
\begin{pmatrix}
 2 &-1 &-1\\
-3t& 2t& 0\\
 2t& -t& 0
\end{pmatrix}
\]
Kontrolle:
\begin{align*}
AA^{-1}
&=
\begin{pmatrix}
 0& 1& 2\\
 0& 2& 3\\
-t& 0& 1
\end{pmatrix}
\cdot
\frac1t
\begin{pmatrix}
 2 &-1 &-1\\
-3t& 2t& 0\\
 2t& -t& 0
\end{pmatrix}
\\
&=
\frac1t
\begin{pmatrix}
0-3t+4t&0+2t-2t&0+0+0\\
0-6t+6t&0+4t-3t&0+0+0\\
-2t+0+2t&t+0-t&t+0+0
\end{pmatrix}
\\
&=\frac1t\begin{pmatrix}t&0&0\\0&t&0\\0&0&t\end{pmatrix}=E
\qedhere
\end{align*}
\end{loesung}

