%Zeilenoperationen (Vereinfachung)
Verwenden Sie Zeilenoperationen, um die folgenden Determinanten in
Determinanten von Dreiecksmatrizen zu verwandeln und damit deren Berechnung
zu vereinfachen.

\vspace*{5pt}
\hbox to\hsize{%
\hbox to0.5\hsize{%
a)\;
\(\displaystyle
\det A
=
\left|\,\begin{matrix*}[r]
1& 2&  9& 1\\
0& 3&  9& 6\\
2& 4& 20& 3\\
3& 6& 27& 0
\end{matrix*}\,\right|
=
\text{?}
\)\hfill}%
\hbox to0.5\hsize{%
b)\;
\(\displaystyle
\det B
=
\left|\,\begin{matrix*}[r]
1&  2&  9& 1\\
0&  3&  9& 6\\
2&  4& 20& 3\\
3& 12& 45& 0
\end{matrix*}\,\right|
=
\text{?}
\)
\hfill}}

\begin{loesung}
\begin{teilaufgaben}
\item
Durch zwei Zeilenoperatationen mit der ersten Zeile, genauer durch Subtraktion
des zweifachen der ersten Zeile von der dritten und des dreifachen von der 
vierten Zeile entsteht eine Dreiecksmatrix:
\[
\det A
=
\left|\,\begin{matrix*}[r]
1& 2&  9& 1\\
0& 3&  9& 6\\
2& 4& 20& 3\\
3& 6& 27& 0
\end{matrix*}\,\right|
=
\left|\,\begin{matrix*}[r]
1& 2&   9&  1\\
0& 3&   9&  6\\
0& 0&   2&  1\\
0& 0&   0& -3
\end{matrix*}\,\right|
=
1\cdot 3 \cdot 2 \cdot (-3)
=
-18.
\]
\item
Mit den gleichen zwei Operationen wie in a) entsteht nicht ganz eine
Dreiecksmatrix, nämlich
\begin{align*}
\det B
&=
\left|\,\begin{matrix*}[r]
1&  2&  9& 1\\
0&  3&  9& 6\\
2&  4& 20& 3\\
3& 12& 45& 0
\end{matrix*}\,\right|
=
\left|\,\begin{matrix*}[r]
1&  2&  9&  1\\
0&  3&  9&  6\\
0&  0&  2&  1\\
0&  6& 18& -3
\end{matrix*}\,\right|.
\intertext{Um auch noch die letzte Zeile zu bereinigen, wird das
zweifache der zweiten Zeile von der vierten Zeile subtrahiert:}
&=
\left|\,\begin{matrix*}[r]
1&  2&  9&   1\\
0&  3&  9&   6\\
0&  0&  2&   1\\
0&  0&  0& -15
\end{matrix*}\,\right|
=
1\cdot 3 \cdot 2 \cdot (-15)
=
-90.
\qedhere
\end{align*}
\end{teilaufgaben}
\end{loesung}
