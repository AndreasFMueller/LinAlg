Berechnen Sie die Determinante der Matrix
\[
A=\begin{pmatrix}
-6&-1& 9&-9\\
-6&-4& 9&-9\\
-2&-8& 2&-7\\
 2& 3&-4&-1
\end{pmatrix}.
\]

\thema{Determinante}
\thema{Entwicklungssatz}
\thema{Gauss-Algorithmus}

\begin{loesung}
Die Determinante kann mit dem Entwicklungssatz berechnet werden, doch
ist dies etwas mühsam, weil in der Matrix keine Nullen vorkommen.
Daher subtrahieren wir erst die die zweite Zeile von der ersten, denn
so entstehen in dieser Zeile drei Nullen:
\begin{align*}
\det(A)
&=
\left|\,\begin{matrix}
-6&-1& 9&-9\\
-6&-4& 9&-9\\
-2&-8& 2&-7\\
 2& 3&-4&-1
\end{matrix}\,\right|
=
\left|\,\begin{matrix}
 0& 3& 0& 0\\
-6&-4& 9&-9\\
-2&-8& 2&-7\\
 2& 3&-4&-1
\end{matrix}\,\right|
=
-3\cdot\left|\,\begin{matrix}
-6& 9&-9\\
-2& 2&-7\\
 2&-4&-1
\end{matrix}\,\right|
\end{align*}
Die $3\times 3$-Determinante könnte man zwar direkt mit der
Sarrus-Formel berechnen, aber wir ziehen vorher noch einen gemeinsamen
Faktor 3 aus der ersten Zeile und einen gemeinsamen Faktor 2 aus der
ersten Spalte:
\begin{align}
\det(A)
&=
-3\cdot 3\cdot 2\cdot\left|\,\begin{matrix}
-1& 3&-3\\
-1& 2&-7\\
 1&-4&-1
\end{matrix}\,\right|
\label{20000037:3det}
\end{align}
Durch ``Anstarren'' kann man jetzt finden, dass das Doppelte der ersten
Zeile zusammen mit der dritten Zeile die zweite Zeile ergibt, die Determinante
muss also verschwinden.

Alternativ könnte man den Gaussalgorithmus anwenden:
\begin{align*}
\begin{tabular}{|>{$}c<{$}>{$}c<{$}>{$}c<{$}|}
\hline
-1& 3&-3\\
-1& 2&-7\\
 1&-4&-1\\
\hline
\end{tabular}
&\Rightarrow
\begin{tabular}{|>{$}c<{$}>{$}c<{$}>{$}c<{$}|}
\hline
 1&-3& 3\\
 0&-1&-4\\
 0&-1&-4\\
\hline
\end{tabular}
\Rightarrow
\begin{tabular}{|>{$}c<{$}>{$}c<{$}>{$}c<{$}|}
\hline
 1&-3& 3\\
 0& 1& 4\\
 0& 0& 0\\
\hline
\end{tabular}\,,
\end{align*}
was auch wieder zeigt, dass die Matrix singulär ist, dass also die
Determinante verschwinden muss.

Natürlich kann man die Determinante (\ref{20000037:3det}) auch mit de
Sarrus-Formel berechnen:
\begin{align*}
\left|\,\begin{matrix}
-1& 3&-3\\
-1& 2&-7\\
 1&-4&-1
\end{matrix}\,\right|
&=
2
+
3\cdot(-7)
+
(-3)\cdot(-1)\cdot(-4)
-
2\cdot(-3)
-
(-4)\cdot(-7)\cdot(-1)
-
3
\\
&=2-21-12+6+28-3=0.
\end{align*}

Ein lustiger Effekt tritt ein, wenn man die Determinante durch Entwicklung
nach der zweiten Spalten zu berechnen versucht.
Der Entwicklungssatz liefert:
\begin{align*}
\det(A)
%\begin{pmatrix}
%-6&-1& 9&-9\\
%-6&-4& 9&-9\\
%-2&-8& 2&-7\\
% 2& 3&-4&-1
%\end{pmatrix}.
=(-1)\cdot
\left|\,\begin{matrix}
-6& 9&-9\\
-2& 2&-7\\
 2&-4&-1
\end{matrix}\,\right|
-(-4)\cdot
\left|\,\begin{matrix}
-6& 9&-9\\
-2& 2&-7\\
 2&-4&-1
\end{matrix}\,\right|
+(-8)\cdot
\left|\,\begin{matrix}
-6& 9&-9\\
-6& 9&-9\\
 2&-4&-1
\end{matrix}\,\right|
-3\cdot
\left|\,\begin{matrix}
-6& 9&-9\\
-6& 9&-9\\
-2& 2&-7
\end{matrix}\,\right|
\end{align*}
Die ersten beiden $3\times 3$-Determinanten sind identisch, sie verschwinden,
wie man mit der Sarrus-Formel nachrechnen kann.
Die letzten zwei $3\times 3$-Determinanten haben zwei gleiche Zeilen und
verschwinden daher auch.
Da alle Unterdeterminanten verschwinden folgt $\det(A)=0$.
\end{loesung}

\begin{bewertung}
Anwendung von Zeilen-Operationen zur Reduktion des Aufwandes
({\bf Z}) 1 Punkt, Anwendung des Entwicklungssatzes ({\bf E}) 1 Punkt,
Sarrus-Formel ({\bf S}) 1 Punkt, Durchführung der Rechnung ({\bf R})
3 Punkte.
\end{bewertung}

