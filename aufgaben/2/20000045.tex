Berechnen Sie die Determinante der Matrix
\[
B
=
\begin{pmatrix}
a-3&  1&  1\\
a-3&a-1&  2\\
a-3&a-1&a+1
\end{pmatrix}.
\]
Für welche Werte von $a$ ist $B$ nicht regulär?

\thema{Determinante}
\thema{Entwicklungssatz}

\begin{loesung}
Wir formen mit Hilfe der Rechenregeln für Determinanten um:
\begin{align*}
\det B
&=
(a-3)\left|\begin{matrix}
1&  1&  1\\
1&a-1&  2\\
1&a-1&a+1
\end{matrix}\right|
=
(a-3)\left|\begin{matrix}
1&  1&  1\\
0&a-2&  1\\
0&a-2&a
\end{matrix}\right|
=
(a-3)\cdot 1\cdot
\left|\begin{matrix}
a-2&  1\\
a-2&  a
\end{matrix}\right|
\\
&=
(a-3)(a-2)\cdot
\left|\begin{matrix}
1&  1\\
1&  a
\end{matrix}\right|
=
(a-3)(a-2)(a-1).
\end{align*}
Die Matrix ist also nicht regulär für $a\in\{1,2,3\}$.
\end{loesung}


