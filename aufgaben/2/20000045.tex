Berechnen Sie die Determinante der Matrix
\[
A=\begin{pmatrix}
1&2&0&0&0\\
2&0&3&4&0\\
0&3&0&0&5\\
0&4&0&0&6\\
0&0&5&6&0
\end{pmatrix}.
\]

\begin{loesung}
Wir entwickeln nach der ersten Spalte
\begin{align*}
\det(A)
&=
1\cdot
\left|\,\begin{matrix}
0&3&4&0\\
3&0&0&5\\
4&0&0&6\\
0&5&6&0
\end{matrix}\,\right|
-2\cdot
\left|\,\begin{matrix}
2&0&0&0\\
3&0&0&5\\
4&0&0&6\\
0&5&6&0
\end{matrix}\,\right|
\\
&=
-3\cdot
\left|\,\begin{matrix}
3&4&0\\
0&0&6\\
5&6&0
\end{matrix}\,\right|
+4\cdot
\left|\,\begin{matrix}
3&4&0\\
0&0&5\\
5&6&0
\end{matrix}\,\right|
-2\cdot
2\cdot
\underbrace{
\left|\,\begin{matrix}
0&0&5\\
0&0&6\\
5&6&0
\end{matrix}\,\right|
}_{\displaystyle =0}
\\
&=
3\cdot6\cdot
\left|\,\begin{matrix}
3&4\\
5&6
\end{matrix}\,\right|
-4\cdot 5\cdot
\left|\,\begin{matrix}
3&4\\
5&6
\end{matrix}\,\right|
\\
&=(3\cdot 6-4\cdot 5)(3\cdot 6-4\cdot 5)=(-2)^2=4.
\end{align*}
Die erste $4\times 4$-Determinante wird nach der ersten Spalte entwickelt,
die zweite nach der ersten Zeile.
Die letzte $3\times 3$ Determinante verschwindet, weil die
ersten zwei Zeilen linear abhängig sind.
Die ersten beiden $3\times 3$-Determinanten werden nach der letzten Spalte entwickelt.
\qedhere
\end{loesung}


\begin{bewertung}
Wahl einer geeigneten Berechnungsmethode ({\bf M}) 1 Punkt,
Teilresultate ({\bf T}) 4 Punkte,
Korrektes Resultat ({\bf D}) 1 Punkt.
\end{bewertung}
