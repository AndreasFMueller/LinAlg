Berechnen Sie die folgende Determinante
\[
\left|\,\begin{matrix}
67&5&0&2\\
0&5&0&0\\
67&0&3&0\\
0&5&0&2\\
\end{matrix}\,\right|.
\]

\thema{Determinante}

\begin{loesung}
Die Determinante kann als Produkt der Pivots bei der Durchführung des
Gauss-Algorithmus gefunden werden, in der aktuellen Form würden aber
etwas unhandliche Brüche entstehen, so dass es sich lohnt, zunächst
die gemeinsamen Faktoren in den Spalten vor die Determinante zu nehmen:
\[
\left|\,\begin{matrix}
67&5&0&2\\
0&5&0&0\\
67&0&3&0\\
0&5&0&2\\
\end{matrix}\,\right|
=
67\cdot5\cdot3\cdot 2\cdot
\left|\,\begin{matrix}
1&1&0&1\\
0&1&0&0\\
1&0&1&0\\
0&1&0&1\\
\end{matrix}\,\right|.
\]
Für diese Matrix ist die Durchführung des Gauss-Algorithmus jetzt
wesentlich einfacher:
\[
\begin{tabular}{|cccc|}
\hline
1&1&0&1\\
0&1&0&0\\
1&0&1&0\\
0&1&0&1\\
\hline
\end{tabular}
\rightarrow
\begin{tabular}{|cccc|}
\hline
1&1&0&1\\
0&1&0&0\\
0&$-1$&1&$-1$\\
0&1&0&1\\
\hline
\end{tabular}
\rightarrow
\begin{tabular}{|cccc|}
\hline
1&1&0&1\\
0&1&0&0\\
0&0&1&$-1$\\
0&0&0&1\\
\hline
\end{tabular}
\]
Da die Matrix jetzt Dreiecksform hat, kann man aufhören
(das Rückwärtseinsetzen ist für die Bestimmung der
Determinante nicht mehr nötig. Da alle Pivots $1$ waren, ist
die Determinante dieser Matrix $1$, also
\[
\left|\,\begin{matrix}
67&5&0&2\\
0&5&0&0\\
67&0&3&0\\
0&5&0&2\\
\end{matrix}\,\right|=2010.
\qedhere
\]
\end{loesung}

