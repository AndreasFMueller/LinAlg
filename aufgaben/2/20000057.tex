Bestimmen Sie den Wert der Determinanten der Matrix
\[
A=\begin{pmatrix}
   1&  a&  2&  7\\
   8&  b&  4&  1\\
   0&  c&  0&  0\\
   6&  d&  3&  1\\
\end{pmatrix}.
\]
Für welche Werte von $a,b,c$ und $d$ ist die Matrix singulär?

\begin{loesung}
Wir entwickeln nach der zweiten Spalte und erhalten
\begin{align*}
\det A
&=
-c\left|\begin{matrix}
   1&  2&  7\\
   8&  4&  1\\
   6&  3&  1
\end{matrix}\right|
\\
&=
-c\bigl(
1\cdot4\cdot1
+
2\cdot1\cdot6
+
7\cdot8\cdot3
-
6\cdot4\cdot7
-
3\cdot1\cdot1
-
1\cdot8\cdot2
\bigr)
\\
&=
-c\bigl(
4+12+168
-168-3-16
\bigr)
=
3c.
\end{align*}
Die Matrix $A$ ist genau dann singulär, wenn $\det(A)=0$ ist, was genau 
für $c=0$ eintritt.
Die Werte der anderen Variablen haben keinen Einfluss darauf, ob $A$
singulär ist.
\end{loesung}

\begin{bewertung}
Entwicklungssatz ({\bf E}) 1 Punkt,
Vorzeichenregel ({\bf V}) 1 Punkt,
Sarrus-Formel ({\bf S}) 1 Punkt,
Berechnung der Determinanten ({\bf R}) 2 Punkt,
Singularität der Matrix $A$ ({\bf Z}) 1 Punkt.
\end{bewertung}


