Auf der Krabbeldecke für Babies
\begin{center}
\includeagraphics[width=0.5\hsize]{alles.jpg}
\quad
\includeagraphics[width=0.36\hsize]{determinante.jpg}
\end{center}
findet man auch eine Determinante.
\begin{teilaufgaben}
\item Welchen Wert hat die Determinante?
\item Die unklare Orientierung der Krabbeldecke könnte dafür sorgen, dass
die Determinante verkehrt herum gelesen wird.
Wie kann man die Werte aller um Vielfache von $90^\circ$ ``gedrehten''
Determinanten bestimmen, ohne erneut zu rechnen?
\end{teilaufgaben}

\thema{Determinante}
\thema{Entwicklungssatz}

\begin{loesung}
\begin{teilaufgaben}
\item
Die Determinante kann mit dem Entwicklungssatz bestimmt werden:
\begin{align}
\left|\begin{matrix}
 1&0&1\\
-2&3&2\\
 4&5&1
\end{matrix}\right|
&=
1\cdot\left|\begin{matrix}3&2\\5&1\end{matrix}\right|
+
1\cdot\left|\begin{matrix}-2&3\\4&5\end{matrix}\right|
=
3-10 -10-12
=
-29.
\label{20000053:det}
\end{align}
\item
Wir müssen die Werte der Determinanten
\[
d_1
=
\left|\begin{matrix}
1&5& 4\\
2&3&-2\\
1&0& 1
\end{matrix}\right|,
\quad
d_2
=
\left|\begin{matrix}
1& 2&1\\
0& 3&5\\
1&-2&4
\end{matrix}\right|,
\quad
d_3
=
\left|\begin{matrix}
4&-2&1\\
5& 3&0\\
1& 2&1
\end{matrix}\right|
\]
aus der bereits bekannten Determinanten \eqref{20000053:det} ableiten.

$d_3$ kann man bekommen, indem man die Matrix transponiert und
dann die erste und letzte Spalte vertauscht. 
Transponieren ändert die Determinante nicht, aber die Vertausch
kehrt das Vorzeichen, also $d_3=29$.

$d_2$ kann man bekommen, indem die Matrix erst transponiert und dann
die erste und letzte Zeile vertauscht.
Wie im Falle von $d_3$ folgt $d_2=29$.

$d_1$ kann man erhalten, indem man die erste und letzte Zeile vertauscht
und ausserdem die erste und letzte Spalte.
Beide Vertauschung führen zu einem Vorzeichenwechsel, insgesamt bleibt
die Determinante gleich. 
Folglich ist $d_1=-29$.
\qedhere
\end{teilaufgaben}
\end{loesung}

