Rapperswil wurde 1358 erstmals urkundlich erwähnt, lange bevor 1972 die
HSR gegründet wurde.
Noch viel weiter zurück liegt der Konflikt der Helvetier mit den
Römern, der in der Schlacht bei Bibracte um 58 BCE gipfelte.
Zusammen mit dem heutigen Datum ergibt das die Matrix
\[
A
=
\begin{pmatrix}
1&3& 5&8\\
1&9& 7&2\\
0&0&-5&8\\
2&0& 1&8
\end{pmatrix}.
\]
\begin{teilaufgaben}
\item
Warum kann man sagen, dass der Wert der Determinante $\det(A)$ dieser Matrix
gerade ist, ohne sie auszurechnen?
\item
Berechnen Sie die Determinante $\det(A)$ dieser Matrix.
\end{teilaufgaben}

\thema{Determinante}
\thema{Entwicklungssatz}

\begin{loesung}
\begin{teilaufgaben}
\item
In der letzten Spalte kommt 2 als gemeinsamer Faktor vor, man kann
ihn vor die Determinante ziehen, also
\[
\det(A)
=
2\cdot
\left|\begin{matrix}
 1&3& 5&  4\\
 1&9& 7&  1\\
 0&0&-5&  4\\
 2&0& 1&  4
\end{matrix}\right|.
\]
Die verbleibende Determinante ist sicher ganzzahlig, also ist $\det(A)$
durch 2 teilbar.
\item
Indem wir die erste Zeile dreimal von der zweiten Zeile subtrahieren,
erhalten wir eine Determinante, die drei Nullen in der zweiten Spalte
enthält, was die Arbeit mit dem Entwicklungssatz etwas verkürzt.
Wir erhalten durch Entwicklung nach der zweiten Spalte und Anwendung
der Sarrus-Formel
\begin{align*}
\det(A)
&=
\left|\begin{matrix}
 1&3& 5&  8\\
-2&0&-8&-22\\
 0&0&-5&  8\\
 2&0& 1&  8
\end{matrix}\right|
=
-3\cdot\left|\begin{matrix}
-2&-8&-22\\
 0&-5&  8\\
 2& 1&  8
\end{matrix}\right|
\\
&=
-3\cdot(
(-2)\cdot(-5)\cdot 8
+
(-8)\cdot 8\cdot 2
+
(-22)\cdot0\cdot 1
-
2\cdot(-5)\cdot(-22)
-
1\cdot 8\cdot (-2)
-
8\cdot 0\cdot (-8)
)
\\
&=
-3(80-128+0-220+16+0)
=
(-3)\cdot(-252)=756.
\qedhere
\end{align*}
\end{teilaufgaben}
\end{loesung}

\begin{bewertung}
Gerader Wert der Determinante ({\bf G}) 1 Punkt,
Wahl einer geeigneten Berechnungsmethode ({\bf M}) 1 Punkt,
Teilresultate ({\bf T}) 3 Punkte,
korrektes Resultat ({\bf D}) 1 Punkt.
\end{bewertung}

