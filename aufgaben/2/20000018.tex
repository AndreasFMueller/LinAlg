Berechnen Sie die Determinante der Matrix $A$,
\[
A=\begin{pmatrix}
4&3&2&1\\
3&4&3&2\\
2&3&4&3\\
1&2&3&4
\end{pmatrix}.
\]

\thema{Determinante}
\thema{Entwicklungssatz}
\thema{Sarrus-Formel}

\begin{loesung}
Wenn man die erste Zeile zur letzten Zeile hinzuaddiert, bekommt man eine
Zeile, in der alle Elemente $5$ sind, 
\[
\det(A)
=
\left|\,
\begin{matrix}
 4& 3& 2& 1\\
 3& 4& 3& 2\\
 2& 3& 4& 3\\
 5& 5& 5& 5
\end{matrix}
\,\right|
=5
\left|\,
\begin{matrix}
4&3&2&1\\
3&4&3&2\\
2&3&4&3\\
1&1&1&1
\end{matrix}
\,\right|
\]
Jetzt kann man durch Subtrahieren von Vielfachen der letzten Zeile in
der letzten Spalte Nullen erzeugen:
\[
\det(A)
=
5
\left|\,
\begin{matrix}
 3&2&1&0\\
 1&2&1&0\\
-1&0&1&0\\
 1&1&1&1
\end{matrix}
\,\right|
=
5
\left|\,
\begin{matrix}
 3&2&1\\
 1&2&1\\
-1&0&1
\end{matrix}
\,\right|
\]
Jetzt kann man die zweite Zeile von der ersten subtrahieren
\[
\det(A)
=
5
\left|\,
\begin{matrix}
 3&2&1\\
 1&2&1\\
-1&0&1
\end{matrix}
\,\right|
=
5
\left|\,
\begin{matrix}
 2&0&0\\
 1&2&1\\
-1&0&1
\end{matrix}
\,\right|
=
10
\left|\,
\begin{matrix}
1&0&0\\
0&2&1\\
0&0&1
\end{matrix}
\,\right|
=20.
\]

Alternativ kann die Determinante auch mit dem Entwicklungssatz berechnen:
\begin{align*}
\det(A)
&=
4
\left|\,
\begin{matrix}
4&3&2\\
3&4&3\\
2&3&4
\end{matrix}
\,\right|
-3
\left|\,
\begin{matrix}
3&3&2\\
2&4&3\\
1&3&4
\end{matrix}
\,\right|
+2
\left|\,
\begin{matrix}
3&4&2\\
2&3&3\\
1&2&4
\end{matrix}
\,\right|
-1
\left|\,
\begin{matrix}
3&4&3\\
2&3&4\\
1&2&3
\end{matrix}
\,\right|
\end{align*}
Die einzelnen $3\times 3$-Determinanten kann man jetzt mit der Sarrus-Formel
ausrechnen:
\begin{align*}
\det(A)
&=
4(4\cdot 4\cdot 4 + 3\cdot 3\cdot 2 +2\cdot 3 \cdot 3
-2\cdot 4\cdot 2-3\cdot 3\cdot 4-4\cdot 3\cdot 3)
\\
&\quad-3(3\cdot 4\cdot 4+3\cdot 3\cdot 1+2\cdot 2\cdot 3
-1\cdot 4\cdot 2-3\cdot 3\cdot 3-4\cdot 2\cdot 3)
\\
&\quad+2(3\cdot 3\cdot 4+4\cdot 3\cdot 1+2\cdot 2\cdot 2
-1\cdot 3\cdot 2-2\cdot 3\cdot 3-4\cdot 2\cdot 4)
\\
&\quad-1(3\cdot 3\cdot 3+4\cdot 4\cdot 1+3\cdot 2\cdot 2
-1\cdot 3\cdot 3-2\cdot 4\cdot 3 - 3\cdot 2\cdot 4)
\\
&=
4(64 + 18 + 18 -16-36-36)
\\
&\quad-3(48+9+12 -8-27-24)
\\
&\quad+2(36+12+8 -6-18-32)
\\
&\quad-1(27+16+12 -9-24 - 24)
\\
&=
4\cdot 12
-3\cdot 10
+2\cdot 0
-1\cdot (-2)
=48-30+2=20.
\qedhere
\end{align*}
\end{loesung}
