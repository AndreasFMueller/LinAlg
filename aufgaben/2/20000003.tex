Sei $A$ eine $n\times n$-Matrix. Warum gilt
\[
\det(\lambda A)=\lambda^n\det(A)?
\]

\thema{Rechnen mit Determinanten}

\begin{loesung}
Durch die Multiplikation mit $\lambda$ werden alle $n$ Pivot-Elemente mit dem
Faktor $\lambda$ multipliziert. Da die Determinante das Produkt der $n$ Pivot-Elemente
ist, wird die Determinante mit dem Faktor $\lambda^n$ multipliziert.

Man kann $\lambda A$ auch als Matrix ansehen, in der jede Zeile mit $\lambda$
multipliziert worden ist. Jede solche Multiplikation verändert die Determinante
um den Faktor $\lambda$. Da die Matrix $n$ Zeilen hat, ist
\[
\det(\lambda A)=\lambda^n \det(A).
\qedhere
\]
\end{loesung}
