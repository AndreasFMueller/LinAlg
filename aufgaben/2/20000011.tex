Sei $M$ eine Matrix, die sich aus einzelnen Blöcken zusammensetzt:
\[
M
=
\begin{pmatrix}
A&B\\
C&D
\end{pmatrix}
\]
Darin ist $A$ eine $n\times n$-Matrix, $B$ eine $n\times m$-Matrix,
$C$ eine $m\times n$-Matrix und $D$ eine $m\times m$-Matrix. Rechnen
Sie nach, dass die Multiplikation von zwei Matrizen dieser Form
wie folgt durchzuführen ist:
\[
M_1M_2=
\begin{pmatrix}
A_1&B_1\\
C_1&D_1
\end{pmatrix}
\begin{pmatrix}
A_2&B_2\\
C_2&D_2
\end{pmatrix}
=\begin{pmatrix}
A_1A_2+B_1C_2&A_1B_2+B_1D_2\\
C_1A_2+D_1C_2&C_1B_2+D_1D_2
\end{pmatrix}.
\]
Bitte beachten Sie, dass $A_i$, $B_i$, $C_i$ und $D_i$  nicht einfach
Zahlen sind, für die dies die bekannte Multiplikationsregel ist,
sondern Matrizen.

\thema{Matrixmultiplikation}

\begin{loesung}
Um das Element in Zeile $i$ und Spalte $k$ von $M_1M_2$ zu bestimmen,
muss man die Zeile $i$ in $M_1$ mit der Spalte $k$ in $M_2$ multiplizieren.
Die Zeile $i$ in $M_1$ setzt sich aus der Zeile $i$ von $A_1$ und $B_1$
zusammen, oder falls $i > n$, aus einer entsprechenden Zeile von
$C_1$ und $D_1$.
Die Spalte $k$ von $M_2$ setzt sich aus der Spalten $k$ von
$A_2$ und $C_2$ zusammen.
Die Produkte bestehen daher aus zwei Summanden, dem Teil
der aus $A_1A_2$
und dem Teil aus $B_1C_2$. Entsprechend für grössere Werte von $i$
und $k$.
\end{loesung}

