F"ur welche Werte von $a$ ist die Matrix 
\[
A=\begin{pmatrix}
0&a&\sqrt{2}\\
a&\sqrt{2}&a\\
\sqrt{2}&a&0
\end{pmatrix}
\]
singul"ar?
Berechnen Sie ausserdem die inverse Matrix von $A$ f"ur alle anderen
Werte von $a$.

\begin{loesung}
Ob die Matrix singul"ar ist, kann man am leichtesten mit der Determinanten
entscheiden. 
Diese ist nach der Sarrus-Formel
\begin{align*}
\det(A)
&=
0\cdot \sqrt{2}\cdot 0+a^2\sqrt{2}+a^2\sqrt{2}-2\sqrt{2}-a^2\cdot 0 - 0\cdot a^2
\\
&=2a^2\sqrt{2}-2\sqrt{2}=2\sqrt{2}(a^2-1).
\end{align*}
Dieser Ausdruck verschwindet genau dann, wenn $a^2=1$,
also sind $a=\pm1$ die beiden Werte, f"ur die die Matrix singul"ar wird.

Die inverse Matrix kann man jetzt, da man die Determinante ja schon berechnet
hat, mit Hilfe von Minoren bestimmen.
\begin{align*}
A^{-1}
&=
\frac1{2\sqrt{2}(a^2-1)}
{\def\arraystretch{2.4}
\begin{pmatrix}
{\def\arraystretch{1}
  \left|\begin{matrix}\sqrt{2}&       a\\       a&       0\end{matrix}\right|}
&{\def\arraystretch{1}
 -\left|\begin{matrix}       a&\sqrt{2}\\       a&       0\end{matrix}\right|}
&{\def\arraystretch{1}
  \left|\begin{matrix}       a&\sqrt{2}\\\sqrt{2}&       a\end{matrix}\right|}
\\
{\def\arraystretch{1}
 -\left|\begin{matrix}       a&       a\\\sqrt{2}&       0\end{matrix}\right|}
&{\def\arraystretch{1}
  \left|\begin{matrix}       0&\sqrt{2}\\\sqrt{2}&       0\end{matrix}\right|}
&{\def\arraystretch{1}
 -\left|\begin{matrix}       0&\sqrt{2}\\       a&       a\end{matrix}\right|}
\\
{\def\arraystretch{1}
  \left|\begin{matrix}       a&\sqrt{2}\\\sqrt{2}&       a\end{matrix}\right|}
&{\def\arraystretch{1}
 -\left|\begin{matrix}       0&       a\\\sqrt{2}&       a\end{matrix}\right|}
&{\def\arraystretch{1}
  \left|\begin{matrix}       0&       a\\       a&\sqrt{2}\end{matrix}\right|}
\end{pmatrix}}
=
\frac1{2\sqrt{2}(a^2-1)}
\begin{pmatrix}
-a^2&a\sqrt{2}&a^2-2\\
a\sqrt{2}&-2&a\sqrt{2}\\
a^2-2&a\sqrt{2}&-a^2
\end{pmatrix}.
\end{align*}
\end{loesung}

\begin{bewertung}
Anwendung der Sarrus-Formel ({\bf S}) 1 Punkt,
Berechnung der Determinante ({\bf D}) 1 Punkt,
Werte von $a$, f"ur die die Matrix singul"ar wird ({\bf A}) 1 Punkt,
Minor-Formel f"ur die Inverse ({\bf M}) 1 Punkt,
Berechnung der Inversen ({\bf I}) 2 Punkte.
\end{bewertung}
