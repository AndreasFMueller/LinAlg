Bestimmen Sie eine Kettenbruchentwicklung für $\!\sqrt{5}$.

\begin{loesung}
Wegen $\!\sqrt{5} \approx 2.236067977499789696$ ist $a_0=2$.
Der nächste Teilnenner wird bestimmt, indem man den Kehrwert von
$\!\sqrt{5}-a_0$ ausrechnet:
\begin{equation}
\frac{1}{\!\sqrt{5}-a_0} = 4.236067977499789696,
\label{20000073:eqn}
\end{equation}
also ist $a_1$ der nächste Teilnenner.
Für die folgenden Teilnenner muss dieser Prozess wiederholt werden.
Weil der Nachkommateil von \eqref{20000073:eqn} der selbe ist wie
der von $\!\sqrt{5}$, wird sich immer wieder der gleiche Teilnenner
ergeben.
Dies kann man auch beweisen:
\begin{align*}
&\text{für $a_0$:}&
\frac{1}{\sqrt{5}-2}
&=
\frac{\sqrt{5}+2}{5-4}
=
\sqrt{5}+2
&&\Rightarrow& a_0&=2
\\
&\text{für $a_k$, $k>0$:}&
\frac{1}{{\color{red}\sqrt{5}+2}-4}
&=
\frac{1}{\sqrt{5}-2}
=
\frac{\sqrt{5}+2}{5-4}
=
{\color{red}\sqrt{5}+2}
&&\Rightarrow& a_k&=4.
\end{align*}
Es folgt, dass
\[
\sqrt{5}
=
[2;4,4,4,4,\dots]
=
2 + \cfrac{1}{4+\cfrac{1}{4 + \cfrac{1}{4+\cfrac{1}{4+\cfrac{1}{\dots}}}}}
.
\qedhere
\]
\end{loesung}
