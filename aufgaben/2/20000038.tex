Berechnen Sie die Determinante 
\[
A=\left|\,\begin{matrix}
 1& 0&-1&0\\
 2&-2& 0&4\\
 0&-3& 3&0\\
-2& 0&-1&3
\end{matrix}\,\right|
\]

\begin{loesung}
Zun"achst kann man die gemeinsamen Faktoren $2$ und $3$ in den Zeilen 2 und 3
herausziehen:
\begin{align*}
\det(A)
&=
2\cdot 3\cdot
\left|\,\begin{matrix}
 1& 0&-1&0\\
 1&-1& 0&2\\
 0&-1& 1&0\\
-2& 0&-1&3
\end{matrix}\,\right|
\end{align*}
Indem man die zweite Zeile zur dritten addiert und das resultate fon der zweiten
zweiten subtrahiert,
kann man die Zahl der Nullen in der zweiten Zeile weiter reduzieren und
nochmals einen Faktor 2 herausziehen:
\begin{align*}
\det(A)
&=
2\cdot 3\cdot
\left|\,\begin{matrix}
 1& 0&-1&0\\
 0& 0& 0&2\\
 1&-1& 0&0\\
-2& 0&-1&3
\end{matrix}\,\right|
=
2\cdot 3\cdot2\cdot
\left|\,\begin{matrix}
 1& 0&-1&0\\
 0& 0& 0&1\\
 1&-1& 0&0\\
-2& 0&-1&3
\end{matrix}\,\right|
\end{align*}
Entwicklung nach der zweiten Zeile gibt
\begin{align*}
\det(A)
&=
2\cdot 3\cdot2\cdot
\left|\,\begin{matrix}
 1& 0&-1\\
 1&-1& 0\\
-2& 0&-1
\end{matrix}\,\right|
=
12\cdot \biggl(
1\cdot
\left|\,\begin{matrix}-1& 0\\ 0&-1\end{matrix}\,\right|
+
(-1)\cdot
\left|\,\begin{matrix} 1&-1\\-2& 0\end{matrix}\,\right|
\biggr)
\\
&=12\cdot(1-(-2))=3\cdot 12=36.
\qedhere
\end{align*}

Nat"urlich kann man die Determinante auch durch direkte Anwendung des
Entwicklungssatzes bestimmen.
Die letzte Spalte enth"alt zwei Nullen, daher bietet sie sich f"ur die
Entwicklung an:
\begin{align*}
\det a
&=
4\cdot
\left|\begin{matrix}
 1& 0&-1\\
 0&-3& 3\\
-2& 0&-1
\end{matrix}\right|
+3\cdot
\left|\begin{matrix}
 1& 0&-1\\
 2&-2& 0\\
 0&-3& 3
\end{matrix}\right|
\\
&=
4\cdot(3+6)+3\cdot(-6+6)=4\cdot 9=36.
\qedhere
\end{align*}
\end{loesung}

\begin{bewertung}
6 Punkte.
\end{bewertung}

