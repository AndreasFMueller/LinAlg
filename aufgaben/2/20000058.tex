Berechnen Sie den Wert der Determinanten der Matrix
\[
A
=
\begin{pmatrix*}[r]
   4 &    6 &   -4 &   -1  \\
   0 &    0 &   -6 &    4  \\
  -6 &    8 &   -7 &    0  \\
  -9 &   -2 &   -4 &    9 
\end{pmatrix*}.
\]



\begin{loesung}
Die Determinante ist am einfachsten mit dem Entwicklungssatz zu berechnen.
Wir entwickeln nach der zweiten Zeile und erhalten
\begin{align*}
\det(A)
&=
-(-6)\cdot
\left|\,\begin{matrix*}[r]
 4 &  6 & -1 \\
-6 &  8 &  0 \\
-9 & -2 &  9
\end{matrix*}\,\right|
+4\cdot
\left|\,\begin{matrix*}[r]
 4 &  6 & -4 \\
-6 &  8 & -7 \\
-9 & -2 & -4
\end{matrix*}\,\right|.
\intertext{Die $3\times 3$-Determinanten können mit der Sarrus-Regel
ausgewertet werden:}
&=
6\cdot\bigl(
288+0-12-72+0+324
\bigr)
+4\cdot\bigl(
-128+378-48-288-56-144
\bigr)
\\
&=
6\cdot 528 +4\cdot (-286)
=
2024.
\qedhere
\end{align*}
\end{loesung}

\begin{bewertung}
Entwicklungssatz ({\bf E}) 1 Punkt,
Vorzeichenregel ({\bf V}) 1 Punkt,
Unterdeterminanten ({\bf U}) 1 Punkt,
Sarrus ({\bf S}) 1 Punkt,
Berechnung der Unterdeterminanten ({\bf B}) 1 Punkt,
Wert der Determinanten ({\bf D}) 1 Punkt.
\end{bewertung}
