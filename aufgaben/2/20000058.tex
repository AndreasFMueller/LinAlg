Die Zahlen $a$, $b$, $c$ und $d$ erfüllen
\[
\frac{a-b}{c-d}=2
\qquad\text{und}\qquad
\frac{a-c}{b-d}=3.
\]
Welche Werte kann
\[
u = \frac{a-d}{b-c}
\]
annehmen?


\begin{hinweis}
Verwenden Sie die Unbekannten $x=a-d$, $y=b-d$ und $z=c-d$.
\end{hinweis}

\begin{loesung}
Die gegebenen Bedingungen werden mit
\begin{align*}
a-b &= x-y \\
a-c &= x-z \\
b-c &= y-z
\end{align*}
zu
\[
\frac{x-y}{z} = 2 ,
\qquad
\frac{x-z}{y} = 3
\qquad\text{und}\qquad
\frac{x}{y-z}=u.
\]
Durch Multiplizieren mit den Nennern kann man diese Bedinungen in die
Form der linearen Gleichungen
\[
\begin{array}{rcrcrcrc}
x&-& y&-&2z&=&0\\
x&-&3y&-& z&=&0\\
x&-&uy&+&uz&=&0
\end{array}.
\]
Der Gauss-Algorithmus ergibt
\begin{align*}
\begin{tabular}{| >{$}c<{$} >{$}c<{$} >{$}c<{$}| >{$}c<{$}|}
\hline
x&y&z&1\\
\hline
1&-1&-2&0\\
1&-3&-1&0\\
1&-u& u&0\\
\hline
\end{tabular}
&
\to
\begin{tabular}{| >{$}c<{$} >{$}c<{$} >{$}c<{$}| >{$}c<{$}|}
\hline
x&y&z&1\\
\hline
1& -1& -2&0\\
0& -2&  1&0\\
0&1-u&2+u&0\\
\hline
\end{tabular}
\to
\begin{tabular}{| >{$}c<{$} >{$}c<{$} >{$}c<{$}| >{$}c<{$}|}
\hline
x&y&z&1\\
\hline
1& -1& -2&0\\
0&  1& -\frac12&0\\
0&  0&\frac{5}{2}+\frac12u&0\\
\hline
\end{tabular}
\to
\begin{tabular}{| >{$}c<{$} >{$}c<{$} >{$}c<{$}| >{$}c<{$}|}
\hline
x&y&z&1\\
\hline
1&  0& -\frac52&0\\
0&  1& -\frac12&0\\
0&  0&\frac{5}{2}+\frac12u&0\\
\hline
\end{tabular}
\end{align*}
Die letzte Gleichung bedeutet
\[
\frac12(5+u)z=0.
\]
Da $z\ne 0$ sein muss, damit die gegebenen Bedingungen überhaupt
sinnvoll sind, folgt, dass $u=-5$ sein muss.
In diesem Fall ist die dritte Gleichung linear unabhängig,
Zu gegebenem $z$ können $x$ und $y$ als
\[
x = \frac52x,\quad y = \frac12z
\]
bestimmt werden.
Zum Beipsiel kann man $d=0$ und $c=2$ wählen, dann müssen
$x=a=5$ und $y=b=1$ sein und es folgt:
\[
\frac{a-d}{b-c}
=
\frac{5-0}{1-2}
=
-5,
\]
womit das gefundene Resultat bestätigt ist.
\end{loesung}
