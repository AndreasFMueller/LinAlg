Für die Kettenbruchentwicklung von $\varphi-1$ ist die $Q$-Matrix
\[
Q_k
=
\begin{pmatrix}
0&b_k\\
1&a_k
\end{pmatrix}
=
\begin{pmatrix}
0&1\\
1&1
\end{pmatrix}
=
Q
\]
\begin{teilaufgaben}
\item
Berechnen Sie die Potenzen  $Q^2$, $Q^3$, $Q^4$,\dots
\item
Lesen Sie in der zweiten Spalte der Matrizen $Q^k$ die Näherungsbrücke 
für $\varphi-1$ ab.
\end{teilaufgaben}

\begin{loesung}
\begin{teilaufgaben}
Die Produkte sind
\begin{align*}
Q^2
&=
\begin{pmatrix}
0&1\\
1&1
\end{pmatrix}
\begin{pmatrix}
0&1\\
1&1
\end{pmatrix}
=
\begin{pmatrix}
1 & 1 \\
1 & 2
\end{pmatrix}
\\
Q^3
&=
\begin{pmatrix}
1&1\\
1&2
\end{pmatrix}
\begin{pmatrix}
0&1\\
1&1
\end{pmatrix}
=
\begin{pmatrix}
1 & 2 \\
2 & 3
\end{pmatrix}
\\
Q^4
&=
\begin{pmatrix}
1 & 2 \\
2 & 3
\end{pmatrix}
\begin{pmatrix}
0&1\\
1&1
\end{pmatrix}
=
\begin{pmatrix}
 2 & 3 \\
 3 & 5
\end{pmatrix}
\\
Q^5
&=
\begin{pmatrix}
 2 & 3 \\
 3 & 5
\end{pmatrix}
\begin{pmatrix}
0&1\\
1&1
\end{pmatrix}
=
\begin{pmatrix}
 3 & 5 \\
 5 & 8
\end{pmatrix}
\\
Q^6
&=
\begin{pmatrix}
 3 & 5 \\
 5 & 8
\end{pmatrix}
\begin{pmatrix}
0&1\\
1&1
\end{pmatrix}
=
\begin{pmatrix}
 5 & 8\\
 8 & 13
\end{pmatrix}
\\
Q^7
&=
\begin{pmatrix}
 5 & 8 \\
 8 & 13
\end{pmatrix}
\begin{pmatrix}
0&1\\
1&1
\end{pmatrix}
=
\begin{pmatrix}
 8  & 13 \\
 13 & 21
\end{pmatrix}
\end{align*}
\item
Die Näherungsbrüche sind
\begin{align*}
\frac{1}{1}     &= 1                                  \\
\frac{1}{2}     &= 0.5                                \\
\frac{2}{3}     &= 0.\underline{6}6666666666666666666 \\
\frac{3}{5}     &= 0.\underline{6}                    \\
\frac{5}{8}     &= 0.\underline{6}25                  \\
\frac{8}{13}    &= 0.\underline{61}538461538461538461 \\
\frac{13}{21}   &= 0.\underline{61}904761904761904761 \\
\frac{21}{34}   &= 0.\underline{61}764705882352941176 \\
\frac{34}{55}   &= 0.\underline{618}18181818181818181 \\
\frac{55}{89}   &= 0.\underline{61}797752808988764044 \\
\frac{89}{144}  &= 0.\underline{6180}5555555555555555 \\
\frac{144}{233} &= 0.\underline{6180}2575107296137339 \\
\frac{233}{337} &= 0.\underline{61803}713527851458885 \\
\varphi-1       &\approx 0.61803398874989484820.
\end{align*}
Die korrekten Stellen sind unterstrichen.
\qedhere
\end{teilaufgaben}
\end{loesung}

