%
% original: 20000001.tex
%
Berechnen Sie die folgenden Determinanten
\begin{teilaufgaben}
\item $A=\left|\,\begin{matrix}1&6\\4&1\end{matrix}\,\right|$
\item $B=\left|\,\begin{matrix}a+b&a\\b&a+b\end{matrix}\,\right|$
\item $C=\left|\,\begin{matrix}2&1&1\\0&5&-2\\1&-3&4\end{matrix}\,\right|$
\end{teilaufgaben}

\thema{Determinante}

\begin{loesung}
\begin{teilaufgaben}
\item
\[
\left|\,\begin{matrix}1&6\\4&1\end{matrix}\,\right|=1\cdot 1-6\cdot 4=1-24=-23.
\]
\item
\[
\left|\,\begin{matrix}a+b&a\\b&a+b\end{matrix}\,\right|=(a+b)^2 -ab=a^2+ab+b^2.
\]
\item Durch Vertauschung der Zeilen erreicht man zunächst die einfachere
Situation, in der wir den Gauss-Algorithmus anwenden können
\begin{align*}
\det(C)&=
\left|\,\begin{matrix}
1&-3&4\\
2&1&1\\
0&5&-2
\end{matrix}\,\right|
\\
&=
\left|\,\begin{matrix}
1&-3&4\\
0&7&-7\\
0&5&-2
\end{matrix}\,\right|
\\
&=
7\left|\,\begin{matrix}
1&-3&4\\
0&1&-1\\
0&0&3
\end{matrix}\,\right|
\\
&=
7\cdot 3
\left|\,\begin{matrix}
1&-3&4\\
0&1&-1\\
0&0&1
\end{matrix}\,\right|
\\
&=21
\left|\,\begin{matrix}
1&0&0\\
0&1&0\\
0&0&1
\end{matrix}\,\right|=21
\qedhere
\end{align*}
\end{teilaufgaben}
\end{loesung}
