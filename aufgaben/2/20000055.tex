Berechnen Sie die Determinante der Matrix
\[
A =
\begin{pmatrix*}[r]
   8 & 6 & 0 &-1 & 8\\
   6 & 0 & 0 & 0 & 0\\
   9 & 8 & 6 & 8 & 0\\
   0 & 2 & 0 & 0 & 0\\
   7 & 0 & 1 &-1 & 8
%ans =  768.00
\end{pmatrix*}.
\]
Ist $A$ regulär?

\begin{loesung}
Wir verwenden den Entwicklungssatz und entwickeln nach der zweiten Zeile
\begin{align*}
\det(A)
&=
-6\cdot
\left|\;\begin{matrix*}[r]
 6 & 0 &-1 & 8\\
 8 & 6 & 8 & 0\\
 2 & 0 & 0 & 0\\
 0 & 1 &-1 & 8
\end{matrix*}\;\right|
\intertext{%
Die $4\times 4$-Determinante lässt sich am besten nach der dritten Zeile
entwicklen:}
\left|\;\begin{matrix*}[r]
 6 & 0 &-1 & 8\\
 8 & 6 & 8 & 0\\
 2 & 0 & 0 & 0\\
 0 & 1 &-1 & 8
\end{matrix*}\;\right|
&=
2
\cdot
\left|\;\begin{matrix*}[r]
 0 &-1 & 8\\
 6 & 8 & 0\\
 1 &-1 & 8
\end{matrix*}\;\right|.
\intertext{Für die $3\times 3$-Determinante bietet sich die Sarrus-Formel
an:}
\left|\;\begin{matrix*}[r]
 0 &-1 & 8\\
 6 & 8 & 0\\
 1 &-1 & 8
\end{matrix*}\;\right|
&=
0\cdot 8 \cdot 8
+
(-1)\cdot 0 \cdot 1
+
8\cdot 6\cdot (-1)
-
1\cdot 8\cdot 8
-
(-1)\cdot 0 \cdot 0
-
8\cdot 6\cdot (-1)
\\
&=
-48-64+48
=
-64
\\
\det(A)
&=
-6\cdot 2\cdot (-64)
=
768.
\end{align*}
Da die Determinante von $0$ verschieden ist, ist $A$ regulär.
\end{loesung}

\begin{bewertung}
Berechnung der Determinante mit dem Entwicklungssatz ({\bf E}) 1 Punkt,
Durchführung der Rechnung ({\bf D}) 2 Punkt,
Sarrus-Formel ({\bf S}) 1 Punkt,
Wert ({\bf W}) 1 Punkt,
$A$ regulär ({\bf R}) 1 Punkt.
\end{bewertung}



