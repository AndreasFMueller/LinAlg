Sei $A$ eine $n\times n$-Matrix und $B$ eine $m\times m$-Matrix. Aus diesen beiden
Matrizen kann man eine neue $(n+m)\times(n+m)$-Matrix $M$ bilden:
\[
M=\begin{pmatrix}
A&0\\
0&B
\end{pmatrix}
\]
Eine solche Matrix heisst Blockmatrix.
\begin{teilaufgaben}
\item
Zeigen Sie:
Ist $B=E$ die $m\times m$-Einheitsmatrix, dann ist $\det(M)=\det(A)$.
\item
Zeigen Sie:
$\det(M)=\det(A)\det(B)$.
\end{teilaufgaben}

\thema{Produktsatz}
\thema{Determinante}
\thema{Entwicklungssatz}

\begin{loesung}
\begin{teilaufgaben}
\item Wir bezeichnen (temporär) mit $E_k$ die $k\times k$-Einheitsmatrix,
und berechnen die Determinante mit Hilfe des Entwicklungssatzes nach der
letzten Spalte:
\[
\left|\,
\begin{matrix}
A&0\\
0&E_m
\end{matrix}
\,\right|
=0\cdot +\dots+(-1)^{2m+2n}1\cdot
\left|\,
\begin{matrix}
A&0\\
0&E_{m-1}
\end{matrix}
\,\right|
=
\left|\,
\begin{matrix}
A&0\\
0&E_{m-2}
\end{matrix}
\,\right|
=\dots=
\det(A).
\]

\item
Man kann die Matrix $M$ in ein Produkt zerlegen:
\[
M=
\begin{pmatrix}
A&0\\
0&B
\end{pmatrix}
=
\begin{pmatrix}
AE&0\\
0&EB
\end{pmatrix}
=
\begin{pmatrix}
A&0\\
0&E
\end{pmatrix}
\begin{pmatrix}
E&0\\
0&B
\end{pmatrix}
\]
Nach der Produktformel für die Determinante gilt jetzt
\[
\det(M) =
\det
\begin{pmatrix}
A&0\\
0&E
\end{pmatrix}
\det
\begin{pmatrix}
E&0\\
0&B
\end{pmatrix}
=\det(A)\det(B),
\]
wobei wir das Resultat von a) verwendet haben.
\qedhere
\end{teilaufgaben}
\end{loesung}

