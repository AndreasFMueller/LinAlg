Ein $2^2\times 2^2$-Sudoku sieht zum Beispiel so aus:
\begin{center}
\begin{tabular}{|cc|cc|}
\hline
1&2&3&4\\
3&4&1&2\\
\hline
2&3&4&1\\
4&1&2&3\\
\hline
\end{tabular}
\end{center}
\begin{teilaufgaben}
\item Berechnen Sie die Determinante dieser Zahlen, betrachtet
als Einträge einer $4\times 4$-Matrix.
\item Wenn man die $2\times 2$-Unterquadrate jeweils über die Diagonalen
die Plätze tauschen lässt, so dass also das Unterquadrat links unten
mit dem Unterquadrat rechts oben den Platz tauscht, und analog die beiden
anderen Unterquadrate, entsteht wieder ein $2^2\times 2^2$-Sudoku.
Was passiert bei diesem Prozess mit der  Determinante?
\end{teilaufgaben}

\themaS{Determinante}
\themaS{Entwicklungssatz}

\begin{loesung}
\begin{teilaufgaben}
\item
Die Determinante kann mit dem Entwicklungssatz bestimmt werden. Wir
entwickeln nach der ersten Spalte:
\begin{align*}
\left|\,
\begin{matrix}
1&2&3&4\\
3&4&1&2\\
2&3&4&1\\
4&1&2&3
\end{matrix}
\,\right|
&=
1\cdot\left|\,
\begin{matrix}
4&1&2\\
3&4&1\\
1&2&3
\end{matrix}
\,\right|
-3\cdot\left|\,
\begin{matrix}
2&3&4\\
3&4&1\\
1&2&3
\end{matrix}
\,\right|
+2\cdot\left|\,
\begin{matrix}
2&3&4\\
4&1&2\\
1&2&3
\end{matrix}
\,\right|
-4\cdot \left|\,
\begin{matrix}
2&3&4\\
4&1&2\\
3&4&1\\
\end{matrix}
\,\right|
\\
&=
1\cdot(48+1+12-8-8-9)
-3\cdot(24+3+24-16-4-27)\\
&\quad+2\cdot(6+6+32-4-8-36)
-4\cdot(2+18+64-12-16-12)
\\
&=1\cdot 36
-3\cdot 4
+2\cdot (-4)
-4\cdot 44
\\
&=
36-12-8-176=-160
\end{align*}


Wir kontrollieren das Resultat mit dem Gauss-Algorithmus, der
die Determinante als das Produkt der Pivots zu bestimmen erlaubt:
\begin{align*}
\begin{tabular}{|>{$}c<{$}>{$}c<{$}>{$}c<{$}>{$}c<{$}|}
\hline
\begin{picture}(0,0)
\color{red}\put(3,4){\circle{12}}
\end{picture}%
1&2&3&4\\
3&4&1&2\\
2&3&4&1\\
4&1&2&3\\
\hline
\end{tabular}
&
\rightarrow
\begin{tabular}{|>{$}c<{$}>{$}c<{$}>{$}c<{$}>{$}c<{$}|}
\hline
1& 2&  3&  4\\
0&-2& -8&-10\\
0&-1& -2& -7\\
0&-7&-10&-13\\
\hline
\end{tabular}
\rightarrow
\begin{tabular}{|>{$}c<{$}>{$}c<{$}>{$}c<{$}>{$}c<{$}|}
\hline
1&2& 3& 4\\
0&1%
\begin{picture}(0,0)
\color{red}\put(-3,4){\circle{12}}
\end{picture}%
& 2& 7\\
0&2& 8&10\\
0&7&10&13\\
\hline
\end{tabular}
\\
&
\rightarrow
\begin{tabular}{|>{$}c<{$}>{$}c<{$}>{$}c<{$}>{$}c<{$}|}
\hline
1&2& 3&  4\\
0&1& 2&  7\\
0&0& 4%
\begin{picture}(0,0)
\color{red}\put(-3,4){\circle{12}}
\end{picture}%
& -4\\
0&0&-4&-36\\
\hline
\end{tabular}
\rightarrow
\begin{tabular}{|>{$}c<{$}>{$}c<{$}>{$}c<{$}>{$}c<{$}|}
\hline
1&2&3&  4\\
0&1&2&  7\\
0&0&1& -1\\
0&0&0&-40%
\begin{picture}(0,0)
\color{red}\put(-9,4){\circle{19}}
\end{picture}%
\\
\hline
\end{tabular}
\end{align*}
Im zweiten Schritt haben wir uns abweichend vom Standardalgorithmus
die Vertauschung der zweiten und dritten Zeile erlaubt, was das Vorzeichen 
der Determinante umkehrt. Wir haben aber auch in drei Zeilen die Vorzeichen
gewechselt, was das Vorzeichen der Determinante ebenfalls dreimal umkehrt,
womit die Determinante insgesamt gleich bleibt.

Die Pivots sind $1$, $1$, $4$ und $-40$, also ist die Determinante $-160$.

\item Die Vertauschung der $2\times 2$-Unterquadrate kann dadurch
ausgeführt werden, dass man die erste Zeile mit der dritten vertauscht,
und die zweite mit der vierten, und analog für die Spalten. Jede
solche Vertauschung kehrt das Vorzeichen, da insgesammt eine gerade
Anzahl von Vertauschung ausgeführt wird, bleibt das Vorzeichen das
selbe.
\qedhere
\end{teilaufgaben}
\end{loesung}
