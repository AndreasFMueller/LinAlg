Für welche Werte von $a$ ist die Matrix 
\[
A=\begin{pmatrix}
0&a&\sqrt{2}\\
a&\sqrt{2}&a\\
\sqrt{2}&a&0
\end{pmatrix}
\]
singulär?
Berechnen Sie ausserdem die inverse Matrix von $A$ für alle anderen
Werte von $a$.

\thema{Sarrus-Formel}
\thema{Determinante mit Parameter}
\thema{inverse Matrix}

\begin{loesung}
Ob die Matrix singulär ist, kann man am leichtesten mit der Determinanten
entscheiden. 
Diese ist nach der Sarrus-Formel
\begin{align*}
\det(A)
&=
0\cdot \sqrt{2}\cdot 0+a^2\sqrt{2}+a^2\sqrt{2}-2\sqrt{2}-a^2\cdot 0 - 0\cdot a^2
\\
&=2a^2\sqrt{2}-2\sqrt{2}=2\sqrt{2}(a^2-1).
\end{align*}
Dieser Ausdruck verschwindet genau dann, wenn $a^2=1$,
also sind $a=\pm1$ die beiden Werte, für die die Matrix singulär wird.

Die inverse Matrix kann man jetzt, da man die Determinante ja schon berechnet
hat, mit Hilfe von Minoren bestimmen.
\begin{align*}
A^{-1}
&=
\frac1{2\sqrt{2}(a^2-1)}
{\def\arraystretch{2.4}
\begin{pmatrix}
{\def\arraystretch{1}
  \left|\begin{matrix}\sqrt{2}&       a\\       a&       0\end{matrix}\right|}
&{\def\arraystretch{1}
 -\left|\begin{matrix}       a&\sqrt{2}\\       a&       0\end{matrix}\right|}
&{\def\arraystretch{1}
  \left|\begin{matrix}       a&\sqrt{2}\\\sqrt{2}&       a\end{matrix}\right|}
\\
{\def\arraystretch{1}
 -\left|\begin{matrix}       a&       a\\\sqrt{2}&       0\end{matrix}\right|}
&{\def\arraystretch{1}
  \left|\begin{matrix}       0&\sqrt{2}\\\sqrt{2}&       0\end{matrix}\right|}
&{\def\arraystretch{1}
 -\left|\begin{matrix}       0&\sqrt{2}\\       a&       a\end{matrix}\right|}
\\
{\def\arraystretch{1}
  \left|\begin{matrix}       a&\sqrt{2}\\\sqrt{2}&       a\end{matrix}\right|}
&{\def\arraystretch{1}
 -\left|\begin{matrix}       0&       a\\\sqrt{2}&       a\end{matrix}\right|}
&{\def\arraystretch{1}
  \left|\begin{matrix}       0&       a\\       a&\sqrt{2}\end{matrix}\right|}
\end{pmatrix}}
=
\frac1{2\sqrt{2}(a^2-1)}
\begin{pmatrix}
-a^2&a\sqrt{2}&a^2-2\\
a\sqrt{2}&-2&a\sqrt{2}\\
a^2-2&a\sqrt{2}&-a^2
\end{pmatrix}.
\end{align*}
Man kann die Inverse Matrix natürlich auch mit dem Gauss-Algorithmus
ausrechnen, das ist in diesem Fall aber ziemlich schmerzhaft.
Wie man den Gaussalgorithmus schrittweise mit Maxima durchführen kann,
zeigt das folgende Maxima-Programm:
\verbatimainput{gauss.maxima}
Es liefert schrittweise die folgenden Tableau:
\begin{align*}
%                    [    0        a     sqrt(2)  1  0  0 ]
%                    [                                    ]
%(%o1)               [    a     sqrt(2)     a     0  1  0 ]
%                    [                                    ]
%                    [ sqrt(2)     a        0     0  0  1 ]
\begin{tabular}{|>{$}c<{$}>{$}c<{$}>{$}c<{$}|>{$}c<{$}>{$}c<{$}>{$}c<{$}|}
\hline
       0&       a&\sqrt{2}& 1& 0& 0\\
       a&\sqrt{2}&       a& 0& 1& 0\\
\sqrt{2}&       a&       0& 0& 0& 1\\
\hline
\end{tabular}
&\rightarrow
%                    [ sqrt(2)     a        0     0  0  1 ]
%                    [                                    ]
%(%o2)               [    a     sqrt(2)     a     0  1  0 ]
%                    [                                    ]
%                    [    0        a     sqrt(2)  1  0  0 ]
\begin{tabular}{|>{$}c<{$}>{$}c<{$}>{$}c<{$}|>{$}c<{$}>{$}c<{$}>{$}c<{$}|}
\hline
\sqrt{2}&       a&       0& 0& 0& 1\\
       a&\sqrt{2}&       a& 0& 1& 0\\
       0&       a&\sqrt{2}& 1& 0& 0\\
\hline
\end{tabular}
\\
%
%                    [       a                       1    ]
%                    [ 1  -------     0     0  0  ------- ]
%                    [    sqrt(2)                 sqrt(2) ]
%(%o4)               [                                    ]
%                    [ a  sqrt(2)     a     0  1     0    ]
%                    [                                    ]
%                    [ 0     a     sqrt(2)  1  0     0    ]
&\rightarrow
\begin{tabular}{|>{$}c<{$}>{$}c<{$}>{$}c<{$}|>{$}c<{$}>{$}c<{$}>{$}c<{$}|}
\hline
1&\frac{a}{\sqrt{2}}&       0& 0& 0&\frac{1}{\sqrt{2}}\\
a&         \sqrt{2} &       a& 0& 1&                 0\\
0&                 a&\sqrt{2}& 1& 0&                 0\\
\hline
\end{tabular}
\\
%
%                  [        a                         1     ]
%                  [ 1   -------      0     0  0   -------  ]
%                  [     sqrt(2)                   sqrt(2)  ]
%                  [                                        ]
%(%o5)             [       2                                ]
%                  [      a  - 2                       a    ]
%                  [ 0  - -------     a     0  1  - ------- ]
%                  [      sqrt(2)                   sqrt(2) ]
%                  [                                        ]
%                  [ 0      a      sqrt(2)  1  0      0     ]
&\rightarrow
\begin{tabular}{|>{$}c<{$}>{$}c<{$}>{$}c<{$}|>{$}c<{$}>{$}c<{$}>{$}c<{$}|}
\hline
1&     \frac{a}{\sqrt{2}}&       0& 0& 0& \frac{1}{\sqrt{2}}\\
a&-\frac{a^2-2}{\sqrt{2}}&       a& 0& 1&-\frac{a}{\sqrt{2}}\\
0&                      a&\sqrt{2}& 1& 0&                  0\\
\hline
\end{tabular}
\\
%
%            [       a                                      1     ]
%            [ 1  -------        0         0      0      -------  ]
%            [    sqrt(2)                                sqrt(2)  ]
%            [                                                    ]
%            [                sqrt(2) a         sqrt(2)     a     ]
%            [ 0     1      - ---------    0  - -------   ------  ]
%(%o9)       [                  2                2         2      ]
%            [                 a  - 2           a  - 2    a  - 2  ]
%            [                                                    ]
%            [              3/2  2    3/2                     2   ]
%            [             2    a  - 2        sqrt(2) a      a    ]
%            [ 0     0     --------------  1  ---------  - ------ ]
%            [                  2               2           2     ]
%            [                 a  - 2          a  - 2      a  - 2 ]
%
&\rightarrow
\begin{tabular}{|>{$}c<{$}>{$}c<{$}>{$}c<{$}|>{$}c<{$}>{$}c<{$}>{$}c<{$}|}
\hline
1&\frac{a}{\sqrt{2}}&                           0&
	0&                       0&\frac{1}{\sqrt{2}}\\
0&                 1&    -\frac{a\sqrt{2}}{a^2-2}&
	0& -\frac{\sqrt{2}}{a^2-2}&     \frac{a}{a^2}\\
0&                 0&2\sqrt{2}\frac{a^2-1}{a^2-2}&
	1& \frac{a\sqrt{2}}{a^2-2}&-\frac{a^2}{a^2-2}\\
\hline
\end{tabular}
\\
%
%       [       a                                                    1         ]
%       [ 1  -------       0             0             0          -------      ]
%       [    sqrt(2)                                              sqrt(2)      ]
%       [                                                                      ]
%       [               sqrt(2) a                    sqrt(2)         a         ]
%       [ 0     1     - ---------        0         - -------       ------      ]
%(%o11) [                 2                           2             2          ]
%       [                a  - 2                      a  - 2        a  - 2      ]
%       [                                                                      ]
%       [                               2                              2       ]
%       [                              a  - 2         a               a        ]
%       [ 0     0          1       --------------  --------   - -------------- ]
%       [                           3/2  2    3/2     2          3/2  2    3/2 ]
%       [                          2    a  - 2     2 a  - 2     2    a  - 2    ]
%
&\rightarrow
\begin{tabular}{|>{$}c<{$}>{$}c<{$}>{$}c<{$}|>{$}c<{$}>{$}c<{$}>{$}c<{$}|}
\hline
1&\frac{a}{\sqrt{2}}&                           0&
	0&                       0&\frac{1}{\sqrt{2}}\\
0&                 1&    -\frac{a\sqrt{2}}{a^2-2}&
	0& -\frac{\sqrt{2}}{a^2-2}&     \frac{a}{a^2}\\
0&                 0&                           1&
	\frac{a^2-2}{2\sqrt{2}(a^2-1)}&\frac{a}{2(a^2-1)}&-\frac{a^2}{2\sqrt{2}(a^2-1)}\\
\hline
\end{tabular}
\\
%        [       a                                           1         ]
%        [ 1  -------  0        0             0           -------      ]
%        [    sqrt(2)                                     sqrt(2)      ]
%        [                                                             ]
%        [                      a           sqrt(2)          a         ]
%        [ 0     1     0     --------     - --------      --------     ]
%(%o13)  [                      2              2             2         ]
%        [                   2 a  - 2       2 a  - 2      2 a  - 2     ]
%        [                                                             ]
%        [                     2                               2       ]
%        [                    a  - 2          a               a        ]
%        [ 0     0     1  --------------   --------   - -------------- ]
%        [                 3/2  2    3/2      2          3/2  2    3/2 ]
%        [                2    a  - 2      2 a  - 2     2    a  - 2    ]
%
&\rightarrow
\begin{tabular}{|>{$}c<{$}>{$}c<{$}>{$}c<{$}|>{$}c<{$}>{$}c<{$}>{$}c<{$}|}
\hline
1&\frac{a}{\sqrt{2}}&                           0&
	0&                       0&\frac{1}{\sqrt{2}}\\
0&                 1&                           0&
	\frac{a}{2(a^2-1)}&-\frac{\sqrt{2}}{2(a^2-1)}&\frac{a}{2(a^2-1)}\\
0&                 0&                           1&
	\frac{a^2-2}{2\sqrt{2}(a^2-1)}&\frac{a}{2(a^2-1)}&-\frac{a^2}{2\sqrt{2}(a^2-1)}\\
\hline
\end{tabular}
\\
%         [                   2                             2    3/2 ]
%         [                  a             a       sqrt(2) a  - 2    ]
%         [ 1  0  0  - --------------   --------   ----------------- ]
%         [             3/2  2    3/2      2              2          ]
%         [            2    a  - 2      2 a  - 2       4 a  - 4      ]
%         [                                                          ]
%         [                 a            sqrt(2)          a          ]
%(%o14)   [ 0  1  0      --------      - --------      --------      ]
%         [                 2               2             2          ]
%         [              2 a  - 2        2 a  - 2      2 a  - 2      ]
%         [                                                          ]
%         [                2                                2        ]
%         [               a  - 2           a               a         ]
%         [ 0  0  1   --------------    --------   - --------------  ]
%         [            3/2  2    3/2       2          3/2  2    3/2  ]
%         [           2    a  - 2       2 a  - 2     2    a  - 2     ]
%
&\rightarrow
\begin{tabular}{|>{$}c<{$}>{$}c<{$}>{$}c<{$}|>{$}c<{$}>{$}c<{$}>{$}c<{$}|}
\hline
1&0&0&
	-\frac{a^2}{2\sqrt{2}(a^2-2)}&\frac{a}{2(a^2-2)}&\frac{a^2-2}{2\sqrt{2}(a^2-1)}\\
0&1&0&
	\frac{a}{2(a^2-1)}&-\frac{\sqrt{2}}{2(a^2-1)}&\frac{a}{2(a^2-1)}\\
0&0&1&
	\frac{a^2-2}{2\sqrt{2}(a^2-1)}&\frac{a}{2(a^2-1)}&-\frac{a^2}{2\sqrt{2}(a^2-1)}\\
\hline
\end{tabular}
\end{align*}
Nimmt man im rechten Teil des Tableau den gemeinsamen Nenner $2\sqrt{2}(a^2-1)$
heraus, kann man wie vorhin die Inverse Matrix ablesen:
\[
A^{-1}
=
\frac1{2\sqrt{2}(a^2-1)}
\begin{pmatrix}
     -a^2&a\sqrt{2}&    a^2-2\\
a\sqrt{2}&       -2&a\sqrt{2}\\
    a^2-2&a\sqrt{2}&     -a^2
\end{pmatrix}.
\qedhere
\]
\end{loesung}

\begin{diskussion}
Man kann die Determinante natürlich auch mit dem Gauss-Algorithmus
berechnen, die Pivots sind
\[
a,\quad a\quad \text{und}\quad 2\sqrt{2}\frac{1-a^2}{a^2},
\]
man erhält als Produkt $\det(A)=2\sqrt(2)(1-a^2)$, das verkehrte Vorzeichen
rührt daher, dass man für die Durchführung des Gauss-Algorithmus zwei
Zeilen vertauschen musste, was das Vorzeichen der Determinante umkehrt.
\end{diskussion}

\begin{bewertung}
Determinantenkriterum ({\bf D}) 1 Punkt,
Berechnung der Determinante zum Beispiel mit der Sarrus-Formel ({\bf S})
1 Punkt,
Werte von $a$, für die die Matrix singulär wird ({\bf A}) 1 Punkt,
Minor-Formel für die Inverse ({\bf M}) 1 Punkt,
Berechnung der Inversen ({\bf I}) 2 Punkte.
\end{bewertung}

