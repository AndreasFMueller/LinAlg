Die $n\times n$-Matrix $A_n$ hat die Form
\[
A_n=\begin{pmatrix}
     1&     0&     0&     0&\dots &     0\\
     1&     2&     0&     0&\dots &     0\\
     1&     2&     3&     0&\dots &     0\\
     1&     2&     3&     4&\dots &     0\\
\vdots&\vdots&\vdots&\vdots&\ddots&\vdots\\
     1&     2&     3&     4&\dots &     n
\end{pmatrix}.
\]
Die Matrix $A_nA_n^t$ hat die Form
\[
A_nA_n^t
=
\begin{pmatrix}
     1&     1&     1&     1& \dots&      1&      1\\
     1&     5&     5&     5& \dots&      5&      5\\
     1&     5&    14&    14& \dots&     14&     14\\
     1&     5&    14&    30& \dots&     30&     30\\
\vdots&\vdots&\vdots&\vdots&\ddots& \vdots& \vdots\\
     1&     5&    14&    30& \dots&q_{n-1}&q_{n-1}\\
     1&     5&    14&    30& \dots&q_{n-1}&    q_n
\end{pmatrix},
\]
darin ist $q_k$ die Summe der ersten $k$ Quadratzahlen:
\[
q_k=\sum_{i=1}^ki^2=\frac{k(k+1)(2k+1)}6.
\qquad
\text{(Wikipedia)}
\]
Berechnen Sie die Determinante von $A_nA^t_n$.

\begin{loesung}
Da $A_n$ eine Dreiecksmatrix ist, ist die Determinante das Produkt der
Diagonalelement, also $\det(A_n)=n!$. 
Die Determinante bleibt beim Transponieren gleich, also folgt aus der
Produktformel
\[
\det(A_nA_n^t)=\det(A_n)\det(A_n^t)=\det(A_n)^2=(n!)^2.
\qedhere
\]
\end{loesung}


