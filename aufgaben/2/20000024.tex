Schreibt man die Zyklen in einem Netzwerk mit $k$ Kanten und $v$ Ecken
als Spalten in eine Matrix Z,
dann lassen sich die Kirchhoff-Gleichungen direkt formulieren.
$Z$ ist natürlich eine Matrix mit $k$ Zeilen, die Matrix $\partial$
des Netzwerks ist eine $v\times k$-Matrix.
Ist $R$ die $k\times k$-Matrix, die auf der Diagonalen die Widerstände der
einzelnen Kanten enthält, dann sind die Maschengleichungen für den
$k$-Vektor der Ströme 
\[
Z^tRI=Z^te,
\]
darin sind $e$ die Spannungsquellen.
Die Knotengleichungen werden zu $\partial I=0$, wobei zu beachten ist,
dass $\partial$ den Rang $k-1$ hat, eine der Gleichungen in $\partial I=0$ 
ist also redundant.

In einer früheren Aufgaben wurden die Zyklen in einem Oktaeder berechnet.
Wir nehmen an, dass alle Kanten des Oktaeders den gleichen Widerstand von
$R_0=1\Omega$ haben. Bestimmen Sie den Widerstand zwischen zwei benachbarten
Punkten des Oktaeders.

\begin{hinweis}
Gehen Sie dazu wie folgt vor.
Zunächst wird die Schaltung modifiziert: zwischen Punkte 1 und 2 wird eine
Spannungsquelle mit Innenwiderstand $R_0$ angeschlossen,
die eine Spannung von einem Volt liefert.
Welcher Strom fliesst aus durch die Spannungsquelle? Welchen Widerstand
hat ein Oktaeder zwischen zwei benachbarten Ecken?
\end{hinweis}

\begin{loesung}
\setcounter{MaxMatrixCols}{20}
Die $\partial$-Matrix des erweiterten Oktaeders hat eine Spalte mehr
für die neue Kante $13$:
\[
\partial = \begin{pmatrix}
-1&-1&-1&-1& 0& 0& 0& 0& 0& 0& 0& 0&-1\\
 1& 0& 0& 0&-1&-1&-1& 0& 0& 0& 0& 0& 1\\
 0& 1& 0& 0& 1& 0& 0&-1&-1& 0& 0& 0& 0\\
 0& 0& 1& 0& 0& 0& 0& 1& 0&-1&-1& 0& 0\\
 0& 0& 0& 1& 0& 1& 0& 0& 0& 1& 0&-1& 0\\
 0& 0& 0& 0& 0& 0& 1& 0& 1& 0& 1& 1& 0
\end{pmatrix}
\]
%
%[
%-1,-1,-1,-1, 0, 0, 0, 0, 0, 0, 0, 0,-1;
% 1, 0, 0, 0,-1,-1,-1, 0, 0, 0, 0, 0, 1;
% 0, 1, 0, 0, 1, 0, 0,-1,-1, 0, 0, 0, 0;
% 0, 0, 1, 0, 0, 1, 0, 0, 0,-1,-1, 0, 0;
% 0, 0, 0, 1, 0, 0, 0, 1, 0, 1, 0,-1, 0;
% 0, 0, 0, 0, 0, 0, 1, 0, 1, 0, 1, 1, 0
%]
%
Der Gauss-Algorithmus liefert das Schlusstableau
\begin{center}
\begin{tabular}{|>{$}c<{$}>{$}c<{$}>{$}c<{$}>{$}c<{$}>{$}c<{$}>{$}c<{$}>{$}c<{$}>{$}c<{$}>{$}c<{$}>{$}c<{$}>{$}c<{$}>{$}c<{$}>{$}c<{$}|}
\hline
x_1&x_2&x_3&x_4&x_5&x_6&x_7&x_8&x_9&x_{10}&x_{11}&x_{12}&x_{13}\\
\hline
   1&  0&  0&  0& -1& -1&  0&  0&  1&  0&  1&  1&  1\\
   0&  1&  0&  0&  1&  0&  0& -1& -1&  0&  0&  0&  0\\
   0&  0&  1&  0&  0&  0&  0&  1&  0& -1& -1&  0&  0\\
   0&  0&  0&  1&  0&  1&  0&  0&  0&  1&  0& -1&  0\\
   0&  0&  0&  0&  0&  0&  1&  0&  1&  0&  1&  1&  0\\
\hline
    &   &   &   &  \color{green}*&  \color{green}*&   &  \color{green}*&  \color{green}*&  \color{green}*&  \color{green}*&  \color{green}*&\color{green}*\\
\hline
\end{tabular}
\end{center}
Die zusätzliche frei wählbare Variable für
die neue Kante $x_{13}$ liefert auch einen
neuen Zyklus $z_8$, der durch die Wahl $x_{13}=1$ und alle anderen
frei wählbaren Variablen $=0$ festgelegt ist. Er verläuft durch die
neue Kante und die erste Kante, zu der die neue Kante parallel
verläuft:
\[
z_8=\begin{pmatrix}
             -1\\
              0\\
              0\\
              0\\
\color{green} 0\\
\color{green} 0\\
              0\\
\color{green} 0\\
\color{green} 0\\
\color{green} 0\\
\color{green} 0\\
\color{green} 0\\
\color{green} 1
\end{pmatrix}
\]
Damit kann man jetzt die $Z$-Matrix aufstellen:
\[
Z^t=\begin{pmatrix}
-1& 1& 0& 0& 1& 0& 0& 0& 0& 0& 0& 0& 0\\
-1& 0& 0& 1& 0& 1& 0& 0& 0& 0& 0& 0& 0\\
 0&-1& 1& 0& 0& 0& 0& 1& 0& 0& 0& 0& 0\\
 1&-1& 0& 0& 0& 0& 1& 0& 1& 0& 0& 0& 0\\
 0& 0&-1& 1& 0& 0& 0& 0& 0& 1& 0& 0& 0\\
 1& 0&-1& 0& 0& 0& 1& 0& 0& 0& 1& 0& 0\\
 1& 0& 0&-1& 0& 0& 1& 0& 0& 0& 0& 1& 0\\
 1& 0& 0& 0& 0& 0& 0& 0& 0& 0& 0& 0& 1
\end{pmatrix}
\]
Damit kann man jetzt die Maschengleichungen aufstellen. Da $R=R_0E$ ist,
wobei $R_0$ der für alle Kanten gleiche Widerstand einer Kante ist,
erhält man:
\[
Z^tR_0EI=R_0Z^tI=Z^te
\]
Die einzige Spannungsquelle ist die in der Kante $x_{13}$, der Vektor
ist $e$ ist also der Standardbasisvektor $e_{13}$. Dann ist aber
$Z^te=Z^te_{13}$ die letzte Spalte von $Z^t$, also der $8$-dimensionale
Standardbasisvektor $e_{8}$. Die Maschengleichungen sind damit:
\[
R_0Z^tI=e_8.
\]
Dazu kommen noch die Knotengleichungen $\partial I=0$, wobei wir die letzte
Gleichung weglassen können. Setzen wir zudem $R_0=1$, dann hat das
gesamte Gleichungssystem die Koeffizientenmatrix
\[
A=\begin{pmatrix}
   1& -1&  0&  0&  1&  0& -0&  0&  0&  0&  0&  0&  0\\
   1& -0& -0& -1&  0&  1& -0&  0&  0&  0&  0&  0&  0\\
   0&  1& -1&  0&  0&  0&  0&  1&  0&  0&  0&  0&  0\\
  -1&  1& -0& -0&  0&  0& -1&  0&  1&  0&  0&  0&  0\\
   0&  0&  1& -1&  0&  0&  0&  0&  0&  1&  0&  0&  0\\
  -1&  0&  1& -0&  0&  0& -1&  0&  0&  0&  1&  0&  0\\
  -1&  0&  0&  1&  0&  0& -1&  0&  0&  0&  0&  1&  0\\
  -1&  0&  0&  0&  0&  0&  0&  0&  0&  0&  0&  0&  1\\
  -1& -1& -1& -1&  0&  0&  0&  0&  0&  0&  0&  0& -1\\
   1&  0&  0&  0& -1& -1& -1&  0&  0&  0&  0&  0&  1\\
   0&  1&  0&  0&  1&  0&  0& -1& -1&  0&  0&  0&  0\\
   0&  0&  1&  0&  0&  0&  0&  1&  0& -1& -1&  0&  0\\
   0&  0&  0&  1&  0&  1&  0&  0&  0&  1&  0& -1&  0
\end{pmatrix}
\]
%
%[
% 1,-1, 0, 0, 1, 0, 0, 0, 0, 0, 0, 0, 0;
% 1, 0,-1, 0, 0, 1, 0, 0, 0, 0, 0, 0, 0;
% 0, 1, 0,-1, 0, 0, 0, 1, 0, 0, 0, 0, 0;
%-1, 1, 0, 0, 0, 0,-1, 0, 1, 0, 0, 0, 0;
% 0, 0, 1,-1, 0, 0, 0, 0, 0, 1, 0, 0, 0;
%-1, 0, 1, 0, 0, 0,-1, 0, 0, 0, 1, 0, 0;
%-1, 0, 0, 1, 0, 0,-1, 0, 0, 0, 0, 1, 0;
%-1, 0, 0, 0, 0, 0, 0, 0, 0, 0, 0, 0, 1;
%-1,-1,-1,-1, 0, 0, 0, 0, 0, 0, 0, 0,-1;
% 1, 0, 0, 0,-1,-1,-1, 0, 0, 0, 0, 0, 1;
% 0, 1, 0, 0, 1, 0, 0,-1,-1, 0, 0, 0, 0;
% 0, 0, 1, 0, 0, 1, 0, 0, 0,-1,-1, 0, 0;
% 0, 0, 0, 1, 0, 0, 0, 1, 0, 1, 0,-1, 0
%]
%
und die rechte Seite $e_8$. Die Lösung dieses Gleichungssystems liefert die
Ströme, wir interessieren uns aber nur für den Strom durch die
Kante $x_{13}$. Die numerische Lösung liefert
\[
I=\begin{pmatrix}
  -0.294118\\
  -0.147059\\
  -0.147059\\
  -0.117647\\
   0.147059\\
   0.147059\\
   0.117647\\
   0.029412\\
  -0.029412\\
   0.029412\\
  -0.029412\\
  -0.058824\\
   0.705882
\end{pmatrix}
\]
Daraus lässt sich ablesen, dass ein Strom von 705mA fliesst,
der Widerstand ist also 1.4167$\Omega$. Da der Innenwiderstand
der Spannungsquelle auch 1$\Omega$ war, folgt, dass der "Aquivalentwiderstand
eines Oktaeders zwischen zwei benachbarten Punkten 0.4167$\Omega$ ist.
\end{loesung}


