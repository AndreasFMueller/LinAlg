Berechnen Sie die Determinante der Matrix
\[
A=
\begin{pmatrix}
   1&  1&  2& -2\\
   9&  4& -6& -9\\
   4&  0&  5& -1\\
   2&  2&  5& -4
\end{pmatrix}
\]

\thema{Determinante}

\begin{loesung}
Wir verwenden eine Mischung des Gauss-Algorithmus und der Umformungsregeln
für Determinanten.
Wir beginnen mit dem Gauss-Algorithmus:
\begin{align*}
\begin{tabular}{|>{$}c<{$} >{$}c<{$} >{$}c<{$} >{$}c<{$}|}
\hline
   1&  1&  2& -2\\
   9&  4& -6& -9\\
   4&  0&  5& -1\\
   2&  2&  5& -4\\
\hline
\end{tabular}
&\rightarrow
\begin{tabular}{|>{$}c<{$} >{$}c<{$} >{$}c<{$} >{$}c<{$}|}
\hline
   1&  1&  2& -2\\
   0& -5&-24&  9\\
   0& -4& -3&  7\\
   0&  0&  1&  0\\
\hline
\end{tabular}
\end{align*}
An dieser Stelle ist es besser, die zweite Zeile von der ersten zu 
subtrahieren, damit dort ein Pivot $-1$ entsteht.
Ausserdem vertauschen wir die dritte und vierte Zeile, weil in der
vierten Zeile bereits ein Element steht, welches als besonders einfaches
Pivot-Element dienen kann.
Die Vertauschung führt ein negatives Vorzeichen ein, wir kompensieren
dies, indem wir das Vorzeichen in der zweiten Zeile kehren.
\begin{align*}
\begin{tabular}{|>{$}c<{$} >{$}c<{$} >{$}c<{$} >{$}c<{$}|}
\hline
   1&  1&  2& -2\\
   0& -1&-21&  2\\
   0& -4& -3&  7\\
   0&  0&  1&  0\\
\hline
\end{tabular}
&\rightarrow
\begin{tabular}{|>{$}c<{$} >{$}c<{$} >{$}c<{$} >{$}c<{$}|}
\hline
   1&  1&  2& -2\\
   0&  1& 21& -2\\
   0&  0&  1&  0\\
   0& -4& -3&  7\\
\hline
\end{tabular}
\rightarrow
\begin{tabular}{|>{$}c<{$} >{$}c<{$} >{$}c<{$} >{$}c<{$}|}
\hline
   1&  1&  2& -2\\
   0&  1& 21& -2\\
   0&  0&  1&  0\\
   0&  0& 81& -1\\
\hline
\end{tabular}
\rightarrow
\begin{tabular}{|>{$}c<{$} >{$}c<{$} >{$}c<{$} >{$}c<{$}|}
\hline
   1&  1&  2& -2\\
   0&  1& 21& -2\\
   0&  0&  1&  0\\
   0&  0&  0& -1\\
\hline
\end{tabular}
\end{align*}
Damit ist nur das allerletzte Pivot, nämlich $-1$, von $1$ verschieden
und wir können schliessen, dass $\det(A)=-1$.
\end{loesung}





