Berechnen Sie die Determinante
\[
\left|\begin{matrix}
3&2&1\\
2&3&2\\
1&2&3
\end{matrix}\right|
.
\]

\begin{loesung}
Man kann die Determinante schrittweise mit Zeilenoperationen umformen,
zum Beispiel im ersten Schritt durch hinzuaddieren der ersten Zeile
zur letzten, oder im dritten durch subtrahieren eines geeigneten
Vielfachen der dritten Zeile von den anderen:
\begin{align*}
\left|\,\begin{matrix}
3&2&1\\
2&3&2\\
1&2&3
\end{matrix}\,\right|
&=
\left|\,\begin{matrix}
3&2&1\\
2&3&2\\
4&4&4
\end{matrix}\,\right|
\\&=
4\cdot \left|\,\begin{matrix}
3&2&1\\
2&3&2\\
1&1&1
\end{matrix}\,\right|
\\&=
4\cdot \left|\,\begin{matrix}
0&-1&-2\\
0& 1& 0\\
1& 1& 1
\end{matrix}\,\right|
\\&=
4\cdot \left|\,\begin{matrix}
1& 1& 1\\
0&-1&-2\\
0& 1& 0
\end{matrix}\,\right|
\\&=
-4\cdot \left|\,\begin{matrix}
1& 1& 1\\
0&-2&-1\\
0& 0& 1
\end{matrix}\,\right|
\\&=
-4\cdot 1\cdot (-2)\cdot 1=8
\end{align*}
Im letzten Schritt hat man die Eigenschaft einer Dreiecksmatrix verwendet,
dass deren Determinante einfach das Produkt der Diagonalelemente ist.
Dies wurde in \"Ubung 4 gefunden.
\end{loesung}
