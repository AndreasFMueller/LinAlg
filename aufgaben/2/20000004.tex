Berechnen Sie die Determinante
\[
\left|\begin{matrix}
3&2&1\\
2&3&2\\
1&2&3
\end{matrix}\right|
.
\]

\thema{Determinante}
\thema{Sarrus-Formel}

\begin{loesung}
Man kann die Determinante schrittweise mit Zeilenoperationen umformen,
zum Beispiel im ersten Schritt durch hinzuaddieren der ersten Zeile
zur letzten, oder im dritten durch subtrahieren eines geeigneten
Vielfachen der dritten Zeile von den anderen:
\begin{align*}
\left|\,\begin{matrix}
3&2&1\\
2&3&2\\
1&2&3
\end{matrix}\,\right|
&=
\left|\,\begin{matrix}
3&2&1\\
2&3&2\\
4&4&4
\end{matrix}\,\right|
\\&=
4\cdot \left|\,\begin{matrix}
3&2&1\\
2&3&2\\
1&1&1
\end{matrix}\,\right|
\\&=
4\cdot \left|\,\begin{matrix}
0&-1&-2\\
0& 1& 0\\
1& 1& 1
\end{matrix}\,\right|
\\&=
4\cdot \left|\,\begin{matrix}
1& 1& 1\\
0&-1&-2\\
0& 1& 0
\end{matrix}\,\right|
\\&=
-4\cdot \left|\,\begin{matrix}
1& 1& 1\\
0&-2&-1\\
0& 0& 1
\end{matrix}\,\right|
\\&=
-4\cdot 1\cdot (-2)\cdot 1=8
\end{align*}
Im letzten Schritt hat man die Eigenschaft einer Dreiecksmatrix verwendet,
dass deren Determinante einfach das Produkt der Diagonalelemente ist.

Alternativ kann man auch die Sarrussche Formel verwenden:
\begin{align*}
\left|\,\begin{matrix}
3&2&1\\
2&3&2\\
1&2&3
\end{matrix}\,\right|
&=
3\cdot 3\cdot 3+2\cdot 2\cdot 1 + 2\cdot 2\cdot 1
-1\cdot 3\cdot 1-2\cdot 2\cdot 3-3\cdot 2\cdot 2
\\
&=27+4+4-3-12-12=8.
\end{align*}

Schliesslich ist es auch möglich, die Determinante mit dem Gauss-Algorithmus
zu bestimmen:
\begin{align*}
\renewcommand{\arraystretch}{1.3}%
\begin{tabular}{|>{$}c<{$}>{$}c<{$}>{$}c<{$}|}
\hline
3
\color{red}\put(-3,4){\circle{12}}
 &2&1\\
2&3&2\\
1&2&3\\
\hline
\end{tabular}
&
\rightarrow
\renewcommand{\arraystretch}{1.3}%
\begin{tabular}{|>{$}c<{$}>{$}c<{$}>{$}c<{$}|}
\hline
1&\frac{2}{3}&\frac{1}{3}\\
0&\frac{5}{3}
\color{red}\put(-3,3){\circle{15}}
             &\frac{4}{3}\\
0&\frac{4}{3}&\frac{8}{3}\\
\hline
\end{tabular}
\rightarrow
\renewcommand{\arraystretch}{1.3}%
\begin{tabular}{|>{$}c<{$}>{$}c<{$}>{$}c<{$}|}
\hline
1&\frac{2}{3}&\frac{1}{3}\\
0&          1&\frac{4}{5}\\
0&          0&\frac{8}{5}
\color{red}\put(-3,3){\circle{15}}
\\
\hline
\end{tabular}
\end{align*}
Auch wenn der Gauss-Algorithmus damit noch nicht zu Ende geführt ist, sind damit
die Pivot-Elemente bekannt, und man kann die Determinante als deren Produkt
berechnen:
\[
\left|\,\begin{matrix}
3&2&1\\
2&3&2\\
1&2&3
\end{matrix}\,\right|
=3\cdot\frac{5}{3}\cdot\frac{8}{5}=8.
\qedhere
\]
\end{loesung}
