Bestimmen Sie den Wert der Determinanten der Matrix
\[
A=\begin{pmatrix}
   4&  0&   8&   2\\
   1&  0&   2&   1\\
   a&  b&   c&   d\\
   3&  0&   7&   5\\
\end{pmatrix}.
\]
Für welche Werte von $a,b,c$ und $d$ ist die Matrix singulär?

\begin{loesung}
Wir entwickeln nach der zweiten Spalte und erhalten
\begin{align*}
\det A
&=
-b\left|\begin{matrix}
   4&  8&  2\\
   1&  2&  1\\
   3&  7&  5
\end{matrix}\right|
\\
&=
-b\bigl(
4\cdot 2\cdot 5
+
8\cdot 1\cdot 3
+
2\cdot 1\cdot 7
-
3\cdot 2\cdot 2
-
7\cdot 1\cdot 4
-
5\cdot 1\cdot 8
\bigr)
\\
&=
-b\bigl(
40+24+14
-12-28-40
\bigr)
=
2b.
\end{align*}
Die Matrix $A$ ist genau dann singulär, wenn $\det(A)=0$ ist, was genau 
für $b=0$ eintritt.
Die Werte der anderen Variablen haben keinen Einfluss darauf, ob $A$
singular ist.
\end{loesung}

\begin{bewertung}
Entwicklungssatz ({\bf E}) 1 Punkt,
Vorzeichenregel ({\bf V}) 1 Punkt,
Sarrus-Formel ({\bf S}) 1 Punkt,
Berechnung der Determinanten ({\bf R}) 2 Punkt,
Singularität der Matrix $A$ ({\bf Z}) 1 Punkt.
\end{bewertung}


