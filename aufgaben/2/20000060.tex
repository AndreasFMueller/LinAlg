Berechnen Sie den Wert der Determinanten der Matrix
\[
A
=
\begin{pmatrix*}[r]
  10 &    9 &    1 &    3  \\
   0 &    2 &   -8 &   -3  \\
   0 &   -1 &   -9 &    7  \\
   1 &   -3 &    0 &    5 
\end{pmatrix*}.
\]



\begin{loesung}
Die Determinante ist am einfachsten mit dem Entwicklungssatz zu berechnen.
Wir entwickeln nach der ersten  Spalte und erhalten
\begin{align*}
\det(A)
&=
10\cdot
\left|\,\begin{matrix*}[r]
   2 &   -8 &   -3  \\
  -1 &   -9 &    7  \\
  -3 &    0 &    5 
\end{matrix*}\,\right|
-1\cdot
\left|\,\begin{matrix*}[r]
   9 &    1 &    3  \\
   2 &   -8 &   -3  \\
  -1 &   -9 &    7 
\end{matrix*}\,\right|.
\intertext{Die $3\times 3$-Determinanten können mit der Sarrus-Regel
ausgewertet werden:}
&=
10\cdot\bigl(
-90+168+0+81+0-40
\bigr)
-1\cdot\bigl(
-504+3-54-24-243-14
\bigr)
\\
&=
10\cdot 119
-1\cdot (-836)
=
2026.
\qedhere
\end{align*}
\end{loesung}

\begin{bewertung}
Entwicklungssatz ({\bf E}) 1 Punkt,
Vorzeichenregel ({\bf V}) 1 Punkt,
Unterdeterminanten ({\bf U}) 1 Punkt,
Sarrus ({\bf S}) 1 Punkt,
Berechnung der Unterdeterminanten ({\bf B}) 1 Punkt,
Wert der Determinanten ({\bf D}) 1 Punkt.
\end{bewertung}
