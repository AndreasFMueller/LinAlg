Berechnen Sie die Determinante der Matrix $A$ und der inversen Matrix $A^{-1}$
\[
A=
\begin{pmatrix}
 0 & 2 & 4 & 0\\
 6 & 0 & 0 & 2\\
 3 & -4 & 2 & -4\\
 0 & 0 & -3 & 1
\end{pmatrix}.
\]
Ist die Matrix $A$ regulär?

\thema{Determinante}
\thema{Entwicklungssatz}
\thema{Produktsatz}

\begin{loesung}
Wir berechnen die Determinante mit Hilfe des Entwicklungssatzes 
und entwickeln nach der vierten Zeile
\begin{align*}
\det(A)
&=
-
(-3)\cdot\left|
\begin{matrix}
 0 & 2 & 0\\
 6 & 0 & 2\\
 3 & -4 & -4\\
\end{matrix}
\right|
+
1\cdot\left|
\begin{matrix}
 0 & 2 & 4\\
 6 & 0 & 0 \\
 3 & -4 & 2\\
\end{matrix}
\right|
\end{align*}
Die beiden $3\times 3$-Determinanten können wiederum mit dem Entwicklungssatz
berechnet werden. Die erste Determinante wird nach der ersten Zeile entwickelt 
und die zweite Determinante nach der zweiten Zeile:
\begin{align*}
\det(A)
&=
3\cdot (-2)\cdot \left|
\begin{matrix}
 6 & 2\\
 3 & -4\\
\end{matrix}
\right|
+
1\cdot (-6)\cdot\left|
\begin{matrix}
2 & 4\\
-4 & 2\\
\end{matrix}
\right|\\
&= -6\cdot (6\cdot (-4) - 2 \cdot 3) - 6 \cdot (2\cdot 2 - 4 \cdot (-4))\\
&= -6\cdot (-30) - 6 \cdot 20\\
&= 180 - 120 = 60.
\end{align*}
Die Matrix ist also regulär.

Aus $\det(A)$ kann nun auch die Determinante der inversen Matrix $A^{-1}$ berechnet werden:
\[
  \det(A^{-1}) = \dfrac{1}{\det(A)} = \dfrac{1}{60}\approx 0.016667.
\]
\end{loesung}

\begin{bewertung}
Entwicklungssatz: Zeile oder Spalte ({\bf Z}) 1 Punkt,
Vorzeichen ({\bf V}) 1 Punkt,
Unterdeterminanten ({\bf U}) 1 Punkt,
Determinantenwert ({\bf D}) 1 Punkt,
Schlussfolgerung regulär ({\bf R}) 1 Punkt,
Determinante der Inversen ({\bf I}) 1 Punkt.
\end{bewertung}

