Berechnen Sie die Determinante der $n\times n$-Matrix
\[
A_n
=
\begin{pmatrix}
     1&     1&     1&     1& \dots&     1&     1\\
     1&     2&     2&     2& \dots&     2&     2\\
     1&     2&     3&     3& \dots&     3&     3\\
     1&     2&     3&     4& \dots&     4&     4\\
\vdots&\vdots&\vdots&\vdots&\ddots&\vdots&\vdots\\
     1&     2&     3&     4& \dots&   n-1&   n-1\\
     1&     2&     3&     4& \dots&   n-1&     n
\end{pmatrix}
\]
\begin{hinweis}
Die Matrix $A_n$ lässt sich als Produkt $A_n=B_nB_n^t$ schreiben, wobei 
$B_n$ eine (untere) Dreiecksmatrix ist.
\end{hinweis}

\themaS{Determinante}
\themaS{Produktsatz}

\begin{loesung}
Durch Probieren findet man, dass $B_n$ aus auf und unter der Diagonalen aus
lauter Einsen besteht, also
\[
B_n
=
\begin{pmatrix}
     1&     0&     0&\dots &     0\\
     1&     1&     0&\dots &     0\\
     1&     1&     1&\dots &     0\\
\vdots&\vdots&\vdots&\ddots&\vdots\\
     1&     1&     1&\dots &     1
\end{pmatrix}
\]
Tatsächlich kann man das Produkt
\begin{align*}
P=B_nB_n^t
&=
\begin{pmatrix}
     1&     0&     0&\dots &     0\\
     1&     1&     0&\dots &     0\\
     1&     1&     1&\dots &     0\\
\vdots&\vdots&\vdots&\ddots&\vdots\\
     1&     1&     1&\dots &     1
\end{pmatrix}
\begin{pmatrix}
     1&     1&     1&\dots &     1\\
     0&     1&     1&\dots &     1\\
     0&     0&     1&\dots &     1\\
\vdots&\vdots&\vdots&\ddots&\vdots\\
     0&     0&     0&\dots &     1
\end{pmatrix}
\end{align*}
direkt berechnen.
Die Zeile $i$ der ersten Matrix enthält genau $i$ Einsen, die Spalte $j$ 
der zweiten Matrix genau $j$.
Das Produkt hat also so viele Summanden $1$, wie die kleinere der beiden
Zahlen angibt.
Das Element $p_{ij}$ der Matrix ist daher
\[
p_{ij}=\min(i,j),
\]
das sind aber genau die Matrixelemente der Matrix $A_n$.

Da man die Matrix $A_n$ als Produkt sehr einfacher Matrizen schreiben kann,
kann man jetzt die Produktformel verwenden, um die Determinante zu berechnen:
\[
\det(A_n)=\det(B_nB_n^t)=\det(B_n)\det(B_n^t)=\det(B_n)^2=1.
\]
Darin wurde verwendet, dass die Determinante von $B_n$ als Determinante
einer Dreiecksmatrix das Produkt der Diagonalelemente ist, also
$\det(B_n)=1$. 
\end{loesung}

