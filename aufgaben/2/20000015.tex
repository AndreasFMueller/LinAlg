Berechnen Sie die folgende Determinante
\[
\left|\,\begin{matrix}
0&1&3&9\\
4&1&4&16\\
0&1&2&4\\
0&1&1&1
\end{matrix}\,\right|.
\]

\thema{Entwicklungssatz}
\thema{Determinante}
\thema{Sarrus-Formel}

\begin{loesung}
Nach dem Entwicklungssatz ist
\[
\left|\,\begin{matrix}
0&1&3&9\\
4&1&4&16\\
0&1&2&4\\
0&1&1&1
\end{matrix}\,\right|
=-4\cdot\left|\,\begin{matrix}
1&3&9\\
1&2&4\\
1&1&1
\end{matrix}\,\right|
=
4\cdot\left|\,\begin{matrix}
1&1&1\\
1&2&4\\
1&3&9
\end{matrix}\,\right|
\]
Die letzte Determinante ist eine Vandermondesche Determinante,
wie sie auch in einer "Ubung berechnet wurde. Sie kann im vorliegenden
Fall aber (ausser mit der Sarrussschen Regel) auch mit Gauss
berechne werden:
\[
\begin{tabular}{|ccc|}
\hline
1&1&1\\
1&2&4\\
1&3&9\\
\hline
\end{tabular}
\rightarrow
\begin{tabular}{|ccc|}
\hline
1&1&1\\
0&1&3\\
0&2&8\\
\hline
\end{tabular}
\rightarrow
\begin{tabular}{|ccc|}
\hline
1&1&1\\
0&1&3\\
0&0&2\\
\hline
\end{tabular}
\]
Daraus liest man die Pivot-Elemente $1$, $1$ und $2$ ab, die
Vandermondesche Determinante muss also den Wert $2$ haben.
Somit gilt
\[
\left|\,\begin{matrix}
0&1&3&9\\
4&1&4&16\\
0&1&2&4\\
0&1&1&1
\end{matrix}\,\right|
= 8
\qedhere
\]
\end{loesung}

