Berechnen Sie die Determinante der Matrix
\[
A=\begin{pmatrix}
    4 &  0 &  7 &  0 \\
    5 &  8 &  5 &  1 \\
    4 &  7 &  4 &  1 \\
    8 &  8 & 10 &  1 
\end{pmatrix}.
\]

\thema{Determinante}
\thema{Entwicklungssatz}

\begin{loesung}
Subtrahiert man das achtfache der letzten Spalte in der zweiten Spalte,
entsteht eine Determinante, die sich mit dem Entwicklungssatz leicht
ausrechnen lässt:
\begin{align*}
\det A
&=
\left|\begin{matrix}
    4 &  0 &  7 &  0 \\
    5 &  0 &  5 &  1 \\
    4 & -1 &  4 &  1 \\
    8 &  0 & 10 &  1 
\end{matrix}\right|
=
-(-1)\left|\begin{matrix}
    4 &  7 &  0 \\
    5 &  5 &  1 \\
    8 & 10 &  1 
\end{matrix}\right|
\\
&=
-\left|\begin{matrix}
    4 &  7\\
    8 & 10
\end{matrix}\right|
+\left|\begin{matrix}
    4 &  7\\
    5 &  5
\end{matrix}\right|
=
-(4\cdot 10-7\cdot 8)+(4\cdot 5-5\cdot 7)
=
16-15=1.
\end{align*}
Dabei wurde die $3\times 3$-Determinante durch Entwicklung nach der letzten
Spalte berechnet.
\end{loesung}


