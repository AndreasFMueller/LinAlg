Betrachten Sie die Matrizen
\[
A=\begin{pmatrix}
13& 9&-3&15\\
 0&11& 7& 5\\
 0& 0&17&13\\
 0& 0& 0&19
\end{pmatrix}
\qquad\text{und}\qquad
B
=
\begin{pmatrix}
   0&  1&  0&  0\\
  -1&  0& -2&  0\\
   0&  2&  0&  1\\
   0&  0&  1&  0
\end{pmatrix}.
\]
\begin{teilaufgaben}
\item Berechnen Sie $\det(A)$.
\item Berechnen Sie $C=BA\transpose{B}$.
\item Berechnen Sie $\det(C)$.
\end{teilaufgaben}

\thema{Produktsatz}
\thema{Matrizenprodukt}

\begin{loesung}
\begin{teilaufgaben}
\item $A$ ist eine Dreiecksmatrix, die Determinante ist das Produkt der
Diagonalelemente:
\[
\det(A) = 13\cdot 11 \cdot 17\cdot 19=46189.
\]
\item Die Matrix $C$ ist
\[
C
=
BA\transpose{B}
=
\begin{pmatrix}
  0& 11&  7&  5\\
-13& -9&-31&-41\\
  0& 22& 14& 29\\
  0&  0& 17& 13
\end{pmatrix}
\begin{pmatrix}
   0& -1&  0&  0\\
   1&  0&  2&  0\\
   0& -2&  0&  1\\
   0&  0&  1&  0
\end{pmatrix}
=
\begin{pmatrix}
   11& -14&  27&   7\\
   -9&  75& -59& -31\\
   22& -28&  73&  14\\
    0& -34&  13&  17
\end{pmatrix}.
\]
\item Aus der Produktformel für Determinanten folgt
\[
\det(C)
=
\det(BA\transpose{B})
=
\det(B)\det(A)\det(\transpose{B})
=
\det(B)\det(A)\det(B)
=
\det(A)\det(B)^2.
\]
Im letzten Schritt wurde $\det(B)=\det(\transpose{B})$ verwendet.
Die Determinante $\det(A)$ auf der rechten Seite wurde in Teilaufgabe a)
bereits berechnet.
Es muss also nur noch $\det(B)$ berechnet werden.
Dies kann mit dem Entwicklungssatz geschehen
\[
\det(B)
=
\left|
\begin{matrix}
 0&  1&  0&  0\\
-1&  0& -2&  0\\
 0&  2&  0&  1\\
 0&  0&  1&  0
\end{matrix}\right|
=
-1\cdot
\left|\begin{matrix}
-1& -2&  0\\
 0&  0&  1\\
 0&  1&  0
\end{matrix}\right|
=
(-1)\cdot(-1)\cdot
\left|\begin{matrix}
0&  1\\
1&  0
\end{matrix}\right|
=(-1)\cdot(-1)\cdot(-1)=-1.
\]
Damit folgt
\[
\det(BA\transpose{B})
=
\det(A)\det(B)^2
=
46189\cdot(-1)^2=46189.\qedhere
\]
\end{teilaufgaben}
\end{loesung}

