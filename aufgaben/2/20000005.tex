Berechnen Sie die folgende Determinante
\[
\left|\,\begin{matrix}
4&6& 0&3\\
7&1& 0&4\\
0&0&11&0\\
8&0& 0&0
\end{matrix}\,\right|.
\]

%   4   6   0   3
%   7   1   0   4
%   0   0  11   0
%   8   0   0   0


\ifthenelse{\boolean{loesungen}}{
\begin{loesung}
Die Determinante kann mit dem Entwicklungssatz bestimmt werden, wobei
als Entwicklung nach der dritten Spalte am zweckm"assigsten ist:
\begin{align*}
\left|\,\begin{matrix}
4&6& 0&3\\
7&1& 0&4\\
0&0&11&0\\
8&0& 0&0
\end{matrix}\,\right|
&=
11\cdot
\left|\,\begin{matrix}
4&6&3\\
7&1&4\\
8&0&0
\end{matrix}\,\right|
\\
&=
11\cdot
8\cdot
\left|\,\begin{matrix}
6&3\\
1&4
\end{matrix}\,\right|
\\
&=
11\cdot
8\cdot
(6\cdot 4-1\cdot 3)=11\cdot 8\cdot 21=1848.
\end{align*}


Alternative k"onnte man die Determinante auch unter Anwendung der
Rechenregeln f\"ur Determinanten berechnen. Dazu vertauschen wir
zun\"achst Zeilen und Spalten, bis wir ann"ahernd Dreiecksform erreichen,
um dann die Determinante in wenigen Gauss-Schritten zu berechnen:
\begin{align*}
\left|\,\begin{matrix}
4&6& 0&3\\
7&1& 0&4\\
0&0&11&0\\
8&0& 0&0
\end{matrix}\,\right|
&=
\left|\,\begin{matrix}
0&0&11&0\\
4&6& 0&3\\
7&1& 0&4\\
8&0& 0&0
\end{matrix}\,\right|
\\
&=
\left|\,\begin{matrix}
11&0&0&0\\
 0&6&3&4\\
 0&1&4&7\\
 0&0&0&8
\end{matrix}\,\right|\\
&=
-\left|\,\begin{matrix}
11&0&0&0\\
 0&1&4&7\\
 0&6&3&4\\
 0&0&0&8
\end{matrix}\,\right|\\
&=
-\left|\,\begin{matrix}
11&0&  0&  0\\
 0&1&  4&  7\\
 0&0&-21&-38\\
 0&0&  0&  8
\end{matrix}\,\right|\\
&-11\cdot 1\cdot (-21)\cdot 8=11\cdot 8\cdot 21=1848
\end{align*}
\end{loesung}
}{
}

