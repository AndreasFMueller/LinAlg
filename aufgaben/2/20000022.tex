Schreibt man die Zyklen in einem Netzwerk mit $k$ Kanten und $v$ Ecken
als Spalten in eine Matrix Z,
dann lassen sich die Kirchhoff-Gleichungen direkt formulieren.
$Z$ ist natürlich eine Matrix mit $k$ Zeilen, die Matrix $\partial$
des Netzwerks ist eine $v\times k$-Matrix.
Ist $R$ die $k\times k$-Matrix, die auf der Diagonalen die Widerstände der
einzelnen Kanten enthält, dann sind die Maschengleichungen für den
$k$-Vektor der Ströme 
\[
\transpose{Z}RJ=\transpose{Z}e,
\]
darin sind $e$ die Spannungsquellen.
Die Knotengleichungen werden zu $\partial J=0$, wobei zu beachten ist,
dass $\partial$ den Rang $k-1$ hat, eine der Gleichungen in $\partial J=0$ 
ist also redundant.

In einer früheren Aufgaben wurden die Zyklen in einem Tetraeder berechnet.
Wir nehmen an, dass alle Kanten des Tetraeder den gleichen Widerstand von
$R_0=1\Omega$ haben. Bestimmen Sie den Widerstand zwischen zwei benachbarten
Punkten des Tetraeders.

\thema{Gauss-Algorithmus}
\thema{Zyklen}

\begin{hinweis}
Gehen Sie dazu wie folgt vor.
Zunächst wird die Schaltung modifiziert: zwischen Punkte 1 und 2 wird eine
Spannungsquelle mit Innenwiderstand $R_0$ angeschlossen,
die eine Spannung von einem Volt liefert.
Welcher Strom fliesst aus durch die Spannungsquelle? Welchen Widerstand
hat ein Tetraeder zwischen zwei benachbarten Ecken?
\end{hinweis}

\begin{loesung}
Die $\partial$-Matrix des erweiterten Tetraeders hat eine Spalte mehr,
für die neue Kante $13$:
\[
\partial = \begin{pmatrix}
-1&-1&-1& 0& 0& 0&-1\\
 1& 0& 0&-1&-1& 0& 1\\
 0& 1& 0& 1& 0&-1& 0\\
 0& 0& 1& 0& 1& 1& 0
\end{pmatrix}
\]
%
%[
%-1,-1,-1, 0, 0, 0,-1;
% 1, 0, 0,-1,-1, 0, 1;
% 0, 1, 0, 1, 0,-1, 0;
% 0, 0, 1, 0, 1, 1, 0
%]
%
Der Gauss-Algorithmus liefert das Schlusstableau
\begin{center}
\begin{tabular}{|>{$}c<{$}>{$}c<{$}>{$}c<{$}>{$}c<{$}>{$}c<{$}>{$}c<{$}>{$}c<{$}|}
\hline
x_1&x_2&x_3&x_4&x_5&x_6&x_7\\
\hline
   1&  0&  0& -1& -1&  0&  1\\
   0&  1&  0&  1&  0& -1&  0\\
   0&  0&  1&  0&  1&  1&  0\\
%   0&  0&  0&  0&  0&  0&  0\\
\hline
    &   &   &  \color{green}*&  \color{green}*&   \color{green}*&  \color{green}*\\
\hline
\end{tabular}
\end{center}
Die zusätzliche frei wählbare Variable für
die neue Kante $x_7$ liefert auch einen
neuen Zyklus $z_4$, der durch die Wahl $x_7=1$ und alle anderen
frei wählbaren Variablen $=0$ festgelegt ist. Er verläuft durch die
neue Kante und die erste Kante, zu der die neue Kante parallel
verläuft:
\[
z_4=\begin{pmatrix}
             -1\\
              0\\
              0\\
\color{green} 0\\
\color{green} 0\\
\color{green} 0\\
\color{green} 1
\end{pmatrix}
\]
Damit kann man jetzt die $Z$-Matrix aufstellen:
\[
\transpose{Z}=\begin{pmatrix}
 1&-1& 0& 1& 0& 0& 0\\
 1& 0&-1& 0& 1& 0& 0\\
 0& 1&-1& 0& 0& 1& 0\\
-1& 0& 0& 0& 0& 0& 1
\end{pmatrix}
\]
Damit kann man jetzt die Maschengleichungen aufstellen. Da $R=R_0E$ ist,
wobei $R_0$ der für alle Kanten gleiche Widerstand einer Kante ist,
erhält man:
\[
\transpose{Z}R_0IJ=R_0\transpose{Z}J=\transpose{Z}e
\]
Die einzige Spannungsquelle ist die in der Kante $x_7$, der Vektor
ist $e$ ist also der Standardbasisvektor $e_7$. Dann ist aber
$\transpose{Z}e=\transpose{Z}e_7$ die letzte Spalte von $\transpose{Z}$,
also der $4$-dimensionale
Standardbasisvektor $e_4$. Die Maschengleichungen sind damit:
\[
R_0\transpose{Z}J=e_4.
\]
Dazu kommen noch die Knotengleichungen $\partial J=0$, wobei wir die letzte
Gleichung weglassen können. Setzen wir zudem $R_0=1$, dann hat das
gesamte Gleichungssystem die Koeffizientenmatrix
\[
A=\begin{pmatrix}
 1&-1& 0& 1& 0& 0& 0\\
 1& 0&-1& 0& 1& 0& 0\\
 0& 1&-1& 0& 0& 1& 0\\
-1& 0& 0& 0& 0& 0& 1\\
-1&-1&-1& 0& 0& 0&-1\\
 1& 0& 0&-1&-1& 0& 1\\
 0& 1& 0& 1& 0&-1& 0
% 1& 0& 0&-1&-1& 0& 1\\
% 0& 1& 0& 1& 0&-1& 0\\
% 0& 0& 1& 0& 1& 1& 0
\end{pmatrix}
\]
%
%[
% 1,-1, 0, 1, 0, 0, 0;
% 1, 0,-1, 0, 1, 0, 0;
% 0, 1,-1, 0, 0, 1, 0;
%-1, 0, 0, 0, 0, 0, 1;
% 1, 0, 0,-1,-1, 0, 1;
% 0, 1, 0, 1, 0,-1, 0;
% 0, 0, 1, 0, 1, 1, 0
%]
%
und die rechte Seite $e_4$. Die Lösung dieses Gleichungssystems liefert die
Ströme, wir interessieren uns aber nur für den Strom durch die
Kante $x_7$. Die numerische Lösung liefert
\[
J=\begin{pmatrix}
          -0.33333\\
          -0.16667\\
          -0.16667\\
\phantom{-}0.16667\\
\phantom{-}0.16667\\
\phantom{-}0.00000\\
\phantom{-}0.66667\\
\end{pmatrix}
\]
Daraus lässt sich ablesen, dass ein Strom von 666.666mA fliesst,
der Widerstand ist also 1.5$\Omega$. Da der Innenwiderstand
der Spannungsquelle auch 1$\Omega$ war, folgt, dass der "Aquivalentwiderstand
eines Oktaeders zwischen zwei benachbarten Punkten 0.5$\Omega$ ist.

Dieses Resultat kann man natürlich auch direkt enthalten. Da die beiden
nicht angeschlossenen Ecken des Tetraeders auf gleichem Potential liegen,
fliesst durch diese Kante kein Strom. Damit ist das Tetraeder äquivalent
zu einer Parallelschaltung eines $R_1=1\Omega$-Widerstands mit zwei
Serienschaltungen von $R_{2,3}=2\Omega$-Widerständen.
Deren Gesamtwiderstand $R$
ist
\[
\frac1R=\frac1{R_1}+\frac1{R_2}+\frac1{R_3}=\frac1{1\Omega}+2\frac1{2\Omega}=\frac2{1\Omega}
\quad
\Rightarrow
\quad
R=\frac12\Omega.
\qedhere
\]
\end{loesung}

