Berechnen Sie die Determinante der folgenden Matrizen mit Hilfe des
Gauss-Algorithmus
\begin{teilaufgaben}
\item
$
\displaystyle
A_1 = \begin{pmatrix}2&0\\3&5\end{pmatrix}
\qquad\text{und}\qquad
A_2 = \begin{pmatrix}2&3\\0&5\end{pmatrix}
$
\item
$\displaystyle
B=\begin{pmatrix}
3&21& 6\\
4&30&14\\
5&41&29
\end{pmatrix}$
\end{teilaufgaben}

\thema{Determinante}
\thema{Gauss-Algorithmus}

\begin{loesung}
\begin{teilaufgaben}
\item
Wir verwenden den Gauss-Algorithmus
\begin{align*}
\begin{tabular}{|>{$}c<{$}>{$}c<{$}|}
\hline
2 \color{red}\put(-3,4){\circle{12}}
&0\\
3\color{blue}\drawline(-8,-1)(-8,10)(1,10)(1,-1)&5\\
\hline
\end{tabular}
&\rightarrow
\begin{tabular}{|>{$}c<{$}>{$}c<{$}|}
\hline
1&0\\
0&5\color{red}\put(-3,4){\circle{12}}\\
\hline
\end{tabular}
\rightarrow
\begin{tabular}{|>{$}c<{$}>{$}c<{$}|}
\hline
1&0\\
0&1\\
\hline
\end{tabular}
\\
\begin{tabular}{|>{$}c<{$}>{$}c<{$}|}
\hline
2 \color{red}\put(-3,4){\circle{12}}
&3\\
0\color{blue}\drawline(-8,-1)(-8,10)(1,10)(1,-1)&5\\
\hline
\end{tabular}
&\rightarrow
\begin{tabular}{|>{$}c<{$}>{$}c<{$}|}
\hline
1&\frac{3}{2}\\
0&5\color{red}\put(-3,4){\circle{12}}\\
\hline
\end{tabular}
\rightarrow
\begin{tabular}{|>{$}c<{$}>{$}c<{$}|}
\hline
1&0\\
0&1\\
\hline
\end{tabular}
\end{align*}
Die Pivot-Elemente sind also in beiden Fällen $\color{red}2$
und $\color{red}5$, die Determinante ist $\det A_1=\det A_2=2\cdot 5=10$.
\item
Der Gauss-Algorithmus
\begin{align*}
\begin{tabular}{|>{$}c<{$}>{$}c<{$}>{$}c<{$}|}
\hline
3\color{red}\put(-3,4){\circle{12}}&21& 6\\
4&30&14\\
5
\begin{picture}(0,0)
\color{blue}\drawline(-8,-1)(-8,25)(1,25)(1,-1)
\end{picture}%
&41&29\\
\hline
\end{tabular}
&
\rightarrow
\begin{tabular}{|>{$}c<{$}>{$}c<{$}>{$}c<{$}|}
\hline
1& 7& 2\\
0& 2\color{red}\put(-3,4){\circle{12}}& 6\\
0& 6\color{blue}\drawline(-8,-1)(-8,10)(1,10)(1,-1)&19\\
\hline
\end{tabular}
\rightarrow
\begin{tabular}{|>{$}c<{$}>{$}c<{$}>{$}c<{$}|}
\hline
1& 7& 2\\
0& 1& 3\\
0& 0& 1\color{red}\put(-3,4){\circle{12}}\\
\hline
\end{tabular}
\end{align*}
liefert die Pivots $\color{red}3$, $\color{red}2$ und $\color{red}1$,
also ist die Determinante $\det B = 3\cdot 2 \cdot 1=6$.
\qedhere
\end{teilaufgaben}
\end{loesung}


