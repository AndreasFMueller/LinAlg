Bestimmen Sie die Determinante der Matrix
\[
A=\begin{pmatrix}
   8& -7&  1& -5\\
   7&  1& -2&  0\\
  -4&  9&  0&  0\\
   0&  8&  6& -2
\end{pmatrix}.
\]

\begin{loesung}
Wir verwenden den Entwicklungssatz und entwickeln die Determinante nach der
dritten Zeile:
\begin{align*}
\det(A)
&=
-4\cdot \left|\;\begin{matrix}
  -7&  1& -5\\
   1& -2&  0\\
   8&  6& -2
\end{matrix}\;\right|
-9\cdot \left|\;\begin{matrix}
   8&  1& -5\\
   7& -2&  0\\
   0&  6& -2
\end{matrix}\;\right|
\\
&=
-4\cdot(
\underbrace{
(-7)\cdot(-2)\cdot(-2) + 0 + (-5)\cdot1\cdot 6
-8\cdot(-2)\cdot(-5)-0-(-2)\cdot1\cdot(-7)}_{\displaystyle=-136}
)
-9\cdot(-164)
\\
&= -4\cdot(-136)-9\cdot(-164)
=
4\cdot 136+9\cdot 164 =544 + 1476 = 2020.
\end{align*}
Für die $3\times 3$-Determinanten haben die Sarrus-Formel verwendet, wobei
wir die zweite nicht im Detail ausgeführt haben.
\end{loesung}

\begin{bewertung}
Entwicklungsssatz ({\bf E}) 1 Punkt,
Wahl einer Zeile ({\bf Z}) 1 Punkt,
Rekursion ({\bf R}) 1 Punkt,
Sarrus-Formel ({\bf S}) 1 Punkt,
Wert der Determinante ({\bf D}) 2 Punkt.
\end{bewertung}


