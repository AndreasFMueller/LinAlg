Gegeben ist das Gleichungssystem
\[
\begin{linsys}{3}
 x&-&t y&=&t\\
tx&+&  y&=&t^2.
\end{linsys}
\]
\begin{teilaufgaben}
\item Für welche Werte von $t$ ist die Matrix dieses Gleichungssystems
regulär?
\item
Verwenden Sie die Kramersche Regel, um 
eine Formel für $x$ und $y$ in Abhängigkeit von $t$ anzugeben.
\end{teilaufgaben}

\themaS{Kramersche Regel}
\themaS{Determinante mit Parameter}

\begin{loesung}
\begin{teilaufgaben}
\item
Die Determinante der Koeffizientenmatrix gibt darüber Auskunft, ob die
Matrix regulär ist. Sie ist
\begin{align*}
\left|\,\begin{matrix}1&-t\\t&1\end{matrix}\,\right|
&=1\cdot 1-(-t)\cdot t=1+t^2.
\end{align*}
Da $1+t^2>0$ für alle $t\in\mathbb R$ ist die Matrix immer regulär und
damit das Gleichungssystem immer eindeutig lösbar.
\item
Wir verwenden die Kramersche Formel, dazu brauchen wir einerseits die bereits
berechnete Determinante der Koeffizientenmatrix des Gleichungssystems
und andererseits die Determinanten von Matrizen, die daraus durch Ersetzen
einer Spalte entstehen:
\begin{align*}
&\text{1.~Spalte ersetzt:}&
\left|\,\begin{matrix}t&-t\\t^2&1\end{matrix}\,\right|
&=t+t^3
\\
&\text{2.~Spalte ersetzt:}&
\left|\,\begin{matrix}1&t\\t&t^2\end{matrix}\,\right|
&=t^2-t^2=0
\end{align*}
Daraus kann man mit Hilfe der Kramerschen Formel die Lösung des
Gleichungssystems ableiten:
\begin{align*}
x&=\frac{t+t^3}{1+t^2}=t,&
y&=0.
\end{align*}
\qedhere
\end{teilaufgaben}
\end{loesung}
