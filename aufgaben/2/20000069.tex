Die Berechnung der Determinante mit dem Entwicklungssatz ist am effizientesten,
wenn eine Zeile oder Spalte mit möglichst vielen Nullen zur Entwicklung verwendet wird.
Welche Zeile oder Spalte verwenden Sie bei der Entwicklung der Determinante
der folgenden Matrizen?
\begin{teilaufgaben}
\item
\(\displaystyle
A
=
\begin{pmatrix*}[r]
    5 &   7 &   9 &   3 \\
    0 &  11 &  11 &   8 \\
    0 &   8 &   0 &   0 \\
   10 &   2 &   0 &  12
\end{pmatrix*}
\)
\item
\(\displaystyle
B
=
\begin{pmatrix*}[r]
   14 &            5 &   0 &   0 \\
    0 &            8 &  15 &  10 \\
   16 &            1 &   3 &   0 \\
    8 & \phantom{0}5 &  12 &   0
\end{pmatrix*}
\)
\end{teilaufgaben}

\begin{loesung}
\begin{teilaufgaben}
\item
Auf Zeile 3 von $A$ stehen ausser in Spalte 2 lauter Nullen.
Die Entwicklung nach der dritten Zeile reduziert sich daher auf einen Term.
\item
In der letzten Spalte stehen drei Nullen, die Entwicklung nach dieser
Spalte reduziert sich daher auf einen Term.
\qedhere
\end{teilaufgaben}
\end{loesung}
