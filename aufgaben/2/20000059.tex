Berechnen Sie den Wert der Determinanten der Matrix
\[
A
=
\begin{pmatrix*}[r]
   8 &    0 &   -4 &    5  \\
  -4 &    0 &    5 &   -4  \\
   0 &    7 &   -1 &   -7  \\
  -1 &    1 &   -9 &   -9 
\end{pmatrix*}
.
\]


\begin{loesung}
Die Determinante ist am einfachsten mit dem Entwicklungssatz zu berechnen.
Wir entwickeln nach der zweiten Spalte und erhalten
\begin{align*}
\det(A)
&=
-
7\cdot
\left|\,\begin{matrix*}[r]
  8 & -4 &  5 \\
 -4 &  5 & -4 \\
 -1 & -9 & -9
\end{matrix*}\,\right|
+
1\cdot
\left|\,\begin{matrix*}[r]
 8 & -4 &  5 \\
-4 &  5 & -4 \\
 0 & -1 & -7
\end{matrix*}\,\right|.
\intertext{Die $3\times 3$-Determinanten können mit der Sarrus-Regel
ausgewertet werden:}
&=
-7\cdot
\bigl(
-360-16+180+25-288+144
\bigr)
+
\bigl(
-280+0+20-0-32+112
\bigr)
\\
&=
-7\cdot (-315)
+(-180)
=
2025.
\qedhere
\end{align*}
\end{loesung}

\begin{bewertung}
Entwicklungssatz ({\bf E}) 1 Punkt,
Vorzeichenregel ({\bf V}) 1 Punkt,
Unterdeterminanten ({\bf U}) 1 Punkt,
Sarrus ({\bf S}) 1 Punkt,
Berechnung der Unterdeterminanten ({\bf B}) 1 Punkt,
Wert der Determinanten ({\bf D}) 1 Punkt.
\end{bewertung}
