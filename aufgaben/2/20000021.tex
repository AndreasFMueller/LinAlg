Hans kritzelt ein paar bedeutende Jahreszahlen aus seinem Leben
auf ein Papier:
\[
\begin{matrix}
1&9&9&9\\
2&0&0&0\\
2&0&0&7\\
2&0&1&4
\end{matrix}
\]
Sofort stellt er sich die naheliegenden Fragen:
\begin{teilaufgaben}
\item Bilden diese Zahlen eine reguläre Matrix? 
\item Welchen Wert hat die Determinante?
\item Was ändert sich, wenn Hans seine Zahlen rückwärts und in
umgekehrter zeitlicher Reihenfolge hinkritzelt?
\end{teilaufgaben}

\thema{Determinante}

\begin{loesung}
\begin{teilaufgaben}
\item
Die Teilaufgabe a) lässt sich sofort lösen, wenn man b) gelöst
hat. Wenn die Determinate nicht verschwindet, ist die Matrix regulär.

Natürlich kann die Regularität auch mit Hilfe des Gauss-Algorithmus
festgestellt werden, was aber etwas mehr Arbeit gibt:
\begin{align*}
\begin{tabular}{|>{$}c<{$}>{$}c<{$}>{$}c<{$}>{$}c<{$}|}
\hline
1&9&9&9\\
2&0&0&0\\
2&0&0&7\\
2&0&1&4\\
\hline
\end{tabular}
&\rightarrow
\begin{tabular}{|>{$}c<{$}>{$}c<{$}>{$}c<{$}>{$}c<{$}|}
\hline
1&  9&  9&  9\\
0&-18&-18&-18\\
0&-18&-18&-11\\
0&-18&-17&-14\\
\hline
\end{tabular}
\rightarrow
\begin{tabular}{|>{$}c<{$}>{$}c<{$}>{$}c<{$}>{$}c<{$}|}
\hline
1&  9&  9&  9\\
0&  1&  1&  1\\
0&  0&  0&  7\\
0&  0&  1&  4\\
\hline
\end{tabular}
\\
&
\rightarrow
\begin{tabular}{|>{$}c<{$}>{$}c<{$}>{$}c<{$}>{$}c<{$}|}
\hline
1&  9&  9&  9\\
0&  1&  1&  1\\
0&  0&  1&  4\\
0&  0&  0&  7\\
\hline
\end{tabular}
\rightarrow
\begin{tabular}{|>{$}c<{$}>{$}c<{$}>{$}c<{$}>{$}c<{$}|}
\hline
1&  9&  9&  9\\
0&  1&  1&  1\\
0&  0&  1&  4\\
0&  0&  0&  1\\
\hline
\end{tabular}
\end{align*}
Offenbar kann der Gaussalgorithmus ohne verschwindende Pivot-Elemente
durchgeführt werden, die Matrix ist also regulär. Die Pivot-Elemente
waren der Reihe nach: $1$, $-18$, $1$ und $7$.
Ausserdem wurde eine Zeilenvertauschung vorgenommen.
\item
Wir berechnen zunächst die Determinante durch Entwicklung
nach der zweiten Zeile:
\begin{align*}
\left|\;
\begin{matrix}
1&9&9&9\\
2&0&0&0\\
2&0&0&7\\
2&0&1&4
\end{matrix}
\;\right|
&=
-2\cdot
\left|\;\begin{matrix}
9&9&9\\
0&0&7\\
0&1&4
\end{matrix}\;\right|
=-2\cdot 9\cdot
\left|\;\begin{matrix}
1&1&1\\
0&0&7\\
0&1&4
\end{matrix}\;\right|
\\
&=
-2\cdot 9\cdot(-7)\cdot
\left|\;\begin{matrix}
1&1\\
0&1
\end{matrix}\;\right|
=
-2\cdot 9\cdot(-7)=126.
\end{align*}
Da die Determinanten nicht verschwindet, ist die Matrix regulär.

Die Determinante kann auch aus den Pivot-Elementen des oben durchgeführten
Gauss-Algorithmus abgeleitet werden. Dabei muss allerdings beachtet werden,
dass dazu auch eine Zeilenvertauschung notwendig war, d.~h.~die Determinante
ist 
\[
\det(A)=\underbrace{1\cdot (-18)\cdot 1\cdot 7}_{\text{Produkt der Pivotelemente}}\cdot \underbrace{(-1)}_{\text{Zeilenvertauschung}}
\]
\item Rückwärts geschriebene Zahlen in umgekehrter Reihenfolge 
ergeben die Matrix
\[
A'=\begin{pmatrix}
4&1&0&2\\
7&0&0&2\\
0&0&0&2\\
9&9&9&1
\end{pmatrix}
\]
Durch 6 Vertauschungen von aufeinanderfolgenden Zeilen und 6
Vertauschungen aufeinanderfolgender Spalten kann $A$ in $A'$
übergeführt werden. Da sich bei Zeilen- oder Spaltenvertauschungen
nur das Vorzeichen der Determinate ändert, ändert sich bei
einer geraden Zahl von Vertauschungen wie beim "Ubergang von $A$
zu $A'$ nichts.
\qedhere
\end{teilaufgaben}
\end{loesung}

\begin{bewertung}
Entwicklungssatz ({\bf E}) 1 Punkt,
rekursive Berechnung der Determinanten ({\bf B}) 1 Punkt,
Wert der Determinanten ({\bf D}) 1 Punkt,
Regularit\ät ({\bf R}) 1 Punkt, 
Rechenregeln für Vertauschungen von Zeilen und Spalten ({\bf V}) 1 Punkt,
Resultat $\det A=\det A'$ ({\bf A}) 1 Punkt.
\end{bewertung}




