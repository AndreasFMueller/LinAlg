Am 21.~Dezember 2020 kam es zu einer ``grossen Konjuktion'' zwischen den
Planeten Jupiter und Saturn, sie kamen sich bis auf 6 Winkelminuten
nahe, das entspricht etwa einem Fünftel des Vollmonddurchmessers.
Bei dieser Gelegenheit konnte man auch mit einem stark vergrössernden
Teleskop beide Planeten gleichzeitig im Gesichtsfeld des Teleskops sehen.
Die Beobachtung war allerdings nicht ganz einfach, weil beide Planeten sehr
tief am Himmel standen.

In einem Koordinatensystem, dessen Nullpunkt in der Sonne liegt, 
waren die Planeten zum Beobachtungszeitpunkt an den Positionen
\[
\vec{p}_{\jupiter}
=
\begin{pmatrix*}[r]
   3.001\\
  -4.122\\
  -0.049
\end{pmatrix*}
,\qquad
\vec{p}_{\saturn}
=
\begin{pmatrix*}[r]
   5.488\\
  -8.345\\
  -0.071
\end{pmatrix*}
,\qquad
\vec{p}_{\earth}
=
\begin{pmatrix*}[r]
  -0.0057\\
   0.9836\\
   0.0000
\end{pmatrix*}
\]
in sogenannten heliozentrischen Koordinaten, die die astronomische Einheit,
den mittleren Abstand von Sonne und Erde als Einheit verwenden.
Darin steht $\jupiter$ für Jupiter, $\saturn$ für Saturn und $\earth$ für
die Erde.

Auf der Erde befindet sich eine Kamera mit einem Chip von $3600\times 2400$
Pixeln und einem Teleskop mit $100000$ Pixeln Brennweite als Objektiv.
Diese Brennweite entspricht bei einer handelsüblichen Spiegelreflexkamera
etwa einer Objektivbrennweite von 150mm.

Die Ausrichtung  der Kamera wird durch die Drehmatrix
\[
D=\begin{pmatrix*}[r]
   0.7481&  0.4364&  0.5000\\
  -0.4277& -0.2591&  0.8660\\
   0.5074& -0.8617& -0.0072
\end{pmatrix*}
\]
beschrieben.
\begin{teilaufgaben}
\item
Finden Sie die Pixelkoordinaten von Jupiter und Saturn auf dem Chip.
\item
Wie weit auseinander in Pixeln befinden sich die Bilder von Jupiter und
Saturn?
\end{teilaufgaben}

\begin{loesung}
Die Kameramatrix ist
\[
K
=
\begin{pmatrix}
f&0&w/2\\
0&f&h/2\\
0&0& 1
\end{pmatrix}
=
\begin{pmatrix}
100000&      &1800\\
      &100000&1200\\
   0  &   0  &  1
\end{pmatrix}
\]
Das Kamerazentrum ist die Position der Erde, also $\vec{c}=\vec{p}_{\earth}$.
Damit kann man jetzt die Kameraprojektionsmatrix
\[
P = K D \begin{pmatrix} E&-\vec{p}_{\earth}\end{pmatrix}
\]
konstruieren.
Die homogenen Koorindaten $\tilde{p}_*$ konstruiert man, indem man den
Vektoren $\vec{p}_*$ eine vierte Komponente mit Wert $1$ hinzufügen.
Damit finden wir dann die homogenen Chipkoordinaten 
\[
\tilde{v}_* = P \tilde{p}_*.
\]
\begin{teilaufgaben}
\item
Die Berechnung ergibt
\[
\begin{aligned}
\tilde{b}_{\jupiter}
&=
\begin{pmatrix*}[r]
   10338.64922\\
    6556.67448\\
       5.92545
\end{pmatrix*}
&&\quad\Rightarrow\quad&
\vec{b}_{\jupiter}
&=
\begin{pmatrix*}[r]
   1744.8\\
   1106.5
\end{pmatrix*}
\\
\tilde{b}_{\saturn}
&=
\begin{pmatrix*}[r]
   19821.23756\\
   13581.64004\\
      10.82647
\end{pmatrix*}
&&\quad\Rightarrow\quad&
\vec{b}_{\saturn}
&=
\begin{pmatrix*}[r]
   1830.8\\
   1254.5
\end{pmatrix*}
\end{aligned}
\]
\item
Die Distanz zwischen den beiden Bildern auf dem Chip ist
\[
|\vec{b}_{\jupiter}-\vec{b}_{\saturn}|
=
\biggl|
\begin{pmatrix*}[r]
   -86.025\\
  -147.957
\end{pmatrix*}
\biggr|
=  171.15.
\]
Damit kann man auch den Winkel $\delta$ zwischen den beiden Punkten berechnen,
man erhält
\[
\tan \alpha \approx \frac{171.15}{f} = 0.00171
\quad\Rightarrow\quad
\alpha\approx
\delta =  0.098^\circ \approx 6'.
\]
Dies passt zur Beschreibung der beobachteten Konjunktion.
\qedhere
\end{teilaufgaben}
\end{loesung}

%f =  100000
%w =  3600
%h =  2400
%ans =  0.30000
%jupiter_xyz =
%
%   3.001054
%  -4.122257
%  -0.049672
%
%saturn_xyz =
%
%   5.488365
%  -8.345376
%  -0.071530
%
%earth_xyz =
%
%  -0.0057861444
%   0.9836756826
%   0.0000037676
%
%separation =  0.10178
%K =
%
%   100000        0     1800
%        0   100000     1200
%        0        0        1
%
%ans =  1.00000
%ans =
%
%   1.0000e+00   1.9075e-16  -4.3368e-18
%   1.9075e-16   1.0000e+00   4.8572e-17
%  -4.3368e-18   4.8572e-17   1.0000e+00
%
%D =
%
%   0.7480707   0.4363523   0.4999869
%  -0.4276592  -0.2591271   0.8660027
%   0.5074425  -0.8616553  -0.0072353
%
%ans =
%
%   1.00000   0.00000   0.00000   0.00579
%   0.00000   1.00000   0.00000  -0.98368
%   0.00000   0.00000   1.00000  -0.00000
%
%P =
%
%   75720.4634779   42084.2539026   49985.6677874  -40959.3159712
%  -42156.9923914  -26946.6939739   86591.5912568   26262.5549021
%       0.5074425      -0.8616553      -0.0072353       0.8505255
%
%jupiter_p =
%
%   3.001054
%  -4.122257
%  -0.049672
%   1.000000
%
%jupiter_b =
%
%   10316.88690
%    6527.12604
%       5.92571
%
%jupiter_b =
%
%   1741.0378
%   1101.4924
%      1.0000
%
%saturn_p =
%
%   5.488365
%  -8.345376
%  -0.071530
%   1.000000
%
%saturn_b =
%
%   19837.83013
%   13576.01865
%      10.82691
%
%saturn_b =
%
%   1832.2708
%   1253.9145
%      1.0000
%
\begin{bewertung}
Kameramatrix ({\bf K}) 1 Punkt,
Projektionsmatrix ({\bf P}) 1 Punkt,
homogene Bildkoordinaten und Pixelkoordinaten \jupiter{} ({\bf J}) 2 Punkt,
homogene Bildkoordinaten und Pixelkoordinaten \saturn{} ({\bf S}) 2 Punkt,
Abstand \jupiter--\saturn{} ({\bf A}) 1 Punkt.
\end{bewertung}

