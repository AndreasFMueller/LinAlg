Betrachten Sie die Matrix
\[
A=\begin{pmatrix}
 2&1\\
-1&t
\end{pmatrix}.
\]
\begin{teilaufgaben}
\item Berechnen Sie das charakteristische Polynom von $A$.
\item Wie muss man $t$ w"ahlen, damit $A$ nur den einen Eigenwert $3$ hat?
\item Finden Sie einen Eigenvektor zu diesem Eigenwert.
\item Ist $A$ in diesem Fall diagonalisierbar?
\end{teilaufgaben}


\begin{loesung}
\begin{teilaufgaben}
\item
Das charakteristische Polynom ist 
\begin{align}
\chi_A(\lambda)
&=
\det(A-\lambda E)
=
\left|\begin{matrix}
2-\lambda &    1     \\
-1        &t-\lambda
\end{matrix}\right|
=
(2-\lambda)(t-\lambda)+1=\lambda^2-(2+t)\lambda +2t+1.
\label{60000028:charpoly}
\end{align}
\item
Setzt man $\lambda=3$ in das charakteristische Polynom \eqref{60000028:charpoly}
ein, erh"alt man
\[
0=\chi_A(3)=9-(2+t)3+2t+1=-t+4\qquad\Rightarrow\qquad t=4.
\]
\item
Einen Eigenvektor kann man mit dem Gauss-Algorithmus finden:
\begin{align*}
\begin{tabular}{|>{$}c<{$}>{$}c<{$}|}
\hline
 2-3&1\\
-1&4-3\\
\hline
\end{tabular}
&\rightarrow
\begin{tabular}{|>{$}c<{$}>{$}c<{$}|}
\hline
-1&1\\
-1&1\\
\hline
\end{tabular}
\rightarrow
\begin{tabular}{|>{$}c<{$}>{$}c<{$}|}
\hline
 1&-1\\
 0& 0\\
\hline
\end{tabular}
\end{align*}
Die zweite Variable ist frei w"ahlbar.
Die erste Zeile besagt, dass die beiden Komponenten des Vektors gleich
sein m"ussen.
Ein m"oglicher Eigenvektor ist damit
\[
v=
\begin{pmatrix}
1\\1
\end{pmatrix}
\]
\item
Es gibt keinen zweiten linear unabh"angigen Eigenvektor zum Eigenwert $3$, 
somit ist $A$ nicht diagonalisierbar.
\qedhere
\end{teilaufgaben}
\end{loesung}

\begin{bewertung}
Charakteristisches Polynome ({\bf X}) 1 Punkt,
$\lambda$ einsetzen ({\bf L}) 1 Punkt,
Wert von $t$ ({\bf T}) 1 Punkt,
Gauss Algorithmus f"ur diese Werte von $\lambda$ und $t$ durchf"uhren
({\bf G}) 1 Punkt,
Eigenvektor ({\bf E}) 1 Punkt,
nicht Diagonalisierbarkeit mit Begr"undung ({\bf D}) 1 Punkt.
\end{bewertung}



