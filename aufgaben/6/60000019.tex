Über einem gleichseitigen Sechseck mit Umkreisradius $1$ 
wird ein gerades Prisma mit einer Höhe von 2 errichtet.
Der Punkt $A$ hat die Koordinaten 
$\left(\frac{\sqrt{3}}{2},\frac{1}{2},1\right)$.

\begin{center}
\begin{tikzpicture}[thick,>=latex]
\begin{scope}[xshift=-3.7cm]
\node at (0,0) {\includeagraphics[width=5cm]{left.jpg}};
\node at (-2.45,-0.6) {$x$};
\node at (1.6,-1.05) {$y$};
\node at (0,2.55) {$z$};
\node at (-1.3,1.1) {$A$};
\node at (0.8,-0.6) {$B$};
\end{scope}
\begin{scope}[xshift=3.7cm]
\node at (0,0) {\includeagraphics[width=5cm]{right.jpg}};
\node at (-2.45,-0.6) {$x$};
\node at (1.6,-1.05) {$y$};
\node at (0,2.55) {$z$};
\node at (1.1,0.95) {$A'$};
\node at (-1.4,0.9) {$B'$};
\end{scope}
\end{tikzpicture}
\end{center}

\begin{teilaufgaben}
\item
Finden Sie eine Drehmatrix $D$, die das Prisma links im Bild auf das
Prisma rechts im Bild abbildet.
\item
Berechnen sie den Drehwinkel dieser Drehmatrix.
\item 
Auf welchen Punkt wird der Punkt $P=(\frac{\sqrt{3}}{3},1,1)$ abgebildet?
\end{teilaufgaben}

\thema{Abbildungsmatrix}
\thema{orthogonale Matrix}
\thema{Drehwinkel}

\begin{loesung}
\begin{teilaufgaben}
\item Die gesuchte Drehmatrix $D$ bildet die Punkte 
$A = \left(\frac{\sqrt{3}}{2},\frac{1}{2},1\right)$,
$B = \left(0,1,0\right)$ und 
$C = \left(\frac{\sqrt{3}}{2},\frac{1}{2},-1\right)$
wie folgt ab:
\[
\begin{aligned}
\vec a&\mapsto \vec a ' = \begin{pmatrix}0\\1\\1\end{pmatrix},
&
\vec b&\mapsto \vec b ' = \begin{pmatrix}\frac{\sqrt{3}}{2}\\0\\\frac{1}{2}\end{pmatrix},
&
\vec c&\mapsto \vec c ' = \begin{pmatrix}0\\-1\\1\end{pmatrix}.
\end{aligned}.
\]
Wir haben also die Abbildungen
\[
\begin{aligned}
D
\underbrace{\begin{pmatrix}
\frac{\sqrt{3}}{2} & 0 &\frac{\sqrt{3}}{2}\\
\frac{1}{2} & 1 & \frac{1}{2}\\
1 &0 & -1
\end{pmatrix}}_{\displaystyle P}
&=
\underbrace{
\begin{pmatrix}
0&\frac{\sqrt{3}}{2}&0\\
1& 0&-1\\
1& \frac{1}{2}&1
\end{pmatrix}}_{\displaystyle P'}
&&\Rightarrow&
D
&=
P'P^{-1}
\end{aligned}.
\]
Wir berechnen die inverse Matrix mit dem Gauss-Algorithmus
\begin{align*}
\begin{tabular}{|>{$}c<{$}>{$}c<{$}>{$}c<{$}|>{$}c<{$}>{$}c<{$}>{$}c<{$}|}
\hline
\frac{\sqrt{3}}{2} & 0 &\frac{\sqrt{3}}{2}&1&0&0\\
\frac{1}{2} & 1 & \frac{1}{2}&0&1&0\\
1 &0 & -1&0&0&1\\
\hline
\end{tabular}
&\rightarrow
\begin{tabular}{|>{$}c<{$}>{$}c<{$}>{$}c<{$}|>{$}c<{$}>{$}c<{$}>{$}c<{$}|}
\hline
1 & 0 &1&\frac{2\sqrt{3}}{3}&0&0\\
0 & 1 & 0&-\frac{\sqrt{3}}{3}&1&0\\
0 &0 & -2&-\frac{2\sqrt{3}}{3}&0&1\\
\hline
\end{tabular}\\
&\rightarrow
\begin{tabular}{|>{$}c<{$}>{$}c<{$}>{$}c<{$}|>{$}c<{$}>{$}c<{$}>{$}c<{$}|}
\hline
1 & 0 &0&\frac{\sqrt{3}}{3}&0&\frac{1}{2}\\
0 & 1 & 0&-\frac{\sqrt{3}}{3}&1&0\\
0 &0 & 1&\frac{\sqrt{3}}{3}&0&-\frac{1}{2}\\
\hline
\end{tabular}
\end{align*}
Kontrolle:
\[
\begin{pmatrix}
\frac{\sqrt{3}}{2} & 0 &\frac{\sqrt{3}}{2}\\
\frac{1}{2} & 1 & \frac{1}{2}\\
1 &0 & -1\\
\end{pmatrix}
\begin{pmatrix}
\frac{\sqrt{3}}{3}&0&\frac{1}{2}\\
-\frac{\sqrt{3}}{3}&1&0\\
\frac{\sqrt{3}}{3}&0&-\frac{1}{2}
\end{pmatrix}
=
\begin{pmatrix}
1&0&0\\
0&1&0\\
0&0&1
\end{pmatrix}.
\]
Daraus erhalten wir die gesuchte Drehmatrix 
\[
D=
\begin{pmatrix}
0&\frac{\sqrt{3}}{2}&0\\
1& 0&-1\\
1& \frac{1}{2}&1
\end{pmatrix}
\begin{pmatrix}
\frac{\sqrt{3}}{3}&0&\frac{1}{2}\\
-\frac{\sqrt{3}}{3}&1&0\\
\frac{\sqrt{3}}{3}&0&-\frac{1}{2}
\end{pmatrix}
=
\begin{pmatrix}
-\frac{1}{2} &\frac{\sqrt{3}}{2}&0\\
0&0&1\\
\frac{\sqrt{3}}{2}&\frac{1}{2}&0
\end{pmatrix}.
\]
Man kann durch Nachrechnen prüfen, dass $DD^t=E$ und $\det(D) = 1$.
Die Matrix $D$ ist also tatsächlich eine Drehmatrix.
\item
Den Drehwinkel von $D$ erhalten wir aus der Spurformel
\begin{align*}
\cos\alpha &= \frac{\operatorname{Spur}D-1}{2} = -\frac{-0.5+0+0-1}{2} = -0.75,
\\
\alpha &=138.59^\circ.
\end{align*}
\item
Die Matrix $D$ kann nun dazu verwendet werden, um das Bild von $P$ zu bestimmen:
\[
\vec p' = D\vec p = 
\begin{pmatrix}
-\frac{1}{2} &\frac{\sqrt{3}}{2}&0\\
0&0&1\\
\frac{\sqrt{3}}{2}&\frac{1}{2}&0
\end{pmatrix}
\begin{pmatrix}
\frac{\sqrt{3}}{3}\\
1\\
1
\end{pmatrix}
= 
\begin{pmatrix}
-\frac{1}{2}\cdot \frac{\sqrt{3}}{3}+\frac{\sqrt{3}}{2}\\
1\\
\frac{\sqrt{3}}{2}\cdot\frac{\sqrt{3}}{3} + \frac{1}{2}
\end{pmatrix}
= 
\begin{pmatrix}
\frac{\sqrt{3}}{3}\\
1\\
1
\end{pmatrix}
\]
Der Punkt $P$ wird folglich auf sich selber abgebildet, 
woraus wir schliessen können, dass der Vektor $\overrightarrow{OP}$
der Drehachse entspricht.
\begin{center}
\begin{tikzpicture}[thick,>=latex]
\begin{scope}[xshift=-3.7cm]
\node at (0,0) {\includeagraphics[width=5cm]{left_sol.jpg}};
\node at (-2.45,-0.6) {$x$};
\node at (1.6,-1.05) {$y$};
\node at (0,2.55) {$z$};
\node at (-1.3,1.1) {$A$};
\node at (0.8,-0.6) {$B$};
\node at (-0.25,1.1) {$P$};
\end{scope}
\begin{scope}[xshift=3.7cm]
\node at (0,0) {\includeagraphics[width=5cm]{right_sol.jpg}};
\node at (-2.45,-0.6) {$x$};
\node at (1.6,-1.05) {$y$};
\node at (0,2.55) {$z$};
\node at (1.1,0.95) {$A'$};
\node at (-1.4,0.9) {$B'$};
\node at (-0.25,1.1) {$P'$};
\end{scope}
\end{tikzpicture}
\end{center}

\end{teilaufgaben}
\end{loesung}

\begin{bewertung}
Drei Abbildungspunkte ({\bf A}) 2 Punkte,
Ermittlung der Abbildungsmatrix ({\bf M}) 2 Punkte,
Spurformel und Drehwinkel ({\bf W}) 1 Punkt,
Bildpunkt von $P$ ({\bf B}) 1 Punkt.
\end{bewertung}

