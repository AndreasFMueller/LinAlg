Ist die Matrix
\[
A=\begin{pmatrix}
\pi&  0&0\\
0  &\pi&1\\
0  &  0&\pi
\end{pmatrix}
\]
diagonalisierbar?

\begin{loesung}
Die Matrix $A$ hat das charakteristische Polynom
\[
\chi_A(\lambda)=\left|\begin{matrix}
\pi-\lambda&  0&0\\
0  &\pi-\lambda&1\\
0  &  0&\pi-\lambda
\end{matrix}\right|
=(\pi-\lambda)^3
\]
mit der dreifachen Nullstelle $\pi$. Es gibt also nur einen Eigenwert
$\lambda=\pi$. Um die Eigenvektoren zu finden, muss man den Gauss-Algorithmus
auf $A-\pi E$ anwenden:
\[
\begin{tabular}{|>{$}c<{$}>{$}c<{$}>{$}c<{$}|}
\hline
0&0&0\\
0&0&1\\
0&0&0\\
\hline
\end{tabular}
\rightarrow
\begin{tabular}{|>{$}c<{$}>{$}c<{$}>{$}c<{$}|}
\hline
0&0&1\\
0&0&0\\
0&0&0\\
\hline
\end{tabular}
\]
Daraus liest man ab, dass die ersten beiden Variablen frei wählbar sind,
und dass die dritte Variable immer verschwindet. Man findet also die 
folgenden zwei möglichen Eigenvektoren
\[
e_1=\begin{pmatrix}
1\\0\\0
\end{pmatrix},\quad
e_2=\begin{pmatrix}
0\\1\\0
\end{pmatrix}.
\]
Natürlich sind auch alle Linearkombinationen davon Eigenvektoren von $A$.
Allerdings bilden die Eigenvektoren keine Basis von $\mathbb R^3$, die
Matrix $A$ ist also nicht diagonalisierbar.
\end{loesung}

