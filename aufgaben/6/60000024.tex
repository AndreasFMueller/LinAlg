Im linken Bild zeigt der Daumen in Richtung der $y$-Achse und die beiden
anderen Finger zeigen in Richtung der $z$-Achse.
Im rechten Bild zeigt der Daumen in Richtung des Punktes $(1,1,1)$, die
beiden anderen Finger in Richtung des Punktes $(1,-2,1)$ (die beiden kleinen
Kugeln).
\begin{center}
\begin{tikzpicture}[>=latex,thick]
\begin{scope}[xshift=-4.3cm]
\node at (0,0) {\includeagraphics[width=8cm]{devil.jpg}};
\node[color=red] at (-3.8,-2.2) {$x$};
\node[color=darkgreen] at (3.8,-2.2) {$y$};
\node[color=blue] at (0.35,3.55) {$z$};
\end{scope}
\begin{scope}[xshift=4.3cm]
\node at (0,0) {\includeagraphics[width=8cm]{devil-rotated.jpg}};
\node[color=red] at (-3.8,-2.2) {$x$};
\node[color=darkgreen] at (3.8,-2.2) {$y$};
\node[color=blue] at (0.35,3.55) {$z$};
\end{scope}
\end{tikzpicture}
\end{center}
\begin{teilaufgaben}
\item Finden Sie eine Drehmatrix, die die Hand im linken Bild auf die
Hand im rechten Bild abbildet.
\item Finden Sie den Drehwinkel.
\end{teilaufgaben}

\thema{Drehmatrix}
\thema{Drehwinkel}

\begin{loesung}
\begin{teilaufgaben}
\item
Die Drehmatrix bildet die beiden Standardbasisvektoren $e_2$ und $e_3$ auf
\[
v_2
=
\frac{1}{\sqrt{3}}
\begin{pmatrix}
1\\1\\1
\end{pmatrix}
\qquad\text{und}\qquad
v_3
=
\frac{1}{\sqrt{6}}
\begin{pmatrix}
1\\-2\\1
\end{pmatrix}
\]
Wir brauchen auch noch einen dritten Vektor, der abgebildet werden soll,
wir verwenden dazu das Vektorprodukt
\[
v_2\times v_3
=
\frac{1}{3\sqrt{2}}
\begin{pmatrix}
1\cdot 1-1\cdot(-2)\\
1\cdot 1-1\cdot 1\\
1\cdot(-2)-1\cdot 1
\end{pmatrix}
=
\frac{1}{3\sqrt{2}}
\begin{pmatrix}
3\\0\\-3
\end{pmatrix}
=
\frac{1}{\sqrt{2}}
\begin{pmatrix}
1\\0\\-1
\end{pmatrix}.
\]
In der Drehmatrix stehen die Bilder der Standardbasisvektoren, also
\[
D
=
\begin{pmatrix}
 \frac{1}{\sqrt{2}}&\frac{1}{\sqrt{3}}&\frac{ 1}{\sqrt{6}}\\
                  0&\frac{1}{\sqrt{3}}&\frac{-2}{\sqrt{6}}\\
-\frac{1}{\sqrt{2}}&\frac{1}{\sqrt{3}}&\frac{ 1}{\sqrt{6}}
\end{pmatrix}
\]
\item
Mit der Drehwinkelformel 
\[
\cos \alpha = \frac{\operatorname{Spur}D-1}{2}
\]
findet man
\begin{align*}
\cos \alpha
&= 
\frac12\biggl(
\frac{1}{\sqrt{2}}
+
\frac{1}{\sqrt{3}}
+
\frac{1}{\sqrt{6}}
-1\biggr)
\\
&=
\frac1{2\sqrt{6}}(\sqrt{3}+\sqrt{2}+1-\sqrt{6})
=
0.34635
\qquad\Rightarrow\qquad
\alpha = 69.736^\circ.
\qedhere
\end{align*}
\end{teilaufgaben}
\end{loesung}

\begin{diskussion}
Man kann natürlich auch noch die Drehachse finden, also einen Vektor
$d$ derart, dass $Dd=d$ ist.
Die numerische Rechnung ergibt den Vektor
\[
d=\begin{pmatrix*}[r]
  -0.74291\\
  -0.59447\\
   0.30772
\end{pmatrix*}.
\]
\end{diskussion}

\begin{bewertung}
\begin{teilaufgaben}
\item
Normierung ({\bf N}) 1 Punkt,
dritter Vektor ({\bf D}) 1 Punkt,
Kreuzprodukt ({\bf K}) 1 Punkt,
Matrix ({\bf M}) 1 Punkt,
\item
Spurformel ({\bf S}) 1 Punkt,
Drehwinkel ({\bf W}) 1 Punkt.
\end{teilaufgaben}
\end{bewertung}





