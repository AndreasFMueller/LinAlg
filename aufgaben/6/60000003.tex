Berechnen Sie Eigenwerte und Eigenvektoren der $n\times n$-Matrix
\[
A=\begin{pmatrix}
a    &1      &      &      \\
     &      a&     1&      \\
     &       &\ddots&     1\\
     &       &      &     a
\end{pmatrix}.
\]

\begin{loesung}
Das charakteristische Polynom ist
\begin{align*}
\operatorname{det}(A-\lambda E)
&=
\operatorname{det}\begin{pmatrix}
a-\lambda&        1&         &         &         \\
         &a-\lambda&        1&         &         \\
         &         &a-\lambda&         &         \\
         &         &         &\ddots   &        1\\
         &         &         &         &a-\lambda
\end{pmatrix}
=(a-\lambda)^n.
\end{align*}
Dieses Polynom hat nur die $n$-fache Nullstelle $a$.
Eigenvektoren sind L"osungen des homogenen Gleichungssystems
mit der Koeffizienten-Matrix
\[
B=A-aE=\begin{pmatrix}
0&1& &      & \\
 &0&1&      & \\
 & &0&     1& \\
 & & &\ddots&1\\
 & & &      &0
\end{pmatrix}.
\]
Das Gleichungssystem kann dadurch gel"ost werden, dass die
letzte Spalte ganz nach rechts permuttiert wird. Dann ist das
Gausstableau
\[
\begin{tabular}{|cccccc|}
\hline
$x_1$&$x_2$&$x_3$&&$x_{n-1}$&$x_n$\\
\hline
0       &1       &0       &$ \dots$&0       &0       \\
0       &0       &1       &$ \dots$&0       &0       \\
0       &0       &0       &$\ddots$&0       &0       \\
$\vdots$&$\vdots$&$\vdots$&$\ddots$&$\ddots$&$\vdots$\\
0       &0       &0       &$ \dots$&0       &1       \\
0       &0       &0       &$ \dots$&0       &0       \\
\hline
\end{tabular}
\rightarrow
\begin{tabular}{|cccccc|}
\hline
$x_2$&$x_3$&&$x_{n-1}$&$x_n$&$x_1$\\
\hline
1       &0       &$ \dots$&0       &0       &0        \\
0       &1       &$ \dots$&0       &0       &0        \\
0       &0       &$\ddots$&0       &0       &0        \\
$\vdots$&$\vdots$&$\ddots$&$\ddots$&$\vdots$&$\vdots$ \\
0       &0       &$ \dots$&0       &1       &0        \\
0       &0       &$ \dots$&0       &0       &0        \\
\hline
\end{tabular}
\]
Daraus liest man ab, dass $x_1$ frei w"ahlbar ist, dass aber
aber alle anderen Unbekannten auf $x_2=x_3=\dots=x_{n-1}=x_n=0$
festgelegt sind. $e_1$ und seine Vielfachen sind also die einzigen
Eigenvektoren von $A$.
\end{loesung}

