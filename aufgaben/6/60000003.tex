Von einer Kamera soll die Kameramatrix
\[
K=\begin{pmatrix}
f&0&m_x\\
0&f&m_y\\
0&0&1
\end{pmatrix}
\]
bestimmt werden.
Zu diesem Zweck wird die Kamera im Nullpunkt des Koordinatensystems
so montiert, dass sie genau in Richtung der $z$-Achse blickt ($D=E$).
In dieser Situation werden die Punkte
\[
\begin{aligned}
P_1&=(1,2,10)
&&\text{und}&
P_2&=(-2,-1,12)
\end{aligned}
\]
im dreidimensionalen Raum auf die Punkte
\[
\begin{aligned}
B_1&=(91,62)
&&\text{und}&
B_2&=(88,59)
\end{aligned}
\]
auf dem Chip abgebildet.
Stellen Sie ein Gleichungssystem auf, mit welchem die Matrix $K$ 
bestimmt werden kann.
Ihre Methode sollte erweiterbar sein, sodass sie auch mit mehr als 
zwei Punkten noch funktioniert.

\thema{Kamerageometrie}
\thema{Least Squares}

\begin{loesung}
Die Theorie der Kamerabbildung sagt, dass ein Punkt des dreidmensionalen
Raumes wie folgt abgebildet wird.
Zunächst muss man den Ortsvektor $\vec{p}$ in einen vierdimensionalen Vektor
$\tilde{p}$ in homogenen Koordinaten umwandeln, indem man eine $1$ 
anhängt.
Auf $\tilde{p}$ wendet man dann erst $\begin{pmatrix}E&-c\end{pmatrix}$,
dann $D$ und schliesslich $K$ an, um den Bildpunkt $\tilde{b}$ in homogenen
Koordinaten zu erhalten.
Da aber $D=E$ und $c=0$ (die Kamera ist im Nullpunkt $c=0$ montiert) ist,
bleibt nur noch
\[
\tilde{b} = K\vec{p}.
\]
Wendet man die Matrix
\[
K=
\begin{pmatrix}
\color{red}f&0&\color{red}m_x\\
0&\color{red}f&\color{red}m_y\\
0&0&1
\end{pmatrix}
\]
auf den Punkt $P$ an, erhält man die Gleichungen
\begin{equation}
\begin{linsys}{3}
{\color{red}f} p_1& &                  &+&{\color{red}m_x}\cdot p_3&=&\tilde{b}_1\\
                  & &{\color{red}f} p_2&+&{\color{red}m_y}\cdot p_3&=&\tilde{b}_2\\
                  & &                  & &                      p_3&=&\tilde{b}_3
\end{linsys}
\label{60000003:equations}
\end{equation}
In der Aufgabenstellung sind die Punkte $B$ nicht in homogenen Koordinaten
gegeben, sondern in gewöhnlichen Koordinaten.
Für die Gleichungen~\eqref{60000003:equations} brauchen wir aber homogene
Koordinaten $\tilde{b}$ und zwar so, dass $\tilde{b_3}=p_3$ ist.
Wir erreichen dies, indem wir mit $p_3$ multipizieren:
\[
\tilde{b}
=
\begin{pmatrix}
\tilde{b}_1\\
\tilde{b}_2\\
\tilde{b}_3
\end{pmatrix}
=
p_3 \begin{pmatrix}
b_1\\b_2\\1
\end{pmatrix}
\qquad\Rightarrow\qquad
\left\{
\begin{aligned}
\tilde{b}_1 &= p_3b_1\\
\tilde{b}_2 &= p_3b_2\\
\end{aligned}
\right.
\]
Damit können wir jetzt die endgültigen Gleichungen hinschreiben:
\begin{equation}
\begin{linsys}{3}
{\color{red}f} p_1&+&{\color{red}m_x}\cdot p_3& &                         &=&p_3b_1\\
{\color{red}f} p_2& &                         &+&{\color{red}m_y}\cdot p_3&=&p_3b_2
\end{linsys}
\label{60000003:final}
\end{equation}
Jetzt können wir das Gauss-Tableau für die Bestimmung von 
$\color{red}f$, $\color{red}m_x$ und $\color{red}m_y$ aufstellen:
\[
\begin{tabular}{|>{$}c<{$}>{$}c<{$}>{$}c<{$}|>{$}c<{$}|}
\hline
\color{red}f&\color{red}m_x&\color{red}m_y& \\
\hline
p_1&p_3&  0&p_3b_1\\
p_2&  0&p_3&p_3b_2\\
\hline
\end{tabular}
\]
Dies wiederholen wir für alle gegebenen Punkte und erhalten so das
Tableau
\[
\begin{tabular}{|>{$}r<{$}>{$}r<{$}>{$}r<{$}|>{$}r<{$}|}
\hline
\color{red}f&\color{red}m_x&\color{red}m_y& \\
\hline
    1&  10&   0& 910\\
    2&   0&  10& 620\\
   -2&  12&   0&1056\\
   -1&   0&  12& 708\\
\hline
\end{tabular}
\]
Dieses überbestimmte Gleichungssystem $Ax=d$ kann nach dem Standardverfahren
gelöst werden, indem man das Gleichungssystem mit Koeffizientenmatrix
\[
\transpose{A}A
=
\begin{pmatrix}
    10&  -14&    8\\
   -14&  244&    0\\
     8&    0&  244
\end{pmatrix}
\]
und rechter Seite
\[
\transpose{A}d
=
\begin{pmatrix}
    -670\\
   21772\\
   14696
\end{pmatrix}
\]
löst.
Die Lösung ist 
\begin{align*}
\color{red}  f&= 10.899,\\
\color{red}m_x&= 89.855\qquad\text{und}\\
\color{red}m_y&= 59.872,
\end{align*}
was ziemlich nahe bei den wahren Parametern
${\color{red}f}=10$,
${\color{red}m_x}=90$ und
${\color{red}m_y}=60$ ist.
\end{loesung}

\begin{bewertung}
Zu bestimmtende Variablen ({\bf V}) 1 Punkt,
Gleichungen ({\bf G}) 1 Punkt,
Homogene Koordinaten anpassen ({\bf H}) 1 Punkt,
überbestimmtes Gauss-Tableau ({\bf T}) 1 Punkt,
Matrix $\transpose{A}A$ ({\bf A}) 1 Punkt,
Vektor $\transpose{A}b$ ({\bf B}) 1 Punkt.
\end{bewertung}


