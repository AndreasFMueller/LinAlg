Ein Fluzeug fängt sich aus dem Sturzflug
(Abbildung~\ref{60000036:fig} links)
in die Normallage
(Abbildung~\ref{60000036:fig} rechts)
auf.
Im Sturzflug zeigt die Nase in die Richtung des Vektors
$\transpose{(1,1,-1)}$ und die Flügel sind horizontal.
\begin{figure}[ht]
\centering
\includeagraphics[]{plane.pdf}
\caption{
Zwei Fluglagen des Flugzeugs in Aufgabe~\ref{60000036}
vor und nach dem Abfangmanöver aus dem Sturzflug.
\label{60000036:fig}}
\end{figure}
\begin{teilaufgaben}
\item Finden Sie eine Drehmatrix, die diese Drehung erzeugt.
\item Wie gross ist der Drehwinkel?
\end{teilaufgaben}

\begin{loesung}
\begin{teilaufgaben}
\item
Es ist einfacher, die Drehung zu finden, die von der Fluglage rechts
in die Fluglage links führt, und dann die inverse Matrix zu bestimmen.
Da die Drehmatrix eine orthogonale Matrix ist, ist die Inverse einfach
die Transponierte.

Wir also die Bilder der Standardbasisvektoren in der Abbilung rechts.
Der erste Vektor ist der rechte Flügel, er wird auf die Richtung
\[
\begin{pmatrix} 1\\-1\\0 \end{pmatrix}
\qquad\Rightarrow\qquad
a_1
=
\frac{1}{\sqrt{2}}
\begin{pmatrix} 1\\-1\\0 \end{pmatrix}
\]
abgebildet.
Der Vektor $e_2$ wird auf die Richtung der Flugzeugnase abgebildet,
also
\[
a_2
=
\frac{1}{\sqrt{3}}
\begin{pmatrix} 1\\1\\-1 \end{pmatrix}.
\]
Der vertikale Vektor $e_3$ ist das Vektorprodukt von $e_1$ und $e_2$,
er wird auf das Vektorprodukt
\[
a_3
=
a_1\times a_2
=
\frac{1}{\sqrt{2}}
\begin{pmatrix} 1\\-1\\0 \end{pmatrix}
\times
\frac{1}{\sqrt{3}}
\begin{pmatrix} 1\\1\\-1 \end{pmatrix}.
=
\frac{1}{\sqrt{6}}
\begin{pmatrix}
(-1)\cdot(-1)-0\cdot 1\\
0\cdot 1 - (-1)\cdot 1\\
1\cdot 1 - 1\cdot(-1)
\end{pmatrix}
=
\frac{1}{\sqrt{6}}
\begin{pmatrix}
1\\
1\\
2
\end{pmatrix}
\]
abgebildet.
Füllt man diese Vektoren in eine Matrix ein, erhält man die Inverse
\[
R^{-1}
=
\begin{pmatrix}
 \frac{1}{\sqrt{2}}& \frac{1}{\sqrt{3}}&\frac{1}{\sqrt{6}}\\
-\frac{1}{\sqrt{2}}& \frac{1}{\sqrt{3}}&\frac{1}{\sqrt{6}}\\
       0           &-\frac{1}{\sqrt{3}}&\frac{2}{\sqrt{6}}
\end{pmatrix}
\]
und damit für die Drehmatrix durch Transponieren
\[
R
=
\transpose{R^{-1}}
=
\begin{pmatrix}
\frac{1}{\sqrt{2}}&-\frac{1}{\sqrt{2}}&       0           \\
\frac{1}{\sqrt{3}}& \frac{1}{\sqrt{3}}&-\frac{1}{\sqrt{3}}\\
\frac{1}{\sqrt{6}}& \frac{1}{\sqrt{6}}& \frac{2}{\sqrt{6}}
\end{pmatrix}.
\]

\item
Den Drehwinkel kann man mit der Spurformel
\[
\cos\alpha
=
\frac{\operatorname{Spur}A-1}{2}
=
\frac{\frac{1}{\sqrt{3}}+\frac{1}{\sqrt{2}}+\frac{2}{\sqrt{6}}-1}{2}
=
0.5505
\qquad\Rightarrow\qquad
\alpha = 56.600^\circ.
\qedhere
\]
\end{teilaufgaben}
\end{loesung}

\begin{bewertung}
\begin{teilaufgaben}
\item
Vorgehensweise ({\bf L}) 1 Punkt,
Bestimmung von drei geeigneten Vektoren ({\bf V}) 1 Punkte,
Methode zur Berechnung der Inversen ({\bf I}) 1 Punkt,
Berechnung der Drehmatrix ({\bf D}) 1 Punkt.
\item
Spurformel ({\bf S}) 1 Punkt,
Winkel ({\bf W}) 1 Punkt.
\end{teilaufgaben}
\end{bewertung}

\begin{diskussion}
Alternativ, aber leider etwas aufwendiger, kann man die Drehmatrix
auch finden, indem man im Flugzeug links leicht zu identifizierende
Vektoren sucht, und ihre Bilder im rechten Bild bestimmt.

Der erste Vektor hat die Richtung der Nase:
\[
\vec{b}_1
=
\frac{1}{\sqrt{3}}\begin{pmatrix*}[r] 1\\1\\-1\end{pmatrix*},
\]
er wird auf $e_2$ abgebildet.
Der zweite Vektor hat die Richtung des linken Flügels:
\[
\begin{pmatrix*}[r]-1\\1\\0\end{pmatrix*}
\qquad\Rightarrow\qquad
\vec{b}_2
=
\frac{1}{\sqrt{2}}
\begin{pmatrix*}[r]-1\\1\\0\end{pmatrix*},
\]
er wird auf $-e_1$ abgebildet.
Der dritte Vektor ist das Vektorprodukt der beiden Vektoren:
\[
\vec{b}_1\times \vec{b}_2
=
\frac{1}{\sqrt{3}}
\begin{pmatrix*}[r] 1\\1\\-1\end{pmatrix*}
\times
\frac{1}{\sqrt{2}}
\begin{pmatrix*}[r]-1\\1\\0\end{pmatrix*}
=
\frac{1}{\sqrt{6}}
\begin{pmatrix}
1\cdot 0-(-1)\cdot 1\\
(-1)\cdot(-1)-0\cdot 1\\
1\cdot 1-(-1)\cdot 1
\end{pmatrix}
=
\frac{1}{\sqrt{6}}
\begin{pmatrix}
1\\1\\2
\end{pmatrix},
\]
er wird auf $e_3$ abgebildet.

Schreibt man die Vektoren als Spalten in eine Matrix $B$, dann ist
jetzt die Matrix $R$ gesucht, die
\[
RB
=
R
\begin{pmatrix}
 \frac{1}{\sqrt{3}}&-\frac{1}{\sqrt{2}}&\frac{1}{\sqrt{6}}\\
 \frac{1}{\sqrt{3}}& \frac{1}{\sqrt{2}}&\frac{1}{\sqrt{6}}\\
-\frac{1}{\sqrt{3}}&       0           &\frac{2}{\sqrt{6}}
\end{pmatrix}
=
\begin{pmatrix}
 0&-1& 0\\
 1& 0& 0\\
 0& 0& 1
\end{pmatrix}
=
C.
\]
Die Drehmatrix ist dann $R=CB^{-1}$.

Die inverse Matrix $B^{-1}$ kann mit dem Gauss-Algorithmus
bestimmt werden.
\bgroup
\renewcommand\arraystretch{1.3}
\begin{align*}
\begin{tabular}{|>{$}c<{$}>{$}c<{$}>{$}c<{$}|>{$}c<{$}>{$}c<{$}>{$}c<{$}|}
\hline
 \frac{1}{\sqrt{3}}&-\frac{1}{\sqrt{2}}&\frac{1}{\sqrt{6}}&1&0&0\\
 \frac{1}{\sqrt{3}}& \frac{1}{\sqrt{2}}&\frac{1}{\sqrt{6}}&0&1&0\\
-\frac{1}{\sqrt{3}}&       0           &\frac{2}{\sqrt{6}}&0&0&1\\[2pt]
\hline
\end{tabular}
&\to
\begin{tabular}{|>{$}c<{$}>{$}c<{$}>{$}c<{$}|>{$}c<{$}>{$}c<{$}>{$}c<{$}|}
\hline
1&-\frac{\sqrt{3}}{\sqrt{2}}&\frac{1}{\sqrt{2}}& \sqrt{3}&0&0\\
0& \frac{2}{\sqrt{2}}       &      0           & -1      &1&0\\
0&-\frac{1}{\sqrt{2}}       &\frac{3}{\sqrt{6}}&  1      &0&1\\[2pt]
\hline
\end{tabular}
\\
\to
\begin{tabular}{|>{$}c<{$}>{$}c<{$}>{$}c<{$}|>{$}c<{$}>{$}c<{$}>{$}c<{$}|}
\hline
1&-\frac{\sqrt{3}}{\sqrt{2}}&\frac{1}{\sqrt{2}}& \sqrt{3}           &     0            &0\\
0&       1                  &      0           &-\frac{1}{\sqrt{2}} &\frac{1}{\sqrt{2}}&0\\
0&       0                  &      1           & \frac{1}{\sqrt{6}} &\frac{1}{\sqrt{6}}&\frac{2}{\sqrt{6}}\\[2pt]
\hline
\end{tabular}
&\to
\begin{tabular}{|>{$}c<{$}>{$}c<{$}>{$}c<{$}|>{$}c<{$}>{$}c<{$}>{$}c<{$}|}
\hline
1&-\frac{\sqrt{3}}{\sqrt{2}}& 0& \sqrt{3}-\frac{1}{2\sqrt{3}} &-\frac{1}{2\sqrt{3}} &-\frac{1}{\sqrt{3}}\\
0&       1                  & 0&-\frac{1}{\sqrt{2}}           & \frac{1}{\sqrt{2}}  & 0\\
0&       0                  & 1& \frac{1}{\sqrt{6}}           & \frac{1}{\sqrt{6}}  & \frac{2}{\sqrt{6}}\\[2pt]
\hline
\end{tabular}
\\
\to
\begin{tabular}{|>{$}c<{$}>{$}c<{$}>{$}c<{$}|>{$}c<{$}>{$}c<{$}>{$}c<{$}|}
\hline
1& 0& 0& \sqrt{3}-\frac{1}{2\sqrt{3}}-\frac{\sqrt{3}}2 &-\frac{1}{2\sqrt{3}}+\frac{\sqrt{3}}{2} &-\frac{1}{\sqrt{3}}\\
0& 1& 0&-\frac{1}{\sqrt{2}}                            & \frac{1}{\sqrt{2}}  & 0\\
0& 0& 1& \frac{1}{\sqrt{6}}                            & \frac{1}{\sqrt{6}}  & \frac{2}{\sqrt{6}}\\[2pt]
\hline
\end{tabular}
&
=
\begin{tabular}{|>{$}c<{$}>{$}c<{$}>{$}c<{$}|>{$}c<{$}>{$}c<{$}>{$}c<{$}|}
\hline
1& 0& 0& \frac{1}{\sqrt{3}} & \frac{1}{\sqrt{3}}  &-\frac{1}{\sqrt{3}}\\
0& 1& 0&-\frac{1}{\sqrt{2}} & \frac{1}{\sqrt{2}}  & 0\\
0& 0& 1& \frac{1}{\sqrt{6}} & \frac{1}{\sqrt{6}}  & \frac{2}{\sqrt{6}}\\[2pt]
\hline
\end{tabular}
\end{align*}
\egroup
Die Drehmatrix kann jetzt durch Berechnung des Produktes
\[
R
=
CB^{-1}
=
\begin{pmatrix}
 0&-1& 0\\
 1& 0& 0\\
 0& 0& 1
\end{pmatrix}
\begin{pmatrix}
 \frac{1}{\sqrt{3}} & \frac{1}{\sqrt{3}}  &-\frac{1}{\sqrt{3}}\\
-\frac{1}{\sqrt{2}} & \frac{1}{\sqrt{2}}  & 0\\
 \frac{1}{\sqrt{6}} & \frac{1}{\sqrt{6}}  & \frac{2}{\sqrt{6}}
\end{pmatrix}
=
\begin{pmatrix}
 \frac{1}{\sqrt{2}} &-\frac{1}{\sqrt{2}}  & 0\\
 \frac{1}{\sqrt{3}} & \frac{1}{\sqrt{3}}  &-\frac{1}{\sqrt{3}}\\
 \frac{1}{\sqrt{6}} & \frac{1}{\sqrt{6}}  & \frac{2}{\sqrt{6}}
\end{pmatrix}
\]
gefunden werden und stimmt mit der früher gefundenen Matrix überein.
\end{diskussion}

