Bestimmen Sie das charakteristische Polynom der $n\times n$-Matrix
\[
A=\begin{pmatrix}
-a_{n-1}&-a_{n-2}&-a_{n-3}&\dots & -a_2 & -a_1 & -a_0 \\
   1    &   0    &   0    &\dots &   0  &   0  &   0  \\
        &   1    &   0    &      &      &      &      \\
        &        &\ddots  &\ddots&      &      &      \\
        &        &        &\ddots&   0  &      &      \\
        &        &        &      &   1  &   0  &      \\
        &        &        &      &      &   1  &   0  
\end{pmatrix}.
\]

\begin{hinweis}Entwicklung nach der letzten Spalte
\end{hinweis}

\begin{loesung}
Es ist die Determinanten
\begin{align*}
\chi_A(\lambda)
=
\det(A-\lambda E)
&=
\left|\begin{matrix}
-a_{n-1}-\lambda&-a_{n-2}&-a_{n-3}&\dots & -a_2   & -a_1   & -a_0  \\
   1            &-\lambda&   0    &\dots &   0    &   0    &   0   \\
                &   1    &-\lambda&      &        &        &       \\
                &        &\ddots  &\ddots&        &        &       \\
                &        &        &\ddots&-\lambda&        &       \\
                &        &        &      &   1    &-\lambda&       \\
                &        &        &      &        &   1    &-\lambda
\end{matrix}\right|
\end{align*}
Schreiben wir $A_k$ f"ur eine $k\times k$-Matrix mit der selben
Struktur wie $A$, also zum Beispiel auch $A=A_n$, dann finden wir
f"ur $n=1$ und $n=2$ die charakteristischen Polynome
\begin{align}
\chi_{A_1}(\lambda)
&=
-a_0-\lambda = -(\lambda+a_0),
\label{60000022:verankerung1}
\\
\chi_{A_2}(\lambda)
&=
\left|\begin{matrix}
-a_1-\lambda&  -a_0  \\
1           &-\lambda
\end{matrix}\right|
=\lambda^2+a_1\lambda+a_0.
\label{60000022:verankerung2}
\end{align}
Daraus und aus dem Resultat von Aufgabe~\ref{60000012} gewinnen wir
die Vermutung, dass 
\[
\chi_{A_k}(\lambda)
=
(-1)^k\bigl(
\lambda^k+a_{k-1}\lambda^{k-1}+a_{k-2}\lambda^{k-2}+\dots a_1\lambda+a_0
\bigr)
\]
gilt.

Die Vermutung kann mit vollst"andiger Induktion bewiesen werden.
Die Formeln (\ref{60000022:verankerung1}) und (\ref{60000022:verankerung2})
dienen als Induktionsverankerung.

F"ur den Induktionsschritt d"urfen wir annehmen, dass die Formel bereits
f"ur $n-1$ bewiesen ist, und m"ussen Sie f"ur den Fall $n$ beweisen.
Der Entwicklungssatz liefert
\begin{align*}
\chi_A(\lambda)
&=
(-1)^{n-1}(-a_0)\left|\begin{matrix}
   1            &-\lambda&   0    &\dots &   0    &   0    \\
                &   1    &-\lambda&      &        &        \\
                &        &\ddots  &\ddots&        &        \\
                &        &        &\ddots&-\lambda&        \\
                &        &        &      &   1    &-\lambda\\
                &        &        &      &        &   1    
\end{matrix}\right|
+(-\lambda)\left|\begin{matrix}
-a_{n-1}-\lambda&-a_{n-2}&-a_{n-3}&\dots & -a_2   & -a_1   \\
   1            &-\lambda&   0    &\dots &   0    &   0    \\
                &   1    &-\lambda&      &        &        \\
                &        &\ddots  &\ddots&        &        \\
                &        &        &\ddots&-\lambda&        \\
                &        &        &      &   1    &-\lambda
\end{matrix}\right|
\\
&=(-1)^n a_0 + (-\lambda)\det(A_{n-1}-\lambda E)
\\
&=(-1)^n a_0 -\lambda (-1)^{n-1}
\bigl(
\lambda^{n-1}+a_{n-1}\lambda^{n-2}+a_{n-2}\lambda^{n-3}+\dots+a_2\lambda+a_1
\bigr)
\\
&=(-1)^n\bigl(
\lambda^n+a_{n-1}\lambda^{n-1}+a_{n-2}\lambda^{n-2}+\dots+a_2\lambda^2+a_1\lambda+a_0
\bigr).
\end{align*}
Damit ist die Vermutung auch f"ur den Fall $n$ bewiesen.
Insbesondere kann jedes beliebige Polynom mit Leitkoeffizient $1$
als charakteristisches Polynom einer Matrix auftreten.
\end{loesung}

