Bei den Tests einer Trägheitsplattform lieferte diese die folgende Matrix,
die die Lage der Achsen der Trägheitsplattform beschreiben.
% 0.650501 0.090217 -0.754128
% -0.719082 0.392770 -0.573283
% 0.244479 0.915201 0.320371
% A=[ 0.650501,0.090217,-0.754128; -0.719082,0.392770,-0.573283; 0.244479,0.915201,0.320371]
\[
A=\begin{pmatrix}
\phantom{-}0.650501&0.090217&          -0.754128\\
  -0.719082&0.392770&          -0.573283\\
\phantom{-}0.244479&0.915201&\phantom{-}0.320371
\end{pmatrix}
\]
\begin{teilaufgaben}
\item Ist $A$ orthogonal (im Rahmen der angegebenen Genauigkeit)?
\item Bestimmen Sie die Determinante von $A$.
\item Ist $A$ eine Drehmatrix? Wenn ja, bestimmen Sie den Drehwinkel.
\end{teilaufgaben}

\thema{orthogonale Matrix}
\thema{Determinante}
\thema{Drehwinkel}
\thema{Sarrus-Formel}

\begin{loesung}
\begin{teilaufgaben}
\item Es ist zu überprüfen, ob $\transpose{A}A=E$ ist, was man direkt nachrechnen kann:
\begin{align*}
A\transpose{A}&=
\begin{pmatrix}
\phantom{-}0.650501&0.090217&          -0.754128\\
  -0.719082&0.392770&          -0.573283\\
\phantom{-}0.244479&0.915201&\phantom{-}0.320371
\end{pmatrix}
\cdot
\begin{pmatrix}
\phantom{-}0.650501&           -0.719082& \phantom{-}0.244479\\
\phantom{-}0.090217& \phantom{-}0.392770& \phantom{-}0.915201\\
  -0.754128&           -0.573283& \phantom{-}0.320371\\
\end{pmatrix}
\\
&=
\begin{pmatrix}
\phantom{-}1.000000455166&          -0.000000163144&0.000000449787\\
  -0.000000163144&\phantom{-}1.000000250390&0.000000329885\\
\phantom{-}0.000000449787&\phantom{-}0.000000329885&1.000000016114
\end{pmatrix}
\simeq E.
\end{align*}
Im Rahmen der Genauigkeit von 6 Stellen, mit denen die Matrix die
Matrix $A$ angegeben wurde, stimmt $\transpose{A}A$ mit $E$ überein.
\item Eine orthogonale Matrix hat Determinante $+1$ oder $-1$. Man muss
also nur noch herausfinden, welcher der Fälle eingetreten ist.

Die Determinante kann in diesem Fall mit der Sarrusschen Regel
berechnet werden, es ergibt sich ein Wert nahe bei 1.
Alternativ kann man auch testen, ob die Determinante $+1$ ist,
indem man nachrrechnet, ob die drei Spaltenvektoren eine
Rechts-System bilden, oder ob die ersten zwei Spalten als Vektorprodukt
die dritte Spalte ergeben. Dies kann man direkt nachrechnen:
\begin{align*}
\begin{pmatrix}
\phantom{-}0.650501\\
  -0.719082\\
\phantom{-}0.244479
\end{pmatrix}
\times
\begin{pmatrix}
0.090217\\
0.392770\\
0.915201
\end{pmatrix}
&=
\begin{pmatrix}
-0.719082\cdot 0.915201 - 0.244479\cdot 0.392770\\
\phantom{-}0.244479\cdot 0.090217 - 0.650501\cdot 0.915201\\
\phantom{-}0.650501\cdot 0.392770 + 0.719082\cdot 0.090217\\
\end{pmatrix}
\\
&=
\begin{pmatrix}
%-0.719082 * 0.915201 - 0.244479 * 0.392770
%0.244479 * 0.090217 - 0.650501 * 0.915201
%0.650501 * 0.392770 + 0.719082 * 0.090217
%
-0.754128582312\\
-0.573283003758\\
\phantom{-}0.320370698564
\end{pmatrix}
\simeq
\begin{pmatrix}
-0.754128\\
-0.573283\\
\phantom{-}0.320371
\end{pmatrix}
\end{align*}
Somit bilden die Spalten der Matrix $A$ ein Rechtssystem, die Determinante
ist $\det(A)=+1$.
\item Wegen $\det(A)=1$ ist die orthogonale Matrix $A$ eine Drehmatrix,
der Drehwinkel $\alpha$ kann aus der Spur mit Hilfe der Formel
\[
\operatorname{Spur}(A)=1+2\cos\alpha\qquad\Rightarrow\qquad
\cos\alpha=\frac{\operatorname{Spur}(A)-1}2
\]
gefunden werden. Der Wert der Spur ist
\[
\operatorname{Spur}(A) = 0.650501  + 0.392770  + 0.320371 = 1.363642.
\]
Also folgt für den Drehwinkel:
\[
\cos\alpha=\frac{1.363642-1}2= 0.181821\qquad\Rightarrow\qquad
\alpha=79.5242^\circ.
\qedhere
\]
\end{teilaufgaben}
\end{loesung}

