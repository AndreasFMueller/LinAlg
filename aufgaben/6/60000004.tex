Ein wesentlicher Schritt des Jacobi-Algorithmus ist, den Winkel
$\alpha$ so zu bestimmen, dass die symmetrische Matrix $A$ durch eine Drehung
um den Winkel $\alpha$ diagonalisiert wird:
\[
\begin{pmatrix}
 \cos\alpha&\sin\alpha\\
-\sin\alpha&\cos\alpha
\end{pmatrix}
\begin{pmatrix}
a_{11}&a_{12}\\
a_{21}&a_{22}
\end{pmatrix}
\begin{pmatrix}
 \cos\alpha&-\sin\alpha\\
 \sin\alpha& \cos\alpha
\end{pmatrix}
=
\begin{pmatrix}
\lambda_1&0\\
0&\lambda_2
\end{pmatrix}.
\]
\begin{teilaufgaben}
\item Bestimmen Sie eine Gleichung für $\alpha$.
\item Lösen Sie die Gleichung in dem Spezialfall
\[
A=\begin{pmatrix}
3&1\\
1&1
\end{pmatrix}
\]
\item Bestimmen Sie Eigenwerte und Eigenvektoren für den
Spezialfall b) mit dem Jacobi-Algorithmus.
\item "Uberprüfen Sie Eigenwerte und Eigenvektoren mit dem klassischen
Algorithmus.
\end{teilaufgaben}

\begin{loesung}
\begin{teilaufgaben}
\item Zunächst sind die Matrizenprodukte auf der linken Seite zu berechnen
\begin{align*}
D_{-\alpha}AD_{\alpha}
&=
\begin{pmatrix}
 a_{11}\cos\alpha+a_{21}\sin\alpha & a_{12}\cos\alpha + a_{22}\sin\alpha\\
-a_{11}\sin\alpha+a_{21}\cos\alpha &-a_{12}\sin\alpha + a_{22}\cos\alpha
\end{pmatrix}
\begin{pmatrix}
 \cos\alpha&-\sin\alpha\\
 \sin\alpha& \cos\alpha
\end{pmatrix}
\end{align*}
Für die Gleichung für $\alpha$ brauchen wir nur das Matrixelement
in der rechten oberen Ecke, also
\begin{align*}
0
&=
-a_{11}\sin\alpha\cos\alpha -a_{21}\sin^2\alpha+a_{12}\cos^2\alpha+a_{22}\sin\alpha\cos\alpha
\\
&=
(a_{22}-a_{11})\sin\alpha\cos\alpha+a_{12}(\cos^2\alpha-\sin^2\alpha)
\\
&=
\frac{a_{22}-a_{11}}2\sin2\alpha +a_{12}\cos 2\alpha
\\
\frac{2a_{12}}{a_{11}-a_{22}}
&=
\tan2\alpha
\\
\alpha&=\frac12\arctan\frac{2a_{12}}{a_{11}-a_{22}}
\end{align*}
\item
In diesem Spezialfall gilt
\[
\alpha=\frac12\arctan\frac{2\cdot 1}{3 - 1}=\frac12\arctan 1=\frac{\pi}8.
\]
\item
Die Spalten der Drehmatrix sind Eigenvektoren:
\[
D_{\frac{\pi}8}=\begin{pmatrix}
   0.92388& -0.38268\\
   0.38268&  0.92388
\end{pmatrix}
\]
Transformiert man die Matrix $A$ damit, erhält man
\[
D^tAD=
\begin{pmatrix}
   3.41421&  0.00000\\
   0.00000&  0.58579
\end{pmatrix}
\]
\item
Die Eigenwerte werden mit dem charakteristischen Polynom bestimmt:
\begin{align*}
0&=\left|\,\begin{matrix}3-\lambda&1\\1&1-\lambda\end{matrix}\,\right|
\\
&=(3-\lambda)(1-\lambda)-1=\lambda^2-4\lambda+2
\\
\lambda_{\pm}
&=
2\pm\sqrt{4-2}=2\pm\sqrt{2}=\begin{cases}
2+\sqrt{2}=3.41421\\
2-\sqrt{2}=0.58579
\end{cases}
\end{align*}
Die Eigenvektoren bestimmt man mit dem Gauss-Algorithmus.
Für $\lambda_+$ erhält man
\begin{align*}
\begin{tabular}{|>{$}c<{$}>{$}c<{$}|}
\hline
3-\lambda_+ & 1\\
1&1-\lambda+\\
\hline
\end{tabular}
&=
\begin{tabular}{|>{$}c<{$}>{$}c<{$}|}
\hline
1-\sqrt{2} & 1\\
1&-(1+\sqrt{2})\\
\hline
\end{tabular}
=
\begin{tabular}{|>{$}c<{$}>{$}c<{$}|}
\hline
1&\frac{1}{1-\sqrt{2}}\\
1&-\sqrt{2}\\
\hline
\end{tabular}
\\
&
=
\begin{tabular}{|>{$}c<{$}>{$}c<{$}|}
\hline
1&\frac{1+\sqrt{2}}{-1}\\
1&-(1+\sqrt{2})\\
\hline
\end{tabular}
\rightarrow
\begin{tabular}{|>{$}c<{$}>{$}c<{$}|}
\hline
1&-1-\sqrt{2}\\
0&0\\
\hline
\end{tabular}
\end{align*}
Die zweite Variable ist also frei wählbar, und damit ist
\[
v_+=\begin{pmatrix}1+\sqrt{2}\\1\end{pmatrix}
\]
ein Eigenvektor. Octave hat auf Länge 1 normierte Eigenvektoren
geliefert, tut man dies hier auch, mit $|v_+|=\sqrt{4+2\sqrt{2}}$
bekommt man den Vektor
\[
\frac{v_+}{|v_+|}=\frac{1}{\sqrt{4+2\sqrt{2}}}\begin{pmatrix}
1+\sqrt{2}\\1\end{pmatrix}
=\begin{pmatrix}
0.92388\\
0.38268
\end{pmatrix},
\]
also genau der Vektor aus der ersten Spalte der Drehmatrix.
Die Berechnung für $\lambda_-$ und $v_-$ verläuft analog.
\qedhere
\end{teilaufgaben}
\end{loesung}

