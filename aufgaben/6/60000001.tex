In dieser Aufgabe wird eine Kamera verwendet, deren Brennweite $f=100$ ist,
der Chip hat die Dimensionen $120\times 90$.
Die Kamera befindet sich im Punkt $(300,100,0)$ und ist auf den Punkt
$Q_0=(-100,800,100)$ gerichtet.
Die Drehmatrix zu diesem Zweck ist 
\[
D=
\begin{pmatrix}
  -0.868243& -0.496139&  0.000000\\
   0.061070& -0.106873&  0.992395\\
  -0.492366&  0.861640&  0.123091
\end{pmatrix}.
\]
\begin{teilaufgaben}
\item Bestimmen Sie die Kameraabbildungsmatrix $K$.
\item Berechnen Sie die Kameraprojektionsmatrix $P$.
\item Kontrollieren Sie, dass der Punkt $Q_0$ auf den Mittelpunkt des
Bildes abgebildet wird.
\item Bestimmen Sie die Bildkoordinaten des Punktes
$Q_1 = (100,600,-5)$.
\item Der Punkt $Q_2$ wird auf $B_2=(77,38)$ abgebildet.
Bestimmen Sie die Gerade, auf der sich $Q_2$ befinden muss.
\end{teilaufgaben}

\thema{Kamerageometrie}

\begin{loesung}
\begin{teilaufgaben}
\item Der Mittelpunkt des Chip hat die Koordinaten $(m_x,m_y)=(60,45)$,
damit wird die Kameramatrix
\[
K
=
\begin{pmatrix}
100&  0&60\\
  0&100&45\\
  0&  0& 1
\end{pmatrix}.
\]
\item Für die Kameraprojektionsmatrix wird die Matrix
\begin{align}
KD
&=
\begin{pmatrix}
100&  0&60\\
  0&100&45\\
  0&  0& 1
\end{pmatrix}
\begin{pmatrix}
  -0.868243& -0.496139&  0.000000\\
   0.061070& -0.106873&  0.992395\\
  -0.492366&  0.861640&  0.123091
\end{pmatrix}
\notag
\\
&=
\begin{pmatrix}
  -116.366272&     2.084532&     7.385489\\
   -16.049420&    28.086485&   104.778650\\
    -0.492366&     0.861640&     0.123091
\end{pmatrix}
\label{60000001:kd}
\end{align}
benötigt.
Damit kann die Kameraprojektionsmatrix berechnet werden, sie ist
\begin{align*}
P
&=
KD\begin{pmatrix}E&-c\end{pmatrix}
\\
&=
\begin{pmatrix}
  -116.36627&  2.08453&   7.38549& 34701.42838\\
   -16.04942& 28.08649& 104.77865&  2006.17753\\
    -0.49237&  0.86164&   0.12309&    61.54575
\end{pmatrix}.
\end{align*}
\item
Zur Kontrolle berechnen wir die Projektion des Punktes $Q_0$ durch die
Abbildung. 
Dazu muss $q_0$ zunächst in homogenen Koordinaten geschrieben werden,
auf die dann $P$ angewendet werden kann:
\[
P
\begin{pmatrix}
-100\\800\\100\\1
\end{pmatrix}
=
\begin{pmatrix}
   48744.23\\
   36558.17\\
     812.40
\end{pmatrix}.
\]
Dies sind die homogenen Koordinaten des Bild-Punktes, um die gewöhnlichen
Koordinaten zu finden, muss durch die dritte Komponente geteilt werden:
\[
b
=
\begin{pmatrix}
60\\45
\end{pmatrix}.
\]
Dies sind die Bildkoordinaten des Mittelpunktes, damit haben
wir bestätigt, dass die Kamera auf den Punkt $Q_0$ gerichtet ist.
\item
Die homogenen Koordinaten des Bildes des Punktes $Q_1$ sind
\[
\tilde b_1
=
P\tilde q_1
=
P\begin{pmatrix}100\\600\\-5\\1\end{pmatrix}
=
\begin{pmatrix}
   24278.593\\
   16729.233\\
     528.678
\end{pmatrix}.
\]
Division durch die dritte Komponente ergibt die Bildkoordinaten
\[
b_1 = \begin{pmatrix}
   45.923\\
   31.644
\end{pmatrix}.
\]
\item
Der Stützvektor ist bereits bekannt, es muss nur noch der Richtungsvektor
gefunden werden.
Dazu wird die Inverse von $KD$ wie in \eqref{60000001:kd} benötigt,
es ist
\[
(KD)^{-1}
=
\begin{pmatrix}
  -0.008682431&  0.000610705&  0.001098205\\
  -0.004961389& -0.001068733&  1.207416804\\
   0.000000000&  0.009923953& -0.323486406
\end{pmatrix}.
\]
Der Richtungsvektor ist dann
\[
r = (KD)^{-1}\tilde b_2 = (KD)^{-1} \begin{pmatrix} 77\\ 38\\ 1\end{pmatrix}
=
\begin{pmatrix}
  -0.6442\\
  \phantom{-}0.7848\\
  \phantom{-}0.0536
\end{pmatrix}.
\]
Die Gerade, auf welcher sich der Punkt $Q_2$ befindet, ist folglich
\[
g = c + sr = 
\begin{pmatrix}
300\\
100\\
0
\end{pmatrix} + s\,
\begin{pmatrix}
-0.6442\\
\phantom{-}0.7848\\
\phantom{-}0.0536
\end{pmatrix}.
\]
\qedhere
\end{teilaufgaben}
\end{loesung}
