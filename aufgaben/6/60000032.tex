Mit Hilfe eines Tastkopfes wurden die Koordinaten $(x_i,y_i,z_i)$,
$i=1,\dots,n$, einiger Punkte auf der Oberfläche eines vertikalen,
zylindrischen Rohres ermittelt.
Stellen Sie ein Gleichungssystem zur Bestimmung der $x$- und $y$-Koordinaten
der Achse und des Radius $r$ des Rohres auf.
Drücken Sie die zugehörigen Matrizen und Vektoren durch $x_i$ und $y_i$ aus.

\thema{Least Squares}

\begin{loesung}
\definecolor{darkred}{rgb}{0.6,0,0}
Es handelt sich um ein Least-Squares-Problem zur Bestimmung der
Unbekannten ${\color{darkred}x_0}$, ${\color{darkred}y_0}$
und ${\color{darkred}r}$.
Offensichtlich tragen die $z$-Koordinaten nichts zu den gesuchten
Daten bei.
Die Projektion des Mantels des Zylinders in die $x$-$y$-Ebene ist
ein Kreis um den Punkt $({\color{darkred}x_0},{\color{darkred}y_0})$
mit Radius ${\color{darkred}r}$.
Es handelt sich also um das in der Vorlesung besprochene Problem,
Mittelpunktskoordinaten und Radius eines Kreises aus Kreispunkten
zu bestimmen.

Die Punkte müssen die Kreisgleichung
\[
(x_i-{\color{darkred}x_0})^2 + (y_i-{\color{darkred}y_0})^2
=
{\color{darkred}r}^2
\]
erfüllen.
Durch Ausmultiplizieren und Umstellen findet man die Gleichungen
\begin{equation}
-2x_i{\color{darkred}x_0}
-2y_i{\color{darkred}y_0}
+({\color{darkred}x_0}^2
+{\color{darkred}y_0}^2
-{\color{darkred}r}^2)
=
-x_i^2
-y_i^2
\label{60000032:eqn}
\end{equation}
Indem man den Klammerausdruck auf der linken Seite zu einer neuen 
Unbekannten 
\[
{\color{darkred}c}
=
{\color{darkred}r}^2
-
{\color{darkred}x_i}^2
-
{\color{darkred}y_i}^2
\]
zusammenfassen, wird die Gleichung~\eqref{60000032:eqn} zu
\[
\begin{linsys}{3}
2x_1{\color{darkred}x_0}
&+&
2y_1{\color{darkred}y_0}
&+&
{\color{darkred}c}
&=&
x_1^2
+y_1^2
\\
2x_2{\color{darkred}x_0}
&+&
2y_2{\color{darkred}y_0}
&+&
{\color{darkred}c}
&=&
x_2^2
+y_2^2
\\
2x_n{\color{darkred}x_0}
&+&
2y_n{\color{darkred}y_0}
&+&
{\color{darkred}c}
&=&
x_n^2
+y_n^2
\\
\end{linsys}
\]
Es ist also das überbestimmte Gleichungssystem
\[
A{\color{darkred}x}
=
\begin{pmatrix}
2x_1&2y_1&1\\
2x_2&2y_2&1\\
\vdots&\vdots&\vdots\\
2x_n&2y_n&1
\end{pmatrix}
\begin{pmatrix}
{\color{darkred}x_0}\\
{\color{darkred}y_0}\\
{\color{darkred}c}
\end{pmatrix}
=
b
\]
zu lösen.
Das Gleichungssystem kann mit Hilfe des bekannten Verfahrens auf das
gewöhnliche $3\times 3$-Gleichungssystem
$\transpose{A}A{\color{darkred}x}=\transpose{A}b$
reduziert werden mit
\[
\transpose{A}A
=
\begin{pmatrix}
\displaystyle4\sum_{i=1}^n x_i^2  & \displaystyle4\sum_{i=1}^n x_iy_i & \displaystyle2\sum_{i=1}^n x_i \\
\displaystyle4\sum_{i=1}^n x_iy_i & \displaystyle4\sum_{i=1}^n x_i^2  & \displaystyle2\sum_{i=1}^n y_i \\
\displaystyle2\sum_{i=1}^n x_i    & \displaystyle2\sum_{i=1}^n y_i    & n
\end{pmatrix}
\qquad\text{und}\qquad
\transpose{A}b
=
\begin{pmatrix}
\displaystyle 2\sum_{i=1}^n x_i(x_i^2+y_i^2)\\
\displaystyle 2\sum_{i=1}^n y_i(x_i^2+y_i^2)\\
\displaystyle \sum_{i=1}^n (x_i^2+y_i^2)
\end{pmatrix}.
\]
Nach der Bestimmung von ${\color{darkred}x_0}$, ${\color{darkred}y_0}$
und ${\color{darkred}c}$ kann ${\color{darkred}r}$ mit Hilfe von
\[
{\color{darkred}r}
=
\sqrt{
{\color{darkred}c}+{\color{darkred}x_0}^2+{\color{darkred}y_0}^2
}
\]
bestimmt werden.
\end{loesung}

\begin{bewertung}
Least Squares ({\bf LS}) 1 Punkt,
Kreisgleichung ({\bf K}) 1 Punkt,
neue Variable $c$ ({\bf C}) 1 Punkt,
Methode $\transpose{A}Ax=\transpose{A}b$ ({\bf M}) 1 Punkt,
Berechnung von $\transpose{A}A$ ({\bf A}) 1 Punkt,
Berechnung von $\transpose{A}b$ ({\bf B}) 1 Punkt.
\end{bewertung}
