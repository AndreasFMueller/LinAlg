Eine Kamera mit einem $640\times 480$-Sensor und einem
Objektiv mit einer Brennweite von 400 Pixeln
ist im Punkt $C=(-4,-3,-2)$ montiert und mit der Matrix
\[
D
=
\begin{pmatrix*}[r]
 0.8742 &  -0.4856 &  0.0000\\
 -0.0000 &  0.0000 &  1.0000\\
 0.4856 &  0.8742 &  0.0000\end{pmatrix*}
\]
auf den Punkt $Q=(1,6,-2)$ ausgerichtet.
Die beiden Punkte $P_1=(9,11,5)$ und $P_2=(-6,17,-9)$
werden durch die Kamera auf die Punkte $B_1$ und $B_2$
abgebildet.
Wegen perspektivischer Verzerrung wird der Mittelpunkt
$M$ der Strecke $P_1P_2$ nicht auf den Mittelpunkt
$M'$ der Strecke $B_1B_2$ auf dem Sensor abgebildet.
Berechnen Sie die Pixelentfernung der Punkte $B_1$
und $B_2$ vom Bild des Mittelpunktes der Strecke
$P_1P_2$.
\begin{figure}[h]
\centering
\begin{tikzpicture}[>=latex,thick]
\begin{scope}[xshift=-5cm]
\node at (0,2.2500) {\includeagraphics[scale=1.2]{pov2025.pdf}};
\end{scope}
\draw (0,0) rectangle (6.0000,4.5000);
\node at (0,0) [left] {$(0,0)$};
\node at (6.0000,4.5000) [right] {$(640,480)$};
\draw[line width=0.3pt] (0,0) -- (6.0000,4.5000);
\draw[line width=0.3pt] (6.0000,0) -- (0,4.5000);
\draw[color=darkred] (3.9187,3.6656) -- (0.3937,0.6562);
\fill[color=white] (2.1562,2.1609) circle[radius=0.08];
\draw[color=darkred] (2.1562,2.1609) circle[radius=0.08];
\fill[color=blue] (2.2594,2.2500)
	circle[radius=0.08];
\node[color=blue] at (2.2594,2.2500) [above] {$M'\mathstrut$};
\fill[color=darkred] (3.9187,3.6656)
	circle[radius=0.08];
\node[color=darkred] at (3.9187,3.6656) [above] {$B_1$};
\fill[color=darkred] (0.3937,0.6562)
	circle[radius=0.08];
\node[color=darkred] at (0.3937,0.6562) [above] {$B_2$};
\end{tikzpicture}
\end{figure}

\begin{loesung}
Die Abbildung der Punkte $P_1$ und $P_2$ erfolgt mit
\[
p_i \mapsto p_i - c \mapsto D(p_i-c) \mapsto KD(p_i-c)
\]
und liefert homogene Koordinaten für die Bildpunkte.
Die Kameramatrix $K$ enthält Brennweite und
Mittelpunktskoordinaten des Sensors:
\[
K = \begin{pmatrix*}[r]
 400 &  0 & 320\\
  0 & 400 & 240\\
  0 &  0 &  1
\end{pmatrix*}.
\]
Angewendet auf die beiden gegebenen Punkte und den Mittelpunkt,
der die Koordinaten $(\frac32,14,-2)$ hat, erhält man
die homogenen Koordinaten der Bildpunkte
\[
\tilde{b}_1
=
\begin{pmatrix*}[r]
7762.9920\\7252.3840\\18.5516
\end{pmatrix*},\quad
\tilde{b}_2
=
\begin{pmatrix*}[r]
699.9360\\1163.0720\\16.5128
\end{pmatrix*}
\qquad\text{und}\qquad
\tilde{b}_m
=
\begin{pmatrix*}[r]
4231.4640\\4207.7280\\17.5322
\end{pmatrix*}.
\]
Division durch die dritte Komponente liefert die
auf ganze Zahlen gerundeten Pixelkoordinaten der
Bildpunkte
\[
b_1=\begin{pmatrix*}[r]
418\\391\end{pmatrix*},\quad
b_2=\begin{pmatrix*}[r]
42\\70\end{pmatrix*}
\qquad\text{und}\qquad
b_m=\begin{pmatrix*}[r]
241\\240\end{pmatrix*}.
\]
Die gesuchten Abstände sind
\begin{align*}
\overline{B_1M'}
&=
|b_1-b_m|
=
\biggl|\begin{pmatrix*}[r]
177\\151
\end{pmatrix*}\biggr|
=
232.6585,
\\
\overline{B_2M'}
&=
|b_2-b_m|
=
\biggl|\begin{pmatrix*}[r]
-199\\-170
\end{pmatrix*}\biggr|
=
261.7270.
\qedhere
\end{align*}
\end{loesung}


\begin{bewertung}
Kameramatrix mit Brennweite ({\bf F}), 1 Punkt,
und Verschiebung für den Mittelpunkt ({\bf V}),
1 Punkt,
homogene Koordinaten der beiden Bildpunkte ({\bf H})
1 Punkt,
Pixelkoordinaten der beiden Bildpunkte ({\bf P})
1 Punkt
Abbildung des Mittelpunktes zwischen $P_1$ und $P_2$
({\bf M}) 1 Punkt,
Pixelabstände der Bildpunkte vom Bild des
Mittelpunktes ({\bf A}) 1 Punkt.
\end{bewertung}

