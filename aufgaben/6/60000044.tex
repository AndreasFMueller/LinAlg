Eine Kamera mit einem $800\times 600$-Chip und einem
Objektiv mit einer Brennweite von 500 Pixeln
ist im Punkt $C=(-4,-7,-3)$ montiert und mit der Matrix
\[
D
=
\begin{pmatrix*}[r]
 0.8396 &  -0.5433 &  0.0000\\
 -0.0000 &  0.0000 &  1.0000\\
 0.5433 &  0.8396 &  0.0000\end{pmatrix*}
\]
auf den Punkt $Q=(7,10,-3)$ ausgerichtet.
Die beiden Punkte $P_1=(13,17,5)$ und $P_2=(1,10,-10)$
werden durch die Kamera auf die Punkte $B_1$ und $B_2$
abgebildet.
Wegen perspektivischer Verzerrung wird der Mittelpunkt
$M$ der Strecke $P_1P_2$ nicht auf den Mittelpunkt
der Strecke $B_1B_2$ auf dem Chip abgebildet,
sondern auf den Punkt $M'$.
Berechnen Sie die Pixelentfernung der Punkte $B_1$
und $B_2$ vom Bild des Mittelpunktes der Strecke
$P_1P_2$.
\begin{figure}[h]
\centering
\begin{tikzpicture}[>=latex,thick]
\begin{scope}[xshift=-5cm]
\node at (0,2.2500) {\includeagraphics[scale=1.2]{pov2026.pdf}};
\end{scope}
\draw (0,0) rectangle (6.0000,4.5000);
\node at (0,0) [left] {$(0,0)$};
\node at (6.0000,4.5000) [right] {$(800,600)$};
\draw[line width=0.3pt] (0,0) -- (6.0000,4.5000);
\draw[line width=0.3pt] (6.0000,0) -- (0,4.5000);
\draw[color=darkred] (3.1575,3.2700) -- (1.8900,0.7050);
\fill[color=white] (2.5237,1.9875) circle[radius=0.08];
\draw[color=darkred] (2.5237,1.9875) circle[radius=0.08];
\fill[color=blue] (2.6925,2.3325)
	circle[radius=0.08];
\node[color=blue] at (2.6925,2.3325) [above] {$M'\mathstrut$};
\fill[color=darkred] (3.1575,3.2700)
	circle[radius=0.08];
\node[color=darkred] at (3.1575,3.2700) [above] {$B_1$};
\fill[color=darkred] (1.8900,0.7050)
	circle[radius=0.08];
\node[color=darkred] at (1.8900,0.7050) [above] {$B_2$};
\end{tikzpicture}
\end{figure}

\begin{loesung}
Die Abbildung der Punkte $P_1$ und $P_2$ erfolgt mit
\[
p_i \mapsto p_i - c \mapsto D(p_i-c) \mapsto KD(p_i-c)
\]
und liefert homogene Koordinaten für die Bildpunkte.
Die Kameramatrix $K$ enthält Brennweite und
Mittelpunktskoordinaten des Sensors:
\[
K = \begin{pmatrix*}[r]
 500 &  0 & 400\\
  0 & 500 & 300\\
  0 &  0 &  1
\end{pmatrix*}.
\]
Angewendet auf die beiden gegebenen Punkte erhält man
die homogenen Koordinaten der Bildpunkte
\[
\tilde{b}_1
=
\begin{pmatrix*}[r]
12371.6000\\12815.9500\\29.3865
\end{pmatrix*},\quad
\tilde{b}_2
=
\begin{pmatrix*}[r]
4276.8300\\1596.9100\\16.9897
\end{pmatrix*}
\qquad\text{und}\qquad
\tilde{b}_m
=
\begin{pmatrix*}[r]
8324.2150\\7206.4300\\23.1881
\end{pmatrix*}.
\]
Division durch die dritte Komponente liefert die
auf ganze Zahlen gerundeten Pixelkoordinaten der
Bildpunkte
\[
b_1=\begin{pmatrix*}[r]
421\\436\end{pmatrix*},\quad
b_2=\begin{pmatrix*}[r]
252\\94\end{pmatrix*}
\qquad\text{und}\qquad
b_m=\begin{pmatrix*}[r]
359\\311\end{pmatrix*}.
\]
Die gesuchten Abstände sind
\begin{align*}
\overline{B_1M'}
&=
|b_1-b_m|
=
\biggl|\begin{pmatrix*}[r]
62\\125
\end{pmatrix*}\biggr|
=
139.5314,
\\
\overline{B_2M'}
&=
|b_2-b_m|
=
\biggl|\begin{pmatrix*}[r]
-107\\-217
\end{pmatrix*}\biggr|
=
241.9463.
\qedhere
\end{align*}
\end{loesung}


\begin{bewertung}
Kameramatrix mit Brennweite ({\bf F}), 1 Punkt,
und Verschiebung für den Mittelpunkt ({\bf V}),
1 Punkt,
homogene Koordinaten der beiden Bildpunkte ({\bf H})
1 Punkt,
Pixelkoordinaten der beiden Bildpunkte ({\bf P})
1 Punkt
Abbildung des Mittelpunktes zwischen $P_1$ und $P_2$
({\bf M}) 1 Punkt,
Pixelabstände der Bildpunkte vom Bild des
Mittelpunktes ({\bf A}) 1 Punkt.
\end{bewertung}

