Berechnen Sie $A^{47}e_1$ für die Matrix
\[
A=\begin{pmatrix}
3&1\\
1&3
\end{pmatrix}
\]

\thema{Eigenvektoren}
\thema{Eigenwerte}
\thema{charakteristisches Polynom}
\thema{Matrixpotenzen}

\begin{loesung}
Die Wirkung von $A$ auf Eigenvektoren ist einfach zu ermitteln,
denn es gilt $Av_{\pm}=\lambda_{\pm}v_{\pm}$.
Das charakteristische Polynom ist
\begin{align*}
\left|\,\begin{matrix}3-\lambda&1\\1&3-\lambda\end{matrix}\,\right|
&=
(3-\lambda)^2-1=\lambda^2-6\lambda+8=0
\\
\lambda_{\pm}&=3\pm\sqrt{9-8}=3\pm1.
\end{align*}
Für die Eigenwert $\lambda_+=4$ und $\lambda_-=2$ ergeben sich die
Eigenvektoren
\[
v_+=\begin{pmatrix}1\\1\end{pmatrix},\qquad
v_-=\begin{pmatrix}1\\-1\end{pmatrix}
\]
Jetzt kann man
\[
e_1=\frac12(v_++v_-)
\]
schreiben, und damit $A^{47}e_1$ ausrechnen
\[
A^{47}e_1=\frac12( \lambda_+^{47}v_++\lambda_-^{47}v_-)
=\frac12\begin{pmatrix}
4^{47}+2^{47}\\
4^{47}-2^{47}
\end{pmatrix}
=\frac{2^{47}}2\begin{pmatrix}
2^{47}+1\\
2^{47}-1
\end{pmatrix}
=2^{46}\begin{pmatrix}
2^{47}+1\\
2^{47}-1
\end{pmatrix}.
\qedhere
\]
\end{loesung}

