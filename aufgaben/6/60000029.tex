Betrachten Sie die Matrix
\[
A=\begin{pmatrix}3&s\\-2&7\end{pmatrix}.
\]

\begin{teilaufgaben}
\item
Berechnen Sie das charakteristische Polynom von $A$.
\item
Wie muss man $s$ wählen, damit $A$ nur einen Eigenwert hat?
\item
Finden Sie einen Eigenvektor zu diesem Eigenwert.
\item
Ist $A$ in diesem Fall diagonalisierbar?
\end{teilaufgaben}

\thema{Eigenwerte}
\thema{Eigenvektoren}
\thema{charakteristisches Polynom}
\thema{diagonalisierbar}

\begin{loesung}
\begin{teilaufgaben}
\item
Das charakteristische Polynom ist
\begin{align*}
\chi_A(\lambda)
&=
\left|\begin{matrix}3-\lambda&s\\-2&7-\lambda\end{matrix}\right|
=
(3-\lambda)(7-\lambda)+2s
=
\lambda^2 - 10\lambda + 21 + 2s
\end{align*}
\item
Das charakteristische Polynom hat genau dann nur eine einzige Nullstelle,
wenn die Diskriminante verschwindet, also
\[
b^2-4ac
=
(-10)^2 -4(21+2s)
=
100-84-8s
=
16-8s
=
0
\qquad
\qquad
\Rightarrow
s=2.
\]
Der Eigenwert ist Nullstelle des charakteristischen Polynoms,
aber weil nur eine einzige Nullstelle existiert, muss $\lambda=-b/2a=-5$
sein.
\item
Den Eigenvektor zum Eigenwert $5$ finden wir mit Hilfe des Gaussalgorithmus
\begin{align*}
\begin{tabular}{|>{$}c<{$}>{$}c<{$}|}
\hline
3-\lambda&2        \\
    -2   &7-\lambda\\
\hline
\end{tabular}
&=
\begin{tabular}{|>{$}c<{$}>{$}c<{$}|}
\hline
-2&2\\
-2&2\\
\hline
\end{tabular}
\rightarrow
\begin{tabular}{|>{$}c<{$}>{$}c<{$}|}
\hline
1&-1\\
0& 0\\
\hline
\end{tabular}
\end{align*}
Daraus liest man ab, dass 
\[
v=\begin{pmatrix}1\\1\end{pmatrix}
\]
ein Eigenvektor ist.
Zur Kontrolle berechnen wir
\[
Av
=
\begin{pmatrix}
3&s\\
-2&7
\end{pmatrix}
\begin{pmatrix}1\\1\end{pmatrix}
=
\begin{pmatrix}3+s\\5\end{pmatrix}.
\]
Für $s=2$ ist die rechte Seite $5v$, wie das für einen Eigenvektor zum
Eigenwert $\lambda=5$ sein muss.
\item
Da es keinen weiteren linear unabhängigen Eigenvektor zum Eigenwert $\lambda=5$
gibt, ist die Matrix $A$ nicht diagonalisierbar.
\qedhere
\end{teilaufgaben}
\end{loesung}

\begin{bewertung}
Charakteristisches Polynom ({\bf X}) 1 Punkt,
Diskrimenantenbedingung ({\bf D}) 1 Punkt,
Wert von $s$ ({\bf S}) 1 Punkt,
Eigenwert ({\bf E}) 1 Punkt,
Eigenvektor ({\bf V}) 1 Punkt,
Nichtdiagonalisierbarkeit ({\bf N}) 1 Punkt.
\end{bewertung}

