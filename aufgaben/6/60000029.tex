Am 10.~Februar 2012 bewegte sich der Planet Venus von der Erde aus gesehen
in einer Entfernung von nur etwa $0.3^\circ$ am Planeten Uranus vorbei.
Da Uranus von blossem Auge kaum sichtbar ist, ist er nicht ganz einfach
zu finden.
Da die Venus sehr hell und unverwechselbar ist, konnte man an diesem
Tag Uranus sehr leicht mit einem Feldstecher oder einem kleinen Teleskop
finden.

Die Planeten befanden sich in dieser Konstellation in sogenannten
heliozentrischen Koordinaten an den Ortsvektoren
\[
\vec{p}_{\venus}
=\begin{pmatrix*}[r]
   0.279\\
   0.665\\
  -0.007
\end{pmatrix*},\quad
\vec{p}_{\uranus}
=\begin{pmatrix*}[r]
   20.023\\
    1.416\\
   -0.254
\end{pmatrix*}
\quad\text{und}\quad
\vec{p}_{\earth}
=\begin{pmatrix*}[r]
  -0.7651\\
   0.6229\\
   0.0000
\end{pmatrix*},
\]
dabei steht $\venus$ für Venus, $\uranus$ für Uranus und $\earth$ für die
Erde.

Ein Kamera auf der Erde mit einem $3600\times 2400$-Pixel Chip und einem
Teleobjektiv mit Brennweite 50000 Pixel wird auf die beiden Planeten
ausgerichtet mit der Drehmatrix
\[
D = \begin{pmatrix*}[r]
  -0.038&  0.999&  0.000\\
   0.012&  0.000&  1.000\\
   0.999&  0.038& -0.012
\end{pmatrix*}.
\]
\begin{teilaufgaben}
\item
An welchen Pixelkoordinaten befinden sich die Bilder der beiden Planeten?
\item
Wie weit auseinander sind die beiden Bilder in Pixeln?
\end{teilaufgaben}

\thema{Kamerageometrie}

\begin{loesung}
Die Kameramatrix ist
\[
K 
=
\begin{pmatrix}
f&0&w/2\\
0&f&h/2\\
0&0& 1
\end{pmatrix}
=
\begin{pmatrix}
50000 &   0   & 1800 \\
  0   & 50000 & 1200 \\
  0   &   0   &  1
\end{pmatrix}.
\]
Das Kamerazentrum ist $\vec{c} = \vec{p}_{\earth}$. 
Die Kameraprojektionsmatrix kann daraus zusammengesetzt werden, sie ist
\[
P = K D \begin{pmatrix} E & -\vec{p}_{\earth} \end{pmatrix}.
\]

Für die Berechnung der Pixelkoordinaten braucht man die homogenen Koordinaten
der Planeten $\tilde{p}_*$, die man aus $\vec{p}_*$ erhält, indem man 
den Vektor $\vec{p}_*$ um eine vierte Komponente mit Wert $1$ erweitert.
Die homogenen Koordinaten der Bilder sind dann $\tilde{b}_* = P \tilde{p}_*$.
Die Pixelkoordinaten $\vec{b}_*$ erhält man dann, indem man $\tilde{b}_*$
durch die dritte Komponente teilt.
\begin{teilaufgaben}
\item
Die Berechnung der Pixelkoordinaten liefert
\begin{align*}
\tilde{b}_{\venus}
&=
\begin{pmatrix*}[r]
   1999.6365\\
   1530.1476\\
      1.0447
\end{pmatrix*}
&&\Rightarrow&
\vec{b}_{\venus}
&=
\begin{pmatrix*}[r]
   1914.0\\
   1464.6
\end{pmatrix*}
\\
\tilde{b}_{\venus}
&=
\begin{pmatrix*}[r]
   37558.8\\
   24733.4\\
      20.8
\end{pmatrix*}
&&\Rightarrow&
\vec{b}_{\venus}
&=
\begin{pmatrix*}[r]
  1805.7\\
  1189.1
\end{pmatrix*}
\end{align*}
\item
Die Differenz der beiden Pixelvektoren ist
\[
\vec{d}
=
\vec{b}_{\venus}-\vec{b}_{\uranus}
=
\begin{pmatrix}
   108.33\\
   275.54
\end{pmatrix}
\qquad\Rightarrow\qquad
|\vec{d}|
=
\biggl|
\begin{matrix*}[r]
   108.33\\
   275.54
\end{matrix*}
\biggr|
=
296.07.
\]
Man kann daraus auch den Winkel $\delta$ zwischen den beiden Planeten
berechnen.
Es ist
\[
\tan\delta
\approx
\frac{296.07}{f}
=
0.0059215
\qquad\Rightarrow\qquad
\delta \approx 0.34^\circ.
\]
Dieser Winkel ist konsistent mit der Beschreibung in der Aufgabe.
\qedhere
\end{teilaufgaben}
\end{loesung}

\begin{bewertung}
\end{bewertung}
\begin{bewertung}
Kameramatrix ({\bf K}) 1 Punkt,
Projektionsmatrix ({\bf P}) 1 Punkt,
Methode zur Berechnung der Pixelkoordinaten ({\bf M}) 1 Punkt,
homogene Bildkoordinaten und Pixelkoordinaten \venus{} ({\bf V}) 1 Punkt,
homogene Bildkoordinaten und Pixelkoordinaten \uranus{} ({\bf U}) 1 Punkt,
Abstand \venus--\uranus{} ({\bf A}) 1 Punkt.
\end{bewertung}

