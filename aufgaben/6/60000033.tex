In einem Gang mit quadratischem Querschnitt $2\times 2$, dessen
Achse die $x$-Achse ist, ist im Punkt 
$C=(2,-1,1)$ eine Überwachungskamera mit einem Objektiv mit
Brennweite $f=2000$ und einem $1920\times 1080$-Chip montiert.
Sie ist mit der Drehmatrix
\[
D=
\begin{pmatrix*}[r]
  -0.44721&  -0.89443&  0.00000\\
  -0.36515&   0.18257&  0.91287\\
  -0.81650&   0.40825& -0.40825
\end{pmatrix*}
\]
auf den Punkt $O=(0,0,0)$ ausgerichtet. 
Eine Lichtschranke zwischen den Punkten $P_1=(0,-1,0)$
und $P_2=(0,1,0)$ soll durch eine Softwarelösung
ersetzt werden, die Bilder der Kamera verarbeitet und Bewegung
im Bereich der Lichtschranke detektiert.
\begin{teilaufgaben}
\item
Bestimmen Sie dazu die Bildpunkte $B_1$ und $B_2$ der Endpunkte
der Lichtschranke, die Koordinaten auf ganze Pixel gerundet.
\item
Der Punkt $O$ sollte auf den Mittelpunkt $M$ des Chip abgebildet werden.
Verwenden Sie dies, um Ihr Resultat wie folgt zu kontrollieren.
Bestimmen Sie den Abstand des Mittelpunktes $M$
des Chip von der Geraden durch die beiden Punkte $B_1$ und $B_2$.
\end{teilaufgaben}

\begin{loesung}
\begin{teilaufgaben}
\item
Die Kameramatrix ist
\[
K
=
\begin{pmatrix}
f&0&m_x\\
0&f&m_y\\
0&0&1
\end{pmatrix}
=
\begin{pmatrix}
2000&0&960\\
  0 &2000&540\\
0 & 0& 1
\end{pmatrix}.
\]
Das Kamerazentrum ist der Vektor
\[
\vec{c}= \begin{pmatrix}2\\-1\\1\end{pmatrix}.
\]
Damit kann man jetzt die Kameraprojektionsmatrix 
\[
P
=
K D \begin{pmatrix} I& -\vec{c}\end{pmatrix}
=
\begin{pmatrix*}[r]
-1678.3\phantom{0000}&-1396.9\phantom{0000}&-391.92\phantom{000}&  2351.5\phantom{000}\\
-1171.2\phantom{0000}&  585.60\phantom{000}&1605.3\phantom{0000}&  1322.7\phantom{000}\\
   -0.81650          &    0.40825          &  -0.40825          &     2.4495
\end{pmatrix*}
\]
berechnen.
Daraus kann man dann die harmonischen Koordinaten der Bildpunkte berechnen:
\begin{align*}
\tilde{b}_1
&=
P\begin{pmatrix}0\\-1\\0\\1\end{pmatrix}
=
\begin{pmatrix*}[r]
   3748.4\phantom{000}\\
   737.12\phantom{00}\\
   2.0412
\end{pmatrix*}
&
\tilde{b}_2
&=
P\begin{pmatrix}0\\1\\0\\1\end{pmatrix}
=
\begin{pmatrix*}[r]
   954.57\phantom{00}\\
   1908.3\phantom{000}\\
   2.8577
\end{pmatrix*}.
\intertext{Nach Division durch die letzte Koordinaten entstehen die 
Pixelkoordinaten der Bilder:}
B_1&=( 1836, 361)
&
B_2&=( 334, 668)
\end{align*}
\item
Die Abstandsformel in der Ebene verwendet den Flächeninhalt $F$ des
Parallelogramms, welches von den Vektoren
\[
\overrightarrow{B_1B_2}
=
\begin{pmatrix*}[r]
  -1502\\
    307
\end{pmatrix*}
\quad\text{und}\quad
\overrightarrow{B_1M}
=
\begin{pmatrix*}[r]
  -876\\
   179
\end{pmatrix*}
\]
aufgespannt wird.
Er ist
\[
F = 
\left|\begin{matrix*}[r]
-1502& -876\\
  307&  179
\end{matrix*}\right|
=
74.
\]
Der Abstand ist jetzt
\[
d
=
\frac{F}{\overline{B_1B_2}}
=
\frac{74}{1533.1}
=0.0483
\]
Als deutlich kleiner als ein Pixel.
\qedhere
\end{teilaufgaben}
\end{loesung}

\begin{bewertung}
Kameraprojektionsmatrix ({\bf K}) 1 Punkt,
Projektionsformel ({\bf P}) 1 Punkt,
Bildpunkte in homogenen Koordinaten ({\bf H}) 1 Punkt,
Bildpunkte in Chip-Koordinaten ({\bf B}) 1 Punkt,
Abstandsformel ({\bf F}) 1 Punkt,
Abstandswert ({\bf A}) 1 Punkt.
\end{bewertung}

