Sei $A$ die Matrix
\[
A=\begin{pmatrix}
1&3\\
4&0
\end{pmatrix}.
\]
\begin{teilaufgaben}
\item Ist $A$ diagonalisierbar?
\item Finden Sie eine Basis, in der $A$ diagonal ist.
\item Wie gross ist der Winkel zwischen den neuen Basisvektoren?
\end{teilaufgaben}

\thema{Eigenwerte}
\thema{Eigenvektoren}
\thema{charakteristisches Polynome}
\thema{diagonalisierbar}

\begin{loesung}
Das charakteristische Polynom ist
\begin{align*}
\det(A-\lambda E)
&=\left|\begin{matrix}
1-\lambda&3\\
4&-\lambda
\end{matrix}\right|
\\
&=-\lambda(1-\lambda)-12=\lambda^2-\lambda-12=(\lambda -4)(\lambda + 3)
\end{align*}
die Eigenwerte sind also $\lambda_1 = 4$ und $\lambda_2=-3$.
Für diese beiden Eigenwert müssen jetzt auch noch Eigenvektoren
gefunden werden. Für $\lambda_1=4$ findet man das Gleichungssystem
\begin{align*}
\begin{tabular}{|cc|c|}
\hline
-3&3&0\\
4&-4&0\\
\hline
\end{tabular}
&\rightarrow
\begin{tabular}{|cc|c|}
\hline
1&-1&0\\
0&0&0\\
\hline
\end{tabular}
\end{align*}
d.~h.~ein Eigenvektor muss einfach beide Komponenten gleich haben, also
zum Beispiel
\[
v_1=\begin{pmatrix}1\\1\end{pmatrix}.
\]
Für $\lambda_2=-3$ findet man entsprechend
\begin{align*}
\begin{tabular}{|cc|c|}
\hline
4&3&0\\
4&3&0\\
\hline
\end{tabular}
&\rightarrow
\begin{tabular}{|cc|c|}
\hline
1&$\frac34$&0\\
0&0&0\\
\hline
\end{tabular}
\end{align*}
also zum Beispiel
\[
v_2=\begin{pmatrix}-\frac34\\1\end{pmatrix}
\]
Zur Kontrolle berechnen wir $Av_1$ und $Av_2$:
\begin{align*}
Av_1&=\begin{pmatrix}4\\4\end{pmatrix}=4v_1=\lambda_1v_1\\
Av_2&=\begin{pmatrix}-\frac34+3\\-3\end{pmatrix}
=\begin{pmatrix}\frac94\\-3\end{pmatrix}
=-3\begin{pmatrix}-\frac34\\1\end{pmatrix}=\lambda_2v_2.
\end{align*}
Die gestellten Fragen können damit wie folgt beantwortet werden:
\begin{teilaufgaben}
\item $A$ ist diagonalisierbar, weil es eine Basis aus Eigenvektoren gibt.
\item Die Vektoren $v_1$ und $v_2$ bilden die gesuchte Basis.
\item Der Zwischenwinkel zwischen $v_1$ und $v_2$ ist
\begin{align*}
\cos\alpha&=\frac{v_1\cdot v_2}{|v_1|\,|v_2|}
=
\frac{-\frac34+1}{\sqrt{2}\frac54}=\frac{1}{5\sqrt{2}}=0.14142\\
\alpha&=81.87^\circ
\qedhere
\end{align*}
\end{teilaufgaben}
\end{loesung}

