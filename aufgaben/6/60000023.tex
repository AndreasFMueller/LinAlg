Das Mooresche Gesetz besagt, dass sich die Komplexität integrierter
Schaltkreise alle 18 Monate verdopple.
Etwas mathematischer kann man das formulieren als eine Gesetzmässigkeit,
nach der die Anzahl $N$ der
Transistoren auf einem Chip exponentiell mit der Zeit anwächst, also
\begin{equation}
N(t) = N_0e^{\lambda t}.
\label{60000023:law}
\end{equation}
Die Konstanten $N_0$ und $\lambda$ sind noch zu bestimmen.
Natürlich kann dieses Gesetz nicht exakt gelten, aber aus einer
Reihe von Zeitpunkten $t_i$, $i=1,\dots,n$, und zugehörigen $N_i$-Werten sollte
man in der Lage sein, Werte von $N_0$ und $\lambda$ zu bestimmen,
für die das Gesetz~\eqref{60000023:law} möglichst genau gilt.
Stellen Sie ein Gleichungssystem dafür auf.

\begin{hinweis}
Wenden Sie den natürlichen Logarithmus an.
\end{hinweis}

\thema{Least Squares}

\begin{loesung}
Wendet man den natürlichen Logarithmus auf die Gleichungen
\[
N_i = N_0e^{\lambda t_i}
\]
an, erhalt man die Gleichungen
\[
\log N_i = \log N_0 + \lambda t_i.
\]
Als lineare Gleichung für die Unbekannten $\color{red}\log N_0$
und $\color{red}\lambda$ geschrieben ist dies
\[
1\cdot
{\color{red}\log N_0}
+
t_i
{\color{red}\lambda}
=
\log N_i
\]
für $i=1,\dots,n$.
Dies ist ein überbestimmtes Gleichungssystem mit der folgenden Matrix $A$
und rechten Seite $b$:
\[
A=
\begin{pmatrix}
1&t_1\\
1&t_2\\
\vdots&\vdots\\
1&t_n
\end{pmatrix}
\qquad\text{und}\qquad
b=
\begin{pmatrix}
\log N_1\\
\log N_2\\
\vdots\\
\log N_n
\end{pmatrix}.
\]
Das überbestimmte Gleichungssystem kann mit der bekannten Methode auf
ein $2\times 2$-Gleichungssystem reduziert werden.
Die Matrix ist
\begin{align*}
\transpose{A}A
&=
\begin{pmatrix}
1&1&\dots&1\\
t_1&t_2&\dots&t_n
\end{pmatrix}
\begin{pmatrix}
1&t_1\\
1&t_2\\
\vdots&\vdots\\
1&t_n
\end{pmatrix}
=
\begin{pmatrix}
n&\displaystyle \sum_{k=1}^n t_k\\
\displaystyle \sum_{k=1}^n t_k&\displaystyle \sum_{k=1}^n t_k^2
\end{pmatrix}
\intertext{und die rechte Seite}
\transpose{A}b
&=
\begin{pmatrix}
1&1&\dots&1\\
t_1&t_2&\dots&t_n
\end{pmatrix}
\begin{pmatrix}\log N_1\\\log N_2\\\vdots\\\log N_n\end{pmatrix}
=
\begin{pmatrix}
\displaystyle \sum_{k=1}^n \log N_k\\
\displaystyle \sum_{k=1}^n t_k \log N_k
\end{pmatrix}.
\end{align*}
Ein solches Gleichungssystem kann mit der Cramerschen Formel sofort gelöst
werden.
Der gemeinsame Nenner ist
\[
\det(\transpose{A}A)
=
n\sum_{k=1}^n t_k^2 - \biggl(\sum_{k=1}^n t_k\biggr)^2.
\]
Einsetzen der rechten Seite liefert die Lösungen
\begin{align*}
{\color{red}\log N_0}
&=
\frac{
\displaystyle
\sum_{k=1}^n \log N_k \cdot \sum_{k=1}^n t_k^2
-
\sum_{k=1}^n t_k\log N_k \cdot \sum_{k=1}^n t_k
}{
\displaystyle
n\sum_{k=1}^n t_k^2 - \biggl(\sum_{k=1}^n t_k\biggr)^2}
\\
\text{und}
\qquad
{\color{red}\lambda}
&=
\frac{
\displaystyle
n \sum_{k=1}^n t_k\log N_k
-
\sum_{k=1}^n t_k \cdot \sum_{k=1}^n \log N_k
}{
\displaystyle
n\sum_{k=1}^n t_k^2 - \biggl(\sum_{k=1}^n t_k\biggr)^2}.
\qedhere
\end{align*}
\end{loesung}

\begin{bewertung}
Least Squares ({\bf LS}) 1 Punkt,
Datengleichungen $N_i = N_0e^{\lambda t_i}$ ({\bf D}) 1 Punkt,
Logarthmierte Gleichungen ({\bf L}) 1 Punkt,
Gleichungssystem für $\log N_0$ und $\lambda$ ({\bf G}) 1 Punkt,
Lösungsprinzip Least Squares $\transpose{A}Ax=\transpose{A}b$ ({\bf P}) 1 Punkt,
$2\times 2$-Gleichungssystem ({\bf G2}) 1 Punkt.
\end{bewertung}



