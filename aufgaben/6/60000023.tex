Die Folge
\[
0,\;
1,\;
5,\;
19,\;
65,\;
211,\;
665,\;
2059,\;
6305,\;\dots
\]
ist durch die Rekursionsformel
\[
x_{n+1}=x_{n\mathstrut}+x_{n-1}
\]
mit den Anfangswerten
\[
x_0=0,\qquad x_1=1
\]
definiert.
Finden Sie eine Formel f"ur $x_n$.

\begin{loesung}
Die Rekursionsformel kann in vektorieller Form als
\[
\begin{pmatrix}x_{n+1}\\x_{n\mathstrut}\end{pmatrix}
=
\underbrace{\begin{pmatrix}
5&-6\\
1& 0
\end{pmatrix}}_{=A}
\begin{pmatrix}x_{n\mathstrut}\\x_{n-1}\end{pmatrix}
\]
geschrieben werden.
Die Matrix $A$ hat das charakteristische Polynom
\begin{align*}
\left|\begin{matrix}
5-\lambda&-6      \\
     1   &-\lambda
\end{matrix}\right|
&=
-(5-\lambda)\lambda+6
=\lambda^2-5\lambda+6=(\lambda-2)(\lambda -3)
\end{align*}
mit den Nullstellen $2$ und $3$.
Die Eigenvektoren finden wir wie folgt. F"ur $\lambda=2$:
\[
\begin{tabular}{|>{$}c<{$}>{$}c<{$}|}
\hline
5-\lambda&-6      \\
   1     &-\lambda\\
\hline
\end{tabular}
=
\begin{tabular}{|>{$}c<{$}>{$}c<{$}|}
\hline
3&-6\\
1&-2\\
\hline
\end{tabular}
\rightarrow
\begin{tabular}{|>{$}c<{$}>{$}c<{$}|}
\hline
1&-2\\
0& 0\\
\hline
\end{tabular}
\qquad\Rightarrow\qquad
v_2=\begin{pmatrix}2\\1\end{pmatrix}
\]
F"ur $\lambda=3$:
\[
\begin{tabular}{|>{$}c<{$}>{$}c<{$}|}
\hline
5-\lambda&-6      \\
   1     &-\lambda\\
\hline
\end{tabular}
=
\begin{tabular}{|>{$}c<{$}>{$}c<{$}|}
\hline
2&-6\\
1&-3\\
\hline
\end{tabular}
\rightarrow
\begin{tabular}{|>{$}c<{$}>{$}c<{$}|}
\hline
1&-3\\
0& 0\\
\hline
\end{tabular}
\qquad\Rightarrow\qquad
v_3=\begin{pmatrix}3\\1\end{pmatrix}
\]
Den Startvektor muss man aus den beiden Eigenvektoren linear kombinieren,
dies geschieht mit dem linearen Gleichungssystem
\begin{align*}
\begin{tabular}{|>{$}c<{$}>{$}c<{$}|>{$}c<{$}|}
\hline
2&3&1\\
1&1&0\\
\hline
\end{tabular}
&
\rightarrow
\begin{tabular}{|>{$}c<{$}>{$}c<{$}|>{$}c<{$}|}
\hline
1& \frac32& \frac12\\
0&-\frac12&-\frac12\\
\hline
\end{tabular}
\rightarrow
\begin{tabular}{|>{$}c<{$}>{$}c<{$}|>{$}c<{$}|}
\hline
1& 0&-1\\
0& 1& 1\\
\hline
\end{tabular}
\end{align*}
Damit kann man die L"osung aufschreiben:
\[
x_n=f(n)=-2^n+3^n.
\]
Wir kontrollieren das Resultat durch Ausrechnen einiger Werte:
\[
\begin{tabular}{l|rrrrrrrrr}
$n$&0&1&2&3&4&5&6&7&8\\
\hline
$x_n$&0&1&5&19&65&211&665&2059&6305\\
$3^n-2^n$&0&1&5&19&65&211&665&2059&6305\\
\end{tabular}
\]
\end{loesung}

