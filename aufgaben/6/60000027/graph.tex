%
% graph.tex -- Resonaz Beispiel
%
% (c) 2020 Prof Dr Andreas Müller, Hochschule Rapperswil
%
\documentclass[tikz]{standalone}
\usepackage{amsmath}
\usepackage{times}
\usepackage{txfonts}
\usepackage{pgfplots}
\usepackage{csvsimple}
\usetikzlibrary{arrows,intersections,math}
\begin{document}
\def\skala{6}
\begin{tikzpicture}[>=latex,thick,scale=\skala]
\definecolor{darkgreen}{rgb}{0,0.6,0}

\def\omeganull{1.00000}
\def\Q{2.00000}
\def\pfad{
	(0,0.45000)
	--(0.02000,0.45016)
	--(0.04000,0.45063)
	--(0.06000,0.45142)
	--(0.08000,0.45253)
	--(0.10000,0.45397)
	--(0.12000,0.45573)
	--(0.14000,0.45783)
	--(0.16000,0.46027)
	--(0.18000,0.46307)
	--(0.20000,0.46623)
	--(0.22000,0.46976)
	--(0.24000,0.47368)
	--(0.26000,0.47800)
	--(0.28000,0.48274)
	--(0.30000,0.48792)
	--(0.32000,0.49356)
	--(0.34000,0.49967)
	--(0.36000,0.50629)
	--(0.38000,0.51344)
	--(0.40000,0.52115)
	--(0.42000,0.52944)
	--(0.44000,0.53836)
	--(0.46000,0.54794)
	--(0.48000,0.55821)
	--(0.50000,0.56921)
	--(0.52000,0.58099)
	--(0.54000,0.59358)
	--(0.56000,0.60703)
	--(0.58000,0.62138)
	--(0.60000,0.63665)
	--(0.62000,0.65289)
	--(0.64000,0.67010)
	--(0.66000,0.68829)
	--(0.68000,0.70744)
	--(0.70000,0.72751)
	--(0.72000,0.74840)
	--(0.74000,0.76997)
	--(0.76000,0.79201)
	--(0.78000,0.81422)
	--(0.80000,0.83621)
	--(0.82000,0.85746)
	--(0.84000,0.87736)
	--(0.86000,0.89516)
	--(0.88000,0.91008)
	--(0.90000,0.92125)
	--(0.92000,0.92790)
	--(0.94000,0.92937)
	--(0.96000,0.92524)
	--(0.98000,0.91538)
	--(1.00000,0.90000)
	--(1.02000,0.87960)
	--(1.04000,0.85492)
	--(1.06000,0.82687)
	--(1.08000,0.79638)
	--(1.10000,0.76436)
	--(1.12000,0.73162)
	--(1.14000,0.69882)
	--(1.16000,0.66651)
	--(1.18000,0.63508)
	--(1.20000,0.60480)
	--(1.22000,0.57587)
	--(1.24000,0.54837)
	--(1.26000,0.52235)
	--(1.28000,0.49781)
	--(1.30000,0.47471)
	--(1.32000,0.45301)
	--(1.34000,0.43264)
	--(1.36000,0.41352)
	--(1.38000,0.39558)
	--(1.40000,0.37875)
	--(1.42000,0.36295)
	--(1.44000,0.34811)
	--(1.46000,0.33417)
	--(1.48000,0.32105)
	--(1.50000,0.30870)
	--(1.52000,0.29706)
	--(1.54000,0.28609)
	--(1.56000,0.27573)
	--(1.58000,0.26594)
	--(1.60000,0.25668)
	--(1.62000,0.24791)
	--(1.64000,0.23961)
	--(1.66000,0.23173)
	--(1.68000,0.22425)
	--(1.70000,0.21715)
	--(1.72000,0.21039)
	--(1.74000,0.20396)
	--(1.76000,0.19783)
	--(1.78000,0.19198)
	--(1.80000,0.18641)
	--(1.82000,0.18109)
	--(1.84000,0.17600)
	--(1.86000,0.17113)
	--(1.88000,0.16648)
	--(1.90000,0.16202)
	--(1.92000,0.15774)
	--(1.94000,0.15364)
	--(1.96000,0.14971)
	--(1.98000,0.14593)
	--(2.00000,0.14230)
}
\def\punkte{
	\punkt{0.02000}{0.45522}
	\punkt{0.04000}{0.44612}
	\punkt{0.06000}{0.45121}
	\punkt{0.08000}{0.47075}
	\punkt{0.10000}{0.46390}
	\punkt{0.12000}{0.44784}
	\punkt{0.14000}{0.45412}
	\punkt{0.16000}{0.44733}
	\punkt{0.18000}{0.47466}
	\punkt{0.20000}{0.48307}
	\punkt{0.22000}{0.48600}
	\punkt{0.24000}{0.45992}
	\punkt{0.26000}{0.48585}
	\punkt{0.28000}{0.47203}
	\punkt{0.30000}{0.50331}
	\punkt{0.32000}{0.47910}
	\punkt{0.34000}{0.51696}
	\punkt{0.36000}{0.48707}
	\punkt{0.38000}{0.51755}
	\punkt{0.40000}{0.53691}
	\punkt{0.42000}{0.54227}
	\punkt{0.44000}{0.53852}
	\punkt{0.46000}{0.54583}
	\punkt{0.48000}{0.56373}
	\punkt{0.50000}{0.55125}
	\punkt{0.52000}{0.59938}
	\punkt{0.54000}{0.57806}
	\punkt{0.56000}{0.60880}
	\punkt{0.58000}{0.62260}
	\punkt{0.60000}{0.64253}
	\punkt{0.62000}{0.66049}
	\punkt{0.64000}{0.68800}
	\punkt{0.66000}{0.67892}
	\punkt{0.68000}{0.70031}
	\punkt{0.70000}{0.73299}
	\punkt{0.72000}{0.74472}
	\punkt{0.74000}{0.77333}
	\punkt{0.76000}{0.77510}
	\punkt{0.78000}{0.81634}
	\punkt{0.80000}{0.83546}
	\punkt{0.82000}{0.83994}
	\punkt{0.84000}{0.86149}
	\punkt{0.86000}{0.91142}
	\punkt{0.88000}{0.89265}
	\punkt{0.90000}{0.91414}
	\punkt{0.92000}{0.94046}
	\punkt{0.94000}{0.92383}
	\punkt{0.96000}{0.92860}
	\punkt{0.98000}{0.91583}
	\punkt{1.00000}{0.88399}
	\punkt{1.02000}{0.86754}
	\punkt{1.04000}{0.85712}
	\punkt{1.06000}{0.82378}
	\punkt{1.08000}{0.80502}
	\punkt{1.10000}{0.76631}
	\punkt{1.12000}{0.72394}
	\punkt{1.14000}{0.70710}
	\punkt{1.16000}{0.67245}
	\punkt{1.18000}{0.62892}
	\punkt{1.20000}{0.60855}
	\punkt{1.22000}{0.58331}
	\punkt{1.24000}{0.56703}
	\punkt{1.26000}{0.51670}
	\punkt{1.28000}{0.49448}
	\punkt{1.30000}{0.48622}
	\punkt{1.32000}{0.47018}
	\punkt{1.34000}{0.43845}
	\punkt{1.36000}{0.39594}
	\punkt{1.38000}{0.40380}
	\punkt{1.40000}{0.38738}
	\punkt{1.42000}{0.35371}
	\punkt{1.44000}{0.33528}
	\punkt{1.46000}{0.34486}
	\punkt{1.48000}{0.33442}
	\punkt{1.50000}{0.30258}
	\punkt{1.52000}{0.30757}
	\punkt{1.54000}{0.28553}
	\punkt{1.56000}{0.26057}
	\punkt{1.58000}{0.26444}
	\punkt{1.60000}{0.26776}
	\punkt{1.62000}{0.26011}
	\punkt{1.64000}{0.23950}
	\punkt{1.66000}{0.24833}
	\punkt{1.68000}{0.23258}
	\punkt{1.70000}{0.22070}
	\punkt{1.72000}{0.21442}
	\punkt{1.74000}{0.18772}
	\punkt{1.76000}{0.18728}
	\punkt{1.78000}{0.18510}
	\punkt{1.80000}{0.16889}
	\punkt{1.82000}{0.16265}
	\punkt{1.84000}{0.18249}
	\punkt{1.86000}{0.18619}
	\punkt{1.88000}{0.15446}
	\punkt{1.90000}{0.18086}
	\punkt{1.92000}{0.14232}
	\punkt{1.94000}{0.14222}
	\punkt{1.96000}{0.15963}
	\punkt{1.98000}{0.15272}
	\punkt{2.00000}{0.15242}
}
\def\pfadneu{
	(0,0.45016)
	--(0.04007,0.45062)
	--(0.06010,0.45140)
	--(0.08014,0.45250)
	--(0.10017,0.45391)
	--(0.12021,0.45565)
	--(0.14024,0.45771)
	--(0.16027,0.46012)
	--(0.18031,0.46287)
	--(0.20034,0.46597)
	--(0.22038,0.46945)
	--(0.24041,0.47330)
	--(0.26045,0.47754)
	--(0.28048,0.48219)
	--(0.30051,0.48727)
	--(0.32055,0.49279)
	--(0.34058,0.49878)
	--(0.36062,0.50525)
	--(0.38065,0.51223)
	--(0.40068,0.51974)
	--(0.42072,0.52782)
	--(0.44075,0.53649)
	--(0.46079,0.54578)
	--(0.48082,0.55573)
	--(0.50086,0.56636)
	--(0.52089,0.57771)
	--(0.54092,0.58982)
	--(0.56096,0.60271)
	--(0.58099,0.61641)
	--(0.60103,0.63094)
	--(0.62106,0.64632)
	--(0.64110,0.66255)
	--(0.66113,0.67960)
	--(0.68116,0.69746)
	--(0.70120,0.71603)
	--(0.72123,0.73522)
	--(0.74127,0.75485)
	--(0.76130,0.77471)
	--(0.78134,0.79449)
	--(0.80137,0.81382)
	--(0.82140,0.83220)
	--(0.84144,0.84908)
	--(0.86147,0.86383)
	--(0.88151,0.87574)
	--(0.90154,0.88415)
	--(0.92157,0.88842)
	--(0.94161,0.88809)
	--(0.96164,0.88285)
	--(0.98168,0.87266)
	--(1.00171,0.85774)
	--(1.02175,0.83853)
	--(1.04178,0.81568)
	--(1.06181,0.78991)
	--(1.08185,0.76202)
	--(1.10188,0.73276)
	--(1.12192,0.70280)
	--(1.14195,0.67272)
	--(1.16199,0.64299)
	--(1.18202,0.61395)
	--(1.20205,0.58588)
	--(1.22209,0.55893)
	--(1.24212,0.53322)
	--(1.26216,0.50879)
	--(1.28219,0.48567)
	--(1.30223,0.46383)
	--(1.32226,0.44324)
	--(1.34229,0.42385)
	--(1.36233,0.40560)
	--(1.38236,0.38844)
	--(1.40240,0.37229)
	--(1.42243,0.35709)
	--(1.44246,0.34279)
	--(1.46250,0.32932)
	--(1.48253,0.31662)
	--(1.50257,0.30465)
	--(1.52260,0.29336)
	--(1.54264,0.28269)
	--(1.56267,0.27260)
	--(1.58270,0.26306)
	--(1.60274,0.25402)
	--(1.62277,0.24546)
	--(1.64281,0.23733)
	--(1.66284,0.22962)
	--(1.68288,0.22229)
	--(1.70291,0.21532)
	--(1.72294,0.20869)
	--(1.74298,0.20237)
	--(1.76301,0.19635)
	--(1.78305,0.19060)
	--(1.80308,0.18511)
	--(1.82312,0.17987)
	--(1.84315,0.17486)
	--(1.86318,0.17006)
	--(1.88322,0.16547)
	--(1.90325,0.16107)
	--(1.92329,0.15685)
	--(1.94332,0.15280)
	--(1.96335,0.14891)
	--(1.98339,0.14518)
	--(2.00342,0.14159)
}
\def\omeganeu{1.00171}
\def\Qneu{1.90609}


\def\punkt#1#2{
\fill[color=red] (#1,#2) circle[radius={0.05/\skala}];
}

\draw[color=darkgreen] (\omeganull,{-0.1/\skala}) -- (\omeganull,1);

\node[color=darkgreen] at (\omeganull,{-0.1/\skala})
	[below] {$\mathstrut\omega_0$};

\punkte

\draw[color=darkgreen,line width=1.0pt] \pfad;

\draw[->] ({-0.1/\skala},0) -- ({2+0.3/\skala},0) coordinate[label={$\omega$}];
\draw[->] (0,{-0.1/\skala},0) -- (0,{1+0.3/\skala}) coordinate[label={right:$W(\omega)$}];

% add image content here

\end{tikzpicture}
\end{document}

