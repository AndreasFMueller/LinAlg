Betrachten Sie die Matrix
\[
A(a)
=
\begin{pmatrix}
 3& a& 1\\
 a&-3& a\\
 1& a& 0
\end{pmatrix}.
\]
\begin{teilaufgaben}
\item
Berechnen Sie Spur und Determinante von $A(a)$.
\item 
F"ur welche Werte von $a$ ist $A(a)$ regul"ar?
\item
F"ur welche Werte des Parameters $a$ hat $A(a)^{-1}$
genau einen negativen Eigenwert?
\end{teilaufgaben}

\begin{loesung}
\begin{teilaufgaben}
\item Spur und Determinante sind
\begin{align}
\operatorname{Spur} A(a)
&=
0,
\label{60000027:tr}
\\
\det A(a)
&=
0 + a^2 + a^2 + 3 - 3 a^2 + 0
=
-a^2+3.
\label{60000027:det}
\end{align}
\item
Damit $A(a)$ regul"ar ist, muss $\det A(a)$
von $0$ verschieden sein.
Die Determinante verschwindet f"ur
\[
\det A(a)
=
-a^2+3
=
0
\qquad
\Rightarrow
\qquad
a=\pm\sqrt{3}.
\]
Die Matrix $A(a)$ ist also f"ur $a\ne\pm\sqrt{3}$ regul"ar.
\item
Die Matrix $A(a)^{-1}$ existiert "uberhaupt nur dann, wenn $A(a)$
regul"ar ist, und wir haben bereits in Teilaufgabe b) gesehen, dass
das f"ur $a\ne\pm\sqrt{3}$ der Fall ist.

Da die Matrix symmetrisch und damit diagonalisierbar ist, sind alle
Eigenwerte rell, und die Eigenwerte von $A(a)^{-1}$ sind die
Kehrwerte der Eigenwerte von $A(a)$, haben also insbesondere
das gleiche Vorzeichen.
Statt der Eigenwerte von $A(a)^{-1}$ k"onnen wir also die Eigenwerte
von $A(a)$ diskutieren.

Die Determinante von $A(a)$ ist das Produkt der Eigenwerte, die Spur
von $A(a)$ ist deren Summe.
Da die Determinante nicht verschwindet, sind alle Eigenwerte von $0$
verschieden.

W"aren alle Eigenwerte positiv, m"usste die Spur positiv sein,
w"aren sie negativ, m"ussten die Spur negativ sein.
Beides widerspricht der Beobachtung in Teilaufgabe a), dass 
$\operatorname{Spur}A(a)=0$ ist.
Die Eigenwerte haben also nicht alle das gleiche Vorzeichen.
Damit bleiben nur zwei Situation: zwei positive und ein negativer
Eigenwert oder umgekehrt.

Zwei negative Eigenwerte f"uhren auf eine positive Determinante,
zwei positive Eigenwerte und ein negativer Eigenwert f"uhren auf
eine negative Determinaten.
Der gesuchte Fall genau eines negativen Eigenwertes ist also derjenige,
wo $\det A(a) <0$ ist.
Die Formel \eqref{60000027:det} f"ur die Determinante zeigt aber:
\[
\det A(a) = -a^2 + 3 < 0
\qquad\Rightarrow\qquad
3 < a^2
\qquad\Rightarrow\qquad
a >\sqrt{3}\quad\vee\quad a < -\sqrt{3}.
\]
Wir finden, dass $A(a)^{-1}$ genau zwei negative Eigenwerte f"ur $a$ im Interval
$(-\sqrt{3},\sqrt{3})$ hat, und genau einen negativen Eigenwert f"ur
$a$ ausserhalb des Intervals $[-\sqrt{3},\sqrt{3}]$.
\qedhere
\end{teilaufgaben}
\end{loesung}

\begin{diskussion}
Im Grenzfall verschwindender Determinante ist noch nicht vollst"andig
gekl"art, was mit den Vorzeichen der Eigenwerte passiert.
Wenn wir davon ausgehen, dass die Eigenwerte stetig mit $a$ varieren,
dann "andert beim "Ubergang genau eine der Eigenwerte das Vorzeichen,
so dass wir an den Stellen $a=\pm\sqrt{3}$ genau eine positive und
eine negative Eigenwerte haben.
Die einzige Alternative w"are, dass alle Eigenwerte verschwinden,
dann w"are die Matrix aber die Nullmatrix, und das ist offensichtlich
nicht der Fall.

Man kann f"ur den Fall $a=\pm\sqrt{3}$ auch das charakteristische Polynom
zum Beispiel mit einem CAS berechnen, man bekommt
\[
\chi_{A(a)}(\lambda)
=
-\lambda^3+(2a^2+10)\lambda-a^2+3.
\]
Setzt man darin $a^2=3$, vereinfacht sich die Eigenwertgleichung auf
\[
-\lambda^3 +16\lambda
=
-\lambda(\lambda^2 -16)=0
\qquad\Rightarrow\qquad \lambda = 0,\pm 4,
\]
was die obige Diskussion best"atigt.
\end{diskussion}

\begin{bewertung}
Spur ({\bf T}) 1 Punkt,
Determinante ({\bf D}) 1 Punkt,
Berechnung der kritischen Werte $a=\pm\sqrt{3}$ ({\bf K}) 1 Punkt,
Diagonalisierbarkeit wegen Symmetrie ({\bf S}) 1 Punkt,
Spur $0$ bedeutet, dass nicht alle Eigenwerte das gleiche Vorzeichen
haben k"onnen ({\bf V}) 1 Punkt,
negative Determinante bedeute genau ein negativer Eigenwert ({\bf N}) 1 Punkt,
Bestimmung der zul"assigen Werte $a$ f"ur negative Determinante ({\bf A})
1 Punkt, jedoch maximal 6 Punkte.
\end{bewertung}

