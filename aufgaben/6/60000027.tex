Betrachten Sie die Matrix
\[
A(a)
=
\begin{pmatrix}
 3& a& 1\\
 a&-3& a\\
 1& a& 0
\end{pmatrix}.
\]
\begin{teilaufgaben}
\item
Berechnen Sie Spur und Determinante von $A(a)$.
\item 
Für welche Werte von $a$ ist $A(a)$ regulär?
\item
Für welche Werte des Parameters $a$ hat $A(a)^{-1}$
genau einen negativen Eigenwert?
\end{teilaufgaben}

\thema{Spur}
\thema{Determinate}
\thema{singulär}
\thema{regulär}
\thema{charakteristisches Polynom}

\begin{loesung}
\begin{teilaufgaben}
\item Spur und Determinante sind
\begin{align}
\operatorname{Spur} A(a)
&=
0,
\label{60000027:tr}
\\
\det A(a)
&=
0 + a^2 + a^2 + 3 - 3 a^2 + 0
=
-a^2+3.
\label{60000027:det}
\end{align}
\item
Damit $A(a)$ regulär ist, muss $\det A(a)$
von $0$ verschieden sein.
Die Determinante verschwindet für
\[
\det A(a)
=
-a^2+3
=
0
\qquad
\Rightarrow
\qquad
a=\pm\sqrt{3}.
\]
Die Matrix $A(a)$ ist also für $a\ne\pm\sqrt{3}$ regulär.
\item
Die Matrix $A(a)^{-1}$ existiert überhaupt nur dann, wenn $A(a)$
regulär ist, und wir haben bereits in Teilaufgabe b) gesehen, dass
das für $a\ne\pm\sqrt{3}$ der Fall ist.

Da die Matrix symmetrisch und damit diagonalisierbar ist, sind alle
Eigenwerte rell, und die Eigenwerte von $A(a)^{-1}$ sind die
Kehrwerte der Eigenwerte von $A(a)$, haben also insbesondere
das gleiche Vorzeichen.
Statt der Eigenwerte von $A(a)^{-1}$ können wir also die Eigenwerte
von $A(a)$ diskutieren.

Die Determinante von $A(a)$ ist das Produkt der Eigenwerte, die Spur
von $A(a)$ ist deren Summe.
Da die Determinante nicht verschwindet, sind alle Eigenwerte von $0$
verschieden.

Wären alle Eigenwerte positiv, müsste die Spur positiv sein,
wären sie negativ, müssten die Spur negativ sein.
Beides widerspricht der Beobachtung in Teilaufgabe a), dass 
$\operatorname{Spur}A(a)=0$ ist.
Die Eigenwerte haben also nicht alle das gleiche Vorzeichen.
Damit bleiben nur zwei Situation: zwei positive und ein negativer
Eigenwert oder umgekehrt.

Zwei negative Eigenwerte führen auf eine positive Determinante,
zwei positive Eigenwerte und ein negativer Eigenwert führen auf
eine negative Determinaten.
Der gesuchte Fall genau eines negativen Eigenwertes ist also derjenige,
wo $\det A(a) <0$ ist.
Die Formel \eqref{60000027:det} für die Determinante zeigt aber:
\[
\det A(a) = -a^2 + 3 < 0
\qquad\Rightarrow\qquad
3 < a^2
\qquad\Rightarrow\qquad
a >\sqrt{3}\quad\vee\quad a < -\sqrt{3}.
\]
Wir finden, dass $A(a)^{-1}$ genau zwei negative Eigenwerte für $a$ im Interval
$(-\sqrt{3},\sqrt{3})$ hat, und genau einen negativen Eigenwert für
$a$ ausserhalb des Intervals $[-\sqrt{3},\sqrt{3}]$.
\qedhere
\end{teilaufgaben}
\end{loesung}

\begin{diskussion}
Im Grenzfall verschwindender Determinante ist noch nicht vollständig
geklärt, was mit den Vorzeichen der Eigenwerte passiert.
Wenn wir davon ausgehen, dass die Eigenwerte stetig mit $a$ varieren,
dann ändert beim "Ubergang genau eine der Eigenwerte das Vorzeichen,
so dass wir an den Stellen $a=\pm\sqrt{3}$ genau eine positive und
eine negative Eigenwerte haben.
Die einzige Alternative wäre, dass alle Eigenwerte verschwinden,
dann wäre die Matrix aber die Nullmatrix, und das ist offensichtlich
nicht der Fall.

Man kann für den Fall $a=\pm\sqrt{3}$ auch das charakteristische Polynom
zum Beispiel mit einem CAS berechnen, man bekommt
\[
\chi_{A(a)}(\lambda)
=
-\lambda^3+(2a^2+10)\lambda-a^2+3.
\]
Setzt man darin $a^2=3$, vereinfacht sich die Eigenwertgleichung auf
\[
-\lambda^3 +16\lambda
=
-\lambda(\lambda^2 -16)=0
\qquad\Rightarrow\qquad \lambda = 0,\pm 4,
\]
was die obige Diskussion bestätigt.
\end{diskussion}

\begin{bewertung}
Spur ({\bf T}) 1 Punkt,
Determinante ({\bf D}) 1 Punkt,
Berechnung der kritischen Werte $a=\pm\sqrt{3}$ ({\bf K}) 1 Punkt,
Diagonalisierbarkeit wegen Symmetrie ({\bf S}) 1 Punkt,
Spur $0$ bedeutet, dass nicht alle Eigenwerte das gleiche Vorzeichen
haben können ({\bf V}) 1 Punkt,
negative Determinante bedeute genau ein negativer Eigenwert ({\bf N}) 1 Punkt,
Bestimmung der zulässigen Werte $a$ für negative Determinante ({\bf A})
1 Punkt, jedoch maximal 6 Punkte.
\end{bewertung}

