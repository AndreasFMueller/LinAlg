Ein gedämpftes schwingungsfähiges System wird von einem Signal
$W(0)\sin\omega t$
konstanter Amplitude $W(0)$ mit Kreisfrequenz $\omega$ angeregt.
Dabei stellt sich nach einiger Zeit eine Schwingung mit Amplitude $W(\omega)$
ein.
\begin{center}
\includeagraphics[]{graph.pdf}
\end{center}
Aus der Theorie ist bekannt, dass die Frequenzabhängigkeit von $W$
gegeben ist durch
\begin{equation}
\frac{W(\omega)}{W(0)}
=
\frac{1}{\sqrt{\biggl(1-\biggl(\displaystyle\frac{\omega}{\omega_0}\biggr)^2\biggr)^2
+
\biggl(\displaystyle\frac{\omega}{Q\omega_0}\biggr)^2}}.
\end{equation}
Man kann dies nach Umformung auch als etwas weniger komplizierte Gleichung
\begin{equation}
\begin{aligned}
\biggl(
\frac{W(0)}{W(\omega)}
\biggr)^2
&=
1
-
2 \biggl(\frac{\omega}{\omega_0}\biggr)^2
+
\biggl(\frac{\omega}{\omega_0}\biggr)^4
+
\biggl(\frac{\omega}{Q\omega_0}\biggr)^2
\\
&=
1
+
\biggl(
\frac{1}{Q^2}
-
2\biggr)\biggl(\frac{\omega}{\omega_0}\biggr)^2
+
\biggl(\frac{\omega}{\omega_0}\biggr)^4
\\
&=
1
+
\omega^2
\biggl(\biggl(\frac{1}{Q^2}-2\biggr)\frac{1}{\omega_0^2}\biggr)
+
\omega^4
\biggl(\frac{1}{\omega_0^4}\biggr)
\end{aligned}
\label{60000027:eqn}
\end{equation}
schreiben.
Die interessanten Grössen in dieser Gleichung sind $\omega_0$, die 
Eigenfrequenz des Systems, und $Q$, die sogenannte Güte des Systems,
sie ist umso grösser je weniger Energie in jeder Schwingung verloren geht.
Je grösser $Q$, desto schmaler ist der Peak in der Abbildung.
Insbesondere bei kleinem $Q$ ist es also schwierig, die Frequenz der
maximalen Amplitude, die Resonanzfrequenz, exakt zu bestimmen.

Gegeben ist jetzt eine Messreihe von $n$ Messwerte $\omega_i$ für die
Anregungsfrequenz und $W_i$ für die resultierende Amplitude.
Wie kann man daraus bestmögliche Werte für $\omega_0$ und $Q$ berechnen?

\begin{hinweis}
Verwenden Sie geeignete Unbekannte anstelle von $\omega_0$ und $Q$,
um lineare Gleichungen zu erhalten.
\end{hinweis}

\thema{Least Squares}

\begin{loesung}
Wir verwenden die Unbekannten 
\[
{\color{red}x}=\frac{1}{\omega_0^2}\biggl(\frac{1}{Q^2}-2 \biggr),
\qquad\text{und}\qquad
{\color{red}y}=\frac{1}{\omega_0^4}.
\]
Wir erhalten dann aus \eqref{60000027:eqn} die Gleichungen
\begin{equation}
\frac{W(0)}{W_i}-1
=
\omega_i^2 {\color{red}x}
+
\omega_i^4 {\color{red}y}
\end{equation}
für $1\le i\le n$.
Dies ist ein überbestimmtes lineares Gleichungssystem mit Matrix
\[
A=\begin{pmatrix}
\omega_1^2&\omega_1^4\\
\omega_2^2&\omega_2^4\\
\vdots   &\vdots    \\
\omega_n^2&\omega_n^4
\end{pmatrix}
\quad\text{und rechter Seite}\quad
b=\begin{pmatrix}
\frac{W(0)}{W_1}-1\\
\frac{W(0)}{W_2}-1\\
\vdots\\
\frac{W(0)}{W_n}-1
\end{pmatrix}.
\]
Es kann mit der Least-Squares-Method gelöst werden.
Die Lösung kann gefunden werden als Lösung des Gleichungssystems
$\transpose{A}Av=\transpose{A}b$.
\[
\transpose{A}A
=
\begin{pmatrix}
\displaystyle\sum_{i=1}^n \omega_i^4 & \displaystyle\sum_{i=1}^n \omega_i^6 \\
\displaystyle\sum_{i=1}^n \omega_i^6 & \displaystyle\sum_{i=1}^n \omega_i^8
\end{pmatrix}
\quad\text{und}\quad
b
=
\begin{pmatrix}
\displaystyle\sum_{i=1}^n \omega_i^2 \biggl(\frac{W(0)}{W_i}-1\biggr) \\
\displaystyle\sum_{i=1}^n \omega_i^4 \biggl(\frac{W(0)}{W_i}-1\biggr)
\end{pmatrix}
\]
Aus der Lösung dieses linearen Gleichungsystems kann man dann 
Schätzungen $\hat{\omega}_0$ für $\omega_0$ und $\hat{Q}$ für $Q$
aus den Formeln
\[
\hat{\omega}_0
=
1/\sqrt[4]{y}
\qquad\text{und}\qquad
\hat{Q}
=
\frac1{\displaystyle\sqrt{2+\frac{x}{\sqrt{y}}}}
\]
erhalten.
\end{loesung}

\begin{bewertung}
Least Squares ({\bf L}) 1 Punkt,
Wahl der Variablen ({\bf V}) 1 Punkt,
Matrix $A$ ({\bf A}) 2 Punkt,
rechte Seite $b$ ({\bf b}) 1 Punkt,
Lösungsmethode $\transpose{A}Av=\transpose{A}b$ ({\bf M}) 1 Punkt.
\end{bewertung}

\begin{diskussion}
Man kann diese Lösung natürlich auch simulieren.
Die Abbildung zeigt in {\color{darkgreen}grün} den exakten Graphen der
Funktion $W(\omega)$ und in {\color{red}rot} die simulierten Messwerte.
In {\color{blue}blau} ist der Graph dargestellt, der zur oben
dargestellten Lösung gefunden wird.
\begin{center}
\includeagraphics[]{vergleich.pdf}
\end{center}
Man kann sehen, dass die Güte $\hat{Q}$, die wesentlich die Höhe des
Maximums von $W(\omega)$ bestimmt relativ schlecht bestimmt wird, aber der
Wert von $\omega_0$ ist sehr genau.
\end{diskussion}

