Betrachten Sie die Matrix
\[
A=
\begin{pmatrix}
\frac1{\sqrt{3}}&\frac1{\sqrt{3}}&\frac1{\sqrt{3}}\\
0&-\frac1{\sqrt{2}}&\frac1{\sqrt{2}}\\
\frac2{\sqrt{6}}&-\frac1{\sqrt{6}}&-\frac1{\sqrt{6}}
\end{pmatrix}
\]
\begin{teilaufgaben}
\item
Ist $A$ orthogonal?
\item
Bestimmen Sie die Determinante von $A$.
\item
Ist $A$ eine Drehmatrix? Wenn ja, bestimmen Sie den Drehwinkel?
\end{teilaufgaben}

\thema{orthogonale Matrix}
\thema{Drehmatrix}
\thema{Drehwinkel}

\begin{loesung}
\begin{teilaufgaben}
\item
Um nachzuprüfen, dass die Matrix orthogonal ist, muss das Produkt
$\transpose{A}A$ berechnet werden:
\begin{align*}
\transpose{A}A
&=
\transpose{
\begin{pmatrix}
\frac1{\sqrt{3}}&\frac1{\sqrt{3}}&\frac1{\sqrt{3}}\\
0&-\frac1{\sqrt{2}}&\frac1{\sqrt{2}}\\
\frac2{\sqrt{6}}&-\frac1{\sqrt{6}}&-\frac1{\sqrt{6}}
\end{pmatrix}}
\begin{pmatrix}
\frac1{\sqrt{3}}&\frac1{\sqrt{3}}&\frac1{\sqrt{3}}\\
0&-\frac1{\sqrt{2}}&\frac1{\sqrt{2}}\\
\frac2{\sqrt{6}}&-\frac1{\sqrt{6}}&-\frac1{\sqrt{6}}
\end{pmatrix}
\\
&=
\begin{pmatrix}
\frac1{\sqrt{3}}&0&\frac2{\sqrt{6}}\\
\frac1{\sqrt{3}}&-\frac1{\sqrt{2}}&-\frac1{\sqrt{6}}\\
\frac1{\sqrt{3}}&\frac1{\sqrt{2}}&-\frac1{\sqrt{6}}
\end{pmatrix}
\begin{pmatrix}
\frac1{\sqrt{3}}&\frac1{\sqrt{3}}&\frac1{\sqrt{3}}\\
0&-\frac1{\sqrt{2}}&\frac1{\sqrt{2}}\\
\frac2{\sqrt{6}}&-\frac1{\sqrt{6}}&-\frac1{\sqrt{6}}
\end{pmatrix}
\\
&=
\begin{pmatrix}
\frac13+\frac46 & \frac13-\frac26         & \frac12 -\frac26       \\
\frac13-\frac26 & \frac13+\frac12+\frac16 & \frac13-\frac12+\frac16\\
\frac13-\frac26 & \frac13-\frac12+\frac16 & \frac13+\frac12+\frac16
\end{pmatrix}
\\
&=\begin{pmatrix}
1&0&0\\
0&1&0\\
0&0&1
\end{pmatrix}
=E
\end{align*}
Somit ist $\transpose{A}A=E$, $A$ ist also orthogonal.
\item
Eine orthogonale Matrix kann nur die Determinante $1$ oder $-1$ haben.
Sie hat Determinante $1$, wenn der dritte Vektor das Vektorprodukt der
ersten zwei ist, also
\begin{align*}
\begin{pmatrix}
\frac1{\sqrt{3}}\\0\\\frac{2}{\sqrt{6}}
\end{pmatrix}\times\begin{pmatrix}
\frac1{\sqrt{3}}\\-\frac1{\sqrt{2}}\\-\frac1{\sqrt{6}}
\end{pmatrix}
&=
\begin{pmatrix}
0\cdot\bigl(-\frac1{\sqrt{6}}\bigr)-\bigl(-\frac1{\sqrt{2}}\bigr)\cdot\frac2{\sqrt{6}}\\
\frac2{\sqrt{6}}\cdot\frac1{\sqrt{3}}-\bigl(-\frac1{\sqrt{6}} \bigr)\cdot\frac1{\sqrt{3}}\\
\frac1{\sqrt{3}}\cdot\bigl(-\frac1{\sqrt{2}}\bigr)-\frac1{\sqrt{3}}\cdot 0
\end{pmatrix}
\\
&=\begin{pmatrix}
\frac1{\sqrt{3}}\\
\frac23\cdot\frac1{\sqrt{2}}+\frac13\cdot\frac1{\sqrt{2}}\\
-\frac1{\sqrt{6}}
\end{pmatrix}
\\
&=\begin{pmatrix}
\frac1{\sqrt{3}}\\
\frac1{\sqrt{2}}\\
-\frac1{\sqrt{6}}
\end{pmatrix}
\\
\end{align*}
Das Vektorprodukt der ersten zwei Spalten ist also tatsächlich
identisch mit dem dritten Vektor, die Determinante muss also $1$ sein.

Natürlich kann man die Determinante auch mit der Sarrusschen Formel ermitteln:
\begin{align*}
\left|
\begin{matrix}
\frac1{\sqrt{3}}&\frac1{\sqrt{3}}&\frac1{\sqrt{3}}\\
0&-\frac1{\sqrt{2}}&\frac1{\sqrt{2}}\\
\frac2{\sqrt{6}}&-\frac1{\sqrt{6}}&-\frac1{\sqrt{6}}
\end{matrix}
\right|
&=
\frac1{\sqrt{3}}\cdot\biggl(-\frac1{\sqrt{2}}\biggr)\cdot\biggl(-\frac1{\sqrt{6}}\biggr)
+
\frac1{\sqrt{3}}\cdot\frac1{\sqrt{2}}\cdot\frac2{\sqrt{6}}
\\
&\qquad 
+
\frac1{\sqrt{3}}\cdot0\cdot\biggl(-\frac1{\sqrt{6}}\biggr)
-
\frac2{\sqrt{6}}\cdot\biggl(-\frac1{\sqrt{2}}\biggr)\cdot\frac1{\sqrt{3}}
\\
&\qquad
-
\biggl(-\frac1{\sqrt{6}}\biggr)\cdot\frac1{\sqrt{2}}\cdot \frac1{\sqrt{3}}
-
\biggl(-\frac1{\sqrt{6}}\biggr)\cdot0\cdot\frac1{\sqrt{3}}
\\
&=
\frac16+\frac26+0+\frac26+\frac16+0=1
\end{align*}
\item
Die Matrix $A$ ist also in $\operatorname{SO}(3)$, also kann man den
Drehwinkel $\alpha$ mit Hilfe der Spur ermitteln:
\begin{align*}
\operatorname{tr} A=1+2\cos\alpha&=\frac1{\sqrt{3}}-\frac1{\sqrt{2}}-\frac1{\sqrt{6}}
\\
&=\frac{\sqrt{2}-\sqrt{3}-1}{\sqrt{6}}=-0.53800480246078477625
\\
\Rightarrow\qquad
\cos\alpha&=\frac12\left(
\frac{\sqrt{2}-\sqrt{3}-1}{\sqrt{6}}-1
\right)=-0.76900240123039238812
\\
\alpha&=2.448075423264433\\
\alpha&=140.2643896827546^\circ
\qedhere
\end{align*}
\end{teilaufgaben}
\end{loesung}

