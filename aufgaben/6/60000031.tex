Im Laufe der Covid-19-Pandemie konnte man immer wieder beobachten, dass
die Fallzahlen wie immer wieder vorhergesagt einen exponentiellen
Verlauf haben, dass sie also einem exponentiellen Gesetz der Form
\begin{equation}
N(t) = N_0\cdot  R_0^{t/T_G}
\label{60000031:funktion}
\end{equation}
folgen, wobei $R_0$ die Reproduktionszahl ist und $T_G$ die
Generationszeit, letztere darf man als bekannt annehmen.
Es ist offensichtlich, dass $R_0>1$ bedeutet, dass die Fallzahlen
zunehmen werden, für $R_0<1$ werden die Fallzahlen abnehmen.

Man kann $R_0$ weiter aufschlüssen in $R_0=\kappa\cdot q\cdot D$, wobei
$\kappa$ die mittlere Anzahl von Kontakten einer infizierten Person
pro Zeiteinheit ist, $D$ die mittlere Zeitdauer, während der eine
infizierte Person ansteckend ist, und $q$ die Wahrscheinlichkeit, dass
bei Kontakt eine Infektion stattfindet.
Die Ausbreitung kann also beeinflusst werden, indem man die Kontakte
$\kappa$ (Lockdown, 2G, Zertifikatspflicht, Abstandsregeln)
und die Wahrscheinlichkeit $q$ (Impfung, Maskenpflicht)
modifiziert, auf $D$ hat man dagegen kaum Einfluss.
Dies haben die Zahlen bei jeder neuen Welle und den zur Eindämmung
ergriffenen Massnahmen jeweils gezeigt.

Um $R_0$ und damit den Erfolg von Massnahmen zu bestimmen,
braucht man Datenpunkte $t_i$ und zugehörige Fallzahlen $N_i$, die in der
Schweiz vom BAG veröffentlicht werden, allerdings ist die Datenqualität
relativ schlecht, weil es anscheinend nicht möglich ist, die Zahlen auch
am Wochenende nach Tagen gesondert zu erheben, was am Montag jeweils zu
einer grossen Spitze führt.
Nehmen Sie für diese Aufgabe an, dass die Daten auf sinnvolle Art und
Weise geglättet worden sind, dass also an jedem Tag ein Paar
$(t_i,N_i)$ zur Verfügung steht.

Beschreiben Sie eine Methode, mit der die Parameter
$R_0$ und $N_0$
%, die den Verlauf der Fallzahlen bestimmen,
innerhalb einer Periode von $n$ Tagen
gleichbleibender Massnahmen bestimmt werden können.
Geben Sie insbesondere die Matrizen und Vektoren eines Gleichungssystems
an, welches gelöst werden muss und Formeln, die $N_0$ und $R_0$ aus
den Lösungen des Gleichungssystems berechnen.

\begin{hinweis}
Wenden Sie den natürlichen Logarithmus an.
\end{hinweis}

\thema{Least Squares}

\begin{loesung}
\definecolor{darkred}{rgb}{0.6,0,0}
Es geht darum, die Parameter der Funktion $N(t)$ in~\eqref{60000031:funktion}
abzuschätzen.
Dabei kann man davon ausgehen, dass die Reproduktionszeit $T_G$ gleich
bleibt, da ja das gleiche Virus die unveränderte Spezies Homo sapiens 
infiziert.
Zu bestimmen sind daher die $N_0$  und $R_0$.
Der natürliche Logarithmus von~\eqref{60000031:funktion} ist
\[
\log N(t)
=
\log {\color{darkred}N_0} + \frac{t}{T_G}\log {\color{darkred}R_0}
\]
Für die gegebenen Daten müssen also die Gleichungen
\[
\begin{linsys}{3}
\frac{t_1 }{t_G} \log {\color{darkred}R_0}
&+&\log {\color{darkred}N_0}
&=&\log N_1
\\
\frac{ t_2 }{t_G} \log {\color{darkred}R_0}
&+&\log {\color{darkred}N_0}
&=&\log N_2
\\
\vdots &&\vdots&&\vdots
\\
\frac{ t_n }{t_G} \log {\color{darkred}R_0}
&+&\log {\color{darkred}N_0}
&=&\log N_n
\end{linsys}
\]
erfüllt sein.
Dies ist ein überbestimmtes Gleichungssystem, es wird im Allgemeinen
keine exakte Lösung haben.
Man kann aber aus
\[
A{\color{darkred}x}
=
\begin{pmatrix}
t_1/T_G & 1\\
t_2/T_G & 1\\
\vdots&\vdots\\
t_n/T_G & 1
\end{pmatrix}
\begin{pmatrix}
\log{\color{darkred}R_0}\\
\log{\color{darkred}N_0}
\end{pmatrix}
=
\begin{pmatrix}
\log N_1\\
\log N_2\\
\vdots\\
\log N_n
\end{pmatrix}
=
b
\]
die Parameter bestimmen, die am nächsten an einer Lösung sind.
Man muss also die Matrix
\[
A^tA
=
\begin{pmatrix}
\displaystyle
\sum_{i=1}^n t_i^2/T_G^2 &
\displaystyle
\sum_{i=1}^n t_i/T_G \\
\displaystyle
\sum_{i=1}^n t_i/T_G   & n 
\end{pmatrix}
\]
und den Vektor der rechten Seiten
\[
A^tb
=
\begin{pmatrix}
\displaystyle
\sum_{i=1}^n \log(N_i)t_i\\
\displaystyle
\sum_{i=1}^n \log N_i
\end{pmatrix}
\]
ermitteln und dann das Gleichungssystem $A^tA{\color{darkred}x} = A^tb$
lösen.
Danach kann man $R_0$ und $N_0$ durch Auflösen von
\begin{align*}
{\color{darkred}x_1}&=\log{\color{darkred}R_0}
&&\Rightarrow&
{\color{darkred}R_0} &= \exp({\color{darkred}x_1})
\\
{\color{darkred}x_2}&=\log{\color{darkred}N_0}
&&\Rightarrow&
{\color{darkred}N_0} &= \exp({\color{darkred}x_2})
\end{align*}
gewinnen.
\end{loesung}

\begin{bewertung}
Least-Squares-Problem ({\bf LS}) 1 Punkt,
Logarithmusgesetze ({\bf L}) 1 Punkt,
Matrix $A$ ({\bf A}) und Vektor $b$ ({\bf B}) je 1 Punkt,
Lösungsverfahren über $A^tA$ und $A^tb$ ({\bf V}) 1 Punkt.
Auflösung nach $R_0$ und $N_0$ ({\bf L}) 1 Punkt.
\end{bewertung}
