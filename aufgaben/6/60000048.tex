Eine Abbildungsmatrix $R$ bildet Vektoren wie folgt ab:
\[
\begin{pmatrix}
1\\0\\0
\end{pmatrix}
\mapsto
\frac13
\begin{pmatrix}
1\\2\\2
\end{pmatrix},
\quad
\frac{1}{\!\sqrt{2}}
\begin{pmatrix}
0\\1\\1
\end{pmatrix}
\mapsto
\frac{1}{3\!\sqrt{2}}
\begin{pmatrix}
4\\-1\\-1
\end{pmatrix},
\quad
\frac{1}{\!\sqrt{2}}
\begin{pmatrix}
0\\-1\\1
\end{pmatrix}
\mapsto
\frac{1}{\!\sqrt{2}}
\begin{pmatrix}
0\\1\\-1
\end{pmatrix}.
\]
Stellen Sie die Gleichungen zur Bestimmung der Abbildungsmatrix auf.

\begin{hinweis}
Die Matrix $R$ muss nicht berechnet werden.
\end{hinweis}

\begin{loesung}
Aus der Abbildungsgleichung folgt die Matrixgleichung für die Abbildungsmatrix
$R$:
\renewcommand{\arraystretch}{1.3}
\[
R
\underbrace{
\begin{pmatrix}
1 & 0                    &  0                    \\
0 & \frac{1}{\!\sqrt{2}} & -\frac{1}{\!\sqrt{2}} \\
0 & \frac{1}{\!\sqrt{2}} &  \frac{1}{\!\sqrt{2}}
\end{pmatrix}
}_{\displaystyle =A}
=
\underbrace{
\begin{pmatrix}
\frac13 &  \frac{4}{3\!\sqrt{2}} &  0                    \\
\frac23 & -\frac{1}{3\!\sqrt{2}} &  \frac{1}{\!\sqrt{2}} \\
\frac23 & -\frac{1}{3\!\sqrt{2}} & -\frac{1}{\!\sqrt{2}}
\end{pmatrix}
}_{\displaystyle =B}.
\]
Die Abbildungsmatrix kann jetzt als $R=BA^{-1}$ berechnet werden.
Die inverse Matrix von $A$ ist
\[
A^{-1}
=
\begin{pmatrix}
1 &  0                    & 0                    \\
0 &  \frac{1}{\!\sqrt{2}} & \frac{1}{\!\sqrt{2}} \\
0 & -\frac{1}{\!\sqrt{2}} & \frac{1}{\!\sqrt{2}}
\end{pmatrix},
\]
weil der $2\times 2$-Block rechts unten eine Drehmatrix ist und
als orthogonale Matrix durch Transponieren invertiert werden kann.
Damit folgt
\begin{align*}
R
&=
\begin{pmatrix}
\frac13 &  \frac{4}{3\!\sqrt{2}} &  0                    \\
\frac23 & -\frac{1}{3\!\sqrt{2}} &  \frac{1}{\!\sqrt{2}} \\
\frac23 & -\frac{1}{3\!\sqrt{2}} & -\frac{1}{\!\sqrt{2}}
\end{pmatrix}
\begin{pmatrix}
1 &  0                    & 0                    \\
0 &  \frac{1}{\!\sqrt{2}} & \frac{1}{\!\sqrt{2}} \\
0 & -\frac{1}{\!\sqrt{2}} & \frac{1}{\!\sqrt{2}}
\end{pmatrix}
=
\begin{pmatrix}
\frac13 &  \frac23 &  \frac23 \\
\frac23 & -\frac23 &  \frac13 \\
\frac23 &  \frac13 & -\frac23
\end{pmatrix}.
\qedhere
\end{align*}
\end{loesung}

