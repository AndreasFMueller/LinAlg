Gegeben ist die Abbildungsmatrix
\[
A=\begin{pmatrix}
\frac12         &-\frac{\sqrt{3}}2\\
\frac{\sqrt{3}}2&\frac12
\end{pmatrix}.
\]
\begin{teilaufgaben}
\item
Ist $A$ eine orthogonale Matrix?
\item
Ist $A\in \operatorname{SO}(2)$?
\item
Ist $A$ eine Drehmatrix? Falls ja berechnen Sie den Drehwinkel.
\end{teilaufgaben}

\thema{Matrizen-Gruppen}

\begin{loesung}
\begin{teilaufgaben}
\item Wir müssen überprüfen, ob $\transpose{A}A=E$ ist.
Dazu berechnen wir
\begin{align*}
\transpose{A}A
&=
\begin{pmatrix}
\frac12         &\frac{\sqrt{3}}2\\
-\frac{\sqrt{3}}2&\frac12
\end{pmatrix}
\begin{pmatrix}
\frac12         &-\frac{\sqrt{3}}2\\
\frac{\sqrt{3}}2&\frac12
\end{pmatrix}
=
\begin{pmatrix}
\frac14+\frac34&-\frac{\sqrt{3}}{4}+\frac{\sqrt{3}}{4}\\
-\frac{\sqrt{3}}{4}+\frac{\sqrt{3}}{4}&\frac34+\frac14
\end{pmatrix}
=
\begin{pmatrix}
1&0\\
0&1
\end{pmatrix}
=E,
\end{align*}
die Matrix $A$ ist also tatsächlich orthogonal.
\item
Eine orthogonale Matrix $A$ ist in $\operatorname{SO}(2)$, wenn die
Determinante $\det(A)=1$ ist.
Wir berechnen daher die Determinante:
\[
\det(A)
=
\left|\,\begin{matrix}
\frac12         &-\frac{\sqrt{3}}2\\
\frac{\sqrt{3}}2&\frac12
\end{matrix}\,\right|
=
\frac14+\frac34
=
1.
\]
Die Matrix $A$ ist folglich in $\operatorname{SO}(2)$.
\item
Eine Drehmatrix ist eine Matrix in $\operatorname{SO}(2)$, also ist
$A$ eine Drehmatrix. 
Der Drehwinkel kann mit Hilfe der Spurformel ermittelt werden:
\[
\cos \alpha
=
\frac{\operatorname{Spur}(A)}2
=
\frac12
\qquad\Rightarrow\qquad
\alpha = 60^\circ.
\qedhere
\]
\end{teilaufgaben}
\end{loesung}



