Ist die Matrix
\[
A=\begin{pmatrix}2&6\\6&7\end{pmatrix}
\]
diagonalisierbar? Wenn ja, geben Sie eine Basis an, in der $A$ diagonal
ist.

\begin{loesung}
Eine geeignete Basis besteht aus den Eigenvektoren von $A$. Weil $A$
symmetrisch ist, l"asst sich immer eine Eigenvektorbasis find.

Das charakteristische Polynom von $A$ ist
\begin{align*}
\det(A-\lambda E)
&=\left|\begin{matrix}2-\lambda&6\\6&7-\lambda\end{matrix}\right|
=(2-\lambda)(7-\lambda)-36=\lambda^2-9\lambda+14-36=\lambda^2-9\lambda-22
\end{align*}
Die Nullstellen k"onnen mit der L"osungsformel f"ur quadratische Gleichungen
gefunden werden:
\begin{align*}
\lambda_{\pm}&=\frac{9}{2}\pm\frac{1}{2}\sqrt{9^2+4 \cdot 22}
=\frac{9}{2}\pm\frac{1}{2}\sqrt{169}=\frac{9\pm13}{2}
=\begin{cases}
11\\-2
\end{cases}
\end{align*}
F"ur jeden Eigenwert muss jetzt mit Hilfe des Gaussalgorithmus ein
Eigenvektor gefunden werden.

F"ur $\lambda_+=11$ findet man
\[
\begin{tabular}{|>{$}c<{$}>{$}c<{$}|}
\hline
2-11&6\\
6&7-11\\
\hline
\end{tabular}
=
\begin{tabular}{|>{$}c<{$}>{$}c<{$}|}
\hline
-9&6\\
6&-4\\
\hline
\end{tabular}
\rightarrow
\begin{tabular}{|>{$}c<{$}>{$}c<{$}|}
\hline
1&-\frac23\\
0&0\\
\hline
\end{tabular}
\]
Die zweite Koordinate ist wie erwartet frei w"ahlbar, indem wir sie auf
$3$ setzen erhalten wir einen Eigenvektor ohne Br"uche
\[
v_+=\begin{pmatrix}2\\3\end{pmatrix}.
\]

F"ur $\lambda_-=-2$ erhalten wir
\[
\begin{tabular}{|>{$}c<{$}>{$}c<{$}|}
\hline
2+2&6\\
6&7+2\\
\hline
\end{tabular}
=
\begin{tabular}{|>{$}c<{$}>{$}c<{$}|}
\hline
4&6\\
6&9\\
\hline
\end{tabular}
\rightarrow
\begin{tabular}{|>{$}c<{$}>{$}c<{$}|}
\hline
1&\frac32\\
0&0\\
\hline
\end{tabular}.
\]
Wieder ist die zweite Variable frei w"ahlbar, die Wahl $-2$ liefert 
den Eigenvektor
\[
v_-=\begin{pmatrix}3\\-2\end{pmatrix}.
\]

Kontrolle: Wir kontrollieren die Rechnung durch Multiplikation von $A$
mit den gefundenen Eigenvektoren:
\begin{align*}
Av_+&=\begin{pmatrix}2&6\\6&7\end{pmatrix}\begin{pmatrix}2\\3\end{pmatrix}
=\begin{pmatrix}4+18\\ 12 + 21\end{pmatrix}
=\begin{pmatrix}22\\ 33\end{pmatrix}=11\begin{pmatrix}2\\3\end{pmatrix}
=\lambda_+v_+,\\
Av_-&=\begin{pmatrix}2&6\\6&7\end{pmatrix}\begin{pmatrix}3\\-2\end{pmatrix}
=\begin{pmatrix}6-12\\18-14 \end{pmatrix}
=\begin{pmatrix}-6\\4 \end{pmatrix}=-2\begin{pmatrix}3\\-2\end{pmatrix}
=\lambda_-v_-.
\end{align*}
Ausserdem sind die beiden Vektoren orthogonal, wie man das von den Eigenvektoren
einer symmetrischen Matrix zu verschiedenen Eigenwerten erwartet:
\[
v_+\cdot v_-=
\begin{pmatrix}2\\3\end{pmatrix}
\cdot
\begin{pmatrix}3\\-2\end{pmatrix}
=2\cdot 3 + 3\cdot(-2)=0.
\qedhere
\]
\end{loesung}

\begin{diskussion}
Man kann sich auch die allgemeinere Frage stellen, welche ganzzahligen
symmetrischen $2\times 2$-Matrizen eigentlich ganzzahlige Eigenwerte haben.
Dazu untersucht man die Matrix
\[
A=\begin{pmatrix}a&b\\b&c\end{pmatrix}
\]
mit dem charakteristischen Polynom
\[
\chi_A(\lambda)=(a-\lambda)(c-\lambda)-b^2=\lambda^2-(a+c)\lambda+ac-b^2.
\]
Die Nullstellen k"onnen mit der L"osungsformel f"ur quadratische 
Gleichungen gefunden werden:
\begin{align*}
\lambda_{\pm}&=\frac{a+c}2\pm\sqrt{\biggl(\frac{a+c}2\biggr)^2-ac+b^2}\\
	&=\frac{a+c}2\pm\frac12\sqrt{a^2+2ac+c^2-4ac+4b^2}\\
	&=\frac{a+c}2\pm\frac12\sqrt{a^2-2ac+c^2+4b^2}\\
	&=\frac{a+c}2\pm\frac12\sqrt{(a-c)^2+4b^2}.
\end{align*}
Wenn die Eigenwerte ganzzahlig werden sollen, dann muss die Quadratwurzel
auf der rechten Seite ganzzahlig sein, es muss also
\[
x=a-c, \quad y=2b ,\quad z=\sqrt{(a-c)^2+4b^2}
\]
ein pythagor"aisches Tripel sein. Schreibt man $m=\frac{a+c}2$, kann man
die Matrix jetzt auch mit dem Tripel $(x,y,z)$ parametrisieren:
\[
A=\begin{pmatrix}
m + \frac{x}2&\frac{y}2\\
\frac{y}2&m-\frac{x}2
\end{pmatrix}
,\qquad
\lambda_{\pm}=m\pm\frac{z}2.
\]
Mit diesen Bezeichungen kann man jetzt auch die Eigenvektoren
berechnen. Wir schreiben das Gauss\-tableau f"ur $A-\lambda_\pm E$:
\[
\begin{tabular}{|>{$}c<{$}>{$}c<{$}|}
\hline
m+\frac{x}2-\lambda_\pm&\frac{y}2\\
\frac{y}2&m-\frac{x}2-\lambda_\pm\\
\hline
\end{tabular}
=
\begin{tabular}{|>{$}c<{$}>{$}c<{$}|}
\hline
m+\frac{x}2-m\mp \frac{z}2&\frac{y}2\\
\frac{y}2&m-\frac{x}2-m\mp\frac{z}2\\
\hline
\end{tabular}
\rightarrow
\begin{tabular}{|>{$}c<{$}>{$}c<{$}|}
\hline
x\mp z&y\\
y&-x\mp z\\
\hline
\end{tabular}
\]
Daraus kann man als m"ogliche Eigenvektoren ablesen:
\[
v_+=\begin{pmatrix}
y\\z-x
\end{pmatrix}
\sim
\begin{pmatrix}
x+z\\y
\end{pmatrix}
,\qquad
v_-=\begin{pmatrix}
-y\\x+z
\end{pmatrix}
\sim
\begin{pmatrix}
x-z\\y
\end{pmatrix}
\]
Die Eigenvektoren m"ussen orthogonal sein, ihr Skalarprodukt muss also
verschwinden:
\begin{align*}
\begin{pmatrix}
y\\z-x
\end{pmatrix}
\cdot
\begin{pmatrix}
-y\\x+z
\end{pmatrix}
&=
-y^2+(z+x)(z-x)=-y^2+z^2-x^2=0,
\\
\begin{pmatrix}
x+z\\y
\end{pmatrix}
\cdot
\begin{pmatrix}
x-z\\y
\end{pmatrix}
&=
(x+z)(x-z)+y^2=x^2-z^2+y^2=0.
\end{align*}
Die Aufgabe entspricht dem pythagor"aischen Tripel $(5,12,13)$.
\end{diskussion}

\begin{bewertung}
Charakteristisches Polynom ({\bf X}) 1 Punkt,
Nullstellen ({\bf N}) 1 Punkt,
Gleichungssysteme f"ur $\lambda_+$ und $\lambda_-$ ({\bf G$\pm$}) je 1 Punkt,
Eigenvektoren ({\bf E$\pm$}) je 1 Punkt.
\end{bewertung}

