Die Durham Bridge ist eine Eisenbahnbrücke in der Stadt Durham
in North Carolina in den USA.
\begin{center}
\includeagraphics[width=\textwidth]{canopener.jpeg}
\end{center}
Die Durchfahrthöhe ist mit $3.76\,\text{m}$ sehr gering, etwa $30\,\text{cm}$
geringer als normalerweise von einer solchen Brücke verlangt wird.
Aus diesem Grund kommt es trotz aufwendiger Warnsysteme
regelmässig zu Unfällen mit zu hohen Fahrzeuge, die von verschiedenen
Webcams festgehalten werden und die Brücke unter dem Namen
{\em can opener bridge} zu einem Internet-Phänomen gemacht haben.
Aus verschiedenen Gründen scheint es nicht möglich zu sein, die Brücke
ausreichend zu erhöhen oder die Strasse abzusenken.

Der Youtube-Kanal \texttt{yovo68} dokumentiert alle Unfälle mit mehreren
Kameras.
Um die Aufzeichnung nur bei bevorstehenden Unfälle auszulösen,
könnte man eine weitere Kamera verwenden, die den heranfahrenden
Verkehr beobachtet.
Die Kamera mit einem Objektiv mit $f=8000$ Brennweite und einem
$1920\times 1080$-Chip ist im Punkt $C=(0,0,12)$ montiert und mit der
Drehmatrix
\[
D=\begin{pmatrix*}[r]
  -0.9806&  0.1961&  0\phantom{.0000}\\
   0.0077&  0.0384& -0.9992\\
  -0.1960& -0.9798& -0.0392
\end{pmatrix*}
\]
orientiert.

\begin{teilaufgaben}
\item
Das Auslösesystem soll die Aufzeichnung immer dann starten, wenn auf den
Bildern eine Bewegung oberhalb einer Geraden zwischen den Punkten
$P_1=(10,-100,13)$ und $P_2=(-10,-100,13)$.
Bestimmen Sie die Punkte $B_1$ und $B_2$ auf dem Chip, auf die $P_1$ und
$P_2$ abgebildet werden.
\item
Bei einem Unfall hat das System nicht ausgelöst. 
Auf Bild der Auslöserkamera kann man Aktivität im Punkt
$B=(b_x,b_y)$ sehen.
Geben Sie eine einfache Formel mit der man testen kann, ob der
Punkt $B$ oberhalb der Verbindungsstrecke $B_1B_2$ liegt.
Wenden Sie die Formel auf den Punkt $B=(666,148)$ an.
\end{teilaufgaben}

\begin{loesung}
\begin{teilaufgaben}
\item
Die Kameramatrix ist
\[
K=\begin{pmatrix}
f&0&m_x\\
0&f&m_y\\
0&0&1
\end{pmatrix}
=
\begin{pmatrix*}[r]
8000&   0&960\\
   0&8000&540\\
   0&   0&  1
\end{pmatrix*}.
\]
Die Projektionsmatrix wird damit
\[
P
=
K D \begin{pmatrix}E&-\vec{c}\end{pmatrix}
=
%\begin{pmatrix*}[r]
%-8032.8\phantom{0000}& 628.29\phantom{000}&  -37.625\phantom{000}&161110\phantom{.0000}\\
%  -44.330\phantom{00}&-221.65\phantom{000}&-8015.0\phantom{00000}& 97067\phantom{.0000}\\
%   -0.19597          &  -0.97983          &   -0.039193          &     4.3896
%\end{pmatrix*}.
\def\p#1{\phantom{#1}}
\begin{pmatrix*}[r]
-8032.8\p{0000}&   628.29\p{000}&   -37.625\p{000}&       451.50\p{000} \\
  -44.330\p{00}&  -221.65\p{000}& -8015.0\p{00000}&     96180\p{.00000} \\
   -0.19597    &    -0.97983    &    -0.039193    &         0.47032 
\end{pmatrix*}
\]
Die Bildpunkte erhält man, indem man die Matrix $P$ auf die
Punkte $P_1$ und $P_2$ mit homogenen Koordinaten 
\begin{align*}
\tilde{p}_1&=\begin{pmatrix*}[r]10\\-100\\1\\1\end{pmatrix*}
&
\tilde{p}_2&=\begin{pmatrix*}[r]-10\\-100\\1\\1\end{pmatrix*}.
\intertext{anwendet. Die homogenen Koordinaten der Bilder sind}
\tilde{b}_1&=P\tilde{p}_1 = \begin{pmatrix*}[r]
  -143190\phantom{.000}
   13707\phantom{.000}
   95.984
\end{pmatrix*}
&
\tilde{b}_2&=P\tilde{p}_2 = \begin{pmatrix*}[r]
   17461\phantom{.000}
   14593\phantom{.000}
      99.903
\end{pmatrix*}.
\intertext{Nach Teilen durch die dritte Komponente bekommt man die
Pixel-Koordinaten:}
B_1&=(-1492,143)
&
B_2&=(175,146)
\end{align*}
(An dieser Stelle ist beim Tippen der Aufgabe ein kleiner Fehler
passiert, der Punkt $C$ war falsch abgetippt worden, was zu einem Punkt $B_1$
ausserhalb des Chips führt. Dies hat jedoch keine Konsequenzen für
die Lösungsmethoden.)
\item
Der Punkt $B$ liegt oberhalb der Verbindungsstrecke, wenn das Dreieck
$\triangle B_1B_2B$ positiv orientiert ist. 
Dies kann man mit der Determinanten testen:
\[
\det(\overrightarrow{B_1B_2},\overrightarrow{B_1B})
=
\left|
\begin{matrix}
b_{2x}-b_{1x}&b_x-b_{1x}\\
b_{2y}-b_{1y}&b_y-b_{1y}
\end{matrix}
\right|
> 0.
\]
Für den Punkt $B=(66,148)$ ergibt sich die Determinante
\[
\left|
\begin{matrix}
1667& 3\\
2158& 5\\
\end{matrix}
\right|
=
2\cdot 1541 - 3\cdot 491 
=
1861 > 0,
\]
das Kriterium ist also erfüllt, der Bildpunkt ist oberhalb der
Verbindungsstrecke.
\qedhere
\end{teilaufgaben}
\end{loesung}

\begin{bewertung}
Kameramatrix ({\bf K}) 1 Punkt,
Kamerazentrum ({\bf C}) 1 Punkt,
Projizierte Punkte in homogenen Koordinaten ({\bf H}) 2 Punkt3,
Pixelkoordinaten der projizierten Punkte ({\bf B}) 1 Punkt,
Kriterium für Lage eines Punktes oberhalb der Verbindungslinie 
({\bf O}) 1 Punkt.
\end{bewertung}


