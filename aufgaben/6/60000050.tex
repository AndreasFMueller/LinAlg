$U^\perp$ ist das Orthogonalkomplement von
\[
U
=
\langle u_1,u_2\rangle
=
%\left\langle
%\bgroup
%\begin{small}
%\begin{pmatrix}1\\1\\0\end{pmatrix},
%\begin{pmatrix}0\\1\\1\end{pmatrix}
%\end{small}
%\egroup
%\right\rangle.
\biggl\langle
\biggl(
\:
\clap{$\displaystyle
\renewcommand{\arraystretch}{0.6}
\begin{array}{c}\scriptstyle 1\\\scriptstyle 1\\\scriptstyle 0\end{array}
$}
\:
\biggr),
\biggl(
\:
\clap{$\displaystyle
\renewcommand{\arraystretch}{0.6}
\begin{array}{c}\scriptstyle 0\\\scriptstyle 1\\\scriptstyle 1\end{array}
$}
\:
\biggr)
\biggr\rangle.
\]
Stellen Sie das Gauss-Tableau auf, mit dem $U^\perp$ bestimmt werden
kann.

\begin{loesung}
Um das Orthogonalkomplement der Vektoren $u_1$ und $u_2$ zu
bestimmen, muss erst die Matrix $A$ so konstruiert werden, dass
die Spalten von $A^t$ die Vektoren $u_1$ und $u_2$ sind, also
\[
A^t
=
\begin{pmatrix}
1 & 0 \\
1 & 1 \\
0 & 1
\end{pmatrix}.
\]
Somit ist
\[
A
=
\begin{pmatrix}
1&1&0\\
0&1&1
\end{pmatrix}.
\]
Das Orthogonalkomplement von $U$ ist der Kern von $A$, $U^\perp=\ker A$,
also die Läsungsmenge des Gleichungssystems $Ax=0$.
Dies entspricht dem Tableau
\begin{align*}
\begin{tabular}{|>{$}c<{$}>{$}c<{$}>{$}c<{$}|>{$}c<{$}|}
\hline
x_1 & x_2 & x_3 & 1 \\
\hline
  1 &   1 &   0 & 0 \\
  0 &   1 &   1 & 0 \\
\hline
\end{tabular}
&\to
\begin{tabular}{|>{$}c<{$}>{$}c<{$}>{$}c<{$}|>{$}c<{$}|}
\hline
x_1 & x_2 & {\color{darkgreen}x_3} & 1 \\
\hline
  1 &   0 & {\color{darkgreen} -1} & 0 \\
  0   & 1 & {\color{darkgreen}  1} & 0 \\
\hline
\end{tabular}
\end{align*}
Daraus kann man die Lösungsmenge
\[
U^\perp
=
\ker A
=
\left\{
\left.
{\color{darkgreen}x_3}
\begin{pmatrix*}[r]
 1\\
-1\\
 1
\end{pmatrix*}
\;
\right|
\;
{\color{darkgreen}x_3}\in\mathbb{R}
\right\}
\]
ablesen.
\end{loesung}

