%
% earth2.tex
%
% (c) 2021 Prof Dr Andreas Müller, OST Ostschweizer Fachhochschule
%
\documentclass[tikz]{standalone}
\usepackage{times}
\usepackage{amsmath}
\usepackage{txfonts}
\usepackage[utf8]{inputenc}
\usepackage{graphics}
\usetikzlibrary{arrows,intersections,math}
\usepackage{ifthen}
\begin{document}

\def\punkt#1#2{
	\fill[color=white,opacity=0.5] #1 circle[radius=0.2];
	\node at #1 {$#2$};
}

\newboolean{showgrid}
\setboolean{showgrid}{false}
%\setboolean{showgrid}{true}
\def\breite{8}
\def\hoehe{4}

\begin{tikzpicture}[>=latex,thick]

% Povray Bild
\begin{scope}[xshift=-4cm]
\node at (0,0) {\includegraphics[width=7.0cm]{earth2.jpg}};
\end{scope}
\begin{scope}[xshift=4cm]
\node at (0,0) {\includegraphics[width=7.0cm]{verdreht2.jpg}};
\end{scope}

% Gitter
\ifthenelse{\boolean{showgrid}}{
\draw[step=0.1,line width=0.1pt] (-\breite,-\hoehe) grid (\breite, \hoehe);
\draw[step=0.5,line width=0.4pt] (-\breite,-\hoehe) grid (\breite, \hoehe);
\draw                            (-\breite,-\hoehe) grid (\breite, \hoehe);
\fill (0,0) circle[radius=0.05];
}{}

\coordinate (A) at (-5.3,-3.3);
\coordinate (B)  at (-1.9,-3.1);
\punkt{(A)}{A}
\punkt{(B)}{B}

\coordinate (x) at (-6.3,-0.7);
\coordinate (y) at (-0.6,-0.3);
\coordinate (z) at (-3.8,3.5);
\punkt{(x)}{x}
\punkt{(y)}{y}
\punkt{(z)}{z}

\coordinate (Aprime) at (3.0,1.8);
\coordinate (Bprime) at (2.5,-1.5);
\coordinate (nprime)  at (7.0,-0.3);
\punkt{(Aprime)}{A\!'}
\punkt{(Bprime)}{B'}
\punkt{(nprime)}{\vec{n}'}

\coordinate (xprime) at (1.7,-0.7);
\coordinate (yprime) at (7.4,-0.3);
\coordinate (zprime) at (4.3,3.6);
\punkt{(xprime)}{x}
\punkt{(yprime)}{y}
\punkt{(zprime)}{z}

\node at (-3.9,1.4) {$(\vec{a}\times\vec{b})^0$};

\end{tikzpicture}

\end{document}

