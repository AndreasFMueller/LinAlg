Gegeben sind die Vektoren
\[
\vec u = \begin{pmatrix}\frac12\\\frac{\sqrt{3}}2\end{pmatrix},
\qquad
\vec v = \begin{pmatrix}\frac{\sqrt{3}}2\\-\frac12\end{pmatrix}
\]
\begin{teilaufgaben}
\item
Finden Sie eine Matrix $A$, die die Standardbasisvektoren $\vec e_1$ und
$\vec e_2$ auf $\vec u$ bzw.~$\vec v$ abbildet.
\item
Ist $A$ eine orthogonale Matrix?
\item
Ist $A\in \operatorname{SO}(2)$?
\item
Ist $A$ eine Drehmatrix?
\item
Finden Sie eine Matrix $B$, die die Vektoren $\vec u$ und $\vec v$ auf
die Standardbasisvektoren abbildet.
\end{teilaufgaben}

\thema{Abbildungsmatrix}
\thema{Matrizen-Gruppen}
\begin{loesung}
\begin{teilaufgaben}
\item Die Spalten von $A$ sind die Bilder der Standardbasisvektoren
\[
A=\begin{pmatrix}
\frac12         &\frac{\sqrt{3}}2\\
\frac{\sqrt{3}}2&-\frac12
\end{pmatrix}.
\]
\item Wir müssen überprüfen, ob $\transpose{A}A=E$ ist.
Dazu berechnen wir
\begin{align*}
\transpose{A}A
&=
\begin{pmatrix}
\frac12         &\frac{\sqrt{3}}2\\
\frac{\sqrt{3}}2&-\frac12
\end{pmatrix}
\begin{pmatrix}
\frac12         &\frac{\sqrt{3}}2\\
\frac{\sqrt{3}}2&-\frac12
\end{pmatrix}
=
\begin{pmatrix}
\frac14+\frac34&\frac{\sqrt{3}}{4}-\frac{\sqrt{3}}{4}\\
\frac{\sqrt{3}}{4}-\frac{\sqrt{3}}{4}&\frac34+\frac14
\end{pmatrix}
=
\begin{pmatrix}
1&0\\
0&1
\end{pmatrix}
=E,
\end{align*}
die Matrix $A$ ist also tatsächlich orthogonal.
\item
Eine orthogonale Matrix $A$ ist in $\operatorname{SO}(2)$, wenn die
Determinante $\det(A)=1$ ist.
Wir berechnen daher die Determinante:
\[
\det(A)
=
\left|\,\begin{matrix}
\frac12         &\frac{\sqrt{3}}2\\
\frac{\sqrt{3}}2&-\frac12
\end{matrix}\,\right|
=
-\frac14-\frac34
=
-1\ne 1.
\]
Die Matrix $A$ ist daher in $\operatorname{O}(2)\setminus\operatorname{SO}(2)$.
\item
Eine Drehmatrix ist eine Matrix in $\operatorname{SO}(2)$, also ist
$A$ keine Drehmatrix.
\item
Die Matrix $A$ bildet die Standardbasisvektoren auf $\vec u$ und $\vec v$
ab, die gesuchte Matrix $B$ soll das Umgekehrte machen.
Die inverse Matrix macht dies, also ist $B=A^{-1}$.
Da aber $A$ eine orthogonale Matrix ist, ist $A^{-1}=\transpose{A}$.
Da die Matrix $A$ sogar symmetrisch ist, folgt $B=A$.
\end{teilaufgaben}
\end{loesung}



