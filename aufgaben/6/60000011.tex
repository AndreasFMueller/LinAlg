Man betrachtet den Vektorraum $V_n$ der ``Exponentialpolynome vom Grad $n$'',
d.~h.~der Funktionen der Form
\[
f(x)=
a_0e^{\frac{x^2}2}
+a_1xe^{\frac{x^2}2}
+a_2x^2e^{\frac{x^2}2}
+a_3x^3e^{\frac{x^2}2}
+\dots
+a_nx^ne^{\frac{x^2}2}
\]
Die Funktionen
$e^{\frac{x^2}2}$,
$xe^{\frac{x^2}2}$,
$x^2e^{\frac{x^2}2}$,
$x^3e^{\frac{x^2}2},\dots,
x^ne^{\frac{x^2}2}$
bilden eine Basis dieses $(n+1)$-dimensionalen Vektorraumes.
\begin{teilaufgaben}
\item Differentiation
ist eine lineare Abbildung.
Zeigen Sie, dass
\[
D\colon V_n\to V_{n+1},
\]
dass also $Df\in V_{n+1}$ falls nur $f\in V_n$. Beim Ableiten wird der Grad
also höchstens um eins grösser.
\item Stellen Sie die $6\times 5$-Matrix von $D\colon V_4\to V_5$ auf.
\item Gibt es eine Funktion $f\in V_4$ so dass $Df=e^{\frac{x^2}2}$?
\end{teilaufgaben}

\thema{Abbildungsmatrix}
\thema{Basis}

\begin{loesung}
\begin{teilaufgaben}
\item Für die Ableitung gilt die Kettenregel:
\begin{equation}
Dx^ke^{\frac{x^2}2}=kx^{k-1}e^{\frac{x^2}2}+x^{k+1}e^{\frac{x^2}2}
\label{kettenregel}
\end{equation}
Wenn also $k\le n$, also $x^ke^{\frac{x^2}2}\in V_n$, dann ist
$Dx^ke^{\frac{x^2}2}\in V_{n+1}.$
\item
In der Spalte der Matrix von $D$ stehen die Bilder der Basisvektoren,
aus der Formel (\ref{kettenregel}) kann man also die Matrix ablesen:
\[
\begin{pmatrix}
0&1&0&0&0\\
1&0&2&0&0\\
0&1&0&3&0\\
0&0&1&0&4\\
0&0&0&1&0\\
0&0&0&0&1
\end{pmatrix}
\]
\item
Es muss ein Vektor gefunden werden, der durch die Matrix $D$ auf
den Vektor $\transpose{\begin{pmatrix}1&0&0&0&0&0\end{pmatrix}}$
abgebildet wird. Es muss also ein lineares Gleichungssystem mit
Matrix $D$ gelöst werden, wir verwenden dafür den Gauss-Algorithmus:
\begin{align*}
\begin{tabular}{|ccccc|c|}
\hline
0&1&0&0&0&1\\
1&0&2&0&0&0\\
0&1&0&3&0&0\\
0&0&1&0&4&0\\
0&0&0&1&0&0\\
0&0&0&0&1&0\\
\hline
\end{tabular}
&\rightarrow
\begin{tabular}{|ccccc|c|}
\hline
1&0&2&0&0&0\\
0&1&0&3&0&0\\
0&0&1&0&4&0\\
0&0&0&1&0&0\\
0&0&0&0&1&0\\
0&1&0&0&0&1\\
\hline
\end{tabular}
\\
&\rightarrow
\begin{tabular}{|ccccc|c|}
\hline
1&0&2&0&0&0\\
0&1&0&3&0&0\\
0&0&1&0&4&0\\
0&0&0&1&0&0\\
0&0&0&0&1&0\\
0&0&0&-3&0&1\\
\hline
\end{tabular}
\\
&\rightarrow
\begin{tabular}{|ccccc|c|}
\hline
1&0&2&0&0&0\\
0&1&0&3&0&0\\
0&0&1&0&4&0\\
0&0&0&1&0&0\\
0&0&0&0&1&0\\
0&0&0&0&0&1\\
\hline
\end{tabular}
\end{align*}
Im ersten Schritt haben wir die erste Zeile ganz nach unten geschoben,
dann hat das Schema bis auf die letzte Zeile bereits die angestrebte
Dreicksform. Die Einsen auf der Diagonalen erlauben dann, die linke
Seite der letzten Zeile zu Null zu machen, dabei wird die $1$ auf der
rechten Seite nicht verändert. Insbesondere hat das Gleichungssystem
also kein Lösung. Man beachte, dass dieses Argument unabhängig von
$n$ funktioniert, es gibt also in keinem der Vektorräume $V_n$
eine Funktion $f$, für die $Df=e^{\frac{x^2}2}$.
\qedhere
\end{teilaufgaben}
\end{loesung}

