Finden Sie eine Basis, in der
\[
A=\begin{pmatrix}
0&1&1\\
1&0&1\\
1&1&0
\end{pmatrix}
\]
diagonal ist.

\begin{loesung}
Das charakteristische Polynom ist
\begin{align*}
\det(A-\lambda E)
&=
\left|\,
\begin{matrix}
-\lambda&1&1\\
1&-\lambda&1\\
1&1&-\lambda
\end{matrix}
\,\right|
\\
&=-\lambda^3+2+3\lambda
\\
&=-\lambda^3+3\lambda+2
\end{align*}
Eine Nullstelle, nämlich $2$, kann man erraten, daher ist das
charakteristische Polynom durch $\lambda-2$ teilbar:
\begin{align*}
-\lambda^3+3\lambda+2
&=-(\lambda-2)(\lambda+1)^2
\end{align*}
also sind die Eigenwerte $-1$ und $2$.

Für $\lambda=2$ sind die Eignvektoren die Nullvektoren der Matrix
\[
\begin{pmatrix}
-2&1&1\\
1&-2&1\\
1&1&-2
\end{pmatrix}
\]
welche man mit dem Gauss-Algorithmus lösen bestimmen kann
\begin{align*}
\begin{tabular}{|ccc|c|}
\hline
$-2$&1&1&0\\
1&$-2$&1&0\\
1&1&$-2$&0\\
\hline
\end{tabular}
&\rightarrow
\begin{tabular}{|ccc|c|}
\hline
1&$-\frac12$&$-\frac12$&0\\
0&$-\frac32$&$\frac32$&0\\
0&$\frac32$&$-\frac32$&0\\
\hline
\end{tabular}
\\
&\rightarrow
\begin{tabular}{|ccc|c|}
\hline
1&$-\frac12$&$-\frac12$&0\\
0&1&$-1$&0\\
0&0&0&0\\
\hline
\end{tabular}
\\
&\rightarrow
\begin{tabular}{|ccc|c|}
\hline
1&0&$-1$&0\\
0&1&$-1$&0\\
0&0&0&0\\
\hline
\end{tabular}
\end{align*}
Also ist
\[
v_1=\begin{pmatrix}1\\1\\1\end{pmatrix}
\]
ein Eigenvektor zum Eigenwert $\lambda=2$.

Für $\lambda=-1$ findet man für das Gleichungssystem
\begin{align*}
\begin{tabular}{|ccc|c|}
\hline
1&1&1&0\\
1&1&1&0\\
1&1&1&0\\
\hline
\end{tabular}
&\rightarrow
\begin{tabular}{|ccc|c|}
\hline
1&1&1&0\\
0&0&0&0\\
0&0&0&0\\
\hline
\end{tabular}
\end{align*}
Man kann also die zweite und dritte Koordiaten frei wählen, und damit
die Eigenvektoren
\[
v_2=\begin{pmatrix}1\\-1\\0\end{pmatrix},\qquad
v_3=\begin{pmatrix}1\\0\\-1\end{pmatrix}
\]
zum Eigenwert $-1$ finden. Die gesuchte Basis besteht aus den
drei Vektoren $v_i$, $B=\{v_1,v_2,v_3\}$.
\end{loesung}

