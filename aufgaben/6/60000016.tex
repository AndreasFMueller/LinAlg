Betrachten Sie die Matrix
\[
A
=
\begin{pmatrix}
3&-3&2\\
2&-2&2\\
2&-3&3
\end{pmatrix}.
\]
%\begin{teilaufgaben}
%\item
Bestimmen Sie alle Eigenwerte und zugehörige Eigenvektoren.
%\item
%Berechnen Sie $A^{10}$ mit Hilfe der Eigenwerte und Eigenvektoren.
%\end{teilaufgaben}

\thema{Eigenwerte}
\thema{Eigenvektoren}
\thema{charakteristisches Polynom}

\begin{hinweis} $2$ ist ein Eigenwert von $A$.
\end{hinweis}

\begin{loesung}
Man beachte, dass $A$ nicht symmetrisch ist, und damit die Existenz der
Eigenwerte nicht automatisch garantiert ist.
%\begin{teilaufgaben}
%\item
Wir bestimmen zunächst das charakteristische Polynom:
\begin{align*}
\chi_A(\lambda)=
\det (A-\lambda E)
&=
\left|\,
\begin{matrix}
3-\lambda&-3&2\\
2&-2-\lambda&2\\
2&-3&3-\lambda
\end{matrix}
\,\right|
\\
&=-(3-\lambda)(2+\lambda)(3-\lambda) -12-12
+4(2+\lambda)+6(3-\lambda)+6(3-\lambda)
\\
&=
-(9-6\lambda+\lambda^2)(2+\lambda)-24 +8+4\lambda +36-12\lambda
\\
&=
-(18-12\lambda+2\lambda^2+9\lambda -6\lambda^2+\lambda^3)+20-8\lambda
\\
&=
-\lambda^3+4\lambda^2-5\lambda+2
\end{align*}
Wir wissen, dass diese Polynom die Nullstelle $\lambda=2$ hat, und
dividieren daher durch $\lambda -2$:
\[
(-\lambda^3+4\lambda^2-5\lambda+2):(\lambda -2)=-\lambda^2+2\lambda-1
=-(\lambda -1)^2
\]
Also gilt
\[
\chi_A(\lambda)=-(\lambda -1)^2(\lambda -2),
\]
die Matrix $A$ hat also die Eigenwerte $1$ und $2$, wobei die Nullstelle
$1$ doppelt zu zählen ist. Wir erwarten, dass es zu dieser Nullstellen
zwei verschiedene Eigenvektoren geben kann.

Zur Berechnung der Eigenvektoren setzen wir jetzt $\lambda$ in die
Gleichungen $(A-\lambda E)x=0$ ein, und bestimmen $x$. Für $\lambda=2$
erhalten wir das Gauss-Tableau:
\[
\begin{tabular}{|>{$}c<{$}>{$}c<{$}>{$}c<{$}|}
\hline
1&-3&2\\
2&-4&2\\
2&-3&1\\
\hline
\end{tabular}
\rightarrow
\begin{tabular}{|>{$}c<{$}>{$}c<{$}>{$}c<{$}|}
\hline
1&-3& 2\\
0& 2&-2\\
0& 3&-3\\
\hline
\end{tabular}
\rightarrow
\begin{tabular}{|>{$}c<{$}>{$}c<{$}>{$}c<{$}|}
\hline
1&-3& 2\\
0& 1&-1\\
0& 0& 0\\
\hline
\end{tabular}
\rightarrow
\begin{tabular}{|>{$}c<{$}>{$}c<{$}>{$}c<{$}|}
\hline
1& 0&-1\\
0& 1&-1\\
0& 0& 0\\
\hline
\end{tabular}
\]
Daraus kann man jetzt ablesen, dass ein Eigenvektor zum Eigenwert $2$
alle Komponenten gleich hat, zum Beispiel:
\[
v_2=\begin{pmatrix}1\\1\\1\end{pmatrix}.
\]

Für den Eigenwert $\lambda=1$ bekommt man dagegen folgendes Gauss-Tableau
\[
\begin{tabular}{|>{$}c<{$}>{$}c<{$}>{$}c<{$}|}
\hline
2&-3&2\\
2&-3&2\\
2&-3&2\\
\hline
\end{tabular}
\rightarrow
\begin{tabular}{|>{$}c<{$}>{$}c<{$}>{$}c<{$}|}
\hline
1&-\frac32&1\\
0& 0&0\\
0& 0&0\\
\hline
\end{tabular}.
\]
Wie erwartet gibt es zwei frei wählbare Variablen, also auch zwei linear
unabhängige Eigenvektoren. Die zwei Eigenvektoren
\[
v_{1,1}=\begin{pmatrix}1\\0\\-1 \end{pmatrix},
\qquad
v_{1,2}=\begin{pmatrix}3\\2\\0 \end{pmatrix}.
\]
Das Resultat können wir durch Auswerten des Produktes $Av$ prüfen.
Dazu schreiben wir die Eigenvektoren in die Spalten einer Matrix $B$:
\[
B=
\begin{pmatrix}
1& 1&3\\
1& 0&2\\
1&-1&0
\end{pmatrix}
\]
und berechnen damit
\[
AB=
\begin{pmatrix}
3&-3&2\\
2&-2&2\\
2&-3&3
\end{pmatrix}
\begin{pmatrix}
1& 1&3\\
1& 0&2\\
1&-1&0
\end{pmatrix}
=
\begin{pmatrix}
2& 1&3\\
2& 0&2\\
2&-1&0
\end{pmatrix}
= B \operatorname{diag}(2,1,1).
\]
Die erste Spalte enthält den Eigenvektor zum Eigenwert $2$, und wird 
daher verdoppelt, die anderen Spalten enthalten Eigenvektoren zum
Eigenwert $1$, sie bleiben unverändert.
%\item
%Das Produkt $A^10$ kann jetzt mit Hilfe der Formel
%\[
%A = B\operatorname{diag}(2,1,1) B^{-1}
%\Rightarrow
%A^{10}=
%B\operatorname{diag}(2,1,1)^{10}B^{-1}
%=
%B\operatorname{diag}(1024,1,1)B^{-1}
%\]
%Die Inverse von $B$ kann zum Beispiel mit dem Gauss-Algorithmus gefunden
%werden:
%\[
%B^{-1}
%=
%\begin{pmatrix}
%   2& -3&  2\\
%   2& -3&  1\\
%  -1&  2& -1
%\end{pmatrix}
%\]
%Und durch ausmultiplizieren:
%\begin{align*}
%A^{10}
%&=
%B
%\operatorname{diag}(1024,0,0)
%B^{-1}
%=
%\begin{pmatrix}
%1& 1&3\\
%1& 0&2\\
%1&-1&0
%\end{pmatrix}
%\begin{pmatrix}1024&0&0\\0&1&0\\0&0&1\end{pmatrix}
%\begin{pmatrix}
%   2& -3&  2\\
%   2& -3&  1\\
%  -1&  2& -1
%\end{pmatrix}
%\\
%&=
%\begin{pmatrix}
%1024&1&3\\
%1024&0&2\\
%1024&-2&0
%\end{pmatrix}
%\begin{pmatrix}
%   2& -3&  2\\
%   2& -3&  1\\
%  -1&  2& -1
%\end{pmatrix}
%=
%\begin{pmatrix}
%2047&-3096&2046\\
%2046&-3068&2046\\
%2046&-3069&2047
%\end{pmatrix}.
%\end{align*}
%\end{teilaufgaben}
\end{loesung}

