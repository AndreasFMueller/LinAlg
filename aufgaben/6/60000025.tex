Für eine Animation muss die Position der Hand zwischen den beiden
Positionen in der Abbildung interpoliert werden.
Die Lösungsstrategie für dieses Problem ist, zunächst eine Drehmatrix
zu finden, die die Hand im linken Bild in die Hand im rechten Bild überführt.
Später kann man dann die Drehung mit verschiedenen Winkeln zwischen $0^\circ$
und der Endlage ausführen, dafür muss man natürlich auch den Drehwinkel
der Matrix kennen.
\begin{center}
\begin{tikzpicture}[>=latex,thick]

\begin{scope}[xshift=-4.3cm]
\node at (0,0) {\includeagraphics[width=7.8cm]{wtf.jpg}};
\node[color=red] at (-3.7,-2.1) {$x$};
\node[color=darkgreen] at (3.1,-3.1) {$y$};
\node[color=blue] at (0.2,3.5) {$z$};
\end{scope}

\begin{scope}[xshift=4.3cm]
\node at (0,0) {\includeagraphics[width=7.8cm]{wtf-rotated.jpg}};
\node[color=red] at (-3.7,-2.1) {$x$};
\node[color=darkgreen] at (3.1,-3.1) {$y$};
\node[color=blue] at (0.2,3.5) {$z$};
\end{scope}

\end{tikzpicture}
\end{center}
Im linken Bild zeigt der Mittelfinger in Richtung der $z$-Achse,
die Handfläche ist parallel zur $x$-$y$-Ebene und der Unterarm hat die
Richtung der $x$-Achse.
Im rechten Bild zeigt der Mittelfinger in Richtung vom Nullpunkt auf
den Punkt $P=(-1,1,2)$ und der Unterarm in Richtung vom Nullpunkt auf
den Punkt $Q=(3,1,1)$.

\begin{teilaufgaben}
\item
Finden Sie eine Drehmatrix $D$, die die Hand im linken Bild in die Hand im
rechten Bild überführt.
\item
Finden Sie den Drehwinkel der Drehmatrix $D$.
\end{teilaufgaben}

\begin{loesung}
\begin{teilaufgaben}
\item
Die gesuchte Drehmatrix $D$ bildet $e_1$ (die Richtung des Unterrarms)
und $e_3$ (die Richtung des Mittelfingers) auf die Vektoren
\[
v_1
=
\frac{1}{\sqrt{11}}
\begin{pmatrix}
3\\1\\1
\end{pmatrix}
\qquad\text{und}\qquad
v_3
=
\frac{1}{\sqrt{6}}
\begin{pmatrix}
-1\\1\\2
\end{pmatrix}
\]
ab.
Für die Abbildungsmatrix brauchen wir noch das Bild des zweiten
Standardbasisvektors $e_2$.
Wegen $e_3\times e_1=e_2$ wird $e_2$ auf 
\[
v_2
=
v_3\times v_1
=
\frac{1}{\sqrt{6}}
\begin{pmatrix}
-1\\1\\2
\end{pmatrix}
\times
\frac{1}{\sqrt{11}}
\begin{pmatrix}
3\\1\\1
\end{pmatrix}
=
\frac{1}{\sqrt{66}}
\begin{pmatrix}
1\cdot 1-2\cdot 1\\
2\cdot 3-(-1)\cdot 1\\
(-1)\cdot 1-1\cdot 3
\end{pmatrix}
=
\frac{1}{\sqrt{66}}
\begin{pmatrix}
-1\\7\\-4
\end{pmatrix}
\]
abgebildet.
Damit folgt jetzt, dass die Drehmatrix
\[
D=
\begin{pmatrix}
\frac{3}{\sqrt{11}}&\frac{-1}{\sqrt{66}}&\frac{-1}{\sqrt{6}}\\
\frac{1}{\sqrt{11}}&\frac{ 7}{\sqrt{66}}&\frac{ 1}{\sqrt{6}}\\
\frac{1}{\sqrt{11}}&\frac{-4}{\sqrt{66}}&\frac{ 2}{\sqrt{6}}
\end{pmatrix}
\]
ist.
\item
Mit der Drehwinkelformel
\[
\cos\alpha = \frac{\operatorname{Spur}A-1}2
\]
kann man jetzt auch den Drehwinkel bestimmen:
\begin{align*}
\cos\alpha
&=
\frac12\biggl(
\frac{3}{\sqrt{11}}+\frac{7}{\sqrt{66}}+\frac{2}{\sqrt{6}}
-1\biggr)
\\
&=0.79134
\qquad\Rightarrow\qquad
\alpha =
37.690^\circ.
\qedhere
\end{align*}
\end{teilaufgaben}
\end{loesung}

\begin{bewertung}
\begin{teilaufgaben}
\item
Normierung ({\bf N}) 1 Punkt,
dritter Vektor ({\bf D}) 1 Punkt,
Kreuzprodukt ({\bf K}) 1 Punkt,
Matrix ({\bf M}) 1 Punkt,
\item
Spurformel ({\bf S}) 1 Punkt,
Drehwinkel ({\bf W}) 1 Punkt.
\end{teilaufgaben}
\end{bewertung}





