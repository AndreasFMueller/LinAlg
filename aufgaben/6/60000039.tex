Das Shamisen
(\begin{CJK}{UTF8}{min}しゃみせん\end{CJK},
\begin{CJK}{UTF8}{min}三味線\end{CJK})
ist eine dreisaitige japanische Laute.
In der Abbildung~\ref{60000039:image} liegt das Shamisen links in
der $x$-$y$-Ebene mit dem Hals des Shamisen in einem Winkel von $45^\circ$ zur
$y$-Achse.
Im rechten Bild befindet es sich in der Stellung, wie es der Spieler
beim Spielen hält, nämlich in der $x$-$z$-Ebene mit einem Winkel
von $60^\circ$ zur Vertikalen.
\begin{teilaufgaben}
\item 
Berechnen Sie den Einheitsvektor mit der Richtung des Halses des
Shamisen für beide Abbildungen.
\item
Finden Sie eine Drehmatrix, die das Shamisen in der Abbildung links
in die Stellung in der Abbildung rechts dreht.
\item
Berechnen Sie den Drehwinkel dieser Drehmatrix.
\end{teilaufgaben}

\begin{figure}[h]
\centering
\includeagraphics[]{shamizen.pdf}
\caption{Das Shamisen liegt in der linken Abbildung in der $x$-$y$-Ebene
in einem Winkel von $45^\circ$ zur $y$-Achse, in der rechten Abbildung
ist es in die Stellung gedreht, in der es der Spieler hält, nämlich in
der $x$-$z$-Ebene mit einem Winkel von $60^\circ$ zur Vertikalen.
\label{60000039:image}}
\end{figure}

\begin{hinweis}
Verwenden Sie den Taschenrechner zur Berechnung der Matrix $D$.
\end{hinweis}

\begin{loesung}
\begin{teilaufgaben}
\item Da die roten Vektoren in den Koordinatenebenen liegen und die
Winkel bekannt sind, können ihre Koordinaten sofort als
\[
\vec{a}_1
=
\begin{pmatrix*}
-\frac{\!\sqrt{2}}2\\
\frac{\!\sqrt{2}}2\\
0
\end{pmatrix*}
\qquad\text{und}\qquad
\vec{b}_1
=
\begin{pmatrix*}
-\frac{\!\sqrt{3}}2\\
0\\
\frac12
\end{pmatrix*}
\]
angegeben werden.
\item
Zur Berechnung der Drehmatrix sind zwei weitere Vektoren nötig.
Wir wählen dazu den Vektor senkrecht auf die Achse in der Ebene 
des Shamisen.
In den beiden Bildern sind dies die Richtungen
\[
\vec{a}_2
=
\begin{pmatrix*}
-\frac{\!\sqrt{2}}2\\
-\frac{\!\sqrt{2}}2\\
0
\end{pmatrix*}
\qquad\text{und}\qquad
\vec{b}_2
=
\begin{pmatrix*}
\frac12\\
0\\
\frac{\!\sqrt{3}}2
\end{pmatrix*}
\]
Als dritten Vektor verwenden wir den Normalenvektor auf der Ebene
des Shamisen.
In der Lage links ist dies der Standardbasisvektor $\vec{e}_3$, in der
rechten Lage ist es $\vec{e}_2$.
Damit erhalten wir die Gleichung
\[
D
\underbrace{
\begin{pmatrix*}
-\frac{\!\sqrt{2}}2 & -\frac{\!\sqrt{2}}2 & 0 \\
 \frac{\!\sqrt{2}}2 & -\frac{\!\sqrt{2}}2 & 0 \\
         0          &          0          & 1
\end{pmatrix*}
}_{\displaystyle =A}
=
\underbrace{
\begin{pmatrix*}
-\frac{\!\sqrt{3}}2 &     \frac12        & 0 \\
        0           &         0          & 1 \\
     \frac12        & \frac{\!\sqrt{3}}2 & 0
\end{pmatrix*}
}_{\displaystyle =B}
\]
für die Drehmatrix $D$.
Durch Auflösen nach $D$ erhalten wir
\[
D
=
BA^{-1}
=
\begin{pmatrix*}[r]
   0.258819& -0.965925&  0.000000\\
   0.000000&  0.000000&  1.000000\\
  -0.965925& -0.258819&  0.000000
\end{pmatrix*}.
\]
\item
Die Spur von $D$ ist das Element in der linken oberen Ecke von $D$.
Daraus kann man den Drehwinkel mit Hilfe der Spurformel finden:
\begin{align*}
\cos\alpha
&=
\frac{\operatorname{Spur}D-1}2
=
\frac{0.258819-1}2
= -0.370590
&&\Rightarrow&
\alpha &= 111.752^\circ
\qedhere
\end{align*}
\end{teilaufgaben}
\end{loesung}

\begin{bewertung}
Bestimmung der roten Vektoren ({\bf V}) je 1 Punkt,
Bestimmung von zwei weiteren Vektorpaaren ({\bf P}) je 1 Punkt,
Bestimmung der Drehmatrix $D$ ({\bf D}) 1 Punkt,
Drehwinkel $\alpha$ ({\bf A}) 1 Punkt.
\end{bewertung}
