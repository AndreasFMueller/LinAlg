%
% shamizen.tex
%
% (c) 2021 Prof Dr Andreas Müller, OST Ostschweizer Fachhochschule
%
\documentclass[tikz]{standalone}
\usepackage{times}
\usepackage{amsmath}
\usepackage{txfonts}
\usepackage[utf8]{inputenc}
\usepackage{graphics}
\usetikzlibrary{arrows,intersections,math}
\usepackage{ifthen}
\begin{document}

\def\punkt#1#2{
	\node at #1 {$#2$};
}

\newboolean{showgrid}
\setboolean{showgrid}{false}
%\setboolean{showgrid}{true}
\def\breite{8}
\def\hoehe{4}

\begin{tikzpicture}[>=latex,thick]

% Povray Bild
\begin{scope}[xshift=-4cm]
\node at (0,0) {\includegraphics[width=8.0cm]{liegend.jpg}};
\end{scope}
\begin{scope}[xshift=4cm]
\node at (0,0) {\includegraphics[width=8.0cm]{spielend.jpg}};
\end{scope}

% Gitter
\ifthenelse{\boolean{showgrid}}{
\draw[step=0.1,line width=0.1pt] (-\breite,-\hoehe) grid (\breite, \hoehe);
\draw[step=0.5,line width=0.4pt] (-\breite,-\hoehe) grid (\breite, \hoehe);
\draw                            (-\breite,-\hoehe) grid (\breite, \hoehe);
\fill (0,0) circle[radius=0.05];
}{}

\coordinate (x) at (-7.9,-2.7);
\coordinate (y) at (-0.6,-2.3);
\coordinate (z) at (-4.8,3.0);
\punkt{(x)}{x}
\punkt{(y)}{y}
\punkt{(z)}{z}
\node at (-2.6,-2.15) {$45^\circ$};

\coordinate (xprime) at (0.1,-2.7);
\coordinate (yprime) at (7.4,-2.3);
\coordinate (zprime) at (3.1,3.0);
\punkt{(xprime)}{x}
\punkt{(yprime)}{y}
\punkt{(zprime)}{z}
\node at (3.5,0.3) {$60^\circ$};


\end{tikzpicture}

\end{document}

