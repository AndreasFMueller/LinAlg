Über einem gleichseitigen Fünfeck in der $x$-$y$-Ebene mit Umkreisradius $1$ 
wird eine Doppelpyramide errichtet, wodurch ein Dekaeder (10-seitiges Polyeder) entsteht.
Alle Kanten des Dekaeders haben die gleiche Länge und die Spitzen der Pyramiden befinden sich
in den Punkten 
\[
\left(0,0,\frac{\sqrt{5}-1}{2}\right) \qquad \text{und}\qquad
\left(0,0,-\frac{\sqrt{5}-1}{2}\right).
\]

\begin{center}
\begin{tikzpicture}[thick,>=latex]
\begin{scope}[xshift=-3.7cm]
\node at (0,0) {\includeagraphics[width=6.8cm]{left.jpg}};
\node at (-3.5,-0.8) {$x$};
\node at (2.3,-1.55) {$y$};
\node at (0,3.65) {$z$};
\node at (1.6,-1.5) {$A$};
\node at (0.3,1.9) {$B$};
\end{scope}
\begin{scope}[xshift=3.7cm]
\node at (0,0) {\includeagraphics[width=6.8cm]{right.jpg}};
\node at (-3.5,-0.8) {$x$};
\node at (2.3,-1.55) {$y$};
\node at (0,3.65) {$z$};
\node at (2.4,0.2) {$A'$};
\node at (0.7,-0.4) {$B'$};
\end{scope}
\end{tikzpicture}
\end{center}

\begin{teilaufgaben}
\item
Finden Sie eine Abbildungsmatrix $M$, die das Dekaeder links im Bild auf das
Dekaeder rechts im Bild abbildet.
\item
Handelt es sich bei der gefundenen Abbildungsmatrix $M$ um eine Drehmatrix?
Falls ja, berechnen sie den Drehwinkel der Drehmatrix.
\end{teilaufgaben}

\thema{Abbildungsmatrix}
\thema{orthogonale Matrix}

\begin{loesung}
\begin{teilaufgaben}
\item Die gesuchte Abbildungsmatrix $M$ bildet die Standardbasis-Vektoren wie folgt ab:
\[
\begin{aligned}
\vec e_1&\mapsto \begin{pmatrix}0\\0\\1\end{pmatrix},
&
\vec e_2&\mapsto \begin{pmatrix}-1\\0\\0\end{pmatrix},
&&\text{und}
&
\vec e_3&\mapsto \begin{pmatrix}0\\1\\0\end{pmatrix}.
\end{aligned}
\]
Folglich ist
\[
M
=
\begin{pmatrix}
0 & -1 & 0\\
0 & 0 & 1\\
1 & 0 & 0
\end{pmatrix}.
\]
Statt den Standardbasis-Vektoren hätte man auch andere Punkte verwenden können.
Die Abbildungsmatrix $M$ bildet die Spitze $\left(0,0,\frac{\sqrt{5}-1}{2}\right)$ auf 
$\left(0,\frac{\sqrt{5}-1}{2},0\right)$ ab und $A = (0,1,0)$ auf $A'=(-1,0,0)$.
Um die Abbildungsmatrix zu bestimmen brauchen wir aber noch einen 
weiteren Punkt. Man liest aus dem Bild ab, dass 
\[
  \vec c = \begin{pmatrix}\cos(\alpha)\\ \sin(\alpha)\\0 \end{pmatrix}
  \quad\text{mit}\quad \alpha = \dfrac{\pi}{2}-\dfrac{2\pi}{5} = \dfrac{\pi}{10}
\]
abgebildet wird auf 
\[
  \vec c\,' = \begin{pmatrix}-\sin(\alpha)\\ 0\\\cos(\alpha) \end{pmatrix}.
\]
Wir haben also die Abbildungen
\[
\begin{aligned}
M
\underbrace{\begin{pmatrix}
0 & 0 &\cos(\alpha)\\
0 & 1 & \sin(\alpha)\\
\frac{\sqrt{5}-1}{2} &0 & 0
\end{pmatrix}}_{\displaystyle P}
&=
\underbrace{
\begin{pmatrix}
0                   &-1&-\sin(\alpha)\\
\frac{\sqrt{5}-1}{2}& 0&0\\
0                   & 0&\cos(\alpha)
\end{pmatrix}}_{\displaystyle P'}
&&\Rightarrow&
M
&=
P'P^{-1}
\end{aligned}.
\]
Wir berechnen die inverse Matrix mit dem Gauss-Algorithmus
\begin{align*}
\begin{tabular}{|>{$}c<{$}>{$}c<{$}>{$}c<{$}|>{$}c<{$}>{$}c<{$}>{$}c<{$}|}
\hline
0 & 0 &\cos(\alpha)&1&0&0\\
0 & 1 & \sin(\alpha)&0&1&0\\
\frac{\sqrt{5}-1}{2} &0 & 0&0&0&1\\
\hline
\end{tabular}
&\rightarrow
\begin{tabular}{|>{$}c<{$}>{$}c<{$}>{$}c<{$}|>{$}c<{$}>{$}c<{$}>{$}c<{$}|}
\hline
\frac{\sqrt{5}-1}{2} &0 & 0&0&0&1\\
0 & 1 & \sin(\alpha)&0&1&0\\
0 & 0 &\cos(\alpha)&1&0&0\\
\hline
\end{tabular}\\
&\rightarrow
\begin{tabular}{|>{$}c<{$}>{$}c<{$}>{$}c<{$}|>{$}c<{$}>{$}c<{$}>{$}c<{$}|}
\hline
1 &0 & 0&0&0&\frac{2}{\sqrt{5}-1}\\
0 & 1 & \sin(\alpha)&0&1&0\\
0 & 0 &\cos(\alpha)&1&0&0\\
\hline
\end{tabular}\\
&\rightarrow
\begin{tabular}{|>{$}c<{$}>{$}c<{$}>{$}c<{$}|>{$}c<{$}>{$}c<{$}>{$}c<{$}|}
\hline
1 &0 & 0&0&0&\frac{2}{\sqrt{5}-1}\\
0 & 1 & 0&-\frac{\sin(\alpha)}{\cos(\alpha)}&1&0\\
0 & 0 &1&\frac{1}{\cos(\alpha)}&0&0\\
\hline
\end{tabular}
\end{align*}
Kontrolle:
\[
\begin{pmatrix}
0 & 0 &\cos(\alpha)\\
0 & 1 & \sin(\alpha)\\
\frac{\sqrt{5}-1}{2} &0 & 0
\end{pmatrix}
\begin{pmatrix}
0&0&\frac{2}{\sqrt{5}-1}\\
-\frac{\sin(\alpha)}{\cos(\alpha)}&1&0\\
\frac{1}{\cos(\alpha)}&0&0\\
\end{pmatrix}
=
\begin{pmatrix}
1&0&0\\
0&1&0\\
0&0&1
\end{pmatrix}.
\]
Daraus erhalten wir, wie bereits oben, die Matrix 
\[
M=
\begin{pmatrix}
0                   &-1&-\sin(\alpha)\\
\frac{\sqrt{5}-1}{2}& 0&0\\
0                   & 0&\cos(\alpha)
\end{pmatrix}
\begin{pmatrix}
0&0&\frac{2}{\sqrt{5}-1}\\
-\frac{\sin(\alpha)}{\cos(\alpha)}&1&0\\
\frac{1}{\cos(\alpha)}&0&0\\
\end{pmatrix}
=
\begin{pmatrix}
0&-1&0\\
0&0&1\\
1&0&0
\end{pmatrix}.
\]
\item
Die Abbildungsmatrix $M$ ist eine Drehmatrix falls gilt: $MM^t= E $ und $\det(M) = 1$.
Wir berechnen also diese beiden Ausdrücke:
\[
MM^t = 
\begin{pmatrix}
0&-1&0\\
0&0&1\\
1&0&0
\end{pmatrix}
\begin{pmatrix}
0&0&1\\
-1&0&0\\
0 &1&0
\end{pmatrix}
=
\begin{pmatrix}
1&0&0\\
0&1&0\\
0 &0&1
\end{pmatrix} = E
\]
\[
\det(M) = 
\begin{vmatrix}
0&-1&0\\
0&0&1\\
1&0&0
\end{vmatrix}
= 
1\cdot(-1)\cdot 1 = -1.
\]
Da $\det(M) = -1 \neq 1$ ist, ist die Abbildungsmatrix $M$ folglich keine Drehmatrix.
\end{teilaufgaben}
\end{loesung}

\begin{bewertung}
Abbildungspunkte $A$ und Spitze ({\bf A}) 1 Punkt,
Dritter Abbildungspunkt ({\bf D}) 1 Punkt,
Ermittlung der Abbildungsmatrix ({\bf M}) 2 Punkte,
Bedingungen für Drehmatrix ({\bf B}) 1 Punkt,
Berechnung und Schlussfolgerung ({\bf S}) 1 Punkt.
\end{bewertung}

