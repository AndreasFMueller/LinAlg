Ist die Matrix
\[
A=\begin{pmatrix}7&4\\4&22\end{pmatrix}
\]
diagonalisierbar? Wenn ja, geben Sie eine Basis an, in der $A$ diagonal
ist.

\begin{loesung}
Eine geeignete Basis besteht aus den Eigenvektoren von $A$. Weil $A$
symmetrisch ist, lässt sich immer eine Eigenvektorbasis find.

Das charakteristische Polynom von $A$ ist
\begin{align*}
\det(A-\lambda E)
&=\left|\begin{matrix}7-\lambda&4\\4&22-\lambda\end{matrix}\right|
=(7-\lambda)(22-\lambda)-16=\lambda^2-29\lambda+154-16=\lambda^2-29\lambda-138
\end{align*}
Die Nullstellen können mit der Lösungsformel für quadratische Gleichungen
gefunden werden:
\begin{align*}
\lambda_{\pm}&=\frac{29}{2}\pm\frac{1}{2}\sqrt{29^2+4 \cdot 138}
=\frac{29}{2}\pm\frac{\sqrt{239}}{2}=\frac{29\pm17}{2}
=\begin{cases}
23\\6
\end{cases}
\end{align*}
Für jeden Eigenwert muss jetzt mit Hilfe des Gaussalgorithmus ein
Eigenvektor gefunden werden.

Für $\lambda_+=19$ findet man
\[
\begin{tabular}{|>{$}c<{$}>{$}c<{$}|}
\hline
7-23&4\\
4&22-23\\
\hline
\end{tabular}
=
\begin{tabular}{|>{$}c<{$}>{$}c<{$}|}
\hline
-16&4\\
4&-1\\
\hline
\end{tabular}
\rightarrow
\begin{tabular}{|>{$}c<{$}>{$}c<{$}|}
\hline
1&-\frac14\\
0&0\\
\hline
\end{tabular}
\]
Die zweite Koordinate ist wie erwartet frei wählbar, indem wir sie auf
$2$ setzen erhalten wir einen Eigenvektor ohne Brüche
\[
v_+=\begin{pmatrix}1\\4\end{pmatrix}.
\]

Für $\lambda_-=6$ erhalten wir
\[
\begin{tabular}{|>{$}c<{$}>{$}c<{$}|}
\hline
7-6&4\\
4&22-6\\
\hline
\end{tabular}
=
\begin{tabular}{|>{$}c<{$}>{$}c<{$}|}
\hline
1&4\\
4&16\\
\hline
\end{tabular}
\rightarrow
\begin{tabular}{|>{$}c<{$}>{$}c<{$}|}
\hline
1&4\\
0&0\\
\hline
\end{tabular}.
\]
Wieder ist die zweite Variable frei wählbar, die Wahl $-1$ liefert 
den Eigenvektor
\[
v_-=\begin{pmatrix}4\\-1\end{pmatrix}.
\]

Kontrolle: Wir kontrollieren die Rechnung durch Multiplikation von $A$
mit den gefundenen Eigenvektoren:
\begin{align*}
Av_+&=\begin{pmatrix}7&4\\4&22\end{pmatrix}\begin{pmatrix}1\\4\end{pmatrix}
=\begin{pmatrix}4+18\\ 12 + 21\end{pmatrix}
=\begin{pmatrix}23\\ 92\end{pmatrix}=23\begin{pmatrix}1\\4\end{pmatrix}
=\lambda_+v_+,\\
Av_-&=\begin{pmatrix}7&4\\4&22\end{pmatrix}\begin{pmatrix}4\\-1\end{pmatrix}
=\begin{pmatrix}24\\-6 \end{pmatrix}
=6\begin{pmatrix}4\\-1\end{pmatrix}
=\lambda_-v_-.
\end{align*}
Ausserdem sind die beiden Vektoren orthogonal, wie man das von den Eigenvektoren
einer symmetrischen Matrix zu verschiedenen Eigenwerten erwartet:
\[
v_+\cdot v_-=
\begin{pmatrix}1\\4\end{pmatrix}
\cdot
\begin{pmatrix}4\\-1\end{pmatrix}
=1\cdot 4 + 4\cdot(-1)=0.
\qedhere
\]
\end{loesung}

\begin{diskussion}
Die Aufgabe entspricht dem pythagoräischen Tripel $(15,8,17)$.
\end{diskussion}

\begin{bewertung}
Charakteristisches Polynom ({\bf X}) 1 Punkt,
Nullstellen ({\bf N}) 1 Punkt,
Gleichungssysteme für $\lambda_+$ und $\lambda_-$ ({\bf G$\pm$}) je 1 Punkt,
Eigenvektoren ({\bf E$\pm$}) je 1 Punkt.
\end{bewertung}

