Ein Fluzeug geht von der normalen Fluglage in Abbildung~\ref{60000035:fig}
links in den Sturzflug über, bei dem die Nase die Richtung des Vektors
$\transpose{(1,1,-1)}$ hat und die Flügel horizontal sind.
\begin{figure}[h]
\centering
\includeagraphics[]{plane.pdf}
\caption{
Zwei Fluglagen des Flugzeugs in Aufgabe~\ref{60000035}
vor und nach dem Übergang in den Sturzflug.
\label{60000035:fig}}
\end{figure}
\begin{teilaufgaben}
\item Finden Sie eine Drehmatrix, die diese Drehung erzeugt.
\item Wie gross ist der Drehwinkel?
\end{teilaufgaben}

\begin{loesung}
\begin{teilaufgaben}
\item
Wir müssen Vektoren finden, auf die die Standardbasisvektoren in der 
linken Abbildungen abgebildet werden.
Der Vektor $\vec{e}_1$ wird auf den Vektor
\[
\vec{a}_1
=
\frac{1}{\sqrt{3}}\begin{pmatrix*}[r] 1\\1\\-1\end{pmatrix*}
\]
abgebildet.
Der zweite Vektor hat die Richtung
\[
\begin{pmatrix*}[r]-1\\1\\0\end{pmatrix*}
\qquad\Rightarrow\qquad
\vec{a}_2
=
\frac{1}{\sqrt{2}}
\begin{pmatrix*}[r]-1\\1\\0\end{pmatrix*}
\]
Der dritte Vektor ist das Vektorprodukt der beiden Vektoren:
\[
\vec{a}_1\times \vec{a}_2
=
\frac{1}{\sqrt{3}}
\begin{pmatrix*}[r] 1\\1\\-1\end{pmatrix*}
\times
\frac{1}{\sqrt{2}}
\begin{pmatrix*}[r]-1\\1\\0\end{pmatrix*}
=
\frac{1}{\sqrt{6}}
\begin{pmatrix}
1\cdot 0-(-1)\cdot 1\\
(-1)\cdot(-1)-0\cdot 1\\
1\cdot 1-(-1)\cdot 1
\end{pmatrix}
=
\frac{1}{\sqrt{6}}
\begin{pmatrix}
1\\1\\2
\end{pmatrix}.
\]
Daraus lässt sich jetzt die Drehmatrix
\[
R = \begin{pmatrix*}[r]
 \frac{1}{\sqrt{3}}&-\frac{1}{\sqrt{2}}&\frac{1}{\sqrt{6}}\\
 \frac{1}{\sqrt{3}}& \frac{1}{\sqrt{2}}&\frac{1}{\sqrt{6}}\\
-\frac{1}{\sqrt{3}}&       0           &\frac{2}{\sqrt{6}}
\end{pmatrix*}
\]
ablesen.
\item
Den Drehwinkel kann man mit der Spurformel
\[
\cos\alpha
=
\frac{\operatorname{Spur}A-1}{2}
=
\frac{\frac{1}{\sqrt{3}}+\frac{1}{\sqrt{2}}+\frac{2}{\sqrt{6}}-1}{2}
=
0.5505
\qquad\Rightarrow\qquad
\alpha = 56.600^\circ.
\qedhere
\]
\end{teilaufgaben}
\end{loesung}

\begin{bewertung}
1.~Spalte normiert ({\bf V1}) 1 Punkt,
horizontaler Vektor ({\bf V2}) 1 Punkt,
dritter Vektor mit Vektorprodukt ({\bf V3}) 1 Punkt,
Drehmatrix ({\bf D}) 1 Punkt,
Spurformel ({\bf S}) 1 Punkt,
Drehwinkel ({\bf W}) 1 Punkt.
\end{bewertung}
