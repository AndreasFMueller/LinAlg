Ist die Matrix
\[
A
=
\begin{pmatrix}
5&1\\
2&6
\end{pmatrix}
\]
diagonalisierbar?
Wenn ja, geben Sie eine Basis an, in der $A$ diagonal ist.

\begin{loesung}
Da die Matrix nicht symmetrisch ist, k"onnen wir die Frage nur durch direktes
Ausrechnen der Eigenwerte und Eigenvektoren entscheiden.

Das charakteristische Polynom ist
\[
\chi_A(\lambda)
=
\left|\,\begin{matrix}5&1\\2&6\end{matrix}\,\right|
=
(5-\lambda)(6-\lambda)-2
=
\lambda^2-11\lambda+28.
\]
Es hat die Nullstellen
\[
\lambda_{\pm}
=
\frac{11}{2}\pm\sqrt{\frac{121}{4}-28}
=
\frac{11}{2}\pm\sqrt{\frac{121-112}{4}}
=
\frac{11}{2}\pm\frac{3}{2}
=
\begin{cases}7\\4\end{cases}
\]
Da die Eigenwerte verschieden sind, gibt es auch zwei linear unabh"angige
Eigenvektoren, die wir finden k"onnen, indem wir das Gleichungssystem
$(A-\lambda)v=0$ mit dem Gauss-Algorithmus l"osen.

F"ur $\lambda_+=7$ finden wir:
\begin{align*}
\begin{tabular}{|>{$}c<{$}>{$}c<{$}|}
\hline
5-\lambda_+&1          \\
2          &6-\lambda_+\\
\hline
\end{tabular}
&\rightarrow
\begin{tabular}{|>{$}c<{$}>{$}c<{$}|}
\hline
5-7&1  \\
2  &6-7\\
\hline
\end{tabular}
\rightarrow
\begin{tabular}{|>{$}c<{$}>{$}c<{$}|}
\hline
-2& 1\\
 2&-1\\
\hline
\end{tabular}
\rightarrow
\begin{tabular}{|>{$}c<{$}>{$}c<{$}|}
\hline
 1&-\frac12\\
 0&0       \\
\hline
\end{tabular}
\end{align*}
Daraus kann man ablesen, dass
\[
v_+=\begin{pmatrix}1\\2\end{pmatrix}
\]
ein Eigenvektor ist.

F"ur $\lambda_-=4$ finden wir:
\begin{align*}
\begin{tabular}{|>{$}c<{$}>{$}c<{$}|}
\hline
5-\lambda_-&1          \\
2          &6-\lambda_-\\
\hline
\end{tabular}
&
\rightarrow
\begin{tabular}{|>{$}c<{$}>{$}c<{$}|}
\hline
5-4&1  \\
2  &6-4\\
\hline
\end{tabular}
\rightarrow
\begin{tabular}{|>{$}c<{$}>{$}c<{$}|}
\hline
1&1\\
2&2\\
\hline
\end{tabular}
\rightarrow
\begin{tabular}{|>{$}c<{$}>{$}c<{$}|}
\hline
1&1\\
0&0\\
\hline
\end{tabular}
\end{align*}
Als zweiten Eigenvektor $v_-$ k"onnen wir also
\[
v_-=\begin{pmatrix}1\\-1\end{pmatrix}
\]
verwenden.
Die gesuchte Eigenbasis ist daher
\[
B
=
\{
v_+,
v_-
\}
=
\biggl\{
\begin{pmatrix}1\\2\end{pmatrix},
\begin{pmatrix}1\\-1\end{pmatrix}
\biggr\}.
\]
Wir kontrollieren die L"osung durch nachrechnen:
\begin{align*}
Av_+
&=
\begin{pmatrix}
5&1\\
2&6
\end{pmatrix}
\begin{pmatrix}1\\2\end{pmatrix}
=
\begin{pmatrix} 7\\14 \end{pmatrix}
=
7
\begin{pmatrix}1\\2\end{pmatrix}
=
\lambda_+v_+,
\\
Av_-
&=
\begin{pmatrix}
5&1\\
2&6
\end{pmatrix}
\begin{pmatrix}1\\-1\end{pmatrix}
=
\begin{pmatrix}4\\-4\end{pmatrix}
=
4\begin{pmatrix}1\\-1\end{pmatrix}
=
\lambda_-v_-.
\qedhere
\end{align*}
\end{loesung}

\begin{bewertung}
\begin{teilaufgaben}
\item
Methode zur Bestimmung der Werte f"ur $c$ ({\bf M}) 1 Punkt,
Werte $1$ und $2$ f"ur $c$ ({\bf C}) 1 Punkt.
\item
F"ur jeden Wert von $c$: L"osung des Gleichungssystems f"ur diesen Wert von $c$ 
({\bf G$\mathstrut_{1,2}$}) 1 Punkt, L"osungsmenge ({\bf L$\mathstrut_{1,2}$})
1 Punkt.
\end{teilaufgaben}
\end{bewertung}

