Im Jahr 1934 beobachteten Wander Johannes de Haas, Jan Hendrik de Boer
und G.~J.~van de Berg, dass der elektrische Widerstand einer 
Gold-Probe mit magnetischen Verunreinigungen nicht wie erwartet bei
tieferer Temperatur immer weiter abnahm, sondern im Gegenteil
logarithmisch anstieg.
1964 konnte der japanische theoretische Physiker Jun Kondo
(\begin{CJK}{UTF8}{min}
近藤\; 淳
\end{CJK})
diesen Effekt erklären als Resultat der Streuung der Leitungselektronen
an lokalisierten magnetischen Elektronen.
Man hat für die Temperaturabhängigkeit des Widerstandes die Formel
\[
\varrho(T)
=
\varrho_0 + aT^2 + c_m\log \frac{\mu}{T} + bT^5
\]
gefunden, der logarithmische dritte Term beschreibt den Kondo-Effekt.
In der Formel müssen die temperaturunabhängigen Koeffizienten
$\varrho_0$, $a$, $c_m$, $\mu$ und $b$ experimentell ermittelt werden.

Zwischen den Koeffizienten gibt es eine Abhängigkeit.
Ändert man $\mu$ um einen Faktor $s$, dann ändert dies den Logarithmus
um einen additiven Wert $\log s$, den man in $\varrho_0$ absorbieren kann.
Man kann sich bei $\mu$ also zunächst auf einen bestimmten Wert festlegen.
Insbesondere darf man also so tun, als ob $\mu$ bekannt wäre.

\begin{teilaufgaben}
\item
Wieviele Messungen sind mindestens nötig, um die Koeffizienten zu bestimmen?
\item
Wie kann man die bestmöglichen Werte für die verbleibenden Koeffizienten
aus einer Messreihe von $n$ Widerständen $\varrho_i$ und zugehörigen
Temperaturen $T_i$ bekommen?
\end{teilaufgaben}

\thema{Least Squares}

\begin{loesung}
Die Koeffizienten müssen die Gleichungen
\begin{equation}
\varrho_i
=
{\color{red}\varrho_0}
+
{\color{red}a}T_i^2
+
{\color{red}c_m}\log\frac{\mu}{T_i}
+
{\color{red}b}T_i^5
\label{60000026:eqn}
\end{equation}
mit Unbekannten in {\color{red}rot} erfüllen.
Um die $4$ verbleibenden Unbekannten zu bestimmen, sind mindestens $4$
Messungen erforderlich, für gute Resultate wird man viel mehr Messungen
machen.

Wegen der Messfehler wird das Gleichungssytem \eqref{60000026:eqn}
nicht exakt gehen, die Gleichungen sind überbestimmt.
Wir verwenden daher das Least-Squares Verfahren.
Die Matrix $A$ und der Vektor $b$ sind
\[
A=
\begin{pmatrix}
1 & T_1^2 & \log\frac{\mu}{T_1} & T_1^5 \\
1 & T_2^2 & \log\frac{\mu}{T_2} & T_2^5 \\
\vdots&\vdots&\vdots&\vdots \\
1 & T_n^2 & \log\frac{\mu}{T_n} & T_n^5 
\end{pmatrix}
\qquad
\text{bzw.}
\qquad
b=
\begin{pmatrix}
\varrho_1\\
\varrho_2\\
\vdots\\
\varrho_n
\end{pmatrix}
\]
Um die Koeffizienten zu bestimmen, muss man das Gleichungssystem
\[
A^tA {\color{red}x} = A^tb
\]
lösen, wobei 
\[
{\color{red}x}
=
\begin{pmatrix}
{\color{red}\varrho_0}\\
{\color{red}a}\\
{\color{red}c_m}\\
{\color{red}b}
\end{pmatrix}
\]
Die Matrix $A^tA$ und den Vektor $A^tb$ kann man explizit berechnen:
\begin{align*}
A^tA
&=
\begin{pmatrix}
n
	&\displaystyle\sum_{i=1}^n T_i^2
		&\displaystyle\sum_{i=1}^n \log\frac{\mu}{T_i}
			&\displaystyle\sum_{i=1}^n T_i^5
\\
\displaystyle\sum_{i=1}^n T_i^2
	&\displaystyle \sum_{i=1}^n T_i^4
		&\displaystyle \sum_{i=1}^n T_i^2\log\frac{\mu}{T_i}
			&\displaystyle \sum_{i=1}^n T_i^7
\\
\displaystyle\sum_{i=1}^n \log\frac{\mu}{T_i}
	&\displaystyle \sum_{i=1}^n T_i^2\log\frac{\mu}{T_i}
		&\displaystyle \sum_{i=1}^n \biggl(\log\frac{\mu}{T_i}\biggr)^2
			&\displaystyle \sum_{i=1}^n T_i^5\log\frac{\mu}{T_i}
\\
\displaystyle\sum_{i=1}^n T_i^5 
	&\displaystyle \sum_{i=1}^n T_i^7
		&\displaystyle \sum_{i=1}^n T_i^5\log\frac{\mu}{T_i}
			&\displaystyle \sum_{i=1}^nT_i^{10}
\end{pmatrix},
&
b
&=
\begin{pmatrix}
\displaystyle\sum_{i=1}^n \varrho_i \\
\displaystyle\sum_{i=1}^n \varrho_i T_i^2 \\
\displaystyle\sum_{i=1}^n \varrho_i \log\frac{\mu}{T_i} \\
\displaystyle\sum_{i=1}^n \varrho_i T_i^5
\end{pmatrix}
\end{align*}
\end{loesung}

\begin{bewertung}
Anzahl Messungen ({\bf N}) 1 Punkt,
Least-Squares-Prinzip ({\bf L}) 1 Punkt,
Matrix $A$ ({\bf A}) 2 Punkte,
Matrix $b$ ({\bf B}) 1 Punkt,
Gleichungssytem $A^tA{\color{red}}x=A^tb$ ({\bf G}) 1 Punkt.
\end{bewertung}
