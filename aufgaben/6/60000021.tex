Ist die Matrix
\[
A
=
\begin{pmatrix}
2&-1\\
3& 6
\end{pmatrix}
\]
diagonalisierbar?
Wenn ja, geben Sie eine Basis an, in der $A$ diagonal ist.

\begin{loesung}
Wir bestimmen zunächst die Eigenwerte von $A$ mit Hilfe des charakteristischen
Polynoms:
\[
\chi_A(\lambda)=\left|\;\begin{matrix}
2-\lambda&        -1\\
        3& 6-\lambda
\end{matrix}\;\right|
=(2-\lambda)(6-\lambda)+3=\lambda^2-8\lambda+12+3
=\lambda^2-8\lambda+15.
\]
Die Nullstellen von $\chi_A(\lambda)$ können mit der Lösungsformel
für quadratische Gleichungen gefunden werden:
\[
\lambda_{\pm}
=
4\pm\sqrt{16-15}=4\pm 1=\begin{cases}\;5\\\;3.\end{cases}
\]
Da die Eigenwerte verschieden sind, werden wir zwei Eigenvektoren finden,
und können daher die Frage nach der Diagonalisierbarkeit bereits mit
``ja'' beantworten.

Um die Eigenvektoren zu bestimmen, lösen wir das Gleichungssystem
$(A-\lambda E)x=0$ mit $\lambda=\lambda_{\pm}$.
Im Falle $\lambda=\lambda_+ = 5$ 
\begin{align*}
\begin{tabular}{|>{$}c<{$}>{$}c<{$}|>{$}c<{$}|}
\hline
2-\lambda_+&         -1&0\\
          3&6-\lambda_+&0\\
\hline
\end{tabular}
&=
\begin{tabular}{|>{$}c<{$}>{$}c<{$}|>{$}c<{$}|}
\hline
-3&-1&0\\
 3& 1&0\\
\hline
\end{tabular}
\\
&\rightarrow
\begin{tabular}{|>{$}c<{$}>{$}c<{$}|>{$}c<{$}|}
\hline
 1& \frac13&0\\
 0& 0&0\\
\hline
\end{tabular}
&
v_+&=\begin{pmatrix}
1\\-3
\end{pmatrix}.
\end{align*}
Zur Kontrolle rechnen wir $Av_+$ nach:
\[
Av_+=
\begin{pmatrix}
2&-1\\ 3&6
\end{pmatrix}
\begin{pmatrix} 1\\-3 \end{pmatrix}
=
\begin{pmatrix}
5\\-15
\end{pmatrix}
=\lambda_+v_+.
\]
Entsprechend finden wir den Eigenvektor für $\lambda=\lambda_-$:
\begin{align*}
\begin{tabular}{|>{$}c<{$}>{$}c<{$}|>{$}c<{$}|}
\hline
2-\lambda_-&         -1&0\\
          3&6-\lambda_-&0\\
\hline
\end{tabular}
&=
\begin{tabular}{|>{$}c<{$}>{$}c<{$}|>{$}c<{$}|}
\hline
-1&-1&0\\
 3& 3&0\\
\hline
\end{tabular}
\\
&\rightarrow
\begin{tabular}{|>{$}c<{$}>{$}c<{$}|>{$}c<{$}|}
\hline
 1& 1&0\\
 0& 0&0\\
\hline
\end{tabular}
&
v_-&=\begin{pmatrix}
1\\-1
\end{pmatrix}.
\end{align*}
Kontrolle:
\[
Av_-
=
\begin{pmatrix}
1&-1\\3&6
\end{pmatrix}
\begin{pmatrix}
1\\-1
\end{pmatrix}
=
\begin{pmatrix}
2\\-2
\end{pmatrix}
=\lambda_-v_-.
\]
Eine Basis, in der $A$ diagonal wird, ist also 
\[
{\cal B}=\left\{
\begin{pmatrix}1\\-3 \end{pmatrix},
\begin{pmatrix}1\\-1 \end{pmatrix}
\right\}.
\qedhere
\]
\end{loesung}

\begin{bewertung}
Ansatz charakteristisches Polynom ({\bf X}) 1 Punkt,
ausmultipiziertes Polynome ({\bf P}) 1 Punkt,
Nullstellen ({\bf N}) 1 Punkt,
Berechnung der Eigenvektoren ($\textbf{E}_1$ und $\textbf{E}_2$) je 1 Punkt,
Basis ({\bf B}) 1 Punkt.
\end{bewertung}

\begin{diskussion}
Man beachte, dass die Matrix $A$ nicht symmetrisch ist, und man daher
auch nicht erwarten kann, dass die Diagonalisierung mit orthogonalen
Eigenvektoren möglich ist.
\end{diskussion}

