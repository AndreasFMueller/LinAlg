Am Golf von Thailand gibt es direkt am Strand ein kleines idyllisches 
Restaurant, welches von der Besitzerin Pi-Gop mit viel Herzblut geführt wird. 
Sie möchte ihre Oase nun etwas aufbessern und zu diesem Zweck am Strand eine
Sonnenuhr bauen. Um herauszufinden wie sie die Stundenskala zeichnen muss,
steckt sie einen langen Stab so in den Sand, dass er genau nach Norden zeigt 
und einen Neigungswinkel von $\varphi = 30^\circ$ hat (siehe Abbildung links).
Der Stab ragt nun noch $1\,$m zum Boden heraus. Zudem Stellt sie $1\,$m vom Stab 
entfernt auf einer Höhe von $0.8\,$m (Punkt $P$ in Abbildung links) eine Kamera auf,
welche eine Brennweite von $f=150$ hat und einen $320\times240$-Chip. Die Ausrichtung 
der Kamera ist durch die Drehmatrix
\[
D
=
\begin{pmatrix}
-1     &    0  &       0\\
0 &   0.5  &  0.866\\
0 &   0.866 &  -0.5\\
\end{pmatrix}
\]
gegeben.
\begin{center}
\definecolor{CadetRed}	{cmyk}{0,1,1,0.290}
\tikzset{
					MyPersp/.style={scale=1,x={(-1.1cm,-0.94cm)},y={(1.598cm,-0.596cm)},
				z={(0cm,1.55cm)}}
					}

\begin{tikzpicture}[thick,>=latex]


\begin{scope}[xshift=-4cm,MyPersp]
\node at (0,0) {\includeagraphics[height=5.5cm,trim={0.5cm 0 0.5cm 0},clip]{uhr9.jpg}};

\node[below]at (1.5,0,0){Osten / $x$};
\node[below] at (0,1.5,0) {Norden/ $y$};
\node[above] at (0,0,1.6){$z$};

\draw[black,dashed](0,0,0)--++(0,-0.87,0)--++(0,0,0.68);
\shade [ball color=CadetRed] (0,-0.87,0.68) circle (2pt)node[left=1pt]{$P$};


\begin{scope}[canvas is yz plane at x=0]

\draw[black](0.7,0)arc(0:30:0.7) node [below=0.5pt,xshift=-3pt] {$\varphi$};
\draw[black,dashed](0.8,0)--(0,0)--({0.8*cos(30)},{0.8*sin(30)});

\end{scope}

\end{scope}

\begin{scope}[xshift=2cm,yshift=0.5cm]

\def\dx{320};
\def\dy{240};
\def\s{3.5};

\node at (0,0) {\includeagraphics[height=\s cm]{view8.jpg}};

\draw[CadetRed,->]({\dx/2*(\s/\dy)},{-\dy/2*(\s/\dy)})--++(-1.5,0)node[below]{$x$};
\draw[CadetRed,->]({\dx/2*(\s/\dy)},{-\dy/2*(\s/\dy)})--++(0,1.5)node[right]{$y$};

\draw[CadetRed,fill]({\dx/2*(\s/\dy)},{-\dy/2*(\s/\dy)})++({-225*(\s/\dy)},{136*(\s/\dy)})circle(1pt)node[above left=-3pt]{$B_8$};

% \draw[CadetRed,fill]({\dx/2*(\s/\dy)},{-\dy/2*(\s/\dy)})++({-197*(\s/\dy)},{137*(\s/\dy)})circle(1pt)node[left]{$B_9$};


\end{scope}

\begin{scope}[xshift=7.5cm,yshift=0.5cm]

\def\dx{320};
\def\dy{240};
\def\s{3.5};

\node at (0,0) {\includeagraphics[height=\s cm]{view9.jpg}};

\draw[CadetRed,->]({\dx/2*(\s/\dy)},{-\dy/2*(\s/\dy)})--++(-1.5,0)node[below]{$x$};
\draw[CadetRed,->]({\dx/2*(\s/\dy)},{-\dy/2*(\s/\dy)})--++(0,1.5)node[right]{$y$};

% \draw[CadetRed,fill]({\dx/2*(\s/\dy)},{-\dy/2*(\s/\dy)})++({-225*(\s/\dy)},{136*(\s/\dy)})circle(1pt)node[left]{$B_8$};

\draw[CadetRed,fill]({\dx/2*(\s/\dy)},{-\dy/2*(\s/\dy)})++({-197*(\s/\dy)},{137*(\s/\dy)})circle(1pt)node[above left=-3pt]{$B_9$};


\end{scope}

\end{tikzpicture}
\end{center}
Pi-Gop hat die Kamera so programmiert, dass sie ab 7 Uhr morgens bis 17 Uhr abends zu jeder vollen
Stunde ein Bild von der Sonnenuhr macht. In jedem der aufgenommenen Bilder (Abbildungen rechts) 
liest Pi-Gop nun die Pixelkoordinaten $B_i$ des Stab-Endes ab. 
\begin{teilaufgaben}
 \item
 Für die Uhrzeiten 8 Uhr und 9 Uhr hat Pi-Gop die Pixelkoordinaten $B_8 = (225, 136 )$ und 
 $B_9 = (197, 137)$ abgelesen.
 Berechnen Sie die den Winkel zwischen diesen beiden Stundenlinien.
 \item
 Um 12 Uhr steht die Sonne genau senkrecht über der Sonnenuhr. Berechnen Sie die Länge des Schattens,
 welchen der Stab erzeugt.
\end{teilaufgaben}

\begin{hinweis}
Berechnen Sie die Matrizen- und Vektoroperationen mit dem Taschenrechner.
\end{hinweis}

\thema{Kamerageometrie}
\thema{Zwischenwinkel}

\begin{loesung}
\begin{teilaufgaben}
\item
Die Kameramatrix $K$ kann aus den Aufgabendaten abgelesen werden, sie ist
\[
K
=
\begin{pmatrix}
150&  0&160\\
  0&150&120\\
  0&  0&  1
\end{pmatrix}
\]
und das Kamerazentrum entspricht dem Punkt $P$
\[
\vec c = \vec p = 
\begin{pmatrix}
0\\ -1\\ 0.8
\end{pmatrix}.
\]
Zunächst müssen wir aus den Punkten $B_8$ und $B_9$ dreidimensionale Vektoren
in homogenen Koordinaten machen, also
\[
\begin{aligned}
\vec b_8 &= \begin{pmatrix}225\\136\\1\end{pmatrix}
&&\text{und}&
\vec b_9 &= \begin{pmatrix}197\\137\\1\end{pmatrix}.
\end{aligned}
\]
Mit der Formel $\vec{r} = (KD)^{-1} \vec b$ können wir zu jedem Punkt
den Richtungsvektor
\[
\begin{aligned}
\vec r_8
&=
\begin{pmatrix}
-0.4333 \\
\phantom{-}0.9194 \\
-0.4076
\end{pmatrix}
&&\text{und}&
\vec r_9 &= 
\begin{pmatrix}
-0.2467 \\
\phantom{-}0.9227 \\
-0.4019
\end{pmatrix}
\end{aligned}
\]
der Geraden von der Kamera in $P$ aus zu den Punkten $S_8$ bzw. $S_9$ berechnen.

Der Punkt $S_8$ liegt auf der Ebene mit $z=0$ und erfüllt die
Gleichung $\vec s_8 = \vec {p} + t_8\vec{r}_8$.
Wir kennen die $z$-Koordinate von $\vec{p}$, sie ist $0.8$.
Die $z$-Koordinate muss daher folgende Gleichung erfüllen:
\[
0.8 + t_8\cdot (-0.4076) = 0
\qquad\Rightarrow\qquad
t_8 = \frac{-0.8}{-0.4076}=1.9625.
\]

Analog gilt $\vec s_9 = \vec {p} + t_9\vec{r}_9$
Die $z$-Koordinate davon ist wiederum
\[
0.8 + t_9\cdot (-0.4019) = 0
\qquad\Rightarrow\qquad
t_9 = \frac{-0.8}{-0.4019}=1.9907.\]

Einsetzen der Werte für $t_8$ und $t_9$ ergibt
\[
\begin{aligned}
S_8&=(-0.8504, 0.8043, 0)
&&\text{und}&
S_9&=(-0.4910, 0.8368, 0)
\end{aligned}
\]
für die beiden Punkte. 

Der Winkel zischen den beiden Stundenlinien 8 Uhr und 9 Uhr kann 
nun mit der Zwischenwinkelformel berechnet werden:
\begin{align*}
\cos\alpha
&=
\frac{\overrightarrow{s_8}\cdot\overrightarrow{s_9}}{|\overrightarrow{s_8}|\cdot|\overrightarrow{s_9}|}
=
\frac{\begin{pmatrix}-0.8504\\ 0.8043\\ 0\end{pmatrix}
\cdot\begin{pmatrix}-0.4910\\ 0.8368\\ 0\end{pmatrix}}
{|\overrightarrow{s_8}|\cdot|\overrightarrow{s_9}|}
=
\frac{1.0906}{1.1705\cdot 0.9703} = 0.9603,
\\
\Rightarrow\qquad
\alpha
&=
16.1937^\circ.
\qedhere
\end{align*}
\item
Die Länge des Schattens um 12 Uhr entspricht der Projektion des ''Stab''-Vektors
\[
  \vec s = \begin{pmatrix} 0\\ \cos(30^\circ) \\ \sin(30^\circ) \end{pmatrix}
  = \begin{pmatrix} 0\\ 0.866 \\ 0.5 \end{pmatrix}
\]
auf die $y$-Achse: $l = 0.866\,$m.
\end{teilaufgaben}
\end{loesung}

\begin{bewertung}
Kameramatrix $K$ ({\bf K}) 1 Punkt,
Punkte $B_i$ in homogenen Koordinaten ({\bf B}) 1 Punkt,
Richtungsvektoren $\vec{r}_i$ ({\bf R}) 1 Punkt,
Weltpunkte $S_i$ ({\bf S}) 1 Punkt,
Zwischenwinkel ({\bf W}) 1 Punkt.
Länge des Schattens ({\bf L}) 1 Punkt,
\end{bewertung}
