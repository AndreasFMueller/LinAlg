Zwei Kameras mit Brennweite $f=100$ und einem $120\times 90$-Chip
sind in den Punkten $C_1=(100,0,0)$ und $C_2=(0,100,0)$ platziert
und mit den Drehmatrizen
\[
\begin{aligned}
D_1
&=
\begin{pmatrix}
   0& -1&  0\\
   0&  0&  1\\
  -1&  0&  0
\end{pmatrix}
&
&\text{und}
&
D_2
&=
\begin{pmatrix}
   0&  0& -1\\
   1&  0&  0\\
   0& -1&  0
\end{pmatrix}
\end{aligned}
\]
ausgerichtet.
Ein Punkt $Q$ wird von den Kameras auf die Bildpunkte
$B_1=(37.778,39.444)$ und $B_2=(66.250,57.500)$ abgebildet.
Bestimmen Sie den Punkt $Q$.

\thema{Kamerageometrie}

\begin{loesung}
Jedes Kamerabild definiert eine Gerade, auf der sich der Punkt $Q$ 
befinden muss.
Wir berechnen daher erst die Geraden und daraus deren Schnittpunkt.
Die Stützvektoren sind $c_1$ und $c_2$, die Richtungsvektoren
$r_i = (KD)^{-1}\tilde b_i$.
Wir brauchen daher zuerst 
\[
K
=
\begin{pmatrix}
100&  0&60\\
  0&100&45\\
  0&  0& 1
\end{pmatrix}.
\]
Für die Richtungsvektoren ergibt die Rechnung
\[
\begin{aligned}
r_1
&=
\begin{pmatrix}
  -1.00000\\
   0.22222\\
  -0.05556
\end{pmatrix}
&
&\text{und}
&
r_2
&=
\begin{pmatrix}
   0.1250\\
  -1.0000\\
  -0.0625
\end{pmatrix}.
\end{aligned}
\]
Die Geradengleichungen sind $c_1+tr_1$ bzw.~$c_2+sr_2$.
Der Schnittpunkt kann als Lösung des Gleichungssystems mit dem
Tableau
\[
\begin{tabular}{|>{$}c<{$} >{$}c<{$} >{$}c<{$} >{$}r<{$} >{$}r<{$}|>{$}r<{$}|}
\hline
x&y&z&    t   &   s   &   \\
\hline
1&0&0& 1.00000& 0.0000&100\\
0&1&0&-0.22222& 0.0000&  0\\
0&0&1& 0.05556& 0.0000&  0\\
1&0&0& 0.00000&-0.1250&  0\\
0&1&0& 0.00000& 1.0000&100\\
0&0&1& 0.00000& 0.0625&  0\\
\hline
\end{tabular}
\]
Im Allgemeinen werden sich die Geraden nicht schneiden, daher
suchen wir eine Lösung im Sinne der kleinsten Quadrate.
Die Rechnung ergibt $Q=(10.000,19.9998,-5.0002)$.
\end{loesung}

