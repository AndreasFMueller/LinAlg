Zwei Kameras mit Brennweite $f=100$ Pixel und einem $120\times 90$-Chip
sind in den Punkten $C_1=(100,0,0)$ und $C_2=(0,100,0)$ platziert
und mit den Drehmatrizen
\[
\begin{aligned}
D_1
&=
\begin{pmatrix}
   0& -1&  0\\
   0&  0&  1\\
  -1&  0&  0
\end{pmatrix}
&
&\text{und}
&
D_2
&=
\begin{pmatrix}
   0&  0& -1\\
   1&  0&  0\\
   0& -1&  0
\end{pmatrix}
\end{aligned}
\]
ausgerichtet.
Ein Punkt $Q$ wird von den Kameras auf die Bildpunkte
$B_1=(36,40)$ und $B_2=(65,58)$ abgebildet.
Bestimmen Sie den Punkt $Q$.

\thema{Kamerageometrie}

\begin{loesung}
Jedes Kamerabild definiert eine Gerade, auf der sich der Punkt $Q$ 
befinden muss.
Wir berechnen daher erst die Geraden und daraus deren Schnittpunkt.
Die Stützvektoren sind $\vec c_1$ und $\vec c_2$, die Richtungsvektoren
$\vec r_i = (KD)^{-1}\tilde b_i$.
Wir brauchen daher zuerst 
\[
K
=
\begin{pmatrix}
100&  0&60\\
  0&100&45\\
  0&  0& 1
\end{pmatrix}.
\]
Für die Richtungsvektoren ergibt die Rechnung
\[
\begin{aligned}
\vec r_1
&=
\begin{pmatrix}
  -1.00\\
   \phantom{-}0.24\\
  -0.05
\end{pmatrix}
&
&\text{und}
&
\vec r_2
&=
\begin{pmatrix}
   \phantom{-}0.13\\
  -1.00\\
  -0.05
\end{pmatrix}.
\end{aligned}
\]
Die Geradengleichungen sind folglich
\[
 \vec p_1 = \vec c_1+t \vec r_1 = 
\begin{pmatrix}
  100\\
   0\\
  0
\end{pmatrix}
+t\cdot
\begin{pmatrix}
  -1.00\\
   \phantom{-}0.24\\
  -0.05
\end{pmatrix}
\]
und
\[
 \vec p_2 = \vec c_2+s \vec r_2 = 
\begin{pmatrix}
   0\\
  100\\
  0
\end{pmatrix}
+s\cdot
\begin{pmatrix}
   \phantom{-}0.13\\
  -1.00\\
  -0.05
\end{pmatrix}.
\]
Der Schnittpunkt sollte als Lösung des folgenden 
Gleichungssystems gefunden werden können:
\[
\begin{pmatrix}
1&0&0& 1 & 0 \\
0&1&0&-0.24 & 0 \\
0&0&1& \phantom{-}0.05 & 0\\
1&0&0& 0 & -0.13\\
0&1&0& 0 & 1\\
0&0&1& 0 & \phantom{-}0.05\\
\end{pmatrix}
\begin{pmatrix}
x\\y\\z\\ t\\s\\ 
\end{pmatrix}
= 
\begin{pmatrix}
100\\
0\\
0\\
0\\
100\\
0\\
\end{pmatrix}
\]
Leider schneiden sich die Geraden im Allgemeinen aber nicht,
weshalb das Gleichungssystems keine Lösung hat. Stattdessen
suchen wir eine Lösung im Sinne der kleinsten Quadrate, also die
Lösung des Gleichungssystems
\[
A^tA\vec x = A^t\vec b
\quad\text{mit}\quad 
A = \begin{pmatrix}
1&0&0& 1 & 0 \\
0&1&0&-0.24 & 0 \\
0&0&1& \phantom{-}0.05 & 0\\
1&0&0& 0 & -0.13\\
0&1&0& 0 & 1\\
0&0&1& 0 & \phantom{-}0.05\\
\end{pmatrix}
\quad\text{und}\quad
\vec b = \begin{pmatrix}
100\\
0\\
0\\
0\\
100\\
0\\
\end{pmatrix}.
\]
Die Rechnung ergibt 
\[
\vec x =  (A^tA)^{-1} A^t\vec b = 
\begin{pmatrix}
 2 & 0 & 0 & 1 & -0.13\\
 0 & 2 & 0 & -0.24 & 1\\
 0 & 0 & 2 & \phantom{-}0.05 & \phantom{-}0.05\\
 1 & -0.24 & 0.05 & 1.0601 & 0\\
 -0.13 & 1 & 0.05 & 0 & 1.0194\\
\end{pmatrix}^{-1}
\begin{pmatrix}
   100\\
   100\\
     0\\
   100\\
   100\\
\end{pmatrix}
=
\begin{pmatrix}
 
   10.2218\\
   21.5260\\
   -4.2063\\
   89.7602\\
   78.4904\\
\end{pmatrix},
\]
woraus die Koordinaten des gesuchten Punkts 
$Q=(10.2218,21.5260,-4.2063)$ abgelesen werden können.
\end{loesung}

