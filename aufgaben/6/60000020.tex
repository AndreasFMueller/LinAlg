Rudolf macht eine Reise nach Paris und besucht bei dieser Gelegenheit 
den Eiffelturm. Als er vor dem riesigen Bauwerk steht, fragt er sich,
wie hoch der Eiffelturm den eigentlich ist. Dabei kommt ihm die Idee,
dass er die Höhe des Eiffelturms mit Hilfe seiner Kamera messen könnte.
Er positioniert dazu seine Kamera $330\,$m vor dem Eiffelturm auf einer
Höhe von $1.7\,$m (Punkt $P$ in Abbildung links) und richtet sie
auf den Eiffelturm. Die Ausrichtung der Kamera ist durch die 
Drehmatrix
\[
D
=
\begin{pmatrix}
0 &    -1   &  0\\
\phantom{-}0.334 &0 & 0.943\\
-0.943 &0 & 0.334\\
\end{pmatrix}
\]
gegeben. Seine Kamera hat eine Brennweite von $f=630$ und einen 
$480\times 640$-Chip.
\begin{center}
\definecolor{CadetRed}	{cmyk}{0,1,1,0.290}
\tikzset{
					MyPersp/.style={scale=1,x={(-0.9cm,-0.2cm)},y={(0.6cm,-0.5cm)},
				z={(0cm,1cm)}}
					}
\begin{tikzpicture}[thick,>=latex]

\begin{scope}[xshift=-2cm, MyPersp]

\coordinate (O) at (0.0,-0.3,-2.6);

\def\dx{8};
\def\dz{1};
\draw[black](O)--++(\dx,0,0)--++(0,0,\dz);
\shade [ball color=CadetRed] ($(O)+(\dx,0,\dz)$) circle (3pt)node[left=1pt]{$P$};


\draw[CadetRed,->](O)--++(2.2,0,0)node[above=-1pt]{$x$};
\draw[CadetRed,->](O)--++(0,2,0)node[above right=-3pt]{$y$};
\draw[CadetRed,->](O)--++(0,0,6.3)node[right]{$z$};

\shade [ball color=CadetRed] (O) circle (2pt)node[above right=-1pt,yshift=-3pt]{$G$};
\shade [ball color=CadetRed] ($(O)+(0,0,5.8)$) circle (2pt)node[left]{$S$};

\node at (0,0) {\includeagraphics[height=7cm]{eiffelturm3D.png}};


\end{scope}
\begin{scope}[xshift=4cm]
\node at (0,0) {\includeagraphics[width=5.5cm]{eiffelturm.JPG}};

\draw[white,fill,opacity=0.3](-5.6/2,-7.5/2)rectangle(5.6/2,7.5/2);

\draw[CadetRed,->](5.5/2,-7.333/2)--++(-1.5,0)node[below]{$x$};
\draw[CadetRed,->](5.5/2,-7.333/2)--++(0,1.5)node[right]{$y$};

\draw[CadetRed,fill](5.5/2-229/480*5.5,-7.333/2+613/480*5.5)circle(1pt)node[right]{$B_S$};
\draw[CadetRed,fill](5.5/2-229/480*5.5,-7.333/2+94/480*5.5)circle(1pt)node[above]{$B_G$};


\end{scope}

\end{tikzpicture}
\end{center}
Im aufgenommenen Bild (Abbildung rechts) liest Rudolf nun die Pixelkoordinaten
der Spitze $B_S = (229,613)$ und des genau darunterliegenden Punktes am Boden
$B_G=(229,94)$ ab. Diese beiden Punkte befinden sich im Weltkoordinatensystem 
auf der $y$-$z$-Ebene ($x=0$).
\begin{teilaufgaben}
 \item
 Berechnen Sie die Höhe des Eiffelturms.
 \item
 Rudolf will wissen, was für ein Gesichtswinkel eine Kamera mindestens haben muss,
 um damit den Eiffelturm aus dieser Distanz photographieren zu können. Berechnen Sie dazu den Winkel $\angle GPS$.
\end{teilaufgaben}


\begin{hinweis}
Berechnen Sie die Matrizen- und Vektoroperationen mit dem Taschenrechner.
\end{hinweis}

\thema{Kamerageometrie}
\thema{Zwischenwinkel}

\begin{loesung}
\begin{teilaufgaben}
\item
Die Kameramatrix $K$ kann aus den Aufgabendaten abgelesen werden, sie ist
\[
K
=
\begin{pmatrix}
630&  0&240\\
  0&630&320\\
  0&  0&  1
\end{pmatrix}
\]
und das Kamerazentrum entspricht dem Punkt $P$
\[
\vec c = \vec p = 
\begin{pmatrix}
330\\ 0\\ 1.7
\end{pmatrix}.
\]
Zunächst müssen wir aus den Punkten $B_S$ und $B_G$ dreidimensionale Vektoren
in homogenen Koordinaten machen, also
\[
\begin{aligned}
\vec b_S &= \begin{pmatrix}229\\613\\1\end{pmatrix}
&&\text{und}&
\vec b_G &= \begin{pmatrix}229\\94\\1\end{pmatrix}.
\end{aligned}
\]
Mit der Formel $\vec{r} = (KD)^{-1} \vec b$ können wir zu jedem Punkt
den Richtungsvektor
\[
\begin{aligned}
\vec r_S
&=
\begin{pmatrix}
-0.7870 \\
\phantom{-}0.0175 \\
\phantom{-}0.7719
\end{pmatrix}
&&\text{und}&
\vec r_G &= 
\begin{pmatrix}
-1.0620 \\
\phantom{-}0.0175 \\
-0.0043
\end{pmatrix}
\end{aligned}
\]
der Geraden von der Kamera in $P$ aus zu den Punkten $S$ bzw. $G$ berechnen.

Der Punkt $S$ liegt auf der Ebene mit $x=0$ und erfüllt die
Gleichung $\vec s = \vec {p} + t_S\vec{r}_S$.
Wir kennen die $x$-Koordinate von $\vec{p}$, sie ist $330$.
Die $x$-Koordinate muss daher folgende Gleichung erfüllen:
\[
330 + t_S\cdot (-0.7870) = 0
\qquad\Rightarrow\qquad
t_S = \frac{-330}{-0.7870}=419.2979.
\]

Analog gilt $\vec g = \vec {p} + t_G\vec{r}_G$.
Die $x$-Koordinate davon ist wiederum
\[
330 + t_G\cdot (-1.0620) = 0
\qquad\Rightarrow\qquad
t_G = \frac{-330}{-1.0620}=310.7459.
\]

Einsetzen der Werte für $t_S$ und $t_G$ ergibt
\[
\begin{aligned}
S&=(0, 7.321, 325.376)
&&\text{und}&
G&=(0, 5.426, 0.370)
\end{aligned}
\]
für die beiden Punkte. 

Die Höhe des Eiffelturms entspricht nun der Länge des Vektors $\overrightarrow{GS}$
\[
  h = |\overrightarrow{GS}| = |\vec s - \vec g| = 
\left|\begin{pmatrix}
0\\
1.895\\
325.006
\end{pmatrix}\right| = \sqrt{0^2 + (1.895)^2 + 325.006^2} = 325.01.
\]
Die Messung von Rudolf ergab also eine Höhe von $325.01\,$m, welche nur geringfügig 
von der wahren Höhe des Eiffelturms von $324.82\,$m abweicht.
\item
Der Winkel kann mit der Zwischenwinkelformel berechnet werden.
Es gilt
\begin{align*}
\cos\alpha
&=
\frac{\vec r_S\cdot\vec r_G}{|\vec r_S|\cdot|\vec r_G|}
=
\frac{\begin{pmatrix}
-0.7870 \\
\phantom{-}0.0175 \\
\phantom{-}0.7719
\end{pmatrix}
\cdot\begin{pmatrix}
-1.0620 \\
\phantom{-}0.0175 \\
-0.0043
\end{pmatrix}}
{|\vec r_S|\cdot|\vec r_G|}
=
\frac{0.8328}{1.1026\cdot 1.0621} = 0.7112,
\\
\Rightarrow\qquad
\alpha
&=
44.6705^\circ.
\qedhere
\end{align*}
\end{teilaufgaben}
\end{loesung}

\begin{bewertung}
Kameramatrix $K$ ({\bf K}) 1 Punkt,
Punkte $B_i$ in homogenen Koordinaten ({\bf B}) 1 Punkt,
Richtungsvektoren $\vec{r}_i$ ({\bf R}) 1 Punkt,
Weltpunkte $G$ und $S$ ({\bf P}) 1 Punkt,
Höhe des Eiffelturm ({\bf H}) 1 Punkt,
Zwischenwinkel ({\bf W}) 1 Punkt.
\end{bewertung}
