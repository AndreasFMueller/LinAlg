Ist die Matrix
\[
A=\begin{pmatrix}
4&-2\\
2&10
\end{pmatrix}
\]
diagonalisierbar? Wenn ja, geben Sie eine Basis an, in der $A$ diagonal
ist.

\begin{loesung}
Zun"achst sind die Eigenwerte mit Hilfe des charakteristischen Polynoms
zu bestimmen:
\begin{align*}
\chi_A(\lambda)
&=
\left|
\begin{matrix}
4-\lambda&-2\\
2&10-\lambda
\end{matrix}
\right|
=
(4-\lambda)(10-\lambda)+4
=
\lambda^2-14\lambda+44
=0
\\
\lambda_{\pm}
&=
7\pm\sqrt{49-44}=7\pm\sqrt{5}.
\end{align*}
Wir haben zwei verschiedene Eigenwerte, wir werden also auch zwei
verschiedene Eigenvektoren finden, und k"onnen damit die Frage, ob 
die Matrix $A$ diagonalisierbar ist, bereits mit ``ja'' beantworten.

Um eine Basis zu finden, in der $A$ diagonal ist, m"ussen wir jetzt
noch die Eigenvektoren ausrechen.

F"ur $\lambda=\lambda_+=7+\sqrt{5}$ finden wir
\begin{align}
\begin{tabular}{|>{$}c<{$}>{$}c<{$}|}
\hline
4-\lambda&-2\\
2&10-\lambda\\
\hline
\end{tabular}
&=
\begin{tabular}{|>{$}c<{$}>{$}c<{$}|}
\hline
-3-\sqrt{5}&-2\\
2&3-\sqrt{5}\\
\hline
\end{tabular}
\rightarrow
\begin{tabular}{|>{$}c<{$}>{$}c<{$}|}
\hline
1&\frac{2}{3+\sqrt{5}}\\
2&3-\sqrt{5}\\
\hline
\end{tabular}
\label{60000020:gaussrechnung}
\end{align}
Den Ausdruck in der rechten oberen Ecke k"onnen wir mit $3-\sqrt{5}$ 
erweitern und erhalten
\begin{align*}
\frac{2}{3+\sqrt{5}}
\cdot
\frac{3-\sqrt{5}}{3-\sqrt{5}}
&=
\frac{2(3-\sqrt{5})}{9-5}
=
\frac{3-\sqrt{5}}2.
\end{align*}
Eingesetzt in die begonnene Gauss-Rechnung (\ref{60000020:gaussrechnung}):
\begin{align*}
&\rightarrow
\begin{tabular}{|>{$}c<{$}>{$}c<{$}|}
\hline
1&\frac{3-\sqrt{5}}2\\
2&3-\sqrt{5}\\
\hline
\end{tabular}
\rightarrow
\begin{tabular}{|>{$}c<{$}>{$}c<{$}|}
\hline
1&\frac{3-\sqrt{5}}2\\
0&0\\
\hline
\end{tabular}
\end{align*}
Daraus k"onnen wir zum Eigenwert $\lambda=\lambda_+=7+\sqrt{5}$ 
den Eigenvektor
\[
v_+
=
\begin{pmatrix}
\frac{3-\sqrt{5}}2\\-1
\end{pmatrix}
=
\begin{pmatrix}
\phantom{-}0.38197\\
-1.00000
\end{pmatrix}
\sim
\begin{pmatrix}
\phantom{-}0.35682\\
-0.93417
\end{pmatrix}
\]
ablesen.

F"ur den Eigenwert $\lambda=\lambda_-=7-\sqrt{5}$ folgt analog:
\begin{align}
\begin{tabular}{|>{$}c<{$}>{$}c<{$}|}
\hline
4-\lambda&-2\\
2&10-\lambda\\
\hline
\end{tabular}
&=
\begin{tabular}{|>{$}c<{$}>{$}c<{$}|}
\hline
-3+\sqrt{5}&-2\\
2&3+\sqrt{5}\\
\hline
\end{tabular}
\rightarrow
\begin{tabular}{|>{$}c<{$}>{$}c<{$}|}
\hline
1&\frac{2}{3-\sqrt{5}}\\
2&3+\sqrt{5}\\
\hline
\end{tabular}
\label{60000020:gaussrechnung2}
\end{align}
Den Ausdruck in der rechten oberen Ecke k"onnen wir mit $3+\sqrt{5}$ 
erweitern und erhalten
\begin{align*}
\frac{2}{3-\sqrt{5}}
\cdot
\frac{3+\sqrt{5}}{3+\sqrt{5}}
&=
\frac{2(3+\sqrt{5})}{9-5}
=
\frac{3+\sqrt{5}}2.
\end{align*}
Eingesetzt in die begonnene Gauss-Rechnung
(\ref{60000020:gaussrechnung2}):
\begin{align*}
&\rightarrow
\begin{tabular}{|>{$}c<{$}>{$}c<{$}|}
\hline
1&\frac{3+\sqrt{5}}2\\
2&3+\sqrt{5}\\
\hline
\end{tabular}
\rightarrow
\begin{tabular}{|>{$}c<{$}>{$}c<{$}|}
\hline
1&\frac{3+\sqrt{5}}2\\
0&0\\
\hline
\end{tabular}
\end{align*}
Daraus k"onnen wir zum Eigenwert $\lambda=\lambda_-=7-\sqrt{5}$ 
den Eigenvektor
\[
v_-
=
\begin{pmatrix}
\frac{3+\sqrt{5}}2\\-1
\end{pmatrix}
=
\begin{pmatrix}
\phantom{-}2.61803\\
-1.00000
\end{pmatrix}
\sim
\begin{pmatrix}
\phantom{-}0.93417\\
-0.35682
\end{pmatrix}
\]
ablesen.

Zur Kontrolle rechnen wir die Wirkung von $A$ auf den beiden Eigenvektoren
nach:
\begin{align*}
Av_+
&=
\begin{pmatrix}4&-2\\2&10\end{pmatrix}
\begin{pmatrix}
\frac{3-\sqrt{5}}2\\-1
\end{pmatrix}
=
\begin{pmatrix}
6-2\sqrt{5}+2\\
3-\sqrt{5}-10
\end{pmatrix}
=
\begin{pmatrix}
8-2\sqrt{5}\\
-7-\sqrt{5}
\end{pmatrix}
=
(7+\sqrt{5})
\begin{pmatrix}
\frac{3-\sqrt{5}}2\\
-1
\end{pmatrix}
=
\lambda_+v_+
\\
Av_-
&=
\begin{pmatrix}4&-2\\2&10\end{pmatrix}
\begin{pmatrix}
\frac{3+\sqrt{5}}2\\-1
\end{pmatrix}
=
\begin{pmatrix}
6+2\sqrt{5}+2\\
3+\sqrt{5}-10
\end{pmatrix}
=
\begin{pmatrix}
8+2\sqrt{5}\\
-7+\sqrt{5}
\end{pmatrix}
=
(7-\sqrt{5})
\begin{pmatrix}
\frac{3+\sqrt{5}}2\\
-1
\end{pmatrix}
=
\lambda_-v_-
\end{align*}
Die Vektoren $\{v_+,v_-\}$ bilden also eine Basis, in der $A$ diagonal ist.
\end{loesung}

\begin{bewertung}
Ansatz charakteristisches Polynom ({\bf X}) 1 Punkt,
ausmultipiziertes Polynome ({\bf P}) 1 Punkt,
Nullstellen ({\bf N}) 1 Punkt,
Berechnung der Eigenvektoren ($\textbf{E}_1$ und $\textbf{E}_2$) je 1 Punkt,
Basis ({\bf B}) 1 Punkt.
\end{bewertung}

\begin{diskussion}
Resultate k"onnen insbesondere dadurch von der Muserl"osung abweichen,
dass die Ausdr"ucke wie $2/(3+\sqrt{5})$ auftauchen. Es ist aber
\[
\frac{2}{3\pm \sqrt{5}}
=
\frac{2}{3\pm \sqrt{5}}\frac{3\mp\sqrt{5}}{3\mp\sqrt{5}}
=
\frac{6\mp2\sqrt{5}}{3^2-5}
=
\frac{3\mp\sqrt{5}}2.
\]
\end{diskussion}


