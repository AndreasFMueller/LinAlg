Eine Kamera mit einem $1024\times 1024$-Chip und einem
Objektiv mit einer Brennweite von 600 Pixeln
ist im Punkt $C=(-8,-6,0)$ montiert und mit der Matrix
\[
D
=
\begin{pmatrix*}[r]
 0.7071 &  -0.7071 &  0.0000\\
 0.1503 &  0.1503 &  0.9771\\
 0.6909 &  0.6909 &  -0.2126\end{pmatrix*}
\]
auf den Punkt $Q=(5,7,-4)$ ausgerichtet.
Die beiden Punkte $P_1=(2,12,-13)$ und $P_2=(-4,18,5)$
werden durch die Kamera auf die Punkte $B_1$ und $B_2$
abgebildet.
Wegen perspektivischer Verzerrung wird der Mittelpunkt
$M$ der Strecke $P_1P_2$ nicht auf den Mittelpunkt
$M'$ der Strecke $B_1B_2$ auf dem Chip abgebildet.
Berechnen Sie die Pixelentfernung der Punkte $B_1$
und $B_2$ vom Bild des Mittelpunktes der Strecke
$P_1P_2$.
\begin{figure}[h]
\centering
\begin{tikzpicture}[>=latex,thick]
\begin{scope}[xshift=-5cm]
\node at (0,3.0000) {\includeagraphics[scale=1.2]{pov2024.pdf}};
\end{scope}
\draw (0,0) rectangle (6.0000,6.0000);
\node at (0,0) [left] {$(0,0)$};
\node at (6.0000,6.0000) [right] {$(1024,1024)$};
\draw[line width=0.3pt] (0,0) -- (6.0000,6.0000);
\draw[line width=0.3pt] (6.0000,0) -- (0,6.0000);
\draw[color=darkred] (2.0977,1.6465) -- (0.2812,4.7461);
\fill[color=white] (1.1895,3.1963) circle[radius=0.08];
\draw[color=darkred] (1.1895,3.1963) circle[radius=0.08];
\fill[color=blue] (1.2773,3.0527)
	circle[radius=0.08];
\node[color=blue] at (1.2773,3.0527) [above] {$M'\mathstrut$};
\fill[color=darkred] (2.0977,1.6465)
	circle[radius=0.08];
\node[color=darkred] at (2.0977,1.6465) [above] {$B_1$};
\fill[color=darkred] (0.2812,4.7461)
	circle[radius=0.08];
\node[color=darkred] at (0.2812,4.7461) [above] {$B_2$};
\end{tikzpicture}
\end{figure}

\begin{loesung}
Die Abbildung der Punkte $P_1$ und $P_2$ erfolgt mit
\[
p_i \mapsto p_i - c \mapsto D(p_i-c) \mapsto KD(p_i-c)
\]
und liefert homogene Koordinaten für die Bildpunkte.
Die Kameramatrix $K$ enthält Brennweite und
Mittelpunktskoordinaten des Sensors:
\[
K = \begin{pmatrix*}[r]
 600 &  0 & 512\\
  0 & 600 & 512\\
  0 &  0 &  1
\end{pmatrix*}.
\]
Angewendet auf die beiden gegebenen Punkte erhält man
die homogenen Koordinaten der Bildpunkte
\[
\tilde{b}_1
=
\begin{pmatrix*}[r]
7925.7280\\6223.4680\\22.1090
\end{pmatrix*},\quad
\tilde{b}_2
=
\begin{pmatrix*}[r]
875.2864\\14816.8264\\18.2822
\end{pmatrix*}
\qquad\text{und}\qquad
\tilde{b}_m
=
\begin{pmatrix*}[r]
4400.5072\\10520.1472\\20.1956
\end{pmatrix*}.
\]
Division durch die dritte Komponente liefert die
auf ganze Zahlen gerundeten Pixelkoordinaten der
Bildpunkte
\[
b_1=\begin{pmatrix*}[r]
358\\281\end{pmatrix*},\quad
b_2=\begin{pmatrix*}[r]
48\\810\end{pmatrix*}
\qquad\text{und}\qquad
b_m=\begin{pmatrix*}[r]
218\\521\end{pmatrix*}.
\]
Die gesuchten Abstände sind
\begin{align*}
\overline{B_1M'}
&=
|b_1-b_m|
=
\biggl|\begin{pmatrix*}[r]
140\\-240
\end{pmatrix*}\biggr|
=
277.8489,
\\
\overline{B_2M'}
&=
|b_2-b_m|
=
\biggl|\begin{pmatrix*}[r]
-170\\289
\end{pmatrix*}\biggr|
=
335.2924.
\qedhere
\end{align*}
\end{loesung}


\begin{bewertung}
Kameramatrix mit Brennweite ({\bf F}), 1 Punkt,
und Verschiebung für den Mittelpunkt ({\bf V}),
1 Punkt,
homogene Koordinaten der beiden Bildpunkte ({\bf H})
1 Punkt,
Pixelkoordinaten der beiden Bildpunkte ({\bf P})
1 Punkt
Abbildung des Mittelpunktes zwischen $P_1$ und $P_2$
({\bf M}) 1 Punkt,
Pixelabstände der Bildpunkte vom Bild des
Mittelpunktes ({\bf A}) 1 Punkt.
\end{bewertung}

