Die Projektion eines Globus auf einen Körper mit 32 Seitenflächen
wie in Abbildung~\ref{60000037:image} links wird so gedreht, dass die Punkte
$A$ (irgendwo im Westen Frankreichs) und $B$ (im Nordwesten Chinas nahe der
Grenze zur Mongolei), beide auf $45^\circ$ nördlicher Breite, auf die
Punkte $A'$ und $B'$ in der Abbildung rechts auf der $x$-$y$-Ebene
abgebildet werden.
Der Punkt $A'$ liegt auf der Winkelhalbierenden zwischen der $x$- und der
negativen $y$-Achse.
\begin{teilaufgaben}
\item
Berechnen Sie den Winkel zwischen den Ortsvektoren
von $A$ und $B$.
\item
Berechnen Sie den Ortsvektor des Punktes $B'$.
\item
Bestimmen Sie die Drehmatrix, die $A$ auf $A'$ und $B$ auf $B'$ abbildet.
\item
Berechnen Sie den Drehwinkel.
\end{teilaufgaben}
\begin{figure}[h]
\centering
\includeagraphics[]{earth1.pdf}
\caption{Drehung der Projektion eines Globus auf ein Polyeder für Aufgabe
\ref{60000037}.
\label{60000037:image}}
\end{figure}

\begin{hinweis}
Die Ecken des Polyeders haben alle Abstand 1 vom Nullpunkt.
Verwenden Sie das Vektorprodukt der Ortsvektoren von $A$ und $B$ als
dritten Vektor zur Bestimmung der Drehmatrix.
Verwenden Sie den Taschenrechner zur Berechnung der Drehmatrix.
\end{hinweis}

\begin{loesung}
Die gegebenen Punkte haben die Koordinaten
\[
A
=
\biggl(\frac{\!\sqrt{2}}{2},0,\frac{\!\sqrt{2}}{2}\biggr),
\qquad
B
=
\biggl(0, \frac{\!\sqrt{2}}{2},\frac{\!\sqrt{2}}{2}\biggr),
\qquad
A'
=
\biggl(-\frac{\!\sqrt{2}}{2},\frac{\!\sqrt{2}}{2},0\biggr),
\]
\begin{teilaufgaben}
\item
Der Winkel zwischen $A$ und $B$ kann mit dem Skalarprodukt der
Ortsvektoren von $A$ und $B$ ermittelt werden:
\[
\cos\alpha
=
\begin{pmatrix}
\frac{\!\sqrt{2}}{2}\\0\\\frac{\!\sqrt{2}}{2}
\end{pmatrix}
\cdot
\begin{pmatrix}
0\\ \frac{\!\sqrt{2}}{2}\\ \frac{\!\sqrt{2}}{2}
\end{pmatrix}
=
\frac12
\qquad\Rightarrow\qquad
\alpha=60^\circ.
\]
\item
Der Winkel zwischen der $x$-Achse und dem Ortsvektor des
Punktes $B'$ ist $\alpha-45^\circ=15^\circ$.
Daraus ergeben sich die Koordinaten $B'=\cos 15^\circ,\sin 15^\circ,0)$
und der Ortsvektor
\[
\vec{b}' = \begin{pmatrix}
\cos 15^\circ \\
\sin 15^\circ \\
0
\end{pmatrix}
\]
\item 
Zur Berechnung der Abbildungsmatrix wird ein dritter Vektor
benötigt, dafür kann das Vektorprodukt verwendet werden.
Das Vektorprodukt der Vektoren $\vec{a}$ und $\vec{b}$ ist
\[
\vec{a}\times\vec{b}
=
\begin{pmatrix}
\frac{\!\sqrt{2}}{2}\\0\\\frac{\!\sqrt{2}}{2}
\end{pmatrix}
\times
\begin{pmatrix}
0\\ \frac{\!\sqrt{2}}{2}\\ \frac{\!\sqrt{2}}{2}
\end{pmatrix}
=
\begin{pmatrix}
-\frac12\\
-\frac12\\
\phantom{-}\frac12
\end{pmatrix}
\quad\Rightarrow\quad
(\vec{a}\times\vec{b})^0
=
\frac{1}{\!\sqrt{3}}
\begin{pmatrix*}[r]
-1\\
-1\\
 1
\end{pmatrix*}.
\]
Der normierte Vektor $(\vec{a}\times\vec{b})^0$ wird von der
gesuchten Drehmatrix $D$ auf den dritten Standardbasisvektor
abgebildet.
Die Drehmatrix $D$ muss daher die Gleichung
\[
D
\underbrace{
\begin{pmatrix*}
\frac{\!\sqrt{2}}2 &         0          & -\frac{1}{\!\sqrt{3}} \\
        0          & \frac{\!\sqrt{2}}2 & -\frac{1}{\!\sqrt{3}} \\
\frac{\!\sqrt{2}}2 & \frac{\!\sqrt{2}}2 &  \frac{1}{\!\sqrt{3}}
\end{pmatrix*}
}_{\displaystyle =B}
=
\underbrace{
\begin{pmatrix*}
 \frac{\!\sqrt{2}}2 & \cos 15^\circ & 0 \\
-\frac{\!\sqrt{2}}2 & \sin 15^\circ & 0 \\
         0          &        0      & 1
\end{pmatrix*}
}_{\displaystyle =B'}
\]
erfüllen.
Sie kann nach $D$ aufgelöst werden und ergibt
\[
D
=
B'B^{-1}
=
\begin{pmatrix*}[r]
   0.2113248654&   0.5773502692&   0.7886751346\\
  -0.7886751346&   0.5773502692&  -0.2113248654\\
  -0.5773502692&  -0.5773502692&   0.5773502692
\end{pmatrix*}
\]
\item
Der Drehwinkel $\beta$ kann aus der Spurformel berechnet werden:
\begin{align*}
\cos\beta
&=
\frac{\operatorname{Spur}(D)-1}{2}
=
\frac{1.3660254038-1}{2}
&&\Rightarrow&
\beta
&=
\arccos(0.1830127019)
\\
&&&&&
=
79.45470941^\circ
\qedhere
\end{align*}
\end{teilaufgaben}
\end{loesung}

\begin{bewertung}
Zwischenwinkel $\alpha$ in Teilaufgabe a) mit dem Skalarprodukt
({\bf A}) 1 Punkt,
Koordinaten des Punktes $B'$ in Teilaufgabe b) ({\bf B}) 1 Punkt,
dritter Vektor mit dem Vektorprodukt ({\bf N}) 1 Punkt,
Gleichung für die Drehmatrix ({\bf G}) 1 Punkt,
Drehmatrix $D$ ({\bf D}) 1 Punkt,
Drehwinkel mit der Spurformel ({\bf S}) 1 Punkt.
\end{bewertung}
