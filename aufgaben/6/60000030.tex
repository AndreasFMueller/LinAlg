Bei der Beschreibung der Quadraturamplitudenmodulation wurde
verwendet, dass die Projektionsmatrix $P$ kann als Linearkombination
\[
P
=
\begin{pmatrix}
1&0\\
0&0
\end{pmatrix}
=
\frac12E+\frac12S,
\qquad
S
=
\begin{pmatrix}
1&0\\
0&-1
\end{pmatrix}
\]
geschrieben werden.
\begin{teilaufgaben}
\item
Rechnen Sie nach, dass $P$ eine Projektionsmatrix ist, d.~h.~$P^2=P$.
\item
Zeigen Sie, dass $S$ eine Spiegelungsmatrix der Form $S=E-2nn^t$ ist,
finden Sie den Vektor $n$.
\item
Rechnen Sie nach, dass $S^2=E$ ist.
\item 
Rechnen Sie nach, dass $SD_{\omega t}=D_{-\omega t}S$.
\item
Die Quadraturamplitudenmodulation hat aus den $I$- und $Q$-Signalen
das modulierte Signal 
\[
D_{\omega t}\begin{pmatrix}I(t)\\Q(t)\end{pmatrix}
\]
gemacht, von denen die erste Komponente übertragen wird, die man mit
$P$ erhalten kann. 
Die Demodulation mit $D_{-\omega t}$ liefert
\[
\begin{pmatrix}
\hat{I}(t)\\
\hat{Q}(t)
\end{pmatrix}
=
D_{-\omega t}PD_{\omega t}
\begin{pmatrix}
I(t)\\
Q(t)
\end{pmatrix}.
\]
Verwenden Sie die obige Darstellung der Matrix $P$ um nachzurechnen,
dass sich $\hat{I}(t)$ und $\hat{Q}(t)$ also Linearkombination der
ursprünglichen Signale und eines mit Frequenz $2\omega$ modulierten
Signales darstellen lassen.
\end{teilaufgaben}

\begin{loesung}
\begin{teilaufgaben}
\item 
Das Quadrat von $P$ ist
\[
P^2=
\begin{pmatrix}1&0\\0&0\end{pmatrix}
\begin{pmatrix}1&0\\0&0\end{pmatrix}
=
\begin{pmatrix}1&0\\0&0\end{pmatrix}
=
P.
\]
\item
Die Matrix $S$ kehrt das Vorzeichen der zweiten Koordinate, dies ist
die Spiegelung an der $x$-Achse. 
Der Vektor
\[
\vec{n}
=
\begin{pmatrix}0\\1\end{pmatrix},
\qquad
|n| = 1,
\]
ergibt die Spiegelungsmatrix
\[
E-2nn^t
=
\begin{pmatrix}1&0\\0&1\end{pmatrix}
-2\begin{pmatrix}0\\1\end{pmatrix}\begin{pmatrix}0&1\end{pmatrix}
=
\begin{pmatrix}1&0\\0&1\end{pmatrix}
-2
\begin{pmatrix}1&0\\0&1\end{pmatrix}
=
\begin{pmatrix}1&0\\0&-1\end{pmatrix}
=
S.
\]
\item
Wir müssen $S^2$ berechnen, es ergibt sich
\[
S^2
=
\begin{pmatrix}1&0\\0&-1\end{pmatrix}
\begin{pmatrix}1&0\\0&-1\end{pmatrix}
=
\begin{pmatrix}1&0\\0&1\end{pmatrix}
=
E.
\]
\item 
Wir müssen die Vertauschung von $S$ mit $D_{\omega t}$ bestimmen.
\begin{align*}
SD_{\omega t}
&=
\begin{pmatrix}1&0\\0&-1\end{pmatrix}
\begin{pmatrix*}[r]
 \cos\omega t & -\sin\omega t \\
 \sin\omega t &  \cos\omega t
\end{pmatrix*}
=
\begin{pmatrix*}[r]
 \cos\omega t & -\sin\omega t \\
-\sin\omega t & -\cos\omega t
\end{pmatrix*},
\\
D_{-\omega t}S
&=
\begin{pmatrix*}[r]
  \cos\omega t &  \sin\omega t \\
 -\sin\omega t &  \cos\omega t
\end{pmatrix*}
\begin{pmatrix}1&0\\0&-1\end{pmatrix}
=
\begin{pmatrix*}[r]
  \cos\omega t & -\sin\omega t \\
 -\sin\omega t & -\cos\omega t
\end{pmatrix*}.
\end{align*}
Daraus kann man ablesen, dass
\(
SD_{\omega t}
=
D_{-\omega t}S
\)
wie behauptet.
\item
Wir berechnen 
\begin{align*}
D_{-\omega t}PD_{\omega t}\begin{pmatrix}I(t)\\Q(t)\end{pmatrix}
&=
D_{-\omega t}\frac12(E+S)D_{\omega t}\begin{pmatrix}I(t)\\Q(t)\end{pmatrix}
=
\biggl(
\frac12
D_{-\omega t}D_{\omega t}
+
\frac12
D_{-\omega t}D_{-\omega t}S
\biggr)
\begin{pmatrix}I(t)\\Q(t)\end{pmatrix}
\\
&=
\frac12
\begin{pmatrix}I(t)\\Q(t)\end{pmatrix}
+
\frac12
D_{-2\omega t}
S
\begin{pmatrix}I(t)\\Q(t)\end{pmatrix}
=
\frac12
\begin{pmatrix}I(t)\\Q(t)\end{pmatrix}
+
\frac12
D_{-2\omega t}
\begin{pmatrix}I(t)\\-Q(t)\end{pmatrix}.
\end{align*}
Der zweite Term beschreibt ein mit Trägerfrequenz $2\omega$ moduliertes Signal.
\qedhere
\end{teilaufgaben}
\end{loesung}

\begin{diskussion}
Die in dieser Aufgabe durchgeführten Berechnungen lassen sich auf
mathematisch äquivalente Art und Weise auch mit komplexen Zahlen durchführen.
\end{diskussion}
