Gegeben sei
\[
A=\begin{pmatrix}
   2&  8&  8\\
   1&  5&  7\\
   0&  1&  5\\
\end{pmatrix},
\qquad
b=\begin{pmatrix}
10\\10\\9
\end{pmatrix}
\]
\begin{teilaufgaben}
\item Finden Sie die $LU$-Zerlegung von $A$.
\item Berechnen Sie $\det(A)$.
\item Lösen Sie das Gleichungssystem $Ly=b$.
\item Lösen Sie das Gleichungssystem $Ux=y$.
\end{teilaufgaben}

\begin{hinweis}
\gaussurl{gausscalc:70000002}
\end{hinweis}

\thema{LU-Zerlegung}

\begin{loesung}
\begin{teilaufgaben}
\item Die LU-Zerlegung wird mit dem Gauss-Algorithmus gefunden
\begin{align*}
\begin{tabular}{|>{$}c<{$}>{$}c<{$}>{$}c<{$}|}
\hline
   2&  8&  8\\
   1&  5&  7\\
   0&  1&  5\\
\hline
\end{tabular}
&\rightarrow
\begin{tabular}{|>{$}c<{$}>{$}c<{$}>{$}c<{$}|}
\hline
   1&  4&  4\\
   0&  1&  3\\
   0&  1&  5\\
\hline
\end{tabular}
\rightarrow
\begin{tabular}{|>{$}c<{$}>{$}c<{$}>{$}c<{$}|}
\hline
   1&  4&  4\\
   0&  1&  3\\
   0&  0&  2\\
\hline
\end{tabular}
\rightarrow
\begin{tabular}{|>{$}c<{$}>{$}c<{$}>{$}c<{$}|}
\hline
   1&  4&  4\\
   0&  1&  3\\
   0&  0&  1\\
\hline
\end{tabular}
\end{align*}
Im Tableau ganz rechts steht $U$, die Pivotspalten bilden $L$, also
\[
L=\begin{pmatrix}
2&0&0\\
1&1&0\\
0&1&2
\end{pmatrix}
,\qquad
U=\begin{pmatrix}
   1&  4&  4\\
   0&  1&  3\\
   0&  0&  1\\
\end{pmatrix}
\]
Kontrolle:
\[
LU=\begin{pmatrix}
2&8&8\\
1&5&7\\
0&1&5
\end{pmatrix}=A.
\]
\item $\det(A)=\det(L)\det(U)=\det(L)=4$.
\item Es ist $Ly=b$ zu lösen. Da $L$ eine Dreicksmatrix ist, kann
man die $y$ direkt bestimmen
\[
\begin{linsys}{5}
2y_1& &   & &    &=&10&\qquad\Rightarrow&y_1&=&5\\
 y_1&+&y_2& &    &=&10&\qquad\Rightarrow&y_2&=&5\\
    & &y_2&+&2y_3&=& 9&\qquad\Rightarrow&y_3&=&2\\
\end{linsys}
\]
\item
Ebenso kann man aus der Dreiecksmatrix $U$ die Lösung $x$ für
$Ux=y$ von unten nach oben bestimmen:
\[
\begin{linsys}{5}
 x_1&+&4x_2&+&4x_3&=& 5&\qquad\Rightarrow&x_1&=& 1\\
    & & x_2&+&3x_3&=& 5&\qquad\Rightarrow&x_2&=&-1\\
    & &    & & x_3&=& 2&\qquad\Rightarrow&x_3&=& 2\\
\end{linsys}
\qedhere
\]
\end{teilaufgaben}
\end{loesung}

