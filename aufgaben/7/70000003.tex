Gegeben sind
\[
A=\begin{pmatrix}1&2\\2&1\\2&1\end{pmatrix},\qquad
b=\begin{pmatrix}2\\2\\2\end{pmatrix}.
\]
Das Gleichungssystem $Ax=b$ ist überbestimmt, es wurde früher
gezeigt, dass die beste Lösung aus dem (nicht überbestimmten)
Gleichungssystem
\[
A^tAx=A^tb
\]
bestimmt werden kann.
\begin{teilaufgaben}
\item Bestimmen Sie $A^tA$ und $A^tb$.
\item $A^tA$ ist eine symmetrische Matrix, finden Sie die
Cholesky-Zerlegung $A^tA=LL^t$.
\item Lösen Sie $Ly=A^tb$.
\item Lösen Sie $L^tx=y$.
\end{teilaufgaben}

\thema{Cholesky-Zerlegung}

\begin{loesung}
\begin{teilaufgaben}
\item
\begin{align*}
A^tA&=
\begin{pmatrix}1&2&2\\2&1&1\end{pmatrix}
\begin{pmatrix}1&2\\2&1\\2&1\end{pmatrix}
=
\begin{pmatrix}
9&6\\
6&6
\end{pmatrix}
\\
A^tb&=
\begin{pmatrix}1&2&2\\2&1&1\end{pmatrix}
\begin{pmatrix}2\\2\\2\end{pmatrix}
=
\begin{pmatrix}
10\\8
\end{pmatrix}
\end{align*}
\item Nach dem in der Vorlesung besprochenen Algorithmus für die
Cholesky-Zerlegung bestimmt man zuerst das Element in der Ecke
links oben:
\[
L=\begin{pmatrix}
3&0\\
2&?
\end{pmatrix}
\]
Aus der Gleichung
\[
LL^t=\begin{pmatrix}9&6\\6&1+?^2\end{pmatrix}=\begin{pmatrix}9&6\\6&6\end{pmatrix}
\]
schliesst man dann, dass $?=\sqrt{2}$, also
\[
L=\begin{pmatrix}
3&0\\2&\sqrt{2}
\end{pmatrix}.
\]
\item $y$ ist Lösung des Gleichungssystems
\[
\begin{pmatrix}
3&0\\
2&\sqrt{2}
\end{pmatrix}y=\begin{pmatrix}10\\8\end{pmatrix}
\quad
\Rightarrow
\quad
y_1=\frac{10}{3},\quad y_2=\frac{4}{3\sqrt{2}}
\]
\item $x$ ist Lösung des Gleichungsystems
\[
\begin{pmatrix}
3&2\\
0&\sqrt{2}
\end{pmatrix}x=\begin{pmatrix}
\frac{10}{3}\\
\frac{4}{3\sqrt{2}}
\end{pmatrix}
\quad\Rightarrow\quad
x_2=\frac23,\quad x_1=\frac23
\qedhere
\]
\end{teilaufgaben}
\end{loesung}

