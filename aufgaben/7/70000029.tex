Für welche Werte von $a$ ist die Matrix
\[
A=\begin{pmatrix}
1   & 1-a       &0       \\
1-a & 2a^2-2a+1 &a-a^2   \\
0   & a-a^2     &a^2-2a+2
\end{pmatrix}
\]
positiv definit?

\begin{loesung}
Die Matrix ist positiv definit, wenn sie sich als $A=L\transpose{L}$
schreiben lässt mit einer unteren Dreiecksmatrix $L$.
Gesucht ist also eine untere Dreiecksmatrix
\[
L=
\begin{pmatrix}
l_{11}&     0&     0\\
l_{21}&l_{22}&     0\\
l_{31}&l_{32}&l_{33}
\end{pmatrix}
\]
derart, dass $A=L\transpose{L}$, also
\[
L\transpose{L}
=
\begin{pmatrix}
l_{11}&     0&     0\\
l_{21}&l_{22}&     0\\
l_{31}&l_{32}&l_{33}
\end{pmatrix}
\begin{pmatrix}
l_{11}&l_{21}&l_{31}\\
     0&l_{22}&l_{32}\\
     0&     0&l_{33}
\end{pmatrix}
=
A.
\]
Daraus gewinnt man Schritt für Schritt
\begin{align*}
a_{11}
&=
% a_11
1
&&=
l_{11}^2
&&\Rightarrow&
l_{11}
&=
1
\\
a_{12}
&=
% a_12
1-a
&&=
l_{11}l_{21}
=
2l_{21}
&&\Rightarrow&
l_{21}
&=
1-a
\\
a_{13}
&=
% a_13
0
&&=
l_{11}l_{31}
=2l_{31}
&&\Rightarrow&
l_{31}
&=
0
\\
a_{22}
&=
% a_22
2a^2-2a+1
&&=
l_{21}^2 + l_{22}^2 = (1-a)^2 + l_{22}^2
&&\Rightarrow&
l_{22}
&=
|a|
\\
a_{23}
&=
% a_23
a-a^2
&&=
l_{21}l_{31}+l_{22}l_{32}
=
|a|l_{32}
&&\Rightarrow& l_{32}
&=
(a-a^2)/|a|
\\
a_{33}
&=
% a_33
a^2-2a+2
&&=
l_{31}^2+l_{32}^2+l_{33}^2
=
(a-a^2)^2/a^2+l_{33}^2
=
a^2-2a+1+l_{33}^2
&&\Rightarrow& l_{33}
&=
1
.
\end{align*}
Damit ist eine Matrix
\[
L
=
\begin{pmatrix*}
1&0&0\\
1-a&|a|&0\\
0&(a-a^2)/|a|&1
\end{pmatrix*}
\]
mit den verlangten Eigenschaften gefunden.
Der Algorithmus funktioniert wegen der Division durch $|a|$ bei der
Berechnung von $l_{32}$ genau dann, wenn $a\ne 0$.
Ausser für $a=0$ ist $A$ also immer positiv definit.
\end{loesung}

\begin{bewertung}
Durchführung der Cholesky-Zerlegung ({\bf C}) 5 Punkte,
Schlussfolgerung ``positiv definit'' ({\bf P}) 1 Punkt.
\end{bewertung}

