Betrachten Sie die Matrix
\[
A=
\begin{pmatrix}
\sqrt{2} & 0 & 0 \\
0 & \frac32 & \frac{3\sqrt{3}}2 \\
-\sqrt{2} & 0 & 0
\end{pmatrix}
\]
\begin{teilaufgaben}
\item Bestimmen Sie die Singulärwerte von $A$.
\item Welchen Rang hat $A$?
\item Bestimmen Sie eine Orthonormalbasis von Kern und Bild von $A$
\item Bestimmen Sie eine Orthonormalbasis von $\operatorname{ker}A^\perp$
und $\operatorname{im}A^\perp$
\item Bestimmen Sie die Pseudoinverse $A^\dagger$ von $A$.
\item Berechnen Sie $P_1=AA^\dagger$ und $P_2=A^\dagger A$.
\item Überprüfen Sie, dass $P_1$ und $P_2$ Projektionen sind, also
$P_1^2=P_2$ und $P_2^2=P_2$ gilt.
\end{teilaufgaben}

\begin{hinweis}
Verwenden sie die Singulärwertzerlegung.
\end{hinweis}

\themaL{Singlarwertzerlegung}{Singulärwertzerlegung}

\begin{loesung}
Alle Teilaufgaben können, gelöst werden, wenn die Singulärwert-Zerlegung 
von $A$ bekannt ist.
Diese kann man finden, indem man das Eigenwertproblem für $B=AA^t$ und $C=A^tA$
löst:
\[
B=AA^t
=
\begin{pmatrix}
 2 & 0 & -2 \\
 0 & 9 &  0 \\
-2 & 0 &  2 
\end{pmatrix},
\qquad
C=A^tA
=
\begin{pmatrix}
4&0&0\\
0&\frac{9}{4}&\frac{9\sqrt{3}}4 \\
0&\frac{9\sqrt{3}}4&\frac{27}4
\end{pmatrix}
\]
Der zweite Standardbasisvektor $e_2$ ist ein Eigenvektor von $B$
zum Eigenwert $9$
und der erste Standardbasisvektor $e_1$ ist ein Eigenvektor von $C$
zum Eigenwert $4$.
Es müssen jetzt nur noch Eignewertprobleme für die verbleibenden
$2\times 2$-Matrizen
\[
B_0
=
\begin{pmatrix}
2&-2\\
-2&2
\end{pmatrix}
\qquad\text{und}\qquad
C_0
=
\begin{pmatrix}
\frac{9}{4}&\frac{9\sqrt{3}}4 \\
\frac{9\sqrt{3}}4&\frac{27}4
\end{pmatrix}
\]
gelöst werden.
\begin{align*}
\chi_{B_0}(\lambda)
&=
\biggl|
\begin{matrix}
2-\lambda&-2\\
-2&2-\lambda
\end{matrix}
\biggr|
&
\chi_{C_0}(\lambda)
&=
\biggl|
\begin{matrix}
\frac{9}4-\lambda & \frac{9\sqrt{3}}4 \\
\frac{9\sqrt{3}}{4} & \frac{27}4-\lambda
\end{matrix}
\biggr|
\\
&=
(2-\lambda)^2 -4
&
&=
\biggl(\frac94-\lambda\biggr)\biggl(\frac{27}4-\lambda\biggr) -\frac{243}{16}
\\
&=
\lambda^2-4\lambda
=
(\lambda-4)\lambda
&
&=
\lambda^2-9\lambda = (\lambda-9)\lambda
\end{align*}
In beiden Fällen ist einer der Eigenwert $0$, der andere ist
$4$ beziehungswiese $9$.
Die zugehörigen Eigenvektoren findet man mit dem Gauss-Algorithmus:
\begin{align*}
\lambda&=0:
&
\begin{tabular}{|>{$}c<{$}>{$}c<{$}|}
\hline
2&-2\\
-2&2\\
\hline
\end{tabular}
&\rightarrow
\begin{tabular}{|>{$}c<{$}>{$}c<{$}|}
\hline
1&-1\\
0&0\\
\hline
\end{tabular}
&
\begin{tabular}{|>{$}c<{$}>{$}c<{$}|}
\hline
\frac94&\frac{9\sqrt{3}}4 \\
\frac{9\sqrt{3}}4&\frac{27}{4}\\
\hline
\end{tabular}
&\rightarrow
\begin{tabular}{|>{$}c<{$}>{$}c<{$}|}
\hline
1&\sqrt{3} \\
0&0\\
\hline
\end{tabular}
\\
&&
\vec{u}_0
&=
\begin{pmatrix}
\frac{\sqrt{2}}2\\
\frac{\sqrt{2}}2
\end{pmatrix}
&
\vec{v}_0
&=
\begin{pmatrix}
\frac{\sqrt{3}}2\\
-\frac12
\end{pmatrix}
\\
\lambda&\ne 0:
&
\begin{tabular}{|>{$}c<{$}>{$}c<{$}|}
\hline
-2&-2\\
-2&-2\\
\hline
\end{tabular}
&\rightarrow
\begin{tabular}{|>{$}c<{$}>{$}c<{$}|}
\hline
1&1\\
0&0\\
\hline
\end{tabular}
&
\begin{tabular}{|>{$}c<{$}>{$}c<{$}|}
\hline
-\frac{27}{4}&\frac{9\sqrt{3}}4\\
\frac{9\sqrt{3}}4&-\frac{9}{4}\\
\hline
\end{tabular}
&\rightarrow
\begin{tabular}{|>{$}c<{$}>{$}c<{$}|}
\hline
1&-\frac{\sqrt{3}}3\\
0&0\\
\hline
\end{tabular}
\\
&&
\vec{u}_4
&=
\begin{pmatrix}
\frac{\sqrt{2}}2\\
-\frac{\sqrt{2}}2
\end{pmatrix}
&
\vec{v}_9
&=
\begin{pmatrix}
\frac12\\
\frac{\sqrt{3}}2
\end{pmatrix}
\end{align*}
Die Eigenvektoren sind bereits auf Länge $1$ normiert.
Die Eigenvektoren muss man jetzt in absteigender Reihenfolge der Eigenwert
ein die Matrizen $U$ und $V$ einfüllen
\begin{align*}
U
&=
\begin{pmatrix}
0&\frac{\sqrt{2}}2 & \frac{\sqrt{2}}2 \\
1&        0        &        0         \\
0&-\frac{\sqrt{2}}2 & \frac{\sqrt{2}}2
\end{pmatrix}
&
V
&=
\begin{pmatrix}
   0             & 1 & 0 \\
\frac12          & 0 & \frac{\sqrt{3}}2 \\
\frac{\sqrt{3}}2 & 0 & -\frac{1}{2}
\end{pmatrix}
\end{align*}
Die Singulärwerte sind
\[
\Sigma
=
\begin{pmatrix}
\sqrt{9} &     0    & 0 \\
    0    & \sqrt{4} & 0 \\
    0    &     0    & 0
\end{pmatrix}
=
\begin{pmatrix}
    3    &     0    & 0 \\
    0    &     2    & 0 \\
    0    &     0    & 0
\end{pmatrix}
\]
Damit ist die Singulärwertzerlegung gefunden, man kann durch Nachrechnen
prüfen, dass
\[
A = U\Sigma V^t.
\]
\begin{teilaufgaben}
\item
Die Singulärwerte von $A$ sind $3$, $2$ und $0$.
\item
Der Rang ist die Anzahl der nicht verschwindenden Singulärwerte, also
$\operatorname{Rang}A=2$.
\item
Die gesuchten Basen können durch Auswahl von Spaltenvektoren aus den 
Matrizen $U$ und $V$ gefunden werden.
\[
\operatorname{ker} A
=
\left\langle
\begin{pmatrix}
0\\ \frac{\sqrt{3}}2 \\ -\frac12
\end{pmatrix}
\right\rangle
\qquad\text{und}\qquad
\operatorname{im} A
=
\left\langle
\begin{pmatrix}
0\\1\\0
\end{pmatrix},
\begin{pmatrix}
\frac{\sqrt{2}}2\\0\\-\frac{\sqrt{2}}2
\end{pmatrix}
\right\rangle.
\]
\item
Die verbleibenden Vektoren bilden Orthonormalbasen der verlangten 
Orthogonalkomplemente:
\[
\operatorname{ker} A^\perp
=
\left\langle
\begin{pmatrix}
0 \\ \frac12 \\ \frac{\sqrt{3}}2
\end{pmatrix},
\begin{pmatrix}
1 \\ 0 \\ 0
\end{pmatrix}
\right\rangle
\qquad\text{und}\qquad
\operatorname{im} A^\perp
=
\left\langle
\begin{pmatrix}
\frac{\sqrt{2}}2 \\ 0 \\ \frac{\sqrt{2}}2
\end{pmatrix}
\right\rangle.
\]
\item
Die Pseudoinverse erhält man aus dem Produkt
\[
A^\dagger
=
V\begin{pmatrix}
\frac13&0&0\\
0&\frac12&0\\
0&0&0
\end{pmatrix}
U^t
=
\begin{pmatrix}
\frac{1}{2\sqrt{2}} &          0       & -\frac{1}{2\sqrt{2}} \\
        0           &       \frac16    & 0                    \\
        0           & \frac1{2\sqrt{3}}& 0
\end{pmatrix}.
\]
\item
Durch Ausmultiplizieren erhält man
\[
P_1
=
\begin{pmatrix}
\frac12 & 0 & -\frac12\\
0&1&0\\
-\frac12&0&\frac12
\end{pmatrix}
\qquad\text{und}\qquad
P_2
=
\begin{pmatrix}
1&0&0\\
0&\frac14&\frac{\sqrt{3}}4\\
0&\frac{\sqrt{3}}4&\frac34
\end{pmatrix}.
\]
\item Die Überprüfung ist eine simple Rechnung.
\qedhere
\end{teilaufgaben}
\end{loesung}
