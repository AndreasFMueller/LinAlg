Finden Sie die QR-Zerlegung der Matrix
\[
A
=
\renewcommand{\arraystretch}{1.4}
\begin{pmatrix}
\frac{\!\sqrt{3}}{2} & \frac{2-\!\sqrt{3}}{2} \\
  \frac{3}{2}        & \frac{2\!\sqrt{3}+1}{2}
\end{pmatrix}.
\]

\begin{hinweis}
Finden sie den zweiten orthonormierten Vektor nicht mit dem Gram-Schmidt-Verfahren
sondern verwenden sie, dass die Vektoren
\[
\begin{pmatrix}v_1\\v_2\end{pmatrix}
\qquad\text{und}\qquad
\begin{pmatrix}-v_2\\v_1\end{pmatrix}
\]
orthogonal sind.
\end{hinweis}

\begin{loesung}
Für die QR-Zerlegung müssen die Spalten der Matrix orthonormalisiert werden.
Wir verwenden dafür das Verfahren von Gram-Schmidt:
\bgroup
\renewcommand{\arraystretch}{1.4}
\begin{align*}
a_1
&=
\begin{pmatrix}
\frac{\!\sqrt{3}}{2} \\
\frac{3}{2}
\end{pmatrix}
&&\Rightarrow&
|a_1|
&=
\!\sqrt{\frac34+\frac94}
=
\!\sqrt{\frac{12}{4}}
=
\!\sqrt{3}
&&\Rightarrow&
q_1
&=
\frac{1}{\!\sqrt{3}}
\begin{pmatrix}
\frac{\!\sqrt{3}}{2} \\
\frac{3}{2}
\end{pmatrix}
=
\begin{pmatrix}
\frac12 \\
\frac{\!\sqrt{3}}{2}
\end{pmatrix}
\intertext{Einen auf $q_1$ orthogonaler Einheitsvektor bekommt man, indem man die
beiden Komponenten vertauscht und bei einer Komponente das Vorzeichen kehrt:}
&
&&&
&&&
&
q_2
&=
\begin{pmatrix}
-\frac{\!\sqrt{3}}{2}\\
 \frac{1}{2}
\end{pmatrix}.
\intertext{Diesen Vektor bekommt man auch auf viel aufwendigere Art mit dem
Gram-Schmidt-Verfahren finden:}
a_2
&=
\begin{pmatrix}
\frac{2-\!\sqrt{3}}{2} \\
\frac{2\!\sqrt{3}-1}{2}
\end{pmatrix}
&&\Rightarrow&
q_2
&=
\frac{
a_2 - (q_1\cdot a_2) q_1
}{\dots}
%\\
%&&&&
%&
\rlap{$\displaystyle
=
\frac{
\begin{pmatrix}
\frac{2-\!\sqrt{3}}{2} \\
\frac{2\!\sqrt{3}-1}{2}
\end{pmatrix}
-
\left(
\frac{2-\!\sqrt{3}}{4}
+
\frac{2\cdot 3-\!\sqrt{3}}{4}
\right)
\begin{pmatrix}
\frac12 \\
\frac{\!\sqrt{3}}{2}
\end{pmatrix}
}{\dots}
$}
\\
&&&&
&=
\frac{
\begin{pmatrix}
\frac{2-\!\sqrt{3}}{2} \\
\frac{2\!\sqrt{3}-1}{2}
\end{pmatrix}
-
\left(
\frac{4-\!\sqrt{3}}{2}
\right)
\begin{pmatrix}
\frac12 \\
\frac{\!\sqrt{3}}{2}
\end{pmatrix}
}{\dots}
%\\
%&&&&
%&
=
\rlap{$\displaystyle
\frac{
\begin{pmatrix}
\frac{4-2\!\sqrt{3}}{4} \\
\frac{4\!\sqrt{3}-2}{4}
\end{pmatrix}
-
\begin{pmatrix}
\frac{4-\!\sqrt{3}}{4}
\\
\frac{4\!\sqrt{3}-\!3}{4}
\end{pmatrix}
}{\dots}
%\\
%&&&&
%&
=
\frac{
\begin{pmatrix}
-\frac{\!\sqrt{3}}{4} \\
\frac{1}{4}
\end{pmatrix}
}{\dots}
=
\begin{pmatrix}
-\frac{\!\sqrt{3}}{2} \\
\frac{1}{2}
\end{pmatrix}.$}
\end{align*}
Daraus kann man ablesen, dass die Matrix $Q$ die Form
\[
Q
=
\begin{pmatrix}
\frac12              & -\frac{\!\sqrt{3}}{2} \\
\frac{\!\sqrt{3}}{2} & \frac{1}{2}
\end{pmatrix}.
\]
Um jetzt $R$ zu bestimmen, berechnen wir
\begin{align*}
A&=QR
&&\Rightarrow&
R&=
Q^tA
=
\begin{pmatrix}
\frac12               & \frac{\!\sqrt{3}}{2} \\
-\frac{\!\sqrt{3}}{2} & \frac{1}{2}
\end{pmatrix}
\begin{pmatrix}
\frac{\!\sqrt{3}}{2} & \frac{2-\!\sqrt{3}}{2} \\
  \frac{3}{2}        & \frac{2\!\sqrt{3}+1}{2}
\end{pmatrix}
=
\begin{pmatrix}
\frac{\!\sqrt{3}+3\!\sqrt{3}}{4} & 
\frac{2-\!\sqrt{3}+6+\!\sqrt{3}}{4}
\\
0 & \frac{3-2\!\sqrt{3}+2\!\sqrt{3}+1}{4}
\end{pmatrix}
=
\begin{pmatrix}
\!\sqrt{3} & 2 \\
0 & 1
\end{pmatrix}
\end{align*}
\egroup
Damit ist die QR-Zerlegung gefunden.
\end{loesung}

