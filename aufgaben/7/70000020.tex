Können Sie die Matrix
\[
A=\begin{pmatrix}
49& -35&-28\\
-35&106&27a + 20\\
-28&27a + 20&9a^2 + 17
\end{pmatrix}
\]
als Produkt $A=BB^t$ schreiben? Geben Sie eine mögliche Matrix $B$ an.

\thema{Cholesky-Zerlegung}

\begin{loesung}
Wir verwenden den Algorithmus für die Cholesky-Zerlegung. Er liefert eine 
untere Dreiecksmatrix $L$ mit $A = LL^t$.

Im ersten Schritt versucht man die erste Spalte von $L$ zu bestimmen.
\[
LL^t=
\begin{pmatrix}
l_{11}&  0&  0\\
l_{21}&  ?&  0\\
l_{31}&  ?&  ?
\end{pmatrix}
\begin{pmatrix}
l_{11}&l_{21}&l_{31}\\
     0&     ?&     ?\\
     0&     0&     ?
\end{pmatrix}
=
\begin{pmatrix}
    l_{11}^2&l_{11}l_{21}&l_{11}l_{31}\\
l_{21}l_{11}&           *&           *\\
l_{31}l_{11}&           *&           *
\end{pmatrix}
=
A
\qquad\Rightarrow\qquad
\left\{
\begin{aligned}
l_{11}&=7\\
l_{21}&=-5\\
l_{31}&=-4
\end{aligned}
\right.
\]
Diese Daten können wir jetzt verwenden, um die zweite Spalte zu bestimmen:
\[
LL^t
=
\begin{pmatrix}
7&     0&0\\
-5&l_{22}&0\\
-4&l_{32}&?
\end{pmatrix}
\begin{pmatrix}
7&    -5&    -4\\
0&l_{22}&l_{32}\\
0&     0&?
\end{pmatrix}
=
\begin{pmatrix}
49&    -35&    -28\\
-35& 25+l_{22}^2  & 20+l_{22}l_{32}\\
-28& 20+l_{32}l_{22}&         *
\end{pmatrix}
=
A
\qquad\Rightarrow\qquad
\left\{
\begin{aligned}
l_{22}&=9\\
l_{32}&=3a
\end{aligned}
\right.
\]
Somit bleibt nur noch das Element $l_{33}$ zu bestimmen:
\[
LL^t=
\begin{pmatrix}
7&     0&0\\
-5&9&0\\
-4&3a&l_{33}
\end{pmatrix}
\begin{pmatrix}
7&    -5&    -4\\
0&9&3a\\
0&     0&l_{33}
\end{pmatrix}
=
\begin{pmatrix}
49&-35 & -28\\
-35&106 &27a+20\\
-28&27a+20&  16+9a^2+l_{33}^2
\end{pmatrix}
=
A
\qquad\Rightarrow\qquad
l_{33}=1
\]
Die gesuchte Matrix $B$ ist also
\[
B = L=
\begin{pmatrix}
7&     0&0\\
-5&9&0\\
-4&3a&1
\end{pmatrix}.
\]
Kontrolle:
\[
\begin{pmatrix}
7&     0&0\\
-5&9&0\\
-4&3a&1
\end{pmatrix}
\begin{pmatrix}
7&  -5& -4\\
0&9&3a\\
0& 0&1
\end{pmatrix}
=
\begin{pmatrix}
49& -35&-28\\
-35&106&27a + 20\\
-28&27a + 20&9a^2 + 17
\end{pmatrix}.
\qedhere
\]
\end{loesung}

\begin{bewertung}
Cholesky-Zerlegung ({\bf C}) 1 Punkt,
Pro richtigem nicht verschwindendem Matrix-Element in $B$ ein Punkt,
maximal 6 Punkte.
\end{bewertung}

