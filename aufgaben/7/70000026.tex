%(%i2) l:matrix([5*u,0,0],[v,3*u,0],[2*u,-v,u])
%                                [ 5 u   0   0 ]
%                                [             ]
%(%o2)                           [  v   3 u  0 ]
%                                [             ]
%                                [ 2 u  - v  u ]
%(%i3) l . transpose(l)
%                        [     2                   2   ]
%                        [ 25 u     5 u v      10 u    ]
%                        [                             ]
%(%o3)                   [         2      2            ]
%                        [ 5 u v  v  + 9 u     - u v   ]
%                        [                             ]
%                        [     2              2      2 ]
%                        [ 10 u     - u v    v  + 5 u  ]
%
Berechnen Sie die Determinante der Matrix 
\[
A=
\begin{pmatrix}
25u^2 &     5uv   & 10u^2 \\
5uv   &  v^2+9u^2 & -uv   \\
10u^2 &     -uv   & v^2+5u^2
\end{pmatrix}
\qquad\text{mit $u>0$}
\]
mit Hilfe der Cholesky-Zerlegung.

\begin{loesung}
Für die Cholesky-Zerlegung setzt man 
\[
L = \begin{pmatrix}
l_{11}& 0 & 0 \\
l_{21}&l_{22}& 0 \\
l_{31}&l_{32}&l_{33}
\end{pmatrix}
\]
und berechnet dann schrittweise die Matrixelemente $l_{ik}$ von $L$
durch Ausmultiplizieren von $LL^t$ und Vergleich mit $A$:
\[
\begin{aligned}
l_{11}^2                   &=  25u^2       &&\Rightarrow&    l_{11} &= 5u  \\
l_{11}l_{21}               &=  5uv         &&\Rightarrow&    l_{21} &= v   \\
l_{11}l_{31}               &=  10u^2       &&\Rightarrow&    l_{31} &= 2u  \\
l_{22}^2+l_{21}^2          &=  v^2 + 9u^2  &&\Rightarrow&           &      \\
l_{22}^2                   &=  9u^2        &&\Rightarrow&    l_{22} &= 3u  \\
l_{22}l_{32}+l_{21}l_{31}  &=  -uv         &&           &           &      \\
3ul_{32}                   &= -uv-2uv      &&\Rightarrow&    l_{32} &= -v  \\
l_{33}^2+l_{32}^2+l_{31}^2 &=  v^2+5u^2    &&           &           &      \\
l_{33}^2                   &=  5u^2-4u^2   &&\Rightarrow&    l_{33} &=  u.
\end{aligned}
\]
Daher ist
\[
L=\begin{pmatrix}
 5u &  0 & 0 \\
  v & 3u & 0 \\
 2u & -v & u
\end{pmatrix}
\Rightarrow
\det L
=
15u^3.
\]
Damit kann man jetzt auch die Determinante von $A$ berechnen:
$\det A = \det LL^t = (\det L)^2 = 15^2u^6 = 225u^6.$
\end{loesung}

\begin{bewertung}
Cholesky-Zerlegung ({\bf C}) 4 Punkte,
Determinante von $L$ ({\bf L}) 1 Punkt,
Determinante von $A$ ({\bf A}) 1 Punkt.
\end{bewertung}

