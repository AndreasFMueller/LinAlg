%(%i2) l:matrix([a,0,0],[-1,1,0],[a,-1,a])
%                                [  a    0   0 ]
%                                [             ]
%(%o2)                           [ - 1   1   0 ]
%                                [             ]
%                                [  a   - 1  a ]
%(%i3) l . transpose(l)
%                         [  2                  2     ]
%                         [ a       - a        a      ]
%                         [                           ]
%(%o3)                    [ - a      2      (- a) - 1 ]
%                         [                           ]
%                         [  2                 2      ]
%                         [ a    (- a) - 1  2 a  + 1  ]
%
Sei $A$ die Matrix
\[
A
=
\begin{pmatrix}
a^2     &   -a    &   a^2    \\
-a      &    2    & -a-1     \\
a^2     &  -a-1   & 2a^2 + 1
\end{pmatrix}
\]
mit $a>0$.
\begin{teilaufgaben}
\item
Ist $A$ positiv definit?
\item
Bestimmen Sie $\det A$.
\end{teilaufgaben}

\thema{Cholesky-Zerlegung}

\begin{loesung}
\begin{teilaufgaben}
\item
Die Cholesky-Zerlegung von $A$, die man mit dem in der Vorlesung
gezeigten Verfahren direkt finden kann, ist
\[
L=
\begin{pmatrix}
 a& 0& 0\\
-1& 1& 0\\
 a&-1& a
\end{pmatrix}
\qquad\Rightarrow\qquad
A
=
\begin{pmatrix}
 a& 0& 0\\
-1& 1& 0\\
 a&-1& a
\end{pmatrix}
\begin{pmatrix}
 a&-1& a\\
 0& 1&-1\\
 0& 0& a
\end{pmatrix}.
\]
Dabei muss man an zwei Stellen eine Quadratwurzel von $a^2$ bestimmen,
welche wegen $a>0$ tatsächlich $a$ ist: $\sqrt{a^2}=a$.
Da die Cholesky-Zerlegung möglich ist, ist $A$ positiv definit.
\item
$\det A = \det L\transpose{L} =(\det L)^2 = (a^2)^2 = a^4$.
\qedhere
\end{teilaufgaben}
\end{loesung}

\begin{bewertung}
Berechnung der Cholesky-Zerlegung ({\bf C}) 4 Punkte,
Aussage darüber, ob $A$ positiv definit ist ({\bf P}) 1 Punkt,
Determinante ({\bf D}) 1 Punkt.
\end{bewertung}
