Ist die Matrix
\[
A=\begin{pmatrix}
1&b&1\\
b&b^2+1&1 \\
1&1&b^2-2b+3
\end{pmatrix}
\]
positiv definit?

\begin{loesung}
Die Matrix ist positiv definit, wenn sich die Cholesky-Zerlegung 
durchführen lässt.
Gesucht ist also eine untere Dreiecksmatrix $L$ derart, dass $L\transpose{L}=A$.
Aus
\[
L
=
\begin{pmatrix}
l_{11}&0&0\\
l_{21}&l_{22}&0\\
l_{31}&l_{32}&l_{33}
\end{pmatrix}
\qquad\Rightarrow\qquad
L\transpose{L}
=
\begin{pmatrix}
l_{11}&0&0\\
l_{21}&l_{22}&0\\
l_{31}&l_{32}&l_{33}
\end{pmatrix}
\begin{pmatrix}
l_{11}&l_{21}&l_{31}\\
0&l_{22}&l_{32}\\
0&0&l_{33}
\end{pmatrix}
\]
bekommt man durch Ausmultiplizieren des Produktes $L\transpose{L}$
\begin{align*}
a_{11}&=1=l_{11}^2 &&\Rightarrow&l_{11}&=1
\\
a_{12}&=b=l_{11}l_{21}&&\Rightarrow&l_{21}&=b
\\
a_{13}&=1=l_{11}l_{31}&&\Rightarrow&l_{31}&=1
\\
a_{22}&=b^2+1=l_{22}^2+l_{21}^2=l_{22}^2+b^2&&\Rightarrow&l_{22}&=1
\\
a_{23}&=1=l_{21}l_{31}+l_{22}l_{32}=b+l_{32}&&\Rightarrow&l_{32}&=1-b
\\
a_{33}&=b^2-2b+3=l_{31}^2+l_{32}^2+l_{33}^2=1+(1-b)^2+l_{33}^2=b^2-2b+2+l_{33}^2
&&\Rightarrow&l_{33}&=1.
\end{align*}
Damit ist gezeigt, dass
\[
L=\begin{pmatrix}
1&0&0\\
b&1&0\\
1&1-b&1
\end{pmatrix}
\]
die untere Dreiecksmatrix der Cholesky-Zerlegung von $A$ ist.
Es folgt, dass $A$ positiv definit ist.
\end{loesung}

\begin{diskussion}
Da die Matrix $L$ unital ist, kann man die Cholesky-Zerlegung in diesem 
Fall auch mit Hilfe der LR-Zerlegung finden.
Da die gefundenen Matrizen $L$ und $R$ übereinstimmen, hat man eine
Cholesky-Zerlegung gefunden und damit auch nachgewiesen, dass die
Matrix positiv definit ist.
\end{diskussion}

\begin{bewertung}
Druchführung der Cholesky-Zerlegung ({\bf C}) 5 Punkte,
Schlussfolgerung ``positiv definit'' ({\bf P}) 1 Punkt.
\end{bewertung}

