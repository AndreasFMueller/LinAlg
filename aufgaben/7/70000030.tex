Für welche Werte von $c$ ist die Matrix
\[
A=\begin{pmatrix}
 4 &     2 &  -2 \\
 2 & c^2+1 & c-1 \\
-2 & c  -1 &   6
\end{pmatrix}
\]
positiv definit?

\begin{loesung}
Die Matrix ist positiv definit, wenn sie sich als $A=L\transpose{L}$
mit einer unteren Dreiecksmatrix $L$ schreiben lässt.
Gesucht ist also eine untere Dreiecksmatrix
\[
L=
\begin{pmatrix}
l_{11}&     0&     0\\
l_{21}&l_{22}&     0\\
l_{31}&l_{32}&l_{33}
\end{pmatrix}
\]
derart, dass $A=L\transpose{L}$, also
\[
L\transpose{L}
=
\begin{pmatrix}
l_{11}&     0&     0\\
l_{21}&l_{22}&     0\\
l_{31}&l_{32}&l_{33}
\end{pmatrix}
\begin{pmatrix}
l_{11}&l_{21}&l_{31}\\
     0&l_{22}&l_{32}\\
     0&     0&l_{33}
\end{pmatrix}
=
A.
\]
Daraus gewinnt man Schritt für Schritt
\begin{equation*}
\renewcommand{\arraycolsep}{3pt}
\begin{array}{rclclclccclcrcl}
a_{11}
&=&
% a_11
4
&=&
l_{11}^2
&&
&&
&&\Rightarrow&&
l_{11}
&=&
2
\\
a_{12}
&=&
% a_12
2
&=&
l_{11}l_{21}
&=&
2l_{21}
&&
&&\Rightarrow&&
l_{21}
&=&
1
\\
a_{13}
&=&
% a_13
-2
&=&
l_{11}l_{31}
&=&
2l_{31}
&&
&&\Rightarrow&&
l_{31}
&=&
-1
\\
a_{22}
&=&
% a_22
c^2+1
&=&
l_{21}^2 + l_{22}^2
&=&
1 + l_{22}^2
&&
&&\Rightarrow&&
l_{22}
&=&
|c|
\\
a_{23}
&=&
% a_23
c-1
&=&
l_{21}l_{31}+l_{22}l_{32}
&=&
-1+|c|l_{32}
&&
&&\Rightarrow&&
l_{32}
&=&
c/|c|
\\
a_{33}
&=&
% a_33
6
&=&
l_{31}^2+l_{32}^2+l_{33}^2
&=&
1+1+l_{33}^2
&=&
2+l_{33}^2
&&\Rightarrow&&
l_{33}
&=&
2
.
\end{array}
\end{equation*}
Damit ist eine Matrix
\[
L
=
\begin{pmatrix*}[r]
2&0&0\\
1&c&0\\
-1&1&2
\end{pmatrix*}
\]
mit den verlangten Eigenschaften gefunden.
Der Algorithmus funktioniert wegen der Division durch $|c|$ bei der
Berechnung von $l_{32}$ genau dann, wenn $c\ne 0$ ist.
Somit ist $A$ genau dann positiv, wenn $c\ne 0$.
\end{loesung}

\begin{bewertung}
Durchführung der Cholesky-Zerlegung ({\bf C}) 5 Punkte,
Schlussfolgerung ``positiv definit'' ({\bf P}) 1 Punkt.
\end{bewertung}

