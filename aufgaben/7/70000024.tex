Finden Sie die Lösung $x$ der Gleichung
\[
Ax
=
b
\qquad\text{mit}\qquad
A
=
\begin{pmatrix}
\frac{5}{\sqrt{2}}          & \frac{5}{\sqrt{2}}          \\
-\frac{5\sqrt{3}}{\sqrt{2}} & -\frac{5\sqrt{3}}{\sqrt{2}}
\end{pmatrix}
\qquad\text{und}\qquad
b
=
\begin{pmatrix}
1\\
2
\end{pmatrix}
\]
für die $|x|$ minimal ist.

\begin{hinweis}
Jacobi-Algorithmus für die Eigenwertprobleme für
$A\transpose{A}$:
\jacobiurl{gausscalc:70000024-ata},
und
$\transpose{A}A$:
\jacobiurl{gausscalc:70000024-taa}
\end{hinweis}

\themaL{Singularwertzerlegung}{Singulärwertzerlegung}

\begin{loesung}
Mit der Pseudoinversen kann man die Lösung als $x=A^\dagger b$ finden.
Dazu muss man zunächst die Eigenwertprobleme für $A\transpose{A}$ und
$\transpose{A}A$ lösen:
\begin{align*}
A\transpose{A}
&=
\begin{pmatrix}
 25        &-25\sqrt{3} \\
-25\sqrt{3}& 75       
\end{pmatrix}
&
\transpose{A}A
&=
\begin{pmatrix}
50&50\\
50&50
\end{pmatrix}
\\
\chi_{A\transpose{A}}(\lambda)
&=
\biggl|
\begin{matrix}
25-\lambda  &-25\sqrt{3} \\
-25\sqrt{3} & 75-\lambda
\end{matrix}
\biggr|
&
\chi_{\transpose{A}A}(\lambda)
&=
\biggl|
\begin{matrix}
50-\lambda&50\\
50&50-\lambda
\end{matrix}
\biggr|
\\
&=
(25-\lambda)(75-\lambda) - 25^2\cdot 3
&
&=
(50-\lambda)^2-50^2
\\
&=
\lambda^2 - 100\lambda
=
\lambda(\lambda-100)
&
&=
\lambda^2-100\lambda
=
\lambda(\lambda-100)
\end{align*}
Die Eigenwerte sind also in beiden Fällen $\lambda=10$ und $\lambda=0$,
wie erwartet.
Die Eigenvektoren sind mit dem Gauss-Algorithmus zu bestimmen:
\begin{align*}
\lambda&=0:
&
\begin{tabular}{|>{$}c<{$}>{$}c<{$}|}
\hline
25&-25\sqrt{3}\\
-25\sqrt{3}&75\\
\hline
\end{tabular}
&\rightarrow
\begin{tabular}{|>{$}c<{$}>{$}c<{$}|}
\hline
1&-\sqrt{3}\\
0&0\\
\hline
\end{tabular}
&
\begin{tabular}{|>{$}c<{$}>{$}c<{$}|}
\hline
50&50\\
50&50\\
\hline
\end{tabular}
&\rightarrow
\begin{tabular}{|>{$}c<{$}>{$}c<{$}|}
\hline
1&1\\
0&0\\
\hline
\end{tabular}
\\
&&
\vec{u}_2
&=
\begin{pmatrix}\frac{\sqrt{3}}{2}\\-\frac{1}{2}\end{pmatrix}
&
\vec{v}_2
&=
\begin{pmatrix}
\frac{\sqrt{2}}2\\
-\frac{\sqrt{2}}2
\end{pmatrix}
\\
\lambda&=100:
&
\begin{tabular}{|>{$}c<{$}>{$}c<{$}|}
\hline
-75 & -25\sqrt{3} \\
-25\sqrt{3}& -25 \\
\hline
\end{tabular}
&\rightarrow
\begin{tabular}{|>{$}c<{$}>{$}c<{$}|}
\hline
1 & \frac{1}{\sqrt{3}} \\
0& 0 \\
\hline
\end{tabular}
&
\begin{tabular}{|>{$}c<{$}>{$}c<{$}|}
\hline
-50&50\\
50&-50\\
\hline
\end{tabular}
&\rightarrow
\begin{tabular}{|>{$}c<{$}>{$}c<{$}|}
\hline
1&-1\\
0& 0\\
\hline
\end{tabular}
\\
&&
\vec{u}_1
&=
\begin{pmatrix}
 \frac12 \\
-\frac{\sqrt{3}}2
\end{pmatrix}
&
\vec{v}_1
&=
\begin{pmatrix}
\frac{\sqrt{2}}2\\
\frac{\sqrt{2}}{2}
\end{pmatrix}
\end{align*}

Für die Singulärwertzerlegung muss man sicherstellen, dass $Av_1=u_1$ ist
(das Vorzeichens eines Eigenvektors lässt sich nicht anders festlegen).
Tatsächlich ist die erste Spalte von $A$ ein Vektor der Länge
$\sqrt{50}=5\sqrt{2}$.
Normiert man die erste Spalte von $A$, erhält man
\[
\begin{pmatrix}
\frac{1}{2}\\-\frac{\sqrt{3}}2 
\end{pmatrix}
=
\vec{u}_1,
\]
die Bedingung ist also erfüllt.

Die Singulärwertzerlegung von $A$ ist daher
\[
A
=
U\Sigma \transpose{V}
\qquad\text{mit}\qquad
U
=
\begin{pmatrix}
\frac{1}{2}         & \frac{\sqrt{3}}{2} \\
-\frac{\sqrt{3}}{2} & \frac{1}{2}       
\end{pmatrix}
\quad \text{und}\quad
V
=
\begin{pmatrix}
\frac{\sqrt{2}}2 & \frac{\sqrt{2}}2 \\
\frac{\sqrt{2}}2 &-\frac{\sqrt{2}}2
\end{pmatrix}.
\]
Die Pseudoinverse kann jetzt als Produkt
\[
A^\dagger
=
V\Sigma^\dagger \transpose{U}
=
V\begin{pmatrix}
\frac{1}{10} & 0 \\
      0      & 0
\end{pmatrix} \transpose{U}
=
\frac{1}{20\sqrt{2}}
\begin{pmatrix}
1&-\sqrt{3}\\
1&-\sqrt{3}
\end{pmatrix}
\]
gefunden werden.
Die gesuchte optimale Lösung  ist jetzt
\[
x
=
A^\dagger b
=
A^\dagger \begin{pmatrix}1\\2\end{pmatrix}
=
\frac{1-2\sqrt{3})}{20\sqrt{2}}
\begin{pmatrix}
1\\1
\end{pmatrix}.
\qedhere
\]
\end{loesung}

