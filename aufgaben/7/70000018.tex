Können Sie die Matrix
\[
A
=
\begin{pmatrix}
a^2&a^2&ab\\
a^2&2a^2&ab+a^2\\
ab&ab+a^2&b^2+2a^2
\end{pmatrix}
\]
als Produkt $A=BB^t$ schreiben?
Wenn ja bestimmen Sie eine solche Matrix $B$.

\thema{Cholesky-Zerlegung}

\begin{loesung}
Wir verwenden den Algorithmus für die Cholesky-Zerlegung, er liefert
eine untere Dreiecksmatrix $L$ mit $A=LL^t$.

Im ersten Schritt versucht man die erste Spalte von $L$ zu bestimmen:
\[
LL^t
=
\begin{pmatrix}
l_{11}&0&0\\
l_{21}&?&0\\
l_{31}&?&?
\end{pmatrix}
\begin{pmatrix}
l_{11}&l_{21}&l_{31}\\
0&?&?\\
0&0&?
\end{pmatrix}
=
\begin{pmatrix}
l_{11}^2&l_{11}l_{21}&l_{11}l_{31}\\
&&\\
&&
\end{pmatrix}
=
A
\qquad\Rightarrow\qquad
\left\{
\begin{aligned}
l_{11}&=a\\
l_{21}&=a\\
l_{31}&=b
\end{aligned}
\right.
\]
Diese Daten können wir jetzt verwenden, um die zweite Spalte zu bestimmen:
\[
LL^t
=
\begin{pmatrix}
a&0&0\\
a&l_{22}&0\\
b&l_{32}&?
\end{pmatrix}
\begin{pmatrix}
a&a&b\\
0&l_{22}&l_{32}\\
0&0&?
\end{pmatrix}
=
\begin{pmatrix}
a^2&a^2&ab\\
a^2&a^2+l_{22}^2& ab+l_{22}l_{32}\\
ab &ab+l_{22}l_{32} &?
\end{pmatrix}
=
A
\qquad\Rightarrow\qquad
\left\{
\begin{aligned}
l_{22}&=a\\
l_{32}&=a
\end{aligned}
\right.
\]
Somit bleibt nur noch das Element $l_{33}$ zu bestimmen:
\[
LL^t
=
\begin{pmatrix}
a&0&0\\
a&a&0\\
b&a&l_{33}
\end{pmatrix}
\begin{pmatrix}
a&a&b\\
0&a&a\\
0&0&l_{33}
\end{pmatrix}
=
\begin{pmatrix}
a^2&a^2&ab\\
a^2&2a^2& ab+a^2\\
ab &ab+a^2&a^2+b^2 +l_{33}^2
\end{pmatrix}
=
A
\qquad\Rightarrow\qquad
l_{33}=a
\]
Kontrolle:
\[
\begin{pmatrix}
a&0&0\\
a&a&0\\
b&a&a
\end{pmatrix}
\begin{pmatrix}
a&a&b\\
0&a&a\\
0&0&a
\end{pmatrix}
=
\begin{pmatrix}
a^2&a^2&ab\\
a^2&2a^2&ab+a^2\\
ab&ab+a^2&b^2+2a^2
\end{pmatrix}
=A.
\qedhere
\]
\end{loesung}

\begin{bewertung}
Cholesky-Zerlegung ({\bf C}) 1 Punkt, pro richtigem nicht verschwindendem
Matrix-Element in $B$ ein Punkt, maximal 6 Punkte.
\end{bewertung}

