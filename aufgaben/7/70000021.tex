Warum ist die Matrix
\[
A
=
\begin{pmatrix}
1&a&a^2\\
a&a^2+1&a^3+a\\
a^2&a^3+a&a^4+a^2+1
\end{pmatrix}
\]
positiv definit?

\thema{Cholesky-Zerlegung}

\begin{loesung}
Eine Matrix ist positiv definit, wenn sie eine Cholesky-Zerlegung hat,
also eine Matrix
\[
L
=
\begin{pmatrix}
l_{11}&0&0\\
l_{21}&l_{22}&0\\
l_{31}&l_{32}&l_{33}
\end{pmatrix}
\]
derart existiert, dass $LL^t=A$ gilt.
Durch Ausführen der Matrixmultiplikation
\[
LL^t
=
\begin{pmatrix}
l_{11}&0&0\\
l_{21}&l_{22}&0\\
l_{31}&l_{32}&l_{33}
\end{pmatrix}
\begin{pmatrix}
l_{11}&l_{21}&l_{31}\\
0&l_{22}&l_{32}\\
0&0&l_{33}
\end{pmatrix}
=
\begin{pmatrix}
1&a&a^2\\
a&a^2+1&a^3+a\\
a^2&a^3+a&a^4+a^2+1
\end{pmatrix}
\]
findet man der Reihe nach:
\begin{align*}
a_{11}&=1&
&=
l_{11}^2 =1
&&\Rightarrow&
l_{11}&=1\\
a_{12}&=a&
&=
l_{11}\cdot l_{21}
=
l_{21}
&&\Rightarrow&
l_{21}&=a\\
a_{13}&=a^2&
&=
l_{11}\cdot l_{31}=1\cdot l_{31}
&&\Rightarrow&
l_{31}&=a^2\\
a_{22}&=a^2+1&
&=
l_{21}^2 + l_{22}^2=a^2 + l_{22}^2
&&\Rightarrow&
l_{22}&=1\\
a_{23}&=a^3+a&
&=
l_{22}\cdot l_{32} + l_{21}\cdot l_{31} 
=
l_{32} + a\cdot a^2
&&\Rightarrow&
l_{32}&=a\\
a_{33}&=a^4+a^2+1&
&=
l_{31}^2 + l_{32}^2 + l_{33}^2=a^4+a^2+l_{33}^2
&&\Rightarrow&
l_{33}&=1
\end{align*}
Somit liefert
\[
L=\begin{pmatrix}
1&0&0\\
a&1&0\\
a^2&a&1
\end{pmatrix}
\]
die Cholesky-Zerlegung der Matrix $A=LL^t$.
Damit folgt jetzt auch, dass die Matrix $A$ positiv definit ist.
\end{loesung}

\begin{bewertung}
Cholesky-Zerlegung ({\bf C}) 1 Punkt,
Durchführung ({\bf D}) 4 Punkte,
Kriterium: wenn die Cholesky-Zerlegung durchführbar ist, ist die
Matrix positiv definit ({\bf K}) 1 Punkt.
\end{bewertung}
