Können Sie die symmetrische Matrix
\[
A=\begin{pmatrix}
 4& 6&10\\
 6&18&39\\
10&39&q^2 + 89
\end{pmatrix}
\]
mit $q>0$ als Produkt $A=BB^t$ schreiben?
Geben Sie eine mögliche Matrix $B$ an.

\thema{Cholesky-Zerlegung}

\begin{loesung}
Man kann sogar eine untere Dreiecksmatrix $L$ mit dieser Eigenschaft finden,
dies ist die Cholesky-Zerlegung von $A$.

Im ersten Schritt sucht man die erste Spalte von $L$ zu bestimmen.
Es muss gelten
\[
LL^t=
\begin{pmatrix}
l_{11}&  0&  0\\
l_{21}&  ?&  0\\
l_{31}&  ?&  ?
\end{pmatrix}
\begin{pmatrix}
l_{11}&l_{21}&l_{31}\\
     0&     ?&     ?\\
     0&     0&     ?
\end{pmatrix}
=
\begin{pmatrix}
    l_{11}^2&l_{11}l_{21}&l_{11}l_{31}\\
l_{21}l_{11}&           *&           *\\
l_{31}l_{11}&           *&           *
\end{pmatrix}
=
\begin{pmatrix}
 4& 6&10\\
 6&18&39\\
10&39&q^2 + 89
\end{pmatrix}
\]
Daraus kann man ablesen, dass $l_{11}=2$ sein muss, und weiter,
dass
$l_{21}=3$ und $l_{31}=5$. Damit ist die erste Spalte bestimmt.

Im zweiten Schritt versucht man, die zweite Spalte zu bestimmen.
Dazu schreibt man wieder
\[
LL^t
=
\begin{pmatrix}
2&     0&0\\
3&l_{22}&0\\
5&l_{32}&?
\end{pmatrix}
\begin{pmatrix}
2&     3&     5\\
0&l_{22}&l_{32}\\
0&     0&?
\end{pmatrix}
=
\begin{pmatrix}
 4&              6&            10\\
 6& 9+l_{22}^2    &15+l_{22}l_{32}\\
10&15+l_{32}l_{22}&             *
\end{pmatrix}
=
\begin{pmatrix}
 4& 6&10\\
 6&18&39\\
10&39& q^2 + 89
\end{pmatrix}
\]
Daraus liest man ab $9+l_{22}^2=9$ und damit $l_{22}=3$, und weiter
$l_{32}=8$.

Im dritten Schritt ist jetzt nur noch das Element unten rechts zu bestimmen:
\[
LL^t=
\begin{pmatrix}
2&0&     0\\
3&3&     0\\
5&8&l_{33}
\end{pmatrix}
\begin{pmatrix}
2&3&     5\\
0&3&     8\\
0&0&l_{33}
\end{pmatrix}
=
\begin{pmatrix}
 4& 6& 10\\
 6&18& 39\\
10&39& 25+64+l_{33}^2
\end{pmatrix}
=
\begin{pmatrix}
 4& 6&10\\
 6&18&39\\
10&39& q^2 + 89
\end{pmatrix}
\]
woraus man ablesen kann: $13+l_{33}^2=13+q^2$ oder $l_{33}^2=q^2$, $l_{33}=q$.
Die gesuchte Matrix $L$ ist also
\[
L=
\begin{pmatrix}
2&0&0\\
3&3&0\\
5&8&q
\end{pmatrix}.
\]
Kontrolle:
\[
\begin{pmatrix}
2&0&0\\
3&3&0\\
5&8&q
\end{pmatrix}
\begin{pmatrix}
2&3&5\\
0&3&8\\
0&0&q
\end{pmatrix}
=
\begin{pmatrix}
 4& 6&10\\
 6&18&39\\
10&39&q^2 + 89
\end{pmatrix}.
\qedhere
\]
\end{loesung}

\begin{bewertung}
Cholesky-Zerlegung ({\bf C}) 1 Punkt,
Pro richtigem nicht verschwindendem Matrix-Element in $B$ ein Punkt,
maximal 6 Punkte.
\end{bewertung}

