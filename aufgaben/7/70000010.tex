Können Sie die symmetrische Matrix
\[
A=\begin{pmatrix}
 9& 3& 12\\
 3& 2&  9\\
12& 9&122
\end{pmatrix}
\]
als Produkt $A=BB^t$ schreiben? Geben Sie eine mögliche Matrix $B$ an.

\thema{Cholesky-Zerlegung}

\begin{loesung}
Man kann sogar eine untere Dreiecksmatrix $L$ mit dieser Eigesnchaft finden,
dies ist die Cholesky-Zerlegung von $A$.

Im ersten Schritt sucht man die erste Spalte von $L$ zu bestimmen.
Es muss gelten
\[
LL^t=
\begin{pmatrix}
l_{11}&  0&  0\\
l_{21}&  ?&  0\\
l_{31}&  ?&  ?
\end{pmatrix}
\begin{pmatrix}
l_{11}&l_{21}&l_{31}\\
     0&     ?&     ?\\
     0&     0&     ?
\end{pmatrix}
=
\begin{pmatrix}
    l_{11}^2&l_{11}l_{21}&l_{11}l_{31}\\
l_{21}l_{11}&           *&           *\\
l_{31}l_{11}&           *&           *
\end{pmatrix}
=
\begin{pmatrix}
 9& 3& 12\\
 3& 2&  9\\
12& 9&122
\end{pmatrix}
\]
Daraus kann man ablesen, dass $l_{11}=3$ sein muss, und weiter,
dass
$l_{21}=1$ und $l_{31}=4$. Damit ist die erste Spalte bestimmt.

Im zweiten Schritt versucht man, die zweite Spalte zu bestimmen.
Dazu schreibt man wieder
\[
LL^t
=
\begin{pmatrix}
3&     0&0\\
1&l_{22}&0\\
4&l_{32}&?
\end{pmatrix}
\begin{pmatrix}
3&     1&     4\\
0&l_{22}&l_{32}\\
0&     0&?
\end{pmatrix}
=
\begin{pmatrix}
 9&3           &          12\\
 3& 1+l_{22}^2  &4+l_{22}l_{32}\\
12& 4+l_{32}l_{22}&         *
\end{pmatrix}
=
\begin{pmatrix}
 9& 3& 12\\
 3& 2&  9\\
12& 9&122
\end{pmatrix}
\]
Daraus liest man ab $1+l_{22}^2=2$ und damit $l_{22}=1$, und weiter
$l_{32}=5$.

Im dritten Schritt ist jetzt nur noch das Element unten rechts zu bestimmen:
\[
LL^t=
\begin{pmatrix}
3&0&     0\\
1&1&     0\\
4&5&l_{33}
\end{pmatrix}
\begin{pmatrix}
3&1&     4\\
0&1&     5\\
0&0&l_{33}
\end{pmatrix}
=
\begin{pmatrix}
 9& 3& 12\\
 3& 2&  9\\
12& 9&16+25+l_{33}^2
\end{pmatrix}
=
\begin{pmatrix}
 9& 3& 12\\
 3& 2&  9\\
12& 9&122
\end{pmatrix}
\]
woraus man ablesen kann: $41+l_{33}^2=122$ oder $l_{33}^2=81$, $l_{33}=9$.
Die gesuchte Matrix $L$ ist also
\[
L=
\begin{pmatrix}
3&0&0\\
1&1&0\\
4&5&9
\end{pmatrix}.
\]
Kontrolle:
\[
\begin{pmatrix}
3&0&0\\
1&1&0\\
4&5&9
\end{pmatrix}
\begin{pmatrix}
3&1&4\\
0&1&5\\
0&0&9
\end{pmatrix}
=
\begin{pmatrix}
 9& 3& 12\\
 3& 2&  9\\
12& 9&122
\end{pmatrix}.
\qedhere
\]
\end{loesung}

\begin{diskussion}
Man beachte, dass die nicht verschwindenden Zahlen in der Matrix $L^t$
die ersten Ziffern der Dezimalentwicklung von $\pi\simeq3.14159$ sind.
\end{diskussion}

\begin{bewertung}
Cholesky-Zerlegung ({\bf C}) 1 Punkt,
Pro richtigem nicht verschwindendem Matrix-Element in $B$ ein Punkt,
maximal 6 Punkte.
\end{bewertung}

