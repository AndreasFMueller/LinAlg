Für welche Werte von $b$ ist die Matrix
\[
A=\begin{pmatrix}
 b^2 & b     & -b^2  \\
 b   & b^2+1 & 0     \\
-b^2 &     0 & 2b^2+1
\end{pmatrix}
\]
positiv definit?

\begin{loesung}
Die Matrix ist positiv definit, wenn sie sich als $A=L\transpose{L}$
schreiben lässt mit einer unteren Dreiecksmatrix $L$.
Gesucht ist also eine untere Dreiecksmatrix
\[
L=
\begin{pmatrix}
l_{11}&     0&     0\\
l_{21}&l_{22}&     0\\
l_{31}&l_{32}&l_{33}
\end{pmatrix}
\]
derart, dass $A=L\transpose{L}$, also
\[
L\transpose{L}
=
\begin{pmatrix}
l_{11}&     0&     0\\
l_{21}&l_{22}&     0\\
l_{31}&l_{32}&l_{33}
\end{pmatrix}
\begin{pmatrix}
l_{11}&l_{21}&l_{31}\\
     0&l_{22}&l_{32}\\
     0&     0&l_{33}
\end{pmatrix}
=
A.
\]
Daraus gewinnt man Schritt für Schritt
\begin{align*}
a_{11}
&=
% a_11
b^2
&&=
l_{11}^2
&&\Rightarrow&
l_{11}
&=
|b|
\\
a_{12}
&=
% a_12
b
&&=
l_{11}l_{21}
=
|b|l_{21}
&&\Rightarrow&
l_{21}
&=
b/|b|
\\
a_{13}
&=
% a_13
-b^2
&&=
l_{11}l_{31}
=|b| l_{31}
&&\Rightarrow&
l_{31}
&=
-b^2/|b|
\\
a_{22}
&=
% a_22
b^2+1
&&=
l_{21}^2 + l_{22}^2 = 1 + l_{22}^2
&&\Rightarrow&
l_{22}
&=
|b|
\\
a_{23}
&=
% a_23
0
&&=
l_{21}l_{31}+l_{22}l_{32}
=
-b
+|b|l_{32}
&&\Rightarrow& l_{32}
&=
 b/|b|
\\
a_{33}
&=
% a_33
2b^2+1
&&=
l_{31}^2+l_{32}^2+l_{33}^2
=
b^2+1+l_{33}^2
&&\Rightarrow& l_{33}
&=
|b|
.
\end{align*}
Damit ist eine Matrix
\[
L
=
\begin{pmatrix*}
|b|      &    0   &   0  \\
b/|b|    &   |b|  &   0  \\
-b^2/|b| &  b/|b| &  |b|
\end{pmatrix*}
\]
mit den genannten Eigenschaften gefunden.
Der Algorithmus funktioniert wegen der Division durch $|b|$ bei der
Gerechnung von $l_{21}$ genau dann, wenn $b\ne 0$.
Damit folgt, dass die Matrix genau dann positiv definit ist, wenn $b\ne 0$
ist.
\[
L
=
\begin{pmatrix*}
b  &  0  &  0  \\
1  &  b  &  0  \\
-b &  1  &  b
\end{pmatrix*}
\]
Für $b>0$ kann man die Matrix $L$ noch etwas einfacher als
schreiben.
\end{loesung}

\begin{bewertung}
Durchführung der Cholesky-Zerlegung ({\bf C}) 5 Punkte,
Schlussfolgerung ``positiv definit'' ({\bf P}) 1 Punkt.
\end{bewertung}

