%
% uebung.tex -- gemeinsame Makros fuer Uebungsblaetter
%
% (c) 2006 Prof. Dr. Andreas Mueller, HSR
% $Id: uebung.tex,v 1.3 2008/01/06 14:05:56 afm Exp $
%
\newcounter{uebungsaufgabe}
\newboolean{loesungen}
% environment fuer uebungsaufgaben
\newenvironment{uebungsaufgaben}{
\begin{list}{\arabic{uebungsaufgabe}.}
  {\usecounter{uebungsaufgabe}
  \setlength{\labelwidth}{2cm}
  \setlength{\leftmargin}{0pt}
  \setlength{\labelsep}{5mm}
  \setlength{\rightmargin}{0pt}
  \setlength{\itemindent}{0pt}
}}{\end{list}\vfill\pagebreak}
% Teilaufgaben
\newenvironment{teilaufgaben}{
\begin{enumerate}
\renewcommand{\labelenumi}{\alph{enumi})}
}{\end{enumerate}}
% Loesung
\NewEnviron{loesung}{%
\begin{proof}[Lösung]%
\renewcommand{\qedsymbol}{$\bigcirc$}
\BODY
\end{proof}}
\NewEnviron{diskussion}{
\BODY
}
\def\keineloesungen{%
\RenewEnviron{loesung}{\relax}
\RenewEnviron{diskussion}{\relax}
}
% Hinweis
\newenvironment{hinweis}{%
\renewcommand{\qedsymbol}{}
\begin{proof}[Hinweis]}{\end{proof}}
% Aufgabe aus der Sammlung wiedergeben
\newcounter{problemcounter}[chapter]
\def\aufgabepath{./}
\def\ainput#1{\input\aufgabepath/#1}
\def\verbatimainput#1{\expandafter\verbatiminput{\aufgabepath/#1}}
\def\aufgabetoplevel#1{%
\expandafter\def\expandafter\inputpath{#1}%
\let\aufgabepath=\inputpath
}
\def\includeagraphics[#1]#2{\expandafter\includegraphics[#1]{\aufgabepath#2}}
% \aufgabe
\newcommand{\aufgabe}[2]{%
\refstepcounter{problemcounter}%
\label{#2}
\bigskip{\parindent0pt\strut}\hbox{\bf\theproblemcounter. }%
\marginpar{\raggedright\tiny #2}%
\expandafter\def\csname currentaufgabe\endcsname{#2}%
\expandafter\def\csname aufgabepath\endcsname{\inputpath/#1/#2/}%
\expandafter\input{\inputpath#1/#2.tex}
\bigskip
}
\renewcommand\theproblemcounter{\thechapter.\arabic{problemcounter}}
% Bewertung
\NewEnviron{bewertung}{\relax}
% oft benutzte Macros
\def\blank{\text{\textvisiblespace}}
%
% macros fuer den thema-Index
%
\newcommand{\openthemaindex}{%
\newwrite\themaindex
\immediate\openout\themaindex=thema.tix}
%
\newcommand{\closethemaindex}{\immediate\closeout\themaindex}
%
\def\thema#1{\immediate\write\themaindex%
{{\thechapter}{#1}{\arabic{problemcounter}}{\thechapter.\arabic{problemcounter}}{\currentaufgabe}}}
\def\themasection#1#2{ \item[#1] #2 }
\newcommand{\printthemata}{
\IfFileExists{./thema.tex}{
\chapter*{Aufgaben nach Themen}
\begin{description}
\input{thema.tex}
\end{description}
}{}
}
