Ein Student muss für seine Bachelorarbeit Beschleunigungen 
eines 2-Achsen Roboters messen und entwickelt dafür ein 
PCB, auf welchem er zwei 1-Achsen-Beschleunigungssensoren
orthogonal zueinander anbringt. Jeder der Sensoren sollte 
ihm den Anteil der Beschleunigung in einer Richtung
($x$- bzw. $y$-Achse) ermitteln, womit er in der Lage ist
Beschleunigungen in der Ebene messen. 
Aufgrund von Fertigungstoleranzen wurden die Sensoren
leider nicht exakt auf dem PCB platziert, wodurch Messfehler
entstehen. Wir können jedoch annehmen, dass die wahren 
Beschleunigungswerte $a_x$ und $a_y$ linear von den gemessenen
Beschleunigungswerten $m_x$ und $m_y$ abhängen.
Um diesen Zusammenhang zu ermitteln, wurden folgende Paare gemessen:
\begin{center}
\begin{tabular}{|>{$}c<{$}>{$}c<{$}|>{$}c<{$}>{$}c<{$}|}
\hline
a_x & a_y & m_x & m_y \\
\hline
9.81 & 0 & 9.83& 1.02\\
0 & 9.81 & -1.11 & 10.21\\
6.94 & -6.94& 6.98 & -6.20\\
\hline
\end{tabular}
\end{center}
Gesucht ist jetzt eine $2 \times 2$-Matrix $C$, die den Zusammenhang
zwischen den wahren Werten $(a_x,a_y)$ und den gemessenen Werten $(m_x,m_y)$
möglichst gut wiedergeben kann. 

\begin{teilaufgaben}
 \item
  Stellen Sie ein Gleichungssystem auf, mit dem sich die bestmöglichen Werte für 
  die Matrixelemente von $C$ bestimmen lassen. 
  Im Gleichungssystem müssen alle Messwerte berücksichtigt werden.
  Zudem soll es erweitert werden können, wenn mehr Daten bekannt werden.
 \item Bestimmen Sie die Matrix $C$.
\end{teilaufgaben}

\begin{hinweis}
Verwenden Sie den Taschenrechner, um das in a) gefundene Gleichungssystem
zu lösen.
\end{hinweis}

\thema{Least Squares}

\begin{loesung}
\begin{teilaufgaben}
\item
Wir suchen eine Matrix, die Beschleunigungs-Vektoren gemäss
\begin{equation}
\begin{pmatrix}a_x\\a_y\end{pmatrix}
=
\underbrace{
\begin{pmatrix}c_{11} & c_{12}\\ c_{21} & c_{22}\end{pmatrix}}_{\displaystyle C}
\begin{pmatrix}m_x\\m_y\end{pmatrix}
\label{40000063:basis1}
\end{equation}
umrechnet.
Die Einträge in $C$ sind die Unbekannten.
Im Idealfall lassen sich die Messwerte exakt wiedergeben, dann
gelten die Gleichungen:
\begin{equation}
\begin{linsys}{4}
{\color{red}c_{11}}m_x&+&{\color{red}c_{12}}m_y
	& &       & &
	&=&a_x\\
       & &       & &       
{\color{red}c_{21}}m_x&+&{\color{red}c_{22}}m_y
	&=&a_y\\
\end{linsys}
\label{40000063:gl}
\end{equation}
Setzt man die drei Messungen ein, erhält man daraus sechs Gleichungen
für die vier gesuchten Matrixeinträge von $C$, also ein überbestimmtes
Gleichungssystem.
Die Matrix $A$ und die rechte Seite $b$ dieses Gleichungssystems sind
\[
A=\begin{pmatrix}
\phantom{-}9.83&\phantom{-}1.02& 0&0\\
0&0& \phantom{-}9.83& \phantom{-}1.02\\
%
-1.11 & \phantom{-}10.21& 0&0\\
0&0& -1.11 & \phantom{-}10.21\\
%
\phantom{-}6.98 & -6.20& 0&0\\
0&0& \phantom{-}6.98 & -6.20\\
%
\end{pmatrix}
,
\qquad
b
=
\begin{pmatrix}
\phantom{-}9.81\\0\\
0\\\phantom{-}9.81\\
\phantom{-}6.94\\ -6.94
\end{pmatrix}
\]
Das gesuchte Gleichungssystem hat die Matrix $A^tA$ und die rechte
Seite $A^tb$:
\[
A^tA
=
\begin{pmatrix}
  146.58 & -44.58  &       0  &       0\\
  -44.58 & 143.72   &      0   &      0\\
         0   &      0  & 146.58 & -44.58\\
         0   &      0  & -44.58 & 143.72\\
\end{pmatrix}
,\qquad
A^tb=\begin{pmatrix}
    144.87\\
  -33.02\\
  -59.33\\
  143.19
  \end{pmatrix}
\]
\item
Um die Matrixelemente von $C$ zu bestimmen muss das Gleichungssystem
$A^tA x = A^tb $ nun gelöst werden. Mann findet
\[
  x = (A^tA)^{-1}A^tb = 
  \begin{pmatrix}
    \phantom{-}1.0141\\
    \phantom{-}0.0848\\
   -0.1123\\
    \phantom{-}0.9614\\ 
  \end{pmatrix},
\]
Woraus sich die folgende Matrix 
\[
  C =  \begin{pmatrix}
    \phantom{-}1.0141 & 0.0848\\
   -0.1123 & 0.9614 
  \end{pmatrix}
\]
ergibt.
\end{teilaufgaben}
\end{loesung}


\begin{bewertung}
Gleichung \eqref{40000063:basis1} für ein Paar von Messpaaren ({\bf M}) 1 Punkt,
Least-Squares Gleichungen \eqref{40000063:gl} ({\bf G}) 1 Punkt,
Matrix $A$ ({\bf A}) 1 Punkt,
Matrix $A^tA$ ({\bf L}) 1 Punkt,
Vektor $A^tb$ ({\bf B}) 1 Punkt,
Matrix $C$ ({\bf C}) 1 Punkt.
\end{bewertung}


