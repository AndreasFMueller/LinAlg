Die Matrix $A$ soll einen Vektor auf die Ebene durch den
Nullpunkt mit der Normalen $\vec n$ projizieren.
\begin{teilaufgaben}
\item
Finden Sie die Abbildungsmatrix für
\[
\vec n=\begin{pmatrix}1\\1\\1\end{pmatrix}.
\]
\item
Wenn man eine Projektion zweimal durchführt, ändert sich nichts mehr,
zweimal projizieren ergibt dasselbe wie nur einmal projizieren.
Kontrollieren Sie, ob $P^2=P$ ist.
\item
Ist diese Matrix orthogonal?
\item
Berechnen Sie die Deteminanten von $P$.
\end{teilaufgaben}

\thema{Abbildungsmatrix}
\thema{Projektion}

\begin{loesung}
\begin{teilaufgaben}
\item
Die Projektion eines Vektors $\vec v$ auf die Ebene entsteht dadurch, dass man
den Vektor $\vec v$ um die Projektion auf die Richtung $\vec n$ verringert.
Es ist also die Matrix der Abbildung
\[
\vec v\mapsto \vec v_{\|} = \vec v - \vec v_{\perp}
\]
zu finden.
Die Projektion $\vec v_{\perp}$ haben wir in der Vorlesung bereits gefunden, es ist
\[
\vec v_{\perp}
=
\vec n^0
(\vec n^0\cdot \vec v)
=
(\vec n^0\transpose{(\vec n^0)})\vec v
\]
Der normierte Normalenvektor ist
\[
\vec n^0 =\frac1{\sqrt{3}}\begin{pmatrix}1\\1\\1\end{pmatrix}
\]
Damit wird die Matrix, die $\vec v_{\perp}$ berechnet:
\[
\vec v_{\perp}
=
\frac13\begin{pmatrix}
1&1&1\\
1&1&1\\
1&1&1
\end{pmatrix}
\vec v.
\]
Daraus ergibt sich für die Projektion auf die Ebene die Matrix
\[
P\vec v
=
\vec v_{\|}
=
\vec v-\vec v_{\perp}
=
\left(I-
\frac13\begin{pmatrix}
1&1&1\\
1&1&1\\
1&1&1
\end{pmatrix}\right)
\vec v
=
\frac13
\begin{pmatrix}
 2&-1&-1\\
-1& 2&-1\\
-1&-1& 2
\end{pmatrix}
\vec v.
\]
Die gesuchte Matrix ist also
\[
P=
\frac13
\begin{pmatrix}
 2&-1&-1\\
-1& 2&-1\\
-1&-1& 2
\end{pmatrix}.
\]
\item
Es ist $PP$ zu berechnen.
\begin{align*}
PP
&=
\frac19
\begin{pmatrix}
 6&-3&-3\\
-3& 6&-3\\
-3&-3& 6
\end{pmatrix}
=
\frac13
\begin{pmatrix}
 2&-1&-1\\
-1& 2&-1\\
-1&-1& 2
\end{pmatrix}
=P.
\end{align*}
\item
Da $\transpose{P}=P$ ist $\transpose{P}P=PP=P\ne I$, die Matrix $P$ ist daher nicht
orthogonal.
\item
Da die Spalten von $P$ alle in der Ebene senkrecht auf $\vec n$ liegen,
sind sie linear abhängig. 
Eine Matrix mir linear abhängigen Spalten hat Determinante $0$.
\end{teilaufgaben}
\end{loesung}

