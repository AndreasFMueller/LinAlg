Die Vektoren
\begin{align*}
b_1&=\frac1{\sqrt{8}} \begin{pmatrix}1&1&1&1&1&1&1&1\end{pmatrix}^t&
b_2&=\frac1{\sqrt{8}} \begin{pmatrix}1&1&1&1&-1&-1&-1&-1\end{pmatrix}^t\\
b_3&=\frac1{\sqrt{4}} \begin{pmatrix}1&1&-1&-1&0&0&0&0\end{pmatrix}^t&
b_4&=\frac1{\sqrt{4}} \begin{pmatrix} 0&0&0&0&1&1&-1&-1\end{pmatrix}^t\\
b_5&=\frac1{\sqrt{2}} \begin{pmatrix}1&-1&0&0&0&0&0&0\end{pmatrix}^t&
b_6&=\frac1{\sqrt{2}} \begin{pmatrix}0&0&1&-1&0&0&0&0\end{pmatrix}^t\\
b_7&=\frac1{\sqrt{2}} \begin{pmatrix}0&0&0&0&1&-1&0&0\end{pmatrix}^t&
b_8&=\frac1{\sqrt{2}} \begin{pmatrix}0&0&0&0&0&0&1&-1\end{pmatrix}^t
\end{align*}
bilden eine orthonormierte Basis eines achtdimensionalen Vektorraumes.
Stellen Sie den Vektor $v$ mit den Komponenten
\[
v_k=(k-4.5)^2,\quad{1\le k\le 8}
\]
als Linearkombination der Vektoren $b_i$ dar.

\thema{Basis}
\thema{Basiszerlegung}

\begin{loesung}
Da die Basisvektoren $b_i$ orthonormiert sind, können die Koeffizienten
in der Entwicklung mit Hilfe des Skalarproduktes ermittelt werden. Der
Vektor $v$ ist
\[
v=\frac14\begin{pmatrix}
49\\
25\\
9\\
1\\
1\\
9\\
25\\
49
\end{pmatrix}
=\frac14v_0
\]
Die Skalarprodukte sind
\begin{align*}
b_1\cdot v_0&=\frac1{\sqrt{8}}2\cdot(1+9+25+49)=\frac{84}{\sqrt{2}}=59.397\\
b_2\cdot v_0&=\frac1{\sqrt{8}}(1+9+25+49)-\frac1{\sqrt{8}}(1+9+25+49)=0\\
b_3\cdot v_0&=\frac12(49+25-9-1)=32\\
b_4\cdot v_0&=-32\\
b_5\cdot v_0&=\frac1{\sqrt{2}}(49-25)=\frac{24}{\sqrt{2}}\\
b_6\cdot v_0&=\frac1{\sqrt{2}}(9-1)=\frac{8}{\sqrt{2}}\\
b_7\cdot v_0&=\frac1{\sqrt{2}}(1-9)=-\frac{8}{\sqrt{2}}\\
b_8\cdot v_0&=\frac1{\sqrt{2}}(25-49)=-\frac{24}{\sqrt{2}}\\
\end{align*}
also gilt
\[
v=
\frac{21}{\sqrt{2}}b_1
+
4b_3
-4b_4
+\frac{6}{\sqrt{2}}b_5
+\frac{2}{\sqrt{2}}b_6
-\frac{2}{\sqrt{2}}b_7
-\frac{6}{\sqrt{2}}b_8.
\]
Numerisch sind die Koeffizienten der $b_i$:
\begin{center}
\begin{tabular}{|c|r|}
\hline
$i$&$b_i\cdot v$\\
\hline
1&$14.849$\\
2&$ 0.000$\\
3&$ 8.000$\\
4&$-8.000$\\
5&$ 4.243$\\
6&$ 1.414$\\
7&$-1.414$\\
8&$-4.243$\\
\hline
\end{tabular}
\end{center}
\end{loesung}

