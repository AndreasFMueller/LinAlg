Statt der Standardbasis $B$ möchte man in einer Anwendung die Basis
$B'$ aus den Basisvektoren
\[
\begin{aligned}
b_1'
&=
\begin{pmatrix}1\\0\end{pmatrix}
&
&\text{und}
&
b_2'
&=
\begin{pmatrix}\frac12\\\frac{\sqrt{3}}2\end{pmatrix}
\end{aligned}
\]
verwenden.
Dies entspricht einem Koordinatensystem mit Achsen, die einen Winkel
von $60^\circ$ ein\-schliessen.
\begin{teilaufgaben}
\item
Berechnen Sie die Transformationsmatrix $T$, die $B$-Koordinaten
in $B'$-Koordinaten umrechnet.
\item
Berechnen Sie die $B'$-Koordinaten der beiden Vektoren
\[
\begin{aligned}
u&=\begin{pmatrix}1\\1\end{pmatrix}
&&\text{und}&
v&=\begin{pmatrix}1\\-1\end{pmatrix}
\end{aligned}
\]
Die Vektoren stehen senkrecht aufeinander.
\item
Das Skalarprodukt hat in $B'$-Koordinaten nicht mehr die einfache
Formel $u^{\prime t}v'$.
Aus der Matrix $T$ lässt sich aber auch eine Formel für die
Koordinatenumrechnung von $B'$-Koordinaten in $B$-Koordinaten finden,
die dazu verwendet werden kann, eine Formel für das Skalarprodukt in
$B'$-Koordinaten aufzustellen.
Dabei stellt sich heraus, dass das Skalarprodukt
als $u'^tGv'$ geschrieben werden kann.
Berechnen Sie die Matrix $G$.
\item 
Berechnen Sie das Skalarprodukt von $u'$ und $v'$ in den $B'$-Koordinaten.
\end{teilaufgaben}

\thema{Basis}
\thema{Basistransformation}

\begin{loesung}
\begin{teilaufgaben}
\item 
Die Transformationsmatrix kann wie üblich mit dem Gauss-Algorithmus
ermittelt werden.
\begin{align*}
\begin{tabular}{|>{$}c<{$}>{$}c<{$}|>{$}c<{$}>{$}c<{$}|}
\hline
1&\frac12         &1&0\\
0&\frac{\sqrt{3}}2&0&1\\
\hline
\end{tabular}
&
\rightarrow
\begin{tabular}{|>{$}c<{$}>{$}c<{$}|>{$}c<{$}>{$}c<{$}|}
\hline
1&0&1&-\frac{1}{\sqrt{3}}\\
0&1&0& \frac{2}{\sqrt{3}}\\
\hline
\end{tabular}
\end{align*}
Daher ist die Koordinatentransformationsmatrix
\[
T
=
\begin{pmatrix}1&-\frac1{\sqrt{3}}\\0&\frac2{\sqrt{3}}\end{pmatrix}.
\]
\item
Anwendung der Transformationsmatrix auf die gegebenen Vektoren $u$
und $v$ liefert
\[
\begin{aligned}
u'
=
Tu
&=
\begin{pmatrix}1&-\frac1{\sqrt{3}}\\0&\frac2{\sqrt{3}}\end{pmatrix}
\begin{pmatrix}1\\1\end{pmatrix}
=
\begin{pmatrix} 1-\frac1{\sqrt{3}}\\\frac2{\sqrt{3}} \end{pmatrix}
&
&\text{und}&
v'
=
Tv
&=
\begin{pmatrix}1&-\frac1{\sqrt{3}}\\0&\frac2{\sqrt{3}}\end{pmatrix}
\begin{pmatrix}1\\-1\end{pmatrix}
=
\begin{pmatrix} 1+\frac1{\sqrt{3}}\\-\frac2{\sqrt{3}} \end{pmatrix}.
\end{aligned}
\]
\item
Die Matrix $T$ rechnet von $B$- auf $B'$-Koordinaten um, die Matrix $T^{-1}$
liefert die Umrechnung in umgekehrter Richtung.
Für das Skalarprodukt gilt
\[
u^t v
=
(T^{-1}u')^t(T^{-1}v')
=
u'^t
\underbrace{ (T^{-1})^t T^{-1}}_{\displaystyle =G} v'
\]
Wenden wir dies auf die oben berechnete Matrix $T$ an, erhalten wir:
\[
G
=
(T^{-1})^tT^{-1}
=
\begin{pmatrix}1&\frac12\\0&\frac{\sqrt{3}}2\end{pmatrix}
\begin{pmatrix}1&0\\\frac12&\frac{\sqrt{3}}2\end{pmatrix}
=
\begin{pmatrix} 1&\frac12\\\frac12&1 \end{pmatrix}.
\]
\item
Unter Verwendung der Matrix $G$ lässt sich das Skalarprodukt jetzt auch
in $B'$-Koordinaten berechnen:
\begin{align*}
u^tv
&=
u'^tGv'
=
\begin{pmatrix} 1-\frac1{\sqrt{3}}&\frac2{\sqrt{3}} \end{pmatrix}
\begin{pmatrix} 1&\frac12\\\frac12&1 \end{pmatrix}
\begin{pmatrix} 1+\frac1{\sqrt{3}}\\-\frac2{\sqrt{3}} \end{pmatrix}
=
\begin{pmatrix} 1-\frac1{\sqrt{3}}&\frac2{\sqrt{3}} \end{pmatrix}
\begin{pmatrix} 1\\
\frac12+\frac1{2\sqrt{3}}-\frac2{\sqrt{3}}
\end{pmatrix}
\\
&=
1-\frac1{\sqrt{3}}
+
\frac1{\sqrt{3}}+\frac13-\frac43
=
0.
\end{align*}
Wie vorhin schon erkannt, stehen, die Vektoren senkrecht aufeinander.
\qedhere
\end{teilaufgaben}
\end{loesung}

