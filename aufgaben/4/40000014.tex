Statt der Standardbasis $B$ möchte man in einer Anwendung die Basis
$C$ aus den Basisvektoren
\[
\begin{aligned}
\vec c_1
&=
\begin{pmatrix}1\\0\end{pmatrix}
&
&\text{und}
&
\vec c_2
&=
\begin{pmatrix}\frac12\\\frac{\sqrt{3}}2\end{pmatrix}
\end{aligned}
\]
verwenden.
Dies entspricht einem Koordinatensystem mit Achsen, die einen Winkel
von $60^\circ$ ein\-schliessen.
\begin{teilaufgaben}
\item
Berechnen Sie die Transformationsmatrix $T$, die $B$-Koordinaten
in $C$-Koordinaten umrechnet.
\item
Berechnen Sie die $C$-Koordinaten ($\vec u_C$ und $\vec v_C$) der beiden Vektoren
\[
\begin{aligned}
\vec u_B&=\begin{pmatrix}1\\1\end{pmatrix}
&&\text{und}&
\vec v_B&=\begin{pmatrix}1\\-1\end{pmatrix}
\end{aligned}
\]
Die Vektoren $\vec u_B$ und $\vec v_B$ stehen senkrecht aufeinander.
\item
Das Skalarprodukt hat in $C$-Koordinaten nicht mehr die einfache
Formel $\transpose{\vec{u}_C} \vec v_C$.
Aus der Matrix $T$ lässt sich aber auch eine Formel für die
Koordinatenumrechnung von $C$-Koordinaten in $B$-Koordinaten finden,
die dazu verwendet werden kann, eine Formel für das Skalarprodukt in
$C$-Koordinaten aufzustellen.
Dabei stellt sich heraus, dass das Skalarprodukt
als $\transpose{\vec{u}_C}G\vec v_C$ geschrieben werden kann.
Berechnen Sie die Matrix $G$.
\item 
Berechnen Sie das Skalarprodukt von $\vec u_C$ und $\vec v_C$ in den $C$-Koordinaten.
\end{teilaufgaben}

\thema{Basis}
\thema{Basistransformation}

\begin{loesung}
\begin{teilaufgaben}
\item 
Die Transformationsmatrix kann mit dem Gauss-Algorithmus
ermittelt werden.
\begin{align*}
\begin{tabular}{|>{$}c<{$}>{$}c<{$}|>{$}c<{$}>{$}c<{$}|}
\hline
1&\frac12         &1&0\\
0&\frac{\sqrt{3}}2&0&1\\
\hline
\end{tabular}
&
\rightarrow
\begin{tabular}{|>{$}c<{$}>{$}c<{$}|>{$}c<{$}>{$}c<{$}|}
\hline
1&0&1&-\frac{1}{\sqrt{3}}\\
0&1&0& \frac{2}{\sqrt{3}}\\
\hline
\end{tabular}
\end{align*}
Daher ist die Koordinaten-Transformationsmatrix
\[
T
=
\begin{pmatrix}1&-\frac1{\sqrt{3}}\\0&\frac2{\sqrt{3}}\end{pmatrix}.
\]
Alternativ könnte die Transformationsmatrix $T$ auch als Inverse der Matrix $C$ mit Hilfe der Minoren wie folgt gefunden werden:
\[
  T = C^{-1}B = C^{-1}I = C^{-1} 
  = \begin{pmatrix}1 & \frac12\\0 & \frac{\sqrt{3}}2\end{pmatrix}^{-1} 
  = \dfrac{2}{\sqrt{3}}\begin{pmatrix}\frac{\sqrt{3}}2 & -\frac12\\0 & 1\end{pmatrix} 
  = \begin{pmatrix}1&-\frac1{\sqrt{3}}\\0&\frac2{\sqrt{3}}\end{pmatrix}
\]
\item
Anwendung der Transformationsmatrix auf die gegebenen Vektoren $\vec u_B$
und $\vec v_B$ liefert
\[
\begin{aligned}
\vec u_C
=
T\vec u_B
&=
\begin{pmatrix}1&-\frac1{\sqrt{3}}\\0&\frac2{\sqrt{3}}\end{pmatrix}
\begin{pmatrix}1\\1\end{pmatrix}
=
\begin{pmatrix} 1-\frac1{\sqrt{3}}\\\frac2{\sqrt{3}} \end{pmatrix}
&
&\text{und}&
\vec v_C
=
T\vec v_B
&=
\begin{pmatrix}1&-\frac1{\sqrt{3}}\\0&\frac2{\sqrt{3}}\end{pmatrix}
\begin{pmatrix}1\\-1\end{pmatrix}
=
\begin{pmatrix} 1+\frac1{\sqrt{3}}\\-\frac2{\sqrt{3}} \end{pmatrix}.
\end{aligned}
\]
\item
Die Matrix $T$ rechnet von $B$- auf $C$-Koordinaten um, die Matrix $T^{-1}$
liefert die Umrechnung in umgekehrter Richtung, also von $C$- auf $B$-Koordinaten.
Für das Skalarprodukt gilt
\[
\transpose{\vec{u}_B} \vec v_B
=
\transpose{(T^{-1}\vec{u}_C)}(T^{-1}\vec v_C)
=
\transpose{\vec{u}_C}
\underbrace{ \transpose{(T^{-1})} T^{-1}}_{\displaystyle =G} \vec v_C.
\]
Wenden wir dies auf die oben berechnete Matrix $T$ an, erhalten wir:
\[
G
=
\transpose{(T^{-1})}T^{-1}
=
\begin{pmatrix}1&0\\\frac12&\frac{\sqrt{3}}2\end{pmatrix}
\begin{pmatrix}1&\frac12\\0&\frac{\sqrt{3}}2\end{pmatrix}
=
\begin{pmatrix} 1&\frac12\\\frac12&1 \end{pmatrix}.
\]
\item
Unter Verwendung der Matrix $G$ lässt sich das Skalarprodukt jetzt auch
in $C$-Koordinaten berechnen:
\begin{align*}
\transpose{\vec{u}_B} \vec v_B
&=
\transpose{\vec{u}_C} G \vec v_C
=
\begin{pmatrix} 1-\frac1{\sqrt{3}}&\frac2{\sqrt{3}} \end{pmatrix}
\begin{pmatrix} 1&\frac12\\\frac12&1 \end{pmatrix}
\begin{pmatrix} 1+\frac1{\sqrt{3}}\\-\frac2{\sqrt{3}} \end{pmatrix}
=
\begin{pmatrix} 1-\frac1{\sqrt{3}}&\frac2{\sqrt{3}} \end{pmatrix}
\begin{pmatrix} 1\\
\frac12+\frac1{2\sqrt{3}}-\frac2{\sqrt{3}}
\end{pmatrix}
\\
&=
1-\frac1{\sqrt{3}}
+
\frac1{\sqrt{3}}+\frac13-\frac43
=
0.
\end{align*}
Wie vorhin schon erkannt, stehen die Vektoren senkrecht aufeinander.
\qedhere
\end{teilaufgaben}
\end{loesung}

