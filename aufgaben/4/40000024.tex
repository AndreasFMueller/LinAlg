Die Punkte
\begin{align*}
A&=(0,0,1),
&
B&=(1,0,0),
&
C&=(0,1,0),
\\
D&=(-1,0,0),
&
E&=(0,-1,0),
&
F&=(0,0,-1)
\end{align*}
sind Ecken des Oktaeders
\begin{center}
\begin{tikzpicture}[scale=1]
\node at (0,0) {\includeagraphics[width=6cm]{oktaeder.jpg}};
\node at (3,-1.3) {$x$};
\node at (1.5,0.4) {$y$};
\node at (-0.1,2.9) {$z$};
\node at (-0.6,2.0) {$A$};
\node at (2.0, -0.8) {$B$};
\node at (0.9,0.6) {$C$};
\node at (-1.9,0.4) {$D$};
\node at (-2.2,-1.2) {$E$};
\node at (-0.0,-2.5) {$F$};
\end{tikzpicture}
\end{center}

\begin{teilaufgaben}
\item Finden Sie eine Drehmatrix $R$, welche den Punkt $A$ auf den Punkt $D$
abbildet und den Schwerpunkt der Seitenfläche $BEF$ nicht bewegt.
\item Auf welchen Punkt wird $F$ abgebildet?
\item Finden Sie den Drehwinkel dieser Drehung.
\end{teilaufgaben}

\thema{Abbildungsmatrix}
\thema{Drehmatrix}
\thema{Drehwinkel}

\begin{loesung}
\begin{teilaufgaben}
\item
Der Schwerpunkt $S$ des Dreiecks $BEF$ hat die Koordinaten
$S=(\frac13,-\frac13,-\frac13)$.
Da mit $S$ auch die Gerade durch $O$ und $S$ von der Drehung nicht bewegt
wird, können wir den Vektor $\transpose{\begin{pmatrix}1&-1&-1\end{pmatrix}}$ als
Achsvektor verwenden.
Um die Drehmatrix $R$ zu bestimmen, haben wir also bis jetzt:
\[
R
\begin{pmatrix}0\\0\\1\end{pmatrix}
=
\begin{pmatrix}-1\\0\\0\end{pmatrix}
\qquad\text{und}\qquad
R
\begin{pmatrix}1\\-1\\-1\end{pmatrix}
=
\begin{pmatrix}1\\-1\\-1\end{pmatrix}.
\qquad\Rightarrow\qquad
R
\begin{pmatrix}
0& 1\\
0&-1\\
1&-1
\end{pmatrix}
=
\begin{pmatrix}
-1& 1\\
 0&-1\\
 0&-1
\end{pmatrix}.
\]
Um dies nach $R$ auflösen zu können, brauchen wir aber noch die Abbildung
eines weiteren, linear unabhängigen Vektors.
Da das Dreieck $BEF$ auf sich selbst abgebildet wird, muss der Punkt
$E$ auf den Punkt $F$ abgebildet werden.
Dies liefert die zusätzliche Bedingung
\[
R
\begin{pmatrix}0\\-1\\0\end{pmatrix}
=
\begin{pmatrix}0\\0\\-1\end{pmatrix}.
\]
Wir können $R$ daher als Lösung
\[
R
\underbrace{
\begin{pmatrix}
0& 1& 0\\
0&-1&-1\\
1&-1& 0
\end{pmatrix}
}_{\displaystyle B_1}
=
\underbrace{
\begin{pmatrix}
-1& 1& 0\\
 0&-1& 0\\
 0&-1&-1
\end{pmatrix}
}_{\displaystyle B_2}
\qquad\Rightarrow\qquad
R=B_2B_1^{-1}
\]
finden.
Die inverse Matrix von $B_1$ kann mit dem Gauss-Algorithmus wie folgt
gefunden wereen:
\begin{align*}
\begin{tabular}{|>{$}c<{$}>{$}c<{$}>{$}c<{$}|>{$}c<{$}>{$}c<{$}>{$}c<{$}|}
\hline
 0& 1& 0& 1& 0& 0\\
 0&-1&-1& 0& 1& 0\\
 1&-1& 0& 0& 0& 1\\
\hline
\end{tabular}
&\rightarrow
\begin{tabular}{|>{$}c<{$}>{$}c<{$}>{$}c<{$}|>{$}c<{$}>{$}c<{$}>{$}c<{$}|}
\hline
 1&-1& 0& 0& 0& 1\\
 0& 1& 0& 1& 0& 0\\
 0&-1&-1& 0& 1& 0\\
\hline
\end{tabular}
\\
&\rightarrow
\begin{tabular}{|>{$}c<{$}>{$}c<{$}>{$}c<{$}|>{$}c<{$}>{$}c<{$}>{$}c<{$}|}
\hline
 1&-1& 0& 0& 0& 1\\
 0& 1& 0& 1& 0& 0\\
 0& 0&-1& 1& 1& 0\\
\hline
\end{tabular}
\\
&\rightarrow
\begin{tabular}{|>{$}c<{$}>{$}c<{$}>{$}c<{$}|>{$}c<{$}>{$}c<{$}>{$}c<{$}|}
\hline
 1&-1& 0& 0& 0& 1\\
 0& 1& 0& 1& 0& 0\\
 0& 0& 1&-1&-1& 0\\
\hline
\end{tabular}
\\
&\rightarrow
\begin{tabular}{|>{$}c<{$}>{$}c<{$}>{$}c<{$}|>{$}c<{$}>{$}c<{$}>{$}c<{$}|}
\hline
 1& 0& 0& 1& 0& 1\\
 0& 1& 0& 1& 0& 0\\
 0& 0& 1&-1&-1& 0\\
\hline
\end{tabular}
\end{align*}
Daraus können wir $B_1^{-1}$ ablesen:
\[
B_1^{-1}
=
\begin{pmatrix}
 1& 0& 1\\
 1& 0& 0\\
-1&-1& 0
\end{pmatrix}.
\]
Die Drehmatrix $R$ ist daher
\[
R
=
B_2B_1^{-1}
=
\begin{pmatrix}
-1& 1& 0\\
 0&-1& 0\\
 0&-1&-1
\end{pmatrix}
\begin{pmatrix}
 1& 0& 1\\
 1& 0& 0\\
-1&-1& 0
\end{pmatrix}
=
\begin{pmatrix}
 0& 0&-1\\
-1& 0& 0\\
 0& 1& 0
\end{pmatrix}.
\]
\item
Die Matrix $R$ kann jetzt dazu verwendet werden, das Bild von $F$ zu bestimmen:
\[
R\vec f
=
\begin{pmatrix}
 0& 0&-1\\
-1& 0& 0\\
 0& 1& 0
\end{pmatrix}
\begin{pmatrix}0\\0\\-1\end{pmatrix}
=
\begin{pmatrix}1\\0\\0\end{pmatrix}
=
\vec{b}.
\]
Der Punkt $F$ wird also auf $B$ abgebildet.
\item
Die Drehwinkelformel angewendet auf die Matrix $R$ liefert
\begin{align*}
\cos\alpha
&=
\frac{\operatorname{Spur}R-1}{2}
=
-\frac12
\\
\Rightarrow\qquad
\alpha
&=
120^\circ.
\qedhere
\end{align*}
\end{teilaufgaben}
\end{loesung}

\begin{diskussion}
Da die Eckpunkte $B$, $C$ und $A$ des Oktaeders als Ortsvektoren die
Standardbasisvektoren haben und wir deren Abbildung auch direkt ermitteln
können, können wir die Drehmatrix auch direkt mit den Bildern der
genannten Punkte füllen.
Die Abbildung der genannten Punkte ist
\[
B\rightarrow E,\qquad
C\rightarrow A,\qquad
A\rightarrow D.
\]
Daraus lesen wir ab
\[
R
=
\begin{pmatrix}
 0& 0&-1\\
-1& 0& 0\\
 0& 1& 0
\end{pmatrix},
\]
in Übereinstimmung mit obiger Lösung.
\end{diskussion}

\begin{bewertung}
\begin{teilaufgaben}
\item
Berechnung der Drehmatrix ({\bf R}) 4 Punkte,
\item
Ermittlung des Bildpunktes ({\bf B}) 1 Punkt,
\item
Berechnung des Drehwinkels ({\bf D}) 1 Punkt.
\end{teilaufgaben}
\end{bewertung}


