Für eine Forschungsarbeit müssen Beschleunigungen sehr exakt 
gemessen werden können. Zu diesem Zweck wird ein rauscharmer
Beschleunigungssensor verwendet, welcher zuvor kalibriert
wird. Als Referenz wird dabei die Erdbeschleunigung verwendet.
Leider reicht es für diese Anwendung aber nicht aus, den 
Standardwert von $9.81\,\frac{\text{m}}{s^2}$ als Referenz 
zu verwenden, weshalb vorab die örtliche Erdbeschleunigung 
ermittelt wird. 

Für die Bestimmung der örtlichen Erdbeschleunigung wird 
eine kleine Stahlkugel im Vakuum fallen gelassen und deren Fall 
mit einer High-Speed-Kamera aufgezeichnet. Die Auswertung
der aufgenommenen Bilder ergab folgende Datenpunkte
\begin{center}
\begin{tabular}{>{$}c<{$}|>{$}r<{$}|>{$}c<{$}}
i& \text{Zeit } t_i\,[\text{s}]& \text{Höhe }h_i\,[\text{m}]\\
\hline
1& 0.00 &1.4009\\
2& 0.05 &1.2176\\
3& 0.10 &1.0097\\
4& 0.15 &0.7774\\
5& 0.20 &0.5209
\end{tabular}
\end{center}
wobei der Zeitpunkt, zu dem die Kugel erstmals auf einem Bild
auftaucht als $t=0$ angenommen wurde.
Zudem ist aus der Physik bekannt, dass die Höhe eines fallenden
Objekts mit dem quadratischen Polynom
\[
  h = -\frac{1}{2}g t^2 + v_0 t + h_0
\]
beschrieben werden kann.


\begin{teilaufgaben}
 \item
  Stellen Sie ein Gleichungssystem auf, mit dem sich die bestmöglichen Werte für 
  $g$, $v_0$, und $h_0$ bestimmen lassen. 
  Im Gleichungssystem müssen alle Messwerte berücksichtigt werden.
  Zudem soll es erweitert werden können, wenn mehr Daten bekannt werden.
 \item Bestimmen Sie die örtliche Erdbeschleunigung $g$.
\end{teilaufgaben}

\begin{hinweis}
Verwenden Sie den Taschenrechner, um das in a) gefundene Gleichungssystem
zu lösen.
\end{hinweis}

\thema{Least Squares}

\begin{loesung}
\begin{teilaufgaben}
\item
Wir möchten gerne $g$, $v_0$ und $h_0$ bestimmen, so dass für
jeden Index $i$ die Gleichung
\[
h_i = {\color{red}h_0} + {\color{red}v_0}t_i -\frac{1}{2}{\color{red}g}t_i^2
\]
möglichst gut erfüllt ist.
Dies liefert uns das Gleichungssystem
\begin{equation}
\begin{linsys}{3}
{\color{red}h_0}&+&t_1{\color{red}v_0}&+&\frac{-1}{2}t_1^2{\color{red}g}&=&h_1\phantom{,}\\
{\color{red}h_0}&+&t_2{\color{red}v_0}&+&\frac{-1}{2}t_2^2{\color{red}g}&=&h_2\phantom{,}\\
{\color{red}h_0}&+&t_3{\color{red}v_0}&+&\frac{-1}{2}t_3^2{\color{red}g}&=&h_3\phantom{,}\\
{\color{red}h_0}&+&t_4{\color{red}v_0}&+&\frac{-1}{2}t_4^2{\color{red}g}&=&h_4\phantom{,}\\
{\color{red}h_0}&+&t_5{\color{red}v_0}&+&\frac{-1}{2}t_5^2{\color{red}g}&=&h_5,\\
\end{linsys}
\label{40000064:gl}
\end{equation}
welches im Sinne der kleinsten Quadrate gelöst werden muss.
In Matrixform ist dies das Gleichungssystem
\[
\underbrace{
\begin{pmatrix}
1& 0.00& \phantom{-}0.00000\\
1& 0.05& -0.00125\\
1& 0.10& -0.00500\\
1& 0.15& -0.01125\\
1& 0.20& -0.02000
\end{pmatrix}
}_{\displaystyle A}
\underbrace{
\begin{pmatrix}
\color{red}h_0\\
\color{red}v_0\\
\color{red}g
\end{pmatrix}
}_{\displaystyle x}
=
\underbrace{
\begin{pmatrix}
1.4009\\
1.2176\\
1.0097\\
0.7774\\
0.5209
\end{pmatrix}
}_{\displaystyle b}.
\]
Um die Unbekannten $g$, $v_0$ und $h_0$ zu bestimmen, muss das
Gleichungssystem $\transpose{A}Ax=\transpose{A}b$ gelöst werden.
Die Matrizenprodukte sind
\[
\transpose{A}A
=
\begin{pmatrix}
5 & 0.5 & -0.0375\\
0.5 & 0.075 & -0.00625\\
-0.0375 & -0.00625 & 5.53125\cdot 10^{-4}
\end{pmatrix},
\qquad
\transpose{A}b
=
\begin{pmatrix}
4.9265\\
0.38264\\
-0.02573425
\end{pmatrix}.
\]
\item
Die Lösung dieses Gleichungssystems mit dem Taschenrechner liefert
\[
  x = (\transpose{A}A)^{-1}\transpose{A}b = 
  \begin{pmatrix}
    1.4\\
    -3.42\\
   9.76 
  \end{pmatrix}.
\]
Folglich ist die beste Schätzung für die
örtliche Beschleunigung $g = 9.76\,\frac{\text{m}}{s^2}$.
\end{teilaufgaben}
\end{loesung}

\begin{bewertung}
Least-Squares Gleichungen \eqref{40000064:gl} ({\bf G}) 1 Punkt,
Matrix $A$ ({\bf A}) 1 Punkt,
Vektor $b$ ({\bf B}) 1 Punkt,
Matrix $\transpose{A}A$ ({\bf L}) 1 Punkt,
Vektor $\transpose{A}b$ ({\bf R}) 1 Punkt,
Lösungsvektor $x$ und Wert $g$ ({\bf X}) 1 Punkt,
\end{bewertung}
