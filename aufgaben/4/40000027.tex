Über einem gleichseitigen Sechseck in der $x$-$y$-Ebene mit Umkreisradius $1$ 
wird eine Pyramide mit Spitze im Punkt $(0,0,1)$ errichtet.
Der Punkt $A$ hat die Koordinaten $(\frac12,\frac{\sqrt{3}}2,0)$.
Finden Sie eine Drehmatrix $R$, die die Pyramide links im Bild in die
Pyramide rechts im Bild abbildet.
Bestimmen sie ausserdem den Drehwinkel von $R$.

\begin{center}
\begin{tikzpicture}[thick,>=latex]

\begin{scope}[xshift=-3.7cm]
\node at (0,0) {\includeagraphics[width=0.4\hsize]{pyramid.jpg}};
\node at (-3.3,-1.7) {$x$};
\node at (3.3,-1.1) {$y$};
\node at (0,3) {$z$};
\node at (1.6,-2) {$A$};
\end{scope}

\begin{scope}[xshift=3.7cm]
\node at (0,0) {\includeagraphics[width=0.4\hsize]{rotated.jpg}};
\node at (-3.3,-1.7) {$x$};
\node at (3.3,-1.1) {$y$};
\node at (0,3) {$z$};
\node at (-0.5,2.3) {$A'$};
\end{scope}

\end{tikzpicture}
\end{center}

\thema{Abbildungsmatrix}
\thema{Drehmatrix}
\thema{inverse Matrix}
\thema{Drehwinkel}

\begin{loesung}
Die Matrix $R$ bildet die Spitze $(0,0,1)$ auf $(1,0,0)$ ab und $A$
auf $A'$.
Um die Abbildungsmatrix zu bestimmen brauchen wir aber noch einen 
weiteren Punkt.
Man liest aus dem Bild ab, dass 
$(1,0,0)$ abgebildet wird auf $(0, \frac{\sqrt{3}}{2},\frac12)$.
Wir haben also die Abbildungen
\[
\begin{aligned}
R
\begin{pmatrix}
1&\frac12         &0\\
0&\frac{\sqrt{3}}2&0\\
0&0               &1
\end{pmatrix}
&=
\begin{pmatrix}
0               &0&1\\
\frac{\sqrt{3}}2&0&0\\
\frac12         &1&0
\end{pmatrix}
&&\Rightarrow&
R
&=
\begin{pmatrix}
0               &0&1\\
\frac{\sqrt{3}}2&0&0\\
\frac12         &1&0
\end{pmatrix}
\begin{pmatrix}
1&\frac12         &0\\
0&\frac{\sqrt{3}}2&0\\
0&0               &1
\end{pmatrix}^{-1}
\end{aligned}.
\]
Wir berechnen die inverse Matrix mit dem Gauss-Algorithmus
\begin{align*}
\begin{tabular}{|>{$}c<{$}>{$}c<{$}>{$}c<{$}|>{$}c<{$}>{$}c<{$}>{$}c<{$}|}
\hline
1&\frac12&0&1&0&0\\
0&\frac{\sqrt{3}}2&0&0&1&0\\
0&0&1&0&0&1\\
\hline
\end{tabular}
&\rightarrow
\begin{tabular}{|>{$}c<{$}>{$}c<{$}>{$}c<{$}|>{$}c<{$}>{$}c<{$}>{$}c<{$}|}
\hline
1&\frac12&0&1&0&0\\
0&1&0&0&\frac{2}{\sqrt{3}}&0\\
0&0&1&0&0&1\\
\hline
\end{tabular}
\rightarrow
\begin{tabular}{|>{$}c<{$}>{$}c<{$}>{$}c<{$}|>{$}c<{$}>{$}c<{$}>{$}c<{$}|}
\hline
1&0&0&1&-\frac{1}{\sqrt{3}}&0\\
0&1&0&0&\frac{2}{\sqrt{3}}&0\\
0&0&1&0&0&1\\
\hline
\end{tabular}
\end{align*}
Kontrolle:
\[
\begin{pmatrix}
1&\frac12&0\\
0&\frac{\sqrt{3}}2&0\\
0&0&1
\end{pmatrix}
\begin{pmatrix}
1&-\frac1{\sqrt{3}}&0\\
0&\frac{2}{\sqrt{3}}&0\\
0&0&1
\end{pmatrix}
=
\begin{pmatrix}
1&0&0\\
0&1&0\\
0&0&1
\end{pmatrix}
=I.
\]
Daraus erhalten wir die Matrix 
\[
R=
\begin{pmatrix}
0               &0&1\\
\frac{\sqrt{3}}2&0&0\\
\frac12         &1&0
\end{pmatrix}
\begin{pmatrix}
1&-\frac1{\sqrt{3}}&0\\
0&\frac{2}{\sqrt{3}}&0\\
0&0&1
\end{pmatrix}
=
\begin{pmatrix}
0&0&1\\
\frac{\sqrt{3}}2&-\frac12&0\\
\frac12&\frac{\sqrt{3}}2&0
\end{pmatrix}.
\]
Man kann durch nachrechnen prüfen, dass $R\transpose{R}=I$ und $\det(R)=1$, die 
Matrix $R$ ist also tatsächlich eine Drehmatrix.
Den Drehwinkel von $R$ erhalten wir aus der Spurformel
\begin{align*}
\cos\alpha &= \frac{\operatorname{Spur}A-1}{2} = -\frac{3}{4},
\\
\alpha &=138.59^\circ.\qedhere
\end{align*}
\end{loesung}

\begin{bewertung}
Abbildungspunkte $A$ und Spitze ({\bf A}) 1 Punkt,
Dritter Abbildungspunkt ({\bf D}) 1 Punkt,
inverse Matrix ({\bf I}) 1 Punkt,
Matrizenprodukt ({\bf P}) 1 Punkt,
Matrix $R$ ({\bf R}) 1 Punkt,
Spurformel und Drehwinkel ({\bf W}) 1 Punkt.
\end{bewertung}

