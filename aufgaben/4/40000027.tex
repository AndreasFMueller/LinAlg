Ein neu entwickeltes Drehzahlmessgerät basiert auf einem
3-Achs-Beschleunigungs\-sensor, der genau auf der Achse der zu
messenden Drehung montiert wird. Diese Achse sei die $x$-Achse.
Er misst laufend die Richtung der Erdbeschleunigung und übermittelt
die Messresultate via Bluetooth.

Der Sensor hat ein eigenes, mit dem Sensor mitdrehendes Koordinatensystem,
dessen $z$-Achse parallel zur Drehachse verläuft.
Ist die Drehachse horizontal, befindet sich der gemessene Vektor
in der $x$-$y$-Ebene des drehenden Koordinatensystems.
Der Drehwinkel ist proportional zur Zeit: $\omega t$, $\omega$ ist die
Winkelgeschwindigkeit.

Nun stellt sich aber heraus, dass die Drehachse
gegen die Horizontale um den Winkel $\alpha$ geneigt ist. 
Finden Sie die Transformationsmatrix, die gestattet, den übermittelten Vektor
aus dem drehenden Koordinatensystem in das feste Koordinatensystem
umzurechnen.

\begin{loesung}
Zunächst sind schon zur Zeit $t=0$ die Koordinatensysteme gegeneinander
verdreht: die $z$-Achse des rotierenden Koordinatensystems ist die 
$x$-Achse des festen Koordinatensystems.
Die Umrechung von rotierendem Koordinatensystem zu festem Koordinatensystem
erfolgt mit der Matrix
\[
Z
=
\begin{pmatrix}
0& 0&1\\
0&-1&0\\
1& 0&0
\end{pmatrix}
\]

Die Drehung um die $z$-Achse ist im rotierenden Koordinatensystem
durch die Transformationsmatrix
\[
D_{z,\omega t}
=
\begin{pmatrix}
\cos\omega t &-\sin\omega t & 0\\
\sin\omega t & \cos\omega t & 0\\
     0       &     0        & 1
\end{pmatrix}
\]
gegeben.
Zur Zeit $\omega t$ zeigt ein Vektor mit Koordinaten $x$ im rotierenden
Koordinatensystem in Richtung $D_{z,\omega t}^{-1}x=D_{z,-\omega t}x$,
dargestellt im rotierenden Koordinatensystem zur Zeit $t=0$.
Im festen Koordinatensystem ist dies $ZD_{z,-\omega t}x$.

Da die Drehachse aber geneigt ist, muss diese Matrix noch mit einer
Drehung um den Winkel $\alpha$ um die $y$-Achse multipliziert werden:
\begin{align*}
T&=
D_{y,\beta}ZD_{z,-\omega t}\\
&=
\begin{pmatrix}
\cos\alpha&0&-\sin\alpha\\
     0    &1&     0     \\
\sin\alpha&0& \cos\alpha
\end{pmatrix}
\begin{pmatrix}
0& 0&1\\
0&-1&0\\
1& 0&0
\end{pmatrix}
\begin{pmatrix}
 \cos\omega t & \sin\omega t & 0\\
-\sin\omega t & \cos\omega t & 0\\
     0       &     0        & 1
\end{pmatrix}
\\
&=
\begin{pmatrix}
\cos\alpha&0&-\sin\alpha\\
     0    &1&     0     \\
\sin\alpha&0& \cos\alpha
\end{pmatrix}
\begin{pmatrix}
     0        &      0       & 1\\
 \sin\omega t &-\cos\omega t & 0\\
 \cos\omega t & \sin\omega t & 0
\end{pmatrix}
\\
&=
\begin{pmatrix}
\sin\alpha\cos\omega t & -\sin\alpha\cos\omega t &\cos\alpha\\
\sin\omega t           & -\cos\omega t           & 0\\
\cos\alpha\cos\omega t &  \cos\alpha\sin\omega t &\sin\alpha
\end{pmatrix}.
\qedhere
\end{align*}
\end{loesung}

