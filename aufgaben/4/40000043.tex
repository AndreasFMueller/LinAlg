Die Punkte
\[
A=( 1, 1, 1),\quad
B=(-1,-1, 1),\quad
C=( 1,-1,-1)
\quad\text{und}\quad
D=(-1, 1,-1)
\]
sind die Ecken eines Tetraeders.
\begin{teilaufgaben}
\item
Finden Sie eine Drehmatrix, die den Punkt $A$ fest l"asst und den
Punkt $B$ in den Punkt $C$ "uberf"uhrt.
\item
Finden Sie den Drehwinkel, der zu dieser Drehmatrix geh"ort.
\end{teilaufgaben}

\begin{hinweis}
Die gesuchte Matrix $R$ f"uhrt auch $C$ in $D$ "uber.
Schreiben Sie die Bedingung, dass $R$ die genannten Punkte in andere
Punkte "uberf"uhrt, als Matrizengleichung%
{} $RB_1=B_2$%
.
\end{hinweis}

\begin{loesung}
\begin{teilaufgaben}
\item
Da die gesuchte Matrix $R$ eine Drehmatrix ist, bleiben L"angen und
Winkel erhalten.
Sie wird wird daher auch den Punkt $C$ in den Punkt $D$ "uberf"uhren.
Wir bezeichnen die Ortsvektoren der Punkte mit entsprechenden kleinen
Buchstaben, also $\vec{a}=\overrightarrow{0A}$.

Gesucht ist also eine Matrix $R$ mit den Eigenschaften:
\begin{equation}
R\vec{a}=\vec{a},\quad
R\vec{b}=\vec{c}\quad\text{und}\quad
R\vec{c}=\vec{d}.
\label{40000043:vektorgleichungen}
\end{equation}
Schreiben wir die Vektoren $\vec{a}$, $\vec{b}$ und $\vec{c}$ in die
Matrix $B_1$ und die Vektoren $\vec{a}$, $\vec{c}$ und $\vec{a}$ in
die Matrix $B_2$, also
\[
B_1
=
\begin{pmatrix}
 1&-1& 1\\
 1&-1&-1\\
 1& 1&-1
\end{pmatrix}
\qquad\text{und}\qquad
B_2
=
\begin{pmatrix}
1& 1&-1\\
1&-1& 1\\
1&-1&-1
\end{pmatrix},
\]
dann lassen sich die
Gleichungen~\eqref{40000043:vektorgleichungen} zusammenfassen in eine
Matrixgleichung
\[
RB_1=B_2
\qquad\Rightarrow\qquad R=B_2B_1^{-1}.
\]
Wir bestimmen daher zun"achst die inverse Matrix $B_1^{-1}$
mit Hilfe des Gauss-Algorithmus:
\begin{align*}
\begin{tabular}{|>{$}c<{$}>{$}c<{$}>{$}c<{$}|>{$}c<{$}>{$}c<{$}>{$}c<{$}|}
\hline
 1&-1& 1& 1& 0& 0\\
 1&-1&-1& 0& 1& 0\\
 1& 1&-1& 0& 0& 1\\
\hline
\end{tabular}
&
\rightarrow
\begin{tabular}{|>{$}c<{$}>{$}c<{$}>{$}c<{$}|>{$}c<{$}>{$}c<{$}>{$}c<{$}|}
\hline
 1&-1& 1& 1& 0& 0\\
 0& 0&-2&-1& 1& 0\\
 0& 2&-2&-1& 0& 1\\
\hline
\end{tabular}
\rightarrow
\begin{tabular}{|>{$}c<{$}>{$}c<{$}>{$}c<{$}|>{$}c<{$}>{$}c<{$}>{$}c<{$}|}
\hline
 1&-1& 1&      1 & 0& 0      \\
 0& 1&-1&-\frac12& 0& \frac12\\
 0& 0&-2&-     1 & 1& 0      \\
\hline
\end{tabular}
\\
&
\rightarrow
\begin{tabular}{|>{$}c<{$}>{$}c<{$}>{$}c<{$}|>{$}c<{$}>{$}c<{$}>{$}c<{$}|}
\hline
 1&-1& 0& \frac12& \frac12& 0      \\
 0& 1& 0&      0 &-\frac12& \frac12\\
 0& 0& 1& \frac12&-\frac12& 0      \\
\hline
\end{tabular}
\rightarrow
\begin{tabular}{|>{$}c<{$}>{$}c<{$}>{$}c<{$}|>{$}c<{$}>{$}c<{$}>{$}c<{$}|}
\hline
 1& 0& 0& \frac12&      0 & \frac12\\
 0& 1& 0&      0 &-\frac12& \frac12\\
 0& 0& 1& \frac12&-\frac12& 0      \\
\hline
\end{tabular}
\end{align*}
Damit kann jetzt auch $R$ berechnet werden:
\begin{align*}
R
=
B_2B_1^{-1}
&=
\begin{pmatrix}
1& 1&-1\\
1&-1& 1\\
1&-1&-1
\end{pmatrix}
\begin{pmatrix}
 \frac12&      0 & \frac12\\
      0 &-\frac12& \frac12\\
 \frac12&-\frac12& 0
\end{pmatrix}
=
\begin{pmatrix}
0&0&1\\
1&0&0\\
0&1&0
\end{pmatrix}.
\end{align*}
\item
Der Drehwinkel kann mit Hilfe der Spurformel ermittelt werden:
\[
\cos \alpha
=
\frac{\operatorname{Spur}R -1 }2
=
-\frac12
\qquad\Rightarrow\qquad
\alpha = 120^\circ.
\qedhere
\]
\end{teilaufgaben}
\end{loesung}

\begin{bewertung}
Bestimmung der Matrizen $B_1$ ($\textbf{B}_1$) und $B_2$ ($\textbf{B}_2$)
je 1 Punkt,
Berechnung der Inversen $B_1^{-1}$  ({\bf I}) 2 Punkte,
Berechnung des Produktes $B_2B_1^{-1}$ ({\bf P}) 1 Punkt,
Anwendung der Spurformel und korrekter Drehwinkel ({\bf D}) 1 Punkt.
\end{bewertung}

