\begin{teilaufgaben}
\item
Finden Sie den Mittelpunkt $M$ des Kreises mit der Gleichung
\begin{equation}
x^2-8x+y^2+6y=0.
\ifthenelse{\boolean{presentation}}{\notag}{
\label{40000080:eqn}
}
\end{equation}
\item
Stellen Sie die Gleichung eines Kreises mit Radius $r$ um den
Mittelpunkt $M$ von a) auf.
\item
Finden Sie den Radius des Kreises von a).
\end{teilaufgaben}

\begin{loesung}
\begin{teilaufgaben}
\item
Der Mittelpunkt kann von den Koeffizienten der linearen Terme
abgelesen werden: $M=(4,-3)$.
\item
Da der Mittelpunkt bekannt ist, kann man die Kreisgleichung
\begin{align}
(\vec{p}-\vec{m})^2&= r^2
\notag
\\
(x-4)^2 + (y+3)^2 &= r^2
\notag
\\
x^2-8x+16 + y^2 + 6y + 9 &= r^2
\label{40000080:eqn2}
\end{align}
aufstellen.
\item
Bringen wir die konstanten Terme in 
\eqref{40000080:eqn2}
auf die rechte Seite, ergibt sich
\begin{align*}
x^2-8x + y^2 + 6y + 25 &= r^2
\\
\Rightarrow\qquad
x^2-8x + y^2 + 6y \phantom{\mathstrut+25} &= r^2-25.
\end{align*}
Da die rechte Seite in \eqref{40000080:eqn} verschwindet, muss
$r^2=25$ bzw.~$r=5$ sein.
\qedhere
\end{teilaufgaben}
\end{loesung}

