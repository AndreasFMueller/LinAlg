Finden Sie die Transformationsmatrix $T$, welche Koordinaten in der Basis
\[
B=\left\{
\begin{pmatrix}
2\\0\\1\\2
\end{pmatrix},\;
\begin{pmatrix}
-4\\-1\\5\\4
\end{pmatrix},\;
\begin{pmatrix}
-3\\1\\0\\5
\end{pmatrix}
\right\}
\]
in Koordinaten in der Basis
\[
B'=\left\{
\begin{pmatrix}
-4\\-1\\5\\4
\end{pmatrix},\;
\begin{pmatrix}
-5\\0\\6\\11
\end{pmatrix},\;
\begin{pmatrix}
6\\1\\-4\\-2
\end{pmatrix}
\right\}
\]
umrechnet.
Bestimmen Sie ausserdem die Koordinaten $\xi$ des Vektors 
\[
v
=
\begin{pmatrix}
3\\2\\-4\\3
\end{pmatrix}
\]
in der Basis $B$ und rechnen Sie sie mit der Transformationsmatrix um in
Koordinaten in der Basis $B'$. Kontrollieren Sie Ihr Resultat.

\begin{loesung}
Die Transformationsmatrix kann mit Hilfe des Gaussalgorithmus gefunden
werden:
\begin{align*}
\begin{tabular}{|>{$}c<{$}>{$}c<{$}>{$}c<{$}|>{$}c<{$}>{$}c<{$}>{$}c<{$}|}
\hline
   -4 & -5 &  6 &  2 & -4 & -3 \\
   -1 &  0 &  1 &  0 & -1 &  1 \\
    5 &  6 & -4 &  1 &  5 & -0 \\
    4 & 11 & -2 &  2 &  4 &  5 \\
\hline
\end{tabular}
&\rightarrow
\begin{tabular}{|>{$}c<{$}>{$}c<{$}>{$}c<{$}|>{$}c<{$}>{$}c<{$}>{$}c<{$}|}
\hline
   1 & 0 & 0 & 1 & 1 &-2 \\
   0 & 1 & 0 & 0 & 0 & 1 \\
   0 & 0 & 1 & 1 & 0 &-1 \\
\hline
   0 & 0 & 0 & \color{red}0 & \color{red}0 & \color{red}0 \\
\hline
\end{tabular}
\end{align*}
Daraus kann man ablesen, dass die Transformationsmatrix 
\[
T
=
\begin{pmatrix}
1&1&-2\\
0&0& 1\\
1&0&-1
\end{pmatrix}
\]
ist. Der Vektor $v$ kann tats"achlich in der Basis $B$ ausgedr"uckt werden,
dazu l"ost man das Gleichungssystem
\begin{align*}
\begin{tabular}{|>{$}c<{$}>{$}c<{$}>{$}c<{$}|>{$}c<{$}|}
\hline
   -4 & -5 &  6 &  3 \\
   -1 &  0 &  1 &  2 \\
    5 &  6 & -4 & -4 \\
    4 & 11 & -2 &  3 \\
\hline
\end{tabular}
&\rightarrow
\begin{tabular}{|>{$}c<{$}>{$}c<{$}>{$}c<{$}|>{$}c<{$}|}
\hline
   1 & 0 & 0 & 1\\
   0 & 1 & 0 &-1\\
   0 & 0 & 1 & 1\\
\hline
   0 & 0 & 0 & \color{red}0 \\
\hline
\end{tabular}
\end{align*}
Die rote $\color{red}0$ zeigt an, dass das Gleichungssystem l"osbar ist,
die Koordinaten von $v$ in der Basis $B$ sind 
\[
\xi
=
\begin{pmatrix}
1\\-1\\1
\end{pmatrix}
\]
Mit der Matrix $T$ erh"alt man daraus die Koordinaten
\[
\xi'=T\xi=
\begin{pmatrix}
1&1&-2\\
0&0& 1\\
1&0&-1
\end{pmatrix}
\begin{pmatrix}
1\\-1\\1
\end{pmatrix}
=
\begin{pmatrix}
-2\\1\\0
\end{pmatrix}.
\]
Zur Kontrolle kann man mit den eben gefundenen Koordinaten $\xi'$ den
zugeh"origen Vektor in $\mathbb R^4$ nachrechnen, es sollte sich $v$
ergeben. Die Rechnung ergibt
\[
B'\xi'=
\begin{pmatrix}
-4&-5& 6\\
-1& 0& 1\\
 5& 6&-4\\
 4&11&-2
\end{pmatrix}
\begin{pmatrix}
-2\\1\\0
\end{pmatrix}
=
\begin{pmatrix}
3\\2\\-4\\3
\end{pmatrix}.
\]
\end{loesung}


