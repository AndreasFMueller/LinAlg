Die Erde f"uhrt wie jeder Kreisel unter Einfluss der Schwerkraft von
Sonne und Mond eine Pr"azessionsbewegung aus.
Die unregelm"assige Gestalt der Erde sowie die komplizierten Prozesse
im Erdinneren f"uhren zu einer noch komplizerten Bewegung.
Eine pr"azise Bestimmung dieser Bewegung ist zum Beispiel n"otig, um
die Genauigkeit des GPS aufrecht zu erhalten.
Dazu wird zum Beispiel von der Website \texttt{maia.usno.navy.mil}
ein Formular angeboten, mit dem man die aktuelle Verdrehung eines
mit der Erde verbundenen Koordinatensystems gegen"uber einem als
ruhend angenommenen, mit Hilfe von fernen Radiogalaxien definierten
Koordinatensystem berechnen lassen kann.
Die genaue Definition des Koordinatensystems ignorieren wir.
Der Output hat die Form einer Drehmatrix.
F"ur den 14.~September 2015 um 08:10 (erste Lektion des Herbstsemsters 2015)
und den 5.~Februar 2016 um 14:30 bekommt man die Matrizen
\begin{align*}
R_1 &= \begin{pmatrix}
\phantom{-}0.999992660750488&0.351397718674194&\phantom{-}0.001526567879330\\
          -0.003513912955321&0.999993825187267&          -0.000044755791542\\
          -0.001526715723890&0.000039391236420&\phantom{-}0.999998833793034
\end{pmatrix},
\tag{2015-09-14 08:10}
\\
R_2 &= \begin{pmatrix}
\phantom{-}0.999992309138471&0.003597169715623&\phantom{-}0.001562700849799\\
          -0.003597099287045&0.999993529271332&          -0.000047876764841\\
          -0.001562862958834&0.000042255206515&\phantom{-}0.999998777836188
\end{pmatrix}.
\tag{2016-02-05 14:30}
\end{align*}
\begin{teilaufgaben}
\item
Wie kann man "uberpr"ufen, ob die Matrizen $R_1$ und $R_2$ tats"achlich
Drehmatrizen sind?
\item
Um welchen Winkel in Winkelsekunden hat sich das Koordinatensystem
zwischen Semesterbeginn und LinAlg-Pr"ufung gedreht?
\end{teilaufgaben}

\begin{hinweis}
Eventuell Matrizenoperationen mit dem Taschenrechner mit mindestens
zehn Stellen Genauigkeit ausf"uhren.
\end{hinweis}

\begin{loesung}
\begin{teilaufgaben}
\item
Drehmatrizen sind orthogonale Matrizen mit Determinaten $+1$.
Man muss also nachrechnen, dass $R_1R_1^t=E$ und $\det R_1=1$
im Rahmen der Rechengenauigkeit gilt. (Es war nicht verlangt,
dass diese Rechnungen durchgef"uhrt werden.)
\item
Wir gehen davon aus, dass beide Matrizen Drehmatrizen sind.
Die Matrix $R=R_2R_1^{-1}$ beschreibt die Drehung vom Zustand zu Beginn
des Semsters bis heute.
\[
R = \begin{pmatrix}
\phantom{-}0.999999995886644& 0.000083191633558&\phantom{-}0.000036136742198\\
          -0.000083191525418& 0.999999996535103&          -0.000002993979435\\
          -0.000036136991146& 0.000002990973152&\phantom{-}0.999999999342587
\end{pmatrix}
\]
Den Drehwinkel k"onnen wir mit der Formel f"ur den Drehwinkel ermitteln:
\begin{align*}
\cos\alpha
&=
\frac{\operatorname{Spur}R-1}2=\frac{2.999999991764334-1}2
=
0.999999995882167
\\
\alpha&=18.7''
\end{align*}
\end{teilaufgaben}
\end{loesung}

\begin{diskussion}
Richtet man eine Spiegelreflexkamera mit einem Normalobjektiv gegen den
Himmel, dann entspricht dieser Winkel ungef"ahr dem Durchmesser eines Pixel.
Messungen dieser Genauigkeit sind heutzutage standard in der Astronomie.
Der 1989 gestartete Hipparcos Satellit hat im Laufe seiner dreij"ahrigen
Mission 118000 Sterne mit einer Genauigkeit $0.001''$ vermessen.
Die aktuell laufende Gaia-Mission der ESA soll von etwa einer Milliarde
Sternen die Position mit einer Genauigkeit von ewa $0.000025''$ messen.
\end{diskussion}



