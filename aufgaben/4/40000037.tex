Die Erde f"uhrt wie jeder Kreisel unter Einfluss der Schwerkraft von
Sonne und Mond eine Pr"azessionsbewegung aus.
Die unregelm"assige Gestalt der Erde sowie die komplizierten Prozesse
im Erdinneren verkomplizieren die Bewegung weiter.
Eine pr"azise Bestimmung dieser Bewegung ist zum Beispiel n"otig, um
die Genauigkeit des GPS aufrecht zu erhalten.

Die Website \texttt{maia.usno.navy.mil} bietet
ein Formular an, mit dem man die aktuelle Verdrehung eines
mit der Erde verbundenen Koordinatensystems gegen"uber einem als
ruhend angenommenen, mit Hilfe von fernen Radiogalaxien definierten
Koordinatensystem berechnen lassen kann.
Die genaue Definition des Koordinatensystems ignorieren wir.
Der Output hat die Form einer Drehmatrix.
F"ur den 14.~September 2015 um 08:10 (erste Lektion des Herbstsemsters 2015)
und den 5.~Februar 2016 um 14:30 (Pr"ufungsende) bekommt man die Matrizen
\begin{align*}
R_1 &= \begin{pmatrix}
\phantom{-}0.999992660750488&0.003513977186741&\phantom{-}0.001526567879330\\
          -0.003513912955321&0.999993825187267&          -0.000044755791542\\
          -0.001526715723890&0.000039391236420&\phantom{-}0.999998833793034
\end{pmatrix},
\tag{2015-09-14 08:10}
\\
R_2 &= \begin{pmatrix}
\phantom{-}0.999992309138471&0.003597169715623&\phantom{-}0.001562700849799\\
          -0.003597099287045&0.999993529271332&          -0.000047876764841\\
          -0.001562862958834&0.000042255206515&\phantom{-}0.999998777836188
\end{pmatrix}.
\tag{2016-02-05 14:30}
\end{align*}
\begin{teilaufgaben}
\item
Wie kann man "uberpr"ufen, ob die Matrizen $R_1$ und $R_2$ tats"achlich
Drehmatrizen sind?
\item
Um welchen Winkel in Winkelsekunden hat sich das Koordinatensystem
zwischen Semesterbeginn und LinAlg-Pr"ufung gedreht?
\end{teilaufgaben}

\begin{hinweis}
F"urhen Sie eventuelle Matrizenoperationen mit dem Taschenrechner
mit mindestens zehn Stellen Genauigkeit aus.
Sie m"ussen die "Uberpr"ufung in a) nicht durchf"uhren, sondern nur
beschreiben, was gemacht werden muss.
\end{hinweis}

\begin{loesung}
\begin{teilaufgaben}
\item
Drehmatrizen sind orthogonale Matrizen mit Determinaten $+1$.
Man muss also nachrechnen, dass $R_1R_1^t=E$ und $\det R_1=1$
im Rahmen der Rechengenauigkeit gilt. (Es war nicht verlangt,
dass diese Rechnungen durchgef"uhrt werden.)
\item
Wir gehen davon aus, dass beide Matrizen Drehmatrizen sind.
Die Matrix $R=R_2R_1^{-1}$ beschreibt die Drehung vom Zustand zu Beginn
des Semsters bis heute.
\[
R = \begin{pmatrix}
\phantom{-}0.999999995886644& 0.000083191633558&\phantom{-}0.000036136742198\\
          -0.000083191525418& 0.999999996535103&          -0.000002993979435\\
          -0.000036136991146& 0.000002990973152&\phantom{-}0.999999999342587
\end{pmatrix}
\]
Den Drehwinkel k"onnen wir mit der Spurformel f"ur den Drehwinkel ermitteln:
\begin{align*}
\cos\alpha
&=
\frac{\operatorname{Spur}R-1}2=\frac{2.999999991764334-1}2
=
0.999999995882167
\\
\alpha&=18.7''
\end{align*}
\end{teilaufgaben}
\end{loesung}

\begin{diskussion}
Es ist nicht ganz korrekt, den zu den Matrizen $R_i$ geh"orenden Drehwinkel
$\alpha_i$ mit der Spurformel zu berechnen, und anzunehmen, dass deren
Differenz der gesuchte Winkel sei.
Dazu m"ussen alle Drehungen die gleiche Drehachse haben, was nicht
vorausgesetzt werden kann.

Richtet man eine Spiegelreflexkamera mit einem Normalobjektiv gegen den
Himmel, dann entspricht dieser Winkel ungef"ahr dem Durchmesser eines Pixel.
Messungen dieser Genauigkeit sind heutzutage Standard in der Astronomie.
Der 1989 gestartete Hipparcos Satellit hat im Laufe seiner dreij"ahrigen
Mission 118000 Sterne mit einer Genauigkeit $0.001''$ vermessen.
Die aktuell laufende Gaia-Mission der ESA soll von etwa einer Milliarde
Sternen die Position mit einer Genauigkeit von ewa $0.000025''$ messen.
\end{diskussion}

\begin{bewertung}
Charakterisierung einer Drehmatrix durch Orthogonalit"at ({\bf O}) 1 Punkt
und Determinante ({\bf D}) 1 Punkt,
Matrix $R$ f"ur die zwischen den zwei Zeitpunkten erfolgte Drehung ({\bf Z})
1 Punkt,
Berechnung von $R$ ({\bf R}) 1 Punkt,
Spurformel ({\bf S}) 1 Punkt,
Drehwinkel $\alpha$ ({\bf A}) 1 Punkt.
\end{bewertung}

