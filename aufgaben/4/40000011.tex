Die Zeitschrift {\em The Economist} hat 1986 den Big Mac Index geschaffen,
angeblich um die Theorie der Wechselkurse ebenso ``leicht verdaulich''
zu machen wie das namensgebende Produkt.
Unter anderem wird regelmässig die Abhängigkeit des Big Mac Preises
vom Bruttoinlandprodukt (BIP) pro Person ermittelt und publiziert.
\begin{center}
\begin{tabular}{|l|r|r|}
\hline
Land&BIP pro Person USD&Big Mac Preis USD\\
\hline
Norwegen&96930&5.21\\
USA&54370&4.93\\
%Euro Raum&40001&4.00\\
Indien&1608&1.90\\
\hline
\end{tabular}
\end{center}
Es wird postuliert, dass es einen linearen Zusammenhang der Form $f(x)=ax+b$
zwischen BIP pro Person und Big Mac Preis gäbe.

\thema{Least Squares}

\begin{teilaufgaben}
\item Stellen Sie ein Gleichungssystem auf, mit dem Sie die Koeffizienten
einer Funktion berechnen können, die den postulierten linearen Zusammenhang
wiedergibt.
\item Lösen das Gleichungssystem (mit dem Taschenrechner).
\item Die Schweiz hat eine BIP pro Kopf von 86468USD, ein Big Mac kostet
in der Schweiz 6.44USD.
Zahlt man in der Schweiz zu viel für einen Big Mac?
\end{teilaufgaben}

\begin{loesung}
\begin{teilaufgaben}
\item
Es wird behauptet, es gäbe einen Zusammenhang zwischen BIP pro Person $x$
und Big Mac Preis $y$ in der Form
\[
y={\color{red}a}x+{\color{red}b}
\]
Aus den angegebenen Daten können wir das Gleichungssystem
\[
\begin{linsys}{2}
96930\color{red}a&+&\color{red}b&=&5.21\\
54370\color{red}a&+&\color{red}b&=&4.93\\
 1608\color{red}a&+&\color{red}b&=&1.90
\end{linsys}
\]
aufstellen.
Es ist ein überbestimmtes Gleichungssstem mit
\[
A=\begin{pmatrix}
96930&1\\
54270&1\\
\phantom{0}1608&1
\end{pmatrix}
\qquad
\text{und}
\qquad
b=\begin{pmatrix}
5.21\\
4.93\\
1.90
\end{pmatrix}
\]
Das Lösungsverfahren für überbestimmte Gleichungssysteme besagt,
dass man stattdessen das Gleichungssystem 
\[
\transpose{A}A\begin{pmatrix}\color{red}a\\\color{red}b\end{pmatrix}
=
\transpose{A}b
\]
lösen muss.
Die Berechnung liefert 
\[
\transpose{A}A=\begin{pmatrix}
          12354107464&          152908\\
\phantom{00000}152908&\phantom{00000}3
\end{pmatrix}
\qquad
\text{und}
\qquad
\transpose{A}b=\begin{pmatrix}
          77610.46\\
\phantom{000}12.04
\end{pmatrix}.
\]
\item
Die Lösung mit dem Computer oder Taschenrechner liefert
\[
{\color{red}a}=0.000035618
\qquad
\text{und}
\qquad
{\color{red}b}=2.1979
\]
\item
Setzt man das BIP pro Person der Schweiz in den linearen Zusammenhang
ein, erhält man für den geschätzten Big Mac Preis 
\[
86468
{\color{red}a}
+
{\color{red}b}
=
5.2777,
\]
der Big Mac Preis in der Schweiz ist also deutlich höher, also die
Theorie vermuten lässt.
\qedhere
\end{teilaufgaben}
\end{loesung}

\begin{diskussion}
Die Antwort in c) lässt sich auch finden, indem man den Big Mac Preis
als unabhängige Variable (das ``$x$'') und das BIP pro Person als
abhängige Variable wählt. 
Es ist jedoch etwas verwegen davon auszugehen, dass das BIP durch den
Big Mac Preis bestimmt wird\dots
\end{diskussion}

\begin{bewertung}
"Uberbestimmtes Gleichungssystem ({\bf "U}) 1 Punkt (es muss darauf hingewiesen
worden sein, dass dies ein überbestimmtes Gleichungssystem ist),
Lösungsverfahren für überbestimmte Gleichungssysteme ({\bf L}) 1 Punkt,
Matrix $A$ und Vektor $b$ ({\bf M}) 1 Punkt,
Matrix $\transpose{A}A$ und Vektor $\transpose{A}b$ ({\bf A}) 1 Punkt,
Lösung für $a$ und $b$ ({\bf X}) 1 Punkt,
Theoretischer Big Mac Preis ({\bf T}) 1 Punkt.
\end{bewertung}

