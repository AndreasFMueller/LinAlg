Der Mittelpunkt des Würfels mit Kantenlänge $2$ befindet sich im
Koordinatenursprung.
Eine Drehung des Raumes führt die Punkte $A$, $B$ und $C$ in
die Punkte $A'$, $B'$ bzw. $C'$ über.

\begin{center}
\includeagraphics[]{wuerfel.pdf}
\end{center}

\begin{teilaufgaben}
\item Finden Sie die Drehmatrix $R$.
\item Wie gross ist der Winkel zwischen den beiden Ortsvektoren $\overrightarrow{OA}$ und $\overrightarrow{OB}$.
\item Wie gross ist der Winkel zwischen den Bildern der Vektoren
\[
  \begin{pmatrix}1\\1\\0\end{pmatrix}\quad\text{und}\quad
  \begin{pmatrix}-1\\-1\\-1\end{pmatrix}
\]
\end{teilaufgaben}

\thema{Drehmatrix}
\thema{Skalarprodukt}
\thema{Abbildungsmatrix}

\begin{loesung}
\begin{teilaufgaben}
\item
Die dargestellte Drehung bildet die Standardbasis-Vektoren wie folgt ab:
\[
\begin{aligned}
\vec e_1&\mapsto \begin{pmatrix}0\\-1\\0\end{pmatrix},
&
\vec e_2&\mapsto \begin{pmatrix}-1\\0\\0\end{pmatrix}
&&\text{und}
&
\vec e_3&\mapsto \begin{pmatrix}0\\0\\-1\end{pmatrix}.
\end{aligned}
\]
Die gesuchte Drehmatrix ist daher
\[
R
=
\begin{pmatrix}
0&-1& 0\\
-1&0& 0\\
0&0& -1\\
\end{pmatrix}.
\]
Alternativ hätte die Aufgabe aber auch wie folgt gelöst werden können:\\
Da die Punkte $A$, $B$ und $C$ mit Ortsvektoren
\[
  \vec a = \begin{pmatrix}1\\-1\\1\end{pmatrix},\quad
  \vec b = \begin{pmatrix}1\\1\\1\end{pmatrix},\quad
  \vec c = \begin{pmatrix}-1\\-1\\-1\end{pmatrix}                          
\]
auf die Punkte $A'$, $B'$ und $C'$ mit Ortsvektoren
\[
  \vec a' = \begin{pmatrix}1\\-1\\-1\end{pmatrix},\quad
  \vec b' = \begin{pmatrix}-1\\-1\\-1\end{pmatrix},\quad
  \vec c' = \begin{pmatrix}1\\1\\1\end{pmatrix}                          
\]
abgebildet werden, müssen folgende Gleichungen erfüllt sein:
\[
  \vec a' = R\vec a, \quad 
  \vec b' = R\vec b, \quad 
  \vec c' = R\vec c.
\]
Leider kann mit diesen Gleichungen die Drehmatrix noch nicht bestimmt werden, 
da die Punkte $A$, $B$ und $C$ in einer Ebene liegen und damit linear abhängig sind.
Wir müssen daher noch ein weiteres Punktepaar aus der Zeichnung herauslesen.
Wir wählen die vordere untere Ecke
\[
\vec d =\begin{pmatrix}1\\-1\\-1\end{pmatrix} \quad\mapsto\quad \vec d'=\begin{pmatrix}1\\-1\\1\end{pmatrix}
\quad\text{mit}\quad\vec d' = R\vec d.
\]
Die drei Gleichungen
\[
  \vec a' = R\vec a, \quad 
  \vec b' = R\vec b, \quad 
  \vec d' = R\vec d
\]
können nun in einer Gleichung zusammengefasst werden zu
\[
\underbrace{
\begin{pmatrix}
\phantom{-}1&-1&\phantom{-}1\\
         - 1&-1&         - 1\\
         - 1&-1&\phantom{-}1
\end{pmatrix}}_{\displaystyle =P\,'}
=
R
\underbrace{
\begin{pmatrix}
\phantom{-}1&\phantom{-}1&\phantom{-}1\\
         - 1&\phantom{-}1&         - 1\\
\phantom{-}1&\phantom{-}1&         - 1
\end{pmatrix}}_{\displaystyle =P},
\]
wobei in den Spalten der Matrix $P$ die Vektoren $\vec a$, $\vec b$ und $\vec d$ enthalten sind
und in den Spalten der Matrix $P'$ die Vektoren $\vec a'$, $\vec b'$ und $\vec d'$. 
Damit kann nun die Drehmatrix gefunden werden als
\[
R = P'P^{-1} = 
\begin{pmatrix}
\phantom{-}1&-1&\phantom{-}1\\
         - 1&-1&         - 1\\
         - 1&-1&\phantom{-}1
\end{pmatrix}
\begin{pmatrix}
\phantom{-}0&         - \frac12&\phantom{-}\frac12\\
\phantom{-}\frac12&\phantom{-}\frac12&\phantom{-}0\\
\phantom{-}\frac12&\phantom{-}0&         - \frac12
\end{pmatrix}
=
\begin{pmatrix}
\phantom{-}0&         - 1&\phantom{-}0\\
         - 1&\phantom{-}0&\phantom{-}0\\
\phantom{-}0&\phantom{-}0&         - 1
\end{pmatrix}.
\]
Wie man sieht, kommt man auf das selbe Ergebnis wie oben.
\item
Die Zwischenwinkelformel muss auf die Vektoren $\vec a$ und
$\vec b$ angewendet werden:
\begin{align*}
|\vec a| &= \sqrt{1^2+(-1)^2+1^2} = \sqrt{3}\\
|\vec b| &= \sqrt{1^2+1^2+1^2} = \sqrt{3}\\
\cos(\alpha)&= \dfrac{\vec a\cdot \vec b}{|\vec a|\cdot |\vec b|} = \dfrac{1\cdot 1 + (-1)\cdot 1 + 1 \cdot 1}{\sqrt{3}\cdot \sqrt{3}} = \dfrac{1}{3}\\
\alpha&=70.529^\circ.
\end{align*}
\item
Die Zwischenwinkelformel muss auf die Vektoren 
\[
R\begin{pmatrix}1\\1\\0\end{pmatrix}
=
\begin{pmatrix}
0&-1& 0\\
-1&0& 0\\
0&0& -1\\
\end{pmatrix}
\begin{pmatrix}1\\1\\0\end{pmatrix}
= 
\begin{pmatrix} -1\\ -1\\ 0\end{pmatrix}
\]
und
\[
R\begin{pmatrix}-1\\-1\\-1\end{pmatrix}
=
\begin{pmatrix}
0&-1& 0\\
-1&0& 0\\
0&0& -1\\
\end{pmatrix}
\begin{pmatrix}-1\\-1\\-1\end{pmatrix}
= 
\begin{pmatrix} 1\\ 1\\ 1\end{pmatrix}
\]
angewendet werden:
\begin{align*}
\cos(\alpha)&= \dfrac{\begin{pmatrix} -1\\ -1\\ 0\end{pmatrix}\cdot 
\begin{pmatrix} 1\\ 1\\ 1\end{pmatrix}}
{\left|\begin{pmatrix} -1\\ -1\\ 0\end{pmatrix}\right|\cdot \left|\begin{pmatrix} 1\\ 1\\ 1\end{pmatrix}\right|} 
= \dfrac{(-1)\cdot 1 + (-1)\cdot 1 + 0 \cdot 1}{\sqrt{(-1)^2+(-1)^2+0^2}\cdot \sqrt{1^2+1^2+1^2}} = \dfrac{-2}{\sqrt{6}}\\
\alpha&=144.736^\circ.
\end{align*}
\qedhere
\end{teilaufgaben}
\end{loesung}


