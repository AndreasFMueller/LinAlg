Finden Sie die Abbildungsmatrix $A$ einer linearen Abbildung von $\mathbb R^2$,
welche die $x$-Komponente halbiert, die $y$-Komponenten verdoppelt und
danach die beiden Komponenten vertauscht.
\begin{teilaufgaben}
\item Berechnen Sie die Determinante von $A$.
\item Wie gross ist der Winkel zwischen den Bildern der Vektoren
$e_1$ und $e_2$?
\item Wie gross ist der Winkel zwischen den Bildern der Vektoren
\[
\begin{pmatrix}1\\1\end{pmatrix}\quad\text{und}\quad
\begin{pmatrix}1\\-1\end{pmatrix}
\]
\end{teilaufgaben}

\thema{Abbildungsmatrix}
\thema{Determinante}
\thema{Zwischenwinkel}

\begin{loesung}
In den Spalten der Abbildungsmatrix stehen die Bilder der Basisvektoren, also
\[
\begin{pmatrix}1\\0\end{pmatrix}\mapsto\begin{pmatrix}0\\\frac12\end{pmatrix}
,\quad
\begin{pmatrix}0\\1\end{pmatrix}\mapsto\begin{pmatrix}2\\0\end{pmatrix}
\qquad
\Rightarrow\qquad
A=\begin{pmatrix}
0&2\\
\frac12&0
\end{pmatrix}
\]
\begin{teilaufgaben}
\item
Die Determinante ist $\det(A)=-2\cdot\frac12=-1.$
\item
Die Vektoren $e_1$ und $e_2$ liegen nach der Transformation wieder auf
den $x$- bzw.~$y$-Achsen, sie schliessen also auch nach der Transformation
einen rechten Winkel ein.
\item
Die Zwischenwinkelformel muss auf die Vektoren $A(e_1+e_2)$ und
$A(e_1-e_2)$ angewendet werden:
\begin{gather}
\begin{aligned}
\left|\begin{pmatrix}2\\\frac12\end{pmatrix}\right|&=\frac{\sqrt{17}}2,&
\left|\begin{pmatrix}-2\\\frac12\end{pmatrix}\right|&=\frac{\sqrt{17}}2,&
\begin{pmatrix}2\\\frac12\end{pmatrix}
\cdot
\begin{pmatrix}-2\\\frac12\end{pmatrix}
&=-4+\frac14=-\frac{15}4.
\end{aligned}
\\
\begin{aligned}
\cos\alpha&=-\frac{\frac{15}{4}}{\frac{\sqrt{17}}{2}\cdot\frac{\sqrt{17}}2}
=-\frac{15}{17}=-0.88235\\
\alpha&=151.927^\circ.
%28.073
\end{aligned}
\end{gather}
\qedhere
\end{teilaufgaben}
\end{loesung}

