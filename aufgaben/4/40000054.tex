Gesucht ist eine Ebene $\sigma$, die durch die Punkt $A=(1,0,0)$
und $B=(0,1,0)$ geht 
und zu den Punkten $C=(c_x,c_y,c_z)$ und $D=(d_x,d_y,d_z)$
den gleichen Abstand hat.
\begin{teilaufgaben}
\item Welche Vorausetzung an die gegenseitige Lage von $C$ und $D$
muss erfüllt sein, damit die Aufgabe nur endlich viele Lösungen hat?
\item Wieviele Lösungen hat die Aufgabe, wenn die in a) gefundene
Voraussetzung erfüllt ist?
\item Erfüllen die Punkte $(2,0,0)$ und $(0,0,2)$ diese Vorausssetzung,
und wenn ja, welche Ebene(n) erfüllen die Bedingung gleichen Abstandes?
\end{teilaufgaben}

\thema{Hessesche Normalform}

\begin{loesung}
\begin{teilaufgaben}
\item
Wir suchen die Ebene in Hessescher Normalform, und verwenden daher einen
Ansatz der Form
\[
ax+by+cz+d=0.
\]
Die beiden Punkte $A$ und $B$ müssen auf der Ebene liegen,
was auf die Gleichungen
\[
\begin{linsys}{4}
a& & & &   &+&d&=&0\\
 & &b& &   &+&d&=&0
\end{linsys}
\]
führt.
Ausserdem müssen die Abstände zu den beiden Punkten $C$ und $D$ gleich
sein:
\[
\begin{linsys}{7}
ac_x&+&bc_y&+&cc_y&+&d&=&\pm ad_x&\pm&bd_y&\pm&cd_z&\pm&d\\
\end{linsys}
\]
oder
\begin{equation}
\begin{linsys}{4}
(c_x\pm d_x)a&+& (c_y\pm d_y)b&+& (c_z\pm d_z)c&+& v_\pm d&=&0\\
\end{linsys}
\label{40000054:vorzeichen}
\end{equation}
wobei
\[
v_\pm=\begin{cases}
2&\\
0&
\end{cases}
\]
Insgesamt haben wir also das Gleichungsystem
\[
\begin{linsys}{4}
a& & & &   &+&d&=&0\\
 & &b& &   &+&d&=&0\\
(c_x\pm d_x)a&+& (c_y\pm d_y)b&+& (c_z\pm d_z)c&+&v_\pm d&=&0
\end{linsys}
\]
mit drei Gleichungen für vier Unbekannte. Wir schreiben die letzte
Gleichung etwas kompakter mit $\delta_i = c_i\pm d_i$:
\[
\begin{linsys}{4}
a& & & &   &+&d&=&0\\
 & &b& &   &+&d&=&0\\
\delta_x a&+& \delta_y b&+& \delta_z c&+&v_\pm d&=&0
\end{linsys}
\]
Das Gausstableau davon ist 
\begin{align*}
\begin{tabular}{|>{$}c<{$}>{$}c<{$}>{$}c<{$}>{$}c<{$}|>{$}c<{$}|}
\hline
       1&       0&       0&    1&0\\
       0&       1&       0&    1&0\\
\delta_x&\delta_y&\delta_z&v_\pm&0\\
\hline
\end{tabular}
&\rightarrow
\begin{tabular}{|>{$}c<{$}>{$}c<{$}>{$}c<{$}>{$}c<{$}|>{$}c<{$}|}
\hline
       1&       0&       0&1                 &0\\
       0&       1&       0&1                 &0\\
       0&       0&\delta_z&v_\pm-\delta_x-\delta_y&0\\
\hline
\end{tabular}
\end{align*}
Diese Gleichung ist nicht mehr (bis auf Vielfache) eindeutig lösbar, wenn
\begin{align*}
\delta_z&=0\\
\delta_x+\delta_y&=v_\pm
\end{align*}
gilt.
Die Bedingung kann man auch so formulieren:
Zwei Punkte mit Ortsvektoren $\vec c$ und $\vec d$ erlauben unendlich viele
Lösungen, wenn die Differenz $\vec\delta =\vec c-\vec d$ horizontal
ist und ausserdem die $x$- und $y$-Komponenten entgegengesetzt 
gleich gross sind. Das sind genau die Vektoren der Geraden
\[
\left\{\left .
\lambda \begin{pmatrix}1\\-1\\0\end{pmatrix}\,\right| \lambda\in\mathbb R\right\}.
\]
\item
Dazu kommt aber noch die
Bedingung an die Hessesche Normalform, dass $\sqrt{a^2+b^2+c^2}=1$ sein
muss.  Die Koeffizienten sind daher bis auf ein Vorzeichen durch die
drei Gleichungen bereits eindeutig bestimmt.
Somit ist für jede Wahl des Vorzeichens in (\ref{40000054:vorzeichen})
die Ebene eindeutig bestimmt, es gibt also zwei Lösungen.
\item
Das Punktepaar $(2,0,0)$ und $(0,0,2)$ erfüllt die Bedingung, es sollte
also zwei Lösungesebenen geben, die man als Lösungen der Tableaux
\begin{align*}
\begin{tabular}{|>{$}c<{$}>{$}c<{$}>{$}c<{$}>{$}c<{$}|>{$}c<{$}|}
\hline
       1&       0&       0&1&0\\
       0&       1&       0&1&0\\
       0&       0&      -2&2&0\\
\hline
\end{tabular}
&\rightarrow
\begin{tabular}{|>{$}c<{$}>{$}c<{$}>{$}c<{$}>{$}c<{$}|>{$}c<{$}|}
\hline
       1&       0&       0& 1&0\\
       0&       1&       0& 1&0\\
       0&       0&       1&-1&0\\
\hline
\end{tabular}
\\
\begin{tabular}{|>{$}c<{$}>{$}c<{$}>{$}c<{$}>{$}c<{$}|>{$}c<{$}|}
\hline
       1&       0&       0&1&0\\
       0&       1&       0&1&0\\
       0&       0&       2&0&0\\
\hline
\end{tabular}
&\rightarrow
\begin{tabular}{|>{$}c<{$}>{$}c<{$}>{$}c<{$}>{$}c<{$}|>{$}c<{$}|}
\hline
       1&       0&       0& 1&0\\
       0&       1&       0& 1&0\\
       0&       0&       1& 0&0\\
\hline
\end{tabular}
\end{align*}
bestimmen kann.
Zusammen mit der Bedingung $\sqrt{a^2+b^2+c^2}=1$ findet man die Ebenen
\[
\begin{linsys}{4}
\frac1{\sqrt{3}}x&+&\frac1{\sqrt{3}}y&-&\frac1{\sqrt{3}}z&-&\frac1{\sqrt{3}}&=&0
\\
\frac1{\sqrt{2}}x&+&\frac1{\sqrt{2}}y& &                 &-&\frac1{\sqrt{2}}&=&0
\end{linsys}
\qedhere
\]
\end{teilaufgaben}
\end{loesung}
