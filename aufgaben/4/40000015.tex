Betrachten Sie das Signal
\[
f(i)=\sin\frac{i\pi}{N}, \qquad 0\le i<N,
\]
wobei $N=2^k$ eine Zweierpotenz ist.
Im Skript im Anhang "uber die Haar-Wavelets wird gezeigt,
wie man die verschiedenen Haar-Wavelet-Koeffizienten
$$a_{0,0},a_{1,0}, a_{2,0},a_{2,1},\dots ,a_{k,0},\dots, a_{k,2^{k-1}-1}$$
berechnen kann.
\begin{teilaufgaben}
\item
Welcher der Waveletkoeffizienten $a_{k,j}, 0\le j<2^{k-1}$ ist betragsm"assig
der gr"osste?
\item
Berechnen Sie den betragsm"assig gr"ossten Waveletkoeffizienten $a_{k,j}$
f"ur $N=1024$.
\end{teilaufgaben}

\begin{loesung}
Die Vektoren der Waveletbasis $\psi_{k,j}$ haben nur zwei nicht verschwindende
Koeffizienten, die $\pm\frac1{\sqrt{2}}$ sein m"ussen, weil ja $|\psi_{j,k}|=1$
gelten muss. Die Wavelet-Koeffizienten ermittelt man, indem man das Signal
skalar mit den $\psi_{k,j}$ multipliziert. Ein solches Skalarprodukt hat
aber wiederum nur zwei Terme, n"amlich:
\begin{equation}
a_{k,j}=f\cdot \psi_{k,j} = \frac1{\sqrt{2}}(f(2j)-f(2j+1))
\label{betragsmaessig}
\end{equation}
\begin{teilaufgaben}
\item der betragsm"assig gr"osste Wert tritt also dort auf, wo die Differenz
(\ref{betragsmaessig}) am gr"ossten ist. Beim Signal handelt es sich um
einen positiven Sinus-Bogen, die Differenz ist dort am gr"ossten, wo die
Steigung am gr"ossten ist, und dies ist ganz am Anfang des Intervals bei
$j=0$. Also ist $a_{k,0}$ der betragsm"assig gr"osste Koeffizient.
\item Der Wert des Koeffizienten ist
\begin{align*}
a_{10,0}&=\frac1{\sqrt{2}}(f(0)-f(1))=\frac1{\sqrt{2}}\biggl(0-\sin\frac{\pi}{1024}\biggr)
=-\frac1{\sqrt{2}}\sin\frac{\pi}{1024}\\
&=-0.00216937303148037122
\qedhere
\end{align*}
\end{teilaufgaben}
\end{loesung}

