In der Vorlesung wurde gezeigt, dass die Spiegelung an einer Ebene
mit der Normalen $\vec n$ mit $|\vec n|=1$ durch die Matrix
\[
S=I-2\vec n \transpose{\vec{n}}
\]
gegeben ist.
\begin{teilaufgaben}
\item Kontrollieren Sie durch nachrechnen, dass zweimalige Spiegelung einen
Vektor nicht ändert.
\item Ist die Matrix $S$ symmetrisch?
\item Ist $S$ orthogonal?
\end{teilaufgaben}

\thema{Spiegelung}

\begin{loesung}
\begin{teilaufgaben}
\item
Wir müssen $S^2$ berechnen:
\begin{align*}
S^2
&=
(I-2\vec{n}\transpose{\vec{n}}) (I-2\vec{n}\transpose{\vec{n}})
=
I
-
4\vec{n}\transpose{\vec{n}}
+
4\vec{n}\underbrace{\transpose{\vec{n}}\vec{n}}_{\displaystyle=1}
\transpose{\vec{n}}
=
I-4\vec{n}\transpose{\vec{n}} +4\vec{n} \transpose{\vec{n}}
=
I.
\end{align*}
\item
Die Matrix ist symmetrisch, wenn $\transpose{S}=S$ ist.
Also
\[
\transpose{S}
=
\transpose{(I-2\vec{n}\transpose{\vec{n}})}
=
\transpose{I} - 2\transpose{(\transpose{\vec{n}})} \transpose{\vec{n}}
=
I-2\vec{n}\transpose{\vec{n}}
=
S.
\]
Die Matrix $S$ ist daher symmetrisch.
\item
Eine Matrix ist orthogonal, wenn $\transpose{A}A=I$ ist.
Da $S$ symmetrisch ist, ist $\transpose{S}=S$ und damit
$\transpose{S}S=SS=I$ nach a).
\end{teilaufgaben}
\end{loesung}

