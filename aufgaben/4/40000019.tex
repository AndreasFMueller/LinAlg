Gegeben sind die Polynome
\[
p_1(x)=(x+1)^2,\qquad
p_2(x)=(x-1)^2,\qquad
p_3(x)=(x-1)(x+1)
\]
\begin{teilaufgaben}
\item
Kann das Polynom
\[
p(x)=x^2+2x+3
\]
als Linearkombination der Polynome $p_i$
geschrieben werden?
\item Kann jedes beliebige Polynom zweiten Grades als Linearkombination der
Polynome $p_i$ geschrieben werden?
\end{teilaufgaben}

\begin{loesung}
\begin{teilaufgaben}
\item
Ausgeschrieben lauten die Polynome
\begin{align*}
p_1(x)&=x^2+2x+1
p_2(x)&=x^2-2x+1
p_3(x)&=x^2-1
\end{align*}
Der Koeffizientenvergleich f"ur eine Linearkombination
$\lambda_1p_1+\lambda_2p_2+\lambda_3p_3=p$ f"uhrt auf das
Gleichungssystem
\[
\begin{linsys}{3}
\lambda_1&+& \lambda_2&+&\lambda_3&=&1\\
2\lambda_1&-&2\lambda_2&+&         &=&2\\
\lambda_1& & \lambda_2&-&\lambda_3&=&3
\end{linsys}
\]
Mit dem Gaussalgorithmus ergibt sich folgende L"osung
\begin{align*}
\begin{tabular}{|ccc|c|}
\hline
1&1&1&1\\
2&$-2$&0&2\\
1&1&$-1$&3\\
\hline
\end{tabular}
&\rightarrow
\begin{tabular}{|ccc|c|}
\hline
1&1&1&1\\
0&$-4$&$-2$&0\\
0&0&$-2$&2\\
\hline
\end{tabular}
\rightarrow
\begin{tabular}{|ccc|c|}
\hline
1&1&1&1\\
0&1&$\frac12$&0\\
0&0&1&$-1$\\
\hline
\end{tabular}
\\
&\rightarrow
\begin{tabular}{|ccc|c|}
\hline
1&1&0&2\\
0&1&0&$\frac12$\\
0&0&1&$-1$\\
\hline
\end{tabular}
\rightarrow
\begin{tabular}{|ccc|c|}
\hline
1&0&0&$\frac32$\\
0&1&0&$\frac12$\\
0&0&1&$-1$\\
\hline
\end{tabular}
\end{align*}
Daraus liest man ab, dass
\[
x^2+2x+3=\frac32(x+1)^2+\frac12(x-1)^2-(x+1)(x-1),
\]
was sich auch durch algebraisches Nachrechnen best"atigen l"asst.
\item
Bei der Durchf"uhrung des Gauss-Algorithmus hat sich gezeigt, dass
die drei Polynome linear unabh"angig sind, da die Matrix reglu"ar
ist. Sie bilden also eine Basis des Vektorraums der Polynome zweiten
Grades, insbesondere kann jedes Polynom zweiten Grades als Linearkombination
der $p_i$ geschrieben werden.
\end{teilaufgaben}
\end{loesung}

