\begin{teilaufgaben}
\item Bestimmen Sie die Parameterdarstellung der Ebene durch die
Punkte $(1,0,0)$, $(0,1,0)$ und $(0,0,1)$.
\item Bestimmen Sie die Normale der Ebene.
\item Wie gross ist der Abstand des Punktes $(1291, 1848, 1918)$ von
der Ebene?
\end{teilaufgaben}

\thema{Ebene}
\thema{Hessesche Normalform}

\begin{loesung}
\begin{teilaufgaben}
\item Als Parameterdarstellung kann man
\[
\begin{pmatrix}1\\0\\0\end{pmatrix}
+s\begin{pmatrix}-1\\1\\0\end{pmatrix}
+t\begin{pmatrix}-1\\0\\1\end{pmatrix}
\]
verwenden.
\item  Der Normalenvektor $\vec n$ erfüllt die Gleichungen
\[
\begin{linsys}{3}
-n_1&+&n_2& &    &=&0\\
-n_1& &   &+&n_3 &=&0
\end{linsys}
\]
Man kann zwar die Lösung
\[
\vec n=
\begin{pmatrix}1\\1\\1\end{pmatrix}
\]
sofort ablesen, wir zeigen hier trotzdem auch noch das Gauss-Tableau
\[
\begin{tabular}{|ccc|c|}
\hline
$-1$&1&0&0\\
$-1$&0&1&0\\
\hline
\end{tabular}
\rightarrow
\begin{tabular}{|ccc|c|}
\hline
1&$-1$&0&0\\
0&$-1$&1&0\\
\hline
\end{tabular}
\rightarrow
\begin{tabular}{|ccc|c|}
\hline
1& 0&$-1$&0\\
0& 1&$-1$&0\\
\hline
\end{tabular}
\]
Es besagt $x=y=z$, alle drei Komponenten sind also gleich,
wie bereits festgestellt.
\item Der Abstand eines Punktes mit Ortsvektor $q$ kann mit Hilfe von
$\vec n$ berechnet werden, wenn $\vec n$ Einheitslänge hat. Der bisher
gefundene Vektor erfüllt diese Bedingung nicht, aber
\[
\vec n^0=\frac1{\sqrt{3}}\begin{pmatrix}1\\1\\1\end{pmatrix}.
\]
hat Länge $1$.
Die Entfernung ist dann
\[
d=\left(\vec q- \begin{pmatrix}1\\0\\0\end{pmatrix}\right)\cdot\vec n^0
=
\frac1{\sqrt{3}}\begin{pmatrix}1290\\1848\\1918\end{pmatrix}
\cdot
\begin{pmatrix}1\\1\\1\end{pmatrix}
=\frac1{\sqrt{3}}(1290+1848+1918)=\frac{5056}{\sqrt{3}}.
\qedhere
\]
\end{teilaufgaben}
\end{loesung}

