Die Ebene $\sigma$ hat die Parameterdarstellung
\[
\vec{p}
=
\vec{p}_0
+
t\vec{u}
+
s\vec{v}
=
\begin{pmatrix}
3\\2\\5
\end{pmatrix}
+
t
\begin{pmatrix*}[r]
3\\-1\\-1
\end{pmatrix*}
+
s
\begin{pmatrix*}[r]
0\\-4\\2
\end{pmatrix*}.
\]
Wählen Sie von den Vektoren
\[
\vec{n}_1
=
\begin{pmatrix*}[r]
1\\-1\\2
\end{pmatrix*},
\qquad
\vec{n}_2
=
\begin{pmatrix*}[r]
-1\\1\\2
\end{pmatrix*}
\qquad\text{oder}\qquad
\vec{n}_3
=
\begin{pmatrix*}[r]
1\\1\\2
\end{pmatrix*}
\]
den geeigneten als Normalenvektor
und konstruieren Sie damit die Normalenform der Ebene $\sigma$.

\begin{loesung}
Der Vektor muss so gewählt werden, dass er auf beiden Richtungsvektoren
senkrecht steht.
Dies ist nur für den dritten Vektor der Fall:
\[
\vec{n}_3\cdot\vec{u}
=
3-1-2
=
0
\qquad\text{und}\qquad
\vec{n}_3\cdot{v}
=
0-4+4
=
0.
\]
Daher ist die Normalenform
\[
0
=
\vec{n}_3\cdot(\vec{p}-\vec{p}_0)
=
\vec{n}_3\cdot\vec{p} - 15
=
x+y+2z-15.
\qedhere
\]
\end{loesung}

