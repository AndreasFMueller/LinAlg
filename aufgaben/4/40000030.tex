In Zaouit Massa in Marokko auf $30^\circ$ n"ordlicher Breite steht abends
um 20:00 Uhr ein Stern im Zenit (genau "uber dem Beobachter). Wann geht
dieser Stern unter?

\begin{loesung}
Es ist zu berechnen, welcher Drehwinkel der Himmelskugel dazu f"uhrt,
dass der Stern auf Horizonth"ohe zu liegen kommt. Zu Beginn hat der
Stern in einem Koordinatensystem mit vertikaler $z$-Achse den
Ortsvektor
\[
\vec s=\begin{pmatrix}0\\0\\1\end{pmatrix}=\vec e_3.
\]
Wir legen die $x$-Achse des Koordinatensystems in Richtung Osten.

Die Drehung der Himmelskugel l"asst sich aber viel leichter in einem
Koordinatensystem beschreiben, welches eine Gerade parallel zur Erdachse
als $z$-Achse verwendet. Wir bezeichnen Vektoren in diesem Koordinatensystem
mit Strichen. Es ist gegen"uber dem urspr"unglichen Koordinatensystem
um einen Winkel $\alpha=60^\circ$ verdreht, Drehachse ist die $x$-Achse.
Die Umrechnung eines Vektors $\vec v$ in dieses Koordinatensystem
erfolgt mit Hilfe einer Drehmatrix.
Die Matrix
\[
T=D_{x,\alpha}=\begin{pmatrix}
1&          0&          0\\
0& \cos\alpha&-\sin\alpha\\
0& \sin\alpha& \cos\alpha
\end{pmatrix}
=
\begin{pmatrix}
1& 0&0\\
0&\frac12&-\frac{\sqrt{3}}2\\
0&\frac{\sqrt{3}}2&\frac12\\
\end{pmatrix}
\]
transformiert den Vektor $\vec e_3$ in den Vektor
\[
\vec v'=T\vec v=
\begin{pmatrix}
0\\-\frac{\sqrt{3}}2\\\frac12 
\end{pmatrix}
\]
die Darstellung der Sternposition im gestrichenen Koordinatensystem.

Jetzt erfolgt die Drehung um den unbekannten Winkel $\varphi$ um
die $z'$-Achse im gestrichenen Koordinatensystem, dies kann durch die
Drehmatrix
\[
D_{z',\varphi}
=
\begin{pmatrix}
\cos\varphi&-\sin\varphi&0\\
\sin\varphi& \cos\varphi&0\\
0&0&1
\end{pmatrix}
\]
Nach der Drehung hat der Stern im gestrichenen Koordinatensystem die
Koordinaten:
\[
D_{z',\varphi}\vec v'
=
\begin{pmatrix}
\cos\varphi&-\sin\varphi&0\\
\sin\varphi& \cos\varphi&0\\
0&0&1
\end{pmatrix}
\begin{pmatrix}
0\\-\frac{\sqrt{3}}2\\\frac12 
\end{pmatrix}
=
\begin{pmatrix}
\frac{\sqrt{3}}2\sin\varphi\\
-\frac{\sqrt{3}}2\cos\varphi\\
\frac12
\end{pmatrix}.
\]

Ob der Stern sich genau auf Horizonth"ohe befindet, kann nur im ungestrichen
Koordinatensystem entschieden werden.
Man muss daher wieder mit $T^{-1}$ zur"ucktransformieren:
\[
T^{-1}D_{z',\varphi}Te_3 = 
\begin{pmatrix}
1& 0&0\\
0&\frac12&\frac{\sqrt{3}}2\\
0&-\frac{\sqrt{3}}2&\frac12\\
\end{pmatrix}
\begin{pmatrix}
\frac{\sqrt{3}}2\sin\varphi\\
-\frac{\sqrt{3}}2\cos\varphi\\
\frac12
\end{pmatrix}.
=
\begin{pmatrix}
\frac{\sqrt{3}}2\sin\varphi\\
-\frac{\sqrt{3}}4\cos\varphi+\frac{\sqrt{3}}4\\
\frac{\sqrt{3}}4\cos\varphi+\frac14
\end{pmatrix}
\]
Jetzt muss $\varphi$ so gew"ahlt werden, dass die $z$-Komponente dieses
Vektors verschwindet:
\begin{align*}
\frac{3}4\cos\varphi+\frac14&=0\\
\cos\varphi&=-\frac1{3}\\
\varphi&=\pm 109.47^\circ.
\end{align*}
Die Antwort ist aber in Stunden gesucht, wir m"ussen also den Winkel
noch mit dem Faktor $\frac{24}{360}=\frac1{15}$ multiplizieren:
\[
\varphi=7.298\text{h}=7\text{h} 17.8\text{min}.
\]
Der Stern wird also um 03:18 Uhr untergehen.
\end{loesung}

\begin{diskussion}
Die Berechnung der Untergangszeit ist nicht ganz korrekt. Die Erde dreht
sich n"amlich nicht einmal in 24 Stunden, sondern einmal in 23.9345 Stunden,
dem siderischen Tag.
Der korrekte Umrechungsfaktor ist daher $\frac{23.9245}{360}$, damit
wird die Untergangszeit 03:16:30.

Ausserdem bewirkt das Ph"anomen der atmosph"arischen Refraktion, dass
Sterne in Horizontn"ahe h"oher erscheinen, als sie tats"achlich
sind, ein Stern ist also noch sichtbar, wenn die Gerade zum Stern bereits
unter dem Horizont liegt.
Sterne gehen also sp"ater unter, also die Geometrie alleine vorhersagen
w"urde.
\end{diskussion}

\begin{bewertung}
Richtung der Drehachse (\textbf{A}) 1 Punkt,
Drehmatrix (\textbf{D}) 1 Punkt,
Umrechnung zwischen den Koordinatensystemen (\textbf{U}) 2 Punkt,
Ermittelung Drehwinkel (\textbf{W}) 1 Punkt,
Berechnung der Untergangszeit (\textbf{Z}) 1 Punkt.
\end{bewertung}

