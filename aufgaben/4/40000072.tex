Im Raum sind die sechs Punkte
\begin{align*}
P_1&=(\phantom{-}1,0,\phantom{-}t),&
	P_2&=(\textstyle-\frac12,\textstyle\frac{\sqrt{3}}2,\phantom{-}t),&
		P_3&=(\textstyle-\frac12,-\textstyle\frac{\sqrt{3}}2, \phantom{-}t),\\
P_4&=(-1,0,-t),&
	P_5&=(\textstyle\phantom{-}\frac12,\textstyle\frac{\sqrt{3}}2,-t),&
		P_6&=(\textstyle\phantom{-}\frac12,\textstyle-\frac{\sqrt{3}}2,-t)\\
\end{align*}
gegeben.
\begin{teilaufgaben}
\item
Wie muss $t$ gewählt werden, damit die Strecke $P_1P_2$ gleich lang ist
wie die Strecke $P_1P_5$?
\item
Welche Winkel hat das Dreieck $\triangle P_2P_3P_4$?
\item
Die Punkte $P_1,\dots,P_6$ sind die Ecken eines wohlbekannten Polyeders.
Wie heisst dieses Polyeder?
\end{teilaufgaben}

\thema{Abstand}
\thema{Zwischenwinkel}

\begin{loesung}
\begin{teilaufgaben}
\item Die Bedingung an $t$ lautet:
\begin{align*}
\overline{P_1P_2}&=\overline{P_1P_5}\\
\left(\frac32\right)^2+\left(\frac{\sqrt{3}}2\right)^2
	&=\left(\frac12\right)^2+\left(\frac{\sqrt{3}}2\right)^2+(2t)^2\\
\frac94+\frac34&=\frac14+\frac34+4t^2\\
t^2&=\frac12&\Rightarrow\qquad t=\frac1{\sqrt{2}}=\frac{\sqrt{2}}2.
\end{align*}
\item Mit $t$ können jetzt auch die Winkel im Dreieck $\triangle P_2P_3P_4$
ausgerechnet werden. Da das Dreieck bezüglich der $x$-$z$-Ebene
symmetrisch ist, und daher die beiden Seiten $P_2P_4$ und $P_3P_4$
gleich lang sind, genügt es, einen der Winkel zu berechnen.
Wir bestimmen den Winkel beim Punkt $P_4$ mit Hilfe des Skalarproduktes.
Dazu brauchen wir die Länge der Seiten $P_2P_4$ und $P_3P_4$, die wegen der
eben erwähnten Symmetrie gleich lang sind:
\begin{align*}
\overline{P_2P_4}=\overline{P_3P_4}
&=|\overrightarrow{P_3P4}|
=
\left|
\begin{pmatrix}
-1\\0\\-t
\end{pmatrix}
-
\begin{pmatrix}
-\frac12\\-\frac{\sqrt{3}}2\\t
\end{pmatrix}
\right|
=
\left|
\begin{pmatrix}
-\frac12\\\frac{\sqrt{3}}2\\-2t
\end{pmatrix}
\right|
=\sqrt{\frac14+\frac34+4t^2}=\sqrt{3}.
\end{align*}
Damit kann man jetzt den Winkel $\angle P_2P_4P_3$ mit der Zwischenwinkelformel
berechnen:
\begin{align*}
\cos\angle P_2P_4P_3
&=\frac{\overrightarrow{P_4P_2}\cdot\overrightarrow{P_4P_3}}{\overline{P_4P_2}\cdot\overline{P_4P_3}}
=\frac13\left(
\begin{pmatrix}
\frac12 \\ \frac{\sqrt{3}}2\\2t
\end{pmatrix}
\cdot
\begin{pmatrix}
\frac12\\-\frac{\sqrt{3}}2\\2t
\end{pmatrix}
\right)
=\frac13\left(
\frac14-\frac34+4t^2
\right)
=
\frac13\left(-\frac12+2\right)=\frac12\\
\angle P_2P_4P_3
=60^\circ
\end{align*}
Insbesondere ist das Dreieck $P_2P_3P_4$ ein gleichseitiges Dreieck.
\item Aus Symmetriegründen sind auch alle anderen Dreieck gleichseitig,
die Punkte bilden also die Ecken eines Oktaeders.
\qedhere
\end{teilaufgaben}
\end{loesung}

\begin{bewertung}
\begin{teilaufgaben}
\item
Abstandsbedingung, Gleichung für $t$ (\textbf{G}) 1 Punkt,
Berechung von $t$ (\textbf{T}) 1 Punkt,
\item
Zwischenwinkelformel (\textbf{Z}) 1 Punkt,
Winkel (\textbf{W}) 1 Punkt,
\item
Symmetrie (\textbf{S}) 1 Punkt,
Oktaeder (\textbf{O}) 1 Punkt.
\end{teilaufgaben}
\end{bewertung}

