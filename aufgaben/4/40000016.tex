Sei $A$ die Matrix
\[
A=\begin{pmatrix}
\frac1{\sqrt{3}}&\frac1{\sqrt{3}}&\frac1{\sqrt{3}}\\
\frac1{\sqrt{6}}&-\frac2{\sqrt{6}}&\frac1{\sqrt{6}}\\
-\frac1{\sqrt{2}}&0&\frac1{\sqrt{2}}
\end{pmatrix}
\]
\begin{teilaufgaben}
\item Ist $A$ orthogonal?
\item Zeigen Sie: $A$ ist keine Drehmatrix.
\item Finden Sie die Matrix $S$ einer Spiegelung an der
Ebene aufgespannt von der $z$-Achse und der Winkelhalbierenden
von $x$- und $y$-Achse.
\item Berechnen Sie $AS$
\item Zeigen Sie: $AS$ ist eine Drehmatrix
\item Bestimmen Sie den Drehwinkel von $AS$.
\end{teilaufgaben}

\thema{orthogonale Matrix}
\thema{Drehmatrix}
\thema{Drehwinkel}
\thema{Spiegelung}

\begin{loesung}
\begin{teilaufgaben}
\item Die Zeilen von $A$ sind Einheitsvektoren und stehen
senkrecht aufeinander. Dies ist nur für die ersten beiden
Zeilen nicht ganz offensichtlich:
\[
\frac1{\sqrt{3}}\cdot\frac1{\sqrt{6}}
+
\frac1{\sqrt{3}}\cdot\frac{-2}{\sqrt{6}}
+
\frac1{\sqrt{3}}\cdot\frac1{\sqrt{6}}
=\frac{1\cdot 1+1\cdot(-2)+1\cdot 1}{\sqrt{3}\cdot\sqrt{6}}=0,
\]
also ist die Matrix orthogonal.
\item $A$ ist keine Drehmatrix, da die Determinante
$\det A=-1$ ist.
\item Die gesuchte Matrix $S$ vertauscht $x$- und $y$-Achse. Da in den
Spalten der Matrix die Bilder der Basisvektoren stehen, muss sein
\[
S=\begin{pmatrix}0&1&0\\1&0&0\\0&0&1\end{pmatrix}.
\]
Selbstverständlich ist dies eine orthogonale Matrix, da bei einer
Spiegelung Längen und Winkel erhalten bleiben.
\item
\[
AS=
\begin{pmatrix}
\frac1{\sqrt{3}}&\frac1{\sqrt{3}}&\frac1{\sqrt{3}}\\
\frac1{\sqrt{6}}&-\frac2{\sqrt{6}}&\frac1{\sqrt{6}}\\
-\frac1{\sqrt{2}}&0&\frac1{\sqrt{2}}
\end{pmatrix}
\begin{pmatrix}0&1&0\\1&0&0\\0&0&1\end{pmatrix}
=\begin{pmatrix}
\frac1{\sqrt{3}}&\frac1{\sqrt{3}}&\frac1{\sqrt{3}}\\
-\frac2{\sqrt{6}}&\frac1{\sqrt{6}}&\frac1{\sqrt{6}}\\
0&-\frac1{\sqrt{2}}&\frac1{\sqrt{2}}
\end{pmatrix}
\]
also die Matrix $A$ in der die ersten beiden Spalten vertauscht
wurden.
\item
$AS$ ist also auch wieder orthogonal, es bleibt zu überprüfen,
ob die Determinante den richtigen Wert hat:
\[
\det(AS)=\det(A)\det(S)=(-1)\cdot(-1)=1
\]
also ist $AS$ ein Drehmatrix. Alternativ kann man auch argumentieren,
dass in $AS$ gegenüber $A$ zwei Spalten vertauscht wurden, was das
Vorzeichen der Determinante umkehrt. Da $\det A=-1$ war, muss also
$AS$ orthogonal sein.
\item
Der Drehwinkel kann mit der Spur bestimmt
werden:
\begin{align*}
\cos \alpha&=\frac{\operatorname{Spur}(A)-1}2=\frac{\frac1{\sqrt{3}}+\frac1{\sqrt{6}}+\frac1{\sqrt{2}}-1}2
=\frac{\frac{\sqrt{2}+1+\sqrt{3}}{\sqrt{6}}-1}2
\\
&=\frac{\sqrt{2}+1+\sqrt{3}-\sqrt{6}}{2\sqrt{6}}\simeq  0.34635267
\\
\alpha&=\pm 69.7356^\circ
\qedhere
\end{align*}
\end{teilaufgaben}
\end{loesung}

