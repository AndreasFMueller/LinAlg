Die Punkte
\[
A=( 1, 1, 1),\quad
B=(-1,-1, 1),\quad
C=( 1,-1,-1)
\quad\text{und}\quad
D=(-1, 1,-1)
\]
sind die Ecken eines Tetraeders.
\begin{center}
\begin{tikzpicture}[scale=1]
\node at (0,0) {\includeagraphics[width=6cm]{tetraeder.jpg}};
\node at (3,0) {$x$};
\node at (0.2,3.1) {$z$};
\node at (-0.1,-0.4) {$y$};
\node at (2.4,2.4) {$A$};
\node at (-2.9,2.3) {$B$};
\node at (2.4,-2.7) {$C$};
\node at (-1.1,-1) {$D$};
\end{tikzpicture}
\end{center}
\begin{teilaufgaben}
\item
Finden Sie eine Drehmatrix, die den Punkt $A$ fest lässt und den
Punkt $B$ in den Punkt $C$ überführt.
\item
Finden Sie den Drehwinkel, der zu dieser Drehmatrix gehört.
\end{teilaufgaben}

\thema{Abbildungsmatrix}
\thema{Drehmatrix}
\thema{Drehwinkel}

\begin{hinweis}
Die gesuchte Abbildungsmatrix $R$ führt auch $C$ in $D$ über und $D$ in $B$.
% Schreiben Sie die Bedingung, dass $R$ die genannten Punkte in andere
% Punkte überführt, als Matrizengleichung%
% {} $P' = RP$%
% .
\end{hinweis}

\begin{loesung}
\begin{teilaufgaben}
\item 
Der Punkt $A$ wird durch die Drehmatrix $R$ wieder auf $A$ abgebildet, woraus 
wir schiessen können, dass $A$ auf der Drehachse liegt, bzw. dass der 
Ortsvektor
\[
  \overrightarrow{0A}  = \vec{a}= \begin{pmatrix}1\\1\\1\end{pmatrix}.
\]
der Drehachse entspricht. Weiter wissen wir, dass die Punkte $B$, $C$ und $D$
jeweils aufeinander abgebildet werden:
\[
  B \rightarrow C,\quad
  C \rightarrow D,\quad
  D \rightarrow B.
\]
Wenn die Drehmatrix $R$ also dreimal angewendet würde, wären alle Punkte
wieder an ihrem ursprünglichen Platz und man hätte insgesamt um $360^\circ$ gedreht.
Der in b) gesuchte Drehwinkel ist damit
\[
  \alpha = \dfrac{360^\circ}{3} = 120^\circ.
\]
Eine Methode um nun auch die Drehmatrix $R$ zu finden ist, zu untersuchen,
auf welche Bildvektoren die Standardbasisvektoren $\vec e_1$, $\vec e_2$ und $\vec e_3$
abgebildet werden. 
Bei einer Drehung um $120^\circ$ um die Achse $\vec a = \begin{pmatrix}1&1&1\end{pmatrix}^t$
werden diese wie folgt abgebildet:
\[
\begin{aligned}
\vec e_1&\mapsto \vec e_2 = \begin{pmatrix}0\\1\\0\end{pmatrix},
&
\vec e_2&\mapsto \vec e_3 = \begin{pmatrix}0\\0\\1\end{pmatrix}
&&\text{und}
&
\vec e_3&\mapsto \vec e_1 = \begin{pmatrix}1\\0\\0\end{pmatrix}.
\end{aligned}
\]
Die gesuchte Drehmatrix ist daher
\[
R
=
\begin{pmatrix}
0&0& 1\\
1&0& 0\\
0&1& 0\\
\end{pmatrix}.
\]
Alternativ hätte die Aufgabe aber auch wie folgt gelöst werden können:\\
Gesucht ist eine Matrix $R$ mit den Eigenschaften:
\begin{equation}
R\vec{a}=\vec{a},\quad
R\vec{b}=\vec{c}\quad\text{und}\quad
R\vec{c}=\vec{d}.
\label{40000016:vektorgleichungen}
\end{equation}
Schreiben wir die Vektoren $\vec{a}$, $\vec{b}$ und $\vec{c}$ in die
Matrix $P$ und die Vektoren $\vec{a}$, $\vec{c}$ und $\vec{d}$ in
die Matrix $P'$, also
\[
P
=
\begin{pmatrix}
 1&-1& 1\\
 1&-1&-1\\
 1& 1&-1
\end{pmatrix}
\qquad\text{und}\qquad
P'
=
\begin{pmatrix}
1& 1&-1\\
1&-1& 1\\
1&-1&-1
\end{pmatrix},
\]
dann lassen sich die
Gleichungen~\eqref{40000016:vektorgleichungen} zusammenfassen in eine
Matrixgleichung
\[
RP=P'
\qquad\Rightarrow\qquad R=P'P^{-1}.
\]
Wir bestimmen daher zunächst die inverse Matrix $P^{-1}$
mit Hilfe des Gauss-Algorithmus:
\begin{align*}
\begin{tabular}{|>{$}c<{$}>{$}c<{$}>{$}c<{$}|>{$}c<{$}>{$}c<{$}>{$}c<{$}|}
\hline
 1&-1& 1& 1& 0& 0\\
 1&-1&-1& 0& 1& 0\\
 1& 1&-1& 0& 0& 1\\
\hline
\end{tabular}
&
\rightarrow
\begin{tabular}{|>{$}c<{$}>{$}c<{$}>{$}c<{$}|>{$}c<{$}>{$}c<{$}>{$}c<{$}|}
\hline
 1&-1& 1& 1& 0& 0\\
 0& 0&-2&-1& 1& 0\\
 0& 2&-2&-1& 0& 1\\
\hline
\end{tabular}
\rightarrow
\begin{tabular}{|>{$}c<{$}>{$}c<{$}>{$}c<{$}|>{$}c<{$}>{$}c<{$}>{$}c<{$}|}
\hline
 1&-1& 1&      1 & 0& 0      \\
 0& 1&-1&-\frac12& 0& \frac12\\
 0& 0&-2&-     1 & 1& 0      \\
\hline
\end{tabular}
\\
&
\rightarrow
\begin{tabular}{|>{$}c<{$}>{$}c<{$}>{$}c<{$}|>{$}c<{$}>{$}c<{$}>{$}c<{$}|}
\hline
 1&-1& 0& \frac12& \frac12& 0      \\
 0& 1& 0&      0 &-\frac12& \frac12\\
 0& 0& 1& \frac12&-\frac12& 0      \\
\hline
\end{tabular}
\rightarrow
\begin{tabular}{|>{$}c<{$}>{$}c<{$}>{$}c<{$}|>{$}c<{$}>{$}c<{$}>{$}c<{$}|}
\hline
 1& 0& 0& \frac12&      0 & \frac12\\
 0& 1& 0&      0 &-\frac12& \frac12\\
 0& 0& 1& \frac12&-\frac12& 0      \\
\hline
\end{tabular}
\end{align*}
Damit kann jetzt auch $R$ berechnet werden:
\begin{align*}
R
=
P'P^{-1}
&=
\begin{pmatrix}
1& 1&-1\\
1&-1& 1\\
1&-1&-1
\end{pmatrix}
\begin{pmatrix}
 \frac12&      0 & \frac12\\
      0 &-\frac12& \frac12\\
 \frac12&-\frac12& 0
\end{pmatrix}
=
\begin{pmatrix}
0&0&1\\
1&0&0\\
0&1&0
\end{pmatrix}.
\end{align*}
\item
Der Drehwinkel kann mit Hilfe der Spurformel ermittelt werden:
\[
\cos \alpha
=
\frac{\operatorname{Spur}R -1 }2
=
-\frac12
\qquad\Rightarrow\qquad
\alpha = 120^\circ.
\qedhere
\]
\end{teilaufgaben}
\end{loesung}

\begin{bewertung}
Bestimmung der Matrizen $P$ ($\textbf{P}$) und $P'$ ($\textbf{P'}$)
je 1 Punkt,
Berechnung der Inversen $P^{-1}$  ({\bf I}) 2 Punkte,
Berechnung des Produktes $P'P^{-1}$ ({\bf P}) 1 Punkt,
Anwendung der Spurformel und korrekter Drehwinkel ({\bf D}) 1 Punkt.
\end{bewertung}

