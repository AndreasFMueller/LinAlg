Ein sogenanntes \emph{autoregressives Modell} ist eine Vorhersagemethode
für zeitabhängige Daten, in der jeder Datenpunkt $x_1,\dots,x_n$
linear von den letzten $k$ Datenpunkten abhängt.
Es besagt dass 
\begin{equation}
x_t
=
\sum_{i=1}^k
\vartheta_ix_{t-i}
=
\vartheta_1x_{t-1}
+
\vartheta_2x_{t-2}
+
\dots
+
\vartheta_kx_{t-k}
\label{60000043:reg}
\end{equation}
gilt für jedes $t$ mit $k < t \le n$.
Finden Sie die Koeffizienten ${\color{darkred}\vartheta_i}$ eines
autoregressiven Modells mit $k=2$ für die Daten
\ainput{ar.tex}
\begin{equation}
x_t
=
\xwerte,
\qquad t=1,\dots,5.
\label{60000043:daten}
\end{equation}

\begin{loesung}
Die Gleichungen~\eqref{60000043:reg} sind $n-k$ Gleichungen
für die $k$ Unbekannten
${\color{darkred}\vartheta_1},\dots,{\color{darkred}\vartheta_k}$
\[
\renewcommand{\arraycolsep}{3pt}
\begin{array}{rcrcccrcrcr}
x_k
{\color{darkred}\vartheta_1}
&+&
x_{k-1}
{\color{darkred}\vartheta_2}
&+&\dots&+&
x_2
{\color{darkred}\vartheta_{k-1}}
&+&
x_1
{\color{darkred}\vartheta_k}
&=&
x_{k+1}
\\
x_{k+1}
{\color{darkred}\vartheta_1}
&+&
x_{k}
{\color{darkred}\vartheta_2}
&+&\dots&+&
x_3
{\color{darkred}\vartheta_{k-1}}
&+&
x_2
{\color{darkred}\vartheta_k}
&=&
x_{k+2}
\\
x_{k+2}
{\color{darkred}\vartheta_1}
&+&
x_{k+1}
{\color{darkred}\vartheta_2}
&+&\dots&+&
x_4
{\color{darkred}\vartheta_{k-1}}
&+&
x_3
{\color{darkred}\vartheta_k}
&=&
x_{k+3}
\\
&& && && && &\vdots&
\\
x_{n-2}
{\color{darkred}\vartheta_1}
&+&
x_{n-3}
{\color{darkred}\vartheta_2}
&+&\dots&+&
x_{n-k}
{\color{darkred}\vartheta_{k-1}}
&+&
x_{n-k-1}
{\color{darkred}\vartheta_k}
&=&
x_{n-1}
\\
x_{n-1}
{\color{darkred}\vartheta_1}
&+&
x_{n-2}
{\color{darkred}\vartheta_2}
&+&\dots&+&
x_{n-k+1}
{\color{darkred}\vartheta_{k-1}}
&+&
x_{n-k}
{\color{darkred}\vartheta_k}
&=&
x_n
\rlap{.}
\end{array}
\]
Dieses Gleichungssystem ist überbestimmt, wenn $n-k>k$ oder $n>2k$ ist.
Die Koeffizientenmatrix in der Vektor der rechten Seiten ist
\[
A
=
\begin{pmatrix*}
x_k     &x_{k-1} &\dots  &x_2       &x_1     \\
x_{k+1} &x_{k  } &\dots  &x_3       &x_2     \\
\vdots  &\vdots  &\ddots &\vdots    &\vdots  \\
x_{n-2} &x_{n-3} &\dots  &x_{n-k} &x_{n-k-1} \\
x_{n-1} &x_{n-2} &\dots  &x_{n-k+1} &x_{n-k}
\end{pmatrix*}
\qquad\text{und}\qquad
b
=
\begin{pmatrix*}
x_{k+1}\\
x_{k+2}\\
\vdots \\
x_{n-1}\\
x_{n}
\end{pmatrix*}.
\]
Die Koeffizienten lassen sich daher mit dem Gleichungssystem
$\transpose{A}A{\color{darkred}\vartheta} = \transpose{A}b$
bestimmen.
Darin gilt
\[
\renewcommand{\arraycolsep}{2pt}
\transpose{A}A
=
\begin{pmatrix}
\displaystyle         \sum_{i=1}^{n-k} x_{k-1+i} x_{k-1+i}
	&\displaystyle\sum_{i=1}^{n-k} x_{k-1+i} x_{k-2+i}
	&\dots
	&\displaystyle\sum_{i=1}^{n-k} x_{k-1+i} x_{  1+i}
	&\displaystyle\sum_{i=1}^{n-k} x_{k-1+i} x_{    i}
\\
\displaystyle         \sum_{i=1}^{n-k} x_{k-2+i} x_{k-1+i}
	&\displaystyle\sum_{i=1}^{n-k} x_{k-2+i} x_{k-2+i}
	&\dots
	&\displaystyle\sum_{i=1}^{n-k} x_{k-2+i} x_{  1+i}
	&\displaystyle\sum_{i=1}^{n-k} x_{k-2+i} x_{    i}
\\
\vdots
	&\vdots
	&\ddots
	&\vdots
	&\vdots
\\
\displaystyle         \sum_{i=1}^{n-k} x_{  2+i} x_{k-1+i}
	&\displaystyle\sum_{i=1}^{n-k} x_{  2+i} x_{k-2+i}
	&\dots
	&\displaystyle\sum_{i=1}^{n-k} x_{  2+i} x_{  1+i}
	&\displaystyle\sum_{i=1}^{n-k} x_{  2+i} x_{    i}
\\
\displaystyle         \sum_{i=1}^{n-k} x_{  1+i} x_{k-1+i}
	&\displaystyle\sum_{i=1}^{n-k} x_{  1+i} x_{k-2+i}
	&\dots
	&\displaystyle\sum_{i=1}^{n-k} x_{  1+i} x_{  1+i}
	&\displaystyle\sum_{i=1}^{n-k} x_{  1+i} x_{    i}
\end{pmatrix},
\quad
\transpose{A}b
=
\begin{pmatrix}
\displaystyle\sum_{i=1}^{n-k} x_{k-1+i} x_{k+i} \\
\displaystyle\sum_{i=1}^{n-k} x_{k-2+i} x_{k+i} \\
\vdots \\
\displaystyle\sum_{i=1}^{n-k} x_{  1+i} x_{k+i} \\
\displaystyle\sum_{i=1}^{n-k} x_{    i} x_{k+i}
\end{pmatrix}.
\]

Für den Fall $k=2$ und $n=5$ werden die Gleichungen etwas übersichtlicher,
nämlich
\[
\renewcommand{\arraycolsep}{2pt}
\begin{array}{rcrcl}
x_2{\color{darkred}\vartheta_1} &+& x_1{\color{darkred}\vartheta_2} &=& x_3 \\
x_3{\color{darkred}\vartheta_1} &+& x_2{\color{darkred}\vartheta_2} &=& x_4 \\
x_4{\color{darkred}\vartheta_1} &+& x_3{\color{darkred}\vartheta_2} &=& x_5
\end{array}
\]
Setzt man die Daten \eqref{60000043:daten} in darin
ein, findet man die Matrizen
\[
A
=
\begin{pmatrix}
x_2&x_1\\
x_3&x_2\\
x_4&x_3
\end{pmatrix}
=
\Amatrix,\quad
b
=
\begin{pmatrix}
x_3\\
x_4\\
x_5
\end{pmatrix}
=
\bvector.
\]
Nach dem Lösungsverfahren für least-squardes-Probleme wird daraus das
Gleichungssystem
\[
\transpose{A}A
=
\Mmatrix,\quad
\transpose{A}b
=
\rvector.
\]
Durch Auflösung des Gleichungssystems findet man
\[
\begin{pmatrix*}
\vartheta_1\\
\vartheta_2
\end{pmatrix*}
=
(\transpose{A}A)^{-1}\transpose{A}b
=
\thetavector.
\]
In der Tat wurden die Daten~\eqref{60000043:daten} mit den Koeffizienten
$\vartheta_1=0$ und $\vartheta_2=-1$ aus den Anfangswerten
$x_1=\xone$ und $x_2=\xtwo$ erzeugt und dann mit normalverteilten Fehlern
gestört.
\end{loesung}

\begin{bewertung}
Gleichungen für die Koeffizienten ${\color{darkred}\vartheta}$
({\bf G}) 1 Punkt,
überbestimmtes Gleichungssystem mit Matrix $A$ ({\bf A}) 1 Punkt,
Vektor $b$ ({\bf B}) 1 Punkt,
Lösungsmethode ({\bf M}) 1 Punkt,
Berechnung der Matrizen $\transpose{A}A$ und des Vektors $\transpose{A}b$
({\bf R}) 1 Punkt,
Lösungsvektor ${\color{darkred}\vartheta}$ ({\bf T}) 1 Punkt.
\end{bewertung}
