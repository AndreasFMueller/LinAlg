Der Mittelpunkt des Würfels mit Kantenlänge $2$ befindet sich im
Koordinatenursprung.
Eine Drehung des Raumes führt die Punkte $A$, $B$ und $C$ in
die Punkte $A'$, $B'=C$ bzw. $C'=B$ über.

\begin{center}
\includeagraphics[]{wuerfel.pdf}
\end{center}

\begin{teilaufgaben}
\item
Man finde die Drehmatrix $R$.
\item
Man berechne den Drehwinkel dieser Drehmatrix.
\end{teilaufgaben}

\begin{loesung}
Die Punkte haben die Koordinaten
\begin{align*}
   A&=(1,\phantom{-}1,1) &   A'&=(-1,-1,-1)          \\
B=C'&=(1,-1,1)           & C=B'&=(\phantom{-}1,-1,-1)
\end{align*}
\begin{teilaufgaben}
\item
Gesucht wird eine Matrix $R$, welche die Vektoren $\overrightarrow{OX}$
in die Vektoren $\overrightarrow{OX'}$ überführt für $X\in\{A,B,C\}$,
also $R\overrightarrow{OX}=\overrightarrow{OX'}$.
Schreibt man die Vektoren in eine Matrix, muss gelten
\[
R
\underbrace{
\begin{pmatrix}
1& 1& 1\\
1&-1&-1\\
1& 1&-1
\end{pmatrix}}_{\displaystyle =B_1}
=
\underbrace{
\begin{pmatrix}
-1& 1& 1\\
-1&-1&-1\\
-1&-1& 1
\end{pmatrix}}_{\displaystyle =B_2}
\qquad \Rightarrow\qquad
R=B_2B_1^{-1}.
\]
Für die Inverse von $B_1$ findet man
\[
B_1^{-1}
=
\frac12
\begin{pmatrix}
0&-1& 1\\
1& 0&-1\\
1& 1& 0
\end{pmatrix}.
\]
Daraus berechnet man die Drehmatrix als
\[
R=B_2B_1^{-1}
=
\begin{pmatrix}
 0&-1& 0\\
-1& 0& 0\\
 0& 0&-1
\end{pmatrix}.
\]
\item
Die Drehwinkelformel gibt für die Matrix $R$
\begin{align*}
\cos \alpha 
&=
\frac{\operatorname{Spur}R-1}2=-1
\qquad\Rightarrow\qquad\alpha=\pi.
\qedhere
\end{align*}
\end{teilaufgaben}
\end{loesung}

