Der Mittelpunkt des Würfels mit Kantenlänge $2$ befindet sich im
Koordinatenursprung.
Eine Drehung des Raumes führt die Punkte $A$, $B$ und $C$ in
die Punkte $A'$, $B'=C$ bzw. $C'=B$ über.

\begin{center}
\includeagraphics[]{wuerfel.pdf}
\end{center}

\begin{teilaufgaben}
\item
Man finde die Drehmatrix $R$.
\item
Man berechne den Drehwinkel dieser Drehmatrix.
\end{teilaufgaben}

\thema{Drehmatrix}
\thema{Drehwinkel}
\thema{Abbildungsmatrix}

\begin{loesung}
Die Punkte haben die Koordinaten
\begin{align*}
A&=(\phantom{-}1,         - 1,\phantom{-}1)
			&A'&=(\phantom{-}1,         - 1,         - 1)\\
B&=(\phantom{-}1,\phantom{-}1,\phantom{-}1)
			&B'&=(         - 1,         - 1,         - 1)\\
C&=(         - 1,         - 1,         - 1)
			&C'&=(\phantom{-}1,\phantom{-}1,\phantom{-}1)
\end{align*}
Durch diese Punkte ist die Drehung zwar festgelegt, aber leider liegen
sie in einer Ebene, bilden also keine Basis.
Wir müssen daher noch ein weiteres Punktepaar aus der Zeichnung
herauslesen.
Wir wählen die vordere untere Ecke
\[
D=(1,-1,-1) \qquad\mapsto\qquad D'=(1,-1,1).
\]
\begin{teilaufgaben}
\item
Gesucht wird eine Matrix $R$, welche die Vektoren $\overrightarrow{OX}$
in die Vektoren $\overrightarrow{OX'}$ überführt für $X\in\{A,B,D\}$,
also $R\overrightarrow{OX}=\overrightarrow{OX'}$.
Schreibt man die Vektoren in eine Matrix, muss gelten
\[
R
\underbrace{
\begin{pmatrix}
\phantom{-}1&\phantom{-}1&\phantom{-}1\\
         - 1&\phantom{-}1&         - 1\\
\phantom{-}1&\phantom{-}1&         - 1
\end{pmatrix}}_{\displaystyle =B_1}
=
\underbrace{
\begin{pmatrix}
\phantom{-}1&-1&\phantom{-}1\\
         - 1&-1&         - 1\\
         - 1&-1&\phantom{-}1
\end{pmatrix}}_{\displaystyle =B_2}
\qquad \Rightarrow\qquad
R=B_2B_1^{-1}.
\]
Für die Inverse von $B_1$ findet man
%   0.00000  -0.50000   0.50000
%   0.50000   0.50000   0.00000
%   0.50000   0.00000  -0.50000
%
%ans =
%
%   0  -1   0
%  -1   0   0
%   0   0  -1
%
\[
B_1^{-1}
=
\frac12
\begin{pmatrix}
\phantom{-}0&         - 1&\phantom{-}1\\
\phantom{-}1&\phantom{-}1&\phantom{-}0\\
\phantom{-}1&\phantom{-}0&         - 1
\end{pmatrix}.
\]
Daraus berechnet man die Drehmatrix als
\[
R=B_2B_1^{-1}
=
\begin{pmatrix}
\phantom{-}0&         - 1&\phantom{-}0\\
         - 1&\phantom{-}0&\phantom{-}0\\
\phantom{-}0&\phantom{-}0&         - 1
\end{pmatrix}.
\]
\item
Die Drehwinkelformel gibt für die Matrix $R$
\begin{align*}
\cos \alpha 
&=
\frac{\operatorname{Spur}R-1}2=-1
\qquad\Rightarrow\qquad\alpha=\pi.
\qedhere
\end{align*}
\end{teilaufgaben}
\end{loesung}

\begin{bewertung}
Matrizen $B_1$ und $B_2$ ({\bf B}) 2 Punkte,
Inverse $B_1^{-1}$ ({\bf I}) 1 Punkt,
Drehmatrix $R$ ({\bf R}) 1 Punkt,
Drehwinkelformel ({\bf F}) 1 Punkt,
Drehwinkel ({\bf W}) 1 Punkt.
\end{bewertung}



