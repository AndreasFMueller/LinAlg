Folgende Abbildungen in der Ebene sollen jeweils mit einer 
einzigen Matrix beschreiben werden.
\begin{teilaufgaben}
\item
Eine Drehung um den Winkel $30^\circ$, genannt $D$.
\item
Eine Verschiebung um den Vektor $\begin{pmatrix}2&1\end{pmatrix}^t$,
genannt $T$.
\item
Eine Skalierung um den Faktor $\frac12$, genannt $S$.
\item
Die Drehung $D$ gefolgt von der Translation $T$ gefolgt von der Skalierung $S$.
\item
Die Translation $T$ gefolgt von der Skalierung $S$ gefolgt von der Drehung $D$.
\item
Die Translation $T$ gefolgt von der Drehung $D$ gefolgt von der Skalierung $S$.
\end{teilaufgaben}

\thema{Abbildungsmatrix}
\thema{Matrizenprodukt}

\begin{loesung}
\begin{teilaufgaben}
\item
Die Drehung kann beschrieben werden durch die Drehmatrix
\[
D=D_{30^\circ}
=
\begin{pmatrix}
\cos 30^\circ & -\sin 30^\circ \\
\sin 30^\circ &  \cos 30^\circ
\end{pmatrix}
=
\begin{pmatrix}
\frac{\sqrt{3}}2 & - \frac12\\
\frac12 & \frac{\sqrt{3}}2
\end{pmatrix}.
\]
\item
Die Verschiebung um einen Vektor kann nicht mit einer $2\times 2$-Matrix
beschrieben werden. Wir verwenden daher den Trick, den Koordinaten der
Ebene eine dritte Komponente zu spendieren, die wir immer auf $1$ setzen.
In diesen Koordinaten kann die Verschiebung durch die Matrix
\[
T=
\begin{pmatrix}
1&0&2\\
0&1&1\\
0&0&1
\end{pmatrix}
\]
beschrieben werden.
\item
Die Skalierung um den Faktor $\frac12$ könnte zwar mit einer $2\times 2$-Matrix
beschrieben werden, doch weil wir später die Skalierung mit der Translation
zusammensetzen wollen, schreiben wir sie bereits jetzt als $3\times 3$-Matrix
\[
S
=
\begin{pmatrix}
\frac12 &     0   & 0 \\
    0   & \frac12 & 0 \\
    0   &     0   & 1 
\end{pmatrix}
.
\]
\item
Verlangt ist die Verknüpfung $STD$, wozu wir zunächst auch die Drehung als
$3\times 3$-Matrix schreiben müssen:
\[
D =
\begin{pmatrix}
\frac{\sqrt{3}}2 & - \frac12        & 0 \\
\frac12          & \frac{\sqrt{3}}2 & 0 \\
      0          &        0         & 1
\end{pmatrix}
.
\]
Damit können wir jetzt $STD$ berechnen:
\begin{align*}
STD
&=
\begin{pmatrix}
\frac12 &     0   & 0 \\
    0   & \frac12 & 0 \\
    0   &     0   & 1 
\end{pmatrix}
\begin{pmatrix}
1&0&2\\
0&1&1\\
0&0&1
\end{pmatrix}
\begin{pmatrix}
\frac{\sqrt{3}}2 & - \frac12        & 0 \\
\frac12          & \frac{\sqrt{3}}2 & 0 \\
      0          &        0         & 1
\end{pmatrix}
\\
&=
\begin{pmatrix}
\frac12 &     0   & 1 \\
    0   & \frac12 & \frac12 \\
    0   &     0   & 1 
\end{pmatrix}
\begin{pmatrix}
\frac{\sqrt{3}}2 & - \frac12        & 0 \\
\frac12          & \frac{\sqrt{3}}2 & 0 \\
      0          &        0         & 1
\end{pmatrix}
\\
&=
\begin{pmatrix}
\frac{\sqrt{3}}4 & -\frac14        & 1       \\
\frac14          &\frac{\sqrt{3}}4 & \frac12 \\
0                &     0           & 1 
\end{pmatrix}.
\end{align*}
\item
Verlangt ist die Verknüpfung
\begin{align*}
DST
&=
\begin{pmatrix}
\frac{\sqrt{3}}2 & - \frac12        & 0 \\
\frac12          & \frac{\sqrt{3}}2 & 0 \\
      0          &        0         & 1
\end{pmatrix}
\begin{pmatrix}
\frac12 &     0   & 0 \\
    0   & \frac12 & 0 \\
    0   &     0   & 1 
\end{pmatrix}
\begin{pmatrix}
1&0&2\\
0&1&1\\
0&0&1
\end{pmatrix}
\\
&=
\begin{pmatrix}
\frac{\sqrt{3}}2 & - \frac12        & 0 \\
\frac12          & \frac{\sqrt{3}}2 & 0 \\
      0          &        0         & 1
\end{pmatrix}
\begin{pmatrix}
\frac12 &     0   & 1 \\
    0   & \frac12 & \frac12 \\
    0   &     0   & 1 
\end{pmatrix}
\\
&=
\begin{pmatrix}
\frac{\sqrt{3}}4 & -\frac14        & \frac{\sqrt{3}}2-\frac14 \\
\frac14          &\frac{\sqrt{3}}4 & \frac12+\frac{\sqrt{3}}4 \\
       0         &        0        & 1
\end{pmatrix}.
\end{align*}
\item
Verlangt ist die Verknüpfung
\begin{align*}
SDT
&=
\begin{pmatrix}
\frac12 &     0   & 0 \\
    0   & \frac12 & 0 \\
    0   &     0   & 1 
\end{pmatrix}
\begin{pmatrix}
\frac{\sqrt{3}}2 & - \frac12        & 0 \\
\frac12          & \frac{\sqrt{3}}2 & 0 \\
      0          &        0         & 1
\end{pmatrix}
\begin{pmatrix}
1&0&2\\
0&1&1\\
0&0&1
\end{pmatrix}
\\
&=
\begin{pmatrix}
\frac{\sqrt{3}}4 & - \frac14        & 0 \\
\frac14          & \frac{\sqrt{3}}4 & 0 \\
      0          &        0         & 1
\end{pmatrix}
\begin{pmatrix}
1&0&2\\
0&1&1\\
0&0&1
\end{pmatrix}
\\
&=
\begin{pmatrix}
\frac{\sqrt{3}}4 & -\frac14         & \frac{\sqrt{3}}2-\frac14         \\
\frac14          & \frac{\sqrt{3}}4 & \frac12+\frac{\sqrt{3}}4 \\
0                & 0                & 1
\end{pmatrix}.
\qedhere
\end{align*}
\end{teilaufgaben}
\end{loesung}


