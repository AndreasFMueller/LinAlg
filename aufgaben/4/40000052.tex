Die Gerade
\[
\vec r
=
\frac12\begin{pmatrix}3\\0\\3\end{pmatrix}
+t
\begin{pmatrix}1\\4\\-1\end{pmatrix}
\]
schneidet die Kugel um den Nullpunkt mit Radius $3$ in zwei Punkten.
Berechnen Sie den Winkel zwischen den Tangentialebenen in diesen
Punkten.

\thema{Zwischenwinkel}
\thema{Kugel}

\begin{loesung}
Einsetzen der Geradengleichung in die Kugelgleichung ergibt
\begin{align*}
(\frac32+t)^2 + 16t^2 + (\frac32 - t)^2&=3^2\\
\frac94+3t+t^2+16t^2+\frac94-3t+t^2&=9\\
\frac92+18t^2&=9\\
18t^2&=\frac92\\
t^2&=\frac14\\
t&=\pm\frac12
\end{align*}
Die Schnittpunkte sind daher
\begin{align*}
t&=\frac12&\Rightarrow\qquad p_1&=\begin{pmatrix}2\\2\\1\end{pmatrix}\\
t&=-\frac12&\Rightarrow\qquad p_2&=\begin{pmatrix}1\\-2\\2\end{pmatrix}
\end{align*}
Diese Vektoren sind auch gleich die Normalen der genannten Tangentialebenen,
also ist der Zwischenwinkel:
\begin{align*}
\cos\alpha&=\frac{p_1\cdot p_2}{|p_1|\;|p_2|} =\frac{2-4+2}{3\cdot 3}=0
\\
\alpha&=90^\circ,
\end{align*}
die beiden Ebenen stehen also senkrecht aufeinander.
\end{loesung}

