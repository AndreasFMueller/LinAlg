%Ein Teleskop wird um eine horizontale Achse drehbar in einer Gabel gelagert,
%die ihrerseits um die vertikale Achse gedreht werden kann. So kann ein
%Tourist an einem Aussichtspunkt jeden beliebigen Punkt beobachten
%(Abbildung~\ref{40000029:altaz} links).
%Für ein grosses Teleskop, welches bei $45^\circ$ nördlicher Breite
%aufgestellt werden soll, ist diese Montierungsart zu simpel.
%Um die
%Drehung der Erde kompensieren zu können, wird die vertikale Achse
%gekippt, so dass sie parallel zur Erdachse liegt. Dreht man das Teleskop mit
%einer Umdrehung pro Tag um diese Achse, zeigt es immer auf den gleichen
%Punkt am Himmel (Abbildung~\ref{40000029:altaz} rechts).
%\begin{figure}[ht]
%\begin{center}
%%\includeagraphics[width=0.266\hsize]{azimutala.jpg}
%\includeagraphics[width=0.2128\hsize]{azimutala.jpg}
%\quad
%\quad
%\quad
%\quad
%%\includeagraphics[width=0.399\hsize]{forksaltazeq.jpg}
%\includeagraphics[width=0.3192\hsize]{forksaltazeq.jpg}
%\end{center}
%\caption{Montierung von Teleskopen: azimutale Montierung (links)
%und äquatoriale Montierung (rechts)\label{40000029:altaz}}
%\end{figure}
Zur Beschreibung der Position von Objekten am Himmel sind zwei
Koordinatensysteme üblich.
\begin{figure}[ht]
\begin{center}
\includeagraphics[width=0.3\hsize]{azimutalcoord.jpg}
\quad
\quad
\includeagraphics[width=0.3\hsize]{equatorialcoord.jpg}
\end{center}
\caption{Koordinatensysteme: azimutal (links) und äquatorial (rechts),
die Richtung nach Osten ist links vorne.
\label{40000029:coord}}
\end{figure}
\begin{itemize}
\item
Das {\it azimutale} Koordinatensystem
(Abbildung \ref{40000029:coord} links)
verwendet eine $x$-Achse, die genau nach Osten zeigt, und eine $z$-Achse,
die nach oben zeigt.
In diesem Koordinatensystem ist es besonders einfach zu entscheiden, ob
ein Himmelskörper über dem Horizont steht: seine $z$-Koordinate ist
positiv.
\item
Das {\it äquatoriale} Koordinatensystem
(Abbildung \ref{40000029:coord} rechts)
verwendet wieder eine $x$-Achse,
die nach Osten zeigt, aber die $z$-Achse ist parallel zur Erdachse.
In diesem Koordinatensystem ist die Bewegung der Objekte, hervorgerufen
durch die Erddrehung, besonders einfach zu berechnen.
\end{itemize}
In den folgenden Teilaufgaben gehen wir davon aus, dass sich der Beobachter
bei $45^\circ$ nördlicher Breite befindet.
\begin{teilaufgaben}
\item Finden Sie die Vektoren der Nordrichtung und der Zenithrichtung
(vertikal nach oben) im azimutalen und im äquatorialen Koordinatensystem.
\item Finden Sie die Transformationsmatrix, mit der Sie Vektoren
vom azimutalen ins äquatoriale Koordinatensystem umrechnen können.
%\item Im äquatorialen Koordinatensystem entspricht $\vec e_3$
%der Erdachse. Welche Koordinaten hat dieser Vektor im azimutalen
%Koordinatensystem?
\item Ein Himmelskörper geht genau im Südosten auf
(Abbildung \ref{40000029:coord}).
Bestimmen Sie einen Einheitsvektor mit Blickrichtung zum
aufgehenden Himmelskörper in beiden Koordinatensystemen.
\item Ist der Himmelskörper nach 8 Stunden schon wieder untergegangen?
\end{teilaufgaben}

\begin{loesung}
\begin{teilaufgaben}
\item Im azimutalen Koordinatensystem sind dies die Vektoren $\vec e_2$
und $\vec e_3$. Im äquatorialen System gilt
\[
\text{Norden:} \quad \begin{pmatrix}0\\ \cos 45^\circ\\\sin 45^\circ\end{pmatrix}
\qquad\text{und}\qquad
\text{Zenith:} \quad \begin{pmatrix}0\\-\sin 45^\circ\\\cos 45^\circ\end{pmatrix}
\]

\item Die Transformation ist eine Drehung um den Winkel $45^\circ$
um die $x$-Achse, sie hat die Matrix:
\[
T=\begin{pmatrix}
1& 0            & 0            \\
0& \cos 45^\circ&-\sin 45^\circ\\
0& \sin 45^\circ& \cos 45^\circ
\end{pmatrix}
\]
%\item Bilder der Standardbasisvektoren sind die Spalten der Matrix, also
%ist das Bild von $\vec e_3$ unter der Matrix $T$ der Vektor
%\[
%\begin{pmatrix}0\\\sin 45^\circ\\\cos 45^\circ\end{pmatrix}.
%\]
\item Im azimutalen Koordinatensystem hat der aufgehende Himmelskörper
die Richtung
\[
v=\begin{pmatrix}
\frac{\sqrt{2}}2\\-\frac{\sqrt{2}}2\\0
\end{pmatrix}.
\]
\item
Die Bewegung der Erde im aquatorialen Koordinatensystem ist eine Drehung
um $ 120^\circ$ um die $z$-Achse, hat also die Matrix
\[
A'=\begin{pmatrix}
 \cos 120^\circ & \sin 120^\circ & 0\\
-\sin 120^\circ & \cos 120^\circ & 0\\
 0              & 0              & 1
\end{pmatrix}
\]
Im azimutalen Koordinatensystem hat die Abbildung die Matrix
$A=T^{-1}A'T.$ Wir müssen dieses Produkt nicht ausrechnen, sondern
sondern nur seine Wirkung auf den Vektor $v$:
\begin{align*}
Av&=
\begin{pmatrix}
1& 0                  & 0                 \\
0& \frac{\sqrt{2}}2   & \frac{\sqrt{2}}2  \\
0&-\frac{\sqrt{2}}2   & \frac{\sqrt{2}}2
\end{pmatrix}
\begin{pmatrix}
 \frac12          & \frac{\sqrt{3}}2 0\\
-\frac{\sqrt{3}}2 & \frac12        & 0\\
 0                & 0              & 1
\end{pmatrix}
\begin{pmatrix}
1& 0                 & 0                 \\
0& \frac{\sqrt{2}}2  &-\frac{\sqrt{2}}2  \\
0& \frac{\sqrt{2}}2  & \frac{\sqrt{2}}2  
\end{pmatrix}v
\\
&=
\begin{pmatrix}
   0.50000 & 0.61237& -0.61237\\
  -0.61237 & 0.75000&  0.25000\\
   0.61237 & 0.25000&  0.75000
\end{pmatrix}v
=\begin{pmatrix}
  -0.079459\\
  -0.963343\\
   0.256236
\end{pmatrix}
\end{align*}
Da die $z$-Komponente positiv ist, steht der Himmelskörper immer noch
über dem Horizont.
\qedhere
\end{teilaufgaben}
\end{loesung}

\begin{bewertung}
\begin{teilaufgaben}
\item 1 Punkt.
\item Drehmatrix oder Matrix aus den in a) gefundenen Spalten, 1 Punkt.
\item 1 Punkt.
\item Drehmatrix mit Winkel $120^\circ$ ({\bf D}) 1 Punkt,
Transformationsformel ({\bf T}) 1 Punkt,
Resultatvektor und Schlussfolgerung ({\bf S}) 1 Punkt.
\end{teilaufgaben}
\end{bewertung}

