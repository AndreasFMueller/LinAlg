Gegeben sind drei Ebenen durch den Nullpunkt mit den Normalenvektoren
\[
\vec{n}_1
=
\begin{pmatrix}1\\2\\2\end{pmatrix}
,\qquad
\vec{n}_2
=
\begin{pmatrix}2\\6\\3\end{pmatrix}
,\qquad
\vec{n}_3
=
\begin{pmatrix}8\\4\\1\end{pmatrix}.
\]
Finden Sie die Parameterdarstellung einer Geraden, deren Punkte von allen drei
Ebenen den gleichen Abstand haben.

\begin{loesung}
Der Abstand eines Punktes mit Koordinaten $(x,y,z)$ von jeder der Ebenen
kann mit Hilfe der Hesseschen Normalform berechnet werden.
Dazu müssen die Vektoren $\vec{n}_i$ normiert werden, ihre Längen sind
\[
|\vec{n}_1| = 3
,\qquad
|\vec{n}_2| = 7
,\qquad
|\vec{n}_3| = 9.
\]
Die Abstandsgleichungen sind dann
\[
\begin{linsys}{3}
\frac13 x &+& \frac23 y &+& \frac23 z &=&     d\phantom{.} \\
\frac27 x &+& \frac67 y &+& \frac37 z &=& \pm d\phantom{.} \\
\frac89 x &+& \frac49 y &+& \frac19 z &=& \pm d.\\
\end{linsys}
\]
Da nicht spezifiziert ist, auf welcher Seite der Ebene die Gerade liegen
soll, müssen genaugenommen verschiedene Vorzeichen in der zweiten und
dritten Gleichung berücksichtigt werden.
Da aber nur eine Gerade verlangt wird, reicht es, nur mit den positiven
Vorzeichen zu rechnen.

Indem man $d$ als eine weitere Unbekannt verwendet, kann man ein
Gleichungssystem mit $3$ Gleichungen und $4$ Unbekannten erhalten,
welches in Tableauform als
\[
\begin{tabular}{|>{$}c<{$}>{$}c<{$}>{$}c<{$}>{$}c<{$}|>{$}r<{$}|}
\hline
1&2&3&\phantom{\pm}3&0\\
2&6&3&\pm7&0\\
8&4&1&\pm9&0\\
\hline
\end{tabular}
\]
geschrieben werden kann.

Wir lösen alle diese Gleichungssysteme mit dem Gaussalgorithmus:
\begin{equation}
\begin{aligned}
\begin{tabular}{|>{$}c<{$}>{$}c<{$}>{$}c<{$}>{$}r<{$}|}
\hline
1&2&3&\phantom{-}3\\
2&6&3&7\\
8&4&1&9\\
\hline
\end{tabular}
&\rightarrow
\begin{tabular}{|>{$}c<{$}>{$}c<{$}>{$}c<{$}>{$}r<{$}|}
\hline
1&0&0&\frac57\\
0&1&0&\frac57\\
0&0&1&\frac37\\
\hline
\end{tabular}
&&\Rightarrow&
\vec{r}
&=
\begin{pmatrix*}[r]
5\\5\\3
\end{pmatrix*}
\\
\begin{tabular}{|>{$}c<{$}>{$}c<{$}>{$}c<{$}>{$}r<{$}|}
\hline
1&2&3&3\\
2&6&3&-7\\
8&4&1&9\\
\hline
\end{tabular}
&\rightarrow
\begin{tabular}{|>{$}c<{$}>{$}c<{$}>{$}c<{$}>{$}r<{$}|}
\hline
1&0&0& \frac{19}7\\
0&1&0&-\frac{30}7\\
0&0&1& \frac{31}7\\
\hline
\end{tabular}
&&\Rightarrow&
\vec{r}
&=
\begin{pmatrix*}[r]
19\\-30\\31
\end{pmatrix*}
\\
\begin{tabular}{|>{$}c<{$}>{$}c<{$}>{$}c<{$}>{$}r<{$}|}
\hline
1&2&3&3\\
2&6&3&-7\\
8&4&1&-9\\
\hline
\end{tabular}
&\rightarrow
\begin{tabular}{|>{$}c<{$}>{$}c<{$}>{$}c<{$}>{$}r<{$}|}
\hline
1&0&0& \frac{ 1}{7}\\
0&1&0&-\frac{27}{7}\\
0&0&1& \frac{37}{7}\\
\hline
\end{tabular}
&&\Rightarrow&
\vec{r}
&=
\begin{pmatrix*}[r]
1\\-27\\37
\end{pmatrix*}
\\
\begin{tabular}{|>{$}c<{$}>{$}c<{$}>{$}c<{$}>{$}r<{$}|}
\hline
1&2&3&3\\
2&6&3&7\\
8&4&1&-9\\
\hline
\end{tabular}
&\rightarrow
\begin{tabular}{|>{$}c<{$}>{$}c<{$}>{$}c<{$}>{$}r<{$}|}
\hline
1&0&0&-\frac{13}{7}\\
0&1&0& \frac{ 8}{7}\\
0&0&1& \frac{ 9}{7}\\
\hline
\end{tabular}
&&\Rightarrow&
\vec{r}
&=
\begin{pmatrix*}[r]
-13\\8\\9
\end{pmatrix*}
\end{aligned}
\label{40000068:eqn}
\end{equation}
Da der Nullpunkt auf allen Ebenen liegt, kann man ihn als Stützvektor 
verwenden.
Die gesuchten Geraden haben daher die Parameterdarstellung
$\vec{p} = t\vec{r}$
für jeden der Vektoren $\vec{r}$ von
\eqref{40000068:eqn}.
\end{loesung}

\begin{bewertung}
Normierung ({\bf N}) 1 Punkt,
Hessesche Normalform der Ebenen ({\bf H}) 1 Punkt,
Gleichungen für gleichen Abstand ({\bf A}) 1 Punkt,
Lösung des Gleichungssystems ({\bf L}) 1 Punkt,
Geradengleichung ({\bf G}) 1 Punkt,
Problem der Vorzeichen der Abstände erkannt ({\bf V}) 1 Punkt.
\end{bewertung}
