Bei der Firma Richter Spielger"ate GmbH kann man das unten abgebildete
kardanisch aufgeh"angte Becken kaufen. 
\begin{center}
\includeagraphics[width=0.5\hsize]{kardan.png}
\end{center}
Der "aussere Rahmen ist fest montiert, der innere Rahmen kann um
die Lager vorne rechts und hinten links um den Winkel $\alpha$
aus der Horizontalen
gedreht werden. Das Becken ganz zuinnerst kann um die Achse
senkrecht dazu um den Winkel $\beta$ aus der Horizontalen gedreht werden.
Welchen
Winkel zur Vertikalen hat die Normale des Beckens (in Abh"angigkeit
von $\alpha$ und $\beta$)?

\begin{loesung}
Wir w"ahlen eine Koordinatensystem, dessen $x$-Achse durch die
erste Drehachse verl"auft, und dessen $z$-Achse vertikal ist.
Die Normale des Beckens entsteht dadurch, dass zuerst der
Einheitsvektor $e_3$ (die Normale des Beckens in der Ausgangslage)
um den Winkel $\beta$ um die $y$-Achse gedreht wird,
und dann der innere Rahmen um den Winkel $\alpha$ um die
$x$-Achse. Die Drehungen k"onnen durch Drehmatrizen beschrieben werden:
\begin{align*}
\vec n
&=
\begin{pmatrix}
1&0&0\\
0&\cos\alpha&-\sin\alpha\\
0&\sin\alpha& \cos\alpha
\end{pmatrix}
\begin{pmatrix}
\cos\beta&0&-\sin\beta\\
0&1&0\\
\sin\beta&0&\cos\beta
\end{pmatrix}\vec e_3
\\
&=
\begin{pmatrix}
1&0&0\\
0&\cos\alpha&-\sin\alpha\\
0&\sin\alpha& \cos\alpha
\end{pmatrix}
\begin{pmatrix}
-\sin\beta\\
0\\
\cos\beta
\end{pmatrix}
=
\begin{pmatrix}
-\sin\beta\\
\sin\alpha\cos\beta\\
\cos\alpha\cos\beta
\end{pmatrix}
\end{align*}
Der Winkel zur Vertikalen kann mit dem Skalarprodukt bestimmt
werden:
\begin{align*}
\cos\gamma
&=
\vec n\cdot\vec e_3=
\begin{pmatrix}
-\sin\beta\\
\sin\alpha\cos\beta\\
\cos\alpha\cos\beta
\end{pmatrix}
\cdot
\begin{pmatrix}0\\0\\1\end{pmatrix}=\cos\alpha\cos\beta
\\
\gamma&=\arccos(\cos\alpha\cos\beta).
\qedhere
\end{align*}
\end{loesung}

\begin{diskussion}
Das Produkt der beiden Drehmatrizen ist
\begin{align*}
\begin{pmatrix}
1&0&0\\
0&\cos\alpha&-\sin\alpha\\
0&\sin\alpha& \cos\alpha
\end{pmatrix}
\begin{pmatrix}
\cos\beta&0&-\sin\beta\\
0&1&0\\
\sin\beta&0&\cos\beta
\end{pmatrix}
&=
\begin{pmatrix}
\cos\beta           &0         &-\sin\beta\\
-\sin\alpha\sin\beta&\cos\alpha&-\sin\alpha\cos\beta\\
 \sin\alpha\sin\beta&\sin\alpha&\cos\alpha\cos\beta
\end{pmatrix}
\end{align*}
Man k"onnte jetzt versucht sein, die L"osung mit Hilfe der Spurformel
zu bestimmen, man bek"ame dann f"ur den Winkel die Formel
\[
\cos\gamma' = \frac{\cos\beta+\cos\alpha+\cos\alpha\cos\beta-1}{2}.
\]
Dies ist aber falsch: die Spurformel berechnet den Drehwinkel $\gamma'$
der Drehung,
die das ganze Koordinatensystem mit in einer Drehung mit dem verdrehten
Koordinatensystem zur Deckung bringt. Die Aufgabe fragt aber nur
nach dem Drehwinkel $\gamma$ der Drehung um eine horizontale Achse,
die die verdrehte Normale
zur"uck in die Vertikale bringt. Dass dies nicht das selbe ist kann
man zum Beispiel daran sehen, dass bei einer Drehung um die $z$-Achse 
die Normale sicher den Winkel 0 zur Vertikalen hat, die Spurformel
liefert aber den Drehwinkel um die $z$-Achse.
\end{diskussion}

\begin{diskussion}
Man beachte, dass f"ur kleine Winkel gilt
\begin{align*}
\cos\gamma=
1-\frac{\gamma^2}2+\frac{\gamma^4}{4!}+\dots
&=
\biggl(
1-\frac{\alpha^2}2+\frac{\alpha^4}{4!}+\dots
\biggl)
\biggl(
1-\frac{\beta^2}2+\frac{\beta^4}{4!}+\dots
\biggl)
\\
&=
1-\frac{\alpha^2+\beta^2}2+\dots
\\
\Rightarrow\qquad
\gamma^2&\simeq\alpha^2+\beta^2
\end{align*}
F"ur kleine Winkel ist der Winkel ungef"ahr der Betrag der Auslenkung aus der
Vertikalen, und die beiden senkrecht stehenden Auslenkungen m"ussen nat"urlich
nach dem Satz des Pythagoras addiert werden.
\end{diskussion}

