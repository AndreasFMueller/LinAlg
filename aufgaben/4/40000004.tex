Polynome vom Grad höchstens 3 kann man durch die Folge der
Koeffizienten als Vektoren beschreiben:
\[
9x^3+10x^2+11x+12\mapsto 
\begin{pmatrix}
12\\11\\10\\9
\end{pmatrix}
\]
Betrachten Sie jetzt die Matrix
\[
D=\begin{pmatrix}
0&1&0&0\\
0&0&2&0\\
0&0&0&3\\
0&0&0&0
\end{pmatrix},
\qquad
I=\begin{pmatrix}
0&0&0&0\\
1&0&0&0\\
0&\frac12&0&0\\
0&0&\frac13&0\\
\end{pmatrix}
\]
Bestimmen Sie Kern und Bild von $D$, $I$, $DI$ und $ID$.

\thema{Abbildungsmatrix}

\begin{loesung}
Wir berechnen zunächst die Matrizenprodukte:
\[
DI=\begin{pmatrix}
1&0&0&0\\
0&1&0&0\\
0&0&1&0\\
0&0&0&0
\end{pmatrix},\qquad
ID=\begin{pmatrix}
0&0&0&0\\
0&1&0&0\\
0&0&1&0\\
0&0&0&1
\end{pmatrix}
\]
Der Kern besteht jeweils aus den Lösungen des homogenen Systems mit der
Matrix.

$D$: Die erste Variable ist frei wählbar, dem Kern von $D$ entsprechen also
die konstanten Polynome. Die Vektoren $e_1$, $e_2$ und $e_3$ sind Basisvektoren
des Bildraumes $\operatorname{im}D$, kubische Polynome kommen also im Bild
nicht vor. Dasselbe gilt für $ID$.

$I$: Die letzte Variable ist frei wählbar, der Kern umfasst also
genau die kubischen Monome. Die Vektoren $e_2$, $e_3$ und $e_4$
sind Basisvektoren des Bildes $\operatorname{im}I$. Dasselbe gilt
für $DI$.
\end{loesung}


