Sei $\vec u$ ein Einheitsvektor und $\varphi$ ein beliebiger Winkel.
Wir setzen $s=\sin\varphi$ und $c=\cos\varphi$. Betrachten Sie jetzt
die Matrix
\[
O=\begin{pmatrix}
c+(1-c)u_1^2    &(1-c)u_1u_2-u_3s   &(1-c)u_1u_3+u_2s\\
(1-c)u_1u_2+u_3s&c+(1-c)u_2^2       &(1-c)u_2u_3-u_1s\\
(1-c)u_1u_3-u_2s&(1-c)u_2u_3+u_1s   &c+(1-c)u_3^2
\end{pmatrix}
\]
Man kann nachrechnen, dass dies eine Drehmatrix ist (nicht verlangt).
\begin{teilaufgaben}
\item Bestimmen Sie den Drehwinkel.
\item Zeigen sie, dass $\vec u$ die Richtung der Drehachse ist.
\end{teilaufgaben}

\thema{Drehwinkel}
\thema{Drehachse}

\begin{loesung}
\begin{teilaufgaben}
\item
Der Drehwinkel kann mit der Spurformel bestimmt werden. Dazu berechnen
wir zunächst die Spur von $O$
\[
\operatorname{Spur}O=
c+(1-c)u_1^2+c+(1-c)u_2^2+c+(1-c)u_3^2=3c+(1-c)(u_1^2+u_2^2+u_3^2)=1+2c
\]
Jetzt wenden wir die Spurformel an:
\[
\cos\alpha=\frac{\operatorname{Spur}O-1}2=c=\cos\varphi,
\]
Der Drehwinkel ist also gerade der Winkel $\varphi.$
\item
Die Richtung $\vec v$ der Drehachse erfüllt die Bedingung $O\vec v=\vec v$,
wir müssen also nachprüfen, ob $O\vec u=\vec u$.
Dazu setzen wir ein:
\begin{align*}
O\vec u
&=
\begin{pmatrix}
c+(1-c)u_1^2    &(1-c)u_1u_2-u_3s   &(1-c)u_1u_3+u_2s\\
(1-c)u_1u_2+u_3s&c+(1-c)u_2^2       &(1-c)u_2u_3-u_1s\\
(1-c)u_1u_3-u_2s&(1-c)u_2u_3+u_1s   &c+(1-c)u_3^2
\end{pmatrix}
\begin{pmatrix}u_1\\u_2\\u_3\end{pmatrix}
\\
&=
\begin{pmatrix}
cu_1+(1-c)u_1^3 +(1-c)u_1u_2^2-u_2u_3s+(1-c)u_1u_3^2+u_2u_3s\\
(1-c)u_1^2u_2+u_1u_3s+cu_2+(1-c)u_2^3+(1-c)u_2u_3^2-u_1u_3s\\
(1-c)u_1^2u_3-u_1u_2s+(1-c)u_2^2u_3+u_1u_2s+cu_3+(1-c)u_3^3
\end{pmatrix}
\\
&=
\begin{pmatrix}
cu_1+(1-c)u_1(u_1^2+u_2^2+u_3^2)\\
(1-c)u_2(u_1^2+u_2^2+u_3^2)+cu_2\\
(1-c)u_3(u_1^2+u_2^2+u_3^2)+cu_3
\end{pmatrix}
\\
&=
\begin{pmatrix}
cu_1+(1-c)u_1\\
(1-c)u_2+cu_2\\
(1-c)u_3+cu_3
\end{pmatrix}
=\begin{pmatrix}u_1\\u_2\\u_3\end{pmatrix}.
\end{align*}
Im letzten Schritt haben wir verwendet, dass $\vec u$ ein Einheitsvektor
ist, dass also insbesondere $u_1^2 + u_2^2 + u_3^2=1$ gilt.
\qedhere
\end{teilaufgaben}
\end{loesung}

\begin{diskussion}
Die Aufgabe hat sich nicht um die Frage gekümmert, wie man die Matrix $O$
finden kann.
Dies soll hier nachgeholt werden.
Der Vektor $\vec r$ soll um die Achse $\vec u$ und den Winkel $\varphi$
gedreht werden.
Dazu zerlegen wir $\vec r$ zuerst die Komponenten $\vec r_{\|}$ parallel
und $\vec r_{\perp}$ senkrecht auf $\vec u$.
Die Komponenten $\vec r_{\|}$ ändert sich bei der Drehung nicht, also
$\vec r_{\|}'=\vec r_{\|}$.

Die Komponenten $\vec r_{\perp}$ wird in der Ebene senkrecht auf
$\vec u$ gedreht.
Diese Drehung ist einfach zu beschreiben, wenn wir einen Vektor senkrecht auf
$\vec u$ und $\vec r_{\perp}$ finden können.
Ein solcher Vektor ist $\vec u\times \vec r$, er hat gerade die richtige 
Länge $|\vec r_{\perp}|$.
Daraus können wir jetzt den gedrehten Vektor 
\[
\vec r' = \vec r_{\|} + c \vec r_{\perp} + s \vec u\times \vec r
\]
zusammensetzen.

Als letzten Schritt müssen wir $\vec r'$ als $O\vec r$ ausdrücken.
Dazu verwenden wir zuerst die bereits bekannten Ausdrücke für $\vec r_{\|}$
und $\vec r_{\perp}$, nämlich
\begin{align*}
\vec r_{\|}
&=
\vec u\vec u^t\vec r
&
&\text{und}&
\vec r_{\perp}
&=
\vec r - \vec u\vec u^t\vec r
=
(E-\vec u\vec u^t)\vec r
\end{align*}
Die Komponente $\vec u\times \vec r$ können wir ebenfalls als Matrix schreiben,
nämlich
\[
\begin{pmatrix}u_1\\u_2\\u_3\end{pmatrix}
\times
\begin{pmatrix}r_1\\r_2\\r_3\end{pmatrix}
=
\begin{pmatrix}
u_2r_3-u_3r_2\\
u_3r_1-u_1r_3\\
u_1r_2-u_2r_1
\end{pmatrix}
=
\begin{pmatrix}
   0&-u_3& u_2\\
 u_3&   0&-u_1\\
-u_2& u_1&   0
\end{pmatrix}
\begin{pmatrix}r_1\\r_2\\r_3\end{pmatrix}.
\]
Jetzt kann man alle drei Matrizen zusammensetzen:
\begin{align*}
\vec r'
&=
\left(
(\vec u\vec u^t) + c(E-\vec u\vec u^t) + 
s
\begin{pmatrix}
   0&-u_3& u_2\\
 u_3&   0&-u_1\\
-u_2& u_1&   0
\end{pmatrix}
\right)
\vec r
\\
&=
\left(
\begin{pmatrix}
u_1^2&u_1u_2&u_1u_3\\
u_2u_1&u_2^2&u_2u_3\\
u_3u_1&u_3u_2&u_3^2
\end{pmatrix}
+c
\left(E-
\begin{pmatrix}
u_1^2&u_1u_2&u_1u_3\\
u_2u_1&u_2^2&u_2u_3\\
u_3u_1&u_3u_2&u_3^2
\end{pmatrix}
\right)
+s
\begin{pmatrix}
   0&-u_3& u_2\\
 u_3&   0&-u_1\\
-u_2& u_1&   0
\end{pmatrix}
\right)
\vec r
\\
&=
\begin{pmatrix}
(1-c)u_1^2+c    &(1-c)u_1u_2-su_3&(1-c)u_1u_3+su_2\\
(1-c)u_1u_2+su_3&(1-c)u_2^2+c    &(1-c)u_2u_3-su_1\\
(1-c)u_1u_3-su_2 &(1-c)u_2u_3+su_1&(1-c)u_3^2+c
\end{pmatrix}  
\vec r.
\end{align*}
Dies ist genau die Matrix der Aufgabenstellung.
\end{diskussion}


