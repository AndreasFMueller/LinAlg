Vom Dodekaeder
\begin{center}
\begin{tikzpicture}
\node at (0,0) {\includeagraphics[width=8cm]{dodekaeder.jpg}};

\node at (4.1,-0.5) {$x$};
\node at (-3,-1.2) {$y$};
\node at (0,3.8) {$z$};

\node[color=orange] at ( 2.5, 1.95) {$A_1$};
\node[color=orange] at ( 1.1, 1.7 ) {$A_2$};
\node[color=orange] at ( 2.5,-2.1 ) {$A_3$};
\node[color=orange] at ( 0.5,-2.7 ) {$A_4$};
\node[color=orange] at (-0.5, 1.9 ) {$A_5$};
\node[color=orange] at (-3.1, 1.95) {$A_6$};
\node[color=orange] at (-0.7,-1.2 ) {$A_7$};
\node[color=orange] at (-3.3,-2.5 ) {$A_8$};

\node[color=green] at ( 0.8, 0.9 ) {$A_9$};
\node[color=green] at (-2.1, 1.0 ) {$A_{10}$};
\node[color=green] at ( 1.1,-1.1 ) {$A_{11}$};
\node[color=green] at (-1.7,-2.1 ) {$A_{12}$};

\node[color=blue] at ( 1.0, 3.3 ) {$A_{13}$};
\node[color=blue] at ( 1.0,-3.8 ) {$A_{14}$};
\node[color=blue] at (-1.2, 3.2 ) {$A_{15}$};
\node[color=blue] at (-1.2,-3.4 ) {$A_{16}$};

\node[color=red] at ( 3.6,-0.1 ) {$A_{17}$};
\node[color=red] at ( 2.5,-0.4 ) {$A_{18}$};
\node[color=red] at (-1.7, 0.2 ) {$A_{19}$};
\node[color=red] at (-3.9,-0.3 ) {$A_{20}$};
\end{tikzpicture}
\end{center}
sind die Punkte
\[
\begin{aligned}
A_1&=(1,1,1),&
A_{10}&=(0,-\varphi,1/\varphi),&
A_{13}&=(1/\varphi,0,\varphi),\\
A_{15}&=(-1/\varphi,0,\varphi),&
A_{18}&=(\varphi,-1/\varphi,0)
\end{aligned}
\qquad
\text{mit}
\quad
\varphi=\frac{1+\sqrt{5}}2
\]
bekannt.
\begin{teilaufgaben}
\item
Finden Sie eine Drehmatrix $R$, welche $A_{18}$ auf $A_{10}$ abbildet
und die Seitenfläche $A_1A_9A_5A_{15}A_{13}$ unverändert lässt.
\item
Finden Sie den Drehwinkel der von $R$ beschriebenen Drehung.
\end{teilaufgaben}

\begin{hinweis}
Führen sie die nötigen Matrizenrechnungen mit dem Taschenrechner durch.
\end{hinweis}

\thema{Abbildungsmatrix}
\thema{Drehmatrix}
\thema{inverse Matrix}
\thema{Drehwinkel}

\begin{loesung}
\begin{teilaufgaben}
\item
Ausser der bereits bekannten Zuordnung $A_{18}\rightarrow A_{10}$,
können wir aus der Zeichnung zusätzlich ablesen
$A_1\rightarrow A_{13}$
und
$A_{13}\rightarrow A_{15}$.
Diese Zuordnungen können wir jetzt vektoriell schreiben:
\begin{equation}
R
\underbrace{
\begin{pmatrix}
1&   \varphi&1/\varphi\\
1&-1/\varphi&        0\\
1&         0&  \varphi
\end{pmatrix}
}_{\displaystyle=B_1}
=
\underbrace{
\begin{pmatrix}
1/\varphi&        0&-1/\varphi\\
        0& -\varphi&         0\\
  \varphi&1/\varphi&   \varphi
\end{pmatrix}
}_{\displaystyle=B_2}
\qquad\Rightarrow\qquad
R=B_2B_1^{-1}.
\label{40000025:b1}
\end{equation}
Die numerische Rechnung ergibt
\begin{align*}
B_1^{-1}
&=
\begin{pmatrix}
   0.30902&  0.80902& -0.11803\\
   0.50000& -0.30902& -0.19098\\
  -0.19098& -0.50000&  0.69098
\end{pmatrix}
\\
R
=
B_2B_1^{-1}
&=
\begin{pmatrix}
   0.61803&  0.00000& -0.61803\\
   0.00000& -1.61803&  0.00000\\
   1.61803&  0.61803&  1.61803
\end{pmatrix}
\begin{pmatrix}
   0.30902&  0.80902& -0.11803\\
   0.50000& -0.30902& -0.19098\\
  -0.19098& -0.50000&  0.69098
\end{pmatrix}
\\
&=
\begin{pmatrix}
   0.30902&  0.80902& -0.50000\\
  -0.80902&  0.50000&  0.30902\\
   0.50000&  0.30902&  0.80902
\end{pmatrix}
\end{align*}
\item
Den Drehwinkel können wir mit der Spurformel berechnen:
\begin{align*}
\cos\alpha
&=
\frac{\operatorname{Spur}-1}{2}
=
\frac{\varphi-1}{2}
=
0.30902
\\
\Rightarrow\qquad
\alpha
&=
72^\circ.
\qedhere
\end{align*}
\end{teilaufgaben}
\end{loesung}

\begin{diskussion}
Die Inverse $B_1^{-1}$ in \eqref{40000025:b1}
können wir auch exakt berechnen.
Dazu beachten wir, dass sich alle Resultate während der Durchführung des
Gaussalgorithmus in der Form $a+b\varphi$ geschrieben werden können,
wobe $a$ und $b$ rationale Zahlen sind.
Zum Beispiel ist
\begin{align*}
\varphi^2
&=
\varphi +1
\\
\frac1{\varphi}
&=
\frac{2}{1+\sqrt{5}}
=
\frac{2(1-\sqrt{5})}{1-5}
=
\frac{\sqrt{5}-1}{2}
=
\varphi-1
\\
\frac{1}{1-2\varphi}
&=
\frac{1}{1-1-\sqrt{5}}
=
\frac{\sqrt{5}}{5}
=
\frac{2\varphi-1}{5}
\\
\frac52\frac1{\varphi+2}
&=
\frac{3-\varphi}2
\end{align*}
Die zweite Beziehung erlaubt, die Terme $1/\varphi$ in $B_1$ zum Verschwinden
zu bringen.
Die letzten zwei Beziehungen sind die Reziproken der Pivot-Elemente, die
bei der Durchführung des Gauss-Algorithmus zur Berechnung der Inversen
auftreten.
In der nachfolgenden Rechnung werden nach jedem Schritt $\varphi^2$
durch $\varphi+1$ ersetzt:
\begin{align*}
\begin{tabular}{|>{$}c<{$}>{$}c<{$}>{$}c<{$}|>{$}c<{$}>{$}c<{$}>{$}c<{$}|}
\hline
1&   \varphi&1/\varphi& 1& 0& 0\\
1&-1/\varphi&        0& 0& 1& 0\\
1&         0&  \varphi& 0& 0& 1\\
\hline
\end{tabular}
&\rightarrow
\begin{tabular}{|>{$}c<{$}>{$}c<{$}>{$}c<{$}|>{$}c<{$}>{$}c<{$}>{$}c<{$}|}
\hline
1&   \varphi&\varphi-1& 1& 0& 0\\
1&-\varphi+1&        0& 0& 1& 0\\
1&         0&  \varphi& 0& 0& 1\\
\hline
\end{tabular}
\\
&\rightarrow
\begin{tabular}{|>{$}c<{$}>{$}c<{$}>{$}c<{$}|>{$}c<{$}>{$}c<{$}>{$}c<{$}|}
\hline
1&   \varphi&\varphi-1& 1& 0& 0\\
0&1-2\varphi&1-\varphi&-1& 1& 0\\
0& - \varphi&        1&-1& 0& 1\\
\hline
\end{tabular}
\\
&\rightarrow
\begin{tabular}{|>{$}c<{$}>{$}c<{$}>{$}c<{$}|>{$}c<{$}>{$}c<{$}>{$}c<{$}|}
\hline
1&\varphi&     \varphi-1&             1&             0& 0\\
0&      1& (3-\varphi)/5&(2\varphi-1)/5&(1-2\varphi)/5& 0\\
0&      0&2(\varphi+2)/5& (\varphi-3)/5& (\varphi+2)/5& 1\\
\hline
\end{tabular}
\\
&\rightarrow
\begin{tabular}{|>{$}c<{$}>{$}c<{$}>{$}c<{$}|>{$}c<{$}>{$}c<{$}>{$}c<{$}|}
\hline
1&\varphi&     \varphi-1&             1&             0&            0\\
0&      1& (3-\varphi)/5&(2\varphi-1)/5&(1-2\varphi)/5&            0\\
0&      0&             1&   \varphi/2-1&      -\frac12&(3-\varphi)/2\\
\hline
\end{tabular}
\\
&\rightarrow
\begin{tabular}{|>{$}c<{$}>{$}c<{$}>{$}c<{$}|>{$}c<{$}>{$}c<{$}>{$}c<{$}|}
\hline
1&\varphi&             0&(2\varphi-1)/2&(\varphi-1)/2&(4-3\varphi)/2\\
0&      1&             0&\frac12       &(1-\varphi)/2&(\varphi-2)/2 \\
0&      0&             1&   \varphi/2-1&      -\frac12&(3-\varphi)/2\\
\hline
\end{tabular}
\\
&\rightarrow
\begin{tabular}{|>{$}c<{$}>{$}c<{$}>{$}c<{$}|>{$}c<{$}>{$}c<{$}>{$}c<{$}|}
\hline
1&0&0&(\varphi-1)/2&\varphi/2    &(3-2\varphi)/2\\
0&1&0&\frac12      &(1-\varphi)/2&(\varphi-2)/2 \\
0&0&1&  \varphi/2-1&     -\frac12&(3-\varphi)/2 \\
\hline
\end{tabular}
\end{align*}
Wir lesen ab 
\[
B_1^{-1}
=
\begin{pmatrix}
(\varphi-1)/2&   \varphi/2 &(3-2\varphi)/2 \\
\frac12      &(1-\varphi)/2&(\varphi-2)/2  \\
(\varphi-2)/2&-\frac12     &(3-\varphi)/2
\end{pmatrix}.
\]
Die Drehmatrix ist
\begin{align*}
R
&=
B_2B_1^{-1}
=
\begin{pmatrix}
(\varphi-1)/2& \varphi/2   & -\frac12    \\
-\varphi/2   &\frac12      &(\varphi-1)/2\\
\frac12      &(\varphi-1)/2&\varphi/2
\end{pmatrix}.
\end{align*}
Um den Drehwinkel zu bestimmen, wenden wir die Spurformel auf $R$ an:
\begin{align*}
\cos\alpha
&=
\frac{\varphi-1}{2}
=
\frac{\sqrt{5}-1}{2}
\\
\Rightarrow\qquad
\alpha
&=
72^\circ.
\end{align*}
\end{diskussion}
