Finden Sie die Tangenten an den Kreis mit Mittelpunkt $M=(3,2)$ und
Radius $r=5$, welche durch den Punkt $P=(13,-3)$ verlaufen.

\thema{Kreis}

\begin{loesung}
Das Problem wäre im Wesentlichen gelöst, wenn der Berührungspunkt $Q$ 
bekannt wäre. Dieser Punkt befindet sich sowohl auf der
Tangente, wie auch auf dem Kreis. Er muss also die Kreis-Gleichung
\[
  (\vec q- \vec m)\cdot(\vec q- \vec m) = r^2
\]
und die Tangenten-Gleichung
\[
  (\vec p- \vec m)\cdot(\vec q- \vec m) = r^2
\]
erfüllen.
Setzt man die gegebenen Werte ein, findet man die Gleichungen
\begin{align*}
 \left(\begin{pmatrix}x\\y\end{pmatrix}- \begin{pmatrix}3\\2\end{pmatrix}\right)\cdot
 \left(\begin{pmatrix}x\\y\end{pmatrix}- \begin{pmatrix}3\\2\end{pmatrix}\right) &= 5^2\\
  (x-3)^2 + (y-2)^2&= 25\\
  x^2-6x + 9 + y^2 -4y +4 &= 25\\
  x^2 + y^2 -6x -4y - 12 &= 0
\end{align*}
und 
\begin{align}
 \left(\begin{pmatrix}x\\y\end{pmatrix}- \begin{pmatrix}3\\2\end{pmatrix}\right)\cdot
 \left(\begin{pmatrix}13\\-3\end{pmatrix}- \begin{pmatrix}3\\2\end{pmatrix}\right) &= 5^2\notag\\
  10\,(x-3) -5 (y-2)&= 25\notag\\
  10x -30 -5y +10 &= 25\notag\\
  10x -5y &= 45\notag\\
  y &= 2x-9.
  \label{equ.y}
\end{align}
Durch einsetzen der zweiten Gleichung in die erste, 
kann man die $x$-Koordinate vom Punkt $Q$ finden:
\begin{align*}
  x^2 + (2x-9)^2 -6x -4(2x-9) - 12 &= 0\\
  x^2 + 4x^2 - 36x + 81 -6x -8x + 36 - 12 &= 0\\
  5x^2 - 50x + 105 &= 0\\
  x^2 - 10x + 21 &= 0\\
  (x-7)(x-3) &= 0
\end{align*}
Es gibt zwei Lösungen für $x$
\[
  \quad x_1 = 7 \quad\text{und}\quad  x_2 = 3.
\]
Die entsprechenden $y$-Koordinaten können mit Gleichung (\ref{equ.y})
gefunden werden:
\[
  \quad y_1 = 5 \quad\text{und}\quad  y_2 = -3.
\]
Die Geradengleichugnen in Parameterform der beiden Tangenten sind somit
\[
\vec t_1 = \vec p + s\cdot (\vec q_{1} -\vec p ) 
= \begin{pmatrix}13\\-3\end{pmatrix} + s\cdot \begin{pmatrix}-6\\8\end{pmatrix}
\]
und
\[
\vec t_2 = \vec p + s\cdot (\vec q_{2} - \vec p ) 
= \begin{pmatrix}13\\-3\end{pmatrix} + s\cdot \begin{pmatrix}-10\\0\end{pmatrix}.
\]
\end{loesung}
