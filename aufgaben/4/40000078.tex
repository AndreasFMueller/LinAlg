Gegeben sind die Vektoren
\[
\vec{a}_1
=
\begin{pmatrix}
 \frac{1}{\sqrt{2}}\\
-\frac{1}{\sqrt{2}}\\
\end{pmatrix}
\qquad\text{und}\qquad
\vec{a}_2
=
\begin{pmatrix}
0\\
1
\end{pmatrix}
.
\]
Beide Vektoren haben Länge $1$.
Bei der Konstruktion einer orthonormierten Basis kann man
also $\vec{b}_1=\vec{a}_1$ wählen.
\begin{teilaufgaben}
\item
Korrigieren Sie den Vektor $\vec{a}_2$ nach dem Verfahren von
Gram-Schmidt, derart, dass der neue Vektor senkrecht auf $\vec{b}_1$
steht.
\item
Finden Sie einen Einheitsvektor $\vec{b}_2$ derart, dass
$\{\vec{b}_1,\vec{b}_2\}$ eine orthonormierte Basis ist.
\end{teilaufgaben}

\begin{loesung}
\begin{teilaufgaben}
\item Es muss die Projektion auf die Reichung $\vec{b}_1$ subtrahiert werden:
\begin{align}
\vec{a}_2-(\vec{b}_1\cdot\vec{a}_2)\,\vec{b}_1
&=
\begin{pmatrix}
0\\
1
\end{pmatrix}
-
\left(
\begin{pmatrix}
0\\
1
\end{pmatrix}
\cdot
\begin{pmatrix}
 \frac{1}{\sqrt{2}}\\
-\frac{1}{\sqrt{2}}\\
\end{pmatrix}
\right)
\begin{pmatrix}
 \frac{1}{\sqrt{2}}\\
-\frac{1}{\sqrt{2}}\\
\end{pmatrix}
\notag
\\
&=
\begin{pmatrix}
0\\
1
\end{pmatrix}
+
\frac{1}{\sqrt{2}}
\begin{pmatrix}
 \frac{1}{\sqrt{2}}\\
-\frac{1}{\sqrt{2}}\\
\end{pmatrix}
=
\begin{pmatrix}
0\\
1
\end{pmatrix}
+
\begin{pmatrix}
 \frac12\\
-\frac12
\end{pmatrix}
=
\begin{pmatrix}
\frac12\\
\frac12
\end{pmatrix}
\label{40000078:eqn}
\end{align}
\item
Der Vektor \eqref{40000078:eqn} ist kein Einheitsvektor, er hat die 
Länge
\[
|\vec{a}_2-(\vec{b}_1\cdot\vec{a}_2)\,\vec{b}_1|
=
\sqrt{(\textstyle\frac12)^2+(\textstyle\frac12)^2}
=
\frac{1}{\!\sqrt{2}}.
\]
Der normierte Vektor ist daher
\[
\vec{b}_2
=
\frac{
\vec{a}_2-(\vec{b}_1\cdot\vec{a}_2)\,\vec{b}_1
}{
|\vec{a}_2-(\vec{b}_1\cdot\vec{a}_2)\,\vec{b}_1|
}
=
\!\sqrt{2}
\begin{pmatrix}
\frac12\\
\frac12
\end{pmatrix}
=
\frac{\!\sqrt{2}}2
\begin{pmatrix}1\\1\end{pmatrix}.
\qedhere
\]
\end{teilaufgaben}
\end{loesung}
