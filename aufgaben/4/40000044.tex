Ein Objekt soll mit Hilfe von $n$ Kameras im dreidimensionalen Raum
verfolgt werden.
Die Software erkennt das Objekt im Bild jeder Kamera $i$ und berechnet
eine Gerade vom Kamerastandort $\vec p_i$ durch das Objekt.
Der Richtungsvektor dieser Geraden ist $\vec r_i$.
Wegen Ungenauigkeiten der Kalibrierung und Objekterkennung schneiden
sich die Geraden nicht in einem Punkt.

Wir betrachten nur den Fall $n=2$ und die Geraden
\[
\vec p_1+t_1\vec r_1
=
\begin{pmatrix}-5\\0\\0\end{pmatrix}
+
t_1\begin{pmatrix}5\\5\\1\end{pmatrix}
\qquad\text{und}\qquad
\vec p_2+t_2\vec r_2
=
\begin{pmatrix}5\\0\\0\end{pmatrix}
+
t_2\begin{pmatrix}-5\\6\\1\end{pmatrix}.
\]

\thema{Least Squares}

\begin{teilaufgaben}
\item
Stellen Sie ein Gleichungssystem auf, mit dem sich eine bestmögliche
Approximation für den gesuchten Schnittpunkt bestimmen lässt.
\item
Finden Sie die Lösung des Gleichungssystems und eine Approximation für den
Schnittpunkt.
\end{teilaufgaben}

\begin{loesung}
\begin{teilaufgaben}
\item
Die Vektorgleichung
\[
\vec p_1+t_1\vec r_1 = \vec p_2 + t_2\vec r_2
\qquad\Leftrightarrow\qquad
t_1\vec r_1 - t_2\vec r_2 = \vec p_2 -\vec p_1
\]
besteht aus drei Gleichungen, die die zwei Parameter $t_1$ und $t_2$ 
bestimmen. 
Als überbestimmtes Gleichungssytem hat es im allgmeinen keine
Lösung, eine Least-Squares-Lösung kann aber immer gefunden werden.

Setzt man die Zahlen ein, erhält man das Gleichungssystem
\[
A
\begin{pmatrix}
t_1\\t_2
\end{pmatrix}
=
\begin{pmatrix}
5& 5\\
5&-6\\
1&-1
\end{pmatrix}
\begin{pmatrix}
t_1\\t_2
\end{pmatrix}
=
\begin{pmatrix}
10\\0\\0
\end{pmatrix}
=
b.
\]
Das Lösungsverfahren für überbestimmte Gleichungssysteme verlangt,
$A^tA$ und $A^tb$ zu berechnen:
\[
A^tA=\begin{pmatrix}51&-6\\-6&62\end{pmatrix}
\qquad\text{und}\qquad
A^tb=\begin{pmatrix}50\\50\end{pmatrix}.
\]
\item
Die Lösung des Gleichungssystems ergibt
\[
t_1 = 1.08765,
\qquad
t_2 = 0.91171.
\]
Die beiden Werte $t_i$ gehören zu den Punkten
\[
\vec p_1+t_1\vec r_1
=
\begin{pmatrix}
   0.43826\\
   5.43826\\
   1.08765
\end{pmatrix}
\qquad\text{und}\qquad
\vec p_2+t_2\vec r_2
=
\begin{pmatrix}
   0.44146\\
   5.47025\\
   0.91171
\end{pmatrix}.
\]
Eine mögliche Schätzung für den Schnittpunkt wäre deren Mittelwert
\begin{equation}
\begin{pmatrix}
   0.43986\\
   5.45425\\
   0.99968
\end{pmatrix}
\label{40000044:loes}
\qedhere
\end{equation}
\end{teilaufgaben}
\end{loesung}

\begin{diskussion}
Alternativ hätte man auch wie folgt vorgehen können.
Gesucht sind nicht nur die Variablen $t_1$ und $t_2$ sondern zusätzlich die
Koordinaten $x$, $y$ und $z$ des gesuchten Schnittpunktes.
Diese könnten mit dem überbestimmten Gleichungssystem
\begin{center}
\begin{tabular}{|>{$}c<{$}>{$}c<{$}>{$}c<{$}>{$}c<{$}>{$}c<{$}|>{$}c<{$}|}
\hline
x&y&z&t_1&t_2&  \\
\hline
1&0&0& -5&  0&-5\\
0&1&0& -5&  0& 0\\
0&0&1& -1&  0& 0\\
1&0&0&  0& -5& 5\\
0&1&0&  0&  6& 0\\
0&0&1&  0&  1& 0\\
\hline
\end{tabular}
\end{center}
bestimmt werden, welches wieder die Lösung
\eqref{40000044:loes}
liefert.
Dieses Verfahren lässt sich aber natürlich auf offensichtliche Weise auf
eine beliebige Anzahl von Kameras verallgemeinern.

Das Gleichungssystem für die Unbekannten $x$, $y$, $z$ und $t_i$ mit
$1\le i\le n$ ist ein $3n\times (3+n)$-Gleichungsystem, für $n>1$ ist
es überbestimmt.
Die Matrix $A$ enthält in den ersten drei Spalten jeweils
Einheitsmatrix-Blöcke.
In der Spalten $i+3$ steht in den Zeilen $3(i-1)+1$ bis $3(i-i)+3$
der Vektor $-\vec r_i$.
Der Rest der Matrix wird mit Nullen gefüllt.
\begin{center}
\begin{tabular}{|>{$}c<{$}>{$}c<{$}>{$}c<{$}|>{$}c<{$}|>{$}c<{$}|>{$}c<{$}|>{$}c<{$}|>{$}c<{$}|}
\hline
x&y&z&t_1&t_2&\dots&t_n&\\
\hline
 &   E  & &-\vec r_1&    0    &\dots &    0    &\vec p_1 \\
 &   E  & &   0     &-\vec r_2&\dots &    0    &\vec p_2 \\
 &\vdots& &         &\vdots   &\ddots&\vdots   &\vdots   \\
 &   E  & &   0     &    0    &\dots &-\vec r_n&\vec p_n \\
\hline
\end{tabular}
\end{center}
Der Vektor $b$ enthält in den Zeilen $3(i-1)+1$ bis $3(i-1)+3$
den Vektor $\vec p_i$.
Die Lösung besteht aus dem optimalen Punkt $P=(x,y,z)$ und den Parameterwerten
$t_i$, die einen Punkt auf der Geraden $i$ möglichst nahe an $P$ ergeben.
\[
A=\begin{pmatrix}
1&0&0&-5& 0\\
0&1&0&-5& 0\\
0&0&1&-1& 0\\
1&0&0& 0& 5\\
0&1&0& 0&-6\\
0&0&1& 0&-1
\end{pmatrix},
\qquad
b=\begin{pmatrix}
-5\\0\\0\\5\\0\\0
\end{pmatrix}
\]
Die Least-Squares-Lösung des überbestimmten Gleichungssystems $Av=b$
kann mit 
\[
A^tA=\begin{pmatrix}
    2 &  0 &  0 & -5 &  5\\
    0 &  2 &  0 & -5 & -6\\
    0 &  0 &  2 & -1 & -1\\
   -5 & -5 & -1 & 51 &  0\\
    5 & -6 & -1 &  0 & 62
\end{pmatrix}
\qquad\text{und}\qquad
A^tb=\begin{pmatrix}
    0 \\
    0 \\
    0 \\
   25 \\
   25
\end{pmatrix}
\]
gefunden werden und ist
\[
\begin{pmatrix}
x\\y\\z\\t_1\\t_2
\end{pmatrix}
=
\begin{pmatrix}
\phantom{-}0.43986\\
\phantom{-}5.45425\\
\phantom{-}0.99968\\
\phantom{-}1.08765\\
         - 0.91171
\end{pmatrix},
\]
was mit der Musterlösung übereinstimmt.
\end{diskussion}

\begin{bewertung}
Least-Squares Ansatz ({\bf L}) 1 Punkt,
Matrix $A$ ({\bf A}) 1 Punkt,
rechte Seite $b$ ({\bf B}) 1 Punkt,
Matrix $A^tA$ ({\bf M}) 1 Punkt,
Vektor $A^tb$ ({\bf R}) 1 Punkt,
Schnittpunkt ({\bf S}) 1 Punkt.
\end{bewertung}


