\definecolor{darkred}{rgb}{0.8,0,0}
Stellen Sie ein Gleichungssystem auf, mit dem die Koeffizienten
$a_0,\dots,a_n$ und $b_0,\dots,b_m$ einer Funktion
\[
f(x)
=
\frac{
a_nx^n+a_{n-1}x^{n-1}+\dots+a_1x+a_0
}{
b_mx^m+b_{m-1}x^{m-1}+\dots+b_1x+b_0
}
\]
bestimmt werden können, die möglichst gut zu gegebenen Datenpunkten
$(x_i,y_i), i=1,\dots,N$ passen.
Die Zahl $N$ der Datenpunkte ist grösser als $n+m+1$.
Das Problem ist in dieser Form nicht eindeutig, da der Bruch beliebig
erweitert werden kann.
Es wird jedoch eindeutig, wenn man $b_m=1$ festlegt.
Der Koeffizient $b_m$ muss also nicht bestimmt werden, sondern darf
als bekannt mit Wert $b_m=1$ angenommen werden.

\begin{hinweis}
Damit die Formeln nicht zu umfangreich werden, dürfen Sie sich auf den Fall
$n=2$ und $m=3$ beschränken.
In diesem Fall sind $n+m+1=6$ Unbekannte zu bestimmen.
\end{hinweis}

\begin{loesung}
Für jeden Datenpunkt $(x_i,y_i)$ sollte gelten
\[
y_i
=
\frac{
a_nx_i^n+a_{n-1}x_i^{n-1}+\dots+a_1x_i+a_0
}{
b_mx_i^m+b_{m-1}x_i^{m-1}+\dots+b_1x_i+b_0
}.
\]
Diese Gleichung ist nicht linear.
Durch Multiplizieren mit dem Nenner wird sie zu
\[
x_i^m
y_i
{\color{darkred}b_m}
+
x_i^{m-1}
y_i
{\color{darkred}b_{m-1}}
+\dots+
x_i
y_i
{\color{darkred}b_1}
+
{\color{darkred}b_0}
=
x_i^n
{\color{darkred}a_n}
+
x_i^{n-1}
{\color{darkred}a_{n-1}}
+\dots+
x_i
{\color{darkred}a_1}
+
{\color{darkred}a_0}.
\]
Diese Gleichung ist linear, aber sie ist nicht in der Standardform, die
wir für ein lineares Gleichungssystem erwarten.
Durch Umstellen und unter Verwendung von ${\color{darkred}b_m}=1$,
erhalten wir
\[
{\color{darkred}a_0}
+
{\color{darkred}a_1}x_i
+\dots+
{\color{darkred}a_{n-1}}x^{n-1}
{\color{darkred}a_{n}}x^n
-
{\color{darkred}b_0}y_i
-
{\color{darkred}b_1}x_iy_i
-\dots-
{\color{darkred}b_{m-1}}x_i^{m-1}y_i
=
{\color{darkred}b_m}x_i^my_i
=
x_i^my_i.
\]
Daraus lässt sich jetzt das Gleichungssystem mit der Koeffizientenmatrix 
$A$ und der rechten Seite $b$ wie folgt ablesen:
\[
\underbrace{
\begin{pmatrix}
1      &x_1    &\dots  &x_1^{n-1} &x_1^n  &-y_1   &-x_1y_1 &\dots &-x_1^{m-1}y_1\\
1      &x_2    &\dots  &x_2^{n-1} &x_2^n  &-y_2   &-x_2y_2 &\dots &-x_2^{m-1}y_2\\
1      &x_3    &\dots  &x_3^{n-1} &x_3^n  &-y_3   &-x_3y_2 &\dots &-x_3^{m-1}y_3\\
\vdots &\vdots &\ddots &\vdots    &\vdots &\vdots &\vdots  &\ddots&\vdots\\
1      &x_N    &\dots  &x_N^{n-1} &x_N^n  &-y_N   &-x_Ny_2 &\dots &-x_N^{m-1}y_N
\end{pmatrix}}_{\displaystyle=A}
\begin{pmatrix}
{\color{darkred}a_0}\\
{\color{darkred}a_1}\\
\vdots\\
{\color{darkred}a_{n-1}}\\
{\color{darkred}a_n}\\
{\color{darkred}b_0}\\
{\color{darkred}b_1}\\
\vdots\\
{\color{darkred}b_{m-1}}
\end{pmatrix}
=
\underbrace{
\begin{pmatrix}
x_1^my_1\\
x_2^my_2\\
x_3^my_3\\
\vdots\\
x_N^my_N
\end{pmatrix}}_{\displaystyle=b}.
\]
Das Gleichungssystem ist jedoch überbstimmt, da es $N$ Gleichungen,
aber nur $n+m+1<N$ Unbekannte hat.
Somit lassen sich daraus die Koeffizienten $a_k$ und $b_k$ noch nicht
exakt bestimmt, dazu muss das
Gleichungssystem mit der Standard-Lösungsmethode für überbestimmte
Gleichungssysteme in ein Gleichungssystem mit $n+m+1$ Gleichungen
umgeformt werden.
Dazu müssen $\transpose{A}A$ und $\transpose{A}b$ berechnet werden.
Die Zeilen von
\[
\transpose{A}
=
\begin{pmatrix}
1             &1             &1            &\dots  &1            \\
x_1           &x_2           &x_3          &\dots  &x_N          \\
%x_1^2&x_2^2&x_3^2&\dots&x_N^2\\
\vdots        &\vdots        &\vdots       &\ddots &\vdots       \\
x_1^{n-1}     &x_2^{n-1}     &x_3^{n-1}    &\dots  &x_N^{n-1}    \\
x_1^{n}       &x_2^{n}       &x_3^{n}      &\dots  &x_N^{n}      \\
-y_1          &-y_2          &-y_3         &\dots  &-y_N         \\
-x_1y_1       &-x_2y_2       &-x_3y_3      &\dots  &-x_Ny_N      \\
\vdots        &\vdots        &\vdots       &\ddots &\vdots       \\
-x_1^{m-1}y_1 &-x_2^{m-1}y_2 &-x_3^{m-1}y_3&\dots  &-x_N^{m-1}y_N
\end{pmatrix}
\]
sind die Spalten von $A$, im Produkt $\transpose{A}A$ treten daher 
Summen von Produkt auf:
\[
\transpose{A}A
=
\begin{small}
\renewcommand{\arraycolsep}{2pt}
\begin{pmatrix*}[c]
N
	& \displaystyle\sum_{i=1}^N x_i
	& \dots
	& \displaystyle\sum_{i=1}^N x_i^{n-1}
	& \displaystyle\sum_{i=1}^N x_i^n
	& -\displaystyle\sum_{i=1}^N y_i
	& -\displaystyle\sum_{i=1}^N x_iy_i
	& \dots
	& -\displaystyle\sum_{i=1}^N x_i^{m-1}y_i
\\
\displaystyle\sum_{i=1}^N x_i
	& \displaystyle\sum_{i=1}^N x_i^2
	& \dots
	& \displaystyle\sum_{i=1}^N x_i^{n}
	& \displaystyle\sum_{i=1}^N x_i^{n+1}
	& -\displaystyle\sum_{i=1}^N x_iy_i
	& -\displaystyle\sum_{i=1}^N x_i^2y_i
	& \dots
	& -\displaystyle\sum_{i=1}^N x_i^{m}y_i
\\
\vdots
	&\vdots
	&\ddots
	&\vdots
	&\vdots
	&\vdots
	&\vdots
	&\ddots
	&\vdots
\\
\displaystyle\sum_{i=1}^N x_i^{n-1}
	& \displaystyle\sum_{i=1}^N x_i^{n}
	& \dots
	& \displaystyle\sum_{i=1}^N x_i^{2n-2}
	& \displaystyle\sum_{i=1}^N x_i^{2n-1}
	& -\displaystyle\sum_{i=1}^N x_i^{n-1}y_i
	& -\displaystyle\sum_{i=1}^N x_i^ny_i
	& \dots
	& -\displaystyle\sum_{i=1}^N x_i^{n+m-2}y_i
\\
\displaystyle\sum_{i=1}^N x_i^{n}
	& \displaystyle\sum_{i=1}^N x_i^{n+1}
	& \dots
	& \displaystyle\sum_{i=1}^N x_i^{2n-1}
	& \displaystyle\sum_{i=1}^N x_i^{2n}
	& -\displaystyle\sum_{i=1}^N x_i^{n}y_i
	& -\displaystyle\sum_{i=1}^N x_i^{n+1}y_i
	& \dots
	& -\displaystyle\sum_{i=1}^N x_i^{n+m-1}y_i
\\
-\displaystyle\sum_{i=1}^N y_i
	& -\displaystyle\sum_{i=1}^N x_iy_i
	& \dots
	& -\displaystyle\sum_{i=1}^N x_i^{n-1}y_i
	& -\displaystyle\sum_{i=1}^N x_i^{n}y_i
	& \displaystyle\sum_{i=1}^N y_i^2
	& \displaystyle\sum_{i=1}^N x_iy_i^2
	& \dots
	& \displaystyle\sum_{i=1}^N x_i^{m-1}y_i^2
\\
-\displaystyle\sum_{i=1}^N x_iy_i
	& -\displaystyle\sum_{i=1}^N x_i^2y_i
	& \dots
	& -\displaystyle\sum_{i=1}^N x_i^{n}y_i
	& -\displaystyle\sum_{i=1}^N x_i^{n+1}y_i
	& \displaystyle\sum_{i=1}^N x_iy_i^2
	& \displaystyle\sum_{i=1}^N x_i^2y_i^2
	& \dots
	& \displaystyle\sum_{i=1}^N x_i^{m}y_i^2
\\
\vdots
	&\vdots
	&\ddots
	&\vdots
	&\vdots
	&\vdots
	&\vdots
	&\ddots
	&\vdots
\\
-\displaystyle\sum_{i=1}^N x_i^{m-1}y_i
	& -\displaystyle\sum_{i=1}^N x_i^my_i
	& \dots
	& -\displaystyle\sum_{i=1}^N x_i^{n+m-2}y_i
	& -\displaystyle\sum_{i=1}^N x_i^{n+m-1}y_i
	& \displaystyle\sum_{i=1}^N x_i^{m-1}y_i^2
	& \displaystyle\sum_{i=1}^N x_i^{m}y_i^2
	& \dots
	& \displaystyle\sum_{i=1}^N x_i^{2m-2}y_i^2
\end{pmatrix*}
\end{small}
\]
Für den Vektor $b$ folgt entsprechend
\[
\transpose{A}b
=
\begin{pmatrix}
-\displaystyle\sum_{i=1}^N x_i^my_i \\
-\displaystyle\sum_{i=1}^N x_i^{m+1}y_i \\
\vdots \\
-\displaystyle\sum_{i=1}^N x_i^{n+m-1}y_i \\
-\displaystyle\sum_{i=1}^N x_i^{n+m}y_i \\
\displaystyle\sum_{i=1}^N x_i^{m}y_i^2 \\
\displaystyle\sum_{i=1}^N x_i^{m+1}y_i^2 \\
\vdots \\
\displaystyle\sum_{i=1}^N x_i^{2m}y_i^2 
\end{pmatrix}.
\]
\begin{figure}
\ainput{punkte.tex}
\definecolor{darkgreen}{rgb}{0,0.6,0}
\centering
\def\dx{4}
\def\dy{0.1}
\def\punkt#1#2{ \fill[color=darkred]
	({\dx*(#1)},{\dy*(#2)}) circle[radius=0.08]; }
\begin{tikzpicture}[>=latex,thick]
\draw[->] (-0.1,0) -- ({\dx*4+0.3},0) coordinate[label={$x$}];
\draw[->] (0,-4.1) -- (0,4.3) coordinate[label={right:$y$}];
\foreach \x in {1,2,3}{
	\draw ({\dx*\x},-0.05) -- ({\dx*\x},0.05);
	\node at ({\dx*\x},0) [below right] {$\x$};
}
\foreach \y in {10,20,30,40}{
	\node at (0,{\dy*\y}) [left] {$\y$};
	\draw (-0.05,{\dy*\y}) -- (0.05,{\dy*\y});
	\node at (0,{-\dy*\y}) [left] {$-\y$};
	\draw (-0.05,{-\dy*\y}) -- (0.05,{-\dy*\y});
}
\begin{scope}
\clip (-0.1,-4.1) rectangle ({\dx*4+0.1},4.1);
\draw[color=blue,line width=1.4pt] plot[domain=0:0.98,samples=50]
	({\dx*\x},{\dy*((\x+1)*\x+1)/(((\x-6)*\x+11)*\x-6)});
\draw[color=blue,line width=1.4pt] plot[domain=1.02:1.98,samples=50]
	({\dx*\x},{\dy*((\x+1)*\x+1)/(((\x-6)*\x+11)*\x-6)});
\draw[color=blue,line width=1.4pt] plot[domain=2.02:2.98,samples=50]
	({\dx*\x},{\dy*((\x+1)*\x+1)/(((\x-6)*\x+11)*\x-6)});
\draw[color=blue,line width=1.4pt] plot[domain=3.02:4,samples=50]
	({\dx*\x},{\dy*((\x+1)*\x+1)/(((\x-6)*\x+11)*\x-6)});
\end{scope}
\draw[color=blue,line width=0.3pt] ({\dx*1},-4.1) -- ({\dx*1},4.1);
\draw[color=blue,line width=0.3pt] ({\dx*2},-4.1) -- ({\dx*2},4.1);
\draw[color=blue,line width=0.3pt] ({\dx*3},-4.1) -- ({\dx*3},4.1);
\begin{scope}
\clip (-0.1,-4.1) rectangle ({\dx*4+0.1},4.1);
\kurven{darkgreen}
\punkte
\end{scope}
\end{tikzpicture}
\caption{Anpassung einer rationalen Funktion an vorgegebene Punkt
\label{60000042:image}}
\end{figure}%
In Abbildung~\ref{60000042:image} wird die Funktion
\[
f(x)
=
\frac{a_2x^2+a_1x+a_0}{x^3+b_2x^2+b_1x+b_0}
\]
an 100 Punkte angepasst, die mit Hilfe der Funktion
\[
y = \frac{x^2+x+1}{x^3-6x^2+11x-6}
\]
erzeugt worden sind.
Den Punkten ist ausserdem ein normalverteilter Fehler überlagert.
\end{loesung}

\begin{bewertung}
Lineare Gleichung ({\bf L}) 1 Punkt,
Standardform mit Variablen links und Konstanten rechts ({\bf S}) 1 Punkt,
Matrix $A$ ({\bf A}) 1 Punkt,
Vektor $b$ ({\bf B}) 1 Punkt,
quadratische Matrix ({\bf Q}) 1 Punkt,
rechte Seite ({\bf R}) 1 Punkt.
\end{bewertung}
