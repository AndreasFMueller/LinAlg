Gegeben ist eine Menge von $n$ Zahlenpaaren $(x_i,y_i)$, $1\le i \le n$.
Finden Sie Koeffizienten $a$ und $b$ so, dass die Gleichungen
\[
ax_i +b=y_i,\qquad 1\le i\le n
\]
``möglichst gut'' erfüllt sind.
Finden sie eine numerische Lösung für die Zahlenwerte der folgenden
Tabelle:
\begin{center}
\begin{tabular}{|>{$}c<{$}|>{$}r<{$}|>{$}r<{$}|}
\hline
i&x_i&y_i\\
\hline
1&0&0.9669848\\
2&1&1.9823312\\
3&2&2.8627041\\
4&3&3.9943533\\
\hline
\end{tabular}
\end{center}

\thema{Least Squares}

\begin{loesung}
Das überbestimmte Gleichungssystem ist
\[
\begin{linsys}{2}
x_1{\color{red}a}&+&{\color{red}b}&=&y_1\\
x_2{\color{red}a}&+&{\color{red}b}&=&y_2\\
x_3{\color{red}a}&+&{\color{red}b}&=&y_3\\
x_4{\color{red}a}&+&{\color{red}b}&=&y_4\\
\end{linsys}
\]
Diese Gleichungssystem hat als Matrix und Vektor der rechten Seiten:
\begin{align*}
A&=\begin{pmatrix}
0&1\\
1&1\\
2&1\\
3&1
\end{pmatrix},
&
b&=
\begin{pmatrix}
0.9669848\\
1.9823312\\
2.8627041\\
3.9943533
\end{pmatrix}
\end{align*}
Das Lösungsverfahren für diese Art von Problem besagt, dass man das
Gleichungssystem
\[
A^tA\begin{pmatrix}{\color{red}a}\\{\color{red}b}\end{pmatrix}=A^tb
\]
lösen muss. Die Matrix $A^tA$ und der Verktor $A^tb$ sind
\begin{align*}
A^tA&=
\begin{pmatrix}
0&1&2&3\\
1&1&1&1
\end{pmatrix}
\begin{pmatrix}
0&1\\
1&1\\
2&1\\
3&1
\end{pmatrix}
=
\begin{pmatrix}
14&6\\
 6&4
\end{pmatrix},
\\
A^tb&=
\begin{pmatrix}
0&1&2&3\\
1&1&1&1
\end{pmatrix}
\begin{pmatrix}
0.9669848\\
1.9823312\\
2.8627041\\
3.9943533
\end{pmatrix}
=\begin{pmatrix}
19.6908\\
9.806373
\end{pmatrix}.
\end{align*}
Die Lösung mit dem Gaussalgorithmus liefert
\begin{align*}
\begin{tabular}{|>{$}c<{$}>{$}c<{$}|>{$}c<{$}|}
\hline
14&6&19.6908\\
 6&4&9.806373\\
\hline
\end{tabular}
&\rightarrow
\begin{tabular}{|>{$}c<{$}>{$}c<{$}|>{$}c<{$}|}
\hline
 1&0&0.99625\\
 0&1&0.95722\\
\hline
\end{tabular}
\end{align*}
Die gesuchten Koeffizienten sind also 
$a=0.99625$ und $b=0.95722$.
\end{loesung}

