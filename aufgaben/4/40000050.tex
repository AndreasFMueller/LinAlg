%a =
%
%   1   4   3  -3
%   0  -3   1   2
%   0   0  -1   4
%
%ans =
%
%    1    0    0   17
%    0    1    0   -2
%    0    0    1   -4
%
%b =
%
%   1   0   0
%   3   1   0
%  -1   2   1
%
%ans =
%
%    1    4    3   -3
%    3    9   10   -7
%   -1  -10   -2   11
Die beiden Ebenen mit den Gleichungen
\[
\begin{linsys}{3}
 x&+&4y&+& 3z&=&-3\\
3x&+&9y&+&10z&=&-7
\end{linsys}
\]
haben eine Schnittgerade $g$.
Ausserdem sei die Ebene $\sigma$ mit der Normalen
\[
\vec n=
\begin{pmatrix}1\\10\\2\end{pmatrix}
\]
durch den Punkt $P=(1,-1,-1)$ gegeben. Finden Sie den Durchstosspunkt
von $g$ durch $\sigma$.

\thema{Durchstosspunkt}

\begin{loesung}
Es geht darum den gemeinsamen Schnittpunkt der beiden Ebenen mit der
dritten gegebenen Ebene zu finden. Von letzterer muss erst noch die
Gleichung gefunden werden. Dazu verwendet man zweckmässigerweise
die Normalenform, also
\[
\begin{pmatrix}1\\10\\2\end{pmatrix}
\cdot\left(
\begin{pmatrix}x\\y\\z\end{pmatrix}
-
\begin{pmatrix}1\\-1\\-1 \end{pmatrix}
\right)
=
x+10y+2z+11=0
\]
Das zu lösende Gleichungssystem ist jetzt also
\[
\begin{linsys}{3}
 x&+&4y&+& 3z&=&-3\\
3x&+&9y&+&10z&=&-7\\
x&+&10y&+& 2z&=&-11\\
\end{linsys}
\]
Die Lösung findet man am einfachsten mit dem Gauss-Algorithmus,
der in diesem Fall wie folgt aussieht:
\begin{align*}
\begin{tabular}{|>{$}c<{$}>{$}c<{$}>{$}c<{$}|>{$}c<{$}|}
\hline
1& 4& 3& -3\\
3& 9&10& -7\\
1&10& 2&-11\\
\hline
\end{tabular}
&
\rightarrow
\begin{tabular}{|>{$}c<{$}>{$}c<{$}>{$}c<{$}|>{$}c<{$}|}
\hline
1& 4& 3& -3\\
0&-3& 1&  2\\
0& 6&-1& -8\\
\hline
\end{tabular}
\rightarrow
\begin{tabular}{|>{$}c<{$}>{$}c<{$}>{$}c<{$}|>{$}c<{$}|}
\hline
1& 4&       3&      -3\\
0& 1&-\frac13&-\frac23\\
0& 0&       1&      -4\\
\hline
\end{tabular}
\\
&
\rightarrow
\begin{tabular}{|>{$}c<{$}>{$}c<{$}>{$}c<{$}|>{$}c<{$}|}
\hline
1& 4&       0&       9\\
0& 1&       0&      -2\\
0& 0&       1&      -4\\
\hline
\end{tabular}
\rightarrow
\begin{tabular}{|>{$}c<{$}>{$}c<{$}>{$}c<{$}|>{$}c<{$}|}
\hline
1& 0&       0&      17\\
0& 1&       0&      -2\\
0& 0&       1&      -4\\
\hline
\end{tabular}
\end{align*}
Der gemeinsame Schnittpunkt ist also $S=(17,-2,-4)$. Zur Kontrolle
setzen wir in die ürsprünglichen Gleichungen ein
\[
\begin{pmatrix}
1&4& 3\\
3&9&10\\
1&10&2
\end{pmatrix}
\begin{pmatrix}17\\-2\\-4\end{pmatrix}
=
\begin{pmatrix}
17-8-12\\
51-18-40\\
17-20-8
\end{pmatrix}
=
\begin{pmatrix}
-3\\
-7\\
-11
\end{pmatrix}.
\qedhere
\]
\end{loesung}

