Betrachten Sie die Matrix
\[
A=
\begin{pmatrix}
1&2&0&0&0\\
2&3&4&0&0\\
0&4&5&6&0\\
0&0&6&7&8\\
0&0&0&8&9
\end{pmatrix}.
\]
\begin{teilaufgaben}
\item Ist $A$ orthogonal?
\item Berechnen Sie die Determinante von $A$.
\end{teilaufgaben}

\begin{loesung}
\begin{teilaufgaben}
\item
Die Spalten einer orthogonalen Matrix sind orthonormiert,
insbesondere haben Sie l"ange $1$. Jede Spalte von $A$ hat mindestens eine
Koordinate mit Betrag $>1$, sie kann also nicht L"ange $1$ haben.
Daher kann $A$ auch nicht orthogonal sein.

Alternativ kann man argumentieren, dass eine orthogonale
Matrix die Gleichung $AA^t=E$ erf"ullen m"usste. Die Matrix $A$
ist symmetrisch, also $A=A^t$. Wir berechnen das Element links
oben im Produkt $AA^t=A^2$:
\[
\begin{pmatrix}
1&2&0&0&0\\
2&3&4&0&0\\
0&4&5&6&0\\
0&0&6&7&8\\
0&0&0&8&9
\end{pmatrix}
\begin{pmatrix}
1&2&0&0&0\\
2&3&4&0&0\\
0&4&5&6&0\\
0&0&6&7&8\\
0&0&0&8&9
\end{pmatrix}
=
\begin{pmatrix}
5&?&?&?&?\\
?&?&?&?&?\\
?&?&?&?&?\\
?&?&?&?&?\\
?&?&?&?&?
\end{pmatrix}
\ne E,
\]
die Matrix $A$ kann also nicht orthogonal sein.
\item
Wir berechnen der Determinante mit dem Entwicklungssatz, indem wir nach der
ersten Spalte entwickeln:
\begin{align*}
\det(A)=
&
=
1\cdot
\left|\,\begin{matrix}
3&4&0&0\\
4&5&6&0\\
0&6&7&8\\
0&0&8&9
\end{matrix}\,\right|
-2\cdot
\left|\,\begin{matrix}
2&0&0&0\\
4&5&6&0\\
0&6&7&8\\
0&0&8&9
\end{matrix}\,\right|
\\
&
=
1\cdot 3\cdot
\left|\,\begin{matrix}
5&6&0\\
6&7&8\\
0&8&9
\end{matrix}\,\right|
-1\cdot 4\cdot
\left|\,\begin{matrix}
4&0&0\\
6&7&8\\
0&8&9
\end{matrix}\,\right|
-2\cdot2\cdot
\left|\,\begin{matrix}
5&6&0\\
6&7&8\\
0&8&9
\end{matrix}\,\right|
\\
&
=
(1\cdot 3-2\cdot 2)
\left|\,\begin{matrix}
5&6&0\\
6&7&8\\
0&8&9
\end{matrix}\,\right|
-4\cdot 4
\left|\,\begin{matrix}
7&8\\
8&9
\end{matrix}\,\right|
\\
&
=
-(5\cdot7\cdot9-8\cdot8\cdot 5-6\cdot 6\cdot 9)-16(7\cdot 9-8\cdot 8)
\\
&
=
-(315-320-324)-16(-1)=-315+320+324+16
\\
&
=345.
\end{align*}
Dabei wurde im zweiten Schritt die erste Determinante durch Entwicklung
nach der ersten Spalten auf $3\times 3$-Determinanten reduziert,
und die zweite durch Entwicklung nach der ersten Zeile. F"ur die
zweite $3\times 3$-Determinante wurde nochmals der Entwicklungssatz angewandt,
die anderen $3\times 3$-Determi\-nanten wurden mit der Sarrusschen Formel
bestimmt.

Alternativ kann die Determinante auch mit dem Gauss-Algorithmus bestimmt
werden.
\begin{align*}
\begin{tabular}{|>{$}c<{$}>{$}c<{$}>{$}c<{$}>{$}c<{$}>{$}c<{$}|}
\hline
1&2&0&0&0\\
2&3&4&0&0\\
0&4&5&6&0\\
0&0&6&7&8\\
0&0&0&8&9\\
\hline
\end{tabular}
&
\rightarrow
\begin{tabular}{|>{$}c<{$}>{$}c<{$}>{$}c<{$}>{$}c<{$}>{$}c<{$}|}
\hline
1&2&0&0&0\\
0&-1&4&0&0\\
0&4&5&6&0\\
0&0&6&7&8\\
0&0&0&8&9\\
\hline
\end{tabular}
\rightarrow
\begin{tabular}{|>{$}c<{$}>{$}c<{$}>{$}c<{$}>{$}c<{$}>{$}c<{$}|}
\hline
1&2&0&0&0\\
0&1&-4&0&0\\
0&0&21&6&0\\
0&0&6&7&8\\
0&0&0&8&9\\
\hline
\end{tabular}
\\
&
\rightarrow
\begin{tabular}{|>{$}c<{$}>{$}c<{$}>{$}c<{$}>{$}c<{$}>{$}c<{$}|}
\hline
1&2&0&0&0\\
0&1&-4&0&0\\
0&0&1&\frac{2}{7}&0\\
0&0&0&\frac{37}{7}&8\\
0&0&0&8&9\\
\hline
\end{tabular}
\rightarrow
\begin{tabular}{|>{$}c<{$}>{$}c<{$}>{$}c<{$}>{$}c<{$}>{$}c<{$}|}
\hline
1&2&0&0&0\\
0&1&-4&0&0\\
0&0&1&\frac{2}{7}&0\\
0&0&0&1&\frac{56}{37}\\
0&0&0&0&-\frac{115}{37}\\
\hline
\end{tabular}
\end{align*}
Aus den Pivot-Elementen kann jetzt die Determinante berechnet
werden:
\[
\det(A)
=
1\cdot(-1)\cdot21\cdot\frac{37}{7}\cdot(-\frac{115}{37})
=3 \cdot 115=345
\]
Insbesondere erh"alt man die gleiche Determinante wie mit dem Entwicklungssatz.
\end{teilaufgaben}
\end{loesung}

