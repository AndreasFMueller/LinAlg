In der Vorlesung wurde gezeigt, dass die Spiegelung an einer Ebene
mit der Normalen $\vec n$ mit $|\vec n|=1$ durch die Matrix
\[
S=E-2\vec n\vec n^t
\]
gegeben ist.
\begin{teilaufgaben}
\item Kontrollieren Sie durch nachrechnen, dass zweimalige Spiegelung einen
Vektor nicht ändert.
\item Ist die Matrix $S$ symmetrisch?
\item Ist $S$ orthogonal?
\end{teilaufgaben}

\begin{loesung}
\begin{teilaufgaben}
\item
Wir müssen $S^2$ berechnen:
\begin{align*}
S^2
&=
(E-2\vec n\vec n^t) (E-2\vec n\vec n^t)
=
E-4\vec n\vec n^t +4\vec n\underbrace{\vec n^t \vec  n}_{\displaystyle=1}\vec n^t
=
E-4\vec n\vec n^t +4\vec n \vec n^t
=
E.
\end{align*}
\item
Die Matrix ist symmetrisch, wenn $S^t=S$ ist.
Also
\[
S^t
=
(E-2\vec n\vec n^t)^t
=
E^t - 2(\vec n^t)^t\vec n^t
=
E-2\vec n\vec n^t
=
S.
\]
Die Matrix $S$ ist daher symmetrisch.
\item
Eine Matrix ist orthogonal, wenn $A^tA=E$ ist.
Da $S$ symmetrisch ist, ist $S^t=S$ und damit
$S^tS=SS=E$ nach a).
\end{teilaufgaben}
\end{loesung}

