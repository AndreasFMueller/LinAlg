Gegeben sind drei Punkte $B_1$, $B_2$ und $B_3$ in der $(x_1,x_2)$-Ebene.
In der Ebene werden {\em baryzentrische Koordinaten} $(w_1,w_2,w_3)$
im Bezug auf diese Punkte so definiert, dass die vorgegebenen Punkte
folgende Koordinaten bekommen:
\begin{center}
\begin{tabular}{|>{$}c<{$}|>{$}c<{$}|}
\hline
\text{Punkte}&\text{baryzentrische Koordinaten}\\
\hline
B_1&(1,0,0)\\
B_2&(0,1,0)\\
B_3&(0,0,1)\\
\hline
\end{tabular}
\end{center}
Die Seite $B_1B_2$ ist also dadurch charakterisiert, dass dort sowohl
$w_3$ immer verschwindet, und analog f"ur die anderen Seiten
(Abbildung~\ref{40000035:bild}).
\begin{figure}
\centering
\includeagraphics[]{bary-1.pdf}
\caption{Barizentrische Koordinaten eines Dreiecks $\triangle B_1B_2B_3$
\label{40000035:bild}}
\end{figure}

Die Koordinaten sind nicht unabh"angig, es gilt immer $w_1+w_2+w_3=1$.
Daher ist es auch nicht m"oglich, eine $3\times 2$-Matrix zu finden, mit
der man die Koordinaten $(x_1,x_2)$ in der Ebene in die drei Koordinaten
$(w_1,w_2,w_3)$ umrechnen kann, eine solche Matrix w"urde den Punkt $(0,0)$
immer auf $(0,0,0)$ abbilden.

Setzt man dagegen f"ur alle Punkte $x_3=1$, dann kann man eine $3\times 3$-Matrix
$A$ angeben,
mit der man $x$-Koordinaten in $w$-Koordinaten umrechnen kann.
\begin{teilaufgaben}
\item
Finden Sie eine solche Matrix.
\item
Berechnen Sie die Matrix f"ur die Punkte $B_1=(4,1)$, $B_2=(16,-1)$ und
$B_3=(10,12)$.
\item 
Wenden Sie die Matrix auf den Schwerpunkt des Dreiecks $\triangle B_1B_2B_3$ an.
Die Schwerpunktskoordinaten sind das arithmetische Mittel der Eckkoordinaten.
\end{teilaufgaben}

\begin{loesung}
\begin{teilaufgaben}
\item
Wir schreiben $\vec b_i$ f"ur die Ortsvektoren der Punkte $B_i$.
Gesucht ist eine Abbildung, die die Vektoren $\vec b_i$ auf die
Standardbasisvektoren abbildet.
Die Umkehrabbildung bildet daher
\[
\begin{aligned}
\vec e_1&\mapsto\vec b_1,&
\vec e_2&\mapsto\vec b_2,&
\vec e_3&\mapsto\vec b_3
\end{aligned}
\]
ab.
Die Matrix dieser Umkehrabbildung ist daher
\[
B
=
\begin{pmatrix}
b_{11}&b_{21}&b_{13}\\
b_{12}&b_{22}&b_{23}\\
     1&     1&     1
\end{pmatrix},
\]
wobei die Zahlen $b_{ij}$ die Komponenten des Vektors $\vec b_j$ sind.
Die Abbildung von $x$-Koordinaten zu $z$-Koordinaten ist die Umkehrabbildung
\[
\begin{pmatrix}
w_1\\w_2\\w_3
\end{pmatrix}
=
B^{-1}\begin{pmatrix}
x_1\\x_2\\1
\end{pmatrix}.
\]
\item
Unter Verwendung der numerischen Werte f"ur die Punkte $B_i$ findet man f"ur
die Matrix
\begin{align*}
B=\begin{pmatrix}
 4&16&10\\
 1&-1&12\\
 1& 1& 1
\end{pmatrix}
\qquad
\Rightarrow
\qquad
A=
B^{-1}=\frac1{144}\begin{pmatrix}
-13&-6&202\\
 11&-6&-38\\
  2&12&-20
\end{pmatrix}.
\end{align*}
\item
Der Schwerpunkt hat den Ortsvektor
\[
\vec s = \frac13(\vec b_1+\vec b_2+\vec b_3)
=
\frac13\left(
\begin{pmatrix}  4\\ 1 \end{pmatrix}
+
\begin{pmatrix} 16\\-1 \end{pmatrix}
+
\begin{pmatrix} 10\\12 \end{pmatrix}
\right)
=
\begin{pmatrix}
10\\4
\end{pmatrix}.
\]
Die Wirkung der Matrix $A$ auf dem Schwerpunkt ist
\[
A\begin{pmatrix}10\\4\\1\end{pmatrix}
=
\frac1{144}\begin{pmatrix}48\\48\\48\end{pmatrix}
=
\begin{pmatrix}
\frac13\\
\frac13\\
\frac13
\end{pmatrix}.
\]
Die baryzentrischen Koordinaten des Schwerpunktes sind immer
$(\frac13,\frac13,\frac13)$.
\end{teilaufgaben}
\end{loesung}

