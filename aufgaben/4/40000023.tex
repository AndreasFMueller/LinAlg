Das Tetraeder hat die Ecken
\[
A=(1,1,1),\quad
B=(-1,-1,1),\quad
C=(1,-1,-1),\quad
D=(-1,1,-1).
\]
\begin{center}
\begin{tikzpicture}[scale=1]
\node at (0,0) {\includeagraphics[width=6cm]{tetraeder.jpg}};
\node at (3,0) {$x$};
\node at (0.2,3.1) {$z$};
\node at (-0.1,-0.4) {$y$};
\node at (2.4,2.4) {$A$};
\node at (-2.9,2.3) {$B$};
\node at (2.4,-2.7) {$C$};
\node at (-1.1,-1) {$D$};
\end{tikzpicture}
\end{center}
\begin{teilaufgaben}
\item Finden Sie eine Drehmatrix, die den Punkt $A$ fest lässt und $B$ in $C$
überführt.
\item Rechnen Sie nach, dass der Drehwinkel dieser Drehmatrix $120^\circ$ ist.
\end{teilaufgaben}

\thema{Abbildungsmatrix}
\thema{Drehmatrix}
\thema{Drehwinkel}

\begin{loesung}
\begin{teilaufgaben}
\item
Aus der Zeichnung können wir ablesen, dass die gesuchte Drehung den Punkt $C$
auf $D$ abbilden muss.
Die gesuchte Abbildungsmatrix $A$ muss daher die Bedingungen
\[
A
\begin{pmatrix}1\\1\\1\end{pmatrix}
=
\begin{pmatrix}1\\1\\1\end{pmatrix},
\qquad
A
\begin{pmatrix}-1\\-1\\1\end{pmatrix}
=
\begin{pmatrix}1\\-1\\-1\end{pmatrix}
\qquad\text{und}\qquad
A
\begin{pmatrix}1\\-1\\-1\end{pmatrix}
=
\begin{pmatrix}-1\\1\\-1\end{pmatrix}
\]
erfüllen.
Als Matrixgleichung geschrieben bedeutet dies
\[
A
\underbrace{
\begin{pmatrix}
1&-1& 1\\
1&-1&-1\\
1& 1&-1
\end{pmatrix}
}_{\displaystyle=B_1}
=
\underbrace{
\begin{pmatrix}
1& 1&-1\\
1&-1& 1\\
1&-1&-1
\end{pmatrix}
}_{\displaystyle=B_2}
\qquad\Rightarrow\qquad
A=B_2B_1^{-1}
\]
Die inverse Matrix von $B_1$ kann mit dem Gaussalgorithmus berechnet werden:
\begin{align*}
\begin{tabular}{|>{$}c<{$}>{$}c<{$}>{$}c<{$}|>{$}c<{$}>{$}c<{$}>{$}c<{$}|}
\hline
1&-1& 1&1&0&0\\
1&-1&-1&0&1&0\\
1& 1&-1&0&0&1\\
\hline
\end{tabular}
&\rightarrow
\begin{tabular}{|>{$}c<{$}>{$}c<{$}>{$}c<{$}|>{$}c<{$}>{$}c<{$}>{$}c<{$}|}
\hline
1&-1& 1& 1&0&0\\
0& 0&-2&-1&1&0\\
0& 2&-2&-1&0&1\\
\hline
\end{tabular}
\\
&\rightarrow
\begin{tabular}{|>{$}c<{$}>{$}c<{$}>{$}c<{$}|>{$}c<{$}>{$}c<{$}>{$}c<{$}|}
\hline
1&-1& 1&       1&0&      0\\
0& 1&-1&-\frac12&0&\frac12\\
0& 0&-2&      -1&1&      0\\
\hline
\end{tabular}
\\
&\rightarrow
\begin{tabular}{|>{$}c<{$}>{$}c<{$}>{$}c<{$}|>{$}c<{$}>{$}c<{$}>{$}c<{$}|}
\hline
1&-1& 0& \frac12&       0&      0\\
0& 1& 0&       0&-\frac12&\frac12\\
0& 0& 1& \frac12&-\frac12&      0\\
\hline
\end{tabular}
\\
&\rightarrow
\begin{tabular}{|>{$}c<{$}>{$}c<{$}>{$}c<{$}|>{$}c<{$}>{$}c<{$}>{$}c<{$}|}
\hline
1& 0& 0& \frac12&       0&\frac12\\
0& 1& 0&       0&-\frac12&\frac12\\
0& 0& 1& \frac12&-\frac12&      0\\
\hline
\end{tabular}
\end{align*}
Daraus liest man ab
\[
B_1^{-1}=
\begin{pmatrix}
\frac12&       0&\frac12\\
      0&-\frac12&\frac12\\
\frac12&-\frac12&      0
\end{pmatrix}.
\]
Jetzt muss nur noch das Matrixprodukt $B_2B_1^{-1}$ mit Hilfe der
Regel ``$\text{Zeile}\times\text{Spalte}$'' berechnet werden:
\[
A
=
B_2B_1^{-1}
=
\begin{pmatrix}
1& 1&-1\\
1&-1& 1\\
1&-1&-1
\end{pmatrix}
\begin{pmatrix}
\frac12&       0&\frac12\\
      0&-\frac12&\frac12\\
\frac12&-\frac12&      0
\end{pmatrix}
=
\begin{pmatrix}
0&0&1\\
1&0&0\\
0&1&0
\end{pmatrix}.
\]
Dies ist die gesuchte Drehmatrix.
\item
Der Drehwinkel kann mit der Drehwinkelformel für eine Drehmatrix
gefunden werden:
\begin{align*}
\cos\alpha
&=
\frac{\operatorname{Spur}A-1}{2}
=
-\frac12
\\
\alpha&=120^\circ.
\qedhere
\end{align*}
\end{teilaufgaben}
\end{loesung}
