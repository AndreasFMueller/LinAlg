Bei der numerischen Berechnung einer Funktionen in einer Programmbibliothek
werden oft Approximationen durch Polynome verwendet.
Dies funktioniert jedoch nur, wenn die Funktionen sich tatsächlich ähnlich
wie ein Polynom verhalten.
Dies ist bei einer Funktion mit Polen nicht möglich.
Eine {\em Padé-Approximation} ist eine Approximation mit einer rationalen
Funktion, die in solchen Fällen besser geeignet sein kann.
\begin{teilaufgaben}
\item
Gegeben sind Paare $(x_i,y_i), i=1,\dots,n$, finden
Sie ein Gleichungssystem für die Koeffizienten $a$ und $b$ derart,
dass die Gleichung $y_i=f(x_i)$ für die Funktion
\[
f(x) = \frac{ax}{x-b}
\]
möglichst genau stimmt.
\item
Wenden Sie dieses Gleichungssystem auf die Wertepaare
\begin{center}
\begin{tabular}{|>{$}c<{$}|>{$}r<{$}>{$}r<{$}|}
\hline
i & x_i & y_i \\
\hline
   1 & 2.8213 &   3.4351 \\
   2 & 1.2019 &  -1.5060 \\
   3 & 3.8204 &   2.0987 \\
\hline
\end{tabular}
\end{center}
an und bestimmen Sie $a$ und $b$ für diesen Fall.
\end{teilaufgaben}

%A =
%
%   2.8213   3.4351
%   1.2019  -1.5060
%   3.8204   2.0987
%
%b =
%
%   9.6914
%  -1.8101
%   8.0179
%
%M =
%
%   24.000   15.899
%   15.899   18.472
%
%v =
%
%   55.798
%   52.844
%
%ans =
%
%   1.0000
%   2.0000


\begin{hinweis}
Verwenden Sie den Taschenrechner zur Lösung des Gleichungssystems.
\end{hinweis}

\begin{loesung}
\definecolor{darkred}{rgb}{0.8,0,0}
\definecolor{darkgreen}{rgb}{0,0.6,0}
\def\dx{3}
\def\a{1.0000}
\def\b{2.0000}
\def\punkt#1#2{ \fill[color=darkred] ({\dx*(#1)},{#2}) circle[radius=0.08]; }
\begin{figure}
\centering
\begin{tikzpicture}[>=latex,thick]

\draw[->] (-0.1,0) -- ({\dx*4+0.3},0) coordinate[label={$x$}];
\draw[->] (0,-4.1) -- (0,4.2) coordinate[label={$y$}];

\begin{scope}
	\clip (0,-4) rectangle ({\dx*4},4);

	\draw[color=blue,line width=0.5pt] ({2*\dx},-4) -- ({2*\dx},4);
	\draw[color=darkgreen,line width=0.5pt] ({\b*\dx},-4) -- ({\b*\dx},4);

	\draw[color=blue,line width=1.4pt]
		plot[domain=0:{2-0.1},samples=50] ({\dx*\x},{\x/(\x-2)});
	\draw[color=blue,line width=1.4pt]
		plot[domain={2+0.1}:4,samples=50] ({\dx*\x},{\x/(\x-2)});

	\draw[color=darkgreen,line width=1.4pt]
		plot[domain=0:{\b-0.1},samples=50] ({\dx*\x},{\a*\x/(\x-\b)});
	\draw[color=darkgreen,line width=1.4pt]
		plot[domain={\b+0.1}:4,samples=50] ({\dx*\x},{\a*\x/(\x-\b)});
\end{scope}

\foreach \x in {1,2,3,4}{
	\node at ({\dx*\x},0) [below] {$\x$};
	\draw ({\dx*\x},-0.05) -- ({\dx*\x},0.05);
}
\foreach \y in {1,...,4}{
	\node at (0,\y) [left] {$\y$};
	\node at (0,-\y) [left] {$-\y$};
	\draw (-0.05,\y) -- (0.05,\y);
	\draw (-0.05,-\y) -- (0.05,-\y);
}

\punkt{2.8213}{3.4351}
\punkt{1.2019}{-1.5060}
\punkt{3.8204}{2.0987}

\end{tikzpicture}
\caption{Anpassung einer Funktion der Form $f(x)=ax/(x-b)$ an die gegeben
der Datenpunkte, 
\label{60000041:figure}}
\end{figure}
\begin{teilaufgaben}
\item
Für jeden Datenpunkt $(x_i,y_i)$ sollte die Gleichung
\[
y_i = \frac{{\color{darkred}a}x_i}{x_i-{\color{darkred}b}}
\quad\Rightarrow\quad
x_iy_i - y_i{\color{darkred}b}={\color{darkred}a}x_i
\quad\Rightarrow\quad
x_i{\color{darkred}a}
+
y_i{\color{darkred}b}
=
x_iy_i
\]
erfüllt sein.
Die Form ganz rechts ist eine lineare Gleichung für die Koeffizienten
${\color{darkred}a}$
und
${\color{darkred}b}$.
In Matrixform geschrieben ist das Gleichungssystem
\[
\begin{pmatrix*}
x_1&y_1\\
x_2&y_2\\
\vdots&\vdots\\
x_n&y_n
\end{pmatrix*}
\begin{pmatrix*}
{\color{darkred}a}\\
{\color{darkred}b}
\end{pmatrix*}
=
\begin{pmatrix*}
x_1y_1\\
x_2y_2\\
\vdots\\
x_ny_n
\end{pmatrix*}.
\quad\Rightarrow\quad
A=\begin{pmatrix*}
x_1&y_1\\
x_2&y_2\\
\vdots&\vdots\\
x_n&y_n
\end{pmatrix*}
,
\quad
b=\begin{pmatrix*}
x_1y_1\\
x_2y_2\\
\vdots\\
x_ny_n
\end{pmatrix*}.
\]
\item
Für die Datenpunkt der Tabelle ergibt sich 
\[
A
=
\begin{pmatrix*}[r]
   2.8213&  3.4351\\
   1.2019& -1.5060\\
   3.8204&  2.0987
\end{pmatrix*},
\quad
b
=
\begin{pmatrix*}[r]
   9.6914\\
  -1.8101\\
   8.0179
\end{pmatrix*}.
\]
Das Standardverfahren zur Lösung eines solchen überbestimmten
Gleichungssystems verwendet die Matrix
\[
M
=
\transpose{A}A
=
\begin{pmatrix*}[r]
   24.000&  15.899\\
   15.899&  18.472
\end{pmatrix*}
\quad\text{und die recchte Seite}\quad
\transpose{A}b
=
\begin{pmatrix*}[r]
   55.798\\
   52.844
\end{pmatrix*}
\]
Lösung des Gleichungssystem liefert die Koeffizienten
\[
\begin{pmatrix*}
{\color{darkred}a}\\
{\color{darkred}b}
\end{pmatrix*}
=
(\transpose{A}A)^{-1}\transpose{A}b
=
\begin{pmatrix*}[r]
   1.0000\\
   2.0000
\end{pmatrix*}.
\]
Gefunden wurde 
${\color{darkred}a} = 1.0000$ und ${\color{darkred}b}=2.0000$,
was genau mit den Werten übereinstimmt, mit denen die
Daten erzeugt worden waren.
In Abbildung~\ref{60000041:figure} ist die Funktion $f(x)=x/(x-2)$
in blau und die Funktion ${\color{darkred}a}x/(x-{\color{darkred}b})$ 
in grün dargestellt.
\qedhere
\end{teilaufgaben}
\end{loesung}

\begin{bewertung}
Einzelne Gleichung für $a$ und $b$ ({\bf G}) 1 Punkt,
Matrix $A$ ({\bf A}) 1 Punkt,
Vektor $b$ ({\bf B}) 1 Punkt,
Lösungsmethode mit $\transpose{A}A$ und $\transpose{A}b$ ({\bf M}) 1 Punkt,
Berechnung der Matrizen ({\bf R}) 1 Punkt,
Lösung der Gleichungen ({\bf L}) 1 Punkt.
\end{bewertung}

\begin{diskussion}
\definecolor{darkred}{rgb}{0.8,0,0}
\definecolor{darkgreen}{rgb}{0,0.6,0}
Mit einer grösseren Zahl von Punkten wird die Anpassung der Funktion
sehr viel genauer.
In Abbildung~\ref{60000069:image} sind $n=100$ Punkte mit normalverteilten
Abweichungen verwendet worden, es ergaben sich die Koeffizienten
${\color{darkred}a}= 1.0036 $
und
${\color{darkred}b}= 1.9997 $,
die auf mindestens drei Nachkommastellen mit den tatsächlichen Werten
übereinstimmen.
\begin{figure}
\def\dx{3}
\def\punkt#1#2{ \fill[color=darkred] ({\dx*(#1)},{#2}) circle[radius=0.08]; }
\ainput{punkte.tex}
\centering
\begin{tikzpicture}[>=latex,thick]

\draw[->] (-0.1,0) -- ({\dx*4+0.3},0) coordinate[label={$x$}];
\draw[->] (0,-4.1) -- (0,4.2) coordinate[label={$y$}];

\begin{scope}
	\clip (0,-4.1) rectangle ({\dx*4.1},4.1);

	\draw[color=blue,line width=0.5pt] ({2*\dx},-4) -- ({2*\dx},4);

	\draw[color=blue,line width=1.4pt]
		plot[domain=0:{2-0.1},samples=50] ({\dx*\x},{\x/(\x-2)});
	\draw[color=blue,line width=1.4pt]
		plot[domain={2+0.1}:4,samples=50] ({\dx*\x},{\x/(\x-2)});

	\kurven

	\punkte
\end{scope}

\foreach \x in {1,2,3,4}{
	\node at ({\dx*\x},0) [below] {$\x$};
	\draw ({\dx*\x},-0.05) -- ({\dx*\x},0.05);
}
\foreach \y in {1,...,4}{
	\node at (0,\y) [left] {$\y$};
	\node at (0,-\y) [left] {$-\y$};
	\draw (-0.05,\y) -- (0.05,\y);
	\draw (-0.05,-\y) -- (0.05,-\y);
}

\end{tikzpicture}
\caption{Anpassung einer Funktion an $n=100$ Punkte des Graphen der
Funktion $y=ax/(x-b)$ mit $a=1$ und $b=2$, mit normalverteilten
Fehlern überlagert sind.
\label{60000069:image}}
\end{figure}
\end{diskussion}
