Gegeben sind die vier Punkte 
$P_1$,
$P_2$,
$P_3$ und $P_4$
mit Ortsvektoren
\begin{align*}
\vec p_1&=\begin{pmatrix} 1\\ 1\\ 1\end{pmatrix},&
\vec p_2&=\begin{pmatrix}-1\\-1\\ 1\end{pmatrix},&
\vec p_3&=\begin{pmatrix}-1\\ 1\\-1\end{pmatrix},&
\vec p_4&=\begin{pmatrix} 1\\-1\\-1\end{pmatrix}.
\end{align*}
Sie bilden die Ecken eines Körpers.
\begin{teilaufgaben}
\item Berechnen Sie die Winkel der dreieckigen Flächen des Körpers.
\item Ein Methan-Molekül kann man sich vorstellen als bestehend aus
einem Kohlenstoff-Atom im Nullpunkt des Koordinatensystems und vier
Wasserstoffatomen in den Punkten $P_i$. Wie gross ist der Winkel
$\sphericalangle P_1 O P_2$?
\item Wie gross ist der Abstand des Punktes $P_1$ von der Seitenfläche
$P_2P_3P_4$?
\item Wie gross ist der Winkel zwischen zwei Seitenflächen des Körpers?
\end{teilaufgaben}

\thema{Zwischenwinkel}
\thema{Abstand}

\begin{loesung}
Die Winkel können mit dem Skalarprodukt berechnet werden, zum
Beispiel für die Vektoren
\begin{align*}
\overrightarrow{P_1P_2}&=\vec p_2-\vec p_1=\begin{pmatrix}-2\\-2\\ 0\end{pmatrix},&
\overrightarrow{P_1P_3}&=\vec p_3-\vec p_1=\begin{pmatrix}-2\\ 0\\-2\end{pmatrix}
\end{align*}
\begin{teilaufgaben}
\item
Die Zwischenwinkelformel ergibt
\begin{align*}
\cos\alpha
&=
\frac{\overrightarrow{P_1P_2}\cdot\overrightarrow{P_1P_3}}{
|\overline{P_1P_2}|\cdot |\overline{P_1P_3}|
}
\\
&=
\frac{
\begin{pmatrix}-2\\-2\\ 0\end{pmatrix}\cdot
\begin{pmatrix}-2\\ 0\\-2\end{pmatrix}
}{
\sqrt{8}\cdot\sqrt{8}
}
=\frac{(-2)\cdot(-2)+(-2)\cdot 0+0\cdot(-2)}{8}
=\frac{4}{8}=\frac12,\\
\Rightarrow\qquad\alpha&=\frac{\pi}3.
\end{align*}
\item
Auch hierfür kann man die Zwischenwinkelformel verwenden:
\begin{align*}
\cos\alpha&=
\frac{\vec p_1\cdot\vec p_2}{|\vec p_1|\cdot |\vec p_2|}
=
\frac{\begin{pmatrix}1\\1\\1\end{pmatrix}\cdot\begin{pmatrix}-1\\-1\\1\end{pmatrix}}{\sqrt{3}\cdot\sqrt{3}}
=
\frac{1\cdot(-1)+1\cdot(-1)+1\cdot 1}{3}=-\frac13
\\
\alpha&=\arccos\biggl(-\frac13\biggr)= 109.47122^\circ.
\end{align*}
\item
Der Schwerpunkt der Seitenfläche $P_2P_3P_4$ hat den Ortsvektor
\[
\vec s=\frac13(\vec p_2+\vec p_3+\vec p_4)
=
\frac13\left(
\begin{pmatrix}-1\\-1\\ 1\end{pmatrix}+
\begin{pmatrix}-1\\ 1\\-1\end{pmatrix}+
\begin{pmatrix} 1\\-1\\-1\end{pmatrix}
\right)
=\frac13
\begin{pmatrix} -1\\-1\\-1 \end{pmatrix}.
\]
Die gesuchte Entfernung ist jetzt
\[
|\vec p_1-\vec s|
=
\left|
\begin{pmatrix}1\\1\\1\end{pmatrix}
-
\frac13\begin{pmatrix} -1\\-1\\-1 \end{pmatrix}
\right|
=
\left|
\frac43\begin{pmatrix}1\\1\\1\end{pmatrix}
\right|
=\frac43\sqrt{3}=2.309401.
\]
\item
Um den Winkel zwischen den Seitenflächen zu berechnen müssen wir die 
Normalen auf den Seitenflächen kennen. Aber der Vektore $\vec p_1$ steht
genau auf der Seitenfläche $\triangle P_2P_3P_4$ senkrecht, und ähnlich
für alle anderen Vektoren $\vec p_i$. Die gesuchten Zwischenwinkel
sind also genau die in Teilaufgabe b) berechneten, mit der Ausnahme, dass
man für die Flächen den Winkel $<90^\circ$ braucht, also $70.52878^\circ$.
\qedhere
\end{teilaufgaben}
\end{loesung}
