Ein Roboter bewegt sich auf der $x$-Achse mit konstanter Geschwindigkeit.
Seine aktuelle Position wird mit Hilfe eines Ultraschall-Entfernungsmessers
verfolgt.
Zu den Zeitpunkten $t_i, i=1,2,3$, werden Impulse ausgesendet,
die sich mit Geschwindigkeit $c$ ausbreiten und
deren Echos zu den Zeiten $T_i$ empfangen werden.
\begin{center}
\begin{tabular}{|>{$}c<{$}|>{$}c<{$}>{$}c<{$}|}
\hline
i&t_i&T_i\\
\hline
1&0&10.06\\
2&1&11.52\\
3&2&13.10\\
\hline
\end{tabular}
\end{center}
Natürlich sind die Werte $T_i$ mit einem unvermeidbaren Messfehler
behaftet.
Stellen Sie ein Gleichungssystem auf, welches Ihnen ermöglicht, Ort und
Geschwindigkeit des Roboters so genau wie möglich zu bestimmen.

\thema{Least Squares}

\begin{loesung}
\begin{figure}
\centering
\begin{tikzpicture}[>= latex,thick]
\draw[->] (-0.1,0)--(10.1,0) coordinate[label={$x$}];
\draw[->] (0,-0.1)--(0,12.1) coordinate[label={left:$t$}];
\draw (-0.1,1)--(0.1,1);
\node at (0,1) [left] {$t_i$};
\draw (-0.1,6.25)--(0.1,6.25);
\node at (0,6.25) [left] {$t_i+\tau_i$};
\draw (-0.1,11.5)--(0.1,11.5);
\node at (0,11.5) [left] {$T_i=t_i+2\tau_i$};
\draw[color=blue] (0,1)--(5.05,6.25)--(0,11.5);
\draw[color=red] (4,0)--(6,12);
\draw (4,-0.1)--(4,0.1);
\node at (4,0) [below] {$\color{red}s_0$};
\draw[line width=0.2] (0,6.25)--(5.05,6.25);
\node at (2.625,6.25) [above] {$c\tau_i$};
\draw[line width=0.2pt] (0,1)--(4.2,1);
\node at (5.05,6.25) [right] {$x = {\color{red}s_0} + {\color{red}v_0}(t_i+\tau_i)$};
\node at (4.2,1) [right] {$x = {\color{red}s_0} + {\color{red}v_0}t_i$};
\draw[line width=0.2pt] (0,11.5)--(5.9,11.5);
\node at (5.9,11.5) [right] {$x = {\color{red}s_0} + {\color{red}v_0}(t_i+2\tau_i)$};
\node at (5.475,8.875) [above,rotate=78.7] {$x={\color{red}s_0}+{\color{red}v_0}t$};
\end{tikzpicture}
\end{figure}
Zur Zeit befindet sich der Roboter an der $x$-Koordinate
\begin{equation}
x = {\color{red}s_0} + {\color{red}v_0}t,
\label{40000053:gleichung}
\end{equation}
die beiden Konstanten
$\color{red}s_0$
und
$\color{red}v_0$
müssen bestimmt werden.
Wenn das Signal, welches zur Zeit $t_i$ ausgesendet wurde, zur Zeit $T_i$
empfangen wurde, dann hat es bis zur Rückkehr die Distanz $T_i-t_i$
zurückgelegt.
Sie $\tau_i$ die Zeit, die das Signal vom Nullpunkt bis zum Roboter
braucht, also $\tau_i = (T_i-t_i)/2$.
Zum Zeitpunkt der Reflektion des Signals befand sich der Roboter also bei
der Koordinaten $\tau_i/c=(T_i - t_i)/2c=(T_i-t_i)/2$.
Der Zeitpunkt der Reflektion ist etwas später als $t_i$, nämlich
um die Zeit, die das Signal braucht, um vom Nullpunkt zum Roboter
zu gelangen, also um $\tau_i/c$.
Die Position zur Zeit $t=t_i + \tau_i$ ist also $x=\tau_i/c$.
Setzen wir diese Werte in die Gleichung
\eqref{40000053:gleichung}
ein, erhalten wir die Gleichungen
\begin{equation}
{\color{red}s_0} + {\color{red}v_0}(t_i+\tau_i) = \frac1c\tau_i
\label{40000053:glsys}
\end{equation}
Der Koeffizient $t_i+\tau_i$ ist
\[
t_i+\tau_i = t_i + \frac12(T_i-t_i)=\frac12(T_i+t_i),
\]
so dass das Gleichungssystem \eqref{40000053:glsys}
zu 
\begin{equation}
{\color{red}s_0} + \frac{T_i+t_i}{2} {\color{red}v_0}
=
\frac{T_i-t_i}{2c}
\label{40000053:glsys2}
\end{equation}
wird.
Für 
$\color{red}s_0$
und
$\color{red}v_0$
finden wird damit das überbestimmte Gleichungssystem
\[
\underbrace{
\begin{pmatrix}
1&5.03\\
1&6.26\\
1&7.55
\end{pmatrix}
}_{\displaystyle=A}
\begin{pmatrix}
\color{red}s_0\\
\color{red}v_0
\end{pmatrix}
=
\begin{pmatrix}
5.03\\
5.26\\
5.55
\end{pmatrix}
=
b.
\]
Das Gleichungssystem zur Bestimmung von $\color{red}s_0$ und $\color{red}v_0$
hat die Matrix
\[
A^tA
=
\begin{pmatrix}
\phantom{0}3.0000&\phantom{0}18.8400\\
          18.8400&          121.4910
\end{pmatrix}
\qquad
\text{und rechte Seite}
\qquad
b
=
\begin{pmatrix}
   \phantom{0}15.840\\
             100.131
\end{pmatrix}.
\]
Die Lösung ist
\[
{\color{red}s_0} =    3.98319
\qquad\text{und}\qquad
{\color{red}v_0} =   0.20650.
\]
Der Simulation lagen die Werte $s_0=4$ und $v=0.2$ zu Grunde, die Messfehler
waren im Mittel $1$ Zeiteinheiten, die Genauigkeit des Resultates kann sich
also durchaus sehen lassen.
\end{loesung}

