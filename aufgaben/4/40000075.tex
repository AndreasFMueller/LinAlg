\definecolor{darkred}{rgb}{0.8,0,0}
Die Länge der Vektoren $\color{darkred}\vec{a}$ und $\color{blue}\vec{b}$
haben die Länge
$|{\color{darkred}\vec{a}}|={\color{red}6}$ und
$|{\color{blue}\vec{b}}|={\color{blue}10}$.
Die Orthogonalprojektion
$p_{{\color{darkred}\vec{a}}}({\color{blue}\vec{b}}))$
von ${\color{blue}\vec{b}}$ auf
die Richtung von ${\color{darkred}\vec{a}}$ hat die Länge 5.

\hbox to\hsize{%
\begin{minipage}{0.5\textwidth}
\begin{teilaufgaben}
\item
Wie gross ist der Zwischenwinkel $\alpha$ zwischen den beiden Vektoren?
\item
Wie lang ist die Orthogonalprojektion $p_{\vec{a}}(\vec{b})$ von $\vec{a}$
auf die Richtung von $\vec{b}$?
\item
Drücken Sie $p_{\vec{a}}(\vec{b})$ als algebraischen Ausdruck durch
$|\vec{a}|$, $|\vec{b}|$ und $p_{\vec{b}}(\vec{a})$ aus.
\end{teilaufgaben}
\end{minipage}%
\begin{minipage}{0.5\textwidth}
\begin{center}
\def\u{0.5}
\def\w{20}
\definecolor{darkred}{rgb}{0.8,0,0}
\pgfmathparse{0.9*\u}
\xdef\r{\pgfmathresult}
\begin{tikzpicture}[>=latex,thick]

\fill[color=violet!20] (0,0)
	-- (\w:{1.4*\u}) arc (\w:{\w+60}:{1.4*\u})
	-- cycle;
\node[color=violet] at ({\w+30}:{1*\u}) {$\alpha$};

\fill[color=gray!40] (\w:{5*\u})
	-- ++({\w+90}:{\r}) arc({\w+90}:{\w+180}:{\r})
	-- cycle;
\fill[color=white] ($(\w:{5*\u})+({\w+135}:{0.6*\r})$) circle[radius=0.05];

\fill[color=gray!40] ({\w+60}:{3*\u})
	-- ++({\w+240}:{\r}) arc ({\w+240}:{\w+330}:{\r})
	-- cycle;
\fill[color=white] ($({\w+60}:{3*\u})+({\w-75}:{0.6*\r})$) circle[radius=0.05];

\draw[color=violet,line width=0.08cm] ({\w-90}:0.05) -- ++(\w:{5*\u});
\node[color=violet] at (\w:{0.6*5*\u}) [below] {$p_{\vec{a}}(\vec{b})$};

\draw[color=violet,line width=0.08cm] ({\w+150}:0.05) -- ++({\w+60}:{3*\u});
\node[color=violet] at ({\w+60}:{0.5*3*\u}) [left] {$p_{\vec{b}}(\vec{a})$};


\draw[line width=0.2pt] ({\w+60}:{10*\u}) -- (\w:{5*\u});
\draw[line width=0.2pt] (\w:{6*\u}) -- ({\w+60}:{3*\u});

\draw[->,color=darkred,line width=1.4pt] (0,0) -- (\w:{6*\u});
\node[color=darkred] at (\w:{6*\u}) [right] {$\vec{a}$};

\draw[->,color=blue,line width=1.4pt] (0,0) -- ({\w+60}:{10*\u});
\node[color=blue] at ({\w+60}:{10*\u}) [left] {$\vec{b}$};

\fill (0,0) circle[radius=0.09];

\end{tikzpicture}
\end{center}
\end{minipage}}

\begin{loesung}
\begin{teilaufgaben}
\item
Der Kosinus von $\alpha$ ist
\begin{equation}
\cos\alpha
=
\frac{p_{\vec{a}}(\vec{b})}{|\vec{b}|}
=
\frac{5}{10}
=
\frac12
\qquad\Rightarrow\qquad
\alpha = 60^\circ.
\label{40000075:cos}
\end{equation}
\item Mit dem Resultat von a) folgt
\begin{equation}
p_{\vec{b}}(\vec{a})
=
|\vec{a}|\cos\alpha
=
6\cos 60^\circ
=
6\cdot\frac12
=
3.
\label{40000075:pba}
\end{equation}
\item
Verwendet man den Ausdruck von \eqref{40000075:cos} für $\cos\alpha$,
wird aus \eqref{40000075:pba}
\[
p_{\vec{b}}(\vec{a})
=
|\vec{a}|\cdot \frac{p_{\vec{a}}(\vec{b})}{|\vec{b}|}.
\qedhere
\]
\end{teilaufgaben}
\end{loesung}

