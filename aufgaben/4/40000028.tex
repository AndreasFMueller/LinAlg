Der Beleuchtungswahn in entwickelten L"andern f"uhrt zu einer spektakul"aren
Lichtverschmutzung, die man erst zu ermessen lernt, wenn ein n"achtlicher
Stromausfall die Lichtverschmutzung ausschaltet. Astrophotographen macht
die Lichtverschmutzung ebenfalls zu schaffen, da sie alle Bilder mit
einem Schleier "uberzieht. Man versucht daher, die Lichtverschmutzung
zu messen und von den Pixelwerten eines Bildes zu subtrahieren. Dabei 
stellt sich aber heraus, dass nicht ein einzelner Wert verwendet werden
kann, n"aher an den Lichtquellen ist die Lichtverschmutzung st"arker.

Statt eines konstanten Wertes versucht man also die Lichtverschmutzung
eines Bildes in der Form
\[
l=ax+by+c
\]
zu berechnen. Darin sind $x$ und $y$ die Koordinaten eines Pixels des
Bildes, $a$, $b$ und $c$ sind unbekannte, zu bestimmende Koeffizienten.
Um $a$, $b$ und $c$ zu finden, betrachtet man die Pixel in den Ecken
des Bildes. Wir gehen davon aus, dass dort nur schwarzer Nachthimmel
abgebildet wird, oder eben nur Lichtverschmutzung.
In einem $1001\times 1001$ Bild wurden in den Ecken folgende Pixelwerte 
gefunden:
\begin{center}
\begin{tabular}{|>{$}r<{$}>{$}r<{$}|>{$}c<{$}|}
\hline
x&y&l\\
\hline
   0&   0&11\\
1000&   0&19\\
   0&1000&26\\
1000&1000&42\\
\hline
\end{tabular}
\end{center}
%Finden Sie m"oglichst gute Werte f"ur $a$, $b$ und $c$.
Stellen Sie ein Gleichungssystem auf, welches die Koeffizienten $a$, $b$ 
und $c$ eindeutig bestimmt.

\begin{loesung}
Die vier Zeilen der Tabelle entsprechen den Gleichungen
\[
\begin{linsys}{3}
   0a&+&   0b&+&c&=&11\\
1000a&+&   0b&+&c&=&19\\
   0a&+&1000b&+&c&=&26\\
1000a&+&1000b&+&c&=&42
\end{linsys}
\]
Dies ist ein Gleichungssystem mit vier Gleichungen f"ur nur drei Unbekannte,
also ein "uberbestimmtes Gleichungssystem mit
\[
A=\begin{pmatrix}
   0&   0&1\\
1000&   0&1\\
   0&1000&1\\
1000&1000&1
\end{pmatrix}
,\qquad\text{und}\qquad b=\begin{pmatrix}
11\\
19\\
26\\
42
\end{pmatrix}
\]
Exakt kann dieses Gleichungssystem nicht erf"ullt werden, aber eine
N"aherungs\-l"osung ist immer m"oglich, indem man das Gleichungssystem
mit Matrix $A^tA$ und rechter Seite $A^tb$ l"ost, also
\begin{align*}
A^tA&=
\begin{pmatrix}
0&1000&   0&1000\\
0&   0&1000&1000\\
1&   1&   1&   1
\end{pmatrix}
\begin{pmatrix}
   0&   0&1\\
1000&   0&1\\
   0&1000&1\\
1000&1000&1
\end{pmatrix}
=
\begin{pmatrix}
2000000&1000000&2000\\
1000000&2000000&2000\\
   2000&   2000&   4
\end{pmatrix},
\\
A^tb&=
\begin{pmatrix}
0&1000&   0&1000\\
0&   0&1000&1000\\
1&   1&   1&   1
\end{pmatrix}
\begin{pmatrix}
11\\19\\26\\42
\end{pmatrix}=\begin{pmatrix}61000\\68000\\98\end{pmatrix}.
\end{align*}
%Die numerische L"osung des Gleichungssystems mit diesen Matrizen ist
%\[
%\begin{pmatrix}
%a\\
%b\\
%c
%\end{pmatrix}
%=
%\begin{pmatrix}
%0.012\\
%0.019\\
%9
%\end{pmatrix}.
%\]
%Die L"osung f"ur die Lichtverschmutzungsfunktion ist also
%\[
%l=0.012 x + 0.019 y + 9.
%\]
\end{loesung}

\begin{bewertung}
Gleichungen f"ur $a$, $b$ und $c$ ({\bf G}) 2 Punkt,
"Uberbestimmtes Gleichungssystem ({\bf U}) 1 Punkt,
Berechnung von $A^tA$ ({\bf A}) 2 Punkt,
Berechung von $A^tb$ ({\bf B}) 1 Punkt.
\end{bewertung}

