Gegeben sind die Punkte $A=(10,5,-7)$ und $B=(1,2,-7)$.
%
% 3, 4, 12, 13
% 5, 12, 13
% P = (5,5,5)
%
\begin{teilaufgaben}
\item
Finden Sie eine Ebene $\sigma$ in Normalenform, so dass für jeden Punkt $P$
auf der Ebene gilt, dass die Winkel $\angle ABP$ und $\angle BAP$ gleich
gross sind.
\item
Finden Sie einen Punkt $P$ auf der Geraden
\begin{equation}
\vec{p}
=
\vec{p}_0 + t\vec{r}
=
\begin{pmatrix}2\\6\\3\end{pmatrix}
+
t
\begin{pmatrix}3\\-1\\2\end{pmatrix},
\label{40000060:gerade}
\end{equation}
der von $A$ und $B$ gleich weit entfernt ist.
\item
Berechnen Sie den Winkel $\angle ABP$.
\end{teilaufgaben}

\thema{Normalenform}
\thema{Durchstosspunkt}
\thema{Zwischenwinkel}

\begin{loesung}
\begin{figure}
\centering
\begin{tikzpicture}[scale=1]
\node at (0,0) {
\includeagraphics[width=0.6\hsize]{ebene.png}
};
\node at (-0.1,-1.9) {$S$};
\node at (2.4,-2.5) {$A$};
\node at (-2.3,-2.9) {$B$};
\node at (+0.1,3.9) {$P$};
\node at (5.1,0.1) {$x$};
%\node at (-1.4,2.0) {$y$};
\node at (-1.6,1.8) {$y$};
\node at (-1.0,3.4) {$\vec{r}$};
\node at (-1.9,2.9) {$P_0$};
\node at (-3.6,4.8) {$z$};
\node at (+0.5,-4.0) {$\sigma$};
\end{tikzpicture}
\caption{Lage der Ebene in Aufgabe~\ref{40000060}
\label{40000060:bild}}
\end{figure}
\begin{teilaufgaben}
\item
Die Punkte $P$ liegen auf der mittelhalbierenden Ebene der Punte $A$ und $B$,
wie in Abbildung~\ref{40000060:bild} dargestellt.
Die Normale dieser Ebene ist der Vektor
\[
\vec{n}
=
\overrightarrow{AB}
=
\begin{pmatrix}
-9\\-3\\0
\end{pmatrix}
\]
Ausserdem enthält die Ebene den Punkt
\[
S = \biggl( \frac{11}2, \frac72, -7 \biggr).
\]
Damit können wir die Ebenengleichung in Normalenform aufstellen:
\begin{equation}
0
=
\vec{n}\cdot (\vec{p}-\vec{s})
=
\begin{pmatrix}-9\\-3\\0\end{pmatrix}
\cdot
\begin{pmatrix}
x-\frac{11}2\\
y-\frac72\\
z+7
\end{pmatrix}
=
-9x-3y
+\frac{99}2
+\frac{21}2
=
-9x-3y+60.
\label{40000060:ebene}
\end{equation}
\item
Einen Schnittpunkt der Geraden
\eqref{40000060:gerade} und der Ebene \eqref{40000060:ebene}
finden wir mit dem Gauss-Tableau:
\begin{align*}
\begin{tabular}{|>{$}c<{$}>{$}c<{$}>{$}c<{$}>{$}c<{$}|>{$}c<{$}|}
\hline
\color{red}x&\color{red}y&\color{red}z&\color{red}t&\\
\hline
 1& 0& 0&-3&  2\\
 0& 1& 0& 1&  6\\
 0& 0& 1&-2&  3\\
-9&-3& 0& 0&-60\\
\hline
\end{tabular}
&\rightarrow
\begin{tabular}{|>{$}c<{$}>{$}c<{$}>{$}c<{$}>{$}c<{$}|>{$}c<{$}|}
\hline
\color{red}x&\color{red}y&\color{red}z&\color{red}t&\\
\hline
 1& 0& 0& -3&  2\\
 0& 1& 0&  1&  6\\
 0& 0& 1& -2&  3\\
 0&-3& 0&-27&-42\\
\hline
\end{tabular}
\\
&\rightarrow
\begin{tabular}{|>{$}c<{$}>{$}c<{$}>{$}c<{$}>{$}c<{$}|>{$}c<{$}|}
\hline
\color{red}x&\color{red}y&\color{red}z&\color{red}t&\\
\hline
 1& 0& 0& -3&  2\\
 0& 1& 0&  1&  6\\
 0& 0& 1& -2&  3\\
 0& 0& 0&-24&-24\\
\hline
\end{tabular}
\\
&\rightarrow
\begin{tabular}{|>{$}c<{$}>{$}c<{$}>{$}c<{$}>{$}c<{$}|>{$}c<{$}|}
\hline
\color{red}x&\color{red}y&\color{red}z&\color{red}t&\\
\hline
 1& 0& 0&  0&  5\\
 0& 1& 0&  0&  5\\
 0& 0& 1&  0&  5\\
 0& 0& 0&  1&  1\\
\hline
\end{tabular}
\end{align*}
Daraus lesen wir ab, dass $P=(5,5,5)$ ist.

Man kann diesen Punkt auch ohne Verwendung der Ebene aus Teilaufgabe a)
bestimmen.
Dazu setze man an, dass die beiden Entferungen gleich sein
müssen, also


%(%i2) a:[10,5,-7]
%(%o2)                            [10, 5, - 7]
%(%i3) b:[1,2,-7]
%(%o3)                             [1, 2, - 7]
%(%i4) p:[2,6,3]+t*[3,-1,2]
%(%o4)                      [3 t + 2, 6 - t, 2 t + 3]
%(%i5) da:a-p
%(%o5)                   [8 - 3 t, t - 1, (- 2 t) - 10]
%(%i6) db:b-p
%(%o6)                 [(- 3 t) - 1, t - 4, (- 2 t) - 10]
%(%i7) da . da
%                           2                 2            2
%(%o7)               (t - 1)  + ((- 2 t) - 10)  + (8 - 3 t)
%(%i8) ratsimp(da . da)
%                                  2
%(%o8)                         14 t  - 10 t + 165
%(%i9) db . db
%                         2                 2                2
%(%o9)             (t - 4)  + ((- 2 t) - 10)  + ((- 3 t) - 1)
%(%i10) ratsimp(db . db)
%                                  2
%(%o10)                        14 t  + 38 t + 117
%(%i11) ratsimp(da . da-db . db)
%(%o11)                             48 - 48 t


\begin{align*}
|\vec{a} - \vec{p}(t)|^2&=|\vec{b}-\vec{p}(t)|^2
\\
\left|
\begin{pmatrix}
8-3t\\t-1\\-10-2t
\end{pmatrix}
\right|^2
&=
\left|
\begin{pmatrix}
-1-3t\\-4+t\\-10-2t
\end{pmatrix}
\right|^2
\\
(8-3t)^2+(t-1)^2+(10+2t)^2
&=
(1+3t)^2 + (t-4)^2+(10+2t)^2
\\
14t^2-10t+165
&=
14t^2+38t+117
\\
28
&=
28t
\qquad\Rightarrow\qquad
t=1
\end{align*}
Setze man dies in die Geradengleichung ein, findet man wieder den Punkt
$P=(5,5,5)$.
\item
Der gesuchte Winkel kann mit der Zwischenwinkelformel bestimmt werden:
\begin{align*}
\cos\alpha
&=
\frac{
\overrightarrow{BA}\cdot \overrightarrow{BP}
}{
|\overrightarrow{BA}|\cdot |\overrightarrow{BP}|
}
=
\frac{
\begin{pmatrix}9\\3\\0\end{pmatrix}
\cdot
\begin{pmatrix}4\\3\\12\end{pmatrix}
}{
\sqrt{90}\cdot 13
}
=
\frac{45}{13\cdot\sqrt{90}}
=
0.36488
\\
\alpha
&=
68.6^\circ.
\qedhere
\end{align*}
\end{teilaufgaben}
\end{loesung}

\begin{bewertung}
\begin{teilaufgaben}
\item
Stützpunkt der Ebene ({\bf S}) 1 Punkt,
Normale ({\bf N}) 1 Punkt,
\item
Abstandsbedingung (entweder als Punkt liegt auf der Ebene aus a) oder
aus Abstandsformel) ({\bf A}) 1 Punkt,
Bestimmung des Punktes ({\bf P}) 1 Punkt,
\item
Zwischenwinkelformel ({\bf Z}) 1 Punkt,
Winkel ({\bf W}) 1 Punit.
\end{teilaufgaben}
\end{bewertung}





