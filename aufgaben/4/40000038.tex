Jedes Jahr im Oktober findet in Morton, Illinois, das ``Punkin Chuckin' ''
Festival steht.
In diesem Rahmen findet auch ein Wettbwerb statt, bei dem die Teilnehmer
ausmarchen, wer mit seiner oft abenteurlichen Konstruktion einen K"urbis
am weitesten schiessen kann.
Ein Teilnehmer versucht den optimalen Abschusswinkel experimentell zu
bestimmen, und misst daher die Schussweite in Abh"angigkeit vom
Abschusswinkel.
Er erh"alt die folgenden Daten:
\begin{center}
\def\c{\phantom{\mathstrut^\circ}}
\begin{tabular}{l|>{$}r<{$}>{$}r<{$}>{$}r<{$}>{$}r<{$}>{$}r<{$}>{$}r<{$}}
Winkel          & 20^\circ& 30^\circ& 40^\circ& 50^\circ& 60^\circ& 70^\circ\\
\hline
Schussweite (ft)&372\c    &462\c    &509\c    &501\c    &437\c    &323\c
\end{tabular}
\end{center}
\begin{teilaufgaben}
\item
Die Daten lassen einen Zusammenhang zwischen Winkel und Schussdistanz
in Form eines quadratischen Polynoms
\begin{equation}
d = aw^2 + bw + c
\label{40000038:q}
\end{equation}
im Winkel $w$ vermuten.
Stellen Sie ein Gleichungssystem zur Bestimmung der Koeffizienten $a$, $b$
und $c$ auf.
\item
Finden Sie den Abschusswinkel, mit dem sich die maximale Schussweite erzielen
l"asst.
\end{teilaufgaben}

\begin{hinweis}
L"osen Sie das Gleichungssystem mit dem Taschenrechner.
\end{hinweis}

\begin{loesung}
\begin{teilaufgaben}
\item
Setzt man die gegebenen Werte aus der Tabelle in den Ansatz~(\ref{40000038:q})
ein, erh"alt man das Gleichungssystem
\[
\underbrace{
\begin{pmatrix}
 400&20&1\\
 900&30&1\\
1600&40&1\\
2500&50&1\\
3600&60&1\\
4900&70&1
\end{pmatrix}}_{\displaystyle = A}
\begin{pmatrix}
a\\
b\\
c
\end{pmatrix}
=
\underbrace{
\begin{pmatrix}
372\\
462\\
509\\
501\\
437\\
323
\end{pmatrix}}_{\displaystyle = b}
\]
Dieses "uberbestimmte Gleichungssystem kann mit Hilfe der
Least-Squares-Methode gel"ost werden, dazu ist das Gleichungssystem
mit Matrix $A^tA$ und rechter Seite $A^tb$ zu l"osen:
\[
A^tA
=
\begin{pmatrix}
46750000&783000&13900\\
  783000& 13900&  270\\
   13900&   270&    6
\end{pmatrix},
\qquad
A^tb
=
\begin{pmatrix}
5787800\\
 115560\\
   2605
\end{pmatrix}.
\]
Die L"osung mit dem Taschenrechner liefert
\[
\begin{pmatrix}
a\\b\\c
\end{pmatrix}
=
\begin{pmatrix}
-0.26054\\
22.49679\\
25.38571
\end{pmatrix}.
\]
\item
Der Scheitel der Parabel mit der Gleichung $d=aw^2 + bw+c$
und damit das gesuchte Maximum liegt bei
\[
w_S=-\frac{b}{2a}=\frac{22.49679}{2\cdot 0.26054}=43.174^\circ.
\qedhere
\]
\end{teilaufgaben}
\end{loesung}

\begin{diskussion}
Die vermutete quadratische Abh"angigkeit zwischen Winkel und Distanz
ist nur eine heuristische Vermutung.
Und f"ur die Extremwerte $w=0$ und $w=90^\circ$ ergeben sich offensichtlich
falsche Werte.

Eine von der Kinematik bessere gest"utzte Hypothese w"are, dass
die Schussdistanz eine Funktion von $\sin \frac{w}2$ ist.
In einer Umgebung des Maximums l"asst sich die Sinus-Funktion jedoch
sehr gut durch eine quadratische Funktion approximieren.
%Die Aufgabe stammt von
%\url{http://www.houstonisd.org/cms/lib2/TX01001591/Centricity/Domain/26026/quadratic\%20regression\%20worksheet.pdf}
\end{diskussion}

\begin{bewertung}
Lineares Gleichungssystem f"ur die Koeffizienten $a$, $b$ und $c$ ({\bf G})
1 Punkt,
L"osungsverfahren Least Squares ({\bf L}) 1 Punkt,
Berechungung $A^tA$ und $A^tb$ ({\bf A}) je 1 Punkt,
Berechnung der Koeffizienten ({\bf K}) 1 Punkt,
$w$-Wert f"ur Maximum ({\bf W}) 1 Punkt.
\end{bewertung}

