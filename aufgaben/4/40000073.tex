Ein Spielzeughersteller produziert einen ``Globus'', der aus gefalteten
Kartonteilen zusammengesetzt ist. Er besteht aus acht ``Schnitzen'',
die jeweils $45^\circ$ in geographischer Länge umfassen. Die Schnitze
sind am "Aquator und bei je $45^\circ$ nördlicher und südlicher Breite
geknickt:
\begin{center}
\includeagraphics[width=0.5\hsize]{globus.jpg}
\end{center}
An jedem Pol entstehen auf diese Weise acht Dreiecke, entlang des
"Aquators reihen sich acht Trapeze auf.
\begin{teilaufgaben}
\item
Bestimmen Sie den Winkel $\beta$ an der Basis der Trapeze am "Aquator.
\item Bestimmen Sie den Winkel $\alpha$ an der Spitze der Dreiecke am Nordpol.
\end{teilaufgaben}

\thema{Zwischenwinkel}
\thema{Drehmatrix}

\begin{loesung}
Wir verwenden ein Koordinatensystem, dessen $x$-Achse durch den 0-Merdian
und den "Aquator (im Punkt $X$) und dessen $z$-Achse durch den Nordpol $N$
verläuft.
Wir nehmen als Radius des ``Globus'' den Wert $1$.

Wir brauchen die Ortsvektoren der Punkte bei $45^\circ$ östlicher Länge
und Breite, und stellen dazu die folgende Tabelle zusammen:
\begin{center}
\begin{tabular}{|>{$}c<{$}|>{$}c<{$}>{$}c<{$}|>{$}c<{$}>{$}c<{$}>{$}c<{$}|}
\hline
\text{Punkt}&\text{Länge}&\text{Breite}&x&y&z\\
\hline
P_1         &0            &45^\circ     &\frac1{\sqrt{2}}&0               &\frac1{\sqrt{2}}\\
P_2         &45^\circ     &0            &\frac1{\sqrt{2}}&\frac1{\sqrt{2}}&0\\
P_3         &45^\circ     &45^\circ     &\frac12         &\frac12         &\frac1{\sqrt{2}}\\
\hline
\end{tabular}
\end{center}
Darin ist nur der letzte Punkt nicht sofort klar. Man kann ihn zum Beispiel
als winkelhalbierenden Vektor zwischen dem zweitletzten Punkt und dem Nordpol
bekommen. Dazu muss man man die Summe des zweitletzten Punktes mit dem Nordpol
bilden und den Vektor auf Länge $1$ bringen.

Oder man dreht den ersten Punkt um die $z$-Achse um den Winkel $45^\circ$,
was man mit einer Drehmatrix wie folgt bewerkstelligen kann:
\[
D_{z,45^\circ}=\begin{pmatrix}
\cos 45^\circ&-\sin 45^\circ& 0\\
\sin 45^\circ& \cos 45^\circ& 0\\
0            &0             & 1
\end{pmatrix}
=
\begin{pmatrix}
\frac1{\sqrt{2}}&-\frac1{\sqrt{2}}& 0\\
\frac1{\sqrt{2}}& \frac1{\sqrt{2}}& 0\\
0               &                 & 1
\end{pmatrix}
,
\qquad
D_{z,45^\circ}\begin{pmatrix}\frac1{\sqrt{2}}\\0\\\frac1{\sqrt{2}}\end{pmatrix}
=
\begin{pmatrix}
\frac12\\\frac12\\\frac1{\sqrt{2}}
\end{pmatrix}
\]

In diesem Koordinatensystem hat der Punkt auf mit Länge 0
und Breite $45^\circ$ den Ortsvektor
\[
v=
\begin{pmatrix}
\frac1{\sqrt{2}}\\0\\\frac1{\sqrt{2}}
\end{pmatrix}.
\]

Ausserdem werden wir noch die Länge $l$ eines Vektors vom Pol zu den Punkten
auf $45^\circ$ Breite brauchen. Diese Länge taucht auch auf dem "Aquator
auf. 
\[
l
=
\overline{NP_1}
=
\left|
\begin{pmatrix}
\frac1{\sqrt{2}}\\0\\\frac1{\sqrt{2}}-1
\end{pmatrix}
\right|
=
\sqrt{\frac12 +\frac12(3-2\sqrt{2})}=\sqrt{2-\sqrt{2}}
\simeq
0.765367
\]

Mit all diesen Punkten kann man jetzt die Teilaufgaben unter Verwendung
der Zwischenwinkelformel mit Hilfe des Skalarproduktes lösen.
\begin{teilaufgaben}
\item
Wir finden mit Hilfe der Zwischenwinkelformel
\begin{align*}
\cos\beta
&=
\frac{\overrightarrow{XP_1}\cdot\overrightarrow{XP_3}}{l^2}
=
\frac{
\begin{pmatrix}\frac1{\sqrt{2}}-1\\0\\\frac1{\sqrt{2}}\end{pmatrix}
\cdot
\begin{pmatrix}\frac1{\sqrt{2}}-1\\\frac1{\sqrt{2}}\\0\end{pmatrix}
}{l^2}
=
\frac{\frac32-\sqrt{2}}{2-\sqrt{2}}
=
\frac{(3-2\sqrt{2})(2+\sqrt{2})}{2(2-\sqrt{2})(2+\sqrt{2})}
\\
&=
\frac{2-\sqrt{2}}{4}
\simeq
0.146447
\\
\beta
&=
\arccos \biggl(\frac{\frac32-\sqrt{2}}{2-\sqrt{2}}\biggr)
\simeq81.5789^\circ.
\end{align*}
\item
Wir finden mit Hilfe der Zwischenwinkelformel:
\begin{align*}
\cos\alpha
&=
\frac{\overrightarrow{NP_1}\cdot \overrightarrow{NP_3}}{
\overline{NP_1}\cdot\overline{NP_3}
}
=\frac{
\begin{pmatrix}\frac1{\sqrt{2}}\\0\\\frac1{\sqrt{2}}-1\end{pmatrix}
\cdot
\begin{pmatrix}\frac12\\\frac12\\\frac1{\sqrt{2}}-1\end{pmatrix}
}{l^2}
=
\frac{\frac1{2\sqrt{2}}+\frac32-\sqrt{2}}{2-\sqrt{2}}
=
\frac{\frac{\sqrt{2}}{4}+\frac32-\frac{4\sqrt{2}}{4}}{2-\sqrt{2}}
=
3 \frac{\frac24-\frac{\sqrt{2}}4}{2-\sqrt{2}}
=
\frac34
\\
\alpha
&=
\arccos\frac34\simeq41.4096^\circ.
\qedhere
\end{align*}
\end{teilaufgaben}
\end{loesung}

\begin{diskussion}
In der Prüfung wurde diese Aufgabe ohne Teilaufgabe b) gestellt, wodurch
die Aufgabe leicher wird, weil es nicht mehr nötig ist, den Punkt
$P_3$ zu bestimmen.
\end{diskussion}

\begin{bewertung}
Identifikation des Winkels ({\bf B}) 1 Punkt,
Zwischenwinkelformel ({\bf Z}) 1 Punkt,
Wahl eines zweckmässigen Koordinatensystems ({\bf K}) 1 Punkt,
Bestimmung der Punkte $P_1$ und $P_2$ ({\bf P}) 1 Punkt,
Skalarprodukt ({\bf S}) 1 Punkt,
Berechnung des Winkels ({\bf W}) 1 Punkt.
\end{bewertung}


