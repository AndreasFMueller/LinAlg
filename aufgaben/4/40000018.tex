Statt der Standardbasis solle die Basis aus den Vektoren
\[
b'_1 = \begin{pmatrix}1\\-1\end{pmatrix},\qquad
b'_2 = \begin{pmatrix}1\\2\end{pmatrix}
\]
verwendet werden.
Dazu m"ussen zu Koordinaten
$\begin{pmatrix}x_1\\x_2\end{pmatrix}$ in der Standardbasis
die neuen Koordinaten in der Form
\[
\begin{pmatrix}x'_1\\x'_2\end{pmatrix}=T
\begin{pmatrix}x_1\\x_2\end{pmatrix}
\]
gefunden werden. Bestimmen Sie die Matrix $T$.

\begin{loesung}
F"ur die Transformation von $B$-Koordinaten in $B'$-Koordinaten verwendet
man das Gauss-Tableau:
\begin{align*}
\begin{tabular}{|>{$}c<{$}>{$}c<{$}|>{$}c<{$}>{$}c<{$}|}
\hline
 1& 1& 1& 0\\
-1& 2& 0& 1\\
\hline
\end{tabular}
&
\rightarrow
\begin{tabular}{|>{$}c<{$}>{$}c<{$}|>{$}c<{$}>{$}c<{$}|}
\hline
 1& 1& 1& 0\\
 0& 3& 1& 1\\
\hline
\end{tabular}
\rightarrow
\begin{tabular}{|>{$}c<{$}>{$}c<{$}|>{$}c<{$}>{$}c<{$}|}
\hline
 1& 0& \frac23&-\frac13\\
 0& 1& \frac13& \frac13\\
\hline
\end{tabular}
\end{align*}
Daraus liest man die Transformationsmatrix
\[
T=\begin{pmatrix}
\frac23&-\frac13\\
\frac13& \frac13
\end{pmatrix}
\]
ab.
\end{loesung}

\begin{diskussion}
Die etwas konventionellere Methode, die Transformationsmatrix zu bestimmen,
dr"uckt zun"achst die Vektoren $b_i'$ durch die Standardbasisvektoren aus:
\begin{align*}
b'_1&=e_1-e_2\\
b'_2&=e_1+2e_2
\end{align*}
F"ur die Transformationsmatrix in Richtung
$\{b'_1,b'_2\}\to\{e_1,e_2\}$ muss die transponiert
Koeffizientenmatrix  davon verwendet werden:
\[
\begin{pmatrix}
1&1\\-1&2
\end{pmatrix}
\]
Die verlangte Transformation verl"auft jedoch in die entgegengesetzte
Richtung, man braucht also die inverse Matrix
\[
T=
\begin{pmatrix}
1&1\\-1&2
\end{pmatrix}^{-1}
=
\begin{pmatrix}
\frac23&-\frac13\\
\frac13&\frac13
\end{pmatrix},
\]
wie man zum Beispiel mit dem Gauss-Algorithmus finden kann.
\end{diskussion}

