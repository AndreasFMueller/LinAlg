Eine Kugel mit Radius $7$ bewegt sich auf der Geraden $g$ durch die Punkte
$A=(67,81,-71)$ und $B=(-33,-43,57)$
und trifft auf die
Ebene $\sigma$ mit der Gleichung
\[
2x + 3y + 6z = 147.
\]
Wo befindet sich das Zentrum der Kugel in dem Moment, wo die Kugel die
Ebene $\sigma$ berührt?

\thema{Hessesche Normalform}

\begin{loesung}
Die Gerade hat die Parameterdarstellung
\[
\vec{p} = \vec{a} + t\vec{r},
\qquad\text{wobei}\qquad
\vec{r}
=
\vec{b} - \vec{a}
=
\begin{pmatrix*}[r]
-100\\
-124\\
128
\end{pmatrix*}.
\]
Die Hessesche Normalform der Ebenengleichung ist
wegen $\sqrt{2^2+3^2+6^2}=7$
\[ 
\frac{2}{7}x+\frac{3}{7}y+\frac{6}{7}z=21.
\]
Die Bedingung, dass die Kugel die Ebene berühren muss ist gleichwertig
mit der Bedingung, dass sich das Zentrum im Abstand $7$ von der
Ebene befinden muss, es muss also gelten
\[
\frac{2}{7}x+\frac{3}{7}y+\frac{6}{7}z=21\pm 7.
\]
Damit haben wir die folgenden Gleichungen zu lösen:
\[
\begin{linsys}{4}
           x& &            & &            &+&100t&=&67\\
            & &           y& &            &+&124t&=&81\\
            & &            & &           z&-&128t&=&-71\\
\frac{2}{7}x&+&\frac{3}{7}y&+&\frac{6}{7}z& &    &=&21\pm 7
\end{linsys}
\]
mit dem Tableau
\begin{align*}
\begin{tabular}{|>{$}c<{$}>{$}c<{$}>{$}c<{$}>{$}r<{$}|>{$}c<{$}|}
\hline
   1   &   0   &   0   & 100&\phantom{-}67\\
   0   &   1   &   0   & 124&\phantom{-}81\\
   0   &   0   &   1   &-128&-71\\
\frac27&\frac37&\frac67&   0&21\pm 7\\
\hline
\end{tabular}
&\rightarrow
\begin{tabular}{|>{$}c<{$}>{$}c<{$}>{$}c<{$}>{$}r<{$}|>{$}c<{$}|}
\hline
   1   &   0   &   0   & 100&\phantom{-}67\\
   0   &   1   &   0   & 124&\phantom{-}81\\
   0   &   0   &   1   &-128&-71\\
\frac27&\frac37&\frac67&   0&\phantom{-}14\\
\hline
\end{tabular}
\rightarrow
\begin{tabular}{|>{$}c<{$}>{$}c<{$}>{$}c<{$}>{$}r<{$}|>{$}c<{$}|}
\hline
   1   &   0   &   0   &   0&  -\phantom{0}8\\
   0   &   1   &   0   &   0& -12\\
   0   &   0   &   1   &   0& \phantom{-} 25\\
   0   &   0   &   0   &   1&\frac{3}{4}\\
\hline
\end{tabular}
\\
&\rightarrow
\begin{tabular}{|>{$}c<{$}>{$}c<{$}>{$}c<{$}>{$}r<{$}|>{$}c<{$}|}
\hline
   1   &   0   &   0   & 100&\phantom{-}67\\
   0   &   1   &   0   & 124&\phantom{-}81\\
   0   &   0   &   1   &-128&-71\\
\frac27&\frac37&\frac67&   0&\phantom{-}28\\
\hline
\end{tabular}
\rightarrow
\begin{tabular}{|>{$}c<{$}>{$}c<{$}>{$}c<{$}>{$}r<{$}|>{$}c<{$}|}
\hline
   1   &   0   &   0   &   0&-58\\
   0   &   1   &   0   &   0&-74\\
   0   &   0   &   1   &   0&\phantom{-}89\\
   0   &   0   &   0   &   1&\frac{5}{4}\\
\hline
\end{tabular}
\end{align*}
Daraus liest man ab, dass jeweils dann, wenn der Kugelmittelpunkt
sich in den Punkten
\[
P_+ = (-58, -74,89)
\qquad\text{und}\qquad
P_- = (-8,-12,25)
\]
befindet,
die Kugel die Ebene berührt.
\end{loesung}

\begin{bewertung}
Hessesche Normalform ({\bf H}) 1 Punkt,
Richtungsvektor der Geraden ({\bf R}) 1 Punkt,
Parameterdarstellung der Geraden ({\bf P}) 1 Punkt,
zwei mögliche Lösungen der Hesseschen Normalform ({\bf Z}) 1 Punkt,
Gauss-Tableau für die Bestimmung des Punktes ({\bf G}) 1 Punkt,
Lösungen für die Punkte ({\bf P}) 1 Punkt.
\end{bewertung}

