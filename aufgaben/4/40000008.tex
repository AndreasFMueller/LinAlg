Finden Sie die Abbildungsmatrix der Projektion auf die Winkelhalbierende
des ersten und dritten Quadranten auf zwei verschiedene Arten:
\begin{teilaufgaben}
\item Direkt
\item Unter Verwendung einer Basistransformation und der Projektion auf
die $x$-Achse, die in der Vorlesung behandelt wurde.
\end{teilaufgaben}

\begin{loesung}
\begin{teilaufgaben}
\item Es müssen die Bilder der Basisvektoren $e_1$ und $e_2$ ermittelt
werden:
\[
e_1\mapsto\begin{pmatrix}\frac{1}2\\\frac{1}2\end{pmatrix}
,\qquad
e_2\mapsto\begin{pmatrix}\frac{1}2\\\frac{1}2\end{pmatrix}
\quad
\Rightarrow
\quad
A=\begin{pmatrix}
\frac{1}2&
\frac{1}2\\
\frac{1}2&
\frac{1}2\\
\end{pmatrix}
\]
\item
Wir wollen die Vektoren
\[
\begin{pmatrix}1\\1\end{pmatrix}
\quad\text{und}\quad
\begin{pmatrix}-1\\1\end{pmatrix}
\]
als Basisvektoren verwenden. Die zugehörige Matrix $B'$ ist
\[
B'=
\begin{pmatrix}
1&-1\\
1&1
\end{pmatrix}
\]
und die Transformationsmatrix ist $T=B'^{-1}$
\[
T=
\frac12\begin{pmatrix}
1&1\\
-1&1
\end{pmatrix}
,\qquad
T^{-1}=
\begin{pmatrix}
1&-1\\
1&1
\end{pmatrix}
\]
Die Abbildungsmatrix $A'$ ist die Projektion auf die $x$-Achse,
also
\[
A'=
\begin{pmatrix}
1&0\\
0&0
\end{pmatrix}
\]
Mit der Transformationsformel kann man jetzt auch $A$ ausrechnen:
\[
A=T^{-1}A'T
=
\begin{pmatrix}
1&-1\\
1&1
\end{pmatrix}
\begin{pmatrix}
1&0\\
0&0
\end{pmatrix}
\frac12\begin{pmatrix}
1&1\\
-1&1
\end{pmatrix}
=
\begin{pmatrix}1&0\\1&0\end{pmatrix}
\frac12\begin{pmatrix}
1&1\\
-1&1
\end{pmatrix}
=
\frac12
\begin{pmatrix}
1&1\\
1&1
\end{pmatrix}
\qedhere
\]
\end{teilaufgaben}
\end{loesung}

