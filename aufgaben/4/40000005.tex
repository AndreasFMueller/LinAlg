Warum ist $\operatorname{ker}B \subset \operatorname{ker}AB$ und
warum ist $\operatorname{im}AB\subset\operatorname{im}A$?

\thema{Kern}
\thema{Bild}

\begin{loesung}
Die Vektoren $x\in\operatorname{ker}B$ werden von der Matrix $B$ zu
Null gemacht: $Bx=0$. Daran ändert sich auch nichts mehr, wenn man jetzt
noch zusätzlich die Matrix $A$ darauf anwendet, also ist $ABx=A(Bx)=A0=0$.
Also ist $x\in\operatorname{ker}AB$.

Alle Vektoren von $\operatorname{im}AB$ entstehen dadurch, dass
man zuerst $B$ auf einen Vektor anwendet und dann $A$. Sie sind
also insbesondere Vektoren, die entstehen, indem man $A$ auf einen
Vektor anwendet, sind also in $\operatorname{im}A$.
\end{loesung}

