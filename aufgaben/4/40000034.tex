Bestimmen Sie die Komponenten eines Vektors $\vec v$ der Länge $2\sqrt{2}$,
der mit $\vec e_1$ den Winkel $60^\circ$, mit $\vec e_2$ den Winkel
$135^\circ$ und mit $\vec e_3$ einen spitzen Winkel einschliesst.

\thema{Skalarprodukt}
\thema{Zwischenwinkel}

\begin{loesung}
In einer Orthonormalbasis können die Komponenten eines Vektors als Projektion
auf den jeweiligen Basisvektor berechnet werden. Da $|\vec e_i| = 1$ gilt also
\[
  v_i = \vec v \bullet \vec e_i.
\]
Da aber zusätzlich gilt
\[
  \vec v \bullet \vec e_i = |\vec v|\cdot |\vec e_i| \cdot \cos(\alpha),
\]
können die Komponenten $v_1$ und $v_2$ direkt berechnet werden als:
\begin{align*}
  v_1 & = |\vec v|\cdot |\vec e_1| \cdot \cos(\alpha) = 2\sqrt{2}\cdot 1 \cdot \cos (60^\circ)=\sqrt{2}\\
  v_2 & = |\vec v|\cdot |\vec e_2| \cdot \cos(\alpha) = 2\sqrt{2}\cdot 1 \cdot \cos (135^\circ)=-2
\end{align*}
Die Projektion $v_3$ auf $\vec e_3$ muss wegen des spitzen Winkels positives Vorzeichen haben.
Ausserdem muss noch die Länge von $\vec v$ stimmen:
\begin{align*}
(\sqrt{2})^2+2^2+v_3^2=(2\sqrt{2})^2\\
6+v_3^2&=8\\
v_3^2&=2\\
v_3&=\sqrt {2}
\end{align*}
Der gesuchte Vektor ist also
\[
\vec v = \begin{pmatrix}\sqrt{2}\\-2\\\sqrt{2}\end{pmatrix}.
\qedhere
\]
\end{loesung}

