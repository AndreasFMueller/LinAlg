Von einem Signal $f(t)$ weiss, man, dass es durch $\sin t$ und $\cos t$ in der
Form
\[
f(t) = a_0 + a_1\cos t+b_1\sin t
\]
beschrieben werden kann, aber die Werte von $a_0$, $a_1$ und $b_1$ sind
vorerst nicht bekannt.
Um sie zu bestimmen, wird das Signal zu verschiedenen Zeitpunkten
gemessen, dabei erhält man die folgenden Messresultate, die natürlich
mit Messfehlern behaftet sind:
\begin{center}
\begin{tabular}{>{$}c<{$}|>{$}c<{$}}
           t&f(t)\\
\hline
           0&-2 \\
 \frac{\pi}4& 2 \\
 \frac{\pi}2& 5 \\
\frac{3\pi}4& 5 \\
\end{tabular}
\end{center}
Stellen Sie ein Gleichungssystem auf, mit dem
die Koeffizienten $a_0$, $a_1$ und $b_1$ möglichst
genau bestimmt werden können.

\begin{hinweis}
Die Lösung des Gleichungssystems ist nicht verlangt.
\end{hinweis}

\begin{loesung}
Wir ergänzen die Wertetabelle durch die Werte der Winkelfunktionen:
\begin{center}
\begin{tabular}{>{$}c<{$}|>{$}c<{$}|>{$}c<{$}>{$}c<{$}}
           t&f(t)&\cos t&\sin t\\
\hline
           0& -2 &                1&               0\\
 \frac{\pi}4&  2 & \frac{\sqrt{2}}2&\frac{\sqrt{2}}2\\
 \frac{\pi}2&  5 &                0&               1\\
\frac{3\pi}4&  5 &-\frac{\sqrt{2}}2&\frac{\sqrt{2}}2\\
\end{tabular}
\end{center}
Die Koeffizienten müssen so bestimmt werden, dass die Gleichungen
\[
f(t)=a_0+a_1\cos t + b_1\sin t
\]
für alle vier gemessenen Werte $t$ stimmen.
Setzen wir die Werte ein, erhalten wir das Gleichungssystem
\[
\begin{linsys}{4}
a_0&+&                a_1& &                   &=&-2\\
a_0&+&\frac{\sqrt{2}}2a_1&+&\frac{\sqrt{2}}2b_1&=& 2\\
a_0& &                   &+&                b_1&=& 5\\
a_0&-&\frac{\sqrt{2}}2a_1&+&\frac{\sqrt{2}}2b_1&=& 5\\
\end{linsys}
\]
Dies sind vier Gleichungen für nur drei Unbekannte, das Gleichungssystem
ist also überbestimmt, in der üblichen Notation
verwenden wir die Matrix $A$ und die rechte Seite $b$ mit
\[
A=\begin{pmatrix}
1&                1&               0\\
1& \frac{\sqrt{2}}2&\frac{\sqrt{2}}2\\
1&                0&               1\\
1&-\frac{\sqrt{2}}2&\frac{\sqrt{2}}2
\end{pmatrix}
\qquad
\text{und}
\qquad
b=\begin{pmatrix}
-2\\
 2\\
 5\\
 5
\end{pmatrix}.
\]
Die beste Lösung findet man mit Hilfe von kleinsten Quadraten, also
als Lösung des Gleichungssystems $A^tAx=A^tb$.
Wir berechnen daher zuerst $A^tA$ und $A^tb$:
\begin{align*}
A^tA&=
\begin{pmatrix}
         4&1&1+\sqrt{2}\\
         1&2&         0\\
1+\sqrt{2}&0&         2
\end{pmatrix}
=\begin{pmatrix}
   4.00000 & 1.00000 & 2.41421 \\
   1.00000 & 2.00000 & 0.00000 \\
   2.41421 & 0.00000 & 2.00000
\end{pmatrix}
\\
A^tb&=
\begin{pmatrix}
10\\
-2-\frac{3\sqrt{2}}2\\
5+\frac{7\sqrt{2}}2
\end{pmatrix}
=\begin{pmatrix}
10.0000\\
-4.1213\\
\phantom{-}9.9497
\end{pmatrix}
\end{align*}
Numerische Lösung des Gleichungssystems liefert 
\begin{align*}
a_0&=\phantom{-}0.085786\\
a_1&=-2.103553\\
b_1&=\phantom{-}4.871320
\end{align*}
\begin{figure}
\centering
\includeagraphics[]{graph-1.pdf}
\caption{An vorgegebene Datenpunkte angepasste harmonische Schwingung.
\label{40000034:1}
}
\end{figure}
In Abbildung~\ref{40000034:1} ist die an die Datenpunkte angepasste
Funktion $f(t)$ graphisch dargestellt.
\end{loesung}

\begin{bewertung}
"Uberbstimmtes Gleichungssystem ({\bf "U}) 1 Punkt (es muss darauf hingewiesen
worden sein, dass dies ein überbestimmtes Gleichungssystem ist),
Lösungsverfahren für überbestimmte Gleichungssysteme ({\bf L}),
Matrix $A$ und Vektor $b$ ({\bf M}) 1 Punkt,
Matrix $A^tA$ und Vektor $A^tb$ ({\bf A}) je 1 Punkt,
Lösung ({\bf X}) 1 Punkt.
\end{bewertung}

