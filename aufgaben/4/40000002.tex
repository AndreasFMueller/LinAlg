Gegeben sind die beiden Vektoren
\[
\vec a
=
\begin{pmatrix}
1\\4\\8
\end{pmatrix}
,\qquad
\vec b
=
\begin{pmatrix}
1\\1\\0
\end{pmatrix}.
\]
Finden Sie eine Basis aus aufeinander senkrecht stehenden
Einheitsvektoren so, dass der erste Basisvektor parallel zu
$\vec a$ ist und der dritte auf der von $\vec a$ und $\vec b$
aufgespannten Ebenen senkrecht steht.

\begin{loesung}
Der erste Basisvektor muss offenbar ein Vielfaches von $\vec a$ sein.
Da $\vec a$ kein Einheitsvektor ist, muss man ihn durch seine Länge
teilen:
\[
|\vec a|=\sqrt{1^2+4^2+8^1}=\sqrt{1+16+64}=\sqrt{81}=9
\quad
\Rightarrow
\quad
\frac19
\begin{pmatrix}
1\\4\\8
\end{pmatrix}.
\]
Der dritte Basisvektor muss auf $\vec a$ und $\vec b$ senkrecht stehen,
das Vektorprodukt der beiden Basisvektoren tut dies:
\[
\vec a\times\vec b=
\begin{pmatrix}
1\\4\\8
\end{pmatrix}
\times
\begin{pmatrix}
1\\1\\0
\end{pmatrix}
=
\begin{pmatrix}
4\cdot 0-1\cdot 8\\
8\cdot 1-0\cdot 1\\
1\cdot 1-1\cdot 4
\end{pmatrix}
=
\begin{pmatrix}
-8\\
8\\
-3
\end{pmatrix}
\]
Da dies ebenfalls kein Einheitsvektor ist, müssen wir ihn auch noch
normieren, seine Länge ist
\[
\sqrt{8^2+8^2+ 3^2}=\sqrt{64+64+9}=\sqrt{137}.
\]
Jetzt fehlt nur noch der zweite Basisvektor.
Da dieser auf den bereits
gefundenen senkrecht stehen muss, kann man ihn ebenfalls mit 
dem Vektorprodukt finden. Da die beiden Faktoren bereits senkrecht
stehen und Länge $1$ haben, wird das Vektorprodukt automatisch
Länge $1$ haben:
\[
\frac19
\begin{pmatrix}
1\\4\\8
\end{pmatrix}
\times
\frac1{\sqrt{137}}\begin{pmatrix}-8\\8\\-3\end{pmatrix}
=
\frac1{9\sqrt{137}}
\begin{pmatrix}
4\cdot(-3)-8\cdot 8\\
8\cdot(-8)-(-3)\cdot1\\
1\cdot 8-(-8)\cdot 4
\end{pmatrix}
=
\frac1{9\sqrt{137}}
\begin{pmatrix}
-76\\
-61\\
40\\
\end{pmatrix}
\]
Tatsächlich ist dies ein Einheitsvektor:
\[
\sqrt{76^2+61^2+40^2}=\sqrt{11097}=9\sqrt{137}.
\]
Die gesuchten Baisvektoren sind also: 
\[
\frac19
\begin{pmatrix}
1\\4\\8
\end{pmatrix}
,
\qquad
\frac1{9\sqrt{137}}
\begin{pmatrix}
-76\\
-61\\
40\\
\end{pmatrix},
\qquad
\frac1{\sqrt{137}}
\begin{pmatrix}
-8\\
8\\
-3
\end{pmatrix}
\]

Alternativ kann man die Vektoren auch mit Hilfe des
Orthonormalisierungsverfahrens von Gram-Schmidt bestimmen.
Die Vektoren $\vec b_1=\vec a$ und $\vec b_2=\vec b$ sind 
zu orthonormalisieren.
Der Vektor $\vec b_1'$ ist wie in der ersten Lösung der auf Länge 1 normierte
Vektor $\vec a$:
\[
\vec b_1' 
=
\frac19
\begin{pmatrix}
1\\4\\8
\end{pmatrix}
\]
Die Formel für den zweiten Vektor lautet
\[
\vec b_2'=\frac{
\vec b_2-(\vec b_1'\cdot \vec b_2)\vec b_1'
}{
|\vec b_2-(\vec b_1'\cdot \vec b_2)\vec b_1'|
}
\]
Setzt man die Vektoren ein, wird der Zähler zu
\[
\vec b_2-(\vec b_1'\cdot \vec b_2)b_1'
=
\begin{pmatrix}
1\\1\\0
\end{pmatrix}
-\left(
\frac19
\begin{pmatrix}
1\\4\\8
\end{pmatrix}
\cdot
\begin{pmatrix}
1\\1\\0
\end{pmatrix}
\right)
\frac19
\begin{pmatrix}
1\\4\\8
\end{pmatrix}
=
\begin{pmatrix}
1\\1\\0
\end{pmatrix}
-
\frac5{81}
\begin{pmatrix}
1\\4\\8
\end{pmatrix}
=
\frac1{81}\begin{pmatrix}
76\\
61\\
-40
\end{pmatrix}
\]
Dieser Vektor muss jetzt noch normiert werden, seien Länge ist
\[
\frac1{81}\sqrt{76^2+61^2+40^2}
=
\frac1{81}\sqrt{11097}
=
\frac19
\sqrt{137}.
\]
Der Vektor $\vec b_2'$ wird dadurch zu
\[
\vec b_2'
=
\frac1{9\sqrt{137}}\begin{pmatrix}
76\\
61\\
-40
\end{pmatrix}
\]
Der dritte Vektor kann nicht durch das Orthonormalisierungsverfahren
gewonnen werden, weil dieses ja nur gegebene Vektoren orthonormalisiert.
Gegeben sind jedoch nur gerade zwei Vektoren. Man muss daher den dritten
Vektor wieder mit dem Vektorprodukt ausrechnen:
\begin{align*}
\vec b_3'
=
\vec b_1'\times\vec b_2'
&=
\frac19
\begin{pmatrix}
1\\4\\8
\end{pmatrix}
\times
\frac1{9\sqrt{137}}\begin{pmatrix}
76\\
61\\
-40
\end{pmatrix}
=
\frac1{9^2\sqrt{137}}
\begin{pmatrix}
4\cdot (-40)-61\cdot 8\\
8\cdot 76-(-40)\cdot 1\\
1\cdot 61-4\cdot 76
\end{pmatrix}
\\
&=
\frac1{9^2\sqrt{137}}
\begin{pmatrix}
-648\\
648\\
-243
\end{pmatrix}
=
\frac1{\sqrt{137}}
\begin{pmatrix}
-8\\
8\\
-3
\end{pmatrix}.
\end{align*}
Bis auf die Vorzeichen der Vektoren, an die die Aufgabe keine Bedingungen
gestellt hat, sind die Lösungen identisch.
\end{loesung}

