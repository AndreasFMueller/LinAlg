Ein 3D-Drucker erzeugt dreidimensionale Objekte, in dem er sie schichtweise
aus Kunststoff auf eine Arbeitsplatte extrudiert.
Dabei ist entscheidend, dass die Druckplatte genau horizontal ist, damit
die geheizte Düse, aus der der flüssige Kunststoff austritt, in konstantem
Abstand über die Platte geführt werden kann.
Die meisten 3D-Drucker stellen dem Benutzer für diesen Zweck einige
mechanische Einstellschrauben zur Verfügung.
Der Kossel-3D-Drucker hingegen verfügt über einen Sensor, mit er den
Abstand der Druckplatte von der Düse messen kann.
Die Firmware steuert daher zu Beginn eines Druckes eine vorgegebene Menge von
Positionen auf der Druckplatte an, und misst den Abstand.
Sie kann dann eine eventuelle Schräglage der Druckplatte berechnen
und kompensieren.
Nehmen Sie an, das beim Anfahren der Messpunkte folgende Höhen
gemessen wurden:
\begin{center}
\begin{tabular}{|>{$}r<{$}>{$}r<{$}|>{$}r<{$}|}
\hline
x\,\text{[mm]}&
y\,\text{[mm]}&
z\,\text{[mm]}\\
\hline
 10& 10&0.015\\
 10&-10&0.022\\
-10& 10&0.017\\
-10&-10&0.022\\
\hline
\end{tabular}
\end{center}
\begin{teilaufgaben}
\item
Stellen Sie ein Gleichungssystem auf, mit dem man die bestmögliche
Schätzung von $z$ als lineare Funktion von $x$ und $y$ finden kann.
\item
Bestimmen Sie den Neigungswinkel der Platte zur Horizontalen.
\end{teilaufgaben}

\thema{Least Squares}

\begin{loesung}
\begin{teilaufgaben}
\item
Gesucht ist eine Formel der Form
\[
z=ax+by+c,
\]
die Koeffizienten $a$, $b$ und $c$ sind zu bestimmen. Einsetzen der Messdaten
führt auf das Gleichungssystem
\[
\begin{linsys}{3}
 10a&+&10b&+&c&=&0.015\\
 10a&-&10b&+&c&=&0.022\\
-10a&+&10b&+&c&=&0.017\\
-10a&-&10b&+&c&=&0.022
\end{linsys}
\]
Dieses Gleichungssystem ist überbestimmt, es hat die Matrix $A$
\[
A
=
\begin{pmatrix}
 10& 10&1\\
 10&-10&1\\
-10& 10&1\\
-10&-10&1
\end{pmatrix}.
\]
Die Theorie besagt, dass sich die bestmögliche Lösung finden lässt,
indem man das Gleichungssystem mit der Matrix $A^tA$ und rechter Seite
$A^tb$ löst:
\begin{align*}
A^tA
&=
\begin{pmatrix}
 10& 10&-10&-10\\
 10&-10& 10&-10\\
  1&  1&  1&  1
\end{pmatrix}.
\begin{pmatrix}
 10& 10&1\\
 10&-10&1\\
-10& 10&1\\
-10&-10&1
\end{pmatrix}
=
\begin{pmatrix}
400&  0&  0\\
  0&400&  0\\
  0&  0&  4
\end{pmatrix}
\\
A^tb
&=
\begin{pmatrix}
 10& 10&-10&-10\\
 10&-10& 10&-10\\
  1&  1&  1&  1
\end{pmatrix}
\begin{pmatrix}
0.015\\0.022\\0.017\\0.022
\end{pmatrix}
=
\begin{pmatrix}
-0.020\\
-0.120\\
0.076
\end{pmatrix}
\end{align*}
Das Gleichungssystem kann dank der Diagonalform von $A^tA$
sofort gelöst werden:
\[
a=-0.00005,\qquad b=-0.00030,\qquad c=0.019.
\]
\item
Die Gleichung der Plattenebene ist
\[
ax+by-z=-c,
\]
die Normale ist also der Vektor
\[
\vec n=\begin{pmatrix}a\\b\\-1\end{pmatrix}.
\]
Gesucht ist der Zwischenwinkel zwischen $\vec n$ und der Vertikalen,
also
\begin{align*}
\cos\alpha&=\frac{\vec n\cdot\vec e_3}{|\vec n|\cdot|\vec e_3|}
=\frac{-1}{\sqrt{a^2+b^2+1}\cdot 1}
=-0.9999999537500032\\
\alpha&=0.01743^\circ.
\qedhere
\end{align*}
\end{teilaufgaben}
\end{loesung}

\begin{bewertung}
\begin{teilaufgaben}
\item
Ansatz für die Gleichung der Plattenebene (\textbf{A}) 1 Punkt,
überbestimmtes Gleichungssystem für die Plattenbene (\textbf{G}) 1 Punkt,
Lösungsverfahren für überbestimmte Gleichungssysteme (\textbf{L}) 1 Punkt,
Berechnung der Koeffizienten $a$, $b$ und $c$ (\textbf{K}) 1 Punkt,
\item
Gleichung der Plattenebene oder Normale (\textbf{N}) 1 Punkt,
Berechnungd es Winkels (\textbf{W}) 1 Punkt.
\end{teilaufgaben}
\end{bewertung}


