Ein Bestückungsautomat ist in der Lage, automatisch Bauteile auf
einer Leiterplatte zu platzieren. Damit dies mit ausreichend hoher
Genauigkeit möglich ist, werden auf der Leiterplatte sogenannte
{\em Passermarken}, english  {\em fiducial markers} angebracht (im Bild
eingekreist):
\begin{center}
\includeagraphics[width=0.3\hsize]{fiducials.jpg}
\end{center}
Der Bestückungsautomat liest die Position der Passermarken, und korrigiert
die Bauteilpositionen, damit die Bauteile am richtigen Ort auf der 
Leiterplatte platziert werden.
Dabei ist eine eventuelle Verschiebung sehr einfach zu korrigieren:
man korrigiert um den Mittelwert der Positionsfehler der Passermarken.
Nach der Verschiebung bleibt aber eine Rotation übrig, die auch
noch korrigiert werden soll. Die Passermarken an den Position $(x_p,y_p)$
sind durch die Drehung um $(\Delta x,\Delta y)$ verschoben:
\begin{center}
\begin{tabular}{|>{$}c<{$}|>{$}r<{$}>{$}r<{$}|>{$}c<{$}>{$}c<{$}|}
\hline
i&x_p&y_p&\Delta x&\Delta y\\
\hline
1&  2&  1& -0.095 & \phantom{-}0.171 \\
2& -1&  1& -0.083 &          - 0.091 \\
\hline
\end{tabular}
\end{center}
Finden Sie den Drehwinkel $\alpha$.

\thema{Least Squares}

\begin{hinweis}
Versuchen Sie nicht, direkt den Drehwinkel $\alpha$ zu bestimmen, sondern
betrachten Sie die Zahlen $\cos\alpha$ und $\sin\alpha$, die als Koeffizienten
in der Drehmatrix auftauchen, als voneinander unabhängige Unbekannte.
\end{hinweis}

\begin{loesung}
Gesucht ist die Drehmatrix
\[
D_\alpha
=
\begin{pmatrix}
 \cos\alpha&-\sin\alpha\\
 \sin\alpha& \cos\alpha
\end{pmatrix},
\]
in der wir gemäss Hinweis die Zahlen $\color{red}\cos\alpha$ und
$\color{red}\sin\alpha$ als unabhängige Unbekannte betrachten.
Für jeden Vektor $(x_p,y_p)$ muss gelten
\begin{equation}
D_\alpha
\begin{pmatrix}x_p\\y_p\end{pmatrix}
-
\begin{pmatrix}x_p\\y_p\end{pmatrix}
=
(D_\alpha-E)
\begin{pmatrix}x_p\\y_p\end{pmatrix}
=
\begin{pmatrix}\Delta x\\\Delta y\end{pmatrix}
\qquad
\Leftrightarrow
\qquad
\begin{aligned}
x_p{\color{red}\cos\alpha}-y_p{\color{red}\sin\alpha}&=x_p+\Delta x\\
x_p{\color{red}\sin\alpha}+y_p{\color{red}\cos\alpha}&=y_p+\Delta y
\end{aligned}
\label{40000033:gleichungen}
\end{equation}
Dies sind zwei Gleichungen für die Unbekannten $\cos\alpha$ und $\sin\alpha$
für jeden Punkt.
Die zwei Punkte aus der Tabelle führen also auf die vier Gleichungen
\begin{equation}
\begin{linsys}{4}
2{\color{red}\cos\alpha}&-& {\color{red}\sin\alpha}&=& 2&-&0.095&=& 1.905\\
 {\color{red}\cos\alpha}&+&2{\color{red}\sin\alpha}&=& 1&+&0.171&=& 1.171\\
-{\color{red}\cos\alpha}&-& {\color{red}\sin\alpha}&=&-1&-&0.083&=&-1.083\\
 {\color{red}\cos\alpha}&-& {\color{red}\sin\alpha}&=& 1&-&0.091&=& 0.909
\end{linsys}
\label{40000033:glsystem}
\end{equation}
Die Matrizen dieses überbestimmten Gleichungssystems sind
\[
A
=
\begin{pmatrix}
\phantom{-}2&         - 1\\
\phantom{-}1&\phantom{-}2\\
         - 1&         - 1\\
\phantom{-}1&         - 1
\end{pmatrix}.
\qquad
b=
\begin{pmatrix}
\phantom{-}1.905\\
\phantom{-}1.171\\
         - 1.083\\
\phantom{-}0.909
\end{pmatrix}
\]
Für die Lösung brauchen wir die Matrix $A^tA$ und den Vektor $A^tb$:
\[
A^tA=
\begin{pmatrix}
\phantom{-}2&\phantom{-}1&-1&\phantom{-}1\\
         - 1&\phantom{-}2&-1&         - 1
\end{pmatrix}
\begin{pmatrix}
\phantom{-}2&         - 1\\
\phantom{-}1&\phantom{-}2\\
         - 1&         - 1\\
\phantom{-}1&         - 1
\end{pmatrix}
=
\begin{pmatrix}
7&0\\
0&7
\end{pmatrix},
\qquad
\qquad
A^tb
=
\begin{pmatrix}
6.973\\ 0.611
\end{pmatrix}.
\]
Das Gleichungssystem ist wegen der Diagonalform von $A^tA$ trivial zu lösen,
\begin{align*}
\color{red} \cos\alpha&=0.99614
&&\Rightarrow&\alpha&=5.0358^\circ
\\
\color{red} \sin\alpha&=0.087286
&&\Rightarrow&
\alpha&=5.0075^\circ
\end{align*}
Offenbar sind die Passmarker um $5^\circ$ gegenüber der Solllage verdreht.
\end{loesung}

\begin{bewertung}
Drehmatrix ({\bf D}) 1 Punkt,
Gleichungen (\ref{40000033:gleichungen}) ({\bf G}) 1 Punkt,
"Uberbestimmtes Gleichungssystem (\ref{40000033:glsystem}) ({\bf "U}) 1 Punkt,
Lösungsverfahren für überbestimmte Gleichungssytem ({\bf V}) 1 Punkt,
Berechnung der Matrizen $A^tA$ und $A^tb$ ({\bf M}) 1 Punkt,
Lösung für den Drehwinkel $\alpha$ ({\bf A}) 1 Punkt.
\end{bewertung}

