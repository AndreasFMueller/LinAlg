Ein Schmidt-Cassegrain-Teleskop hat einen gefalteten Strahlengang,
man kann daher die Brennweite nicht direkt messen.
Um die Brennweite $f$ zu bestimmen, verwendet man folgende Methode:
Man richtet das Teleskop auf einen Stern am Himmelsäquator.
Durch die Erddrehung bewegt
er sich mit einer Winkelgeschwindigkeit von
$\omega=360^\circ/24\,h=2\pi/86400\,s$.
Mit einem Okular mit einer eingebauten Messplatte (Abbildung)
beobachtet man die Bewegung des Sterns entlang der horizontalen Skala
in der Mitte.
Er bewegt sich mit der Geschwindigkeit $v = f\omega$.
Ein Teilstrich dieser Skala ist 0.1\,mm.
In regelmässigen Zeitabständen liest man die Position des Sterns ab
und erhält die folgenden Datenpunkte:
\begin{center}
\begin{tabular}{ccc}
\begin{minipage}{0.25\hsize}
\includeagraphics[width=\hsize]{20040512Micro3.jpg}
\end{minipage}&\qquad\qquad\qquad&
\begin{minipage}{0.2\hsize}
\begin{tabular}{>{$}c<{$}|>{$}r<{$}|>{$}c<{$}}
i& t_i\,[\text{s}]&x_i\,[\text{mm}]\\
\hline
1& 0&1.30\\
2&10&2.24\\
3&20&3.21\\
4&30&4.02\\
5&40&5.05
\end{tabular}
\end{minipage}
\end{tabular}
\end{center}
Man würde erwarten, dass die Position $x$ linear von der Zeit abhängt,
also $x=x_0 + vt$.
Nun wurde aber experimentell festgestellt, dass die Abbildungsfehler
von Teleskop und Okular zusammenwirken und dafür sorgen, dass die
Bewegung des Sterns nicht linear ist.
Wir nehmen daher in nächster Näherung an, dass $x$ durch ein
quadratisches Polynom
\begin{equation}
x=x_0+vt + bt^2
\label{40000058:quadratisch}
\end{equation}
beschrieben wird.

\begin{teilaufgaben}
\item
Stellen Sie ein Gleichungssystem auf, mit dem sich die bestmöglichen
Werte für $x_0$, $v$ und $b$ bestimmen lassen.
\item
Bestimmen Sie die Brennweite $f$ in mm.
\end{teilaufgaben}

\begin{hinweis}
Verwenden Sie den Taschenrechner, um das in a) gefundene Gleichungssystem
zu lösen.
\end{hinweis}

\thema{Least Squares}

\begin{loesung}
\begin{teilaufgaben}
\item
Wir möchten gerne $x_0$, $v$ und $b$ bestimmen, so dass für
jeden Index $i$ die Gleichung
\[
x_i = {\color{red}x_0} + {\color{red}v}t_i + {\color{red}b}t_i^2
\]
möglichst gut erfüllt ist.
Dies liefert uns das Gleichungssystem
\begin{equation}
\begin{linsys}{3}
{\color{red}x_0}&+&t_1{\color{red}v}&+&t_1^2{\color{red}b}&=&x_1\phantom{,}\\
{\color{red}x_0}&+&t_2{\color{red}v}&+&t_2^2{\color{red}b}&=&x_2\phantom{,}\\
{\color{red}x_0}&+&t_3{\color{red}v}&+&t_3^2{\color{red}b}&=&x_3\phantom{,}\\
{\color{red}x_0}&+&t_4{\color{red}v}&+&t_4^2{\color{red}b}&=&x_4\phantom{,}\\
{\color{red}x_0}&+&t_5{\color{red}v}&+&t_5^2{\color{red}b}&=&x_5,
\end{linsys}
\label{40000058:gl}
\end{equation}
welches im Sinne der kleinsten Quadrate gelöst werden muss.
In Matrixform ist dies das Gleichungssystem
\[
\underbrace{
\begin{pmatrix}
1& 0&   0\\
1&10& 100\\
1&20& 400\\
1&30& 900\\
1&40&1600
\end{pmatrix}
}_{\displaystyle=A}
\underbrace{
\begin{pmatrix}
\color{red}x_0\\
\color{red}v\\
\color{red}b
\end{pmatrix}
}_{\displaystyle=x}
=
\underbrace{
\begin{pmatrix}
1.30\\
2.24\\
3.21\\
4.02\\
5.05
\end{pmatrix}
}_{\displaystyle=b}.
\]
Um die Unbekannten $x_0$, $v$ und $b$ zu bestimmen, muss das
Gleichungssystem $A^tAx=A^tb$ gelöst werden.
Die Matrizenprodukte sind
\[
A^tA
=
\begin{pmatrix}
   5&   100&   3000\\
 100&  3000& 100000\\
3000&100000&3540000
\end{pmatrix},
\qquad
A^tb
=
\begin{pmatrix}
\phantom{000}15.82\\
\phantom{00}409.2\phantom{0}\\
13206.\phantom{00}
\end{pmatrix}.
\]
Die Lösung dieses Gleichungssystems mit dem Taschenrechner liefert
\[
x_0 = 1.3109,\quad
v=0.09222,\quad
b=0.000014287.
\]
\item
Aus dem Wert von $v=0.09222$ erhalten wir für die Brennweite
\[
f
=
v/\omega
=
0.09222\cdot \frac{86400}{2\pi}
=
1268.1\, \text{mm}
\qedhere
\]
\end{teilaufgaben}
\end{loesung}

\thema{Least Squares}

\begin{bewertung}
Least-Squares Gleichungen \eqref{40000058:gl} ({\bf G}) 1 Punkt,
Matrix $A$ ({\bf A}) 1 Punkt,
Matrix $A^tA$ ({\bf C}) 1 Punkt,
Vektor $A^tb$ ({\bf B}) 1 Punkt,
Wert $v$ ({\bf V}) 1 Punkt,
Brennweite $f$ ({\bf F}) 1 Punkt.
\end{bewertung}
