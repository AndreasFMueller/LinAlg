Finden Sie die Gleichung einer Kugel mit Mittelpunkt auf der Geraden
mit Parameterdarstellung
\[
\begin{pmatrix}x\\y\\z\end{pmatrix}
=
\begin{pmatrix} 7\\0\\0 \end{pmatrix}
+t
\begin{pmatrix}1\\0\\3\end{pmatrix}
\]
die die beiden Ebenen mit Gleichungen
\[
\begin{linsys}{4}
4x&+&7y&-&4z&-&18&=&0\\
4x&-&8y&+& z&+&27&=&0
\end{linsys}
\]
berührt.

\thema{Hessesche Normalform}
\thema{Kugel}

\begin{loesung}
Der Mittelpunkt hat gleichen Abstand von beiden Ebenen. 
Diese Bedingung lässt sich am leichtesten mit der Hesseschen Normalform
formulieren.
Dazu müssen die Ebenen-Gleichungen in Normalform, durch die Länge der
Normalen-Vektoren geteilt werden. Die Normalen der beiden Ebenen sind
\[
\vec n_1 = \begin{pmatrix}4\\7\\-4\end{pmatrix}\quad \text{und}\quad
\vec n_2 = \begin{pmatrix}4\\-8\\1\end{pmatrix}
\]
und die Längen der Normalen-Vektoren
\begin{align*}
|\vec n_1| &= \sqrt{4^2+7^2+(-4)^2} = \sqrt{81} = 9\\
|\vec n_2| &= \sqrt{4^2+(-8)^2+1^2} = \sqrt{81} = 9\\
\end{align*}
Die Hessenschen Normalformen lauten jetzt
\[
\begin{linsys}{4}
\frac{4}{9}x&+&\frac{7}{9}y&-&\frac{4}{9}z&-&2&=&0\\
\frac{4}{9}x&-&\frac{8}{9}y&+&\frac{1}{9}z&+&3&=&0.
\end{linsys}
\]
Der Mittelpunkt muss von beiden Ebenen den gleichen Abstand haben,
also müssen beide Hesseschen Normalformen den gleichen Wert für den Abstand
liefern, also
\begin{align*}
\frac{4}{9}x+\frac{7}{9}y-\frac{4}{9}z-2&=
\frac{4}{9}x-\frac{8}{9}y+\frac{1}{9}z+3
\\
\frac{15}{9}y-\frac{5}{9}z&=5
\\
15y-5z&=45.
\end{align*}
Setzt man die Geradengleichung darin ein, erhält man die Gleichung
\begin{align*}
15\cdot0-5\cdot3t&=45\\
-15t&=45
\qquad\Rightarrow\qquad t=-3.
\end{align*}
Um nun den Mittelpunkt zu berechnen, muss $t=-3$ wieder 
in die Geradengleichung eingesetzt werden:
\[
\vec m= 
\begin{pmatrix} 7\\0\\0 \end{pmatrix}
+(-3)\cdot
\begin{pmatrix}1\\0\\3\end{pmatrix}
=
\begin{pmatrix}4\\0\\-9\end{pmatrix}.
\]
Um die Kugelgleichung aufstellen zu können, müssen wir auch noch den
Radius kennen.
Dieser ist aber auch der Abstand des Mittelpunktes von den Ebenen,
den wir mit der Hesseschen Normalform berechnen können:
\[
r
=
\frac{4}{9}\cdot 4+\frac{7}{9}\cdot 0-\frac{4}{9}\cdot (-9)-2
=
\frac{16}{9}+4-2
=
\frac{16+18}{9}
=
\frac{34}{9}.
\]
Damit können wir jetzt die Kugelgleichung aufstellen:
\[
\left|\vec p - \vec m \right|^2 = r^2\quad\Rightarrow\quad
\left|\,\vec p - \begin{pmatrix}4\\0\\-9\end{pmatrix}
\,\right|^2=\left(\frac{34}{9}\right)^2.
\qedhere
\]
\end{loesung}
