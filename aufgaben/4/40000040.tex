Bestimmen Sie die Matrix $D$ einer Drehung um die Achse mit Richtungsvektor
\[
v=\begin{pmatrix}1\\1\\0\end{pmatrix}
\]
um den Winkel $\alpha=180^\circ$.
\begin{teilaufgaben}
\item Ist $D\in\text{SO}(3)$?
\item "Uberpr"ufen Sie mit der Drehwinkel-Formel, dass der Drehwinkel
tats"achlich $180^\circ$ ist.
\end{teilaufgaben}

\begin{loesung}
Diese Drehung bildet die Standardbasis-Vektoren wie folgt ab:
\[
\begin{aligned}
e_1&\mapsto e_2
&
e_2&\mapsto e_1
&
e_3&\mapsto -e_3.
\end{aligned}
\]
Sie hat daher die Matrix
\[
D
=
\begin{pmatrix}
0&1& 0\\
1&0& 0\\
0&0&-1
\end{pmatrix}
\]
\begin{teilaufgaben}
\item Die Spalten von $D$ sind offensichtlich orthonormiert, also ist $D$
orthogonal, $D\in\textrm{O}(3)$.
Alternativ kann man auch nachrechnen, dass
\[
D^tD
=
\begin{pmatrix}
0&1& 0\\
1&0& 0\\
0&0&-1
\end{pmatrix}
\begin{pmatrix}
0&1& 0\\
1&0& 0\\
0&0&-1
\end{pmatrix}
=
\begin{pmatrix}
1&0&0\\
0&1&0\\
0&0&1
\end{pmatrix}
=E,
\]
was auch zeigt, dass die Matrix orthogonal ist.

Damit $D\in\textrm{SO}(3)$ muss aber auch $\det(D)=1$ sein:
\[
\left|
\begin{matrix}
0&1& 0\\
1&0& 0\\
0&0&-1
\end{matrix}\right|
=
\left|\begin{matrix}0&1\\1&0\end{matrix}\right|\cdot (-1)
=(-1)\cdot(-1)=1.
\]
Also folgt $D\in\textrm{SO}(3)$.
\item
Die Drehwinkelformel liefert
\[
\cos\alpha = \frac{\operatorname{Spur}(D)-1}2=\frac{-1-1}2=-1,
\]
oder $\alpha=180^\circ$.
\end{teilaufgaben}
\end{loesung}

