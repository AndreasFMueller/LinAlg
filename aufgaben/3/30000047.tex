Die Lösungsmenge des Gleichungssystems 
\[
\begin{linsys}{3}
4x&+& 8y&-&12z&=&16\\
7x&+&17y&-&15z&=&7
\end{linsys}
\]
ist eine Gerade.
\begin{teilaufgaben}
\item
Verwenden Sie den Gauss-Algorithmus, um dafür eine Parameterdarstellung
zu finden.
\item
Schneidet diese Gerade die Gerade mit der Parameterdarstellung
\[
\begin{pmatrix}
-18\\
 19\\
  7
\end{pmatrix}
+t
\begin{pmatrix}
-3\\
 4\\
 2
\end{pmatrix}
\]
und wenn ja, in welchem Punkt?
\end{teilaufgaben}

\thema{Gerade}
\thema{Schnittpunkt}

\begin{loesung}
\begin{teilaufgaben}
\item
Wir Lösen das Gleichungssystem mit dem Gauss-Algorithmus:
\begin{align*}
\begin{tabular}{|>{$}c<{$}>{$}c<{$}>{$}c<{$}|>{$}c<{$}|}
\hline
4& 8&-12&16\\
7&17&-15& 7\\
\hline
\end{tabular}
&
\rightarrow
\begin{tabular}{|>{$}c<{$}>{$}c<{$}>{$}c<{$}|>{$}c<{$}|}
\hline
1& 2&- 3&  4\\
0& 3&  6&-21\\
\hline
\end{tabular}
\rightarrow
\begin{tabular}{|>{$}c<{$}>{$}c<{$}>{$}c<{$}|>{$}c<{$}|}
\hline
1& 0&- 7& 18\\
0& 1&  2&- 7\\
\hline
\end{tabular}
\end{align*}
Die dritte Variable ist offenbar frei wählbar, wir nenne Sie $s$, obwohl
sie natürlich mit der Variablen $z$ übereinstimmt.
Die Lösungsmenge ist
\[
\mathbb L
=
\left\{
\left.
\begin{pmatrix}18\\-7\\0\end{pmatrix}
+s
\begin{pmatrix}7\\-2\\1\end{pmatrix}
\;
\right|
s\in\mathbb R
\right\},
\]
also genau eine Parameterdarstellung einer Geraden.
\item
Um den Schnittpunkt zu finden, verwenden wir das Gausstableau
\[
\begin{tabular}{|>{$}c<{$}>{$}c<{$}>{$}c<{$}>{$}c<{$}>{$}c<{$}|>{$}c<{$}|}
\hline
1&0&0&-7& 0& 18\\
0&1&0& 2& 0& -7\\
0&0&1&-1& 0&  0\\
1&0&0& 0& 3&-18\\
0&1&0& 0&-4& 19\\
0&0&1& 0&-2&  7\\
\hline
\end{tabular}
\rightarrow\cdots\rightarrow
\begin{tabular}{|>{$}c<{$}>{$}c<{$}>{$}c<{$}>{$}c<{$}>{$}c<{$}|>{$}c<{$}|}
\hline
1&0&0& 0& 0& -3\\
0&1&0& 0& 0& -1\\
0&0&1& 0& 0& -3\\
0&0&0& 1& 0& -3\\
0&0&0& 0& 1& -5\\
0&0&0& 0& 0&  0\\
\hline
\end{tabular},
\]
in dem die vierte Spalte für die Variable $s$ steht und die fünfte
Spalte für $t$.
Aus diesem Tableau kann man ablesen, dass ein Schnittpunkt existiert,
dass er für $s=-3$ und $t=-5$ erfolgt, und dass der Schnittpunkt
$S=(-3,-1,-3)$ ist.
\qedhere
\end{teilaufgaben}
\end{loesung}




