Kann der Vektor
\[
\begin{pmatrix}
2\\-5\\3
\end{pmatrix}
\]
als Linearkombination der Vektoren
\[
u_1=\begin{pmatrix}
1\\-3\\2
\end{pmatrix},
\qquad
u_2=\begin{pmatrix}
2\\-4\\-1
\end{pmatrix},
\qquad\text{und}\qquad
u_3=\begin{pmatrix}
1\\-5\\7
\end{pmatrix}
\]
geschrieben werden? Wenn ja, finden Sie die Koeffizienten, wenn nein,
warum nicht? Bilden die Vektoren $u_i$ eine Basis?

\thema{Basis}
\thema{Basiszerlegung}

\begin{loesung}
Wenn die gesuchte Darstellung möglich ist, dann gibt es Zahlen
$x_1$, $x_2$ und $x_3$ mit der Eigenschaft, dass
\[
\begin{pmatrix}
2\\-5\\3
\end{pmatrix}
=x_1u_1+x_2u_2+x_3u_3.
\]
Diese  Zahlen kann man finden als Lösungen eines Gleichungssystems,
mit dem Gauss-Algorithmus ergibt sich:
\begin{align*}
\begin{tabular}{|ccc|c|}
\hline
1&2&1&2\\
$-3$&$-4$&$-5$&$-5$\\
2&$-1$&7&3\\
\hline
\end{tabular}
&\rightarrow
\begin{tabular}{|ccc|c|}
\hline
1&2&1&2\\
0&2&$-2$&1\\
0&$-5$&5&$-1$\\
\hline
\end{tabular}
\\
&\rightarrow
\begin{tabular}{|ccc|c|}
\hline
1&2&1&2\\
0&1&$-1$&$\frac12$\\
0&0&0&$\frac32$\\
\hline
\end{tabular}
\end{align*}
Aus dieser Rechnung kann man ablesen, dass das Gleichungssystem singulär
ist, insbesondere sind die drei Vektoren linear abhängig, bilden also
keine Basis. Ausserdem kann man aus der letzten Zeile sehen, die der
Gleichung $0=\frac32$ entspricht, dass das ursprüngliche Gleichungssytem
keine Lösung haben kann. Die gesuchte Darstellung ist also nicht möglich.
\end{loesung}

