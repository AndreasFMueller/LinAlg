Flat-Earther glauben, die Erde sei eine Scheibe von 200000km Radius,
mit Zentrum im Nordpol.
Die HSR befindet sich in diesem Modell in einem Punkt etwa 4750km
vom Nordpol entfernt.

Den S"udpol gebe es dagegen nicht, stattdessen wurde dort eine Eiswand
verhindern, dass man den Kontinent Antarctica "uberqueren kann.
Die Sonne bewege sich einmal am Tag entlang eines Kreises von variablem
Radius etwa 5000km "uber der Erdscheibe.
Die Jahreszeiten entstehen dadurch, dass der Radius dieses Kreises
"uber das Jahr schwankt.
Zur Zeit der Tag- und Nachtgleiche ist der Radius 10000km.

\begin{teilaufgaben}
\item
In diesem Modell geht die Sonne nie unter (was die Flat-Earther anscheinen
nicht st"ort).
Wie gross ist der kleinstm"ogliche Winkel "uber dem Horizont, unter dem
man die Sonne von der HSR aus sehen kann.
\item
Sei $P$ der Punkt auf dem "Aquator mit dem gleichen Breitengrad wie
die HSR.
Am Tag der Tag- und Nachtgleiche befindet sich in diesem Modell
die Sonne bei ``Sonnenaufgang'' morgens um 06:00 Uhr "uber einem Strahl
vom Nordpol aus, der senkrecht ist auf dem Strahl vom Nordpol zu $P$.
Um welche Zeit ist von $P$ aus gesehen der Winkel zwischen der Position
der Sonne und der Position bei ``Sonnenaufgang'' $90^\circ$?
Man vergleiche das mit der Realit"at, in der die Sonne an jedem Tag,
nicht nur zur Tag- und Nachtgleiche, genau um 12:00 
$90^\circ$ von der Position entfernt ist, an der sie um 06:00 am
gleichen Tag war.
\end{teilaufgaben}

\begin{loesung}
Wir verwenden ein Koordinatensystem mit Nullpunkt im Nordpol,
dessen $x$-Achse durch die HSR geht, die $z$-Achse ist senkrecht
auf der Erdscheibe im Nordpol.
Die HSR hat in diesem Koordinatensystem den Ortsvektor
\[
\vec h=\begin{pmatrix}4750\\0\\0\end{pmatrix}.
\]
\begin{teilaufgaben}
\item
Am tiefsten Punkt befindet sich die Sonne diametral gegen"uber der HSR,
sie hat den Ortsvektor
\[
\vec s_{\text{Mitternacht}}=\begin{pmatrix}-10000\\0\\5000\end{pmatrix}.
\]
Der gesuchte Winkel $\alpha$ kann mit einem rechtwinkligen Dreieck berechnet
werden:
\[
\tan\alpha = \frac{5000}{4750-(-10000)}=0.33898
\qquad\Rightarrow\qquad
\alpha=18.726^\circ.
\]
\item
Wir messen die Zeit mit Hilfe des Winkels $\varphi$ zwischen der Richtung
der Sonnenposition um Mitternacht, die in Teilaufgabe a) verwendet wurde,
und der aktuellen Position.
Der Ortsvektor der Sonnen zur Zeit $\varphi$ ist
\[
\vec{s}(\varphi) =
\begin{pmatrix}
-10000\cdot\cos\varphi\\
10000\cdot\sin\varphi\\
5000
\end{pmatrix}.
\]
Gesucht ist dasjenige $\varphi$, f"ur das 
\[
(\vec s(90^\circ) - \vec p)\cdot(\vec s(\varphi)-\vec p)=0
\]
gilt.
Wir l"osen das Problem gleich f"ur beliebige Punkte auf dem Meridian
durch die HSR, und setzen daher 
\[
\vec p=\begin{pmatrix}10000r\\0\\0\end{pmatrix},
\]
so dass f"ur $r=1$ der in der Aufgabe verlangte Punkt auf dem "Aquator
verwendet wird.

Setzt man diese Werte in die Vektorgleichung ein, erh"alt man
\begin{align*}
0
&=
\left(
\begin{pmatrix}
-10000\cdot\cos 90^\circ\\
 10000\cdot\sin 90^\circ\\
5000
\end{pmatrix}
-\begin{pmatrix}10000r\\0\\0\end{pmatrix}
\right)
\cdot
\left(
\begin{pmatrix}
-10000\cdot\cos\varphi\\
 10000\cdot\sin\varphi\\
5000
\end{pmatrix}
-\begin{pmatrix}10000r\\0\\0\end{pmatrix}
\right)
\\
&=
\begin{pmatrix}
-10000r\\
 10000\\
 5000
\end{pmatrix}
\cdot
\begin{pmatrix}
-10000r
 10000\cdot\cos\varphi\\
-10000\cdot\sin\varphi\\
  5000
\end{pmatrix}
=
10000^2
\begin{pmatrix}
r\\
1\\
0.5
\end{pmatrix}
\cdot
\begin{pmatrix}
r
+
\cos\varphi\\
\sin\varphi\\
0.5
\end{pmatrix}
\\
&=
10000^2 ( r^2 + r\cos\varphi+\sin\varphi +0.25)
\\
&=
10000^2 ( r^2 + r\cos\varphi\pm\sqrt{1-\cos^2\varphi}+0.25)
\end{align*}
Bringt man die Wurzel auf eine Seite
\begin{align*}
r\cos\varphi + 0.25+r^2&=\mp\sqrt{1-\cos^2\varphi}\\
u\cos\varphi + v&=\mp\sqrt{1-\cos^2\varphi}
\end{align*}
mit
\[
\begin{aligned}
u&=r,&
v&=0.25 + r^2.
\end{aligned}
\]
Quadrieren wir die Gleichung und fassen Potenzen von $\cos\varphi$
zusammen, erhalten wir
\begin{align*}
u^2\cos^2\varphi + 2uv\cos\varphi+v^2&=1-\cos^2\varphi
\\
(u^2+1)\cos^2\varphi +2uv\cos\varphi+v^2-1&=0,
\end{align*}
eine quadratische Gleichung in $\cos\varphi$.

Setzt man die numerischen Werte $r=1$ ein, erh"alt man f"ur
den Winkel
\[
\cos\varphi= -0.9557189138830738
\qquad\Rightarrow\qquad
\varphi = 10.859\text{h}.
\]
Da die Sonne um 06:00 bereits "uber dem Horizont ist (mehr als $18^\circ$,
wie wir aus Teilaufgabe a) wissen), ist dies nicht die richtige Zeit.
Wir brauchen einen Winkel, dessen $\sin\varphi<0$ ist, der aber
den gleichen $\cos\varphi$ hat.
Der gesuchte Winkel ist $24\text{h}-\varphi=13.141\text{h}$, die
in der Aufgabenstellung beschriebene Situation tritt also erst um
$13:08:28$ ein. 
\end{teilaufgaben}
Man m"usste meinen, dass diese Unstimmigkeiten selbst einem Flat Earther 
auffallen m"ussten.
Eine schnelle Suche auf Youtube zeigt jedoch, dass Flat Earther in
erstaunlicher Zahl Experimente genau dieser Art durchf"uhren, und keine
Diskrepanzen feststellen.

F"uhrt man die Rechnung f"ur verschiedene Werte von $r$ durch erkennt man,
dass f"ur gem"assigte Breiten (Werte um 0.5) der Fehler erstaunlich klein
ist.
\end{loesung}


\begin{bewertung}
\end{bewertung}

