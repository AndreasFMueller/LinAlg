Flat-Earther glauben, die Erde sei eine Scheibe von 20'000km Radius,
mit Zentrum im Nordpol.
Die HSR befindet sich in diesem Modell in einem Punkt etwa 4'750km
vom Nordpol entfernt.

Den Südpol gebe es dagegen nicht, stattdessen würde dort eine Eiswand
verhindern, dass man den Kontinent Antarctica überqueren kann.
Die Sonne bewege sich einmal am Tag entlang eines Kreises von variablem
Radius etwa 5'000km über der Erdscheibe.
Die Jahreszeiten entstehen dadurch, dass der Radius dieses Kreises
über das Jahr schwankt.
Zur Zeit der Tag- und Nachtgleiche ist der Radius 10'000km.

\begin{teilaufgaben}
\item
In diesem Modell geht die Sonne nie unter (was die Flat-Earther anscheinen
nicht stört).
Wie gross ist der kleinstmögliche Winkel über dem Horizont, unter dem
man die Sonne von der HSR aus sehen kann.
\item
Sei $P$ der Punkt auf dem "Aquator mit dem gleichen Längengrad wie
die HSR.
Am Tag der Tag- und Nachtgleiche befindet sich in diesem Modell
die Sonne bei ``Sonnenaufgang'' morgens um 06:00 Uhr über einem Strahl
vom Nordpol aus, der senkrecht ist auf dem Strahl vom Nordpol zu $P$.
Um welche Zeit ist von $P$ aus gesehen der Winkel zwischen der Position
der Sonne und der Position bei ``Sonnenaufgang'' $90^\circ$?
Man vergleiche das mit der Realität, in der die Sonne an jedem Tag,
nicht nur zur Tag- und Nachtgleiche, genau um 12:00 
$90^\circ$ von der Position entfernt ist, an der sie um 06:00 am
gleichen Tag war.
\end{teilaufgaben}

\thema{Drehmatrix}

\begin{loesung}
Wir verwenden ein Koordinatensystem mit Nullpunkt im Nordpol,
dessen $x$-Achse durch die HSR geht, die $z$-Achse ist senkrecht
auf der Erdscheibe im Nordpol.
Die HSR hat in diesem Koordinatensystem den Ortsvektor
\[
\vec h=\begin{pmatrix}4750\\0\\0\end{pmatrix}.
\]
\begin{teilaufgaben}
\item
Am tiefsten Punkt befindet sich die Sonne diametral gegenüber der HSR,
sie hat den Ortsvektor
\[
\vec s_{\text{Mitternacht}}=\begin{pmatrix}-10000\\0\\5000\end{pmatrix}.
\]
Der gesuchte Winkel $\alpha$ kann mit einem rechtwinkligen Dreieck berechnet
werden:
\[
\tan\alpha = \frac{5000}{4750-(-10000)}=0.33898
\qquad\Rightarrow\qquad
\alpha=18.726^\circ.
\]
\item
Wir messen die Zeit mit Hilfe des Winkels $\varphi$ zwischen der Richtung
der Sonnenposition um Mitternacht, die in Teilaufgabe a) verwendet wurde,
und der aktuellen Position.
Der Ortsvektor der Sonnen zur Zeit $\varphi$ ist
\[
\vec{s}(\varphi) =
\begin{pmatrix}
-10000\cdot\cos\varphi\\
10000\cdot\sin\varphi\\
5000
\end{pmatrix}.
\]
Gesucht ist dasjenige $\varphi$, für das 
\[
(\vec s(90^\circ) - \vec p)\cdot(\vec s(\varphi)-\vec p)=0
\]
gilt.
Wir lösen das Problem gleich für beliebige Punkte auf dem Meridian
durch die HSR, und setzen daher 
\[
\vec p=\begin{pmatrix}10000r\\0\\0\end{pmatrix},
\]
so dass für $r=1$ der in der Aufgabe verlangte Punkt auf dem "Aquator
verwendet wird.

Setzt man diese Werte in die Vektorgleichung ein, erhält man
\begin{align*}
0
&=
\left(
\begin{pmatrix}
-10000\cdot\cos 90^\circ\\
 10000\cdot\sin 90^\circ\\
5000
\end{pmatrix}
-\begin{pmatrix}10000r\\0\\0\end{pmatrix}
\right)
\cdot
\left(
\begin{pmatrix}
-10000\cdot\cos\varphi\\
 10000\cdot\sin\varphi\\
5000
\end{pmatrix}
-\begin{pmatrix}10000r\\0\\0\end{pmatrix}
\right)
\\
&=
\begin{pmatrix}
-10000r\\
 10000\\
 5000
\end{pmatrix}
\cdot
\begin{pmatrix}
-10000r
 10000\cdot\cos\varphi\\
-10000\cdot\sin\varphi\\
  5000
\end{pmatrix}
=
10000^2
\begin{pmatrix}
r\\
1\\
0.5
\end{pmatrix}
\cdot
\begin{pmatrix}
r
+
\cos\varphi\\
\sin\varphi\\
0.5
\end{pmatrix}
\\
&=
10000^2 ( r^2 + r\cos\varphi+\sin\varphi +0.25)
\\
&=
10000^2 ( r^2 + r\cos\varphi\pm\sqrt{1-\cos^2\varphi}+0.25)
\end{align*}
Bringt man die Wurzel auf eine Seite
\begin{align*}
r\cos\varphi + 0.25+r^2&=\mp\sqrt{1-\cos^2\varphi}\\
u\cos\varphi + v&=\mp\sqrt{1-\cos^2\varphi}
\end{align*}
mit
\[
\begin{aligned}
u&=r,&
v&=0.25 + r^2.
\end{aligned}
\]
Quadrieren wir die Gleichung und fassen Potenzen von $\cos\varphi$
zusammen, erhalten wir
\begin{align*}
u^2\cos^2\varphi + 2uv\cos\varphi+v^2&=1-\cos^2\varphi
\\
(u^2+1)\cos^2\varphi +2uv\cos\varphi+v^2-1&=0,
\end{align*}
eine quadratische Gleichung in $\cos\varphi$.

Setzt man die numerischen Werte $r=1$ ein, erhält man für
den Winkel
\[
\cos\varphi= -0.9557189138830738
\qquad\Rightarrow\qquad
\varphi = 10.859\text{h}.
\]
Da die Sonne um 06:00 bereits über dem Horizont ist (mehr als $18^\circ$,
wie wir aus Teilaufgabe a) wissen), ist dies nicht die richtige Zeit.
Wir brauchen einen Winkel, dessen $\sin\varphi<0$ ist, der aber
den gleichen $\cos\varphi$ hat.
Der gesuchte Winkel ist $24\text{h}-\varphi=13.141\text{h}$, die
in der Aufgabenstellung beschriebene Situation tritt also erst um
$13:08:28$ ein. 
\end{teilaufgaben}
Man müsste meinen, dass diese Unstimmigkeiten selbst einem Flat Earther 
auffallen müssten.
Eine schnelle Suche auf Youtube zeigt jedoch, dass Flat Earther in
erstaunlicher Zahl Experimente genau dieser Art durchführen, und keine
Diskrepanzen feststellen.

Führt man die Rechnung für verschiedene Werte von $r$ durch erkennt man,
dass für gemässigte Breiten (Werte um 0.5) der Fehler erstaunlich klein
ist.
\end{loesung}


\begin{bewertung}
Position der Sonne um Mitternacht ({\bf P}) 1 Punkt,
Elevation ({\bf E}) 1 Punkt,
Teilaufgabe b) 4 Punkte.
\end{bewertung}

