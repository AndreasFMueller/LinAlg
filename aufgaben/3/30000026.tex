Gegeben sind die Vektoren
\[
v_1
=
\begin{pmatrix*}[r]
100\\
0
\end{pmatrix*}
\quad\text{und}\quad
v_2
=
\begin{pmatrix*}[r]
0\\
100
\end{pmatrix*}
\quad\text{und die Matrix}\quad
D
=
\begin{pmatrix}
0 & -1 \\
1 &  0
\end{pmatrix},
\]
die eine Drehung um $90^\circ$ beschreibt.
\begin{teilaufgaben}
\item
Stellen Sie die $3\times 3$-Matrizen $T_i$ auf, die eine Verschiebung
um $v_i$ beschreiben, und die $3\times 3$-Matrix $O$, die Drehung um
$90^\circ$ beschreibt.
\item
Bestimmen Sie die Zusammensetzung von $T_2$ gefolgt von $O$ gefolgt
von $T_1$.
\end{teilaufgaben}

\begin{loesung}
\begin{teilaufgaben}
\item
Die Matrizen sind
\[
T_1
=
\begin{pmatrix*}[r]
1 & 0 & 100 \\
0 & 1 &   0 \\
0 & 0 &   1
\end{pmatrix*},
\quad
T_2
=
\begin{pmatrix*}[r]
1 & 0 &   0 \\
0 & 1 & 100 \\
0 & 0 &   1
\end{pmatrix*}
\quad\text{und}\quad
O
=
\begin{pmatrix*}[r]
0 & -1 & 0 \\
1 &  0 & 0 \\
0 &  0 & 1
\end{pmatrix*}.
\]
\item
Die Zusammesetzung ist
\begin{align*}
T_1OT_2
&=
\begin{pmatrix*}[r]
1 & 0 & 100 \\
0 & 1 &   0 \\
0 & 0 &   1
\end{pmatrix*}
\begin{pmatrix*}[r]
0 & -1 & 0 \\
1 &  0 & 0 \\
0 &  0 & 1
\end{pmatrix*}
\begin{pmatrix*}[r]
1 & 0 &   0 \\
0 & 1 & 100 \\
0 & 0 &   1
\end{pmatrix*}
=
\begin{pmatrix*}[r]
1 & 0 & 100 \\
0 & 1 &   0 \\
0 & 0 &   1
\end{pmatrix*}
\begin{pmatrix*}[r]
0 & -1 & -100 \\
1 &  0 &    0 \\
0 &  0 &    1
\end{pmatrix*}
=
\begin{pmatrix*}[r]
0 & -1 & 0 \\
1 &  0 & 0 \\
0 &  0 & 1
\end{pmatrix*}.
\end{align*}
Dies ist wieder die Drehung um $90^\circ$.
\qedhere
\end{teilaufgaben}
\end{loesung}
