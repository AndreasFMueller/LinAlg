Ortonormalisieren Sie die drei Vektoren
\[
\vec{a}_1=\begin{pmatrix}1\\1\\0\end{pmatrix},\qquad
\vec{a}_2=\begin{pmatrix}0\\1\\1\end{pmatrix},\qquad
\vec{a}_3=\begin{pmatrix}1\\0\\1\end{pmatrix}.
\]

\thema{Orthonormalisierung}

\begin{loesung}
Wir verwenden das Gram-Schmidtsche Orthogonalisierungsverfahren:
\begin{align*}
\vec{b}_1
&=
\frac{\vec{a}_1}{|\vec{a}_1|}
=
\frac1{\sqrt{2}}\begin{pmatrix}1\\1\\0\end{pmatrix}
\\
\vec{b}_2
&=
\frac{\vec{a}_2 - (\vec{a}_2\cdot\vec{b}_1)\vec{b}_1}{\dots}
=
\frac{\displaystyle\begin{pmatrix}0\\1\\1\end{pmatrix}-\frac1{\sqrt{2}}\frac1{\sqrt{2}}\begin{pmatrix}1\\1\\0\end{pmatrix}}{\dots}
=
\frac{\displaystyle\frac12\begin{pmatrix}-1\\1\\2\end{pmatrix}}{\dots}
=
\frac{1}{\sqrt{6}}\begin{pmatrix}-1\\1\\2\end{pmatrix}
\\
\vec{b}_3
&=
\frac{\vec{a}_3-(\vec{a}_3\cdot\vec{b}_1)\vec{b}_1-(\vec{a}_3\cdot\vec{b}_2)\vec{b}_2}{\dots}
=
\left(
\begin{pmatrix}1\\0\\1\end{pmatrix}
-\frac1{\sqrt{2}}\frac1{\sqrt{2}}\begin{pmatrix}0\\1\\1\end{pmatrix}
-\frac1{\sqrt{6}}\frac1{\sqrt{6}}\begin{pmatrix}-1\\1\\2\end{pmatrix}
\right)^0
=
\begin{pmatrix}4\\-4\\4\end{pmatrix}^0
=
\begin{pmatrix}1\\-1\\1\end{pmatrix}^0
\\
&=
\frac1{\sqrt{3}} \begin{pmatrix}1\\-1\\1\end{pmatrix}.
\qedhere
\end{align*}
\end{loesung}

