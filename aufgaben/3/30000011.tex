Im Vektorraum $\mathbb R^2$ soll statt der Standardbasis die Basis $C$ mit
Bassivektoren
\[
\begin{pmatrix}
\frac12\\
\frac{\sqrt{3}+1}2
\end{pmatrix},\quad
\begin{pmatrix}
-1\\
0
\end{pmatrix}
\]
verwendet werden.
Bestimmen Sie die Transformationsmatrix.
Welche Komponenten hat der Vektor
\[
x=
\begin{pmatrix}
-\frac12\\
\frac{\sqrt{3}+1}2
\end{pmatrix}
\]
in der Basis $C$?

\thema{Basis}
\thema{Abbildungsmatrix}
\thema{inverse Matrix}
\thema{Minoren}

\begin{loesung}
Den Basen entsprechen die Matrizen
\[
B=\begin{pmatrix}1&0\\0&1\end{pmatrix},\qquad
C=\begin{pmatrix}
\frac12&-1\\
\frac{\sqrt{3}+1}2&0
\end{pmatrix}.
\]
In der Vorlesung wurde gefunden, dass Transformationsmatrix in diesem
Fall ($B=E$) die Inverse von $C$ ist, also
\[
T=C^{-1}=\frac2{\sqrt{3}+1}\begin{pmatrix}
0&1\\
-\frac{\sqrt{3}+1}2&\frac12
\end{pmatrix}
=\begin{pmatrix}
0&\frac{2}{\sqrt{3}+1}\\
-1&\frac1{\sqrt{3}+1}
\end{pmatrix}
=
\begin{pmatrix}
0&\sqrt{3}-1\\
-1&\frac{\sqrt{3}-1}{2}
\end{pmatrix}
\]
Die Inverse wurde hier mit Hilfe von Minoren berechnet.
Mit der Matrix $T$ kann man jetzt auch den gegebenen Vektor $x$  in
die neue Basis umrechnen:
\[
y=
Tx
=
\begin{pmatrix}
0&\sqrt{3}-1\\
-1&\frac{\sqrt{3}-1}{2}
\end{pmatrix}
\begin{pmatrix}
-\frac12\\
\frac{\sqrt{3}+1}2
\end{pmatrix}
=
\begin{pmatrix}1\\
\frac12+\frac12\end{pmatrix}
=\begin{pmatrix}1\\1\end{pmatrix}.
\qedhere
\]
\end{loesung}

