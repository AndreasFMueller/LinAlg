Stellen Sie die Gleichungen der Geraden auf, die zur Geraden mit der
Gleichung
\[
6x-8y-13=0
\]
den Abstand $4.5$ haben.

\thema{Hessesche Normalform}

\begin{loesung}
Gesucht ist die Menge der Punkte, deren Ortsvektoren von der gegebenen
Geraden eine bestimmten Abstand haben, dazu brauchen wir zunächst die
Abstandsformel. Diese erhalten wir, indem wir die Gleichung so skalieren,
dass die Koeffizienten von $x$ und $y$ eine Einheitsvektor bilden,
wegen $\sqrt{6^2+8^2}=10$ ist das
\[
\frac{6}{10}x-\frac{8}{10}y-\frac{13}{10}=0
\]
Die Abstandsformel ist also
\[
d =\frac{6}{10}x-\frac{8}{10}y-\frac{13}{10}
\]
Die beiden Ebenen sind
\begin{align*}
4.5&=\frac{6}{10}x-\frac{8}{10}y-\frac{13}{10}&&\Rightarrow&
6x-8y&=13+45=58
\\
-4.5&=\frac{6}{10}x-\frac{8}{10}y-\frac{13}{10}&&\Rightarrow&
6x-8y&=13-45=-32
\qedhere
\end{align*}
\end{loesung}

