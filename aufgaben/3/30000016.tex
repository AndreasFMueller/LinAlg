Der Mittelpunkt einer Kugel bewegt sich auf der Geraden durch
die Punkt $A=(2,2,1)$ und $B=(0,0,\frac12)$. Welchen Durchmesser
darf sie höchstens haben, wenn sie noch durch das dreieckig
Fenster mit Ecken $C=(2,0,0)$, $D=(0,2,0)$ und $E=(0,0,2)$ passen
soll?

\thema{Abstand windschiefer Geraden}

\begin{loesung}
Die Frage ist, wie gross der minimale Abstand zwischen
der Geraden $AB$ und jeder der Kanten des Dreiecks, also
der Geraden $CD$, $DE$ und $CE$ ist.

Den Abstand zweier windschiefer Geraden kann man mit dem
Abstandsformel
\[
\frac{(\vec r_1\times \vec r_2)\cdot (\vec p_1-\vec p_2)}{|\vec r_1\times \vec r_2|}
\]
ausrechnen.
Dazu braucht man die Richtungsvektoren aller beteiligten Geraden.
\begin{align*}
\overrightarrow{AB}
&=
\begin{pmatrix} -2\\-2\\-\frac12 \end{pmatrix}
\\
\overrightarrow{CD}
&=
\begin{pmatrix} -2\\2\\0\end{pmatrix}
\\
\overrightarrow{DE}
&=
\begin{pmatrix} 0\\-2\\2\end{pmatrix}
\\
\overrightarrow{CE}
&=
\begin{pmatrix} -2\\0\\2\end{pmatrix}
\end{align*}
Jetzt sind die Vektorprodukte von $\overrightarrow{AB}$ mit den
anderen Vektoren auszurechnen:
\begin{align*}
\overrightarrow{AB}\times\overrightarrow{CD}
&=
\begin{pmatrix} -2\\-2\\-\frac12 \end{pmatrix}
\times
\begin{pmatrix} -2\\2\\0\end{pmatrix}
%=
%\begin{pmatrix}
%(-2)\cdot 0-2\cdot(-\frac12)\\
%(-\frac12)\cdot(-2)-0\cdot(-2)\\
%(-2)\cdot 2-(-2)\cdot(-2)
%\end{pmatrix}
=
\begin{pmatrix}
1\\1\\-8
\end{pmatrix},
&
|\overrightarrow{AB}\times\overrightarrow{CD}|
&=
\sqrt{66}
\\
\overrightarrow{AB}\times\overrightarrow{DE}
&=
\begin{pmatrix} -2\\-2\\-\frac12 \end{pmatrix}
\times
\begin{pmatrix} 0\\-2\\2\end{pmatrix}
=
\begin{pmatrix} -5\\4\\4\end{pmatrix},
&
|\overrightarrow{AB}\times\overrightarrow{DE}|
&=
\sqrt{57}
\\
\overrightarrow{AB}\times\overrightarrow{CE}
&=
\begin{pmatrix} -2\\-2\\-\frac12 \end{pmatrix}
\times
\begin{pmatrix} -2\\0\\2\end{pmatrix}
=
\begin{pmatrix} -4\\5\\-4\end{pmatrix},
&
|\overrightarrow{AB}\times\overrightarrow{CE}|
&=
\sqrt{57}
\end{align*}
Als Ausgangspunkt für die Gerade $AB$ nehmen wir $A$, für
die anderen Geraden jeweils die alphabetisch frühere Ecke.
\begin{align*}
\overrightarrow{AC}
&=
\begin{pmatrix}0\\-2\\-1\end{pmatrix}
&
\overrightarrow{AC}
\cdot
\frac{\overrightarrow{AB}\times\overrightarrow{CD}}{|\overrightarrow{AB}\times\overrightarrow{CD}|}
&=
\frac{0-2+8}{\sqrt{66}}=\frac6{\sqrt{66}}=0.73855
\\
\overrightarrow{AD}
&=
\begin{pmatrix}-2\\0\\-1\end{pmatrix}
&
\overrightarrow{AD}
\cdot
\frac{ \overrightarrow{AB}\times\overrightarrow{DE}}{|\overrightarrow{AB}\times\overrightarrow{DE}|}
&=
\frac{10+0-4}{\sqrt{57}}=\frac{6}{\sqrt{57}}=0.79472
\\
\overrightarrow{AC}
&=
\begin{pmatrix}0\\-2\\-1\end{pmatrix}
&
\overrightarrow{AC}
\cdot
\frac{ \overrightarrow{AB}\times\overrightarrow{CE}}{|\overrightarrow{AB}\times\overrightarrow{CE}|}
&=
\frac{0-10+4}{\sqrt{57}}=-\frac{6}{\sqrt{57}}=-0.79472
\end{align*}
Der geringste Abstand entsteht also zur Kante $CD$, der Radius darf
$r=\frac6{\sqrt{66}}$ nicht überschreiten, wenn die Kugel die Kante
in der $x$-$y$-Ebene nicht berühren soll.
\end{loesung}

