Die Vektoren
\[
\vec u=\begin{pmatrix}1\\1\\1\end{pmatrix}
\qquad\text{und}\qquad
\vec v=\begin{pmatrix}1\\-1\\1\end{pmatrix}
\]
liegen in einer Ebene $\sigma$.
In welchem Punkt schneidet die Normale auf die Ebene $\sigma$
durch den Punkt $(1,2,3)$ die Ebene mit der Gleichung $x=5$?

\begin{loesung}
Die Normale hat die Richtung
\[
\vec n=\vec u\times \vec v=
\begin{pmatrix}1\\1\\1\end{pmatrix}
\times
\begin{pmatrix}1\\-1\\1\end{pmatrix}
=
\begin{pmatrix}
1\cdot 1-(-1)\cdot 1\\
1\cdot 1-1\cdot 1\\
1\cdot (-1)-1\cdot 1
\end{pmatrix}
=
\begin{pmatrix}
2\\0\\-2
\end{pmatrix}.
\]
Die Normale hat die Parameterdarstellung
\[
\begin{pmatrix}
1\\
2\\
3
\end{pmatrix}+t\begin{pmatrix}2\\0\\-2\end{pmatrix}
\]
Gesucht ist also der Parameter $t$ so, dass $x=5=1+2t$, also $t=2$. Der
Schnittpunkt ist damit $(5,2,-1)$.
\end{loesung}

\begin{bewertung}
Normale als Vektorprodukt der Vektoren $\vec u$ und $\vec v$ ({\bf N})
1 Punkt,
Berechnung des Vektorprodukts ({\bf V}) 1 Punkt,
Parameterdarstellung der Geraden ({\bf P}) 1 Punkt,
Schnittbedingung $x = 5$ ({\bf B}) 1 Punkt,
Berechnung von $t$ ({\bf T}) 1 Punkt,
Schnittpunkt ({\bf S}) 1 Punkt.
\end{bewertung}


