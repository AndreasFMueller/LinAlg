Wir betrachten den würfelförmigen Hohlraum mit Ecken $(\pm1,\pm1,\pm1)$
(Abbildung links).
Im Inneren befindet sich ein gerades Rohr mit Durchmesser $0.1$ zwischen
den Punkten $(0.5,-1,-0.5)$ und $(-0.5,1,0.5)$.
Nun soll ein weiteres Rohr mit dem gleichen Durchmesser installiert werden.
Es soll durch eine Öffnung im Punkt $P=(-0.5,-0.5,-1)$ eingeführt werden und 
muss einen Punkt an der Rückwand ($z=1$) erreichen, welcher aber von aussen nicht
erkennbar ist.
Daher wird eine Kamera mit einem $320\times 180$-Chip und Brennweite $f=50$
in der Öffnung bei $P$ platziert.
\begin{center}
\begin{tikzpicture}[scale=1]
\node at (0,0) {
\includeagraphics[width=0.35\hsize]{box.jpg}
};
\node at (8.5,0) {
\includeagraphics[width=0.622222\hsize]{inside.jpg}
};
\node at (-2.2,-0.7) {$x$};
\node at (-0,3.2) {$y$};
\node at (+2.2,0.1) {$z$};
\node at (+2.3,-0.6) {$\leftarrow P$};
%\node at (9.25,0.4) {$B$};
\node at (7.75,0.4) {$B$};
\end{tikzpicture}
\end{center}
Auf dem Kamerabild (Abbildung rechts) hat der Zielpunkt die Koordinaten
$B=(179,109)$.
Die Ausrichtung der Kamera wird durch die Drehmatrix
\[
D
=
\begin{pmatrix}
   0.99504 & 0.00000 &         - 0.09950\\
   0.00000 & 1.00000 &\phantom{-}0.00000\\
   0.09950 & 0.00000 &\phantom{-}0.99504
\end{pmatrix}
\]
gegeben.
Kann das neue Rohr installiert werden, ohne vom bereits vorhandenen
Rohr blockiert zu werden?

\begin{hinweis}
Berechnen Sie die Matrizen- und Vektoroperationen mit dem Taschenrechner.
\end{hinweis}

\thema{Kamerageometrie}
\thema{Abstand windschiefer Geraden}

\begin{loesung}
Das bestehende Rohr hat die Geradengleichung in Parameterdarstellung
\[
\vec p
=
\vec p_0
+
t\vec r_0
=
\begin{pmatrix}
\phantom{-}0.5\\
-1\phantom{.0}\\
-0.5
\end{pmatrix}
+t\begin{pmatrix}
-1\\
\phantom{-}2\\
\phantom{-}1
\end{pmatrix}.
\]

Die Achse des neuen Rohrs ist eine Gerade durch den Punkt $P$, 
beziehungsweise das Kamerazentrum mit Ortsvektor
\[
\vec c
=
\begin{pmatrix}
-0.5\\
-0.5\\
-1\phantom{.0}
\end{pmatrix}.
\]
Der Richtungsvektor ist $\vec r$. Er muss aus dem Bildpunkt $B$
bestimmt werden.

Die Kameramatrix kann aus den gegebenen Daten abgelesen werden, 
sie ist
\[
K
=
\begin{pmatrix}
50& 0&160\\
 0&50& 90\\
 0& 0&  1
\end{pmatrix}.
\]
Die Gleichungen der Kamerageometrie liefern den Richtungsvektor
\[
\vec r
=
(KD)^{-1}\begin{pmatrix}179\\109\\1\end{pmatrix}
=
\begin{pmatrix}
   0.47761 \\
   0.38000 \\
   0.95723
\end{pmatrix}.
\]

Ob sich das neue Rohr installieren lässt hängt davon ab, ob die beiden
Geraden einen windschiefen Abstand $\ge 0.1$ haben.
Dafür können wir die Abstandsformel für den windschiefen Abstand
\[
d
=
\frac{(\vec r\times\vec r_0)\cdot (\vec c - \vec p_0)}{|\vec r\times \vec r_0|}
\]
verwenden.
Die Rechnung ergibt
\[
|d| = 0.060027 < 0.1,
\]
das Rohr lässt sich also nicht installieren, weil der windschiefe Abstand
zu klein ist.
\end{loesung}

\begin{bewertung}
Parameterdarstellung der bestehenden Gerade ({\bf B}) 1 Punkt,
Kamerazentrum und Stützvektor der Gerade neuen ({\bf C}) 1 Punkt,
Kameramatrix ({\bf K}) 1 Punkt,
Richtungsvektor der neuen Geraden ({\bf R}) 1 Punkt,
Abstandsformel für windschiefe Geraden ({\bf A}) 1 Punkt,
Abstandsberechnung und Entscheidung ({\bf E}) 1 Punkt.
\end{bewertung}

