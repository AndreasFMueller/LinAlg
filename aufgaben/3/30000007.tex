Im allgemeinen ist $(\vec a\cdot \vec b )^2\ne \vec a^2\vec b^2$.
Finden Sie Bedingungen, unter denen Gleichheit trotzdem gilt.

\thema{Skalarprodukt}

\begin{loesung}
Nach Definition ist
\[
\vec a\cdot\vec b=|\vec a|\;|\vec b|\cos\alpha ,
\]
also findet man durch quadrieren
\[
(\vec a\cdot\vec b)^2
=
|\vec a|^2|\vec b|^2 \cos^2 \alpha
=
\vec a^2\vec b^2 \cos^2 \alpha.
\]
Das sind genau die Terme, die gemäss Aufgabenstellung verglichen werden
sollen. Offenbar sind sie genau dann gleich, wenn $\cos^2\alpha=1$,
also
$\cos\alpha=\pm 1$. Die Vektoren müssen also parallel oder
entegegengesetzt gerichtet sein.
\end{loesung}

