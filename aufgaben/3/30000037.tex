Die Vektoren
\[
\vec a
=
\begin{pmatrix}2\\9\\1\end{pmatrix}
\qquad
\text{und}
\qquad
\vec b
=
\begin{pmatrix}9\\1\\8\end{pmatrix}
\]
spannen im Punkt $0$ eine Ebene $\sigma$ auf. Wie weit ist der Punkt
$P=(157,-17,-1)$ von der Ebene $\sigma$ entfernt?

\begin{loesung}
Die Normale der Ebene ist
\[
\vec n=
\vec a\times\vec b
=
\begin{pmatrix}2\\9\\1\end{pmatrix}
\times
\begin{pmatrix}9\\1\\8\end{pmatrix}
=
\begin{pmatrix}
9\cdot 8-1\cdot 1\\
1\cdot 9-2\cdot 8\\
2\cdot 1-9\cdot 9
\end{pmatrix}
=
\begin{pmatrix}
71\\-7\\-79
\end{pmatrix}
\]
Um den Abstand des Punktes $P$ von der Ebene zu berechnen, verwenden
wir die Abstandsformel
\begin{align*}
d&=\frac{\vec p\cdot (\vec a\times\vec b)}{|\vec a\times\vec b|}
=\frac1{\sqrt{5041+49+6241}}\begin{pmatrix}157\\-17\\-1\end{pmatrix}\cdot
\begin{pmatrix}
71\\-7\\-79
\end{pmatrix}
=\frac{11147+119+79}{\sqrt{11331}}=\frac{11345}{\sqrt{11331}}=106.58.
\qedhere
\end{align*}
\end{loesung}

\begin{bewertung}
Normale (\textbf{N}) 1 Punkt, Normierung (\textbf{E}) 1 Punkt, 
Ortsvektor (\textbf{O}) 1 Punkt,
Abstandsformel (\textbf{A}) 1 Punkt, Skalarprodukt (\textbf{S}) 1 Punkt,
Abstand (\textbf{D}) 1 Punkt.
\end{bewertung}

