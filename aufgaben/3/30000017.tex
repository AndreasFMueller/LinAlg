Liegt der Ursprung  auf der Ebene durch die Punkte
$A=(-2, 0, 1)$, $B=(4,0,-2)$ und $C=(-1,-4,3)$?

\thema{Ebene}

\begin{loesung}
Dieser Fall tritt ein, wenn die Ortsvektoren der drei Punkte
in einer Ebene liegen, die drei Ortsvektoren also linear abhängig
sind. Dies kann man zum Beispiel mit dem Gaussalgorithmus testen:
\begin{align*}
\begin{tabular}{|ccc|}
\hline
$-2$&4&$-1$\\
0&0&$-4$\\
1&$-2$&3\\
\hline
\end{tabular}
&\rightarrow
\begin{tabular}{|ccc|}
\hline
1&$-2$&3\\
0&0&$-4$\\
$-2$&4&$-1$\\
\hline
\end{tabular}
\\
&\rightarrow
\begin{tabular}{|ccc|}
\hline
1&$-2$&3\\
0&0&$-4$\\
0&0&5\\
\hline
\end{tabular}
\\
&\rightarrow
\begin{tabular}{|ccc|}
\hline
1&$-2$&3\\
0&0&1\\
0&0&0\\
\hline
\end{tabular}
\end{align*}
Die Matrix hat also nur den Rang $2$, insbesondere ist sie singulär,
die Vektoren sind linear abhängig, der Ursprung ist also ein Punkt
der Ebene.

Alternativ kann man die Ebene in Parameterdarstellung finden:
\begin{align*}
\vec r&=\overrightarrow{OA}+\lambda \overrightarrow{AB}+\mu\overrightarrow{AC}\\
&=
\begin{pmatrix}
-2\\0\\1
\end{pmatrix}
+
\lambda
\begin{pmatrix}
6\\0\\-3
\end{pmatrix}
+
\mu
\begin{pmatrix}
1\\-4\\2
\end{pmatrix}
\end{align*}
und $\lambda$ und $\mu$ so zu bestimmen versuchen, dass
der Nullvektor entsteht, also
\[
\begin{pmatrix}
-2\\0\\1
\end{pmatrix}
+
\lambda
\begin{pmatrix}
6\\0\\-3
\end{pmatrix}
+
\mu
\begin{pmatrix}
1\\-4\\2
\end{pmatrix}
=
\begin{pmatrix}
0\\0\\0
\end{pmatrix}.
\]
Dies führt auf das überbestimmte Gleichungssystem
\[
\begin{linsys}{3}
6\lambda&+&\mu&=&2\\
&-&\mu&=&0\\
-3\lambda&+&2\mu&=&-1\\
\end{linsys}
\]
Aus der zweiten Gleichung folgt $\mu=0$. Damit werden die anderen
zwei Gleichungen zu $3\lambda=1$ oder $\lambda=\frac13$. Die Gleichungen
sind also lösbar, somit liegt der Nullpunkt auf der Ebene.

Schliesslich kann auch der Normalenvektor bestimmt werden:
\[
\vec n=
\begin{pmatrix}
6\\0\\-3
\end{pmatrix}
\times
\begin{pmatrix}
1\\-4\\2
\end{pmatrix}
=\begin{pmatrix}
0\cdot 2-(-4)\cdot(-3)\\
(-3)\cdot 1-2\cdot 6\\
6\cdot(-4)-1\cdot 0
\end{pmatrix}
=
\begin{pmatrix}
-12\\
-15\\
-24
\end{pmatrix}
\]
Der Ursprung liegt in der Ebene, wenn $\overrightarrow{OA}$ auf
$\vec n$ senkrecht steht:
\[
\vec n\cdot\overrightarrow{OA}
=
\begin{pmatrix}
-12\\ -15\\ -24 \end{pmatrix}
\cdot
\begin{pmatrix}
-2\\0\\1
\end{pmatrix}
=(-12)\cdot(-2)+(-15)\cdot 0+(-24)\cdot 1=0,
\]
der Ursprung liegt also tatsächlich auf der Ebenen.
\end{loesung}

