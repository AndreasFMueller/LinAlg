Finden Sie die Ebenengleichung in der Form
\[
ax+by+cz=d
\]
f"ur die Ebene mit Parameterdarstellung
\[
{\vec{p}}_0 +s{\vec{r}}_1 +t{\vec{r}}_2
=
\begin{pmatrix} 1\\-3\\-3\end{pmatrix}
+s
\begin{pmatrix} 2\\-5\\-2\end{pmatrix}
+t
\begin{pmatrix}-3\\12\\22\end{pmatrix}.
\]

\begin{loesung}
Die Punkte $\vec{p}_0$,
$\vec{p}_0+\vec{r}_1$ und 
$\vec{p}_0+\vec{r}_2$ m"ussen sich auf der Ebene befinden, daher muss das
Gleichungssystem
\[
\begin{linsys}{4}
  a&-&3b&-& 3c&-&d&=&0\\
 3a&-&8b&-& 5c&-&d&=&0\\
-2a&+&9b&+&19c&-&d&=&0
\end{linsys}
\]
f"ur die Koeffizienten $a$ bis $d$ erf"ullt sein.
Mit dem Gaussalgorithmus findet man
\begin{align*}
\begin{tabular}{|>{$}c<{$}>{$}c<{$}>{$}c<{$}>{$}c<{$}|>{$}c<{$}|}
\hline
  1& -3& -3& -1& 0\\
  3& -8& -5& -1& 0\\
 -2&  9& 19& -1& 0\\
\hline
\end{tabular}
&\rightarrow
\begin{tabular}{|>{$}c<{$}>{$}c<{$}>{$}c<{$}>{$}c<{$}|>{$}c<{$}|}
\hline
  1&  0&  0& 86& 0\\
  0&  1&  0& 38& 0\\
  0&  0&  1& -9& 0\\
\hline
\end{tabular}
\end{align*}
Daraus liest man ab, dass $d$ frei w"ahlbar ist, und mit der Wahl
$d=-1$ die Koeffizienten 
\[
a=86,\qquad
b=38,\qquad
c=-9
\]
sind.
Also ist die gesuchte Geradengleichung
\[
86x + 38y -9z=-1.
\qedhere
\]
\end{loesung}

