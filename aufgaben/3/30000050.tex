Die Vektoren
\[
\vec u=\begin{pmatrix}1\\1\\1\end{pmatrix}
\qquad\text{und}\qquad
\vec v=\begin{pmatrix}1\\-1\\1\end{pmatrix}
\]
liegen in einer Ebene $\sigma$. Der Punkt $P$ hat die Koordinaten $(1,2,3)$.
\begin{teilaufgaben}
\item Wie weit ist $P$ von der Ebene $\sigma$ entfernt?
\item Wie weit ist $P$ von jeder der Geraden durch $0$ mit Richtungsvektor
$\vec{u}$ bzw.~$\vec{v}$ entfernt?
\end{teilaufgaben}

\begin{loesung}
\begin{teilaufgaben}
\item Wir verwenden die hessesche Normalform, um den Abstand zu bestimmen.
Dazu brauchen wir die Normale, sie ist
\[
\vec n=\vec u\times \vec v=
\begin{pmatrix}1\\1\\1\end{pmatrix}
\times
\begin{pmatrix}1\\-1\\1\end{pmatrix}
=
\begin{pmatrix}
1\cdot 1-(-1)\cdot 1\\
1\cdot 1-1\cdot 1\\
1\cdot (-1)-1\cdot 1
\end{pmatrix}
=
\begin{pmatrix}
2\\0\\-2
\end{pmatrix}
\quad\Rightarrow\quad
\vec{n}^0
=
\frac{\sqrt{2}}2
\begin{pmatrix}1\\0\\-1\end{pmatrix}.
\]
Der Abstand wird mit Hilfe des Skalarproduktes gefunden:
\[
d
=
\vec{n}^0\cdot \vec{p}
=
\frac{\sqrt{2}}2
\begin{pmatrix}1\\0\\-1\end{pmatrix}
\begin{pmatrix}1\\2\\3\end{pmatrix}
=
\frac{\sqrt{2}}2(1-3)=-\sqrt{2}.
\]
\item
Die Abstandsformel Punkt-Gerade l"ost dieses Problem:
\begin{align*}
d_{\vec{u}}
&=
\frac{|\vec{p}\times \vec{u}|}{|\vec{u}|}
&
d_{\vec{v}}
&=
\frac{|\vec{p}\times \vec{v}|}{|\vec{v}|}
\\
&=
\frac{1}{\sqrt{3}}
\left|
\begin{pmatrix}1\\2\\3\end{pmatrix}
\times
\begin{pmatrix}1\\1\\1\end{pmatrix}
\right|
=
\frac{1}{\sqrt{3}}
\left|\begin{pmatrix}-1\\2\\-1\end{pmatrix}\right|
&
&=
\frac{1}{\sqrt{3}}
\left|
\begin{pmatrix}1\\2\\3\end{pmatrix}
\times
\begin{pmatrix}1\\-1\\1\end{pmatrix}
\right|
=
\frac{1}{\sqrt{3}}
\left|\begin{pmatrix}5\\2\\-3\end{pmatrix}\right|
\\
&=
\frac{\sqrt{6}}{\sqrt{3}}=\sqrt{2},
&
&=
\frac{\sqrt{25+4+9}}{\sqrt{3}}
=
\frac{\sqrt{38}}{\sqrt{3}}.
\qedhere
\end{align*}
\end{teilaufgaben}
\end{loesung}

