Die Cheops-Pyramide hat einen quadratischen Grundriss mit Seitenlänge
$a=440\,\text{Ellen}$, sie ist $h=280\,\text{Ellen}$ hoch.
\begin{center}
\includeagraphics[width=0.5\hsize]{cheops.jpg}
\end{center}
\begin{teilaufgaben}
\item
Berechnen Sie den Winkel zwischen zwei Seitenflächen.
\item
Die Königinnenkammer befindet sich genau unter der Spitze $61\,\text{Ellen}$
über der Grundfläche. Wieviele Ellen Stein müsste man durchbohren, wenn
man einen Tunnel senkrecht auf die Seitenfläche bis zur Königinnenkammer
bohren wollte?
\end{teilaufgaben}

\thema{Zwischenwinkel}
\thema{Abstand}

\begin{loesung}
Wir verwenden ein Koordinatensystem mit Nullpunkt im Zentrum der
Grundfläche. Die Ecken der Pyramide sind
\begin{align*}
E_1&=(-220,-220, 0),&
E_2&=( 220,-220, 0),&
E_3&=(-220, 220, 0),&
E_4&=( 220, 220, 0).
\end{align*}
Die Spitze der Pyramide ist $S=(0,0,280)$.
Für beide Teilaufgaben brauchen wir die Normalen auf Seitenflächen.
Wir berechnen die Normale von $\triangle E_1E_2S$:
\begin{align*}
\vec n
&=
\overrightarrow{SE_1}\times \overrightarrow{SE_2}
=
\begin{pmatrix}-220\\-220\\-280\end{pmatrix}
\times
\begin{pmatrix} 220\\-220\\-280\end{pmatrix}
=
\begin{pmatrix}
220\cdot 280-220\cdot280\\
-280\cdot220-280\cdot220\\
220\cdot220+220\cdot 220
\end{pmatrix}
=
\begin{pmatrix} 0\\ -123200\\ 96800 \end{pmatrix}
=
8800
\begin{pmatrix} 0\\ -14\\ 11 \end{pmatrix}
\end{align*}
\begin{teilaufgaben}
\item
Der Zwischenwinkel für die Flächen $\triangle E_1E_2S$ und
$\triangle E_2E_3S$ ist der Zwischenwinkel der Normalen der beiden
Dreiecke, also von
\[
\vec n_1
=
\begin{pmatrix} 0\\-14\\11 \end{pmatrix}
\qquad\text{und}\qquad
\vec n_2
=
\begin{pmatrix} 14\\0\\11 \end{pmatrix}.
\]
Dieser ist nach der Zwischenwinkelformel:
\begin{align*}
\cos\alpha
&=
\frac{\vec n_1\cdot \vec n_2}{|\vec n_1|\cdot|\vec n_2|}
=
\frac{11^2}{11^2+14^2}=\frac{121}{317}=0.38170
\\
\Rightarrow\qquad
\alpha
&=
67.561^\circ
\end{align*}
Diesem spitzen Winkel entspricht ein stumpfer Winkel von $112.44^\circ$.
\item
Die Königinnenkammer hat die Koordinaten $Q=(0,0,61)$.
Wir müssen den Abstand des Punktes $Q$ von der Ebene des Dreiecks
$\triangle SE_1E_2$ bestimmen.
Dazu stellen wir zuerst die Ebenengleichung in hessescher Normalform
auf:
\[
\vec n_1^0 \cdot (\vec p-\vec s)=0
\]
Setzt man darin die Königinnenkammer ein,  also $\vec p=\vec q$, dann
erhält man den gesuchten Abstand:
\[
d
=
\vec n_1^0 \cdot (\vec q-\vec s)
=
\frac1{\sqrt{14^2+11^2}}
\begin{pmatrix}0\\-14\\11 \end{pmatrix}
\cdot
\left(
\begin{pmatrix}0\\0\\61\end{pmatrix}
-
\begin{pmatrix}0\\0\\280\end{pmatrix}
\right)
=
\frac1{\sqrt{317}}
\begin{pmatrix}0\\-14\\11 \end{pmatrix}
\cdot
\begin{pmatrix}0\\0\\-219\end{pmatrix}
=
-\frac{2409}{\sqrt{317}}=-135.30
\]
Da sich die Königinnenkammer im Inneren der Pyramide befindet, wir aber
für die Ebene die nach aussen zeigende Normale verwendet haben, ergibt
sich erwartungsgemäss ein negativer Abstand.
Man müsste also durch $135.3\,\text{Ellen}$ Stein bohren, um die Aussenfläche
zu erreichen.
\qedhere
\end{teilaufgaben}
\end{loesung}

\begin{bewertung}
Normale auf Seitenfläche ({\bf N}) 1 Punkt,
Zwischenwinkelformel ({\bf ZW}) 1 Punkt,
Berechhnung des Zwischenwinkels ({\bf A}) 1 Punkt,
Abstandsformel oder Ebenengleichung ({\bf E}) 1 Punkt,
Position der Königinnenkammer ({\bf Q}) 1 Punkt,
Länge des Tunnels ({\bf L}) 1 Punkt.
\end{bewertung}

