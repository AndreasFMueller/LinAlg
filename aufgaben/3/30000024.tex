%Transformationsformel
Die Drehung um $90^\circ$ im Gegenuhrzeigersinn wird in kartesischen
Koordinaten durch die Matrix
\[
\renewcommand{\arraystretch}{1.2}
D
=
\begin{pmatrix*}[r]
0&-1\\
1& 0
\end{pmatrix*}
\]
beschrieben.
Jetzt soll Sie in einem neuen Koordinatensystem beschrieben werden,
die kartesischen Koordinaten werden mit der Matrix
\[
\renewcommand{\arraystretch}{1.2}
T
=
\begin{pmatrix*}[r]
1&-1\\
1& 3
\end{pmatrix*}
\quad\text{mit der Inversen}\quad
T^{-1}
=
\begin{pmatrix*}[r]
 \frac34&\frac14\\
-\frac14&\frac14
\end{pmatrix*}
\]
in die neuen Koordinaten umgerechnet.
Bestimmen Sie die Abbildungsmatrix im neuen Koordinatensystem.

\begin{loesung}
Die neue Abbildungsmatrix ist
\renewcommand{\arraystretch}{1.2}
\begin{align*}
D'
&=
TDT^{-1}
=
\begin{pmatrix*}[r]
1&-1\\
1& 3
\end{pmatrix*}
\begin{pmatrix*}[r]
0&-1\\
1& 0
\end{pmatrix*}
\begin{pmatrix*}[r]
 \frac34&\frac14\\
-\frac14&\frac14
\end{pmatrix*}
\\
&=
\begin{pmatrix*}[r]
1&-1\\
1& 3
\end{pmatrix*}
\begin{pmatrix*}[r]
\frac14&-\frac14\\
\frac34& \frac14
\end{pmatrix*}
\\
&=
\begin{pmatrix*}[r]
-\frac12&-\frac12\\
\frac52&\frac12
\end{pmatrix*}.
\qedhere
\end{align*}
\end{loesung}
