Die Ebene $\sigma_1$ sei die Ebene durch die Punkte $(1,0,0)$,
$(0,1,0)$ und $(0,0,1)$.
Die Ebene $\sigma_2$ hat die Gleichung
\[
x+2y+3z=4.
\]
Die Ebene $\sigma_3$ steht auf beiden Ebenen senkrecht und geht
durch den Nullpunkt des Koordinatensystems.
\begin{teilaufgaben}
\item
Finden Sie den Fusspunkt $F$ des Lotes vom Punkt $P=(10,-1,6)$ auf die
Ebene $\sigma_3$.
\item
Bestimmen Sie ausserdem die Länge des Lotes von $P$ auf die Ebene $\sigma_3$.
\end{teilaufgaben}

\thema{Vektorprodukt}
\thema{Abstand}

\begin{loesung}
\begin{teilaufgaben}
\item
Die Normale $\vec n_3$ von $\sigma_3$ steht auf den Normalen
$\vec n_1$ von $\sigma_1$ und $\vec n_2$ von $\sigma_2$.
Sie kann mit dem Vektorprodukt gefunden werden:
\begin{align*}
\vec n_3
&=
\vec n_1 \times \vec n_2
=
\begin{pmatrix}1\\1\\1\end{pmatrix}
\times
\begin{pmatrix}1\\2\\3\end{pmatrix}
=
\begin{pmatrix}
1\\-2\\1
\end{pmatrix},
\\
|\vec n_3|^2
&=
1+2^2 + 1=6.
\end{align*}
Da die Ebene $\sigma_3$ durch den Nullpunkt geht, ist ihre Gleichung
$\vec n_3\cdot\vec r=0$.
Der Fusspunkt $F$ des Lotes liegt auf der Ebene $\sigma_3$, es gilt
daher $\vec f\cdot \vec n_3 = 0$.
Ausserdem muss $\overrightarrow{PF}$ parallel zu $\vec n_3$ sein,
oder 
\[
\vec f = \vec p + t\vec n_3
\]
mit noch zu bestimmendem  Parameter $t$.
Multiplizieren wir diese Gleichung skalar mit $\vec n_3$ erhält man
\begin{align*}
\vec f\cdot \vec n_3
&=
\vec p\cdot\vec n_3 + t\vec n_3\cdot\vec n_3
\\
0
&=
\vec p\cdot\vec n_3 + t|\vec n_3|^2
\\
t
&=
-\frac{\vec p\cdot\vec n_3}{|\vec n_3|^2}
=
-\frac{10+2+6}{6}=-3
\end{align*}
Somit ist
\begin{align*}
\vec f
&=
\vec p -3\vec n_3
=
\begin{pmatrix}10\\-1\\6\end{pmatrix}
-3
\begin{pmatrix}1\\-2\\1\end{pmatrix}
=
\begin{pmatrix}7\\5\\3\end{pmatrix}
\end{align*}
\item
Die Strecke $PF$ ist dreimal so lang wie der Vektor $\vec n_3$, also ist
\[
\overline{PF}
=
3|\vec n_3|
=
3\sqrt{6}
=
7.348469.
\qedhere
\]
\end{teilaufgaben}
\end{loesung}

\begin{bewertung}
Normale $\vec n_1$ ({\bf N}) 1 Punkt,
Normale $\vec n_3$ (Vektorprodukt) ({\bf V}) 1 Punkt,
Geradengleichung für Lot ({\bf G}) 1 Punkt,
Ebenengleichung für $\sigma_3$ ($\mathbf{\Sigma}$) 1 Punkt,
Fusspunkt ({\bf F}) 1 Punkt,
Länge des Lotes ({\bf L}) 1 Punkt,
\end{bewertung}


