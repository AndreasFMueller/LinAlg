In einem zugegebenermassen nicht unbedingt zweckm"assig gew"ahlten
Koordinatensystem haben die horizontalen Kanten eines Wohnzimmers
die Darstellung
\[
\vec v_1=\begin{pmatrix}
1\\1\\1
\end{pmatrix}
\qquad\text{und}\qquad
\vec v_2=\begin{pmatrix}
1\\-2\\1
\end{pmatrix}.
\]
\begin{teilaufgaben}
\item Stehen die Kanten senkrecht aufeinander?
\item Finden Sie einen Vektor parallel zur vertikalen Kante
des Zimmers.
\item Finden Sie eine Gleichung f"ur die Ebene des Wohnzimmerbodens, wenn
sich der Koordinatennullpunkt in einer unteren Ecke befindet.
\item In Zimmer soll jetzt ein Weihnachtsbaum aufgestellt werden.
Leider gelingt es nicht, den Baum in genau vertikale Position zu
bringen. Man muss sich damit begn"ugen, dass es zwei Richtungen
gibt, aus denen der Baum gerade aussieht. Finden Sie Einheitsvektoren
f"ur diese Richtungen,
wenn der Stamm des Baums parallel ist zu
\[
\vec b=\begin{pmatrix}
1\\0\\2
\end{pmatrix}.
\]
\end{teilaufgaben}

\begin{loesung}
\begin{teilaufgaben}
\item Das Skalarprodukt ist
\[
\vec v_1\cdot\vec v_2
=
\begin{pmatrix}
1\\1\\1
\end{pmatrix}
\cdot
\begin{pmatrix}
1\\-2\\1
\end{pmatrix}
=1\cdot 1+1\cdot(-2)+1\cdot 1=1-2+1=0,
\]
also stehen die Vektoren senkrecht aufeinander.
\item Ein solcher Vektor kann mit dem Vektorprodukt gefunden werden:
\[
\vec n=\vec v_1\times\vec v_2
=
\begin{pmatrix}
1\\1\\1
\end{pmatrix}
\times
\begin{pmatrix}
1\\-2\\1
\end{pmatrix}
=
\begin{pmatrix}
1\cdot 1-1\cdot(-2)\\
1\cdot 1-1\cdot 1\\
1\cdot(-2)-1\cdot 1
\end{pmatrix}
=
\begin{pmatrix}
3\\0\\-3
\end{pmatrix}
\]
\item
Die Ebene besteht aus den Punkten mit Ortsvektoren $\vec p$
senkrecht zum in b) gefunden Vektor,
die Ebenengleichung ist also
\[
\vec n\cdot \vec p =0\quad\Rightarrow\quad 3x-3z=0.
\]
\item
Es muss die Projektion des Vektors auf die von den beiden Zimmerkanten
aufgespannte Ebene gefunden werden. Dazu muss die Komponenten
subtrahiert werden, die parallel zu dem in Teilaufgabe b) gefundenen
Vektor ist. Diese Komponente ist
\begin{align*}
\vec b_{\|}
&=
\left(\frac{\vec n}{|\vec n|}\cdot \vec b\right)\frac{\vec n}{|\vec n|}
=
\frac{\vec n\cdot\vec b}{\vec n\cdot \vec n} \vec n
\\
&
=
\frac{3\cdot 1+0\cdot 0+(-3)\cdot 2}{3^2+0^2+3^2}\begin{pmatrix}3\\0\\-3\end{pmatrix}
=
\frac{-3}{18}\begin{pmatrix}3\\0\\-3\end{pmatrix}
=
\frac{-1}{2}\begin{pmatrix}1\\0\\-1\end{pmatrix}
=
\begin{pmatrix}-\frac12\\0\\\frac12\end{pmatrix}
\end{align*}
Diese Komponente muss von $\vec b$ subtrahiert werden:
\[
\vec b -\vec b_{\|}
=\begin{pmatrix}1\\0\\2\end{pmatrix}
-
\begin{pmatrix}-\frac12\\0\\\frac12\end{pmatrix}
=
\begin{pmatrix}\frac32\\0\\\frac32\end{pmatrix}
\]
Man kontrolliert auch leicht, dass
$\vec b -\vec b_{\|} = v_1+\frac12v_2$, dass also der gefundene
Vektor tats"achlich in der verlangten Ebene liegt.
Er ist aber noch kein Einheitsvektor, durch Normieren finden wir
die zwei gesuchten Richtungen
\[
\vec r=\pm\frac1{\sqrt{2}}\begin{pmatrix}1\\0\\1\end{pmatrix}.
\qedhere
\]
\end{teilaufgaben}
\end{loesung}

