\bgroup
\definecolor{darkred}{rgb}{0.8,0,0}
Gegeben sind die beiden Basen
\[
\mathcal{B}
=
\left\{
\vec{b}_1
=
\begin{pmatrix*}[r]3\\-2\end{pmatrix*}
,
\vec{b}_2
=
\begin{pmatrix}1\\2\end{pmatrix}
\right\}
\qquad\text{und}\qquad
\mathcal{C}
=
\left\{
\vec{c}_1
=
\begin{pmatrix}4\\0\end{pmatrix}
,
\vec{c}_2
=
\begin{pmatrix}0\\8\end{pmatrix}
\right\}.
\]
Stellen Sie das Tableau zur Berechnung der Transformationsmatrix $T$ auf,
um Koordinaten in der Basis $\color{blue}\mathcal{C}$ in Koordinaten
in der Basis $\color{darkred}\mathcal{B}$ umzurechnen.
\egroup

\begin{loesung}
Das Tableau und die Anwendung des Gauss-Algorithmus ist
\definecolor{darkred}{rgb}{0.8,0,0}
\[
\renewcommand{\arraystretch}{1.2}
\begin{tabular}{|>{$}r<{$}>{$}r<{$}|>{$}r<{$}>{$}r<{$}|}
\hline
{\color{darkred}y_1}&
{\color{darkred}y_2}&
{\color{blue}x_1}&
{\color{blue}x_2}\\
\hline
 3& 1& 4& 0\\
-2& 2& 0& 8\\
\hline
\end{tabular}
%
\rightarrow
%
\begin{tabular}{|>{$}r<{$}>{$}r<{$}|>{$}r<{$}>{$}r<{$}|}
\hline
{\color{darkred}y_1}&
{\color{darkred}y_2}&
{\color{blue}x_1}&
{\color{blue}x_2}\\
\hline
 1& \frac13& \frac43& 0\\
 0& \frac83& \frac83& 8\\
\hline
\end{tabular}
%
\rightarrow
%
\begin{tabular}{|>{$}r<{$}>{$}r<{$}|>{$}r<{$}>{$}r<{$}|}
\hline
{\color{darkred}y_1}&
{\color{darkred}y_2}&
{\color{blue}x_1}&
{\color{blue}x_2}\\
\hline
 1& \frac13& \frac43& 0\\
 0& 1& 1& 3\\
\hline
\end{tabular}
%
\rightarrow
%
\begin{tabular}{|>{$}r<{$}>{$}r<{$}|>{$}r<{$}>{$}r<{$}|}
\hline
{\color{darkred}y_1}&
{\color{darkred}y_2}&
{\color{blue}x_1}&
{\color{blue}x_2}\\
\hline
 1& 0& 1& -1\\
 0& 1& 1&  3\\
\hline
\end{tabular}.
\]
Die Transformationsmatrix ist daher
\[
T
=
\begin{pmatrix*}[r]
1&-1\\
1& 3
\end{pmatrix*}.
\qedhere
\]
\end{loesung}


