Gegeben sind die Punkte 
\[
\begin{aligned}
A&=(-1,-2,-3)&
B&=(1,1,2)   &
C&=(-2,3,-4) \\
D&=(0,0,-3)  &
E&=(12,9,1)  &
F&=(9,13,-9)
\end{aligned}
\]
\begin{teilaufgaben}
\item
Finden Sie Normalenform der Gleichung einer Ebene durch den Punkt $E$,
die zu den Geraden $AB$ und $CD$ parallel ist.
\item
Wie weit ist der Punkt $F$ von der Ebene entfernt?
\end{teilaufgaben}

\begin{loesung}
\begin{teilaufgaben}
\item
Die Ebenengleichung ist in Parameterdarstellung 
\[
\vec{p}
=
\vec{e}+t(\vec{b}-\vec{a})+s(\vec{d}-\vec{c})
=
\begin{pmatrix}12\\9\\1\end{pmatrix}
+t
\begin{pmatrix}2\\3\\5\end{pmatrix}
+s
\begin{pmatrix}2\\-3\\1\end{pmatrix}.
\]
Die Ebene hat als Normale das Vektorprodukt der Vektoren $\overrightarrow{AB}$
und $\overrightarrow{CD}$:
\begin{align*}
\vec{n}
&=
\overrightarrow{AB}
\times
\overrightarrow{CD}
=
\begin{pmatrix}2\\3\\5\end{pmatrix}
\times
\begin{pmatrix}2\\-3\\1\end{pmatrix}
=
\begin{pmatrix} 18\\8\\-12 \end{pmatrix}.
\end{align*}
Die Normalenform der Ebenengleichung ist daher
\begin{equation}
\vec{n}\cdot (\vec{p}-\vec{e})=0.
\label{30000061:normalenform}
\end{equation}
\item
Um den Abstand zu berechnen, brauchen wir die Hessesche Normalform
von \eqref{30000061:normalenform}.
Dazu multiplizieren wir aus
\[
\vec{n}\cdot (\vec{p}-\vec{e})
=
\begin{pmatrix} 18\\8\\-12 \end{pmatrix}
\cdot\left(
\begin{pmatrix}x\\y\\z\end{pmatrix}
-
\begin{pmatrix}12\\9\\1\end{pmatrix}
\right)
=
18x+8y-12z-276.
\]
Die Hessesche Normalform erhalten wir, indem wir durch die Norm
\[
|\vec{n}|
=
\sqrt{18^2+8^2+12^2}
=
\sqrt{532}
\]
der Koeffizenten der Koordinaten dividieren.
Die Entfernung des Punktes $F$ von der Ebene ist daher
\[
d
=
\frac{1}{|\vec{n}|}
(\vec{n}\cdot\vec{f}-\vec{n}\cdot\vec{e})
=
\frac{1}{|\vec{n}|}
\vec{n}\cdot(\vec{f}-\vec{e})
=
\frac1{\sqrt{532}}
\begin{pmatrix}18\\8\\-12\end{pmatrix}
\cdot
\begin{pmatrix}-3\\4\\-10\end{pmatrix}
=
\frac{98}{\sqrt{532}}
=
4.2488.
\qedhere
\]
\end{teilaufgaben}
\end{loesung}

\begin{bewertung}
Richtungsvektoren ({\bf R}) 1 Punkt,
Normale ({\bf N}) 1 Punkt,
Ebenengleichung ({\bf E}) 1 Punkt,
Länge von $\vec{n}$ ({\bf L}) 1 Punkt,
Hessesche Normalform ({\bf H}) 1 Punkt,
Abstand ({\bf A}) 1 Punkt.
\end{bewertung}


