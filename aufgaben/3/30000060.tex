Eine sogenannte Blasenkammer kann dazu verwendet werden, 
die Bahnen subatomarer Teilchen sichtbar zu machen.
Sie funktioniert wie folgt.
In eine mit flüssigem Wasserstoff gefüllte Kammer werden Elementarteilchen
injiziert.
Kurz zuvor wird der Druck stark reduziert, so dass die Temperatur
der Flüssigkeit jetzt über dem Siedepunkt liegt.
Die Teilchen ionisieren einzelne Wasserstoff-Moleküle, welche als
Keime für Gasblasen dienen.
Wenige Millisekunden später werden die Gasblasen mit Blitzlicht
sichtbar gemacht und mit mehreren Kameras aufgenommen.
Am CERN in Genf wurde 1973 mit der grossen Blasenkammer {\em Gargamelle} das
$Z$-Boson nachgewiesen.

Eine Kamera mit einem $320\times 180$-Chip und Brennweite $f=135$ 
beobachtet eine würfelförmige Blasenkammer (Abbildung links) mit den Ecken
$(\pm1,\pm1,\pm1)$ von einem Beobachtungsfenster im Punkt
$P_0=(0.5,0.5,-1)$ aus und hat die nachstehende Aufnahme (rechts) gemacht.
\begin{center}
\begin{tikzpicture}[thick,>=latex,scale=1.1]

\begin{scope}[xshift=-3cm]
\node at (0,0) {
\includeagraphics[width=0.33\hsize]{chamber.jpg}
};
\node at (-2.15,-1.00) {$x$};
\node at ( 1.45,-1.50) {$z$};
\node at (-0.05, 1.99) {$y$};
\node at (-1.20,-1.05) {$I$};
\node at ( 0.30,-1.20) {$P_0$};
\node at ( 0.75, 0.93) {$P_1$};
\node at (-0.60, 1.70) {$P_2$};
\node at (-0.80, 0.40) {$S$};
\end{scope}

\begin{scope}[xshift=5cm]
\node at (0,0) {
\includeagraphics[width=0.583\hsize]{capture.jpg}
};
\node at (-4.45,-0.8) {$I$};
\node at ( 0.10, 1.9) {$B_2$};
\node at ( 1.50, 0.6) {$B_1$};
\node at (-0.70, 0.3) {$S$};
\end{scope}

\end{tikzpicture}
\end{center}
Ein Teilchen tritt beim Punkt $I=(1,0,0)$ in die Kammer ein.
Beim Punkt $S=(\frac18,\frac18,\frac38)$ zerfällt es.
Auf der Aufnahme sind zwei Spuren zu sehen, die bei den Pixelkoordinaten
$B_1=(204,114)$ und $B_2=(160,150)$ enden, weil dort die Teilchen die Kammer
verlassen.
Die Orientierung der Kamera wird durch die Drehmatrix
\[
D=\begin{pmatrix}
0.894&\phantom{-}0.000&\phantom{-}0.447\\
0.183&          -0.913&          -0.365\\
0.408&\phantom{-}0.408&          -0.816
\end{pmatrix}
\]
gegeben.
\begin{teilaufgaben}
\item
Berechnen Sie die Koordinaten der Austrittspunkte $P_1$ und $P_2$ der Teilchen.
\item
Ein Physiker will wissen, welchen Winkel die Bahnen der Zerfallsprodukte
einschliessen.
Berechnen Sie den Winkel $\angle P_1SP_2$.
\end{teilaufgaben}

\begin{hinweis}
Berechnen Sie die Matrizen- und Vektoroperationen mit dem Taschenrechner.
\end{hinweis}

\begin{loesung}
\begin{teilaufgaben}
\item
Die Kameramatrix $K$ kann aus den Aufgabendaten abgelesen werden, sie ist
\[
K
=
\begin{pmatrix}
135&  0&160\\
  0&135& 90\\
  0&  0&  1
\end{pmatrix}.
\]
Zunächst müssen wir aus den Punkten $B_i$ dreidimensionale Vektoren
in homogenen Koordinaten machen, also
\[
\begin{aligned}
\vec b_1 &= \begin{pmatrix}204\\114\\1\end{pmatrix}
&&\text{und}&
\vec b_2 &= \begin{pmatrix}160\\150\\1\end{pmatrix}.
\end{aligned}
\]
Mit der Formel $\vec{r} = (KD)^{-1} \vec b$ können wir zu jedem Punkt $P_i$
den Richtungsvektor
\[
\begin{aligned}
\vec r_1
&=
\begin{pmatrix}
\phantom{-}0.73258 \\
\phantom{-}0.24636 \\
-0.73602
\end{pmatrix}
&&\text{und}&
\vec r_2 &= \begin{pmatrix} 
\phantom{-}0.48963\\
\phantom{-}0.00283\\
         - 0.97926
\end{pmatrix}
\end{aligned}
\]
der Geraden von der Kamera in $P_0$ aus zum Punkt $P_i$ berechnen.

Der Punkt $P_1$ liegt auf der Ebene mit $x=-1$ und er erfüllt die
Gleichung $\vec{p}_1 = \vec{p}_0 + t\vec{r}_1$.
Wir kennen die $x$-Koordinate von $\vec{p}_0$, sie ist $0.5$.
Die $x$-Koordinate muss daher die Gleichung 
\[
0.5 + t\cdot 0.73258 = -1
\qquad\Rightarrow\qquad
t = \frac{-1.5}{0.73258}=-2.0476.
\]

Analog gilt $\vec{p}_2=\vec{p}_0+s\vec{r}_2$.
Die $z$-Koordinate davon ist
\[
1=-1+s\cdot (-0.97926)
\qquad\Rightarrow\qquad
s = \frac2{-0.97926}=-2.0424.
\]

Einsetzen der Werte für $t$ und $s$ ergibt
\[
\begin{aligned}
P_1&=(-1, -0.004, -0.507)
&&\text{und}&
P_2&=(-0.5,0.494, 1)
\end{aligned}
\]
für die gesuchten Punkte.
\item
Der Winkel kann mit der Zwischenwinkelformel berechnet werden.
Es gilt
\begin{align*}
\cos\alpha
&=
\frac{\overrightarrow{SP_1}\cdot\overrightarrow{SP_2}}{|\overrightarrow{SP_1}|\cdot|\overrightarrow{SP_2}|}
=
\frac{0.73786}{1.1401\cdot 0.9579} = 0.67564,
\\
\Rightarrow\qquad
\alpha
&=
47.496^\circ.
\qedhere
\end{align*}
\end{teilaufgaben}
\end{loesung}

\begin{diskussion}
Man kann die Punkte $P_i$ natürlich auch mit der konventionellen Methode
als Durchstoss\-punkte der Geraden $\vec{p}_0+t\vec{r}_i$ mit einer
der Seitenebenen der Kammer berechnen.
Für $P_1$ verwendet man die Ebenengleichung $x=-1$
und für $P_2$ die Ebenengleichung $z=1$.
So entstehen die Gleichungssysteme
\begin{align*}
&
\begin{tabular}{| >{$}c<{$} >{$}c<{$} >{$}c<{$} >{$}r<{$}| >{$}r<{$}|}
\hline
x&y&z&t&\\
\hline
1&0&0&-0.73258&0.5\\
0&1&0&-0.24636&0.5\\
0&0&1& 0.73602&-1\\
1&0&0& 0.00000&-1\\
\hline
\end{tabular}
&&
\begin{tabular}{| >{$}c<{$} >{$}c<{$} >{$}c<{$} >{$}r<{$}| >{$}r<{$}|}
\hline
x&y&z&s&\\
\hline
1&0&0&-0.48963&0.5\\
0&1&0&-0.00283&0.5\\
0&0&1& 0.97926&-1\\
0&0&1& 0.00000& 1\\
\hline
\end{tabular}
\\
\intertext{mit den Lösungen}
&
\begin{tabular}{| >{$}c<{$} >{$}c<{$} >{$}c<{$} >{$}r<{$}| >{$}r<{$}|}
\hline
x&y&z&t&\\
\hline
1&0&0&0& -1.00000\\
0&1&0&0& -0.00444\\
0&0&1&0& 0.50704\\
0&0&0&1& -2.04756\\
\hline
\end{tabular}
&&
\begin{tabular}{| >{$}c<{$} >{$}c<{$} >{$}c<{$} >{$}r<{$}| >{$}r<{$}|}
\hline
x&y&z&s&\\
\hline
1&0&0&0& -0.50000\\
0&1&0&0& 0.49422\\
0&0&1&0& 1.00000\\
0&0&0&1& -2.04236\\
\hline
\end{tabular}
\end{align*}
in Übereinstimmung mit den oben gefundenen Daten.
\end{diskussion}

\begin{bewertung}
Kameramatrix $K$ ({\bf K}) 1 Punkt,
Punkte $B_i$ in homogenen Koordinaten ({\bf B}) 1 Punkt,
Richtungsvektoren $\vec{r}_i$ ({\bf R}) 1 Punkt,
Punkt $P_1$ ($\text{\bf P}_1$) 1 Punkt,
Punkt $P_2$ ($\text{\bf P}_2$) 1 Punkt,
Zwischenwinkel ({\bf W}) 1 Punkt.
\end{bewertung}


