Schneidet die Gerade durch die Punkte $P=(2,0,0)$ und $Q=(7,4,-5)$
die Ebene durch
die drei Punkte $A=(-2,0,1)$, $B=(4,0,-2)$ und $C=(-1,-4,3)$?

\thema{Ebene}
\thema{Durchstosspunkt}

\begin{loesung}
Die Parameterdarstellung für die Gerade $PQ$ ist
\[
\vec r=
\begin{pmatrix} 2\\0\\0 \end{pmatrix}
+\lambda \overrightarrow{PQ}
=
\begin{pmatrix} 2\\0\\0 \end{pmatrix}
+\lambda
\begin{pmatrix}  5\\4\\-5\end{pmatrix}.
\]
Die Parameterdarstellung für die Ebene durch $A$, $B$ und $C$ ist
\[
\vec r
=
\begin{pmatrix}6\\0\\-3\end{pmatrix}
+\mu \overrightarrow{AB}
+\nu \overrightarrow{AC}
=
\begin{pmatrix}-2\\0\\1\end{pmatrix}
+\mu \begin{pmatrix}6\\0\\-3\end{pmatrix}
+\nu \begin{pmatrix}1\\-4\\2\end{pmatrix}
\]
Wenn sich Gerade und Ebene schneiden sollen, muss es eine Lösung
$(\lambda,\mu,\nu)$ des Gleichungssystems
\[
\begin{linsys}{3}
-5\lambda&+&6\mu&+& \nu&=& 4\\
-4\lambda& &    &-&4\nu&=& 0\\
5\lambda&-&3\mu&+&2\nu&=&-1
\end{linsys}
\]
Die reduzierte Form ergibt
\begin{center}
\begin{tabular}{|ccc|c|}
\hline
1&0&$-1$&0\\
0&1& 1&0\\
0&0& 0&0\\
\hline
\end{tabular}
\end{center}
insbesondere ist das Gleichungssystem singulär und es gibt keine
Lösung, also können sich Gerade und Ebene auch nicht schneiden.

Alternativ kann man auch wie folgt vorgehen.
Ein Schnittpunkt kann auf zwei Arten entstehen, einerseits indem
die Gerade $PQ$ in der Ebene liegt, oder dadurch, dass der Vektor
von $P$ nach $Q$ mit der Ebene einen positiven Winkel bildet. Der
letzte Fall tritt ein, wenn die drei Vektoren
\[
\overrightarrow{AB}
\text{, }
\overrightarrow{AC}
\text{ und }
\overrightarrow{PQ}
\]
linear unabhängig sind. Der Punkt $P$ ist nicht in der Ebene enthalten,
wenn die drei Vektoren
\[
\overrightarrow{AB}
\text{, }
\overrightarrow{AC}
\text{ und }
\overrightarrow{AP}
\]
linear unabhängig sind.
Die genannten Vektoren sind
\[
\overrightarrow{AB}=\begin{pmatrix}6\\0\\-3\end{pmatrix}, \qquad
\overrightarrow{AC}=\begin{pmatrix}1\\-4\\2\end{pmatrix}, \qquad
\overrightarrow{AP}=\begin{pmatrix}4\\0\\2\end{pmatrix}, \qquad
\overrightarrow{PQ}=\begin{pmatrix}5\\4\\-5\end{pmatrix}, \qquad
\]
Es ist also nur zu bestimmen, ob die Matrizen
\[
S=\begin{pmatrix}
6&1&4\\
0&-4&0\\
-3&2&2
\end{pmatrix}
\qquad
T=\begin{pmatrix}
6&1&5\\
0&-4&4\\
-3&2&-5
\end{pmatrix}
\]
singulär sind. Wir berechnen dies mit Hilfe des Gauss-Algorithmus.
Für die erste Matrix ergibt dieser, leicht modifiziert im ersten
Schritt, wo wir zur Vereinfachung die erste Spalte durch $-3$, die letze Zeile
mit der  erste Zeile vertauschen und die zweite Zeile durch $-4$ teilen:
\begin{align*}
\begin{tabular}{|ccc|}
\hline
6&1&4\\
0&$-4$&0\\
$-3$&2&2\\
\hline
\end{tabular}
&\rightarrow
\begin{tabular}{|ccc|}
\hline
1&2&2\\
0&1&0\\
$-2$&1&4\\
\hline
\end{tabular}
\rightarrow
\begin{tabular}{|ccc|}
\hline
1&2&2\\
0&1&0\\
0&5&8\\
\hline
\end{tabular}
\rightarrow
\begin{tabular}{|ccc|}
\hline
1&2&2\\
0&1&0\\
0&0&8\\
\hline
\end{tabular}
\end{align*}
Die Vektoren sind also linear unabhängig, der Punkt $P$ liegt ausserhalb
der Ebene. Für die zweite Matrix erhalten wir auf ähnliche Weise:
\begin{align*}
\begin{tabular}{|ccc|}
\hline
6&1&5\\
0&$-4$&4\\
$-3$&2&$-5$\\
\hline
\end{tabular}
&\rightarrow
\begin{tabular}{|ccc|}
\hline
1&2&$-5$\\
0&1&$-1$\\
$-2$&1&5\\
\hline
\end{tabular}
\rightarrow
\begin{tabular}{|ccc|}
\hline
1&2&$-5$\\
0&1&$-1$\\
0&5&$-5$\\
\hline
\end{tabular}
\rightarrow
\begin{tabular}{|ccc|}
\hline
1&2&$-5$\\
0&1&$-1$\\
0&0&0\\
\hline
\end{tabular}
\end{align*}
Da die Matrix singulär ist, folgt, dass der Vektor von $P$ nach $Q$ parallel
zur Ebene verläuft, dass also die Gerade $PQ$ eine zur Ebene parallele
Gerade ist.
\end{loesung}

