Berechnen Sie den Zwischenwinkel zwischen den Vektoren
\[
\vec a
=
\begin{pmatrix}-6\\8\\0\end{pmatrix}
,\qquad
\vec b
=
\begin{pmatrix}3\\-4\\12\end{pmatrix}
\]

\thema{Zwischenwinkel}
\thema{Skalarprodukt}

\begin{loesung}
Das Skalarprodukt ist
\[
\vec a\cdot\vec b=(-6)\cdot 3+8\cdot(-4)+0\cdot 12=-18-32=-50.
\]
Die Länge der Vektoren ist
\begin{align*}
\vec a\cdot\vec a&=6^2+8^2=36+64=100&\Rightarrow\quad|\vec a|&=10\\
\vec b\cdot\vec b&=3^2+4^2+12^2=9+16+144=169&\Rightarrow\quad|\vec b|&=13
\end{align*}
Für den Cosinus des Zwischenwinkels gilt also
\[
\cos \alpha=\frac{-50}{10\cdot 13}=-0.38461538461538461538
\]
und damit für den Zwischenwinkel
\[
\alpha=112.619865^\circ.
\qedhere
\]
\end{loesung}

