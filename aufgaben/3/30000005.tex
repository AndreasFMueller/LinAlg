Die Gerade
\[
\vec r
=
\begin{pmatrix}6\\0\\3\end{pmatrix}
+t
\begin{pmatrix}4\\-1\\1\end{pmatrix}
\]
schneidet die Kugel um den Nullpunkt mit Radius $3$ in zwei Punkten.
Berechnen Sie den Winkel zwischen den Tangentialebenen in diesen
Punkten.

\thema{Zwischenwinkel}
\thema{Kugel}

\begin{loesung}
Die Gleichung der Kugel ist $\vec r^2=9$. Wir setzen die Geradengleichung
darin ein:
\begin{align*}
9&=\left(
\begin{pmatrix}6\\0\\3\end{pmatrix}+t\begin{pmatrix}4\\-1\\1\end{pmatrix}
\right)^2
\\
&=
\begin{pmatrix}6\\0\\3\end{pmatrix}
\cdot
\begin{pmatrix}6\\0\\3\end{pmatrix}
+2t
\begin{pmatrix}6\\0\\3\end{pmatrix}
\cdot
\begin{pmatrix}4\\-1\\1\end{pmatrix}
+t^2
\begin{pmatrix}4\\-1\\1\end{pmatrix}
\cdot
\begin{pmatrix}4\\-1\\1\end{pmatrix}
\\
&=(36+9)+2(24+3)t+(16+1+1)t^2\\
&=45+54t+18t^2
\\
0&=18t^2+54+36
\\
0&=t^2+3t+2=(t+1)(t+2)
\end{align*}
Die Lösungen sind
\[
t_{\pm}=\frac{-3\pm\sqrt{3^2-4\cdot1\cdot 2}}{2\cdot 1}
=\frac{-3\pm1}{2}
=\begin{cases}
-1\\
-2
\end{cases}
\]
Setzt man die beiden $t$-Werte ein, ergeben sich die Punkte
\begin{align*}
\vec r_+&=
\begin{pmatrix}6\\0\\3\end{pmatrix}
-
\begin{pmatrix}4\\-1\\1\end{pmatrix}
=
\begin{pmatrix}2\\1\\2\end{pmatrix}
\\
\vec r_-&=
\begin{pmatrix}6\\0\\3\end{pmatrix}
-2
\begin{pmatrix}4\\-1\\1\end{pmatrix}
=
\begin{pmatrix}-2\\2\\1\end{pmatrix}
\end{align*}
Diese Vektoren sind auch die Normalen auf den Tangentialebenen,
somit ist der Schnittwinkel der Tangetialebenen einfach der
Zwischenwinkel der beiden Vektoren, den man mit der Zwischenwinkelformel
berechnen kann:
\[
\cos \alpha=
\frac{\vec r_+\cdot\vec r_-} {|\vec r_+|\;|\vec r_-|}
=\frac{-4+2+2}{9\cdot 9}=0
\]
Die beiden Ebenen stehen also senkrecht.
\end{loesung}

