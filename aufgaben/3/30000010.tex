Für den dreidimensionalen Raum möchte man die Vektoren
\[
B=\left\{
b_1=\begin{pmatrix}0\\0\\1\end{pmatrix},
b_2=\begin{pmatrix}-1\\-3\\-1\end{pmatrix},
b_3=\begin{pmatrix}1\\2\\5\end{pmatrix}
\right\}
\]
als Basis verwenden. Bilden sie tatsächlich eine Basis?

\thema{Basis}

\begin{loesung}
Die drei Vektoren bilden eine Basis, wenn jeder Vektor $v$ eindeutig
als Linearkombination der Vektoren $b_i$ geschrieben werden kann.
Dazum muss das Gleichungssystem $Bx = v$ für jeden Vektor eine
eindeutige Lösung haben. Dies tritt genau dann ein, wenn die Matrix
$B$ regulär ist. Das kann man aber mit dem Gauss-Algorithmus prüfen:
\[
\begin{tabular}{|>{$}c<{$}>{$}c<{$}>{$}c<{$}|}
\hline
0&-1&1\\
0&-3&2\\
1&-1&5\\
\hline
\end{tabular}
\rightarrow
\begin{tabular}{|>{$}c<{$}>{$}c<{$}>{$}c<{$}|}
\hline
1&-1&5\\
0&-1&1\\
0&-3&2\\
\hline
\end{tabular}
\rightarrow
\begin{tabular}{|>{$}c<{$}>{$}c<{$}>{$}c<{$}|}
\hline
1&-1&5\\
0&1&-1\\
0&0&-1\\
\hline
\end{tabular}
\]
Weiter müssen wir den Gaussalgorithmus gar nicht treiben, man kann
schon jetzt sehen dass $\det(B)\ne 0$, also $B$ regulär ist.
\end{loesung}

