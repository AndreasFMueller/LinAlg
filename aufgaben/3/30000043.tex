In einem Raum ist zwischen den Punkten $A=(1,2,9)$ und $B=(5,5,1)$
ein gerades Rohr mit Durchmesser $1$ installiert.
Jetzt sollen weitere solche Rohre installiert werden:
%\begin{teilaufgaben}
%\item
Ist es m"oglich, zwischen den Punkten zwischen $C=(3,0,0)$ und $D=(9,10,4)$
ein zweites gerades Rohr mit Durchmesser $1$ zu installieren, so dass sich
die Rohre nicht gegenseitig in den Weg kommen?
%\item
%Zwischen den Punkt $E=(4,0,3)$ und $F=(2,10,7)$ soll ein drittes
%solches gerades Rohr mit Durchmesser $1$ installiert werden.
%Es ist bekannt dass sich die Rohre schneiden, ja dies ist sogar 
%beabsichtigt, die Rohre sollen sich genau zentral durchstossen.
%Geht das?
%\item
%Wenn sich die beiden Rohre in b) genau zentral treffen: in welchem Winkel
%schneiden sie sich?
%\end{teilaufgaben}

\begin{loesung}
\begin{figure}
\includeagraphics[width=0.5\hsize]{tubes.jpg}
\includeagraphics[width=0.5\hsize]{intersect.jpg}
\caption{3D-Darstellung der R"ohren aus Aufgabe~\ref{30000043}.
Rechts die beiden Schnittstellen im Detail.
\label{30000043:tubes}}
\end{figure}
%\begin{teilaufgaben}
%\item
Das ist genau dann m"oglich, wenn der windschiefe Abstand der beiden
Geraden $AB$ und $CD$ gr"osser als $1$ ist.
Wir brauchen die Parameterdarstellungen der beiden Geraden:
\[
\begin{aligned}
&AB:&\vec{p}=\vec{p}_1+t\vec{r}_1
&=
\begin{pmatrix}1\\2\\9\end{pmatrix} + t\begin{pmatrix}\phantom{-}4\\\phantom{-}3\\-8\end{pmatrix}
\\
&CD:&\vec{p}=\vec{p}_2+t\vec{r}_2
&=
\begin{pmatrix}3\\0\\0\end{pmatrix} + t\begin{pmatrix}6\\10\\4\end{pmatrix}
\end{aligned}
\]
Der windschiefe Abstand ist
\begin{align*}
d
&=
\frac{(\vec{r}_1\times\vec{r}_2)\cdot (\vec{p}_2 - \vec{p}_1)}{|\vec{r}_1\times \vec{r}_2|}
\end{align*}
Wir berechnen zun"achst das Vektorprodukt:
\begin{align*}
\vec{r}_1\times \vec{r}_2
&=
\begin{pmatrix}\phantom{-}4\\\phantom{-}3\\-8\end{pmatrix}
\times
\begin{pmatrix}6\\10\\4\end{pmatrix}
=
{\def\arraystretch{2.2}
\begin{pmatrix}
{\def\arraystretch{1.0}
\left|\,\begin{matrix}\phantom{-}3&          10\\         - 8&\phantom{1}4\end{matrix}\,\right|}\\
{\def\arraystretch{1.0}
\left|\,\begin{matrix}         - 8&\phantom{1}4\\\phantom{-}4&\phantom{1}6\end{matrix}\,\right|}\\
{\def\arraystretch{1.0}
\left|\,\begin{matrix}\phantom{-}4&\phantom{1}6\\\phantom{-}3&          10\end{matrix}\,\right|}
\end{pmatrix}}
=
\begin{pmatrix}
\phantom{-}92\\-64\\\phantom{-}22
\end{pmatrix}.
\end{align*}
Seine L"ange ist 
\[
|\vec{r}_1\times\vec{r}_2|
=
\sqrt{92^2+64^2+22^2}=\sqrt{8464+4096+484}=\sqrt{13044}=114.2103.
\]
Damit kann jetzt der windschiefe Abstand berechnet werden
\begin{align*}
d
&=
\frac{(\vec{r}_1\times\vec{r}_2)\cdot (\vec{p}_2 - \vec{p}_1)}{|\vec{r}_1\times \vec{r}_2|}
=
\frac1{\sqrt{13044}}
\begin{pmatrix}92\\-64\\22\end{pmatrix}
\cdot
\begin{pmatrix}2\\-2\\-9 \end{pmatrix}
\\
&=
\frac{92\cdot 2+64\cdot 2-22\cdot 9}{\sqrt{13044}}=\frac{114}{114.2103}=0.99816.
\end{align*}
Der windschiefe Abstand ist also ein kleines bisschen zu klein, die Rohre
werden sich in die Quere kommen.
% \item
% Wir m"ussen herausfinden, ob sich die Geraden $AB$ und $EF$ schneiden.
% Die Parameterdarstellung der Gerade $EF$ ist
% \[
% \begin{aligned}
% &EF:&\vec{p}=\vec{p}_3+t\vec{r}_3
% &=
% \begin{pmatrix}4\\0\\3\end{pmatrix} + s\begin{pmatrix}-2\\10\\4\end{pmatrix}.
% \end{aligned}
% \]
% Das kombinierte Tableau zur Bestimmung des Schnittpunktes lautet
% \begin{align*}
% \begin{tabular}{|>{$}r<{$} >{$}r<{$} >{$}r<{$} >{$}r<{$} >{$}r<{$}|>{$}r<{$}|}
% \hline
% x&y&z&t&s&\\
% \hline
% 1&0&0&-4&  0&1\\
% 0&1&0&-3&  0&2\\
% 0&0&1& 8&  0&9\\
% 1&0&0& 0&  2&4\\
% 0&1&0& 0&-10&0\\
% 0&0&1& 0& -4&3\\
% \hline
% \end{tabular}
% &
% \rightarrow
% \begin{tabular}{|>{$}r<{$} >{$}r<{$} >{$}r<{$} >{$}r<{$} >{$}r<{$}|>{$}r<{$}|}
% \hline
% x&y&z&t&s&\\
% \hline
% 1&0&0&-4&  0& 1\\
% 0&1&0&-3&  0& 2\\
% 0&0&1& 8&  0& 9\\
% 0&0&0& 4&  2& 3\\
% 0&0&0& 3&-10&-2\\
% 0&0&0&-8& -4&-6\\
% \hline
% \end{tabular}
% \rightarrow
% \begin{tabular}{|>{$}r<{$} >{$}r<{$} >{$}r<{$} >{$}r<{$} >{$}r<{$}|>{$}r<{$}|}
% \hline
% x&y&z&t&s&\\
% \hline
% 1&0&0&-4&            0&          1\\
% 0&1&0&-3&            0&          2\\
% 0&0&1& 8&            0&          9\\
% 0&0&0& 1&      \frac12&    \frac34\\
% 0&0&0& 0&-\frac{23}{2}&-\frac{17}4\\
% 0&0&0& 0&            0&          0\\
% \hline
% \end{tabular}
% \end{align*}
% Weiter muss man nicht rechnen, die Nullzeile zeigt bereits an, dass eine
% L"osung existiert, die Geraden schneiden sich also.
% %rref(A)
% %ans =
% %
% %   1.00000   0.00000   0.00000   0.00000   0.00000   3.26087
% %   0.00000   1.00000   0.00000   0.00000   0.00000   3.69565
% %   0.00000   0.00000   1.00000   0.00000   0.00000   4.47826
% %   0.00000   0.00000   0.00000   1.00000   0.00000   0.56522
% %   0.00000   0.00000   0.00000   0.00000   1.00000   0.36957
% %   0.00000   0.00000   0.00000   0.00000   0.00000   0.00000
% %\item
% %Der Schnittwinkel kann aus den Richtungsvektoren $\vec{r}_1$ und
% %$\vec{r}_2$ der Geraden $AB$ und $EF$ berechnet werden:
% %\begin{align*}
% %\cos\alpha
% %&=
% %\frac{\vec{r}_1\cdot\vec{r}_2}{|\vec{r}_1|\cdot|\vec{r}_2|}
% %=
% %\frac{4\cdot(-2)+3\cdot 10+(-8)\cdot4}{\sqrt{16+9+64}\sqrt{4+100+16}}
% %=
% %\frac{-10}{\sqrt{89\cdot 120}}
% %=-0.096764\\
% %\Rightarrow\qquad
% %\alpha&=95.553^\circ
% %\end{align*}
% %Nat"urlich ist auch $\alpha=84.447^\circ$ eine korrekte Antwort.
% \end{teilaufgaben}
\end{loesung}

\begin{bewertung}
Parameterdarstellungen der Geraden ({\bf P}) 2 Punkt,
Abstand windschiefer Geraden ({\bf W}) 1 Punkt,
Vektorprodukt ({\bf v}) 1 Punkt,
Berechnung des Abstandes ({\bf A}) 1 Punkt,
Schlussfolgerung ({\bf S}) 1 Punkt.
\end{bewertung}

