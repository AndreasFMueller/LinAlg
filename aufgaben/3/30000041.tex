Die Punkte
\begin{align*}
P_1&=(1,0),
&P_2&=\biggl(\frac{-1+\sqrt{5}}4, \frac{\sqrt{10+2\sqrt{5}}}4 \biggr),
&P_3&=\biggl(\frac{-1-\sqrt{5}}4, \frac{\sqrt{10-2\sqrt{5}}}4 \biggr),\\
&&P_5&=\biggl(\frac{-1+\sqrt{5}}4,-\frac{\sqrt{10+2\sqrt{5}}}4 \biggr),
&P_4&=\biggl(\frac{-1-\sqrt{5}}4,-\frac{\sqrt{10-2\sqrt{5}}}4 \biggr)
\end{align*}
bilden in der $x$-$y$-Ebene ein regelm"assiges F"unfeck mit Seitenl"ange
\[
a=\sqrt{\frac{5-\sqrt{5}}2}.
\]
Es ist dem Einheitskreis einbeschrieben.
"Uber diesem F"unfeck  soll jetzt eine f"unfseitige Pyramide mit
Spitze $S$ errichtet werden.
Die Kanten der Pyramide sollen alle die gleiche L"ange haben.
\begin{teilaufgaben}
\item
Wie hoch ist die Pyramide?
\item
Bestimmen Sie die Normale der Ebene durch die Punkte
$S$, $P_3$ und $P_4$?
\item
Wie weit ist der Punkt $P_1$ von der Ebene durch die Punkte
$S$, $P_3$ und $P_4$ entfernt?
\end{teilaufgaben}

\begin{hinweis}
F"ur die Normale in Teilaufgabe b) reicht ein numerisches Resultat
mit 5 Stellen Genauigkeit.
\end{hinweis}


\begin{loesung}
\begin{teilaufgaben}
\item Die H"ohe kann mit dem Satz von Pythagoras ausgerechnet werden:
\[
h^2+1^2=a^2\quad a=\sqrt{a^2-1}
=
\sqrt{\frac{5-\sqrt{5}}2-1}
=
\sqrt{\frac{3-\sqrt{5}}2}.
\]
Wegen 
\[
\biggl(\frac{\sqrt{5}-1}2\biggr)^2=\frac{5-2\sqrt{5}+1}{4}=\frac{3-\sqrt{5}}2
\]
ist auch $a=(\sqrt{5}-1)/2$.
\item
Wir brauchen die Normale der Ebene durch die genannten Punkte.
Wir bekommen Sie als Vektorprodukt der Vektoren
$\overrightarrow{SP_3}$
und
$\overrightarrow{SP_4}$:
\begin{align*}
\vec n
&=
\left(
\begin{pmatrix}
\frac{-1-\sqrt{5}}4\\
\frac{\sqrt{10-2\sqrt{5}}}4\\
0
\end{pmatrix}
-
\begin{pmatrix}
0\\0\\\sqrt{\frac{3-\sqrt{5}}2}
\end{pmatrix}
\right)
\times
\left(
\begin{pmatrix}
\frac{-1-\sqrt{5}}4\\
-\frac{\sqrt{10-2\sqrt{5}}}4\\
0
\end{pmatrix}
-
\begin{pmatrix}
0\\0\\\sqrt{\frac{3-\sqrt{5}}2}
\end{pmatrix}
\right)
\\
&=
\begin{pmatrix}
\frac{-1-\sqrt{5}}4\\
\frac{\sqrt{10-2\sqrt{5}}}4\\
-\sqrt{\frac{3-\sqrt{5}}2}
\end{pmatrix}
\times
\begin{pmatrix}
\frac{-1-\sqrt{5}}4\\
-\frac{\sqrt{10-2\sqrt{5}}}4\\
-\sqrt{\frac{3-\sqrt{5}}2}
\end{pmatrix}
\\
&=\begin{pmatrix}
  -0.80902\\
  \phantom{-}0.58779\\
  -0.61803   
\end{pmatrix}
\times
\begin{pmatrix}
  -0.80902\\
  -0.58779\\
  -0.61803
\end{pmatrix}
=\begin{pmatrix}
  -0.72654\\
   0\\
   0.95106    
\end{pmatrix}.
\end{align*}
Die Normierung der Normalen ist nicht wichtig, wir h"atten f"ur die
Berechnung auch Vielfache der Faktoren im Vektorprodukt verwenden k"onnen.
\item
Die Abstandsformel liefert:
\begin{align*}
d&=\frac{\vec n\cdot \overrightarrow{SP_1}}{|\vec n|}
=
\begin{pmatrix}
-0.60706\\
\phantom{-}0.00000\\
\phantom{-}0.79465
\end{pmatrix}
\cdot
\begin{pmatrix}
-1\\0\\\sqrt{\frac{3-\sqrt{5}}2}
\end{pmatrix}
=
\begin{pmatrix}
-0.60706\\
\phantom{-}0.00000\\
\phantom{-}0.79465
\end{pmatrix}
\cdot
\begin{pmatrix}
-1.00000\\
\phantom{-}0.00000\\
\phantom{-}0.61803
\end{pmatrix}
\\
&=
0.60706+0.49112
=
1.0982.
\end{align*}
\end{teilaufgaben}
\end{loesung}

\begin{diskussion}
Solche f"unfseitigen Pyramiden enstehen als Ecken eines Ikosaeders.
\end{diskussion}

\begin{bewertung}
\begin{teilaufgaben}
\item
H"ohe der Pyramide ({\bf H}) 1 Punkt,
\item
Normale mit Hilfe des Vektorproduktes ({\bf N}) 1 Punkt,
Vektorproduktformel ({\bf V}) 1 Punkt,
Korrektes Resultat der Rechnung ({\bf R}) 1 Punkt,
\item
Abstandsformel ({\bf F}) 1 Punkt,
Abstand ({\bf A}) 1 Punkt.
\end{teilaufgaben}
\end{bewertung}




