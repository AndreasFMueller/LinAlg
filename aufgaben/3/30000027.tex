Bestimmen Sie die beiden winkelhalbierenden Geraden der zwei Geraden
mit den Gleichungen
\[
\begin{linsys}{2}
3x&+&4y&=&5\phantom{.}\\
\text{und}\qquad 4x&-&3y&=&-5.
\end{linsys}
\]

\begin{loesung}
Die Winkelhalbierenden sind die Mengen der Punkte, die von beiden
Geraden den gleichen Abstand haben. Die Hessesche Normalform erlaubt,
den Abstand eines Punktes von der Geraden zu bestimmen, dazu müssen
die Koeffizienten der Gleichungen so normalisiert werden, dass sie
einen Vektor der Länge 1 bilden:
\begin{align*}
H_1(x,y)&=\frac35x+\frac45y-1\\
H_2(x,y)&=\frac45x-\frac35y+1
\end{align*}
Die Hesseschen Normalformen $H_1$ und $H_2$ berechnen den Abstand
des Punktes $(x,y)$ von der Geraden. Die Punkte der Winkelhalbierenden
sind also diejeinigen Paare $(x,y)$, für die $H_1(x,y)=\pm H_2(x,y)$
\begin{align*}
w_1
&=
\{(x,y)\,|\, H_1(x,y)=H_2(x,y)\}
\\
&=
\left\{(x,y)\,\left|\, 
\frac35x+\frac45y-1=\frac45x-\frac35y+1\right.\right\}
\\
&=
\left\{(x,y)\,\left|\,
-\frac15x+\frac75y-2=0
\right.\right\}
\\
w_2&=\{(x,y)\,|\, H_1(x,y)=-H_2(x,y)\}
\\
&=
\left\{(x,y)\,\left|\, 
\frac35x+\frac45y-1=-\frac45x+\frac35y-1\right.\right\}
\\
&=
\left\{(x,y)\,\left|\,
\frac75x+\frac15y=0
\right.\right\}
\end{align*}
Die gesuchten Geraden haben also die Gleichungen
\[
-x+7y-10=0
\qquad
\text{und}
\qquad
7x+y=0
\]
Alternativ kann man die gesuchten Winkelhalbierenden auch Hilfe
vektorgeometrischer "Uberlegungen wie folgt finden.
Zunächst braucht man den Schnittpunkt der beiden gegebenen Geraden.
Diesen kann man zum Beispiel mit Hilfe der Kramerschen Regel
finden
\begin{align*}
x&=\frac{\left|\,\begin{matrix}5&4\\-5&-3\end{matrix}\,\right|}{\left|\,\begin{matrix}3&4\\4&-3\end{matrix}\,\right|}
=\frac{-15+20}{-9-16}=\frac{5}{-25}=-\frac15\\
y&=\frac{\left|\,\begin{matrix}3&5\\4&-5\end{matrix}\,\right|}{\left|\,\begin{matrix}3&4\\4&-3\end{matrix}\,\right|}
=\frac{-15-20}{-25}=\frac{35}{25}=\frac75
\end{align*}
Nun braucht man nur noch die Normalen der beiden Geraden. Bekannt sind die Normalen
\[
\vec n_1=\begin{pmatrix}3\\4\end{pmatrix}
\qquad
\text{und}
\qquad
\vec n_2=\begin{pmatrix}4\\-3\end{pmatrix}
\]
der gegebenen Geraden. Da diese beiden Vektoren die gleiche Länge haben,
sind die gesuchten Normalen der Winkelhalbierenden
\[
\vec m_{\pm}=\vec n_1\pm \vec n_2=\begin{cases}
\begin{pmatrix}7\\1\end{pmatrix}\\
\begin{pmatrix}-1\\7\end{pmatrix}
\end{cases}.
\]
Die Gleichung einer Gearden mit Normale $\vec n$ durch den Punkt $\vec q$ ist
\begin{align*}
\vec n\cdot(\vec p-\vec q)&=0,&\vec p&=\begin{pmatrix}x\\y\end{pmatrix}
\\
\vec n\cdot p&=\vec n\cdot\vec q
\end{align*}
Da wir Normale und Schnittpunkt kennen, können wir jetzt auch die Gleichung
aufstellen:
\begin{align*}
w_1:\qquad \vec m_+\cdot\vec p&=\vec m_+\cdot\begin{pmatrix}\frac15\\\frac75\end{pmatrix}&
   7x+y&=-\frac75+\frac{7}5=0
\\
w_2:\qquad\vec m_-\cdot\vec p&=\vec m_-\cdot\begin{pmatrix}\frac15\\\frac75\end{pmatrix}&
   -x+7y&=\frac15+\frac{49}5=10,
\end{align*}
also dieselben Gleichungen wie in der Lösung mit der Hesseschen Normalform.

Da die
beiden Winkelhalbierenden senkrecht aufeinander stehen, können die beiden
Vektoren $\vec m_{\pm}$ auch als Richtungsvektoren der gesuchten Winkelhalbierenden
betrachtet werden. Damit kann man die beiden Winkelhalbierenden auch in
Parameterdarstellung bekommen:
\begin{align*}
w_1&=\left\{
\left.
\begin{pmatrix}-\frac15\\\frac75\end{pmatrix}+t\begin{pmatrix}7\\1\end{pmatrix}
\,
\right|\,t\in\mathbb R
\right\}\qquad\text{und}
\\
w_2&=\left\{
\left.
\begin{pmatrix}-\frac15\\\frac75\end{pmatrix}+t\begin{pmatrix}-1\\7\end{pmatrix}
\,
\right|\,t\in\mathbb R
\right\}.
\qedhere
\end{align*}
\end{loesung}

