Wo durchstösst die Gerade durch den Punkt $(0,0,1)$ mit Richtungsvektor
\[
\begin{pmatrix}1\\1\\-1\end{pmatrix}
\]
die Ebene durch die Punkte $O$, $(1,0,1)$ und $(0,1,1)$?

\thema{Durchstosspunkt}
\thema{Ebene}

\begin{loesung}
Die Ortsvektoren von Punkten in der Ebene sind
\[
t
\begin{pmatrix}1\\0\\1\end{pmatrix}
+
s
\begin{pmatrix}0\\1\\1\end{pmatrix}
\]
Punkte auf der Geraden haben die Form
\[
\begin{pmatrix}0\\0\\1\end{pmatrix}+u\begin{pmatrix}1\\1\\-1\end{pmatrix}.
\]
Der Durchstosspunkt kann also als Lösung der Gleichung
\begin{align*}
t\begin{pmatrix}1\\0\\1\end{pmatrix}
+
s\begin{pmatrix}0\\1\\1\end{pmatrix}
&=
\begin{pmatrix}0\\0\\1\end{pmatrix}+u\begin{pmatrix}1\\1\\-1\end{pmatrix}
\\
t\begin{pmatrix}1\\0\\1\end{pmatrix}
+
s\begin{pmatrix}0\\1\\1\end{pmatrix}
-u\begin{pmatrix}1\\1\\-1\end{pmatrix}
&=
\begin{pmatrix}0\\0\\1\end{pmatrix}
\end{align*}
gefunden werden.
Man muss also das folgende Gleichungssystem lösen:
\begin{align*}
\begin{tabular}{|ccc|c|}
\hline
1&0&-1&0\\
0&1&-1&0\\
1&1&1&1\\
\hline
\end{tabular}
&\rightarrow
\begin{tabular}{|ccc|c|}
\hline
1&0&-1&0\\
0&1&-1&0\\
0&1&2&1\\
\hline
\end{tabular}
\\
&\rightarrow
\begin{tabular}{|ccc|c|}
\hline
1&0&-1&0\\
0&1&-1&0\\
0&0&3&1\\
\hline
\end{tabular}
\\
&\rightarrow
\begin{tabular}{|ccc|c|}
\hline
1&0&0&$\frac13$\\
0&1&0&$\frac13$\\
0&0&1&$\frac13$\\
\hline
\end{tabular}
\end{align*}
Somit ist $u=\frac13$, also
\[
\begin{pmatrix}0\\0\\1\end{pmatrix}+\frac13\begin{pmatrix}1\\1\\-1\end{pmatrix}
=
\begin{pmatrix}\frac13\\\frac13\\\frac23\end{pmatrix}
\]
der Ortsvektor des Durchstosspunktes.

Noch etwas ökonomischer ist das folgende Verfahren, welches auch in der
Vorlesung besprochen wurde. Die Variablen werden {\color{red}} markiert.
Die Ebenengleichung liefert Ortsvektoren von Punkten auf der Ebene
\[
\begin{pmatrix}
{\color{red}x}\\
{\color{red}y}\\
{\color{red}z}
\end{pmatrix}
=
{\color{red}t}
\begin{pmatrix}1\\0\\1\end{pmatrix}
+
{\color{red}s}
\begin{pmatrix}0\\1\\1\end{pmatrix}.
\]
Die Geradengleichung entsprechend
\[
\begin{pmatrix}
{\color{red}x}\\
{\color{red}y}\\
{\color{red}z}
\end{pmatrix}
=
\begin{pmatrix}0\\0\\1\end{pmatrix}+{\color{red}}{\color{red}u}\begin{pmatrix}1\\1\\-1\end{pmatrix}.
\]
Wir haben also insgesamt sechs Gleichungen mit sechs Unbekannten.
Das Gauss-Tableau dieses Gleichungssystems ist
\[
\begin{tabular}{|>{$}c<{$}>{$}c<{$}>{$}c<{$}>{$}c<{$}>{$}c<{$}>{$}c<{$}|>{$}c<{$}|}
\hline
{\color{red}x}&{\color{red}y}&{\color{red}z}&{\color{red}t}&{\color{red}s}&{\color{red}u}&\\
\hline
1&0&0&-1& 0& 0& 0\\
0&1&0& 0&-1& 0& 0\\
0&0&1&-1&-1& 0& 0\\
1&0&0& 0& 0&-1& 0\\
0&1&0& 0& 0&-1& 0\\
0&0&1& 0& 0& 1& 1\\
\hline
\end{tabular}
\rightarrow
\begin{tabular}{|>{$}c<{$}>{$}c<{$}>{$}c<{$}>{$}c<{$}>{$}c<{$}>{$}c<{$}|>{$}c<{$}|}
\hline
{\color{red}x}&{\color{red}y}&{\color{red}z}&{\color{red}t}&{\color{red}s}&{\color{red}u}&\\
\hline
1&0&0&0&0&0&\frac13\\
0&1&0&0&0&0&\frac13\\
0&0&1&0&0&0&\frac23\\
0&0&0&1&0&0&\frac13\\
0&0&0&0&1&0&\frac13\\
0&0&0&0&0&1&\frac13\\
\hline
\end{tabular}
\]
Daraus liest man ab, dass $(\frac13,\frac13,\frac23)$ der Schnittpunkt ist,
und dass er auf der Ebene mit den Parameterwerten $t=\frac13$  und $s=\frac13$
und auf der Geraden mit dem Parameterwert $u=\frac13$ erreicht wird.
\end{loesung}

