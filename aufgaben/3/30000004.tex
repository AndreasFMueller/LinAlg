Finden Sie eine Parameterdarstellung der Schnittgeraden der Ebenen
durch die Punkte $A=(6,4,7)$, $B=(9,2,9)$ und $C=(1,7,0)$ bzw.~$P=(2,2,4)$,
$Q=(6,13,4)$, und $R=(1,3,7)$.

\thema{Schnittgerade}

\begin{loesung}
Wir benötigen zunächst Parameterdarstellungen für die Ebenen:
\[
\begin{pmatrix}6\\4\\7\end{pmatrix}
+t
\begin{pmatrix}3\\-2\\2\end{pmatrix}
+s
\begin{pmatrix}-5\\3\\-7\end{pmatrix}
,\qquad
\begin{pmatrix}2\\2\\4\end{pmatrix}
+u
\begin{pmatrix}4\\11\\0\end{pmatrix}
+v
\begin{pmatrix}-1\\1\\3\end{pmatrix}
\]
Gesucht werden jetzt Werte für $t$, $s$, $u$ und $v$, für die
die beidden Ausdrücke übereinstimmen, also
\begin{align*}
\begin{pmatrix}6\\4\\7\end{pmatrix}
+t
\begin{pmatrix}3\\-2\\2\end{pmatrix}
+s
\begin{pmatrix}-5\\3\\-7\end{pmatrix}
&=
\begin{pmatrix}2\\2\\4\end{pmatrix}
+u
\begin{pmatrix}4\\11\\0\end{pmatrix}
+v
\begin{pmatrix}-1\\1\\3\end{pmatrix}
\\
t
\begin{pmatrix}3\\-2\\2\end{pmatrix}
+s
\begin{pmatrix}-5\\3\\-7\end{pmatrix}
+u
\begin{pmatrix}-4\\-11\\0\end{pmatrix}
+v
\begin{pmatrix}1\\-1\\-3\end{pmatrix}
&=
\begin{pmatrix}2\\2\\4\end{pmatrix}
-
\begin{pmatrix}6\\4\\7\end{pmatrix}
=
\begin{pmatrix}-4\\-2\\-3\end{pmatrix}
\end{align*}
Es ist also das lineare Gleichungssystem
\begin{center}
\begin{tabular}{|cccc|c|}
\hline
$ 3$&$-5$&$ -4$&$ 1$&$-4$\\
$-2$&$ 3$&$-11$&$-1$&$-2$\\
$ 2$&$-7$&$  0$&$-3$&$-3$\\
\hline
\end{tabular}
\end{center}
Dieses Gleichungssystem löst man am einfachsten mit Octave, und findet
\verbatiminput{1.m}
also
\begin{align*}
t&=-\frac13-2v\\
s&=\frac13-v\\
u&=\frac13
\end{align*}
Setzt man die Beziehung für $u$ in die Parameterdarstellung der
zweiten Ebene  ein, findet man als Parameterdarstellung für die
Schnittgerade:
\[
\left\{\left.
\begin{pmatrix}\frac{10}{3}\\\frac{17}{3}\\4\end{pmatrix}
+v
\begin{pmatrix}-1\\1\\3\end{pmatrix}
\right|
v\in\mathbb R
\right\}
\]

Alternativ kann man das Problem auch lösen, indem man die beiden
Ebenen als sechs Gleichungen mit sieben Unbekannten $x,y,z,u,v,s,t$
schreibt,
\begin{align*}
\begin{pmatrix}
{\color{red}x}\\
{\color{red}y}\\
{\color{red}z}
\end{pmatrix}=
\begin{pmatrix}6\\4\\7\end{pmatrix}
+{\color{red}t}
\begin{pmatrix}3\\-2\\2\end{pmatrix}
+{\color{red}s}
\begin{pmatrix}-5\\3\\-7\end{pmatrix}
\\
\begin{pmatrix}
{\color{red}x}\\
{\color{red}y}\\
{\color{red}z}
\end{pmatrix}=
\begin{pmatrix}2\\2\\4\end{pmatrix}
+{\color{red}u}
\begin{pmatrix}4\\11\\0\end{pmatrix}
+{\color{red}v}
\begin{pmatrix}-1\\1\\3\end{pmatrix}
\end{align*}
Das zugehörige Gauss-Tableau ist:
\begin{center}
\begin{tabular}{|>{$}c<{$}>{$}c<{$}>{$}c<{$}>{$}c<{$}>{$}c<{$}>{$}c<{$}>{$}c<{$}|>{$}c<{$}|}
\hline
{\color{red} x} &  {\color{red}y} &  {\color{red}z} &  {\color{red}t} &  {\color{red}s} &  {\color{red}u} &  {\color{red}v} &\\
\hline
1&0&0& -3 &  5 &   0 &  0 &6\\
0&1&0&  2 & -3 &   0 &  0 &4\\
0&0&1& -2 &  7 &   0 &  0 &7\\
1&0&0&  0 &  0 &  -4 &  1 &2\\
0&1&0&  0 &  0 & -11 & -1 &2\\
0&0&1&  0 &  0 &   0 & -3 &4\\
\hline
\end{tabular}
\end{center}
Wendet man den Gauss-Algorithmus darauf an, ergibt sich
\begin{center}
\begin{tabular}{|ccccccc|c|}
\hline
$x$& $y$& $z$& $t$& $s$& $u$& $v$&\\
\hline
1&0&0&0&0&0&$ 1$&$\frac{10}{3}$\\
0&1&0&0&0&0&$-1$&$\frac{17}{3}$\\
0&0&1&0&0&0&$-3$&$4$\\
0&0&0&1&0&0&$ 2$&$-\frac{1}{3}$\\
0&0&0&0&1&0&$ 1$&$ \frac{1}{3}$\\
0&0&0&0&0&1&$ 0$&$ \frac{1}{3}$\\
\hline
\end{tabular}
\end{center}
Aus den ersten drei Zeilen kann man wieder die Lösung ablesen.

Die unteren drei Zeilen braucht man
nicht, sie geben die Abhängigkeit der Variablen $t$, $s$ und $u$
von $v$ wieder, welche zwar auch mit denjenigen im ersten Lösungsweg
übereinstimmen, aber bei diesem zweiten Lösungsweg nicht
explizit benötig werden. In der ersten Lösung waren diese
Abhängigkeiten nur benötigt werden, um sie später in die
Ebenengleichungen einsetzen zu können. In der zweiten Lösung
hat das der Gauss-Algorithmus bereits übernommen.
\end{loesung}

