Bestimmen Sie die gegenseitige Lage der Geraden
\[
\begin{pmatrix}0\\2\\1\end{pmatrix}
+t
\begin{pmatrix}2.5\\5\\-3\end{pmatrix}
\qquad
\text{und}
\qquad
\begin{pmatrix}2\\6\\1\end{pmatrix}
+t
\begin{pmatrix}-5\\-10\\6\end{pmatrix}
\]
und ihren Abstand.

\thema{Abstand windschiefer Geraden}

\begin{loesung}
Die beiden Richtungsvektoren sind parallel.
Wir schreiben für den gemeinsamen Richtungsvektor
\[
\vec v=
\begin{pmatrix}-5\\-10\\6\end{pmatrix},
\]
um Brüche zu vermeiden.

Zur Bestimmung des
Abstands kann das Vektorprodukt verwendet werden, dazu braucht man
die Differenz zwischen zwei Punkten auf der Geraden:
\[
\vec d=
\begin{pmatrix}0\\2\\1\end{pmatrix}
-\begin{pmatrix}2\\6\\1\end{pmatrix}
=
\begin{pmatrix}-2\\-4\\0\end{pmatrix}
\]
Das Vektorprodukt ist
\[
\vec v\times \vec d
=
\begin{pmatrix}-5\\-10\\6\end{pmatrix}
\times
\begin{pmatrix}-2\\-4\\0\end{pmatrix}
=
\begin{pmatrix}24\\-12\\0\end{pmatrix}
\]
Damit wird der Abstand
\begin{align*}
d
=
\frac{|\vec v\times \vec d|}{|\vec v|}
=\frac{\sqrt{24^2+12^2}}{\sqrt{5^2+10^2+6^2}}
=\frac{\sqrt{720}}{\sqrt{161}}
\simeq 2.114722
\qedhere
\end{align*}
\end{loesung}

