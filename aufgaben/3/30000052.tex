Gegeben ist die Ebene $\sigma_1$ mit der Gleichung
\begin{equation}
x+3y-2z=2
\label{30000052:ebenengleichung}
\end{equation}
sowie die Ebene $\sigma_2$ durch die Punkte $A=(6,0,0)$, $B=(0,2,0)$ und
$C=(0,0,-3)$.
\begin{teilaufgaben}
\item
Finden Sie die Normale der Ebene $\sigma_2$.
\item 
Finden Sie die Schnittgerade der beiden Ebenen.
\item 
Berechnen Sie den Abstand des Punktes $A$ von der Ebene $\sigma_1$.
\end{teilaufgaben}

\begin{loesung}
\begin{teilaufgaben}
\item
Die Normale kann man mit Hilfe des Vektorproduktes bestimmt werden
\[
\vec{n}_2
=
\overrightarrow{AB}\times\overrightarrow{AC}
=
\begin{pmatrix}
-6\\
 2\\
 0
\end{pmatrix}
\times
\begin{pmatrix}
-6\\
 0\\
-3
\end{pmatrix}
=
\begin{pmatrix}
-6\\
-18\\
12
\end{pmatrix}
=
-6\cdot\begin{pmatrix}
1\\
3\\
-2
\end{pmatrix}.
\]
\item
Die Normale $\vec{n}_2$ der Ebene $\sigma_2$ ist ein Vielfaches der
Normalen
\[
\vec{n}_1=\begin{pmatrix}1\\3\\-2\end{pmatrix}
\]
der Ebene $\sigma_1$, die Ebenen sind daher parallel.
Die Ebenen schneiden sich, wenn einer der Punkte der Ebene $\sigma_2$ auch
auf der Ebene $\sigma_1$ liegt. 
Setzt man allerdings $A$ in die Gleichung von $\sigma_1$ ein, erh"alt man
\[
6+3\cdot 0-2\cdot 0 \ne 2.
\]
Somit schneiden sich die Ebenen nicht.
\item
Der Abstand kann mit der Hesseschen Normalform berechnet werden.
Dazu muss die Ebenengleichung~\eqref{30000052:ebenengleichung}
so dividiert werden, dass die Koeffizienten der Koordinaten einen
Einheitsvektor bilden, also durch
\[
\sqrt{1+3^2+2^2}=\sqrt{14}.
\]
Dann ist der gesuchte Abstand:
\[
l
=
\frac1{\sqrt{14}}\cdot 6
+\frac3{\sqrt{14}}\cdot 0
-\frac2{\sqrt{14}}\cdot 0
-\frac2{\sqrt{14}}
=
\frac4{\sqrt{14}}.
\qedhere
\]
\end{teilaufgaben}
\end{loesung}

\begin{bewertung}
Normale mit Vektorprodukt ({\bf N}) 1 Punkt,
Berechnung des Vektorproduktes ({\bf V}) 1 Punkt,
Erkenntnis, dass die Ebenen die gleiche Normale haben und damit
parallel sind ({\bf P}) 1 Punkt,
Erkenntnis, dass die Ebenen keinen gemeinsamen Punkt haben,
also disjunkt sind ({\bf D}) 1 Punkt,
Hessesche Normalform ({\bf H}) 1 Punkt,
korrekter Abstand ({\bf A}) 1 Punkt.
\end{bewertung}


