Stellen Sie das Tableau auf, mit dem die Schnittmenge der beiden Geraden
\[
\vec{p}
=
\begin{pmatrix}1\\9\end{pmatrix}
+
t
\begin{pmatrix}2\\1\end{pmatrix}
\quad
\text{und}
\quad
\vec{p}
=
\begin{pmatrix*}[r]4\\-1\end{pmatrix*}
+
t
\begin{pmatrix*}[r]-7\\1\end{pmatrix*}
\]
gefunden werden kann.

\begin{loesung}
\[
\renewcommand{\arraystretch}{1.2}
\begin{tabular}{|>{$}r<{$} >{$}r<{$} >{$}r<{$} >{$}r<{$}|>{$}r<{$}|}
\hline
x_1&x_2&t_1&t_2&1\\
\hline
1&0&-9& 0& 2\\
0&1&-1& 0& 1\\
1&0& 0& 7& 4\\
0&1& 0&-1&-1\\
\hline
\end{tabular}
\quad\rightarrow\quad
\begin{tabular}{|>{$}r<{$} >{$}r<{$} >{$}r<{$} >{$}r<{$}|>{$}r<{$}|}
\hline
x_1&x_2&t_1&t_2&1\\
\hline
1&0& 0& 0&-\frac{19}{4}\\
0&1& 0& 0& \frac{1}{4}\\
0&0& 0& 0&-\frac{3}{4}\\
0&0& 0& 1& \frac{5}{4}\\
\hline
\end{tabular}
\]
woraus man den Schnittpunkt $(\frac{-19}{4},\frac{1}{4})$ ablesen kann,
der mit $t_1=-\frac{3}{4}$ bzw.~$t_2=\frac{5}{4}$ erreicht wird.
\end{loesung}
