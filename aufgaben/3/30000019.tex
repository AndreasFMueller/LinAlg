Gegeben sind die Vektoren
\[
\vec u = \begin{pmatrix}\frac12\\\frac{\sqrt{3}}2\end{pmatrix},
\qquad
\vec v = \begin{pmatrix}\frac{\sqrt{3}}2\\-\frac12\end{pmatrix}
\]
\begin{teilaufgaben}
\item
Finden Sie eine Matrix $A$, die die Standardbasisvektoren $\vec e_1$ und
$\vec e_2$ auf $\vec u$ bzw.~$\vec v$ abbildet.
\item
Finden Sie eine Matrix $B$, die die Vektoren $\vec u$ und $\vec v$ auf
die Standardbasisvektoren $\vec e_1$ und $\vec e_2$ abbildet.
\end{teilaufgaben}

\begin{hinweis}
b) \gaussurl{gausscalc:30000019}
\end{hinweis}

\thema{Abbildungsmatrix}

\begin{loesung}
\begin{teilaufgaben}
\item Die Spalten von $A$ sind die Bilder der Standardbasisvektoren
\[
A=\begin{pmatrix}
\frac12         &\frac{\sqrt{3}}2\\
\frac{\sqrt{3}}2&-\frac12
\end{pmatrix} 
\]
\item
Die Matrix $A$ bildet die Standardbasisvektoren auf $\vec u$ und $\vec v$
ab, die gesuchte Matrix $B$ soll das Umgekehrte machen.
Die inverse Matrix macht dies, also ist 
\[
B=A^{-1} = \dfrac{4}{-1-3}\begin{pmatrix}
-\frac12         &-\frac{\sqrt{3}}2\\
-\frac{\sqrt{3}}2&\frac12
\end{pmatrix} 
= \begin{pmatrix}
\frac12         &\frac{\sqrt{3}}2\\
\frac{\sqrt{3}}2&-\frac12
\end{pmatrix} .
\]
Die Inverse wurde hier mit Hilfe der Minoren berechnet.
\qedhere
\end{teilaufgaben}
\end{loesung}



