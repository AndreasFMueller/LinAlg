RGB-Farben bilden einen Vektorraum, Farbtransformationen können daher 
mit Matrizen durchgeführt werden. 
\definecolor{farbe1}{rgb}{0.6,0.4,0.8}%
\definecolor{farbe2}{rgb}{0.83,0.48,0.48}%
Wenden Sie die Matrix
\[
\begin{pmatrix*}[r]
   0.33& -0.24&  0.91\\
   0.91&  0.33& -0.24\\
  -0.24&  0.91&  0.33
\end{pmatrix*}
\quad
\text{auf die Farbe}
\quad
{\color{farbe1}
f_1=\begin{pmatrix} 153\\102\\204\end{pmatrix}
}
\qquad
\qrcode[height=1.6cm]{https://www.w3schools.com/colors/colors_rgb.asp}
\quad
\qrcode[height=1.6cm]{https://www.visibone.com/colorlab/}
\]
an.
\begin{teilaufgaben}
\item Verwenden Sie den RGB-Rechner aus dem QR-Code, um die neue Farbe
${\color{farbe2}f_2}=D{\color{farbe1}f_1}$
zu bestimmen.
\item $D$ ist ungefähr eine Drehung um $90^\circ$, suchen Sie eine
ähnliche Farbe in \url{https://www.visibone.com/colorlab/} (2.~QR-Code)
und überzeugen Sie sich, dass sie etwa $90^\circ$ verdreht auf dem
Farbkreis liegt.
\end{teilaufgaben}

\begin{loesung}
Die Anwendung der Matrix ergibt:
\[
D
f_1
=
D\begin{pmatrix} 153\\102\\204\end{pmatrix}
=
{\color{farbe2}
\begin{pmatrix}
  211.65\\
  123.93\\
  123.42
\end{pmatrix}
\approx
\begin{pmatrix}
  212\\
  124\\
  123
\end{pmatrix}
}
\]
Die gefundene Farbe ist nahe bei der Farbe \texttt{\#cc6666}, die auf
Visibone \emph{light dull red} genannt wird, und die von der
Farbe \texttt{\#9966cc}, genannt \emph{light dull violet}, etwa
$90^\circ$ im Gegenuhrzeigersinn entfernt ist.
\end{loesung}
