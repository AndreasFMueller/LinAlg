Um die Position eines Objekts im Raum zu bestimmen, werden zwei Kameras
verwendet.
Die Software erkennt das Objekt auf beiden Bildern und berechnet
die Gerade von der Kamera durch das Objekt.
Die Position des Objektes ist also der Schnittpunkt der Geraden.

Wegen Ungenauigkeiten in der Kamerakalibrierung schneiden sich die Geraden
nicht. 
Um den Fehler abzuschätzen, sollen sie den Abstand der beiden Geraden
\[
\begin{pmatrix} 100\\0\\0 \end{pmatrix}
+
t\begin{pmatrix} 1\\7\\ 101\end{pmatrix}
\qquad\text{und}\qquad
\begin{pmatrix} 0\\0\\ 100 \end{pmatrix}
+
t\begin{pmatrix} 99\\7\\1 \end{pmatrix}
\]
berechnen.

\thema{Abstand windschiefer Geraden}
\thema{Vektorprodukt}

\begin{loesung}
Dazu kann man die Formel für den windschiefen Abstand verwenden.
Das Vektorprodukt der beiden Richtungsvektoren ist
\[
\vec n
=
\vec r_1
\times
\vec r_2
=
\begin{pmatrix} 1\\7\\ 101\end{pmatrix}
\times
\begin{pmatrix} 99\\7\\1 \end{pmatrix}
=
\begin{pmatrix} -700\\9998\\-686 \end{pmatrix}
\]
Die Differenz der Stützvektoren
\[
\vec p_1-\vec p_2
=
\begin{pmatrix} 100\\0\\0 \end{pmatrix}
-
\begin{pmatrix} 0\\0\\ 100 \end{pmatrix}
=
\begin{pmatrix} 100\\0\\-100 \end{pmatrix}
\]
muss jetzt darauf projiziert werden, es ist
\[
d
=
\frac{\vec n}{|\vec n|}\cdot (\vec p_1-\vec p_2)
=
\frac{1}{|\vec n|}(-1400)=-0.13936.
\]
Der Abstand der beiden Geraden ist also $0.13936$.
\end{loesung}

\begin{bewertung}
Abstandsformel ({\bf F}) 1 Punkt,
Vektorprodukt ({\bf P}) 1 Punkt,
Länge des Vektorproduktes ({\bf L}) 1 Punkt,
Differenz der Stützpunkte ({\bf D}) 1 Punkt,
Abstand ({\bf A}) 2 Punkte.
\end{bewertung}

