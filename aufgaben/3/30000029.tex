Bestimmen Sie die gegenseitige Lage der Geraden
\[
p_1+tr_1
=
\begin{pmatrix} 3\\ 0\\ 0 \end{pmatrix}
+t
\begin{pmatrix} 2\\ 2\\ 3 \end{pmatrix}
\qquad
\text{und}
\qquad
p_2+tr_2
=
\begin{pmatrix} 4\\ 4\\ 0 \end{pmatrix}
+t
\begin{pmatrix} 1\\ 4\\ 3 \end{pmatrix}
\]
und ihren Abstand.

\thema{Abstand windschiefer Geraden}

\begin{loesung}
Die Richtungsvektoren sind nicht linear abhängig, also sind
die Geraden vermutlich windschief. Wir berechnen ihren Abstand mit
Hilfe der Formel für den Abstand zweier Geraden. Dazu brauchen
wir zunächste die Normale auf der von den beiden Richtungsvektoren
aufgespannte Ebene:
\begin{align*}
n&=r_1\times r_2 =\begin{pmatrix} 2\\ 2\\ 3 \end{pmatrix}
\times
\begin{pmatrix} 1\\ 4\\ 3 \end{pmatrix}
=
\begin{pmatrix} -6\\-3\\6 \end{pmatrix}
=3
\begin{pmatrix} -2\\-1\\2 \end{pmatrix}
\\
n^0&=\frac{n}{|n|}=
\frac13
\begin{pmatrix} -2\\-1\\2 \end{pmatrix}
\end{align*}
Der vorzeichenbehaftete Abstand ist dann
\[
d=n^0\cdot(p_1-p_2)=
\frac13
\begin{pmatrix} -2\\-1\\2 \end{pmatrix}
\cdot
\begin{pmatrix} -1\\-4\\0\end{pmatrix}
=\frac13(2+4)=2.
\]
Die Geraden sind also tatsächlich windschief, und
ihr Abstand ist $2$.
\end{loesung}

