Orthonormalisieren Sie die folgenden Vektoren
\[
\vec a_1=\begin{pmatrix}1\\1\\1\end{pmatrix},\quad
\vec a_2=\begin{pmatrix}0\\1\\1\end{pmatrix},\quad
\vec a_3=\begin{pmatrix}0\\0\\1\end{pmatrix}
\]

\thema{Orthonormalisierung}

\begin{loesung}
Wir verwenden die Formeln des Gram-Schmidtschen Orthogonalisierungsverfahrens:
\begin{align*}
\vec b_1&=\frac{\vec a_1}{|\vec a_1|}=\frac1{\sqrt{3}}\begin{pmatrix}1\\1\\1\end{pmatrix}
\\
\vec b_2&=\frac{\vec a_2-(\vec a_2\cdot\vec b_1)\vec b_1}{\dots}
=
\frac{\begin{pmatrix}0\\1\\1\end{pmatrix}-\frac{2}{\sqrt{3}}\cdot\frac1{\sqrt{3}}\begin{pmatrix}1\\1\\1\end{pmatrix}}{\dots}
=
\frac1{\sqrt{6}}\begin{pmatrix}-2\\1\\1\end{pmatrix}
\\
\vec b_3
&=
\frac{\vec a_3-(\vec a_3\cdot\vec b_1)\vec b_1-(\vec a_3\cdot\vec b_2)\vec b_2}{\dots}
=
\frac{\begin{pmatrix}0\\0\\1\end{pmatrix}
-\frac1{\sqrt{3}}\cdot\frac1{\sqrt{3}}\begin{pmatrix}1\\1\\1\end{pmatrix}
-\frac1{\sqrt{6}}\cdot\frac1{\sqrt{6}}\begin{pmatrix}-2\\1\\1\end{pmatrix}}{\dots}\\
&=
\frac{
\begin{pmatrix}0\\0\\1\end{pmatrix}
-\frac13\begin{pmatrix}1\\1\\1\end{pmatrix}
-\frac16\begin{pmatrix}-2\\1\\1\end{pmatrix}}{\dots}
=
\frac1{\sqrt{2}}\begin{pmatrix}0\\-1\\1\end{pmatrix}
\qedhere
\end{align*}
\end{loesung}

