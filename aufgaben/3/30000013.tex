Stellen Sie die Gleichung der Tagenten des Kreises mit der Gleichung
\[
(x-2)^2+(y-5)^2=16
\]
auf, die parallel ist zur Geraden mit der Gleichung
\[
4x-3y+1=0
\]

\begin{loesung}
Es gibt zwei solche Tangenten. Da die Tangente den gleichen Normalenvektor
hat wie die vorgegebene Gerade, hat sie bis auf den konstanten Term
auch die gleiche Hessesche Normalform. Der Einheitsnormalenvektor
ist
\[
\frac15\begin{pmatrix}4\\-3\end{pmatrix}.
\]
Die Gleichung einer solchen Geraden durch den Mittelpunkt hat
die Form $\frac45x-\frac35y+e=0$, wobei $e$ so gew"ahlt werden
muss, dass der Mittelpunkt $(2,5)$ des Kreises die Gleichung
erf"ullt, also
\[
-e=\frac85-\frac{15}5=-\frac75.
\]
Da der Radius des Kreises $4$ ist, sind die gesuchten Geraden gegeben
durch die Gleichungen
\[
\begin{linsys}{3}
\frac45x&-&\frac35y&+&\frac75&=&4\\
\frac45x&-&\frac35y&+&\frac75&=&-4
\end{linsys}
\]
oder
\[
\begin{linsys}{3}
4x&-&3y&=&13\\
4x&-&3y&=&-27
\end{linsys}
\]
\end{loesung}
