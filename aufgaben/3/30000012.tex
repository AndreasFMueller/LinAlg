Bestimmen Sie die Schnittpunkte des Kreises um den Punkt $M=(2,3)$
mit Radius $5$ mit der Geraden, die durch die Punkte $(1,-4)$
und $(5,-1)$ geht.

\begin{loesung}
Der Richtungsvektor der Geraden ist $\vec r=\begin{pmatrix}4\\3\end{pmatrix}$,
die Parameterdarstellung der Geraden ist also
\[
\begin{pmatrix}x\\y\end{pmatrix}
=
\begin{pmatrix}1\\-4\end{pmatrix}
+t
\begin{pmatrix}4\\3\end{pmatrix}
=
\begin{pmatrix}4t+1\\3t-4\end{pmatrix}
\]
Setzt man dies in die Kreisgleichung
\begin{align*}
(\vec p-\vec m)\cdot(\vec p-\vec m)&=r^2\\
\left(
\begin{pmatrix}x\\y\end{pmatrix}
-
\begin{pmatrix}2\\3\end{pmatrix}
\right)
\cdot
\left(
\begin{pmatrix}x\\y\end{pmatrix}
-
\begin{pmatrix}2\\3\end{pmatrix}
\right)
&=r^2
\\
(x-2)^2+(y-3)^2&=25
\end{align*}
ein, erhält man die quadratische Gleichung
\begin{align*}
(4t-1)^2+(3t-7)^2&=25
\\
25t^2-50t+25&=0
\\
t^2-2t+1&=0
\\
(t-1)^2&=0
\\
\Rightarrow\qquad t&=1
\end{align*}
Somit ist der Schnittpunkt gerade $(5,-1)$. Durch Einsetzen kann
man sich überzeugen, dass dieser Punkt $5$ Einheiten vom Mittelpunkt
entfernt ist. Da er die einzige Lösung ist, muss es sich auch um
einen Berührpunkt handeln.
\end{loesung}

