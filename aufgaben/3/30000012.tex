Kann der Vektor
\[
v = \begin{pmatrix*}[r]
2\\-5\\3
\end{pmatrix*}
\]
als Linearkombination der Vektoren
\[
u_1=\begin{pmatrix*}[r]
1\\-3\\2
\end{pmatrix*},
\qquad
u_2=\begin{pmatrix*}[r]
2\\-4\\-1
\end{pmatrix*},
\qquad\text{und}\qquad
u_3=\begin{pmatrix*}[r]
1\\-5\\7
\end{pmatrix*}
\]
geschrieben werden? Wenn ja, finden Sie die Koeffizienten, wenn nein,
warum nicht? Bilden die Vektoren $u_i$ eine Basis?

\begin{hinweis}
\gaussurl{gausscalc:30000012}
\end{hinweis}

\thema{Basis}
\thema{Basiszerlegung}

\begin{loesung}
Wenn die gesuchte Darstellung möglich ist, dann gibt es Zahlen
$x_1$, $x_2$ und $x_3$ mit der Eigenschaft, dass
\[
v =x_1u_1+x_2u_2+x_3u_3.
\]
Diese Zahlen kann man finden als Lösungen des Gleichungssystems
\[
\begin{pmatrix*}[r]
1 & 2 & 1\\
-3 & -4 & -5\\
2 & -1 & 7
\end{pmatrix*}
\begin{pmatrix}
x_1\\x_2\\x_3
\end{pmatrix}
=
\begin{pmatrix}
2\\-5\\3
\end{pmatrix}.
\]
Mit dem Gauss-Algorithmus ergibt sich:
\begin{align*}
\begin{tabular}{|rrr|r|}
\hline
1&2&1&2\\
$-3$&$-4$&$-5$&$-5$\\
2&$-1$&7&3\\
\hline
\end{tabular}
&\rightarrow
\begin{tabular}{|rrr|r|}
\hline
1&2&1&2\\
0&2&$-2$&1\\
0&$-5$&5&$-1$\\
\hline
\end{tabular}
\\
&\rightarrow
\begin{tabular}{|rrr|r|}
\hline
1&2&1&2\\
0&1&$-1$&$\frac12$\\
0&0&0&$\frac32$\\
\hline
\end{tabular}
\end{align*}
Aus dieser Rechnung kann man ablesen, dass das Gleichungssystem singulär
ist. Insbesondere sind die drei Vektoren linear abhängig und spannen daher
nicht den ganzen Raum $\mathbb{R}^3$ auf, sondern nur eine Ebene im dreidimensionalen 
Raum. Die Vektoren $u_i$ bilden daher keine Basis. 
Ausserdem kann man aus der letzten Zeile sehen, die der
Gleichung $0=\frac32$ entspricht, dass das Gleichungssystem
keine Lösung hat. Die gesuchte Darstellung ist also nicht möglich, da sich der Vektor $v$
ausserhalb des von den Vektoren $u_i$ aufgespannten Raumes befindet.
\end{loesung}

