Betrachten Sie die Punkte
\begin{align*}
P_1&=(1,0),
&P_2&=\biggl(\frac{-1+\sqrt{5}}4, \frac{\sqrt{10+2\sqrt{5}}}4 \biggr),
&P_3&=\biggl(\frac{-1-\sqrt{5}}4, \frac{\sqrt{10-2\sqrt{5}}}4 \biggr),\\
&&P_5&=\biggl(\frac{-1+\sqrt{5}}4,-\frac{\sqrt{10+2\sqrt{5}}}4 \biggr),
&P_4&=\biggl(\frac{-1-\sqrt{5}}4,-\frac{\sqrt{10-2\sqrt{5}}}4 \biggr)
\end{align*}
\begin{teilaufgaben}
\item Wie weit vom Ursprung sind die Punkte $P_1$ und $P_2$ 
entfernt?
\item Sind die Winkel zwischen den Ortsvektoren von aufeinanderfolgenden
Punkten gleich, also $\angle P_1OP_2=\angle P_2OP_3=\angle P_3OP_4\dots$?
\item Was für eine Figur bilden die fünf Punkte?
\end{teilaufgaben}

\thema{Skalarprodukt}
\thema{Zwischenwinkel}

\begin{hinweis}
Die Punkte $P_4$ und $P_5$ entstehen aus $P_3$ bzw.~$P_2$ durch
Spiegelung an der $x$-Achse.
\end{hinweis}

\begin{loesung}
\begin{teilaufgaben}
\item Die Länge der Strecken $OP_2$ und $OP_3$ kann mit dem Satz von
Pythagoras berechnet werden:
\begin{align*}
|OP_2|&=
\biggl(\frac{-1+\sqrt{5}}4\biggr)^2+ \biggl(\frac{\sqrt{10+2\sqrt{5}}}4 \biggr)^2
\\
&=\frac{1-2\sqrt{5}+5}{16}+\frac{10+2\sqrt{5}}{16}=\frac{1-2\sqrt{5}+5+10+2\sqrt{5}}{16}=\frac{16}{16}=1\\
|OP_3|&=
\biggl(\frac{-1-\sqrt{5}}4\biggr)^2+\biggl( \frac{\sqrt{10-2\sqrt{5}}}4 \biggr)^2
\\
&=\frac{1+2\sqrt{5}+5}{16}+\frac{10-2\sqrt{5}}{16}=\frac{1+2\sqrt{5}+5+10-2\sqrt{5}}{16}=\frac{16}{16}=1,
\end{align*}
Beide Vektoren sind Einheitsvektoren.
\item Wenn die Skalarprodukte der Ortsvektoren gleich sind, dann sind auch
die Winkel gleich:
\begin{align*}
\overrightarrow{OP_1}\cdot\overrightarrow{OP_2}
&=
\frac{-1+\sqrt{5}}4\\
\overrightarrow{OP_2}\cdot\overrightarrow{OP_3}
&=
\frac{-1+\sqrt{5}}4
\cdot
\frac{-1-\sqrt{5}}4
+
\frac{\sqrt{10+2\sqrt{5}}}4
\cdot
\frac{\sqrt{10-2\sqrt{5}}}4
\\
&=
\frac{1-5}{16}+\frac{\sqrt{100-20}}{16}
=\frac{-1}{4}+\frac{\sqrt{5\cdot 16}}{16}
=\frac{-1+\sqrt{5}}{4},
\\
\overrightarrow{OP_3}\cdot\overrightarrow{OP_4}
&=
\frac{-1-\sqrt{5}}4\cdot
\frac{-1-\sqrt{5}}4
-
 \frac{\sqrt{10-2\sqrt{5}}}4
\cdot
\frac{\sqrt{10-2\sqrt{5}}}4
\\
&=\frac{1+2\sqrt{5}+5}{16}-\frac{10-2\sqrt{5}}{16}
=\frac{6+2\sqrt{5}-10+2\sqrt{5}}{16}
=\frac{-4+4\sqrt{5}}{16}=\frac{-1+\sqrt{5}}{4}.
\end{align*}
\item
Die Punkte liegen alle auf dem Einheitskreis und schliessen jeweils
den gleichen Winkel ein, also bilden sie ein regelmässiges Fünfeck.
\qedhere
\end{teilaufgaben}
\end{loesung}

\begin{bewertung}
Längenberechnung mit Pythagoras/Skalarprodukt ({\bf P}) 1 Punkt,
korrekte Werte ({\bf L}) 1 Punkt,
Zwischenwinkelformel ({\bf Z}) 1 Punkt,
Berechnung von zwei Winkeln mit Resultat $72^\circ$ ({\bf W}) 1 Punkt,
Beantwortung der Frage, ob Winkel gleich sind ({\bf F}) 1 Punkt,
regelmässiges Fünfeck ({\bf 5}) 1 Punkt.
\end{bewertung}

