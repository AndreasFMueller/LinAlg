Sei $\sigma$ die Ebene durch den Punkt $(-1,-1,-1)$ mit der Normalen 
\[
\vec n=\begin{pmatrix}2\\2\\1\end{pmatrix}
\]
und $g$ die Gerade durch die Punkte $(1,1,1)$ und $(-1,-1,1)$.
Die Ebene $\sigma$ und die Gerade $g$ schneiden sich im Punkt $P$.
Wie weit ist der Punkt $P$ von der Ebene durch die Punkte
$(1,0,0)$, $(0,1,0)$ und $(0,0,1)$ entfernt?

\thema{Durchstosspunkt}
\thema{Hessesche Normalform}

\begin{loesung}
Als erstes ist der Durchstosspunkt $P$ zu bestimmen.
Dazu brauchen wir zunächst die Geradengleichung in Parameterform:
\[
\begin{pmatrix} x\\y\\z \end{pmatrix}
=
\begin{pmatrix}1\\1\\1\end{pmatrix}+t
\left(
\begin{pmatrix}1\\1\\1\end{pmatrix}
-
\begin{pmatrix}-1\\-1\\1\end{pmatrix}
\right)
=
\begin{pmatrix}1\\1\\1\end{pmatrix}+t
\begin{pmatrix}2\\2\\0\end{pmatrix}
\]
Ausserdem brauchen wir die Ebenengleichung. Da wir einen Punkt und
die Normale kennen, können wir die Gleichung in der Normalenform
ansetzen:
\[
0=\vec n\cdot \left(\vec p-\begin{pmatrix}-1\\-1\\-1\end{pmatrix}\right)
=
\begin{pmatrix}2\\2\\1\end{pmatrix}\cdot\left(
\begin{pmatrix}x\\y\\z\end{pmatrix}-\begin{pmatrix}-1\\-1\\-1\end{pmatrix}
\right)
=2x+2y+z+5
\]
Daraus ergibt sich jetzt das Gleichungssystem
\[
\begin{linsys}{4}
 x& &  & & &-&2t&=&1\\
  & & y& & &-&2t&=&1\\
  & &  & &z& &  &=&1\\
2x&+&2y&+&z& &  &=&-5\\
\end{linsys}
\]
Es kann mit dem Gaussalgorithmus gelöst werden:
\begin{align*}
\begin{tabular}{|>{$}c<{$}>{$}c<{$}>{$}c<{$}>{$}c<{$}|>{$}c<{$}|}
\hline
1&0&0&-2& 1\\
0&1&0&-2& 1\\
0&0&1& 0& 1\\
2&2&1& 0&-5\\
\hline
\end{tabular}
&\rightarrow
\begin{tabular}{|>{$}c<{$}>{$}c<{$}>{$}c<{$}>{$}c<{$}|>{$}c<{$}|}
\hline
1&0&0&-2& 1\\
0&1&0&-2& 1\\
0&0&1& 0& 1\\
0&2&1& 4&-7\\
\hline
\end{tabular}
\rightarrow
\begin{tabular}{|>{$}c<{$}>{$}c<{$}>{$}c<{$}>{$}c<{$}|>{$}c<{$}|}
\hline
1&0&0&-2& 1\\
0&1&0&-2& 1\\
0&0&1& 0& 1\\
0&0&1& 8&-9\\
\hline
\end{tabular}
\\
&\rightarrow
\begin{tabular}{|>{$}c<{$}>{$}c<{$}>{$}c<{$}>{$}c<{$}|>{$}c<{$}|}
\hline
1&0&0&-2& 1\\
0&1&0&-2& 1\\
0&0&1& 0& 1\\
0&0&0& 8&-10\\
\hline
\end{tabular}
\rightarrow
\begin{tabular}{|>{$}c<{$}>{$}c<{$}>{$}c<{$}>{$}c<{$}|>{$}c<{$}|}
\hline
1&0&0& 0&-\frac32\\
0&1&0& 0&-\frac32\\
0&0&1& 0&       1\\
0&0&0& 1&-\frac54\\
\hline
\end{tabular}
\end{align*}
Die Ebene durch die drei genannten Punkte kam mehrmals in der Vorlesung
vor, sie hat die Normale
\[
\vec n'=\begin{pmatrix}1\\1\\1\end{pmatrix}.
\]
Die Normalenform ist
\[
0
=
\vec n'\cdot \left(\begin{pmatrix}x\\y\\z\end{pmatrix}-\begin{pmatrix}1\\0\\0\end{pmatrix}\right)
=
\begin{pmatrix}1\\1\\1\end{pmatrix}
\cdot \left(\begin{pmatrix}x\\y\\z\end{pmatrix}-\begin{pmatrix}1\\0\\0\end{pmatrix}\right)
=
x+y+z-1.
\]
Diese Form der Ebenengleichung ist jedoch nicht geeignet, um den Abstand
zu berechnen, dazu braucht man die Hessesche Normalform. Diese liegt
dann vor, wenn die Koeffizienten von $x$, $y$ und $z$ einen
Einheitsvektor bilden. Im vorliegenden Fall erreicht man dies durch
Division durch $\sqrt{ä3}$:
\[
\frac1{\sqrt{3}}x
+
\frac1{\sqrt{3}}y
+
\frac1{\sqrt{3}}z
-
\frac1{\sqrt{3}}
=0
\]
Für Punkte nicht auf der Ebene liefert die linke Seite den Abstand,
also 
\begin{align*}
d&=
\frac1{\sqrt{3}}
\cdot
\biggl(-\frac32\biggr)
+
\frac1{\sqrt{3}}
\cdot
\biggl(-\frac32\biggr)
+
\frac1{\sqrt{3}}
\cdot 1
-\frac1{\sqrt{3}}
\\
&=\frac1{\sqrt{3}}\biggl(
-\frac32-\frac32+1-1
\biggr)
=-\frac3{\sqrt{3}}=-\sqrt{3}
\end{align*}
Der gesuchte Abstand ist also $\sqrt{3}$.
\end{loesung}
