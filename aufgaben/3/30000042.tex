In einem Krisengebiet wird ein Passagierflugzeug von Rebellen abgeschossen.
Die Behörden versuchen, aus Zeugenaussagen möglichst genau die
Geometrie des Abschusses zu rekonstruieren.
In einer Kleinstadt haben zwei Zeugen $A$ und $B$ das Flugzeug just in
dem Moment hinter dem Dach eines Wohnblocks verschwinden sehen, 
als es von der Rebellenrakete getroffen wurde.
Abbildung~\ref{30000042:grundriss} zeigt die Situation im Grundriss.
\begin{figure}[h]
\centering
\includeagraphics[]{plan-1.pdf}
\caption{Grundriss von Gebäude und Beobachterpositionen
\label{30000042:grundriss}}
\end{figure}
Zeuge $B$ befand sich zum Zeitpunkt des Unglücks auf einer Leiter,
er wollte eine Glühbirne auswechseln. 
Die Informationen legen eine Gerade fest, auf der sich das Ereignis
abgespielt haben muss.
Finden Sie eine Parameterdarstellung derselben.

\thema{Schnittgerade}

\begin{hinweis}
Lösen Sie die Gleichungssysteme in dieser Aufgabe numerisch und geben Sie
die Resultate mit 5 signifikanten Ziffern an.
\end{hinweis}

\begin{loesung}
Gesucht ist die Schnittgerade der Ebenen aufgespannt von Beobachter und
jeweiliger Dachkannte.
Die Ebenen können in Form einer Parameterdarstellung angegeben werden,
wobei die beiden Richtungen zu den vom Beobachter aus sichtbaren 
Gebäudeecken zeigen.
Für Beobachter $A$ ist dies
\[
\begin{pmatrix}
x\\y\\z
\end{pmatrix}
=
\begin{pmatrix}
3\\-4\\0
\end{pmatrix}
+s
\begin{pmatrix}
-3\\4\\4
\end{pmatrix}
+t
\begin{pmatrix}
9\\-1\\4
\end{pmatrix}
\]
Für Beobachter $B$ ergibt sich die Ebene
\[
\begin{pmatrix}
x\\y\\z
\end{pmatrix}
=
\begin{pmatrix}
1\\5\\2
\end{pmatrix}
+u
\begin{pmatrix}
-1\\-5\\2
\end{pmatrix}
+v
\begin{pmatrix}
4\\7\\2
\end{pmatrix}
\]
Durch Gleichsetzen ergeben sich drei Gleichungen für die vier Unbekannten
$s$, $t$, $u$ und $v$:
\[
\begin{linsys}{5}
-3s&+&9t&+& u&-&4v&=& 2\\
 4s&-&1t&+&5u&-&7v&=&-9\\
 4s&+&4t&-&2u&-&2v&=&-2
\end{linsys}
\]
Mit diesen Gleichungen können drei Parameter
eliminiert werden, es bleibt eine Parameterdarstellung für eine Gerade.
Der Gaussalgorithmus liefert nach einiger Rechnung das Tableau
\[
\begin{tabular}{|>{$}c<{$}>{$}c<{$}>{$}c<{$}>{$}c<{$}>{$}c<{$}|>{$}c<{$}}
\hline
1&0&0&-0.53374&-1\\
0&1&0&-0.50307&\phantom{-}0\\
0&0&1&-1.07362&-1\\
\hline
\end{tabular}
=
\begin{tabular}{|>{$}c<{$}>{$}c<{$}>{$}c<{$}>{$}c<{$}>{$}c<{$}|>{$}c<{$}}
\hline
1&0&0&-\frac{ 87}{163}&-1\\
0&1&0&-\frac{ 82}{163}&\phantom{-}0\\
0&0&1&-\frac{175}{163}&-1\\
\hline
\end{tabular}
\qquad
\Rightarrow
\qquad
\begin{aligned}
s&=0.53374v+1=\frac{ 87}{163}v+1\\
t&=0.50307v\phantom{\mathstrut+1}=\frac{ 82}{163}v\\
u&=1.07362v+1=\frac{175}{163}v+1
\end{aligned}
\]
Jetzt muss man den Ausdruck für $u$ noch in die zweite Ebenengleichung
einsetzen, um die Geradengleichung zu finden.

Besonders elegant lässt sich das Problem mit einem einzigen Tableau
lösen:
\begin{align*}
\begin{tabular}{|>{$}c<{$}>{$}c<{$}>{$}c<{$}>{$}c<{$}>{$}c<{$}>{$}c<{$}>{$}c<{$}|>{$}c<{$}|}
\hline
x&y&z&s&t&u&v&\\
\hline
1&0&0& 3&-9& 0& 0& 3\\
0&1&0&-4& 1& 0& 0&-4\\
0&0&1&-4&-4& 0& 0& 0\\
1&0&0& 0& 0& 1&-4& 1\\
0&1&0& 0& 0& 5&-7& 5\\
0&0&1& 0& 0&-2&-2& 2\\
\hline
\end{tabular}
&\rightarrow
\begin{tabular}{|>{$}c<{$}>{$}c<{$}>{$}c<{$}>{$}c<{$}>{$}c<{$}>{$}c<{$}>{$}c<{$}|>{$}c<{$}|}
\hline
x&y&z&s&t&u&v&\\
\hline
1&0&0&0&0&0&-\frac{477}{163}&0\\
0&1&0&0&0&0&-\frac{266}{163}&0\\
0&0&1&0&0&0&-\frac{676}{163}&4\\
0&0&0&1&0&0&-\frac{ 87}{163}&1\\
0&0&0&0&1&0&-\frac{ 82}{163}&0\\
0&0&0&0&0&1&-\frac{175}{163}&1\\
\hline
\end{tabular}
\end{align*}
Daraus kann man die Parameterdarstellung der Geraden ablesen:
\[
\begin{pmatrix}
x\\y\\z
\end{pmatrix}
=
\begin{pmatrix}
0\\0\\4
\end{pmatrix}
+v\begin{pmatrix}
2.92638\\
1.63190\\
4.14724
\end{pmatrix}
=
\begin{pmatrix}
0\\0\\4
\end{pmatrix}
+\frac{v}{163}\begin{pmatrix}
477\\266\\676
\end{pmatrix}.
\]
\begin{figure}
\centering
\includeagraphics[]{plan-2.pdf}
\caption{Grundriss mit eingezeichneter Geraden (blau)
\label{30000042:gerade}}
\end{figure}
In Abbildung~\ref{30000042:gerade} ist die Gerade blau im Grundriss eingezeichnet.
\end{loesung}

\begin{bewertung}
Lösungsgerade als Schnittgerade zweier Ebenen ({\bf G}) 1 Punkt,
Gleichungen der Ebenen ({\bf E}) 3 Punkte,
Ebene finden als Lösung eines Gleichungssystems von 6 Gleichungen
mit 7 Unbekannten ({\bf S}) 1 Punkt,
Lösungsgerade ({\bf L}) 1 Punkt.
\end{bewertung}

