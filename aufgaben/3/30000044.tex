Gegeben sind die Punkte $A=(1,2)$, $B=(11,2)$ und $C=(7,10)$
in der Ebene.
Sie bilden ein Dreieck $\triangle ABC$.
\begin{teilaufgaben}
\item Berechnen Sie die Mittelpunkte $M_{AB}$, $M_{BC}$ und $M_{CA}$ aller drei
Seiten.
\item Bestimmen Sie f"ur jede Seite einen Vektor $\vec{v}_{AB}$, $\vec{v}_{BC}$
und $\vec{v}_{CA}$, der auf der entsprechenden Seite senkrecht steht.
\item Berechnen Sie den Schnittpunkt $M$ der beiden Mittelsenkrechten auf
$BC$ und $AC$.
\item Berechnen Sie die Entfernung von $M$ von allen drei Ecken des Dreiecks.
\end{teilaufgaben}

\begin{loesung}
$M$ ist nat"urlich der Umkreismittelpunkt, der von allen drei Ecken gleich
weit entfernt ist.
Wir verwenden die "ubliche Konvention, dass Ortsvektoren eines Punktes
mit dem zugeh"origen Kleinbuchstaben bezeichnet werden, also
\[
\begin{aligned}
\vec{a}&=\begin{pmatrix} 1\\ 2\end{pmatrix},&
\vec{b}&=\begin{pmatrix}11\\ 2\end{pmatrix}&
&\text{und}&
\vec{c}&=\begin{pmatrix} 7\\10\end{pmatrix}.
\end{aligned}
\]
Die Punkte sind in Abbildung~\ref{30000044:dreieck} dargestellt.
\begin{teilaufgaben}
%
% a)
%
\item Die Mittelpunkte haben die Ortsvektoren
\[
\begin{aligned}
\vec{m}_{AB}&=\frac{\vec{a}+\vec{b}}2=\begin{pmatrix}6\\2\end{pmatrix},
&
\vec{m}_{BC}&=\frac{\vec{b}+\vec{c}}2=\begin{pmatrix}9\\6\end{pmatrix}
&
&\text{und}&
\vec{m}_{CA}&=\frac{\vec{c}+\vec{a}}2=\begin{pmatrix}4\\6\end{pmatrix}.
\end{aligned}
\]
%
% b)
%
\item Einen Vektor, der orthogonal zu einer Kante des Dreiecks k"onnen
wir dadurch finden, dass wir die zwei Komponenten vertauschen und eine
mit einem negativen Vorzeichen versehen:
\[
\begin{aligned}
\vec{b}-\vec{a}&=\begin{pmatrix}10\\ 0\end{pmatrix}&
\vec{c}-\vec{b}&=\begin{pmatrix}-4\\ 8\end{pmatrix}&
&\text{und}&
\vec{a}-\vec{c}&=\begin{pmatrix}-6\\-8\end{pmatrix}
\\
   \vec{v}_{AB}&=\begin{pmatrix} 0\\ 1\end{pmatrix}&
   \vec{v}_{BC}&=\begin{pmatrix} 2\\ 1\end{pmatrix}&
&&
   \vec{v}_{CA}&=\begin{pmatrix} 4\\-3\end{pmatrix}
\end{aligned}
\]
Beliebige Vielfache dieser Vektoren sind nat"urlich auch korrekte 
Antworten.
%
% c)
%
\item
Die Mittelsenkrechten auf $BC$ und $AC$ haben die Parameterdarstellung
\[
\begin{aligned}
\begin{pmatrix}x\\y\end{pmatrix}
&=
\begin{pmatrix}9\\6\end{pmatrix}+t\begin{pmatrix}2\\ 1\end{pmatrix}
&&\text{und}&
\begin{pmatrix}x\\y\end{pmatrix}
&=
\begin{pmatrix}4\\6\end{pmatrix}+s\begin{pmatrix}4\\-3\end{pmatrix}.
\end{aligned}
\]
Um einen Schnittpunkt zu finden, muss ein Gleichungssystem mit vier
Unbekannten $x$, $y$, $t$ und $s$ gel"ost werden:
\begin{align*}
\begin{tabular}{|>{$}c<{$}>{$}c<{$}>{$}c<{$}>{$}c<{$}|>{$}c<{$}|}
\hline
x&y&t&s&\\
\hline
1&0&-2& 0&9\\
0&1&-1& 0&6\\
1&0& 0&-4&4\\
0&1& 0& 3&6\\
\hline
\end{tabular}
&
\rightarrow
\begin{tabular}{|>{$}c<{$}>{$}c<{$}>{$}c<{$}>{$}c<{$}|>{$}c<{$}|}
\hline
x&y&t&s&\\
\hline
1&0&-2& 0& 9\\
0&1&-1& 0& 6\\
0&0& 2&-4&-5\\
0&0& 1& 3& 0\\
\hline
\end{tabular}
\rightarrow
\begin{tabular}{|>{$}c<{$}>{$}c<{$}>{$}c<{$}>{$}c<{$}|>{$}c<{$}|}
\hline
x&y&t&s&\\
\hline
1&0&-2& 0& 9\\
0&1&-1& 0& 6\\
0&0& 1&-2&-\frac52\\
0&0& 0& 5& \frac52\\
\hline
\end{tabular}
\\
&
\rightarrow
\begin{tabular}{|>{$}c<{$}>{$}c<{$}>{$}c<{$}>{$}c<{$}|>{$}c<{$}|}
\hline
x&y&t&s&\\
\hline
1&0&-2& 0& 9\\
0&1&-1& 0& 6\\
0&0& 1& 0&-\frac32\\
0&0& 0& 1& \frac12\\
\hline
\end{tabular}
\rightarrow
\begin{tabular}{|>{$}c<{$}>{$}c<{$}>{$}c<{$}>{$}c<{$}|>{$}c<{$}|}
\hline
x&y&t&s&\\
\hline
1&0& 0& 0& 6\\
0&1& 0& 0& \frac92\\
0&0& 1& 0&-\frac32\\
0&0& 0& 1& \frac12\\
\hline
\end{tabular}
\end{align*}
also ist $M=(6,\frac92)$.
%
% d)
%
\item
Der Abstand ist
\begin{align*}
|\vec{m}-\vec{a}|
&=
\left|\begin{pmatrix}6\\\frac92\end{pmatrix}-\begin{pmatrix}1\\2\end{pmatrix}\right|
=
\frac12\left|\begin{pmatrix}10\\5\end{pmatrix}\right|
=
\frac{\sqrt{125}}2=\frac52\sqrt{5}
= 5.59016994374947,
\\
|\vec{m}-\vec{b}|
&=
\left|\begin{pmatrix}6\\\frac92\end{pmatrix}-\begin{pmatrix}11\\2\end{pmatrix}\right|
=
\frac12\left|\begin{pmatrix}-10\\5\end{pmatrix}\right|
=
\frac{\sqrt{125}}2=\frac52\sqrt{5}
= 5.59016994374947,
\\
|\vec{m}-\vec{c}|
&=
\left|\begin{pmatrix}6\\\frac92\end{pmatrix}-\begin{pmatrix}7\\10\end{pmatrix}\right|
=
\frac12\left|\begin{pmatrix}-2\\-11\end{pmatrix}\right|
=
\frac{\sqrt{125}}2=\frac52\sqrt{5}
= 5.59016994374947.
\end{align*}
Der Punkt $M$ hat also von allen drei Ecken den gleichen Abstand.
Nat"urlich ist es nicht n"otig, alle drei Abst"ande auszurechnen, denn
aus der Kontruktion erkennt man ja, dass $M$ der Umkreismittelpunkt ist
und dass daher alle drei Abst"ande gleich gross sind.
\qedhere
\end{teilaufgaben}
\begin{figure}
\centering
\includeagraphics[]{graph-1.pdf}
\caption{Dreieck und Mittelsenkrechten zu Aufgabe~\ref{30000044}
\label{30000044:dreieck}}
\end{figure}
\end{loesung}

\begin{diskussion}
Zuf"alligerweise ist das Dreieck gleichschenklig, denn
\begin{align*}
\overline{AB}
&=
10
\\
\overline{AC}
&=
\sqrt{(7-1)^2 + (10-2)^2}=\sqrt{36 + 64}=\sqrt{100}=10.
\end{align*}
Das bedeutet zum Beispiel, dass die Mittelsenkrechte auf $BC$ auch
Winkelhalbierende ist.
Man kann also insbesondere die Geraden ziemlich falsch parametrisieren,
oder sogar die falschen Ausgangspunkte (Ecken statt Seitenmittelpunkte)
nehmen, und trotzdem einen Punkt bekommen, der von $B$ und $C$ gleich
weit weg ist. Typischerweise ist er dann von $A$ doch nicht gleich weit
entfernt.
\end{diskussion}

\begin{bewertung}
Mittelpunkte ({\bf M}) 1 Punkt, Orthogonaltrick (Komponenten tauschen, 
Vorzeichenwechsel bei einer Komponente) ({\bf O}) 1 Punkt,
Parameterdarstellung ({\bf P}) 1 Punkt, L"osung des Gleichungsystems
zum Beispiel mit Gauss ({\bf G}) 1 Punkt, Bestimmung des Mittelpunktes
({\bf M}) 1 Punkt, Abstand ({\bf A}) 1 Punkt.
\end{bewertung}

