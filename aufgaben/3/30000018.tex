Bestimmen Sie die Abbildungsmatrizen folgender Drehungen des dreidimensionalen Raumes $\mathbb{R}^3$.
\begin{teilaufgaben}
\item Drehung um den Winkel $\alpha$ um die $x$-Achse.
\item Drehung um den Winkel $\beta$ um die $y$-Achse.
\item Drehung um den Winkel $\gamma=180^\circ$ um die Achse mit Richtungsvektor
\[
v=\begin{pmatrix}1\\1\\0\end{pmatrix}.
\]
\end{teilaufgaben}

\thema{Abbildungsmatrix}
\thema{Drehmatrix}

\begin{loesung}
\begin{teilaufgaben}
\item Die Drehung um die $x$-Achse bildet die Standardbasis-Vektoren wie folgt ab:
\[
\begin{aligned}
\vec e_1&\mapsto \vec e_1 = \begin{pmatrix}1\\0\\0\end{pmatrix},
&
\vec e_2&\mapsto \begin{pmatrix}0\\\cos(\alpha)\\\sin(\alpha)\end{pmatrix}
&&\text{und}
&
\vec e_3&\mapsto \begin{pmatrix}0\\-\sin(\alpha)\\\phantom{-}\cos(\alpha)\end{pmatrix}.
\end{aligned}
\]
Sie hat daher die Matrix
\[
D_{\alpha,x}
=
\begin{pmatrix}
1&0& 0\\
0&\cos(\alpha)& -\sin(\alpha)\\
0&\sin(\alpha)&\phantom{-}\cos(\alpha)
\end{pmatrix}.
\]
\item Die Drehung um die $y$-Achse bildet die Standardbasis-Vektoren wie folgt ab:
\[
\begin{aligned}
\vec e_1&\mapsto \begin{pmatrix}\phantom{-}\cos(\beta)\\0\\-\sin(\beta)\end{pmatrix},
&
\vec e_2&\mapsto \vec e_2 = \begin{pmatrix}0\\1\\0\end{pmatrix}
&&\text{und}
&
\vec e_3&\mapsto \begin{pmatrix}\sin(\beta)\\0\\\cos(\beta)\end{pmatrix}.
\end{aligned}
\]
Sie hat daher die Matrix
\[
D_{\beta,y}
=
\begin{pmatrix}
\phantom{-}\cos(\beta)&0& \sin(\beta)\\
0&1& 0 \\
-\sin(\beta)&0&\cos(\beta)
\end{pmatrix}.
\]
\item Diese Drehung bildet die Standardbasis-Vektoren wie folgt ab:
\[
\begin{aligned}
\vec e_1&\mapsto \vec e_2 = \begin{pmatrix}0\\1\\0\end{pmatrix},
&
\vec e_2&\mapsto \vec e_1 = \begin{pmatrix}1\\0\\0\end{pmatrix}
&&\text{und}
&
\vec e_3&\mapsto -\vec e_3  = \begin{pmatrix}0\\0\\-1\end{pmatrix}.
\end{aligned}
\]
Sie hat daher die Matrix
\[
D
=
\begin{pmatrix}
0&1& 0\\
1&0& 0\\
0&0&-1
\end{pmatrix}.
\]
\end{teilaufgaben}
\end{loesung}

