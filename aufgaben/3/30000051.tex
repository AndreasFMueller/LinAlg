In der Verfassung Nepals wird die Konstruktion des roten Teils der
Flagge Nepals wie folgt beschrieben:
\begin{enumerate}
\item Am unteren Rand eines karmesinroten Tuches zeichne man die Strecke
$AB$ von links nach rechts, deren L"ange die Breite der Flagge wird.
\item Von $A$ aus zeichne man die Strecke $AC$ senkrecht auf $AB$ nach
oben derart, dass die L"ange von $AC$ gleich lang ist wie $AB$ plus
ein Drittel der L"ange von $AB$.
Auf der Strecke $AC$ trage man von $A$ aus die Strecke $AD$ ab, die
gleich lang sein soll wie $AB$.
Zeichne die Strecke $BD$.
\item Auf der Strecke $BD$ trage von $B$ aus die L"ange der Strecke $AB$ ab,
der neue Punkt heisst $E$.
\item Von $E$ aus schneidet eine horizontale Gerade die Strecke $AC$ im
Punkt $F$.
Die Strecke $FG$ ist parallel zu und gleich lang wie $AB$
\item Verbinde $C$ mit $G$.
\end{enumerate}
\begin{center}
\includeagraphics[width=0.2\hsize]{NepalFlagPicture2.png}
\qquad
\qquad
\qquad
\includeagraphics[]{plan-1.pdf}
\end{center}
Verwenden Sie $A$ als Nullpunkt eines Koordinatensystems und
$\vec{e}_1=\overrightarrow{AB}$
als ersten Basisvektor.
\begin{teilaufgaben}
\item
Dr"ucken Sie die Konstruktionsschritte 3 bis 5 mit Hilfe von Vektoren aus
und berechnen Sie die Koordinaten der Punkte $E$, $F$ und $G$.
\item
Berechnen Sie den Umfang der Nepalesischen Flagge in Einheiten der Strecke
$AB$.
\end{teilaufgaben}

\begin{loesung}
\begin{teilaufgaben}
\item
\renewcommand{\labelenumii}{\arabic{enumii}.}
\begin{enumerate}
\item Der erste Basisvektor ist gerade $\vec{e}_1=\overrightarrow{AB}$.
\item $\overrightarrow{AC}=\frac43 \vec{e}_2$.
Da $AD$ die L"ange von $AB$ hat, ist $\overrightarrow{AD}=\vec{e}_2$.
\item Der Punkt $E$ liegt auf der Geraden $BD$, also gibt es eine
Zahl  $t$ so, dass
\[
\overrightarrow{BE}
=
\vec{e}_1+t\overrightarrow{BD}
=
\vec{e}_1+t(\vec{e}_2-\vec{e}_1)
=
\begin{pmatrix}1\\0\end{pmatrix}+t\begin{pmatrix}-1\\1\end{pmatrix}
\]
ist.
Ausserdem muss die L"ange stimmen.
Da die L"ange von $BE$ gleich ist wie die von $AB$, kann man f"ur $t$
\[
t=\frac{\overline{AB}}{\overline{BD}}
\]
verwenden, also $1/\sqrt{2}$.
Der Ortsvektor von $E$ ist daher
\[
\overrightarrow{AE}
=
\vec{e}_1+\frac1{\sqrt{2}}\overrightarrow{BD}
=
\vec{e}_1+\frac1{\sqrt{2}}(\vec{e}_2-\vec{e}_1)
=
\begin{pmatrix}1\\0\end{pmatrix}+\frac1{\sqrt{2}}\begin{pmatrix}-1\\1\end{pmatrix}
\]
\item Der Punkt $F$ entsteht als Schnittpunkt der horizontalen Geraden von
$E$ aus mit der vertikalen Achse:
\[
\overrightarrow{AE}+t\vec{e}_1 = s\vec{e}_2
\]
oder als Gleichungssytem geschrieben:
\[
\left.
\begin{linsys}{2}
1-\frac1{\sqrt{2}}&+&t&=&0\\
                  &-&t&=&s
\end{linsys}
\right\}
\quad\Rightarrow\quad
t=\frac{1-\sqrt{2}}{\sqrt{2}},
\;
s=\frac{\sqrt{2}-1}{\sqrt{2}}.
\]
Daraus erh"alt man durch Einsetzen den Ortsvektor von $F$:
\[
\overrightarrow{AF}
=
\begin{pmatrix}
0\\\frac1{\sqrt{2}}
\end{pmatrix}
\]
\item Der Ortsvektor von $G$ ist
\[
\overrightarrow{AG}
=
\overrightarrow{AF}+\vec{e}_1
=
\begin{pmatrix}1\\\frac1{\sqrt{2}}\end{pmatrix}.
\]
\end{enumerate}
\item
Der Umfang $U$ ist
\begin{align*}
U
&=
\overline{AB} + \overline{BE} + \overline{EG} + \overline{GC} +\overline{AC}
\\
&=
1 + 1 + \frac1{\sqrt{2}}+\overline{GC} + \frac43.
\end{align*}
wobei f"ur den dritten Term verwendet wurde, dass $E$ die Strecke
$FG$ im gleichen Verh"altnis teilt wie $F$ die Strecke $DA$.
Die L"ange der Strecke kann mit dem Satz des Pythagoras berechnet
werden:
\[
\overline{GC}
=
\sqrt{1+\left(\frac43-\frac1{\sqrt{2}}\right)^2}
\]
und damit f"ur den Umfang
\begin{align*}
U
&=
 2
+ \frac1{\sqrt{2}}
+ \sqrt{1+\left(\frac43-\frac1{\sqrt{2}}\right)^2}
+ \frac43
\\
&=
\frac{10}{3}
+ \frac1{\sqrt{2}}
+ \sqrt{1+\left(\frac43-\frac1{\sqrt{2}}\right)^2}
\\
&=
5.22033828580995072213
\qedhere
\end{align*}
\end{teilaufgaben}
\end{loesung}

\begin{bewertung}
Jeder der drei Konstruktionssschritte ($\textbf{S}_i$, $1\le i\le 3$) ein Punkt,
Koordinaten von $E$ ({\bf K}) 1 Punkt,
Umfang ({\bf U}) 2 Punkte.
\end{bewertung}

