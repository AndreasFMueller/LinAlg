Ein vom Punkt $Q_1=(7,5,5)$ ausgehender Lichtstrahl wird im Punkt $P$
einer Kugel vom Radius $3$ mit Mittelpunkt im Ursprung des Koordinatensystems
reflektiert und trifft dann im Punkt $Q_2=(-1,7,7)$ ein.
Bestimmen Sie den Punkt $P$.

\thema{Kugel}

\begin{loesung}
Die beiden Punkte $Q_1$ und $Q_2$ sind gleich weit von $O$ entfernt:
\begin{align*}
\left|\,\begin{pmatrix} 7\\5\\5 \end{pmatrix} \,\right|
&=\sqrt{7^2+5^2+5^2}=\sqrt{99},
&
\left|\,\begin{pmatrix}-1\\7\\7 \end{pmatrix} \,\right|
&=\sqrt{1^2+7^2+7^2}=\sqrt{99}.
\end{align*}
Also ist der Reflexionspunkt $P$ der Durchstosspunkt der Winkelhalbierenden
der beiden Ortsvektoren $\vec q_1=\overrightarrow{OQ}_1$ und
$\vec q_2=\overrightarrow{OQ}_2$. Die Winkelhalbierende ist
die Gerade mit Parameterdarstellung
\[
\vec p(t)=t\frac{\vec q_1+\vec q_2}2=t\begin{pmatrix} 3\\6\\6 \end{pmatrix}.
\]
Der Parameter $t$ muss jetzt so bestimmt werden, dass $\vec p(t)$ die Länge
$3$ hat:
\begin{align*}
|\vec p(t)|&=t\sqrt{3^2+6^2+6^2}=t\sqrt{81}=9t=3
&&\Rightarrow&
t&=\frac13
\end{align*}
Damit folgt für den Punkt:
\[
\vec p({\textstyle\frac13})=\begin{pmatrix}1\\2\\2\end{pmatrix}
\qquad\Rightarrow\qquad
P=(1,2,2).
\qedhere
\]
\end{loesung}
