%
% aufgabensammlung.tex -- Aufgabensammlung zur Linearen Algebra
%
% (c) 2017 Prof. Dr. Andreas Mueller, HSR
%
\documentclass[a4paper,12pt]{book}
\usepackage{german}
\usepackage[utf8]{inputenc}
\usepackage[T1]{fontenc}
\usepackage{times}
\usepackage{geometry}
\geometry{papersize={210mm,297mm},total={160mm,240mm},top=31mm,bindingoffset=15mm,marginparwidth=9mm}
\usepackage{alltt}
\usepackage{verbatim}
\usepackage{fancyhdr}
\usepackage{amsmath}
\usepackage{amssymb}
\usepackage{amsfonts}
\usepackage{amsthm}
\usepackage{textcomp}
\usepackage{graphicx}
\usepackage{array}
\usepackage{ifthen}
\usepackage{multirow}
\usepackage{txfonts}
%\usepackage[basic]{circ}
\usepackage[all]{xy}
\usepackage{algorithm}
\usepackage{algorithmic}
\usepackage{makeidx}
\usepackage{paralist}
\usepackage{enumitem}
\usepackage{multicol}
\usepackage{tikz}
\usetikzlibrary{arrows,3d,calc}
\usepackage{color}
\usepackage[colorlinks=true]{hyperref}
\hypersetup{
    linktoc=all,
    linkcolor=blue
}
\usepackage{environ}
\usepackage{epic}
\usepackage{etoolbox}
\makeindex
\begin{document}
\pagestyle{fancy}
\lhead{Aufgabensammlung}
\rhead{}
\frontmatter
\newcommand\HRule{\noindent\rule{\linewidth}{1.5pt}}
\begin{titlepage}
\vspace*{\stretch{1}}
\HRule
\vspace*{2pt}
\begin{flushright}
{\Huge
Lineare Algebra:\\
\bigskip
Aufgabensammlung}
\end{flushright}
\HRule
\begin{flushright}
\vspace{30pt}
\LARGE
Andreas Müller
und
Tabea Gassmann-Méndez
\end{flushright}
\vspace*{\stretch{2}}
\begin{center}
Hochschule für Technik, Rapperswil, 2011-2019
\end{center}
\end{titlepage}
\rhead{Inhaltsverzeichnis}
\tableofcontents
\newenvironment{beispiel}[1][Beispiel]{%
\begin{proof}[#1]%
\renewcommand{\qedsymbol}{$\bigcirc$}
}{\end{proof}}
\mainmatter
\input ../skript/linsys.tex
\input uebungen.tex
\openthemaindex
\setboolean{loesungen}{true}
\aufgabetoplevel{./}
\allowdisplaybreaks
\chapter{Lineare Gleichungssysteme}
\lhead{Kapitel \thechapter}
\rhead{Lineare Gleichungssysteme}
\input 1.tex
\chapter{Determinanten}
\lhead{Kapitel \thechapter}
\rhead{Determinanten}
\input 2.tex
\chapter{Affine Vektorgeometrie}
\lhead{Kapitel \thechapter}
\rhead{Affine Vektorgeometrie}
\input 3.tex
\chapter{Orthogonalität}
\lhead{Kapitel \thechapter}
\rhead{Orthogonalität}
\input 4.tex
\chapter{Orientierung}
\lhead{Kapitel \thechapter}
\rhead{Orientierung}
\input 5.tex
\chapter{Eigenschaften linearer Abbildungen}
\lhead{Kapitel \thechapter}
\rhead{Eigenschaften linearer Abbildungen}
\input 6.tex
\chapter{Matrixzerlegungen}
\lhead{Kapitel \thechapter}
\rhead{Matrixzerlegungen}
\input 7.tex
\chapter{Eigenwerte und Eigenvektoren}
\lhead{Kapitel \thechapter}
\rhead{Eigenwerte und Eigenvektoren}
%\input 8.tex
\chapter{Octave}
\lhead{Kapitel \thechapter}
\rhead{Octave}
Die Aufgaben in diesem Kapitel sind als Lernaufgabe gedacht, mit denen
man sich in die Benutzung des Programms Octave einführen lassen kann.

\bigskip
\input o.tex
\closethemaindex
\printthemata
\input aufgabensammlung.ind
\end{document}
