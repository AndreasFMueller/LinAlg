%
% aufgabensammlung.tex -- Aufgabensammlung zur Linearen Algebra
%
% (c) 2017 Prof. Dr. Andreas Mueller, HSR
%
\documentclass[a4paper,12pt]{book}
\usepackage{german}
\usepackage[utf8]{inputenc}
\usepackage[T1]{fontenc}
\usepackage{times}
\usepackage{geometry}
\geometry{papersize={210mm,297mm},total={170mm,240mm},top=31mm,bindingoffset=15mm,marginparwidth=9mm}
\usepackage{alltt}
\usepackage{verbatim}
\usepackage{fancyhdr}
\usepackage{amsmath}
\usepackage{amssymb}
\usepackage{amsfonts}
\usepackage{amsthm}
\usepackage{mathtools}
\usepackage{textcomp}
\usepackage{graphicx}
\usepackage{array}
\usepackage{ifthen}
\usepackage{multirow}
\usepackage{txfonts}
%\usepackage[basic]{circ}
\usepackage[all]{xy}
\usepackage{algorithm}
\usepackage{algorithmic}
\usepackage{makeidx}
\usepackage{paralist}
\usepackage{enumitem}
\usepackage{multicol}
\usepackage{CJKutf8}
\usepackage{wasysym}
\usepackage{tikz}
\usetikzlibrary{arrows,3d,calc}
\usepackage{color}
\usepackage[colorlinks=true]{hyperref}
\hypersetup{
    linktoc=all,
    linkcolor=blue
}
\usepackage{environ}
\usepackage{epic}
\usepackage{etoolbox}
\makeindex
\begin{document}
\pagestyle{fancy}
\lhead{Aufgabensammlung}
\rhead{}
\frontmatter
\newcommand\HRule{\noindent\rule{\linewidth}{1.5pt}}
\begin{titlepage}
\vspace*{\stretch{1}}
\HRule
\vspace*{2pt}
\begin{flushright}
{\Huge
Lineare Algebra:\\
\bigskip
Aufgabensammlung}
\end{flushright}
\HRule
\begin{flushright}
\vspace{30pt}
\LARGE
Andreas Müller
und
Tabea Gassmann-Méndez
\end{flushright}
\vspace*{\stretch{2}}
\begin{center}
OST Ostschweizer Fachhochschule, Rapperswil, 2011-2022
\end{center}
\end{titlepage}
\rhead{Inhaltsverzeichnis}
\tableofcontents
\newenvironment{beispiel}[1][Beispiel]{%
\begin{proof}[#1]%
\renewcommand{\qedsymbol}{$\bigcirc$}
}{\end{proof}}
\mainmatter
% labels
\makeatletter
\newcommand{\customlabel}[2]{%
\protected@write \@auxout {}{\string \newlabel {#1}{{#2}{}{}{}{}}}}
\makeatother
\input{gausslabels.tex}
\def\gaussurl#1{
\edef\glnumber{\getrefnumber{#1}}
\url{https://linalg.ch/gauss/?id=\glnumber}}
\def\jacobiurl#1{
\edef\glnumber{\getrefnumber{#1}}
\url{https://linalg.ch/jacobi/?id=\glnumber}}
%
%%%%%%%%%%%%%%%%%%%%%%%
%% Copyleft
%% Walter A. Kehowski
%% Department of Mathematics
%% Glendale Community College
%% walter.kehowski@gcmail.maricopa.edu
%% \begin{linsys}{2}
%% -x & + & 4y & = & 8\\
%% -3x & - & 2y & = & 6
%% \end{linsys}
%%%%%%%%%%%%%%%%%%%%%%%
%\makeatletter
%% math-mode column types ------------------
\newcolumntype{\linsysR}{>{$}r<{$}}
\newcolumntype{\linsysL}{>{$}l<{$}}
\newcolumntype{\linsysC}{>{$}c<{$}}
\newenvironment{linsys}[1]{%
\begin{tabular}{*{#1}{\linsysR@{\;}\linsysC}@{\;}\linsysR}}%
{\end{tabular}}
%\makeatother
\endinput

%
% uebung.tex -- gemeinsame Makros fuer Uebungsblaetter
%
% (c) 2006 Prof. Dr. Andreas Mueller, HSR
% $Id: uebung.tex,v 1.3 2008/01/06 14:05:56 afm Exp $
%
\newcounter{uebungsaufgabe}
\newboolean{loesungen}
% environment fuer uebungsaufgaben
\newenvironment{uebungsaufgaben}{
\begin{list}{\arabic{uebungsaufgabe}.}
  {\usecounter{uebungsaufgabe}
  \setlength{\labelwidth}{2cm}
  \setlength{\leftmargin}{0pt}
  \setlength{\labelsep}{5mm}
  \setlength{\rightmargin}{0pt}
  \setlength{\itemindent}{0pt}
}}{\end{list}\vfill\pagebreak}
% Teilaufgaben
\newenvironment{teilaufgaben}{
\begin{enumerate}
\renewcommand{\labelenumi}{\alph{enumi})}
}{\end{enumerate}}
% Loesung
\NewEnviron{loesung}{%
\begin{proof}[Lösung]%
\renewcommand{\qedsymbol}{$\bigcirc$}
\BODY
\end{proof}}
\NewEnviron{diskussion}{
\BODY
}
\def\keineloesungen{%
\RenewEnviron{loesung}{\relax}
\RenewEnviron{diskussion}{\relax}
}
% Hinweis
\newenvironment{hinweis}{%
\renewcommand{\qedsymbol}{}
\begin{proof}[Hinweis]}{\end{proof}}
% Aufgabe aus der Sammlung wiedergeben
\newcounter{problemcounter}[chapter]
\def\aufgabepath{./}
\def\ainput#1{\input\aufgabepath/#1}
\def\verbatimainput#1{\expandafter\verbatiminput{\aufgabepath/#1}}
\def\aufgabetoplevel#1{%
\expandafter\def\expandafter\inputpath{#1}%
\let\aufgabepath=\inputpath
}
\def\includeagraphics[#1]#2{\expandafter\includegraphics[#1]{\aufgabepath#2}}
% \aufgabe
\newcommand{\aufgabe}[2]{%
\refstepcounter{problemcounter}%
\label{#2}
\bigskip{\parindent0pt\strut}\hbox{\bf\theproblemcounter. }%
\marginpar{\raggedright\tiny #2}%
\expandafter\def\csname aufgabepath\endcsname{\inputpath/#1/#2/}%
\expandafter\input{\inputpath#1/#2.tex}
\bigskip
}
\renewcommand\theproblemcounter{\thechapter.\arabic{problemcounter}}
% Bewertung
\NewEnviron{bewertung}{\relax}
% oft benutzte Macros
\def\blank{\text{\textvisiblespace}}

\openthemaindex
\setboolean{loesungen}{true}
\aufgabetoplevel{./}
\allowdisplaybreaks
\chapter{Lineare Gleichungssysteme}
\lhead{Kapitel \thechapter}
\rhead{Lineare Gleichungssysteme}
\input{1.tex}
\chapter{Determinanten}
\lhead{Kapitel \thechapter}
\rhead{Determinanten}
\input{2.tex}
\chapter{Affine Vektorgeometrie}
\lhead{Kapitel \thechapter}
\rhead{Affine Vektorgeometrie}
\input{3.tex}
\chapter{Orthogonalität}
\lhead{Kapitel \thechapter}
\rhead{Orthogonalität}
\input{4.tex}
\chapter{Orientierung}
\lhead{Kapitel \thechapter}
\rhead{Orientierung}
\input{5.tex}
\chapter{Eigenschaften linearer Abbildungen}
\lhead{Kapitel \thechapter}
\rhead{Eigenschaften linearer Abbildungen}
\input{6.tex}
\chapter{Matrixzerlegungen}
\lhead{Kapitel \thechapter}
\rhead{Matrixzerlegungen}
\input{7.tex}
\chapter{Eigenwerte und Eigenvektoren}
\lhead{Kapitel \thechapter}
\rhead{Eigenwerte und Eigenvektoren}
\input{8.tex}
\chapter{Octave}
\lhead{Kapitel \thechapter}
\rhead{Octave}
Die Aufgaben in diesem Kapitel sind als Lernaufgaben gedacht, mit denen
man sich in die Benutzung des Programms Octave einführen lassen kann.

\bigskip
\input{o.tex}
\closethemaindex
\printthemata
\input aufgabensammlung.ind
\end{document}
