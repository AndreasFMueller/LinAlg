Um ein lineares Gleichungssystem zu l"osen, braucht man ausser
der Koeffizienten-Matrix auch noch die rechten Seiten, einzugeben
als Spaltenvektor {\tt b}, und einen Operator, mit dem sich
Gleichungssysteme l"osen lassen.
\begin{teilaufgaben}
\item In der Hilfe-Funktion \verb+help '\'+ findet man nicht
viel Information, daf"ur in der Online-Dokumentation. Der
Befehl \verb+doc left+ liefert Information "uber die Links-Division.
Lesen sie nach, was \verb+A \ b+ tut.
\item L"osen Sie das Gleichungssytem
\[
\begin{linsys}{3}
 & &y&-&z&=&2\\
 & &y&+&z&=&4\\
-x&+&&+&z&=&2
\end{linsys}
\]
\end{teilaufgaben}

\begin{loesung}
\begin{teilaufgaben}
\item
\item
\verbatimainput{aufg6b.m}
\end{teilaufgaben}
\end{loesung}

