Machen Sie sich mit dem Hilfesystem von Octave vertraut. Zu fast
jeder Funktion kann man  mit dem {\tt help}-Befehl Hilfe abrufen.
Um Hilfe zu einer Funktion {\tt foo} abzurufen, geben Sie den Befehl
\verb+help("foo")+
ein.
\begin{teilaufgaben}
\item
Lesen Sie nach, was die Funktionen {\tt ones}, {\tt zeros} und {\tt eye}
tun.
\item
Definieren Sie eine $3\times 3$-Matrix, die aus lauter Einsen
besteht.
\item
Definieren Sie eine Matrix {\tt E}, die den Inhalt
\[
\begin{pmatrix}
1&0&0\\
0&1&0\\
0&0&1
\end{pmatrix}
\]
hat.
\end{teilaufgaben}

\begin{loesung}
\begin{teilaufgaben}
\item
Mit den Funktionen {\tt ones} und {\tt zeros} kann man Matrizen
bestehend aus lauter Einsen bzw.~Nullen erzeugen.
Die Funktion {\tt eye} erlaubt, die Einheitsmatrix zu erzeugen.
\item
\verbatimainput{aufg2b.m}
\item
\verbatimainput{aufg2c.m}
\end{teilaufgaben}
\end{loesung}

