In der Vorlesung haben Sie gelernt, eine Matrix auf ``Treppenform''
zu bringen: auf der Diagonalen Einsen darunter und darüber
ausschliesslich Nullen. Auf Englisch heisst diese Form ``reduced
row echelon form'', sie kann mit der Funktion {\tt rref} berechnet
werden.
\begin{teilaufgaben}
\item Lesen Sie in der Hilfe nach, wie die Funktion {\tt rref}
verwendet wird.
\item Wenden Sie die Funktion auf die Matrizen
\[
\begin{pmatrix}
1&1&-2&4&5\\
2&2&-3&1&3\\
3&3&-4&-2&1
\end{pmatrix},
\begin{pmatrix}
1&2&1&1\\
3&7&6&5\\
-2&-1&7&4
\end{pmatrix}
\]
an.
\item
Verwenden Sie diese Funktion, um die Gleichungssyteme aus "Ubung
1 nochmals zu lösen.
\end{teilaufgaben}

\thema{Octave}

\begin{loesung}
\begin{teilaufgaben}
\item
\item
\verbatimainput{aufg4b.m}
\end{teilaufgaben}
\end{loesung}

