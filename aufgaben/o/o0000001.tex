Octave kann mit Matrizen arbeiten, doch wie gibt man Matrizen ein?
Die Eingabe
\begin{verbatim}
octave:1> A=[1,2;3,4]
A =

   1  2
   3  4
\end{verbatim}
erzeugt eine Matrix, auf die fortan unter dem Namen {\tt A}
Bezug genommen werden kann. Geben Sie einfach den Namen der
Matrix auf der Kommandozeile an, um ihren Inhalt angezeigt zu
erhalten.

\thema{Octave}

Aufgabe: Geben sie die Koeffizienten-Matrizen, die in der "Ubungsserie 1
vorgekommen
sind, unter verschiedenen Namen in Octave ein. "Uberprüfen Sie die
Eingabe, indem Sie sich die Matrizen anzeigen lassen.

\begin{loesung}
Die Matrix
\[
\begin{pmatrix}
1&2&1\\
0&1&2\\
1&2&2
\end{pmatrix}
\]
kann so eingegeben werden:
\verbatimainput{aufg1.m}
\end{loesung}

